
\documentclass{sebase}
%%%%%%%%%%%%%%%%%%%%%%%%%%%%%%%%%%%%%%%%%%%%%%%%%%%%%%%%%%%%%%%%%%%%%%%%%%%%%%%%%%%%%%%%%%%%%%%%%%%%%%%%%%%%%%%%%%%%%%%%%%%%%%%%%%%%%%%%%%%%%%%%%%%%%%%%%%%%%%%%%%%%%%%%%%%%%%%%%%%%%%%%%%%%%%%%%%%%%%%%%%%%%%%%%%%%%%%%%%%%%%%%%%%%%%%%%%%%%%%%%%%%%%%%%%%%
\usepackage{SERES10list}
\usepackage{hyperref}
\usepackage{multicol}
\usepackage{graphicx}
\usepackage{wrapfig}

%TCIDATA{OutputFilter=LATEX.DLL}
%TCIDATA{Version=5.00.0.2606}
%TCIDATA{<META NAME="SaveForMode" CONTENT="1">}
%TCIDATA{BibliographyScheme=BibTeX}
%TCIDATA{Created=Tuesday, December 11, 2012 12:46:20}
%TCIDATA{LastRevised=Thursday, November 19, 2015 11:56:54}
%TCIDATA{<META NAME="GraphicsSave" CONTENT="32">}
%TCIDATA{<META NAME="DocumentShell" CONTENT="Style Editor\resume test 5 shl">}
%TCIDATA{Language=American English}
%TCIDATA{CSTFile=AG.cst}

\input{tcilatex}

\begin{document}


%TCIMACRO{\TeXButton{TeX field}{\begin{wrapfigure}[0]{r}[0pt]{0cm}}}%
%BeginExpansion
\begin{wrapfigure}[0]{r}[0pt]{0cm}%
%EndExpansion

\FRAME{itbpF}{1.8766in}{1.8766in}{0in}{}{}{pic4c.jpg}{\special{language
"Scientific Word";type "GRAPHIC";maintain-aspect-ratio TRUE;display
"USEDEF";valid_file "F";width 1.8766in;height 1.8766in;depth
0in;original-width 3.6979in;original-height 3.6979in;cropleft "0";croptop
"1";cropright "1";cropbottom "0";filename
'../2012USCstuff/resume/pic4c.jpg';file-properties "NPEU";}}

%TCIMACRO{\TeXButton{TeX field}{\end{wrapfigure}}}%
%BeginExpansion
\end{wrapfigure}%
%EndExpansion

{\huge Anthony Liang, Ph.D.}\vspace*{0.1in}

{\Large Lecturer}\vspace*{0.3in}

{\large No. 33, Ln. 186, Yude 2nd Rd., North District}

{\large Tainan City, Taiwan 70450}

{\large Cellphone: 0975954735}

{\large Email: whymranderson@gmail.com}

\vspace*{0.1in}

\QTR{lightgrey}{\rule[0.05in]{7.5in}{0.01in}}

\QTR{smallcapitalgrey}{Education}

\begin{itemize}
\item Georgia Institute of Technology, Atlanta, Georgia, 2003 - 2009

\quad Ph.D. in Physics

\item National Taiwan University, Taipei, Taiwan, 1999 - 2003

\quad Bachelor of Science in Physics

\item National Tainan First Senior High, Tainan, Taiwan 1997 - 1999

\bigskip
\end{itemize}

\QTR{smallcapitalgrey}{Work Experience Summary}

\begin{itemize}
\item Freelance Physicist, Jan 2013 - current

\quad \U{88dc}\U{8db3}\U{4e86}\U{8f49}\U{52d5}\U{7684}\U{725b}\U{9813}%
\U{5c24}\U{62c9}\U{65b9}\U{7a0b}\U{929c}\U{63a5}\U{5230}\U{822a}\U{592a}%
\U{9818}\U{57df}\U{7684}\U{59ff}\U{614b}\U{4f30}\U{6e2c}\U{8207}\U{6a5f}%
\U{68b0}\U{9818}\U{57df}\U{7684}\U{8f49}\U{52d5}\U{7a4d}\U{5206}\U{5668}%
\U{7684}\U{6a4b}\U{6a11}\U{3002}\U{4e26}\U{4e14}\U{5b8c}\U{6210}\U{4e00}%
\U{5957}\U{5c08}\U{984c}\U{6559}\U{6750}\U{53ca}\U{8edf}\U{9ad4}GS Simulation%
\U{9069}\U{5408}\U{7576}\U{4f5c}\U{5927}\U{5b78}\U{529b}\U{5b78}\U{4e2d}%
\U{9640}\U{87ba}\U{8ab2}\U{7a0b}\U{4e2d}\U{7684}\U{9032}\U{968e}\U{6559}%
\U{6750}\U{3002}\U{6559}\U{6750}\U{5305}\U{542b}\U{5f71}\U{7247}\U{6559}%
\U{5b78}\U{3001}\U{7d19}\U{672c}\U{6559}\U{6750}\U{3001}\U{4ee5}\U{53ca}%
\U{4e00}\U{5957}\U{4ee5}python\U{7a0b}\U{5f0f}\U{8a9e}\U{8a00}\U{5beb}%
\U{6210}\U{7684}\U{9640}\U{87ba}\U{5100}\U{6a21}\U{64ec}\U{7a0b}\U{5f0f}%
\U{ff0c}\U{9069}\U{5408}\U{7576}\U{4f5c}\U{5be6}\U{9a57}\U{7528}\U{6559}%
\U{6750}\U{3002}\U{6a21}\U{64ec}\U{7a0b}\U{5f0f}\U{4e26}\U{5305}\U{542b}%
\U{59ff}\U{614b}\U{4f30}\U{6e2c}\U{8207}\U{8f49}\U{52d5}\U{7a4d}\U{5206}%
\U{5668}\U{7684}\U{89e3}\U{7b97}\U{6cd5}\U{ff0c}\U{56e0}\U{6b64}\U{4e5f}%
\U{662f}\U{4e00}\U{500b}\U{4e0d}\U{540c}\U{8f49}\U{52d5}\U{529b}\U{5b78}%
\U{65b9}\U{6cd5}\U{7684}\U{6bd4}\U{8f03}\U{5e73}\U{53f0}\U{3002}\U{6559}%
\U{6750}\U{7372}\U{5f97}\U{6210}\U{5927}\U{6a5f}\U{68b0}\U{7cfb}\U{85cd}%
\U{5146}\U{6770}\U{6559}\U{6388}\U{7684}\U{9ad8}\U{7b49}\U{52d5}\U{529b}%
\U{5b78}\U{8ab2}\U{7a0b}\U{9078}\U{7528}\U{ff0c}\U{4e26}\U{4e14}\U{7372}%
\U{5f97}\U{5728}\U{4e2d}\U{7814}\U{9662}\U{8cc7}\U{5275}\U{6240}\U{66f9}%
\U{6631}\U{535a}\U{58eb}\U{7684}\U{7814}\U{7a76}\U{5c0f}\U{7d44}\U{4e0a}%
\U{5c08}\U{984c}\U{6f14}\U{8b1b}\U{3002}

\item Bridged up the connection from rotational Newton-Euler equations to
orientation estimation in the field of electrical or aerospace engineering,
and to rigid body integrator in the field of mechanical engineering.
Completed a suite of teaching materials ideal for supplementing gyroscope
dynamics in university physics/mechanics. Materials include a 100-minute
lecturing film, a carefully-typeset instructional note, and a simulation
program developed with python programming language ideal as a labratory
companion. Material is choosen by Dr. Lan in the mechanical department of
NCKU as a special topic, and is presented at Academia Sinica's ITRI center
in Dr. Tsao's group.

\item Postdoctoral Research Associate

\quad University of Southern California, March 2010 - December 2012

\quad \quad \quad \quad Studied interactions of metal nanoclusters with
photons and electrons at different temperatures in a molecular beam.

\item Lecturer

\quad University of Southern California, September 2010 -- May 2011

\quad \quad \quad \quad Taught two semesters of introductory physics to life
science undergraduates.

\item Research Assistant

\quad Georgia Institute of Technology, September 2003 -- December 2009

\quad \quad \quad \quad Studied ferroelectricity, electric and magnetic
susceptibilities in metal nanoclusters at low temperature and measured
electric susceptibility of sodium metal clusters at 20 $\unit{K}$.

\item Substitute teacher

\qquad ACI Institute, May 2013
\end{itemize}

\bigskip

\QTR{smallcapitalgrey}{Work Experience}

\begin{description}
\item \quad \underline{\textbf{Postdoctoral Research Associate}} 
{\normalsize at University of Southern California, March 2010 - December 2012%
}
\end{description}

\begin{itemize}
\item Planned, tested, and built a high temperature (1000 degC)
resistively-heated source including heat flow design/analysis. This system
is contamination free (oxygen, hydroxide) and highly-reactive lithium metal
is melted in this system to generate lithium clusters by supersonic jet
expansion. The aerodynamics of supersonic/subsonic jet expansion is studied
and tested in order to optimize expansion parameters. Upgraded a linear
Wiley-Mclaren TOF\ mass spectrometer and mastered a magnetron ion sputtering
source. Mastered a helium leak detector. Maintained/upgraded/built
components of our molecular beam apparatus.

\item I introduced a statistical data reduction method to successfully
observed homonuclear aluminum clusters for the first time in our lab.

\item Studied and optimized the growth of nanoparticles/clusters in a cold
helium flow. This includes subjects like atom aggregation theory in free
space, gas dynamics, and ensemble temperatures on different degrees of
freedom (rotation, vibration, and electronic) of clusters and their
couplings through collisions.

\item Mastered a quadruple mass analyzer. Upgraded a home-written LabVIEW
control program and its NI communication hardware to acquire data (DAQ).
Experts in nuclear instrumentation module (NIM) digital/analog electronic
instruments.

\item Used software SIMION: 1. to simulate ion optics and lenses to improve
ion focusing. 2. to improve the design of a set of electric deflector plates
to steer the charged particles. 3. to improve the design of our voltage
acceleration mesh grids.

\item Constructed a clusters deposition chamber, for studying proximity
effect of soft-landed superconducting clusters and fullerene compounds.

\item Measured and studied size-dependent electron-impact ionization
potentials (1st, 2nd, 3rd) of selected silver clusters at 77 $\unit{K}$.\cite%
{Halder:2012:article-Double-and-triple-ionization-}

\item Constructed single-photon ionization potential experiments for
aluminum clusters with our OPO\ laser.

\item Many projects can be found in a \href{https://sites.google.com/site/vitalyslab/%
}{\underline{\color{blue}\smash{webpage}}} I\ built for my advisor in this
lab. Constructed/maintained machine's basic electricity and plumbing.
\end{itemize}

\begin{description}
\item \quad \underline{\textbf{Lecturer}}{\normalsize \ at USC, September
2010 -- May 2011}
\end{description}

\begin{itemize}
\item I was given the opportunity to lecture two semesters of introductory
physics course to medical-school undergraduates in University of Southern
California. The first class has sixty students and the second has ninety.
Official course evaluation can be provided.
\end{itemize}

\begin{description}
\item \quad \underline{\textbf{Research Assistant}}{\normalsize \ at Georgia
Institute of Technology, September 2003 -- December 2009}
\end{description}

\begin{itemize}
\item Completed highest accurate and most comprehensive measurements of
static electric dipole polarizability of neutral sodium clusters ( Na$_{1}$%
\symbol{126}Na$_{250}$) to date, at a low temperature 20 $\unit{K}$. The
polarizability of certain unattended cluster sizes exhibits features of
electronic shell structure (aka magic numbers). This result matched with
previous measurements of separation energy on the same clusters by another
group. Theoretical explanation is still anticipated for this unexpected
finding.\QTR{smallcapitalgrey}{\cite%
{Bowlan:2011:article-How-Metallic-are-Small-Sodium}} Also studied electric
susceptibilities of sodium cluster' oxides and hydroxide (Na$_{n}$O, Na$_{n}$%
OH). Performed numerical simulation on cluster's molecular dynamics under
electromagnetic fields.

\item Performed measurements of electric dipole moments of neutral niobium
and alloy clusters to study electron pairing effect in ferroelectric
systems, and studied their temperature effect. Similar quench and
enhancement behaviors by dopant atoms (Mg, Au, Al, O, etc.) are observed in
these ferroelectric clusters as in bulk superconductors. Distinct pairing
gap energy in odd and even cluster sizes are observed in ionization
potential measurements.\cite%
{Yin:2008:article-Electron-pairing-in-ferroelectri}

\item Performed measurements of magnetic moments, polarizabilities, and
ionization potentials, of cobalt and iron clusters (Co$_{N}$, Fe$_{N}$) at a
low 20 $\unit{K}$ temperature.\QTR{smallcapitalgrey}{\cite%
{Xu:2011:article-Metastability-of-Free-Cobalt-and-}}

\item Mastered a '\textit{position-sensitive}' time-of-flight mass
spectrometer (TOF-MS).

\item Mastered a laser vaporization/ablation source (UV/visible/IR) to
produce supersaturated metal vapor plume in a temperature-controlled helium
flow to generate metal clusters.

\item Mastered/maintained a OPO laser, a gas excimer laser (ArF, KrF, F2),
and two Nd:YAG lasers (1064 and it's 532, 355, 266 harmonics). Performed
power and energy measurements.

\item Mastered and developed a MATLAB mass-spectra analyzing program coded
in-house for analyzing cluster's mass spectra data. The program automates
files in/out and statistical analysis so it can save time on large data sets.

\item Mastered a home-written LabVIEW control program and its communication
hardware to acquire data, translate micrometers, trigger lasers \& step
motors, perform high frequency switching of high voltages, etc. Performed
synchronization on various pulsed signals and triggers for control and
communication. Installed a photon counter to our cluster beam apparatus.
\end{itemize}

\bigskip

\QTR{smallcapitalgrey}{Appointments and elected positions}

\begin{itemize}
\item Intel${\tiny \registered }$ International Science and Engineering Fair
(ISEF), 2011, Grand award judge

\item California State Science Fair, 2010;2012;2013, Judge

\item NTU soccer club, 2001-2002, President

\begin{itemize}
\item Major accomplishments: Successfully held NTU soccer leagues involving
20 teams. Me and vice-president Cheng-Hao Chien introduced an adaptive match
scheduling method and solved the long-lasting match teams' no-show problems
due to students' tight schedules and this completely eliminated event
interruption. The\ event is so successful that it has been continued to date.
\end{itemize}

\item Georgia Tech Taiwan Student Association, 2006 Fall, President
\end{itemize}

\bigskip

\QTR{smallcapitalgrey}{Job Skills}

%TCIMACRO{\TeXButton{TeX field}{\begin{multicols}{2}}}%
%BeginExpansion
\begin{multicols}{2}%
%EndExpansion

\begin{itemize}
\item Complex problem solving

\item Synthesizing information

\item Mathematical skills

\item Lab or instrumentation skills

\item Knowledge of physics principles

\item Communication

\item Good documentation skill
\end{itemize}

%TCIMACRO{\TeXButton{TeX field}{\end{multicols}}}%
%BeginExpansion
\end{multicols}%
%EndExpansion

\bigskip

\QTR{smallcapitalgrey}{Technical Skills }\textit{(years of experience):}

%TCIMACRO{\TeXButton{TeX field}{\begin{multicols}{2}}}%
%BeginExpansion
\begin{multicols}{2}%
%EndExpansion

\begin{itemize}
\item Spectral data analysis (6)

\item Physical modeling (6)

\item Molecular beam photon and electron spectroscopy (9)

\item High vacuum systems (9)

\item Cryogenics (9)

\item Mass spectrometers: TOF (9), QMA (3), leak detector (4)

\item Sputtering technique: laser ablation (6), magnetron ion sputtering (3)

\item Numerical simulation (1)

\item Tool machining: lathe and mill (5)
\end{itemize}

\emph{Computer-aided skills}

\begin{itemize}
\item MATLAB (8), Python (3)

\item ProgeCAD, AutoDesk Inventor 3D (5)

\item LabVIEW automation \& data acquisition (4)

\item SIMION 7 3D (1)

\item Scientific Workplace (LATEX typeset) (4)

\item Origin, Canvas (5), Matplotlib (1), PStricks (\TEXTsymbol{<}1)

\item MS Word, Excel, Powerpoint (\TEXTsymbol{>}10)
\end{itemize}

%TCIMACRO{\TeXButton{TeX field}{\end{multicols}}}%
%BeginExpansion
\end{multicols}%
%EndExpansion

\bigskip

\QTR{smallcapitalgrey}{Publications\nocite%
{Xu:2007:article-Nonclassical-dipoles-in-cold-niob}}

\bibliographystyle{gatech-thesis}
\bibliography{MBE2py}

\bigskip

\QTR{smallcapitalgrey}{References}: Professor \href{http://physics.usc.edu/~kresin/%
}{\underline{\color{blue}\smash{Vitaly Kresin}}}, Professor \href{http://dornsife.usc.edu/cf/phys/faculty_display.cfm?person_ID=1003141%
}{\underline{\color{blue}\smash{Douglas Burke}}}, Professor \href{https://www.physics.gatech.edu/user/walter-de-heer%
}{\underline{\color{blue}\smash{Walter deHeer}}}\bigskip

\QTR{smallcapitalgrey}{Personal:}

\textsf{Language: English (proficient second language), Chinese (native)}

\textsf{My blog: \href{http://whymranderson.blogspot.com}{\underline{%
\color{blue}\smash{http://whymranderson.blogspot.com/}}}.}

\textsf{Residence \& Work Permit: Taiwan Resident Certificate and Work
Permit for Foreigners (under Employment Service Act regulation number 51) }

\bigskip

\QTR{smallcapitalgrey}{Community:}

StackExchange.com

Github.com/matplotlib

\bigskip

\QTR{smallcapitalgrey}{Recent Participations, Presentations \& Conferences}

\textsf{2015.9 \U{9ad8}\U{503c}\U{5316}\U{8eca}\U{7528}\U{7167}\U{660e}%
\U{5546}\U{6a5f}\U{8207}\U{6280}\U{8853}\U{7814}\U{8a0e}\U{6703}\U{ff0c}%
\U{7d93}\U{6fdf}\U{90e8}\U{5357}\U{53f0}\U{7063}\U{5275}\U{65b0}\U{5712}%
\U{5340}}

\textsf{2015.8 GS Simulation\U{8a0e}\U{8ad6}\U{63a8}\U{5ee3}\U{6703}\U{ff0c}%
\U{53f0}\U{5357}\U{5e02}\U{7acb}\U{7e3d}\U{5716}\U{66f8}\U{9928}}

\textsf{2015.7 GS Simulation presented in Tainan Python's User Group in
isrlab X hackerspace}

\textsf{2015.7 GS Simulation presented in Academia Sinica CITI center Dr.
Tsao's group as a special talk}

\textsf{2015.5 GS Simulation presented in NCKU Mechanical Dept. professor
Lan's advanced mechanics class as a special talk}

\textsf{2015.4 \U{5168}\U{7403}\U{88fd}\U{9020}\U{696d}2015\U{667a}\U{52d5}%
\U{5316}\U{9ad8}\U{5cf0}\U{8ad6}\U{58c7}\U{ff0c}\U{81fa}\U{5357}\U{751f}%
\U{6d3b}\U{7f8e}\U{5b78}\U{9928}}

\textsf{2014.12 \U{5de5}\U{7814}\U{9662}\U{5357}\U{5206}\U{9662}\U{5275}%
\U{65b0}\U{6280}\U{8853}\U{767c}\U{8868}\U{66a8}\U{7814}\U{8a0e}\U{6703}%
\U{ff0c}\U{5de5}\U{7814}\U{9662}\U{516d}\U{7532}\U{9662}\U{5340}}

\textsf{2013.12 ITRI South Campus Technology and Innovation Expo, Tainan,
Taiwan}

\textsf{2013.4 California State Science Fair: Judge, Product Science}

\textsf{2013.3 Comsol Multiphysics Workshop, El Segundo}

\textsf{2013.3 MathWorks}$\registered $\textsf{\ virtual conference 2013}

\textsf{2012.10 AVS SCC symposium @UCLA}

\textsf{2012.6 APS DAMOP, abstract \#P3.00001, \href{https://sites.google.com/site/vitalyslab/home/damop2012%
}{\underline{\color{blue}\smash{photos}}}}

\QTR{lightgrey}{\rule[-0.03in]{7.5in}{0.01in}}

\hspace{5.25in}CV of Anthony Liang, Nov. 2014

\end{document}
