% EPC flow charts
% Author: Fabian Schuh
\documentclass{article}
\usepackage[inner=0.5in,outer=0.5in,top=0.5in,bottom=0.5in]{geometry}
\usepackage{xeCJK} % 分開設置中英文字型
\setCJKmainfont{微軟正黑體} % 設定中文字型
\usepackage{enumitem}
\usepackage{tikz}
\usetikzlibrary{positioning,arrows}
\usetikzlibrary{shapes.multipart}

\tikzset{
    state/.style={
           rectangle,
           rounded corners,
           draw=black,
           minimum height=2em,
           inner sep=2pt,
           text centered,
           },
}

\setlist[enumerate]{topsep = 0pt, noitemsep}%remove extra empty spaces above enumerate envr, and make the items more compact

\begin{document}
\tikzstyle{abstract}=[rectangle, draw=black, rounded corners,  anchor=center, text width=3cm,text centered,rectangle split, rectangle split parts=2]

\begin{tikzpicture}[item/.style={draw=black, rounded corners,  anchor=center, text width=8.5cm,align = center,rectangle split, rectangle split parts=#1, rectangle split part align={center, left} }]

\begin{scope}[node distance=0.5cm]
\node [item=4](a) at (1,1){%
            \textbf{擴充}
            \nodepart{two}
            \begin{enumerate}
            	\item gyro sensor轉動驗證,準確性歸納。應用?
            	\item Chatter ring, fidget spinner模擬驗證。
            	\item 軟體持續改良
            	\item 結合疊格來做立體作圖?
            	\item 教材製作
            	\item 增加藍芽傳輸(沒有wifi時)
            	\item 與ejss準確度比較
            	\item 偏移誤差近似與改進(與ejss,sensor轉動驗證均有關係)
           \end{enumerate}
           \nodepart{three}\textbf{待修正事項}
	\nodepart{four}
	\begin{enumerate}
            \item GUI修補
            \item GUI module加說明
            \item py2exe作業系統可執行檔,軟體化
	\end{enumerate}
            };
\node [state,below=of a](b)   
{git log}edge [<-,>=stealth'] (a);

\node [item=2,below=of b](c)
{%
\textbf{模擬軟體擴充}
\nodepart{two}
\begin{enumerate}
            \item 增加新模擬(python)
            \item 寫程式說明(comment \& documentation \& update log)
\end{enumerate}
}edge [<-,>=stealth'] (b);

\node [state,below=of c](d)   
{git log,做成exe}edge [<-,>=stealth'] (c);

\node [item=2,below=of d](e)
{%
\textbf{pdf文件}
\nodepart{two}
\begin{enumerate}
            \item 製作要放入文件的圖片(疊格技術運用)
            \item 以Scientific Word軟體撰寫擴充的新內容,然後使用本工作室開發的自動化流程來進行章節或全書的編譯(使用Xelatex引擎),來產生PDF技術文件。
            \item 寫軟體操作說明
\end{enumerate}
}edge [<-,>=stealth'] (d);

\node [state,below=of e](f)   
{git log,列印,技術文件裝訂}edge [<-,>=stealth'] (e);

\node [item=2,below=of f](g)
{%
\textbf{加入網頁}
\nodepart{two}
\begin{enumerate}
            \item 製作網頁可觀看的示範影片
            \item Landing page的修改。以LYX增加Landing page的新內容,然後利用LYX的HTML export功能,若有數學要加上使用Mathjax的選項。然後git push上pyanywere網站。
            \item 有時也需要把技術文件po上網頁,做成blog文章,此時就利用SW中的export HTML功能。需要時要微調。例如production chapter。
\end{enumerate}
}edge [<-,>=stealth'] (f);


%draw the curve line
%\draw [->,out = 180, in = 0] (h) to  (f);

\node [inner sep = 0pt, right=of g](gic1){};
\node [inner sep = 0pt, right=of a](gic2){};

\node [item=2,right=of gic2](h)
{%
\textbf{推廣(找專人負責)}
\nodepart{two}
\begin{enumerate}
            \item 陀螺儀應用體驗店
            \item 廣告文宣製作,以landing page製作flyer。將LYX中landing page的內容export成xetex碼,然後在texlive按照我們整理好的方式修改成Texlive的xelatex可編譯碼,然後編譯成PDF flyer。列印後發放。
            \item gss logo不同型式的製作,名片電繡等製作。
\end{enumerate}
};%edge [<-,>=stealth'] (f);


\draw [rounded corners] (g) --  (gic1.center) -- (gic2.center) -- (h);


\end{scope}
\end{tikzpicture}
\end{document}