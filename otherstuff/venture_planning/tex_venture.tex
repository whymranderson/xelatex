% EPC flow charts
% Author: Fabian Schuh
\documentclass{article}
\usepackage[inner=0.5in,outer=0.5in,top=0.25in,bottom=0.25in]{geometry}
\usepackage{xeCJK} % 分開設置中英文字型
\setCJKmainfont{微軟正黑體} % 設定中文字型
\usepackage{enumitem}
\usepackage{graphicx}
\usepackage{tikz}
\usetikzlibrary{positioning,arrows}
\usetikzlibrary{shapes.multipart}

\tikzset{
    state/.style={
           rectangle,
           rounded corners,
           draw=black,
           minimum height=2em,
           inner sep=2pt,
           text centered,
           },
}
\tikzset{
    phantom/.style={
           rectangle,
           rounded corners,
           minimum height=2em,
           inner sep=2pt,
           text centered,
           },
}
\setlist[enumerate]{topsep = 0pt, noitemsep}%remove extra empty spaces above enumerate envr, and make the items more compact

\begin{document}
\tikzstyle{abstract}=[rectangle, draw=black, rounded corners,  anchor=center, text width=3cm,text centered,rectangle split, rectangle split parts=2]

\begin{tikzpicture}[item/.style={draw=black, rounded corners,  anchor=center, text width=8.5cm,align = center,rectangle split, rectangle split parts=#1, rectangle split part align={center, left} }]

\begin{scope}[node distance=5mm and 5mm]

\node [ item=2](a) at (1,1) {%
            \textbf{未完成方向}
            \nodepart{two}
            \begin{enumerate}
            	\item 應用subdoc讓case\_g20.tex及其他可以獨立運行,就不用去擔心編譯時期相對路徑找不到檔問題。
            	\item share community, tikz, LYX doc, ...
            	\item Math.lyx與.pdf清楚化
            	\item LYX做webpost時的標題列css模組化
            	\item 軟體學習、紀錄、分享。
            	\item sciencefair typeset conti, 將疊格繪圖方程modulize.
           \end{enumerate}
            };

\node [phantom, inner sep = 0pt, right=of a.north east](center_point){};

\node [above = of center_point, align = center] (title){\includegraphics[width=0.4\textwidth]{../../figs/tex_logo.png}\\ \LARGE 事業部規劃};


\node [ item=2, right = of center_point, anchor = north west](unique) {%
            \textbf{特色}
            \nodepart{two}
            \begin{enumerate}
            	\item 使用LYX與SW,高度自動化(省下很多細微瑣碎步驟)。
            	\item 與不同應用做方便性整合,如django,web blog等等。
           \end{enumerate}
            };

\node [ item=2,below = of a](small) {%
            \textbf{未完成細項}
            \nodepart{two}
            \begin{enumerate}
            	\item 總經排
            	\item chemgreek目前是放在同目錄下,需裝成package。
            	\item LYX export包含有被伺服圖檔的html檔,目前已有一寫下的流程,但還是要做成自動化較方便。
            	\item vec A不對稱問題?
            	\item 順著陀螺外型的powered by LYX SW字串,path?
            	\item 加入tikz說明書,node position,anchor參數必須要在above right之類參數之後才會有效。
		\item texlive 2017 installed, but texworks fails to start?! instead using gvim now.
	    \end{enumerate}
            };

\node [ item=2, below = of unique](LYXcustomization) {%
            \textbf{LYX特殊客製化紀錄}
            \nodepart{two}
            \begin{enumerate}
            	\item 在LYX中直接加入ScriBD的pdf preview視窗指令,LYX export html後即可成為一具有Scribd pdf預覽視窗的網頁。範例請見網站的疊格服務介紹網頁。
           	\begin{enumerate}
           		\item 步驟:使用LYX的program listing指令,在指令中貼上Scribd的embed pdf指令,存檔。使用LYX的export html指令,我有新增一複製指令,所以html檔除了存在原本檔案位置,也會存到django的template資料夾。去template夾中,打開LYX2HTML\_str\_replace.py檔,確認fname是該html檔名,關閉py檔。執行python LYX2HTML\_str\_replace.py檔。
           	\end{enumerate}
           \end{enumerate}
            };

\node [ item=2, below = of LYXcustomization](SWcustomization) {%
            \textbf{SW特殊客製化紀錄}
            \nodepart[text width = 8.2cm]{two}
            \begin{enumerate}
            	\item 在SW中若要加入連結,由於SW本身的inserted hypertext所插入的msihyperref指令在xelatex下編譯會多產生兩個??,因此只好用平常的hyperref指令,並且這樣的話也可以很方便的把連結加顏色換型變得很漂亮。不過這樣的話以後要輸入就要去複製一個平常的hyperref指令,然後更改其中的連結成想要的。是有點麻煩,之後再想辦法。
           \end{enumerate}
            };


\end{scope}
\end{tikzpicture}
\end{document}
