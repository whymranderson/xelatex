% EPC flow charts
% Author: Fabian Schuh
\documentclass{article}
\usepackage[inner=0.5in,outer=0.5in,top=0.25in,bottom=0.25in]{geometry}
\usepackage{xeCJK} % 分開設置中英文字型
\setCJKmainfont{微軟正黑體} % 設定中文字型
\usepackage{enumitem}
\usepackage{graphicx}
\usepackage{tikz}
\usetikzlibrary{positioning,arrows}
\usetikzlibrary{shapes.multipart}

\tikzset{
    state/.style={
           rectangle,
           rounded corners,
           draw=black,
           minimum height=2em,
           inner sep=2pt,
           text centered,
           },
}
\tikzset{
    phantom/.style={
           rectangle,
           rounded corners,
           minimum height=2em,
           inner sep=2pt,
           text centered,
           },
}
\setlist[enumerate]{topsep = 0pt, noitemsep}%remove extra empty spaces above enumerate envr, and make the items more compact

\begin{document}
\tikzstyle{abstract}=[rectangle, draw=black, rounded corners,  anchor=center, text width=3cm,text centered,rectangle split, rectangle split parts=2]

\begin{tikzpicture}[item/.style={draw=black, rounded corners,  anchor=center, text width=8.5cm,align = center,rectangle split, rectangle split parts=#1, rectangle split part align={center, left} }]

\begin{scope}[node distance=5mm and 5mm]

\node [ item=4](a) at (1,1) {%
            \textbf{weekly routine}
            \nodepart{two}
            \begin{enumerate}
            	\item arrange for collected readings from bloggers to update essay post.
            	\item pyany's grabing and pulling need to be automized.
            	\item admire, find and fix any typo or mistake.
            	\item expand some posts, ex. the csgo post.
            	\item look up collected volcab and write post.(四五月有不少還沒查的)
            	\item still need to close out DIY post.
           \end{enumerate}
           \nodepart{three}\textbf{特色}
	\nodepart{four}
            \begin{enumerate}
            	\item 使用LYX與SW,高度自動化產生django-compatible html(省下很多細微瑣碎步驟)。
            	\item 讓SW與LYX產生的網頁能夠正確顯示。如SW export links時有點問題,要記錄下可行步驟,並嘗試自動化。
           \end{enumerate}
            };

\node [phantom, inner sep = 0pt, right=of a.north east](center_point){};

\node [above = of center_point, align = center] (title){\parbox[c][][c]{0.1\textwidth}{\includegraphics[width=0.1\textwidth]{../../figs/logo_June27_2016.png}}\parbox[c][][c]{10cm}{\Huge website planning}};


\node [ item=2, right = of center_point, anchor = north west](small) {%
            \textbf{重點步驟}
            \nodepart[text width = 8.2cm]{two}
            \begin{enumerate}
            	\item when generating html files from LYX for the website, a few steps need to be followed in order to make it work. Most webpages on this site are generated by LYX, even the django-compatible html pages. In order to make this work a few things has to be followed during the LYX exporting process.
            	\item To generate webpost that has the head banner, after LYX html export, run LYX2HTML\_header\_replace.py
            	\item To generate webpost that has Scribd pdf preview embed code, after LYX html export, run LYX2HTML\_str\_replace.py
           \end{enumerate}
            };

\end{scope}
\end{tikzpicture}
\end{document}