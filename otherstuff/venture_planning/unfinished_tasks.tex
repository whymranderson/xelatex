% EPC flow charts
% Author: Fabian Schuh
\documentclass{article}
\usepackage{myflowchart}

\begin{document}

\begin{tikzpicture}

\begin{scope}[node distance=5mm and 5mm]

\node [ item=2](a) at (1,1) {%
            \textbf{2017未完成的較複雜項目}
            \nodepart[text width = 8.2cm]{two}
            \begin{enumerate}
            	\item 車燈gyro儀
             	\item django doc improve $\bullet\bullet\circ$
		\begin{enumerate}
			\item 加新model到full example
			\item 可小修m2m section(extra field那段)
			\item django static file那邊的說明可以再清楚一點?說明django static file的做法是,他希望你把需要伺服的圖檔放在你引用此圖的檔案(如你的html pages)的相同資料夾下,建立一個名為static的資料夾,然後圖檔放在此static資料夾之下(可以在子目錄下)。你說,那我有好多個html檔案都在不同地方,不同資料夾裡面,這些檔案都引用不同的圖檔的話,那不是有一堆static資料夾四散各地?是的,就是這樣。django在執行collectstatic指令時,就是把四散各處的static資料夾以及裡面的需要伺服的圖檔,全部copy到同一個static資料夾下,而此static資料夾會放在我們指定的被伺服的資料夾。你說,那html檔裡的圖檔連結又不是指向這個被伺服的資料夾,對,他會指向之前四散各地的static資料夾連結嘛。但django還會做一個步驟。他會把在html中遇到的圖檔連結,只要遇到有static字串包含在裡面,就會把static前面的超連結字串取代為我們設定的伺服連結。嘿,這樣就可以連到需要的圖檔囉,對吧。劃出例子?
		\end{enumerate}
            	\item 椅子削平回家教人
            	\item 滾筒洗衣機要清洗
            	\item 印表機level up
		\begin{enumerate}
			\item 等十分鐘熱了再印顏色較深
			\item 圖形有白色條紋,見fidget gyro圖
		\end{enumerate}
		\item 有水電的地面工作區域 $\bullet\bullet\bullet\circ\circ\circ$
		\item high definition投影
                \item vr,gyro體驗館,原力推starwar?
		\item 乾洗液(目前溫水加乾淨布擦)
                \item 相機插電
                \item 好筆的墨水,整理墨水post,入網站。
		\item 空壓機排水閥修
                \item bank order checks
                \item 英文整理還有訂下一個時間去查
                \item 回spyder+manage.py as startupscript問題
		\item 印烤箱貼紙,製作
		\item 筆記本的強力夾邊設計及製作,要不會卡其他書以及書包,現在的夾子會卡到然後破壞其他書及書包。
		\item 低收補助寫下文章
		\item oldnew文章寫成報案書

	    \end{enumerate}
            };

\node [ item=2, right = of a.north east, anchor = north west](continue) {%
            \textbf{continue..}
            \nodepart[text width = 8.2cm]{two}
            \begin{enumerate}
		\setcounter{enumi}{22}
		\item T60相關
		\begin{enumerate}
			\item T60電池找(型號已找到,4-17-2017 note)
			\item 液晶模組diy post未完成的另一半
		\end{enumerate}
            	\item load GUI demo setting 1234.
            	\item 慢速轉10幾圈,差五度內。快速轉十圈,可差到30度。可能原因見5/10/17。run\_avg有問題?
            	\item swp加link要變容易。
            	\item 仿宋體試5/16/17
            	\item science fair typeset continue.
            	\item 特殊的tex學習法寫下,排版是一門專業
		\begin{enumerate}
			\item xeCJK模組改良,自動抓取常用的中文字型,讓初學者可以引用package後就可以使用,不用再去設定字形。這會讓中文化更加方便容易與普及。多數網路例子都是英文編譯模式,中文模式要自己選字形,但怎麼選?中文使用設置要變容易,預設大家都有的幾個字型?learn ifexist isthenesle,check ifexist in moderncv package。
			\item 心法:學tex人會覺得,常常會遇到連一個簡單的問題都會覺得怎麼那麼麻煩,比如說啊我今天要最快速排出一個非常簡單的文件,所以用usepackage{minimal}應該就夠了,我只需要最簡單的排版就好,結果發現一直編譯不過,卡在字形大小tiny出錯。結果居然是在minimal package裡面沒有定義tiny字形大小。心法就是,就像做硬體一樣,要做一精密精細的作品所需要的工具本來就很多很繁瑣,有時你認為簡單的東西,只要有一個小地方你沒有想到,但這小地方其實需要不少工具,那就不能怪工具不好,是你自己沒想到而已。多數人以為做東西很簡單,那是因為他們只會用幾樣很簡單的工具,就想要應付各式各樣的狀況,結果只做得出一些很爛很醜又不耐用的作品。做東西都是如此,忌氣急敗壞。心法。
		\end{enumerate}            	
		\item gui module sphinx doc comment補齊與整理。
		\item db\_note整理,看如何推銷與賣(可能太難,因為不是你的領域,離你太遙遠,無認識的人)。
		\item db模型改良,足球聯賽應用。但現在又變成大會定時間,是否可應用上大會定?
		\item py2exe繼續否?
		\item 滾動碰撞模型建立持續否?是。
		\item share pstricks and tikz graphs
		\item logo3D OpenGL engrave continue
		\item csgo event scraping continue?
		\item python import問題,學習os insert?
		\item 目前matplotlib2pgf text label margin是用annotation,但text method好像有offset還是leftright margin可用?
		\item note有不少要修改,然後要重印note $\bullet\bullet\bullet\circ\circ\circ$
		\item gitlogs page目前還需要跑grab\_gitlogs.bat去收集並產生文字檔,是否可以自動化,一禮拜做一次?
		\item 單軸角速度計進階版證明。8/16/17
		\item $L,\omega,z$互相繞的想法有錯,make a demo,8/16/17
		\item $\omega,z$ not centered around L check
           \end{enumerate}
            };
\end{scope}
\end{tikzpicture}


\begin{tikzpicture}
\begin{scope}[node distance=5mm and 5mm]


\node [ item=2](a) at (1,1) {%
	\textbf{2013未完成項目}
	\nodepart[text width = 8.2cm]{two}
	\begin{enumerate}
		\item 5/30/13 滾動角解析式停住
		\item 8/26/13 嘗試compost加入DIYpost,並加並沒有那麼久
		\item 10/1/13 事實上一點點收入是有達成的,list
		\item 12/15/13, 10/1/13, 想要以高技術賺錢或想以技術養一個家,換個說法就行的通了,想要以高技術來交換以物易物,這是很合理的,因為高技術代表高效率。這是因為現在獲得錢的交換條件不平等。
		\item 12/26/13 值得做coin滾動實驗。
	\end{enumerate}
        };

\node [ item=2, below of = a](2017) {%
	\textbf{2017未完成項目}
	\nodepart[text width = 8.2cm]{two}
	\begin{enumerate}
		\item 5/30/13 滾動角解析式停住
		\item 8/26/13 嘗試compost加入DIYpost,並加並沒有那麼久
		\item 10/1/13 事實上一點點收入是有達成的,list
		\item 12/15/13, 10/1/13, 想要以高技術賺錢或想以技術養一個家,換個說法就行的通了,想要以高技術來交換以物易物,這是很合理的,因為高技術代表高效率。只是現在獲得錢的交換條件不平等。
		\item 12/26/13 值得做coin滾動實驗。
	\end{enumerate}
        };

\end{scope}
\end{tikzpicture}

\end{document}
