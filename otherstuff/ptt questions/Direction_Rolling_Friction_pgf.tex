
\documentclass{article}
%%%%%%%%%%%%%%%%%%%%%%%%%%%%%%%%%%%%%%%%%%%%%%%%%%%%%%%%%%%%%%%%%%%%%%%%%%%%%%%%%%%%%%%%%%%%%%%%%%%%%%%%%%%%%%%%%%%%%%%%%%%%%%%%%%%%%%%%%%%%%%%%%%%%%%%%%%%%%%%%%%%%%%%%%%%%%%%%%%%%%%%%%%%%%%%%%%%%%%%%%%%%%%%%%%%%%%%%%%%%%%%%%%%%%%%%%%%%%%%%%%%%%%%%%%%%
\usepackage[ignoreall]{geometry}
\usepackage{graphicx}
\usepackage{pgf}
\usepackage{multicol}

%TCIDATA{OutputFilter=LATEX.DLL}
%TCIDATA{Version=5.00.0.2606}
%TCIDATA{<META NAME="SaveForMode" CONTENT="1">}
%TCIDATA{BibliographyScheme=Manual}
%TCIDATA{Created=Wednesday, April 06, 2016 11:13:15}
%TCIDATA{LastRevised=Friday, May 27, 2016 16:48:06}
%TCIDATA{<META NAME="GraphicsSave" CONTENT="32">}
%TCIDATA{<META NAME="DocumentShell" CONTENT="Standard LaTeX\Blank - Standard LaTeX Article">}
%TCIDATA{CSTFile=40 LaTeX article.cst}

\newtheorem{theorem}{Theorem}
\newtheorem{acknowledgement}[theorem]{Acknowledgement}
\newtheorem{algorithm}[theorem]{Algorithm}
\newtheorem{axiom}[theorem]{Axiom}
\newtheorem{case}[theorem]{Case}
\newtheorem{claim}[theorem]{Claim}
\newtheorem{conclusion}[theorem]{Conclusion}
\newtheorem{condition}[theorem]{Condition}
\newtheorem{conjecture}[theorem]{Conjecture}
\newtheorem{corollary}[theorem]{Corollary}
\newtheorem{criterion}[theorem]{Criterion}
\newtheorem{definition}[theorem]{Definition}
\newtheorem{example}[theorem]{Example}
\newtheorem{exercise}[theorem]{Exercise}
\newtheorem{lemma}[theorem]{Lemma}
\newtheorem{notation}[theorem]{Notation}
\newtheorem{problem}[theorem]{Problem}
\newtheorem{proposition}[theorem]{Proposition}
\newtheorem{remark}[theorem]{Remark}
\newtheorem{solution}[theorem]{Solution}
\newtheorem{summary}[theorem]{Summary}
\newenvironment{proof}[1][Proof]{\noindent\textbf{#1.} }{\ \rule{0.5em}{0.5em}}
\input{../../tcilatex}
\setlength{\columnseprule}{1pt}



\begin{document}


\part{The direction of rolling friction w/o slippling}

How to determine the direction of friction force acting on a rolling object?
This is important and is ensential to solving the dynamics of rolling
motions.

\begin{case}
Round object freely rolling down the hill\newline
\newline
The only force that makes the object rotate is friction so friction has to
go up the hill. This friction force is exerted on the wheel by the slope.

\input{../../../../Scripts/cordtrans/cases_fig_only/whatever.pgf}
\end{case}

\newpage

\begin{case}
Object is forced to roll up the hill initially but external force is removed
once the object is going upward. We are considering the later part of the
motion when the external force is removed, so only gravitation is in place.
The wheel is still rolling up the hill.\newline
\newline
The rotation of the object slows down as it climbs up the hill. Friction is
the only force that produces a torque to slow down the rotation. So it needs
to go against the rotating direction. So the friction force acting on the
wheel is up the hill.

%% Creator: Matplotlib, PGF backend
%%
%% To include the figure in your LaTeX document, write
%%   \input{<filename>.pgf}
%%
%% Make sure the required packages are loaded in your preamble
%%   \usepackage{pgf}
%%
%% Figures using additional raster images can only be included by \input if
%% they are in the same directory as the main LaTeX file. For loading figures
%% from other directories you can use the `import` package
%%   \usepackage{import}
%% and then include the figures with
%%   \import{<path to file>}{<filename>.pgf}
%%
%% Matplotlib used the following preamble
%%   \usepackage{fontspec}
%%   \setmainfont{Times New Roman}
%%   \setsansfont{Verdana}
%%   \setmonofont{Courier New}
%%
\begingroup%
\makeatletter%
\begin{pgfpicture}%
\pgfpathrectangle{\pgfpointorigin}{\pgfqpoint{4.000000in}{4.000000in}}%
\pgfusepath{use as bounding box}%
\begin{pgfscope}%
\pgfsetbuttcap%
\pgfsetroundjoin%
\definecolor{currentfill}{rgb}{1.000000,1.000000,1.000000}%
\pgfsetfillcolor{currentfill}%
\pgfsetlinewidth{0.000000pt}%
\definecolor{currentstroke}{rgb}{1.000000,1.000000,1.000000}%
\pgfsetstrokecolor{currentstroke}%
\pgfsetdash{}{0pt}%
\pgfpathmoveto{\pgfqpoint{0.000000in}{0.000000in}}%
\pgfpathlineto{\pgfqpoint{4.000000in}{0.000000in}}%
\pgfpathlineto{\pgfqpoint{4.000000in}{4.000000in}}%
\pgfpathlineto{\pgfqpoint{0.000000in}{4.000000in}}%
\pgfpathclose%
\pgfusepath{fill}%
\end{pgfscope}%
\begin{pgfscope}%
\pgfpathrectangle{\pgfqpoint{0.500000in}{0.400000in}}{\pgfqpoint{3.100000in}{3.200000in}} %
\pgfusepath{clip}%
\pgfsetbuttcap%
\pgfsetroundjoin%
\definecolor{currentfill}{rgb}{0.000000,0.000000,1.000000}%
\pgfsetfillcolor{currentfill}%
\pgfsetlinewidth{1.003750pt}%
\definecolor{currentstroke}{rgb}{0.000000,0.000000,0.000000}%
\pgfsetstrokecolor{currentstroke}%
\pgfsetdash{}{0pt}%
\pgfpathmoveto{\pgfqpoint{2.871466in}{2.163611in}}%
\pgfpathlineto{\pgfqpoint{2.771677in}{2.026452in}}%
\pgfpathlineto{\pgfqpoint{2.744982in}{2.072688in}}%
\pgfpathlineto{\pgfqpoint{1.999238in}{1.642132in}}%
\pgfpathlineto{\pgfqpoint{1.983738in}{1.668979in}}%
\pgfpathlineto{\pgfqpoint{2.729482in}{2.099535in}}%
\pgfpathlineto{\pgfqpoint{2.702788in}{2.145771in}}%
\pgfpathclose%
\pgfusepath{stroke,fill}%
\end{pgfscope}%
\begin{pgfscope}%
\pgfpathrectangle{\pgfqpoint{0.500000in}{0.400000in}}{\pgfqpoint{3.100000in}{3.200000in}} %
\pgfusepath{clip}%
\pgfsetbuttcap%
\pgfsetroundjoin%
\definecolor{currentfill}{rgb}{0.000000,0.000000,1.000000}%
\pgfsetfillcolor{currentfill}%
\pgfsetlinewidth{1.003750pt}%
\definecolor{currentstroke}{rgb}{0.000000,0.000000,0.000000}%
\pgfsetstrokecolor{currentstroke}%
\pgfsetdash{}{0pt}%
\pgfpathmoveto{\pgfqpoint{1.647044in}{0.639444in}}%
\pgfpathlineto{\pgfqpoint{1.578155in}{0.794444in}}%
\pgfpathlineto{\pgfqpoint{1.631544in}{0.794444in}}%
\pgfpathlineto{\pgfqpoint{1.631544in}{2.252151in}}%
\pgfpathlineto{\pgfqpoint{1.662544in}{2.252151in}}%
\pgfpathlineto{\pgfqpoint{1.662544in}{0.794444in}}%
\pgfpathlineto{\pgfqpoint{1.715933in}{0.794444in}}%
\pgfpathclose%
\pgfusepath{stroke,fill}%
\end{pgfscope}%
\begin{pgfscope}%
\pgfpathrectangle{\pgfqpoint{0.500000in}{0.400000in}}{\pgfqpoint{3.100000in}{3.200000in}} %
\pgfusepath{clip}%
\pgfsetbuttcap%
\pgfsetroundjoin%
\definecolor{currentfill}{rgb}{1.000000,0.000000,0.000000}%
\pgfsetfillcolor{currentfill}%
\pgfsetlinewidth{1.003750pt}%
\definecolor{currentstroke}{rgb}{1.000000,0.000000,0.000000}%
\pgfsetstrokecolor{currentstroke}%
\pgfsetdash{}{0pt}%
\pgfpathmoveto{\pgfqpoint{3.317511in}{3.216595in}}%
\pgfpathlineto{\pgfqpoint{3.224643in}{2.964113in}}%
\pgfpathlineto{\pgfqpoint{3.160060in}{3.075975in}}%
\pgfpathlineto{\pgfqpoint{2.414316in}{2.645419in}}%
\pgfpathlineto{\pgfqpoint{2.371260in}{2.719994in}}%
\pgfpathlineto{\pgfqpoint{3.117004in}{3.150549in}}%
\pgfpathlineto{\pgfqpoint{3.052421in}{3.262411in}}%
\pgfpathclose%
\pgfusepath{stroke,fill}%
\end{pgfscope}%
\begin{pgfscope}%
\pgfpathrectangle{\pgfqpoint{0.500000in}{0.400000in}}{\pgfqpoint{3.100000in}{3.200000in}} %
\pgfusepath{clip}%
\pgfsetrectcap%
\pgfsetroundjoin%
\pgfsetlinewidth{1.003750pt}%
\definecolor{currentstroke}{rgb}{0.000000,0.000000,1.000000}%
\pgfsetstrokecolor{currentstroke}%
\pgfsetdash{}{0pt}%
\pgfpathmoveto{\pgfqpoint{0.500000in}{0.794444in}}%
\pgfpathlineto{\pgfqpoint{3.482976in}{0.794444in}}%
\pgfusepath{stroke}%
\end{pgfscope}%
\begin{pgfscope}%
\pgfpathrectangle{\pgfqpoint{0.500000in}{0.400000in}}{\pgfqpoint{3.100000in}{3.200000in}} %
\pgfusepath{clip}%
\pgfsetrectcap%
\pgfsetroundjoin%
\pgfsetlinewidth{1.003750pt}%
\definecolor{currentstroke}{rgb}{0.000000,0.000000,1.000000}%
\pgfsetstrokecolor{currentstroke}%
\pgfsetdash{}{0pt}%
\pgfpathmoveto{\pgfqpoint{3.482976in}{0.794444in}}%
\pgfpathlineto{\pgfqpoint{3.482976in}{2.516667in}}%
\pgfusepath{stroke}%
\end{pgfscope}%
\begin{pgfscope}%
\pgfpathrectangle{\pgfqpoint{0.500000in}{0.400000in}}{\pgfqpoint{3.100000in}{3.200000in}} %
\pgfusepath{clip}%
\pgfsetrectcap%
\pgfsetroundjoin%
\pgfsetlinewidth{1.003750pt}%
\definecolor{currentstroke}{rgb}{0.000000,0.000000,1.000000}%
\pgfsetstrokecolor{currentstroke}%
\pgfsetdash{}{0pt}%
\pgfpathmoveto{\pgfqpoint{0.500000in}{0.794444in}}%
\pgfpathlineto{\pgfqpoint{3.482976in}{2.516667in}}%
\pgfusepath{stroke}%
\end{pgfscope}%
\begin{pgfscope}%
\pgfpathrectangle{\pgfqpoint{0.500000in}{0.400000in}}{\pgfqpoint{3.100000in}{3.200000in}} %
\pgfusepath{clip}%
\pgfsetrectcap%
\pgfsetroundjoin%
\pgfsetlinewidth{1.003750pt}%
\definecolor{currentstroke}{rgb}{0.000000,0.000000,0.000000}%
\pgfsetstrokecolor{currentstroke}%
\pgfsetdash{}{0pt}%
\pgfpathmoveto{\pgfqpoint{0.672222in}{0.794444in}}%
\pgfpathlineto{\pgfqpoint{0.671986in}{0.803458in}}%
\pgfpathlineto{\pgfqpoint{0.671279in}{0.812447in}}%
\pgfpathlineto{\pgfqpoint{0.670102in}{0.821386in}}%
\pgfpathlineto{\pgfqpoint{0.668459in}{0.830251in}}%
\pgfpathlineto{\pgfqpoint{0.666354in}{0.839019in}}%
\pgfpathlineto{\pgfqpoint{0.663793in}{0.847664in}}%
\pgfpathlineto{\pgfqpoint{0.660783in}{0.856163in}}%
\pgfpathlineto{\pgfqpoint{0.657333in}{0.864494in}}%
\pgfpathlineto{\pgfqpoint{0.653451in}{0.872632in}}%
\pgfpathlineto{\pgfqpoint{0.649149in}{0.880556in}}%
\pgfusepath{stroke}%
\end{pgfscope}%
\begin{pgfscope}%
\pgfpathrectangle{\pgfqpoint{0.500000in}{0.400000in}}{\pgfqpoint{3.100000in}{3.200000in}} %
\pgfusepath{clip}%
\pgfsetrectcap%
\pgfsetroundjoin%
\pgfsetlinewidth{1.003750pt}%
\definecolor{currentstroke}{rgb}{0.000000,0.000000,0.000000}%
\pgfsetstrokecolor{currentstroke}%
\pgfsetdash{}{0pt}%
\pgfpathmoveto{\pgfqpoint{2.335933in}{2.252151in}}%
\pgfpathlineto{\pgfqpoint{2.327451in}{2.359917in}}%
\pgfpathlineto{\pgfqpoint{2.302216in}{2.465029in}}%
\pgfpathlineto{\pgfqpoint{2.260848in}{2.564900in}}%
\pgfpathlineto{\pgfqpoint{2.204367in}{2.657070in}}%
\pgfpathlineto{\pgfqpoint{2.134162in}{2.739269in}}%
\pgfpathlineto{\pgfqpoint{2.051962in}{2.809474in}}%
\pgfpathlineto{\pgfqpoint{1.959793in}{2.865955in}}%
\pgfpathlineto{\pgfqpoint{1.859922in}{2.907323in}}%
\pgfpathlineto{\pgfqpoint{1.754810in}{2.932558in}}%
\pgfpathlineto{\pgfqpoint{1.647044in}{2.941040in}}%
\pgfpathlineto{\pgfqpoint{1.539278in}{2.932558in}}%
\pgfpathlineto{\pgfqpoint{1.434165in}{2.907323in}}%
\pgfpathlineto{\pgfqpoint{1.334295in}{2.865955in}}%
\pgfpathlineto{\pgfqpoint{1.242125in}{2.809474in}}%
\pgfpathlineto{\pgfqpoint{1.159926in}{2.739269in}}%
\pgfpathlineto{\pgfqpoint{1.089721in}{2.657070in}}%
\pgfpathlineto{\pgfqpoint{1.033239in}{2.564900in}}%
\pgfpathlineto{\pgfqpoint{0.991871in}{2.465029in}}%
\pgfpathlineto{\pgfqpoint{0.966636in}{2.359917in}}%
\pgfpathlineto{\pgfqpoint{0.958155in}{2.252151in}}%
\pgfusepath{stroke}%
\end{pgfscope}%
\begin{pgfscope}%
\pgfpathrectangle{\pgfqpoint{0.500000in}{0.400000in}}{\pgfqpoint{3.100000in}{3.200000in}} %
\pgfusepath{clip}%
\pgfsetrectcap%
\pgfsetroundjoin%
\pgfsetlinewidth{1.003750pt}%
\definecolor{currentstroke}{rgb}{0.000000,0.000000,0.000000}%
\pgfsetstrokecolor{currentstroke}%
\pgfsetdash{}{0pt}%
\pgfpathmoveto{\pgfqpoint{0.958155in}{2.252151in}}%
\pgfpathlineto{\pgfqpoint{0.966636in}{2.144385in}}%
\pgfpathlineto{\pgfqpoint{0.991871in}{2.039272in}}%
\pgfpathlineto{\pgfqpoint{1.033239in}{1.939402in}}%
\pgfpathlineto{\pgfqpoint{1.089721in}{1.847232in}}%
\pgfpathlineto{\pgfqpoint{1.159926in}{1.765033in}}%
\pgfpathlineto{\pgfqpoint{1.242125in}{1.694828in}}%
\pgfpathlineto{\pgfqpoint{1.334295in}{1.638346in}}%
\pgfpathlineto{\pgfqpoint{1.434165in}{1.596979in}}%
\pgfpathlineto{\pgfqpoint{1.539278in}{1.571743in}}%
\pgfpathlineto{\pgfqpoint{1.647044in}{1.563262in}}%
\pgfpathlineto{\pgfqpoint{1.754810in}{1.571743in}}%
\pgfpathlineto{\pgfqpoint{1.859922in}{1.596979in}}%
\pgfpathlineto{\pgfqpoint{1.959793in}{1.638346in}}%
\pgfpathlineto{\pgfqpoint{2.051962in}{1.694828in}}%
\pgfpathlineto{\pgfqpoint{2.134162in}{1.765033in}}%
\pgfpathlineto{\pgfqpoint{2.204367in}{1.847232in}}%
\pgfpathlineto{\pgfqpoint{2.260848in}{1.939402in}}%
\pgfpathlineto{\pgfqpoint{2.302216in}{2.039272in}}%
\pgfpathlineto{\pgfqpoint{2.327451in}{2.144385in}}%
\pgfpathlineto{\pgfqpoint{2.335933in}{2.252151in}}%
\pgfusepath{stroke}%
\end{pgfscope}%
\begin{pgfscope}%
\pgfpathrectangle{\pgfqpoint{0.500000in}{0.400000in}}{\pgfqpoint{3.100000in}{3.200000in}} %
\pgfusepath{clip}%
\pgfsetbuttcap%
\pgfsetroundjoin%
\definecolor{currentfill}{rgb}{0.000000,0.000000,1.000000}%
\pgfsetfillcolor{currentfill}%
\pgfsetlinewidth{0.501875pt}%
\definecolor{currentstroke}{rgb}{0.000000,0.000000,0.000000}%
\pgfsetstrokecolor{currentstroke}%
\pgfsetdash{}{0pt}%
\pgfsys@defobject{currentmarker}{\pgfqpoint{-0.041667in}{-0.041667in}}{\pgfqpoint{0.041667in}{0.041667in}}{%
\pgfpathmoveto{\pgfqpoint{0.000000in}{-0.041667in}}%
\pgfpathcurveto{\pgfqpoint{0.011050in}{-0.041667in}}{\pgfqpoint{0.021649in}{-0.037276in}}{\pgfqpoint{0.029463in}{-0.029463in}}%
\pgfpathcurveto{\pgfqpoint{0.037276in}{-0.021649in}}{\pgfqpoint{0.041667in}{-0.011050in}}{\pgfqpoint{0.041667in}{0.000000in}}%
\pgfpathcurveto{\pgfqpoint{0.041667in}{0.011050in}}{\pgfqpoint{0.037276in}{0.021649in}}{\pgfqpoint{0.029463in}{0.029463in}}%
\pgfpathcurveto{\pgfqpoint{0.021649in}{0.037276in}}{\pgfqpoint{0.011050in}{0.041667in}}{\pgfqpoint{0.000000in}{0.041667in}}%
\pgfpathcurveto{\pgfqpoint{-0.011050in}{0.041667in}}{\pgfqpoint{-0.021649in}{0.037276in}}{\pgfqpoint{-0.029463in}{0.029463in}}%
\pgfpathcurveto{\pgfqpoint{-0.037276in}{0.021649in}}{\pgfqpoint{-0.041667in}{0.011050in}}{\pgfqpoint{-0.041667in}{0.000000in}}%
\pgfpathcurveto{\pgfqpoint{-0.041667in}{-0.011050in}}{\pgfqpoint{-0.037276in}{-0.021649in}}{\pgfqpoint{-0.029463in}{-0.029463in}}%
\pgfpathcurveto{\pgfqpoint{-0.021649in}{-0.037276in}}{\pgfqpoint{-0.011050in}{-0.041667in}}{\pgfqpoint{0.000000in}{-0.041667in}}%
\pgfpathclose%
\pgfusepath{stroke,fill}%
}%
\begin{pgfscope}%
\pgfsys@transformshift{1.647044in}{2.252151in}%
\pgfsys@useobject{currentmarker}{}%
\end{pgfscope}%
\end{pgfscope}%
\begin{pgfscope}%
\pgfpathrectangle{\pgfqpoint{0.500000in}{0.400000in}}{\pgfqpoint{3.100000in}{3.200000in}} %
\pgfusepath{clip}%
\pgfsetbuttcap%
\pgfsetroundjoin%
\definecolor{currentfill}{rgb}{0.000000,0.500000,0.000000}%
\pgfsetfillcolor{currentfill}%
\pgfsetlinewidth{0.501875pt}%
\definecolor{currentstroke}{rgb}{0.000000,0.000000,0.000000}%
\pgfsetstrokecolor{currentstroke}%
\pgfsetdash{}{0pt}%
\pgfsys@defobject{currentmarker}{\pgfqpoint{-0.041667in}{-0.041667in}}{\pgfqpoint{0.041667in}{0.041667in}}{%
\pgfpathmoveto{\pgfqpoint{0.000000in}{-0.041667in}}%
\pgfpathcurveto{\pgfqpoint{0.011050in}{-0.041667in}}{\pgfqpoint{0.021649in}{-0.037276in}}{\pgfqpoint{0.029463in}{-0.029463in}}%
\pgfpathcurveto{\pgfqpoint{0.037276in}{-0.021649in}}{\pgfqpoint{0.041667in}{-0.011050in}}{\pgfqpoint{0.041667in}{0.000000in}}%
\pgfpathcurveto{\pgfqpoint{0.041667in}{0.011050in}}{\pgfqpoint{0.037276in}{0.021649in}}{\pgfqpoint{0.029463in}{0.029463in}}%
\pgfpathcurveto{\pgfqpoint{0.021649in}{0.037276in}}{\pgfqpoint{0.011050in}{0.041667in}}{\pgfqpoint{0.000000in}{0.041667in}}%
\pgfpathcurveto{\pgfqpoint{-0.011050in}{0.041667in}}{\pgfqpoint{-0.021649in}{0.037276in}}{\pgfqpoint{-0.029463in}{0.029463in}}%
\pgfpathcurveto{\pgfqpoint{-0.037276in}{0.021649in}}{\pgfqpoint{-0.041667in}{0.011050in}}{\pgfqpoint{-0.041667in}{0.000000in}}%
\pgfpathcurveto{\pgfqpoint{-0.041667in}{-0.011050in}}{\pgfqpoint{-0.037276in}{-0.021649in}}{\pgfqpoint{-0.029463in}{-0.029463in}}%
\pgfpathcurveto{\pgfqpoint{-0.021649in}{-0.037276in}}{\pgfqpoint{-0.011050in}{-0.041667in}}{\pgfqpoint{0.000000in}{-0.041667in}}%
\pgfpathclose%
\pgfusepath{stroke,fill}%
}%
\begin{pgfscope}%
\pgfsys@transformshift{1.991488in}{1.655556in}%
\pgfsys@useobject{currentmarker}{}%
\end{pgfscope}%
\end{pgfscope}%
\begin{pgfscope}%
\pgftext[x=0.844444in,y=0.794444in,left,bottom]{{\sffamily\fontsize{20.000000}{24.000000}\selectfont \(\displaystyle \theta\)}}%
\end{pgfscope}%
\begin{pgfscope}%
\pgftext[x=1.647044in,y=2.252151in,right,bottom]{{\sffamily\fontsize{20.000000}{24.000000}\selectfont \(\displaystyle CM\)}}%
\end{pgfscope}%
\begin{pgfscope}%
\pgftext[x=2.737232in,y=2.086111in,left,top]{{\sffamily\fontsize{20.000000}{24.000000}\selectfont \(\displaystyle F_{fr}\)}}%
\end{pgfscope}%
\begin{pgfscope}%
\pgftext[x=1.647044in,y=0.794444in,left,top]{{\sffamily\fontsize{20.000000}{24.000000}\selectfont \(\displaystyle mg\)}}%
\end{pgfscope}%
\begin{pgfscope}%
\pgftext[x=3.138532in,y=3.285484in,left,bottom]{{\sffamily\fontsize{20.000000}{24.000000}\selectfont \(\displaystyle v\)}}%
\end{pgfscope}%
\end{pgfpicture}%
\makeatother%
\endgroup%

\end{case}

\newpage

\begin{case}
External force $F_{p}$ passing through CM point\newline
%% Creator: Matplotlib, PGF backend
%%
%% To include the figure in your LaTeX document, write
%%   \input{<filename>.pgf}
%%
%% Make sure the required packages are loaded in your preamble
%%   \usepackage{pgf}
%%
%% Figures using additional raster images can only be included by \input if
%% they are in the same directory as the main LaTeX file. For loading figures
%% from other directories you can use the `import` package
%%   \usepackage{import}
%% and then include the figures with
%%   \import{<path to file>}{<filename>.pgf}
%%
%% Matplotlib used the following preamble
%%   \usepackage{fontspec}
%%   \setmainfont{Times New Roman}
%%   \setsansfont{Verdana}
%%   \setmonofont{Courier New}
%%
\begingroup%
\makeatletter%
\begin{pgfpicture}%
\pgfpathrectangle{\pgfpointorigin}{\pgfqpoint{3.000000in}{2.000000in}}%
\pgfusepath{use as bounding box}%
\begin{pgfscope}%
\pgfsetbuttcap%
\pgfsetroundjoin%
\definecolor{currentfill}{rgb}{1.000000,1.000000,1.000000}%
\pgfsetfillcolor{currentfill}%
\pgfsetlinewidth{0.000000pt}%
\definecolor{currentstroke}{rgb}{1.000000,1.000000,1.000000}%
\pgfsetstrokecolor{currentstroke}%
\pgfsetdash{}{0pt}%
\pgfpathmoveto{\pgfqpoint{0.000000in}{0.000000in}}%
\pgfpathlineto{\pgfqpoint{3.000000in}{0.000000in}}%
\pgfpathlineto{\pgfqpoint{3.000000in}{2.000000in}}%
\pgfpathlineto{\pgfqpoint{0.000000in}{2.000000in}}%
\pgfpathclose%
\pgfusepath{fill}%
\end{pgfscope}%
\begin{pgfscope}%
\pgfpathrectangle{\pgfqpoint{0.375000in}{0.200000in}}{\pgfqpoint{2.325000in}{1.600000in}} %
\pgfusepath{clip}%
\pgfsetbuttcap%
\pgfsetroundjoin%
\definecolor{currentfill}{rgb}{1.000000,0.000000,0.000000}%
\pgfsetfillcolor{currentfill}%
\pgfsetlinewidth{1.003750pt}%
\definecolor{currentstroke}{rgb}{1.000000,0.000000,0.000000}%
\pgfsetstrokecolor{currentstroke}%
\pgfsetdash{}{0pt}%
\pgfpathmoveto{\pgfqpoint{2.357708in}{1.000000in}}%
\pgfpathlineto{\pgfqpoint{2.183333in}{0.922500in}}%
\pgfpathlineto{\pgfqpoint{2.183333in}{0.982563in}}%
\pgfpathlineto{\pgfqpoint{1.537500in}{0.982563in}}%
\pgfpathlineto{\pgfqpoint{1.537500in}{1.017437in}}%
\pgfpathlineto{\pgfqpoint{2.183333in}{1.017437in}}%
\pgfpathlineto{\pgfqpoint{2.183333in}{1.077500in}}%
\pgfpathclose%
\pgfusepath{stroke,fill}%
\end{pgfscope}%
\begin{pgfscope}%
\pgfpathrectangle{\pgfqpoint{0.375000in}{0.200000in}}{\pgfqpoint{2.325000in}{1.600000in}} %
\pgfusepath{clip}%
\pgfsetrectcap%
\pgfsetroundjoin%
\pgfsetlinewidth{1.003750pt}%
\definecolor{currentstroke}{rgb}{0.000000,0.000000,1.000000}%
\pgfsetstrokecolor{currentstroke}%
\pgfsetdash{}{0pt}%
\pgfpathmoveto{\pgfqpoint{0.568750in}{0.515625in}}%
\pgfpathlineto{\pgfqpoint{2.506250in}{0.515625in}}%
\pgfusepath{stroke}%
\end{pgfscope}%
\begin{pgfscope}%
\pgfpathrectangle{\pgfqpoint{0.375000in}{0.200000in}}{\pgfqpoint{2.325000in}{1.600000in}} %
\pgfusepath{clip}%
\pgfsetrectcap%
\pgfsetroundjoin%
\pgfsetlinewidth{1.003750pt}%
\definecolor{currentstroke}{rgb}{0.000000,0.000000,0.000000}%
\pgfsetstrokecolor{currentstroke}%
\pgfsetdash{}{0pt}%
\pgfpathmoveto{\pgfqpoint{2.021875in}{1.000000in}}%
\pgfpathlineto{\pgfqpoint{2.015912in}{1.075773in}}%
\pgfpathlineto{\pgfqpoint{1.998168in}{1.149680in}}%
\pgfpathlineto{\pgfqpoint{1.969081in}{1.219902in}}%
\pgfpathlineto{\pgfqpoint{1.929368in}{1.284708in}}%
\pgfpathlineto{\pgfqpoint{1.880005in}{1.342505in}}%
\pgfpathlineto{\pgfqpoint{1.822208in}{1.391868in}}%
\pgfpathlineto{\pgfqpoint{1.757402in}{1.431581in}}%
\pgfpathlineto{\pgfqpoint{1.687180in}{1.460668in}}%
\pgfpathlineto{\pgfqpoint{1.613273in}{1.478412in}}%
\pgfpathlineto{\pgfqpoint{1.537500in}{1.484375in}}%
\pgfpathlineto{\pgfqpoint{1.461727in}{1.478412in}}%
\pgfpathlineto{\pgfqpoint{1.387820in}{1.460668in}}%
\pgfpathlineto{\pgfqpoint{1.317598in}{1.431581in}}%
\pgfpathlineto{\pgfqpoint{1.252792in}{1.391868in}}%
\pgfpathlineto{\pgfqpoint{1.194995in}{1.342505in}}%
\pgfpathlineto{\pgfqpoint{1.145632in}{1.284708in}}%
\pgfpathlineto{\pgfqpoint{1.105919in}{1.219902in}}%
\pgfpathlineto{\pgfqpoint{1.076832in}{1.149680in}}%
\pgfpathlineto{\pgfqpoint{1.059088in}{1.075773in}}%
\pgfpathlineto{\pgfqpoint{1.053125in}{1.000000in}}%
\pgfusepath{stroke}%
\end{pgfscope}%
\begin{pgfscope}%
\pgfpathrectangle{\pgfqpoint{0.375000in}{0.200000in}}{\pgfqpoint{2.325000in}{1.600000in}} %
\pgfusepath{clip}%
\pgfsetrectcap%
\pgfsetroundjoin%
\pgfsetlinewidth{1.003750pt}%
\definecolor{currentstroke}{rgb}{0.000000,0.000000,0.000000}%
\pgfsetstrokecolor{currentstroke}%
\pgfsetdash{}{0pt}%
\pgfpathmoveto{\pgfqpoint{1.053125in}{1.000000in}}%
\pgfpathlineto{\pgfqpoint{1.059088in}{0.924227in}}%
\pgfpathlineto{\pgfqpoint{1.076832in}{0.850320in}}%
\pgfpathlineto{\pgfqpoint{1.105919in}{0.780098in}}%
\pgfpathlineto{\pgfqpoint{1.145632in}{0.715292in}}%
\pgfpathlineto{\pgfqpoint{1.194995in}{0.657495in}}%
\pgfpathlineto{\pgfqpoint{1.252792in}{0.608132in}}%
\pgfpathlineto{\pgfqpoint{1.317598in}{0.568419in}}%
\pgfpathlineto{\pgfqpoint{1.387820in}{0.539332in}}%
\pgfpathlineto{\pgfqpoint{1.461727in}{0.521588in}}%
\pgfpathlineto{\pgfqpoint{1.537500in}{0.515625in}}%
\pgfpathlineto{\pgfqpoint{1.613273in}{0.521588in}}%
\pgfpathlineto{\pgfqpoint{1.687180in}{0.539332in}}%
\pgfpathlineto{\pgfqpoint{1.757402in}{0.568419in}}%
\pgfpathlineto{\pgfqpoint{1.822208in}{0.608132in}}%
\pgfpathlineto{\pgfqpoint{1.880005in}{0.657495in}}%
\pgfpathlineto{\pgfqpoint{1.929368in}{0.715292in}}%
\pgfpathlineto{\pgfqpoint{1.969081in}{0.780098in}}%
\pgfpathlineto{\pgfqpoint{1.998168in}{0.850320in}}%
\pgfpathlineto{\pgfqpoint{2.015912in}{0.924227in}}%
\pgfpathlineto{\pgfqpoint{2.021875in}{1.000000in}}%
\pgfusepath{stroke}%
\end{pgfscope}%
\begin{pgfscope}%
\pgfpathrectangle{\pgfqpoint{0.375000in}{0.200000in}}{\pgfqpoint{2.325000in}{1.600000in}} %
\pgfusepath{clip}%
\pgfsetbuttcap%
\pgfsetroundjoin%
\definecolor{currentfill}{rgb}{0.000000,0.000000,1.000000}%
\pgfsetfillcolor{currentfill}%
\pgfsetlinewidth{0.501875pt}%
\definecolor{currentstroke}{rgb}{0.000000,0.000000,0.000000}%
\pgfsetstrokecolor{currentstroke}%
\pgfsetdash{}{0pt}%
\pgfsys@defobject{currentmarker}{\pgfqpoint{-0.041667in}{-0.041667in}}{\pgfqpoint{0.041667in}{0.041667in}}{%
\pgfpathmoveto{\pgfqpoint{0.000000in}{-0.041667in}}%
\pgfpathcurveto{\pgfqpoint{0.011050in}{-0.041667in}}{\pgfqpoint{0.021649in}{-0.037276in}}{\pgfqpoint{0.029463in}{-0.029463in}}%
\pgfpathcurveto{\pgfqpoint{0.037276in}{-0.021649in}}{\pgfqpoint{0.041667in}{-0.011050in}}{\pgfqpoint{0.041667in}{0.000000in}}%
\pgfpathcurveto{\pgfqpoint{0.041667in}{0.011050in}}{\pgfqpoint{0.037276in}{0.021649in}}{\pgfqpoint{0.029463in}{0.029463in}}%
\pgfpathcurveto{\pgfqpoint{0.021649in}{0.037276in}}{\pgfqpoint{0.011050in}{0.041667in}}{\pgfqpoint{0.000000in}{0.041667in}}%
\pgfpathcurveto{\pgfqpoint{-0.011050in}{0.041667in}}{\pgfqpoint{-0.021649in}{0.037276in}}{\pgfqpoint{-0.029463in}{0.029463in}}%
\pgfpathcurveto{\pgfqpoint{-0.037276in}{0.021649in}}{\pgfqpoint{-0.041667in}{0.011050in}}{\pgfqpoint{-0.041667in}{0.000000in}}%
\pgfpathcurveto{\pgfqpoint{-0.041667in}{-0.011050in}}{\pgfqpoint{-0.037276in}{-0.021649in}}{\pgfqpoint{-0.029463in}{-0.029463in}}%
\pgfpathcurveto{\pgfqpoint{-0.021649in}{-0.037276in}}{\pgfqpoint{-0.011050in}{-0.041667in}}{\pgfqpoint{0.000000in}{-0.041667in}}%
\pgfpathclose%
\pgfusepath{stroke,fill}%
}%
\begin{pgfscope}%
\pgfsys@transformshift{1.537500in}{1.000000in}%
\pgfsys@useobject{currentmarker}{}%
\end{pgfscope}%
\end{pgfscope}%
\begin{pgfscope}%
\pgfpathrectangle{\pgfqpoint{0.375000in}{0.200000in}}{\pgfqpoint{2.325000in}{1.600000in}} %
\pgfusepath{clip}%
\pgfsetbuttcap%
\pgfsetroundjoin%
\definecolor{currentfill}{rgb}{0.000000,0.500000,0.000000}%
\pgfsetfillcolor{currentfill}%
\pgfsetlinewidth{0.501875pt}%
\definecolor{currentstroke}{rgb}{0.000000,0.000000,0.000000}%
\pgfsetstrokecolor{currentstroke}%
\pgfsetdash{}{0pt}%
\pgfsys@defobject{currentmarker}{\pgfqpoint{-0.041667in}{-0.041667in}}{\pgfqpoint{0.041667in}{0.041667in}}{%
\pgfpathmoveto{\pgfqpoint{0.000000in}{-0.041667in}}%
\pgfpathcurveto{\pgfqpoint{0.011050in}{-0.041667in}}{\pgfqpoint{0.021649in}{-0.037276in}}{\pgfqpoint{0.029463in}{-0.029463in}}%
\pgfpathcurveto{\pgfqpoint{0.037276in}{-0.021649in}}{\pgfqpoint{0.041667in}{-0.011050in}}{\pgfqpoint{0.041667in}{0.000000in}}%
\pgfpathcurveto{\pgfqpoint{0.041667in}{0.011050in}}{\pgfqpoint{0.037276in}{0.021649in}}{\pgfqpoint{0.029463in}{0.029463in}}%
\pgfpathcurveto{\pgfqpoint{0.021649in}{0.037276in}}{\pgfqpoint{0.011050in}{0.041667in}}{\pgfqpoint{0.000000in}{0.041667in}}%
\pgfpathcurveto{\pgfqpoint{-0.011050in}{0.041667in}}{\pgfqpoint{-0.021649in}{0.037276in}}{\pgfqpoint{-0.029463in}{0.029463in}}%
\pgfpathcurveto{\pgfqpoint{-0.037276in}{0.021649in}}{\pgfqpoint{-0.041667in}{0.011050in}}{\pgfqpoint{-0.041667in}{0.000000in}}%
\pgfpathcurveto{\pgfqpoint{-0.041667in}{-0.011050in}}{\pgfqpoint{-0.037276in}{-0.021649in}}{\pgfqpoint{-0.029463in}{-0.029463in}}%
\pgfpathcurveto{\pgfqpoint{-0.021649in}{-0.037276in}}{\pgfqpoint{-0.011050in}{-0.041667in}}{\pgfqpoint{0.000000in}{-0.041667in}}%
\pgfpathclose%
\pgfusepath{stroke,fill}%
}%
\begin{pgfscope}%
\pgfsys@transformshift{1.537500in}{0.515625in}%
\pgfsys@useobject{currentmarker}{}%
\end{pgfscope}%
\end{pgfscope}%
\begin{pgfscope}%
\pgftext[x=1.537500in,y=1.000000in,right,bottom]{{\sffamily\fontsize{14.000000}{16.800000}\selectfont \(\displaystyle CM\)}}%
\end{pgfscope}%
\begin{pgfscope}%
\pgftext[x=2.183333in,y=0.903125in,left,top]{{\sffamily\fontsize{14.000000}{16.800000}\selectfont \(\displaystyle F_{p}\)}}%
\end{pgfscope}%
\end{pgfpicture}%
\makeatother%
\endgroup%
\newline
Friction is still the only force that produces an angular acceleration, so
friction should have the same direction as angular acceleration. If the
object is moving to the right and rotating faster, friction acting on the
object goes towards left. If the object is moving to the left and rotation
slow down because of the external force slowing it down, the friction force
goes towards the left, to produce a torque and a angular acceleration to
slow down the rotation.\newline
%% Creator: Matplotlib, PGF backend
%%
%% To include the figure in your LaTeX document, write
%%   \input{<filename>.pgf}
%%
%% Make sure the required packages are loaded in your preamble
%%   \usepackage{pgf}
%%
%% Figures using additional raster images can only be included by \input if
%% they are in the same directory as the main LaTeX file. For loading figures
%% from other directories you can use the `import` package
%%   \usepackage{import}
%% and then include the figures with
%%   \import{<path to file>}{<filename>.pgf}
%%
%% Matplotlib used the following preamble
%%   \usepackage{fontspec}
%%   \setmainfont{Times New Roman}
%%   \setsansfont{Verdana}
%%   \setmonofont{Courier New}
%%
\begingroup%
\makeatletter%
\begin{pgfpicture}%
\pgfpathrectangle{\pgfpointorigin}{\pgfqpoint{4.000000in}{3.970000in}}%
\pgfusepath{use as bounding box}%
\begin{pgfscope}%
\pgfsetbuttcap%
\pgfsetroundjoin%
\definecolor{currentfill}{rgb}{1.000000,1.000000,1.000000}%
\pgfsetfillcolor{currentfill}%
\pgfsetlinewidth{0.000000pt}%
\definecolor{currentstroke}{rgb}{1.000000,1.000000,1.000000}%
\pgfsetstrokecolor{currentstroke}%
\pgfsetdash{}{0pt}%
\pgfpathmoveto{\pgfqpoint{0.000000in}{0.000000in}}%
\pgfpathlineto{\pgfqpoint{4.000000in}{0.000000in}}%
\pgfpathlineto{\pgfqpoint{4.000000in}{3.970000in}}%
\pgfpathlineto{\pgfqpoint{0.000000in}{3.970000in}}%
\pgfpathclose%
\pgfusepath{fill}%
\end{pgfscope}%
\begin{pgfscope}%
\pgfpathrectangle{\pgfqpoint{0.500000in}{0.397000in}}{\pgfqpoint{3.100000in}{3.176000in}} %
\pgfusepath{clip}%
\pgfsetbuttcap%
\pgfsetroundjoin%
\definecolor{currentfill}{rgb}{0.000000,0.000000,1.000000}%
\pgfsetfillcolor{currentfill}%
\pgfsetlinewidth{1.003750pt}%
\definecolor{currentstroke}{rgb}{0.000000,0.000000,0.000000}%
\pgfsetstrokecolor{currentstroke}%
\pgfsetdash{}{0pt}%
\pgfpathmoveto{\pgfqpoint{1.171667in}{1.339167in}}%
\pgfpathlineto{\pgfqpoint{1.404167in}{1.442500in}}%
\pgfpathlineto{\pgfqpoint{1.404167in}{1.362417in}}%
\pgfpathlineto{\pgfqpoint{2.050000in}{1.362417in}}%
\pgfpathlineto{\pgfqpoint{2.050000in}{1.315917in}}%
\pgfpathlineto{\pgfqpoint{1.404167in}{1.315917in}}%
\pgfpathlineto{\pgfqpoint{1.404167in}{1.235833in}}%
\pgfpathclose%
\pgfusepath{stroke,fill}%
\end{pgfscope}%
\begin{pgfscope}%
\pgfpathrectangle{\pgfqpoint{0.500000in}{0.397000in}}{\pgfqpoint{3.100000in}{3.176000in}} %
\pgfusepath{clip}%
\pgfsetbuttcap%
\pgfsetroundjoin%
\definecolor{currentfill}{rgb}{1.000000,0.000000,0.000000}%
\pgfsetfillcolor{currentfill}%
\pgfsetlinewidth{1.003750pt}%
\definecolor{currentstroke}{rgb}{1.000000,0.000000,0.000000}%
\pgfsetstrokecolor{currentstroke}%
\pgfsetdash{}{0pt}%
\pgfpathmoveto{\pgfqpoint{3.143611in}{1.985000in}}%
\pgfpathlineto{\pgfqpoint{2.911111in}{1.881667in}}%
\pgfpathlineto{\pgfqpoint{2.911111in}{1.961750in}}%
\pgfpathlineto{\pgfqpoint{2.050000in}{1.961750in}}%
\pgfpathlineto{\pgfqpoint{2.050000in}{2.008250in}}%
\pgfpathlineto{\pgfqpoint{2.911111in}{2.008250in}}%
\pgfpathlineto{\pgfqpoint{2.911111in}{2.088333in}}%
\pgfpathclose%
\pgfusepath{stroke,fill}%
\end{pgfscope}%
\begin{pgfscope}%
\pgfpathrectangle{\pgfqpoint{0.500000in}{0.397000in}}{\pgfqpoint{3.100000in}{3.176000in}} %
\pgfusepath{clip}%
\pgfsetbuttcap%
\pgfsetroundjoin%
\definecolor{currentfill}{rgb}{0.000000,0.000000,0.000000}%
\pgfsetfillcolor{currentfill}%
\pgfsetlinewidth{1.003750pt}%
\definecolor{currentstroke}{rgb}{0.000000,0.000000,0.000000}%
\pgfsetstrokecolor{currentstroke}%
\pgfsetdash{}{0pt}%
\pgfpathmoveto{\pgfqpoint{2.669032in}{2.655354in}}%
\pgfpathlineto{\pgfqpoint{2.568506in}{2.646888in}}%
\pgfpathlineto{\pgfqpoint{2.604801in}{2.698739in}}%
\pgfpathlineto{\pgfqpoint{2.567765in}{2.724664in}}%
\pgfpathlineto{\pgfqpoint{2.569246in}{2.726780in}}%
\pgfpathlineto{\pgfqpoint{2.606282in}{2.700855in}}%
\pgfpathlineto{\pgfqpoint{2.642578in}{2.752706in}}%
\pgfpathclose%
\pgfusepath{stroke,fill}%
\end{pgfscope}%
\begin{pgfscope}%
\pgfpathrectangle{\pgfqpoint{0.500000in}{0.397000in}}{\pgfqpoint{3.100000in}{3.176000in}} %
\pgfusepath{clip}%
\pgfsetbuttcap%
\pgfsetroundjoin%
\definecolor{currentfill}{rgb}{0.000000,0.000000,1.000000}%
\pgfsetfillcolor{currentfill}%
\pgfsetlinewidth{1.003750pt}%
\definecolor{currentstroke}{rgb}{0.000000,0.000000,0.000000}%
\pgfsetstrokecolor{currentstroke}%
\pgfsetdash{}{0pt}%
\pgfpathmoveto{\pgfqpoint{1.171667in}{1.339167in}}%
\pgfpathlineto{\pgfqpoint{1.404167in}{1.442500in}}%
\pgfpathlineto{\pgfqpoint{1.404167in}{1.362417in}}%
\pgfpathlineto{\pgfqpoint{2.050000in}{1.362417in}}%
\pgfpathlineto{\pgfqpoint{2.050000in}{1.315917in}}%
\pgfpathlineto{\pgfqpoint{1.404167in}{1.315917in}}%
\pgfpathlineto{\pgfqpoint{1.404167in}{1.235833in}}%
\pgfpathclose%
\pgfusepath{stroke,fill}%
\end{pgfscope}%
\begin{pgfscope}%
\pgfpathrectangle{\pgfqpoint{0.500000in}{0.397000in}}{\pgfqpoint{3.100000in}{3.176000in}} %
\pgfusepath{clip}%
\pgfsetbuttcap%
\pgfsetroundjoin%
\definecolor{currentfill}{rgb}{1.000000,0.000000,0.000000}%
\pgfsetfillcolor{currentfill}%
\pgfsetlinewidth{1.003750pt}%
\definecolor{currentstroke}{rgb}{1.000000,0.000000,0.000000}%
\pgfsetstrokecolor{currentstroke}%
\pgfsetdash{}{0pt}%
\pgfpathmoveto{\pgfqpoint{3.143611in}{1.985000in}}%
\pgfpathlineto{\pgfqpoint{2.911111in}{1.881667in}}%
\pgfpathlineto{\pgfqpoint{2.911111in}{1.961750in}}%
\pgfpathlineto{\pgfqpoint{2.050000in}{1.961750in}}%
\pgfpathlineto{\pgfqpoint{2.050000in}{2.008250in}}%
\pgfpathlineto{\pgfqpoint{2.911111in}{2.008250in}}%
\pgfpathlineto{\pgfqpoint{2.911111in}{2.088333in}}%
\pgfpathclose%
\pgfusepath{stroke,fill}%
\end{pgfscope}%
\begin{pgfscope}%
\pgfpathrectangle{\pgfqpoint{0.500000in}{0.397000in}}{\pgfqpoint{3.100000in}{3.176000in}} %
\pgfusepath{clip}%
\pgfsetbuttcap%
\pgfsetroundjoin%
\definecolor{currentfill}{rgb}{0.000000,0.000000,0.000000}%
\pgfsetfillcolor{currentfill}%
\pgfsetlinewidth{1.003750pt}%
\definecolor{currentstroke}{rgb}{0.000000,0.000000,0.000000}%
\pgfsetstrokecolor{currentstroke}%
\pgfsetdash{}{0pt}%
\pgfpathmoveto{\pgfqpoint{2.669032in}{2.655354in}}%
\pgfpathlineto{\pgfqpoint{2.568506in}{2.646888in}}%
\pgfpathlineto{\pgfqpoint{2.604801in}{2.698739in}}%
\pgfpathlineto{\pgfqpoint{2.567765in}{2.724664in}}%
\pgfpathlineto{\pgfqpoint{2.569246in}{2.726780in}}%
\pgfpathlineto{\pgfqpoint{2.606282in}{2.700855in}}%
\pgfpathlineto{\pgfqpoint{2.642578in}{2.752706in}}%
\pgfpathclose%
\pgfusepath{stroke,fill}%
\end{pgfscope}%
\begin{pgfscope}%
\pgfpathrectangle{\pgfqpoint{0.500000in}{0.397000in}}{\pgfqpoint{3.100000in}{3.176000in}} %
\pgfusepath{clip}%
\pgfsetrectcap%
\pgfsetroundjoin%
\pgfsetlinewidth{1.003750pt}%
\definecolor{currentstroke}{rgb}{0.000000,0.000000,1.000000}%
\pgfsetstrokecolor{currentstroke}%
\pgfsetdash{}{0pt}%
\pgfpathmoveto{\pgfqpoint{0.758333in}{1.339167in}}%
\pgfpathlineto{\pgfqpoint{3.341667in}{1.339167in}}%
\pgfusepath{stroke}%
\end{pgfscope}%
\begin{pgfscope}%
\pgfpathrectangle{\pgfqpoint{0.500000in}{0.397000in}}{\pgfqpoint{3.100000in}{3.176000in}} %
\pgfusepath{clip}%
\pgfsetrectcap%
\pgfsetroundjoin%
\pgfsetlinewidth{1.003750pt}%
\definecolor{currentstroke}{rgb}{0.000000,0.000000,0.000000}%
\pgfsetstrokecolor{currentstroke}%
\pgfsetdash{}{0pt}%
\pgfpathmoveto{\pgfqpoint{2.050000in}{2.889167in}}%
\pgfpathlineto{\pgfqpoint{2.105185in}{2.887481in}}%
\pgfpathlineto{\pgfqpoint{2.160165in}{2.882430in}}%
\pgfpathlineto{\pgfqpoint{2.214734in}{2.874033in}}%
\pgfpathlineto{\pgfqpoint{2.268689in}{2.862321in}}%
\pgfpathlineto{\pgfqpoint{2.321828in}{2.847338in}}%
\pgfpathlineto{\pgfqpoint{2.373954in}{2.829139in}}%
\pgfpathlineto{\pgfqpoint{2.424872in}{2.807793in}}%
\pgfpathlineto{\pgfqpoint{2.474392in}{2.783379in}}%
\pgfpathlineto{\pgfqpoint{2.522329in}{2.755988in}}%
\pgfpathlineto{\pgfqpoint{2.568506in}{2.725722in}}%
\pgfusepath{stroke}%
\end{pgfscope}%
\begin{pgfscope}%
\pgfpathrectangle{\pgfqpoint{0.500000in}{0.397000in}}{\pgfqpoint{3.100000in}{3.176000in}} %
\pgfusepath{clip}%
\pgfsetrectcap%
\pgfsetroundjoin%
\pgfsetlinewidth{1.003750pt}%
\definecolor{currentstroke}{rgb}{0.000000,0.000000,0.000000}%
\pgfsetstrokecolor{currentstroke}%
\pgfsetdash{}{0pt}%
\pgfpathmoveto{\pgfqpoint{2.695833in}{1.985000in}}%
\pgfpathlineto{\pgfqpoint{2.687882in}{2.086031in}}%
\pgfpathlineto{\pgfqpoint{2.664224in}{2.184573in}}%
\pgfpathlineto{\pgfqpoint{2.625442in}{2.278202in}}%
\pgfpathlineto{\pgfqpoint{2.572490in}{2.364611in}}%
\pgfpathlineto{\pgfqpoint{2.506673in}{2.441673in}}%
\pgfpathlineto{\pgfqpoint{2.429611in}{2.507490in}}%
\pgfpathlineto{\pgfqpoint{2.343202in}{2.560442in}}%
\pgfpathlineto{\pgfqpoint{2.249573in}{2.599224in}}%
\pgfpathlineto{\pgfqpoint{2.151031in}{2.622882in}}%
\pgfpathlineto{\pgfqpoint{2.050000in}{2.630833in}}%
\pgfpathlineto{\pgfqpoint{1.948969in}{2.622882in}}%
\pgfpathlineto{\pgfqpoint{1.850427in}{2.599224in}}%
\pgfpathlineto{\pgfqpoint{1.756798in}{2.560442in}}%
\pgfpathlineto{\pgfqpoint{1.670389in}{2.507490in}}%
\pgfpathlineto{\pgfqpoint{1.593327in}{2.441673in}}%
\pgfpathlineto{\pgfqpoint{1.527510in}{2.364611in}}%
\pgfpathlineto{\pgfqpoint{1.474558in}{2.278202in}}%
\pgfpathlineto{\pgfqpoint{1.435776in}{2.184573in}}%
\pgfpathlineto{\pgfqpoint{1.412118in}{2.086031in}}%
\pgfpathlineto{\pgfqpoint{1.404167in}{1.985000in}}%
\pgfusepath{stroke}%
\end{pgfscope}%
\begin{pgfscope}%
\pgfpathrectangle{\pgfqpoint{0.500000in}{0.397000in}}{\pgfqpoint{3.100000in}{3.176000in}} %
\pgfusepath{clip}%
\pgfsetrectcap%
\pgfsetroundjoin%
\pgfsetlinewidth{1.003750pt}%
\definecolor{currentstroke}{rgb}{0.000000,0.000000,0.000000}%
\pgfsetstrokecolor{currentstroke}%
\pgfsetdash{}{0pt}%
\pgfpathmoveto{\pgfqpoint{1.404167in}{1.985000in}}%
\pgfpathlineto{\pgfqpoint{1.412118in}{1.883969in}}%
\pgfpathlineto{\pgfqpoint{1.435776in}{1.785427in}}%
\pgfpathlineto{\pgfqpoint{1.474558in}{1.691798in}}%
\pgfpathlineto{\pgfqpoint{1.527510in}{1.605389in}}%
\pgfpathlineto{\pgfqpoint{1.593327in}{1.528327in}}%
\pgfpathlineto{\pgfqpoint{1.670389in}{1.462510in}}%
\pgfpathlineto{\pgfqpoint{1.756798in}{1.409558in}}%
\pgfpathlineto{\pgfqpoint{1.850427in}{1.370776in}}%
\pgfpathlineto{\pgfqpoint{1.948969in}{1.347118in}}%
\pgfpathlineto{\pgfqpoint{2.050000in}{1.339167in}}%
\pgfpathlineto{\pgfqpoint{2.151031in}{1.347118in}}%
\pgfpathlineto{\pgfqpoint{2.249573in}{1.370776in}}%
\pgfpathlineto{\pgfqpoint{2.343202in}{1.409558in}}%
\pgfpathlineto{\pgfqpoint{2.429611in}{1.462510in}}%
\pgfpathlineto{\pgfqpoint{2.506673in}{1.528327in}}%
\pgfpathlineto{\pgfqpoint{2.572490in}{1.605389in}}%
\pgfpathlineto{\pgfqpoint{2.625442in}{1.691798in}}%
\pgfpathlineto{\pgfqpoint{2.664224in}{1.785427in}}%
\pgfpathlineto{\pgfqpoint{2.687882in}{1.883969in}}%
\pgfpathlineto{\pgfqpoint{2.695833in}{1.985000in}}%
\pgfusepath{stroke}%
\end{pgfscope}%
\begin{pgfscope}%
\pgfpathrectangle{\pgfqpoint{0.500000in}{0.397000in}}{\pgfqpoint{3.100000in}{3.176000in}} %
\pgfusepath{clip}%
\pgfsetbuttcap%
\pgfsetroundjoin%
\definecolor{currentfill}{rgb}{0.000000,0.000000,1.000000}%
\pgfsetfillcolor{currentfill}%
\pgfsetlinewidth{0.501875pt}%
\definecolor{currentstroke}{rgb}{0.000000,0.000000,0.000000}%
\pgfsetstrokecolor{currentstroke}%
\pgfsetdash{}{0pt}%
\pgfsys@defobject{currentmarker}{\pgfqpoint{-0.041667in}{-0.041667in}}{\pgfqpoint{0.041667in}{0.041667in}}{%
\pgfpathmoveto{\pgfqpoint{0.000000in}{-0.041667in}}%
\pgfpathcurveto{\pgfqpoint{0.011050in}{-0.041667in}}{\pgfqpoint{0.021649in}{-0.037276in}}{\pgfqpoint{0.029463in}{-0.029463in}}%
\pgfpathcurveto{\pgfqpoint{0.037276in}{-0.021649in}}{\pgfqpoint{0.041667in}{-0.011050in}}{\pgfqpoint{0.041667in}{0.000000in}}%
\pgfpathcurveto{\pgfqpoint{0.041667in}{0.011050in}}{\pgfqpoint{0.037276in}{0.021649in}}{\pgfqpoint{0.029463in}{0.029463in}}%
\pgfpathcurveto{\pgfqpoint{0.021649in}{0.037276in}}{\pgfqpoint{0.011050in}{0.041667in}}{\pgfqpoint{0.000000in}{0.041667in}}%
\pgfpathcurveto{\pgfqpoint{-0.011050in}{0.041667in}}{\pgfqpoint{-0.021649in}{0.037276in}}{\pgfqpoint{-0.029463in}{0.029463in}}%
\pgfpathcurveto{\pgfqpoint{-0.037276in}{0.021649in}}{\pgfqpoint{-0.041667in}{0.011050in}}{\pgfqpoint{-0.041667in}{0.000000in}}%
\pgfpathcurveto{\pgfqpoint{-0.041667in}{-0.011050in}}{\pgfqpoint{-0.037276in}{-0.021649in}}{\pgfqpoint{-0.029463in}{-0.029463in}}%
\pgfpathcurveto{\pgfqpoint{-0.021649in}{-0.037276in}}{\pgfqpoint{-0.011050in}{-0.041667in}}{\pgfqpoint{0.000000in}{-0.041667in}}%
\pgfpathclose%
\pgfusepath{stroke,fill}%
}%
\begin{pgfscope}%
\pgfsys@transformshift{2.050000in}{1.985000in}%
\pgfsys@useobject{currentmarker}{}%
\end{pgfscope}%
\end{pgfscope}%
\begin{pgfscope}%
\pgfpathrectangle{\pgfqpoint{0.500000in}{0.397000in}}{\pgfqpoint{3.100000in}{3.176000in}} %
\pgfusepath{clip}%
\pgfsetbuttcap%
\pgfsetroundjoin%
\definecolor{currentfill}{rgb}{0.000000,0.500000,0.000000}%
\pgfsetfillcolor{currentfill}%
\pgfsetlinewidth{0.501875pt}%
\definecolor{currentstroke}{rgb}{0.000000,0.000000,0.000000}%
\pgfsetstrokecolor{currentstroke}%
\pgfsetdash{}{0pt}%
\pgfsys@defobject{currentmarker}{\pgfqpoint{-0.041667in}{-0.041667in}}{\pgfqpoint{0.041667in}{0.041667in}}{%
\pgfpathmoveto{\pgfqpoint{0.000000in}{-0.041667in}}%
\pgfpathcurveto{\pgfqpoint{0.011050in}{-0.041667in}}{\pgfqpoint{0.021649in}{-0.037276in}}{\pgfqpoint{0.029463in}{-0.029463in}}%
\pgfpathcurveto{\pgfqpoint{0.037276in}{-0.021649in}}{\pgfqpoint{0.041667in}{-0.011050in}}{\pgfqpoint{0.041667in}{0.000000in}}%
\pgfpathcurveto{\pgfqpoint{0.041667in}{0.011050in}}{\pgfqpoint{0.037276in}{0.021649in}}{\pgfqpoint{0.029463in}{0.029463in}}%
\pgfpathcurveto{\pgfqpoint{0.021649in}{0.037276in}}{\pgfqpoint{0.011050in}{0.041667in}}{\pgfqpoint{0.000000in}{0.041667in}}%
\pgfpathcurveto{\pgfqpoint{-0.011050in}{0.041667in}}{\pgfqpoint{-0.021649in}{0.037276in}}{\pgfqpoint{-0.029463in}{0.029463in}}%
\pgfpathcurveto{\pgfqpoint{-0.037276in}{0.021649in}}{\pgfqpoint{-0.041667in}{0.011050in}}{\pgfqpoint{-0.041667in}{0.000000in}}%
\pgfpathcurveto{\pgfqpoint{-0.041667in}{-0.011050in}}{\pgfqpoint{-0.037276in}{-0.021649in}}{\pgfqpoint{-0.029463in}{-0.029463in}}%
\pgfpathcurveto{\pgfqpoint{-0.021649in}{-0.037276in}}{\pgfqpoint{-0.011050in}{-0.041667in}}{\pgfqpoint{0.000000in}{-0.041667in}}%
\pgfpathclose%
\pgfusepath{stroke,fill}%
}%
\begin{pgfscope}%
\pgfsys@transformshift{2.050000in}{1.339167in}%
\pgfsys@useobject{currentmarker}{}%
\end{pgfscope}%
\end{pgfscope}%
\begin{pgfscope}%
\pgfpathrectangle{\pgfqpoint{0.500000in}{0.397000in}}{\pgfqpoint{3.100000in}{3.176000in}} %
\pgfusepath{clip}%
\pgfsetrectcap%
\pgfsetroundjoin%
\pgfsetlinewidth{1.003750pt}%
\definecolor{currentstroke}{rgb}{0.000000,0.000000,1.000000}%
\pgfsetstrokecolor{currentstroke}%
\pgfsetdash{}{0pt}%
\pgfpathmoveto{\pgfqpoint{0.758333in}{1.339167in}}%
\pgfpathlineto{\pgfqpoint{3.341667in}{1.339167in}}%
\pgfusepath{stroke}%
\end{pgfscope}%
\begin{pgfscope}%
\pgfpathrectangle{\pgfqpoint{0.500000in}{0.397000in}}{\pgfqpoint{3.100000in}{3.176000in}} %
\pgfusepath{clip}%
\pgfsetrectcap%
\pgfsetroundjoin%
\pgfsetlinewidth{1.003750pt}%
\definecolor{currentstroke}{rgb}{0.000000,0.000000,0.000000}%
\pgfsetstrokecolor{currentstroke}%
\pgfsetdash{}{0pt}%
\pgfpathmoveto{\pgfqpoint{2.050000in}{2.889167in}}%
\pgfpathlineto{\pgfqpoint{2.105185in}{2.887481in}}%
\pgfpathlineto{\pgfqpoint{2.160165in}{2.882430in}}%
\pgfpathlineto{\pgfqpoint{2.214734in}{2.874033in}}%
\pgfpathlineto{\pgfqpoint{2.268689in}{2.862321in}}%
\pgfpathlineto{\pgfqpoint{2.321828in}{2.847338in}}%
\pgfpathlineto{\pgfqpoint{2.373954in}{2.829139in}}%
\pgfpathlineto{\pgfqpoint{2.424872in}{2.807793in}}%
\pgfpathlineto{\pgfqpoint{2.474392in}{2.783379in}}%
\pgfpathlineto{\pgfqpoint{2.522329in}{2.755988in}}%
\pgfpathlineto{\pgfqpoint{2.568506in}{2.725722in}}%
\pgfusepath{stroke}%
\end{pgfscope}%
\begin{pgfscope}%
\pgfpathrectangle{\pgfqpoint{0.500000in}{0.397000in}}{\pgfqpoint{3.100000in}{3.176000in}} %
\pgfusepath{clip}%
\pgfsetrectcap%
\pgfsetroundjoin%
\pgfsetlinewidth{1.003750pt}%
\definecolor{currentstroke}{rgb}{0.000000,0.000000,0.000000}%
\pgfsetstrokecolor{currentstroke}%
\pgfsetdash{}{0pt}%
\pgfpathmoveto{\pgfqpoint{2.695833in}{1.985000in}}%
\pgfpathlineto{\pgfqpoint{2.687882in}{2.086031in}}%
\pgfpathlineto{\pgfqpoint{2.664224in}{2.184573in}}%
\pgfpathlineto{\pgfqpoint{2.625442in}{2.278202in}}%
\pgfpathlineto{\pgfqpoint{2.572490in}{2.364611in}}%
\pgfpathlineto{\pgfqpoint{2.506673in}{2.441673in}}%
\pgfpathlineto{\pgfqpoint{2.429611in}{2.507490in}}%
\pgfpathlineto{\pgfqpoint{2.343202in}{2.560442in}}%
\pgfpathlineto{\pgfqpoint{2.249573in}{2.599224in}}%
\pgfpathlineto{\pgfqpoint{2.151031in}{2.622882in}}%
\pgfpathlineto{\pgfqpoint{2.050000in}{2.630833in}}%
\pgfpathlineto{\pgfqpoint{1.948969in}{2.622882in}}%
\pgfpathlineto{\pgfqpoint{1.850427in}{2.599224in}}%
\pgfpathlineto{\pgfqpoint{1.756798in}{2.560442in}}%
\pgfpathlineto{\pgfqpoint{1.670389in}{2.507490in}}%
\pgfpathlineto{\pgfqpoint{1.593327in}{2.441673in}}%
\pgfpathlineto{\pgfqpoint{1.527510in}{2.364611in}}%
\pgfpathlineto{\pgfqpoint{1.474558in}{2.278202in}}%
\pgfpathlineto{\pgfqpoint{1.435776in}{2.184573in}}%
\pgfpathlineto{\pgfqpoint{1.412118in}{2.086031in}}%
\pgfpathlineto{\pgfqpoint{1.404167in}{1.985000in}}%
\pgfusepath{stroke}%
\end{pgfscope}%
\begin{pgfscope}%
\pgfpathrectangle{\pgfqpoint{0.500000in}{0.397000in}}{\pgfqpoint{3.100000in}{3.176000in}} %
\pgfusepath{clip}%
\pgfsetrectcap%
\pgfsetroundjoin%
\pgfsetlinewidth{1.003750pt}%
\definecolor{currentstroke}{rgb}{0.000000,0.000000,0.000000}%
\pgfsetstrokecolor{currentstroke}%
\pgfsetdash{}{0pt}%
\pgfpathmoveto{\pgfqpoint{1.404167in}{1.985000in}}%
\pgfpathlineto{\pgfqpoint{1.412118in}{1.883969in}}%
\pgfpathlineto{\pgfqpoint{1.435776in}{1.785427in}}%
\pgfpathlineto{\pgfqpoint{1.474558in}{1.691798in}}%
\pgfpathlineto{\pgfqpoint{1.527510in}{1.605389in}}%
\pgfpathlineto{\pgfqpoint{1.593327in}{1.528327in}}%
\pgfpathlineto{\pgfqpoint{1.670389in}{1.462510in}}%
\pgfpathlineto{\pgfqpoint{1.756798in}{1.409558in}}%
\pgfpathlineto{\pgfqpoint{1.850427in}{1.370776in}}%
\pgfpathlineto{\pgfqpoint{1.948969in}{1.347118in}}%
\pgfpathlineto{\pgfqpoint{2.050000in}{1.339167in}}%
\pgfpathlineto{\pgfqpoint{2.151031in}{1.347118in}}%
\pgfpathlineto{\pgfqpoint{2.249573in}{1.370776in}}%
\pgfpathlineto{\pgfqpoint{2.343202in}{1.409558in}}%
\pgfpathlineto{\pgfqpoint{2.429611in}{1.462510in}}%
\pgfpathlineto{\pgfqpoint{2.506673in}{1.528327in}}%
\pgfpathlineto{\pgfqpoint{2.572490in}{1.605389in}}%
\pgfpathlineto{\pgfqpoint{2.625442in}{1.691798in}}%
\pgfpathlineto{\pgfqpoint{2.664224in}{1.785427in}}%
\pgfpathlineto{\pgfqpoint{2.687882in}{1.883969in}}%
\pgfpathlineto{\pgfqpoint{2.695833in}{1.985000in}}%
\pgfusepath{stroke}%
\end{pgfscope}%
\begin{pgfscope}%
\pgfpathrectangle{\pgfqpoint{0.500000in}{0.397000in}}{\pgfqpoint{3.100000in}{3.176000in}} %
\pgfusepath{clip}%
\pgfsetbuttcap%
\pgfsetroundjoin%
\definecolor{currentfill}{rgb}{1.000000,0.000000,0.000000}%
\pgfsetfillcolor{currentfill}%
\pgfsetlinewidth{0.501875pt}%
\definecolor{currentstroke}{rgb}{0.000000,0.000000,0.000000}%
\pgfsetstrokecolor{currentstroke}%
\pgfsetdash{}{0pt}%
\pgfsys@defobject{currentmarker}{\pgfqpoint{-0.041667in}{-0.041667in}}{\pgfqpoint{0.041667in}{0.041667in}}{%
\pgfpathmoveto{\pgfqpoint{0.000000in}{-0.041667in}}%
\pgfpathcurveto{\pgfqpoint{0.011050in}{-0.041667in}}{\pgfqpoint{0.021649in}{-0.037276in}}{\pgfqpoint{0.029463in}{-0.029463in}}%
\pgfpathcurveto{\pgfqpoint{0.037276in}{-0.021649in}}{\pgfqpoint{0.041667in}{-0.011050in}}{\pgfqpoint{0.041667in}{0.000000in}}%
\pgfpathcurveto{\pgfqpoint{0.041667in}{0.011050in}}{\pgfqpoint{0.037276in}{0.021649in}}{\pgfqpoint{0.029463in}{0.029463in}}%
\pgfpathcurveto{\pgfqpoint{0.021649in}{0.037276in}}{\pgfqpoint{0.011050in}{0.041667in}}{\pgfqpoint{0.000000in}{0.041667in}}%
\pgfpathcurveto{\pgfqpoint{-0.011050in}{0.041667in}}{\pgfqpoint{-0.021649in}{0.037276in}}{\pgfqpoint{-0.029463in}{0.029463in}}%
\pgfpathcurveto{\pgfqpoint{-0.037276in}{0.021649in}}{\pgfqpoint{-0.041667in}{0.011050in}}{\pgfqpoint{-0.041667in}{0.000000in}}%
\pgfpathcurveto{\pgfqpoint{-0.041667in}{-0.011050in}}{\pgfqpoint{-0.037276in}{-0.021649in}}{\pgfqpoint{-0.029463in}{-0.029463in}}%
\pgfpathcurveto{\pgfqpoint{-0.021649in}{-0.037276in}}{\pgfqpoint{-0.011050in}{-0.041667in}}{\pgfqpoint{0.000000in}{-0.041667in}}%
\pgfpathclose%
\pgfusepath{stroke,fill}%
}%
\begin{pgfscope}%
\pgfsys@transformshift{2.050000in}{1.985000in}%
\pgfsys@useobject{currentmarker}{}%
\end{pgfscope}%
\end{pgfscope}%
\begin{pgfscope}%
\pgfpathrectangle{\pgfqpoint{0.500000in}{0.397000in}}{\pgfqpoint{3.100000in}{3.176000in}} %
\pgfusepath{clip}%
\pgfsetbuttcap%
\pgfsetroundjoin%
\definecolor{currentfill}{rgb}{0.000000,0.750000,0.750000}%
\pgfsetfillcolor{currentfill}%
\pgfsetlinewidth{0.501875pt}%
\definecolor{currentstroke}{rgb}{0.000000,0.000000,0.000000}%
\pgfsetstrokecolor{currentstroke}%
\pgfsetdash{}{0pt}%
\pgfsys@defobject{currentmarker}{\pgfqpoint{-0.041667in}{-0.041667in}}{\pgfqpoint{0.041667in}{0.041667in}}{%
\pgfpathmoveto{\pgfqpoint{0.000000in}{-0.041667in}}%
\pgfpathcurveto{\pgfqpoint{0.011050in}{-0.041667in}}{\pgfqpoint{0.021649in}{-0.037276in}}{\pgfqpoint{0.029463in}{-0.029463in}}%
\pgfpathcurveto{\pgfqpoint{0.037276in}{-0.021649in}}{\pgfqpoint{0.041667in}{-0.011050in}}{\pgfqpoint{0.041667in}{0.000000in}}%
\pgfpathcurveto{\pgfqpoint{0.041667in}{0.011050in}}{\pgfqpoint{0.037276in}{0.021649in}}{\pgfqpoint{0.029463in}{0.029463in}}%
\pgfpathcurveto{\pgfqpoint{0.021649in}{0.037276in}}{\pgfqpoint{0.011050in}{0.041667in}}{\pgfqpoint{0.000000in}{0.041667in}}%
\pgfpathcurveto{\pgfqpoint{-0.011050in}{0.041667in}}{\pgfqpoint{-0.021649in}{0.037276in}}{\pgfqpoint{-0.029463in}{0.029463in}}%
\pgfpathcurveto{\pgfqpoint{-0.037276in}{0.021649in}}{\pgfqpoint{-0.041667in}{0.011050in}}{\pgfqpoint{-0.041667in}{0.000000in}}%
\pgfpathcurveto{\pgfqpoint{-0.041667in}{-0.011050in}}{\pgfqpoint{-0.037276in}{-0.021649in}}{\pgfqpoint{-0.029463in}{-0.029463in}}%
\pgfpathcurveto{\pgfqpoint{-0.021649in}{-0.037276in}}{\pgfqpoint{-0.011050in}{-0.041667in}}{\pgfqpoint{0.000000in}{-0.041667in}}%
\pgfpathclose%
\pgfusepath{stroke,fill}%
}%
\begin{pgfscope}%
\pgfsys@transformshift{2.050000in}{1.339167in}%
\pgfsys@useobject{currentmarker}{}%
\end{pgfscope}%
\end{pgfscope}%
\begin{pgfscope}%
\pgftext[x=2.308333in,y=2.889167in,left,bottom]{{\sffamily\fontsize{20.000000}{24.000000}\selectfont \(\displaystyle \alpha\)}}%
\end{pgfscope}%
\begin{pgfscope}%
\pgftext[x=2.050000in,y=1.985000in,right,bottom]{{\sffamily\fontsize{20.000000}{24.000000}\selectfont \(\displaystyle CM\)}}%
\end{pgfscope}%
\begin{pgfscope}%
\pgftext[x=1.404167in,y=1.210000in,left,top]{{\sffamily\fontsize{20.000000}{24.000000}\selectfont \(\displaystyle F_{fr}\)}}%
\end{pgfscope}%
\begin{pgfscope}%
\pgftext[x=2.911111in,y=1.855833in,left,top]{{\sffamily\fontsize{20.000000}{24.000000}\selectfont \(\displaystyle F_{p}\)}}%
\end{pgfscope}%
\begin{pgfscope}%
\pgftext[x=2.308333in,y=2.889167in,left,bottom]{{\sffamily\fontsize{20.000000}{24.000000}\selectfont \(\displaystyle \alpha\)}}%
\end{pgfscope}%
\begin{pgfscope}%
\pgftext[x=2.050000in,y=1.985000in,right,bottom]{{\sffamily\fontsize{20.000000}{24.000000}\selectfont \(\displaystyle CM\)}}%
\end{pgfscope}%
\begin{pgfscope}%
\pgftext[x=1.404167in,y=1.210000in,left,top]{{\sffamily\fontsize{20.000000}{24.000000}\selectfont \(\displaystyle F_{fr}\)}}%
\end{pgfscope}%
\begin{pgfscope}%
\pgftext[x=2.911111in,y=1.855833in,left,top]{{\sffamily\fontsize{20.000000}{24.000000}\selectfont \(\displaystyle F_{p}\)}}%
\end{pgfscope}%
\end{pgfpicture}%
\makeatother%
\endgroup%
%
\[
\left\{ 
\begin{array}{c}
F_{p}-F_{fr}=ma \\ 
F_{fr}\cdot R=I\alpha =I\frac{a}{R}%
\end{array}%
\right. 
\]
\end{case}

\newpage

\begin{case}
An external torque (not force), like on car tire.\newline
%% Creator: Matplotlib, PGF backend
%%
%% To include the figure in your LaTeX document, write
%%   \input{<filename>.pgf}
%%
%% Make sure the required packages are loaded in your preamble
%%   \usepackage{pgf}
%%
%% Figures using additional raster images can only be included by \input if
%% they are in the same directory as the main LaTeX file. For loading figures
%% from other directories you can use the `import` package
%%   \usepackage{import}
%% and then include the figures with
%%   \import{<path to file>}{<filename>.pgf}
%%
%% Matplotlib used the following preamble
%%   \usepackage{fontspec}
%%   \setmainfont{Times New Roman}
%%   \setsansfont{Verdana}
%%   \setmonofont{Courier New}
%%
\begingroup%
\makeatletter%
\begin{pgfpicture}%
\pgfpathrectangle{\pgfpointorigin}{\pgfqpoint{3.000000in}{1.500000in}}%
\pgfusepath{use as bounding box}%
\begin{pgfscope}%
\pgfsetbuttcap%
\pgfsetroundjoin%
\definecolor{currentfill}{rgb}{1.000000,1.000000,1.000000}%
\pgfsetfillcolor{currentfill}%
\pgfsetlinewidth{0.000000pt}%
\definecolor{currentstroke}{rgb}{1.000000,1.000000,1.000000}%
\pgfsetstrokecolor{currentstroke}%
\pgfsetdash{}{0pt}%
\pgfpathmoveto{\pgfqpoint{0.000000in}{0.000000in}}%
\pgfpathlineto{\pgfqpoint{3.000000in}{0.000000in}}%
\pgfpathlineto{\pgfqpoint{3.000000in}{1.500000in}}%
\pgfpathlineto{\pgfqpoint{0.000000in}{1.500000in}}%
\pgfpathclose%
\pgfusepath{fill}%
\end{pgfscope}%
\begin{pgfscope}%
\pgfpathrectangle{\pgfqpoint{0.375000in}{0.150000in}}{\pgfqpoint{2.325000in}{1.200000in}} %
\pgfusepath{clip}%
\pgfsetbuttcap%
\pgfsetroundjoin%
\definecolor{currentfill}{rgb}{0.000000,0.000000,0.000000}%
\pgfsetfillcolor{currentfill}%
\pgfsetlinewidth{1.003750pt}%
\definecolor{currentstroke}{rgb}{0.000000,0.000000,0.000000}%
\pgfsetstrokecolor{currentstroke}%
\pgfsetdash{}{0pt}%
\pgfpathmoveto{\pgfqpoint{1.733924in}{0.908131in}}%
\pgfpathlineto{\pgfqpoint{1.658529in}{0.901782in}}%
\pgfpathlineto{\pgfqpoint{1.685750in}{0.940670in}}%
\pgfpathlineto{\pgfqpoint{1.675830in}{0.947614in}}%
\pgfpathlineto{\pgfqpoint{1.676941in}{0.949201in}}%
\pgfpathlineto{\pgfqpoint{1.686861in}{0.942257in}}%
\pgfpathlineto{\pgfqpoint{1.714083in}{0.981145in}}%
\pgfpathclose%
\pgfusepath{stroke,fill}%
\end{pgfscope}%
\begin{pgfscope}%
\pgfpathrectangle{\pgfqpoint{0.375000in}{0.150000in}}{\pgfqpoint{2.325000in}{1.200000in}} %
\pgfusepath{clip}%
\pgfsetrectcap%
\pgfsetroundjoin%
\pgfsetlinewidth{1.003750pt}%
\definecolor{currentstroke}{rgb}{0.000000,0.000000,1.000000}%
\pgfsetstrokecolor{currentstroke}%
\pgfsetdash{}{0pt}%
\pgfpathmoveto{\pgfqpoint{0.568750in}{0.265625in}}%
\pgfpathlineto{\pgfqpoint{2.506250in}{0.265625in}}%
\pgfusepath{stroke}%
\end{pgfscope}%
\begin{pgfscope}%
\pgfpathrectangle{\pgfqpoint{0.375000in}{0.150000in}}{\pgfqpoint{2.325000in}{1.200000in}} %
\pgfusepath{clip}%
\pgfsetrectcap%
\pgfsetroundjoin%
\pgfsetlinewidth{1.003750pt}%
\definecolor{currentstroke}{rgb}{0.000000,0.000000,0.000000}%
\pgfsetstrokecolor{currentstroke}%
\pgfsetdash{}{0pt}%
\pgfpathmoveto{\pgfqpoint{1.320881in}{0.858310in}}%
\pgfpathlineto{\pgfqpoint{1.334677in}{0.882354in}}%
\pgfpathlineto{\pgfqpoint{1.351130in}{0.904665in}}%
\pgfpathlineto{\pgfqpoint{1.370026in}{0.924949in}}%
\pgfpathlineto{\pgfqpoint{1.391115in}{0.942941in}}%
\pgfpathlineto{\pgfqpoint{1.414122in}{0.958405in}}%
\pgfpathlineto{\pgfqpoint{1.438746in}{0.971139in}}%
\pgfpathlineto{\pgfqpoint{1.464663in}{0.980975in}}%
\pgfpathlineto{\pgfqpoint{1.491535in}{0.987786in}}%
\pgfpathlineto{\pgfqpoint{1.519009in}{0.991481in}}%
\pgfpathlineto{\pgfqpoint{1.546725in}{0.992012in}}%
\pgfpathlineto{\pgfqpoint{1.574321in}{0.989372in}}%
\pgfpathlineto{\pgfqpoint{1.601434in}{0.983596in}}%
\pgfpathlineto{\pgfqpoint{1.627709in}{0.974760in}}%
\pgfpathlineto{\pgfqpoint{1.652803in}{0.962979in}}%
\pgfpathlineto{\pgfqpoint{1.676385in}{0.948408in}}%
\pgfusepath{stroke}%
\end{pgfscope}%
\begin{pgfscope}%
\pgfpathrectangle{\pgfqpoint{0.375000in}{0.150000in}}{\pgfqpoint{2.325000in}{1.200000in}} %
\pgfusepath{clip}%
\pgfsetrectcap%
\pgfsetroundjoin%
\pgfsetlinewidth{1.003750pt}%
\definecolor{currentstroke}{rgb}{0.000000,0.000000,0.000000}%
\pgfsetstrokecolor{currentstroke}%
\pgfsetdash{}{0pt}%
\pgfpathmoveto{\pgfqpoint{2.021875in}{0.750000in}}%
\pgfpathlineto{\pgfqpoint{2.015912in}{0.825773in}}%
\pgfpathlineto{\pgfqpoint{1.998168in}{0.899680in}}%
\pgfpathlineto{\pgfqpoint{1.969081in}{0.969902in}}%
\pgfpathlineto{\pgfqpoint{1.929368in}{1.034708in}}%
\pgfpathlineto{\pgfqpoint{1.880005in}{1.092505in}}%
\pgfpathlineto{\pgfqpoint{1.822208in}{1.141868in}}%
\pgfpathlineto{\pgfqpoint{1.757402in}{1.181581in}}%
\pgfpathlineto{\pgfqpoint{1.687180in}{1.210668in}}%
\pgfpathlineto{\pgfqpoint{1.613273in}{1.228412in}}%
\pgfpathlineto{\pgfqpoint{1.537500in}{1.234375in}}%
\pgfpathlineto{\pgfqpoint{1.461727in}{1.228412in}}%
\pgfpathlineto{\pgfqpoint{1.387820in}{1.210668in}}%
\pgfpathlineto{\pgfqpoint{1.317598in}{1.181581in}}%
\pgfpathlineto{\pgfqpoint{1.252792in}{1.141868in}}%
\pgfpathlineto{\pgfqpoint{1.194995in}{1.092505in}}%
\pgfpathlineto{\pgfqpoint{1.145632in}{1.034708in}}%
\pgfpathlineto{\pgfqpoint{1.105919in}{0.969902in}}%
\pgfpathlineto{\pgfqpoint{1.076832in}{0.899680in}}%
\pgfpathlineto{\pgfqpoint{1.059088in}{0.825773in}}%
\pgfpathlineto{\pgfqpoint{1.053125in}{0.750000in}}%
\pgfusepath{stroke}%
\end{pgfscope}%
\begin{pgfscope}%
\pgfpathrectangle{\pgfqpoint{0.375000in}{0.150000in}}{\pgfqpoint{2.325000in}{1.200000in}} %
\pgfusepath{clip}%
\pgfsetrectcap%
\pgfsetroundjoin%
\pgfsetlinewidth{1.003750pt}%
\definecolor{currentstroke}{rgb}{0.000000,0.000000,0.000000}%
\pgfsetstrokecolor{currentstroke}%
\pgfsetdash{}{0pt}%
\pgfpathmoveto{\pgfqpoint{1.053125in}{0.750000in}}%
\pgfpathlineto{\pgfqpoint{1.059088in}{0.674227in}}%
\pgfpathlineto{\pgfqpoint{1.076832in}{0.600320in}}%
\pgfpathlineto{\pgfqpoint{1.105919in}{0.530098in}}%
\pgfpathlineto{\pgfqpoint{1.145632in}{0.465292in}}%
\pgfpathlineto{\pgfqpoint{1.194995in}{0.407495in}}%
\pgfpathlineto{\pgfqpoint{1.252792in}{0.358132in}}%
\pgfpathlineto{\pgfqpoint{1.317598in}{0.318419in}}%
\pgfpathlineto{\pgfqpoint{1.387820in}{0.289332in}}%
\pgfpathlineto{\pgfqpoint{1.461727in}{0.271588in}}%
\pgfpathlineto{\pgfqpoint{1.537500in}{0.265625in}}%
\pgfpathlineto{\pgfqpoint{1.613273in}{0.271588in}}%
\pgfpathlineto{\pgfqpoint{1.687180in}{0.289332in}}%
\pgfpathlineto{\pgfqpoint{1.757402in}{0.318419in}}%
\pgfpathlineto{\pgfqpoint{1.822208in}{0.358132in}}%
\pgfpathlineto{\pgfqpoint{1.880005in}{0.407495in}}%
\pgfpathlineto{\pgfqpoint{1.929368in}{0.465292in}}%
\pgfpathlineto{\pgfqpoint{1.969081in}{0.530098in}}%
\pgfpathlineto{\pgfqpoint{1.998168in}{0.600320in}}%
\pgfpathlineto{\pgfqpoint{2.015912in}{0.674227in}}%
\pgfpathlineto{\pgfqpoint{2.021875in}{0.750000in}}%
\pgfusepath{stroke}%
\end{pgfscope}%
\begin{pgfscope}%
\pgfpathrectangle{\pgfqpoint{0.375000in}{0.150000in}}{\pgfqpoint{2.325000in}{1.200000in}} %
\pgfusepath{clip}%
\pgfsetbuttcap%
\pgfsetroundjoin%
\definecolor{currentfill}{rgb}{0.000000,0.000000,1.000000}%
\pgfsetfillcolor{currentfill}%
\pgfsetlinewidth{0.501875pt}%
\definecolor{currentstroke}{rgb}{0.000000,0.000000,0.000000}%
\pgfsetstrokecolor{currentstroke}%
\pgfsetdash{}{0pt}%
\pgfsys@defobject{currentmarker}{\pgfqpoint{-0.041667in}{-0.041667in}}{\pgfqpoint{0.041667in}{0.041667in}}{%
\pgfpathmoveto{\pgfqpoint{0.000000in}{-0.041667in}}%
\pgfpathcurveto{\pgfqpoint{0.011050in}{-0.041667in}}{\pgfqpoint{0.021649in}{-0.037276in}}{\pgfqpoint{0.029463in}{-0.029463in}}%
\pgfpathcurveto{\pgfqpoint{0.037276in}{-0.021649in}}{\pgfqpoint{0.041667in}{-0.011050in}}{\pgfqpoint{0.041667in}{0.000000in}}%
\pgfpathcurveto{\pgfqpoint{0.041667in}{0.011050in}}{\pgfqpoint{0.037276in}{0.021649in}}{\pgfqpoint{0.029463in}{0.029463in}}%
\pgfpathcurveto{\pgfqpoint{0.021649in}{0.037276in}}{\pgfqpoint{0.011050in}{0.041667in}}{\pgfqpoint{0.000000in}{0.041667in}}%
\pgfpathcurveto{\pgfqpoint{-0.011050in}{0.041667in}}{\pgfqpoint{-0.021649in}{0.037276in}}{\pgfqpoint{-0.029463in}{0.029463in}}%
\pgfpathcurveto{\pgfqpoint{-0.037276in}{0.021649in}}{\pgfqpoint{-0.041667in}{0.011050in}}{\pgfqpoint{-0.041667in}{0.000000in}}%
\pgfpathcurveto{\pgfqpoint{-0.041667in}{-0.011050in}}{\pgfqpoint{-0.037276in}{-0.021649in}}{\pgfqpoint{-0.029463in}{-0.029463in}}%
\pgfpathcurveto{\pgfqpoint{-0.021649in}{-0.037276in}}{\pgfqpoint{-0.011050in}{-0.041667in}}{\pgfqpoint{0.000000in}{-0.041667in}}%
\pgfpathclose%
\pgfusepath{stroke,fill}%
}%
\begin{pgfscope}%
\pgfsys@transformshift{1.537500in}{0.750000in}%
\pgfsys@useobject{currentmarker}{}%
\end{pgfscope}%
\end{pgfscope}%
\begin{pgfscope}%
\pgfpathrectangle{\pgfqpoint{0.375000in}{0.150000in}}{\pgfqpoint{2.325000in}{1.200000in}} %
\pgfusepath{clip}%
\pgfsetbuttcap%
\pgfsetroundjoin%
\definecolor{currentfill}{rgb}{0.000000,0.500000,0.000000}%
\pgfsetfillcolor{currentfill}%
\pgfsetlinewidth{0.501875pt}%
\definecolor{currentstroke}{rgb}{0.000000,0.000000,0.000000}%
\pgfsetstrokecolor{currentstroke}%
\pgfsetdash{}{0pt}%
\pgfsys@defobject{currentmarker}{\pgfqpoint{-0.041667in}{-0.041667in}}{\pgfqpoint{0.041667in}{0.041667in}}{%
\pgfpathmoveto{\pgfqpoint{0.000000in}{-0.041667in}}%
\pgfpathcurveto{\pgfqpoint{0.011050in}{-0.041667in}}{\pgfqpoint{0.021649in}{-0.037276in}}{\pgfqpoint{0.029463in}{-0.029463in}}%
\pgfpathcurveto{\pgfqpoint{0.037276in}{-0.021649in}}{\pgfqpoint{0.041667in}{-0.011050in}}{\pgfqpoint{0.041667in}{0.000000in}}%
\pgfpathcurveto{\pgfqpoint{0.041667in}{0.011050in}}{\pgfqpoint{0.037276in}{0.021649in}}{\pgfqpoint{0.029463in}{0.029463in}}%
\pgfpathcurveto{\pgfqpoint{0.021649in}{0.037276in}}{\pgfqpoint{0.011050in}{0.041667in}}{\pgfqpoint{0.000000in}{0.041667in}}%
\pgfpathcurveto{\pgfqpoint{-0.011050in}{0.041667in}}{\pgfqpoint{-0.021649in}{0.037276in}}{\pgfqpoint{-0.029463in}{0.029463in}}%
\pgfpathcurveto{\pgfqpoint{-0.037276in}{0.021649in}}{\pgfqpoint{-0.041667in}{0.011050in}}{\pgfqpoint{-0.041667in}{0.000000in}}%
\pgfpathcurveto{\pgfqpoint{-0.041667in}{-0.011050in}}{\pgfqpoint{-0.037276in}{-0.021649in}}{\pgfqpoint{-0.029463in}{-0.029463in}}%
\pgfpathcurveto{\pgfqpoint{-0.021649in}{-0.037276in}}{\pgfqpoint{-0.011050in}{-0.041667in}}{\pgfqpoint{0.000000in}{-0.041667in}}%
\pgfpathclose%
\pgfusepath{stroke,fill}%
}%
\begin{pgfscope}%
\pgfsys@transformshift{1.537500in}{0.265625in}%
\pgfsys@useobject{currentmarker}{}%
\end{pgfscope}%
\end{pgfscope}%
\begin{pgfscope}%
\pgftext[x=1.673125in,y=0.943750in,left,bottom]{{\sffamily\fontsize{20.000000}{24.000000}\selectfont \(\displaystyle \tau\)}}%
\end{pgfscope}%
\begin{pgfscope}%
\pgftext[x=1.537500in,y=0.750000in,right,top]{{\sffamily\fontsize{14.000000}{16.800000}\selectfont \(\displaystyle CM\)}}%
\end{pgfscope}%
\end{pgfpicture}%
\makeatother%
\endgroup%
\newline
In this case, the rule of friction is not to cause rotation; the rotation
about the center is done by the external torque. From 2nd law, in order to
have no slipping, a translational force is needed to accelerate the CM of
the object to move. Otherwise the object will slip. Friction in this case
acts as this \emph{force}, to give the CM of the object an acceleration to
the right. One can judge from the relative movement between the object and
the ground too. Friction prevents the object from skidding at the contact.
So friction acting on the object goes to the right, while the mutual
friction force acting on the ground goes to the left.%
\begin{eqnarray*}
&&\left\{ 
\begin{array}{c}
F_{fr}=ma \\ 
\tau -F_{fr}\cdot R=I\alpha =I\frac{a}{R}%
\end{array}%
\right. \\
&\Rightarrow &a=\frac{\tau }{mR+\frac{I}{R}}
\end{eqnarray*}%
%% Creator: Matplotlib, PGF backend
%%
%% To include the figure in your LaTeX document, write
%%   \input{<filename>.pgf}
%%
%% Make sure the required packages are loaded in your preamble
%%   \usepackage{pgf}
%%
%% Figures using additional raster images can only be included by \input if
%% they are in the same directory as the main LaTeX file. For loading figures
%% from other directories you can use the `import` package
%%   \usepackage{import}
%% and then include the figures with
%%   \import{<path to file>}{<filename>.pgf}
%%
%% Matplotlib used the following preamble
%%   \usepackage{fontspec}
%%   \setmainfont{Times New Roman}
%%   \setsansfont{Verdana}
%%   \setmonofont{Courier New}
%%
\begingroup%
\makeatletter%
\begin{pgfpicture}%
\pgfpathrectangle{\pgfpointorigin}{\pgfqpoint{4.000000in}{2.600000in}}%
\pgfusepath{use as bounding box}%
\begin{pgfscope}%
\pgfsetbuttcap%
\pgfsetroundjoin%
\definecolor{currentfill}{rgb}{1.000000,1.000000,1.000000}%
\pgfsetfillcolor{currentfill}%
\pgfsetlinewidth{0.000000pt}%
\definecolor{currentstroke}{rgb}{1.000000,1.000000,1.000000}%
\pgfsetstrokecolor{currentstroke}%
\pgfsetdash{}{0pt}%
\pgfpathmoveto{\pgfqpoint{0.000000in}{0.000000in}}%
\pgfpathlineto{\pgfqpoint{4.000000in}{0.000000in}}%
\pgfpathlineto{\pgfqpoint{4.000000in}{2.600000in}}%
\pgfpathlineto{\pgfqpoint{0.000000in}{2.600000in}}%
\pgfpathclose%
\pgfusepath{fill}%
\end{pgfscope}%
\begin{pgfscope}%
\pgfpathrectangle{\pgfqpoint{0.500000in}{0.260000in}}{\pgfqpoint{3.100000in}{2.080000in}} %
\pgfusepath{clip}%
\pgfsetbuttcap%
\pgfsetroundjoin%
\definecolor{currentfill}{rgb}{0.000000,0.000000,1.000000}%
\pgfsetfillcolor{currentfill}%
\pgfsetlinewidth{1.003750pt}%
\definecolor{currentstroke}{rgb}{0.000000,0.000000,0.000000}%
\pgfsetstrokecolor{currentstroke}%
\pgfsetdash{}{0pt}%
\pgfpathmoveto{\pgfqpoint{2.928333in}{0.654167in}}%
\pgfpathlineto{\pgfqpoint{2.695833in}{0.550833in}}%
\pgfpathlineto{\pgfqpoint{2.695833in}{0.630917in}}%
\pgfpathlineto{\pgfqpoint{2.050000in}{0.630917in}}%
\pgfpathlineto{\pgfqpoint{2.050000in}{0.677417in}}%
\pgfpathlineto{\pgfqpoint{2.695833in}{0.677417in}}%
\pgfpathlineto{\pgfqpoint{2.695833in}{0.757500in}}%
\pgfpathclose%
\pgfusepath{stroke,fill}%
\end{pgfscope}%
\begin{pgfscope}%
\pgfpathrectangle{\pgfqpoint{0.500000in}{0.260000in}}{\pgfqpoint{3.100000in}{2.080000in}} %
\pgfusepath{clip}%
\pgfsetbuttcap%
\pgfsetroundjoin%
\definecolor{currentfill}{rgb}{1.000000,0.000000,0.000000}%
\pgfsetfillcolor{currentfill}%
\pgfsetlinewidth{1.003750pt}%
\definecolor{currentstroke}{rgb}{1.000000,0.000000,0.000000}%
\pgfsetstrokecolor{currentstroke}%
\pgfsetdash{}{0pt}%
\pgfpathmoveto{\pgfqpoint{3.143611in}{1.300000in}}%
\pgfpathlineto{\pgfqpoint{2.911111in}{1.196667in}}%
\pgfpathlineto{\pgfqpoint{2.911111in}{1.276750in}}%
\pgfpathlineto{\pgfqpoint{2.050000in}{1.276750in}}%
\pgfpathlineto{\pgfqpoint{2.050000in}{1.323250in}}%
\pgfpathlineto{\pgfqpoint{2.911111in}{1.323250in}}%
\pgfpathlineto{\pgfqpoint{2.911111in}{1.403333in}}%
\pgfpathclose%
\pgfusepath{stroke,fill}%
\end{pgfscope}%
\begin{pgfscope}%
\pgfpathrectangle{\pgfqpoint{0.500000in}{0.260000in}}{\pgfqpoint{3.100000in}{2.080000in}} %
\pgfusepath{clip}%
\pgfsetbuttcap%
\pgfsetroundjoin%
\definecolor{currentfill}{rgb}{0.000000,0.000000,0.000000}%
\pgfsetfillcolor{currentfill}%
\pgfsetlinewidth{1.003750pt}%
\definecolor{currentstroke}{rgb}{0.000000,0.000000,0.000000}%
\pgfsetstrokecolor{currentstroke}%
\pgfsetdash{}{0pt}%
\pgfpathmoveto{\pgfqpoint{2.669032in}{1.970354in}}%
\pgfpathlineto{\pgfqpoint{2.568506in}{1.961888in}}%
\pgfpathlineto{\pgfqpoint{2.604801in}{2.013739in}}%
\pgfpathlineto{\pgfqpoint{2.567765in}{2.039664in}}%
\pgfpathlineto{\pgfqpoint{2.569246in}{2.041780in}}%
\pgfpathlineto{\pgfqpoint{2.606282in}{2.015855in}}%
\pgfpathlineto{\pgfqpoint{2.642578in}{2.067706in}}%
\pgfpathclose%
\pgfusepath{stroke,fill}%
\end{pgfscope}%
\begin{pgfscope}%
\pgfpathrectangle{\pgfqpoint{0.500000in}{0.260000in}}{\pgfqpoint{3.100000in}{2.080000in}} %
\pgfusepath{clip}%
\pgfsetbuttcap%
\pgfsetroundjoin%
\definecolor{currentfill}{rgb}{0.000000,0.000000,0.000000}%
\pgfsetfillcolor{currentfill}%
\pgfsetlinewidth{1.003750pt}%
\definecolor{currentstroke}{rgb}{0.000000,0.000000,0.000000}%
\pgfsetstrokecolor{currentstroke}%
\pgfsetdash{}{0pt}%
\pgfpathmoveto{\pgfqpoint{2.335707in}{1.494175in}}%
\pgfpathlineto{\pgfqpoint{2.235181in}{1.485710in}}%
\pgfpathlineto{\pgfqpoint{2.271476in}{1.537560in}}%
\pgfpathlineto{\pgfqpoint{2.234440in}{1.563485in}}%
\pgfpathlineto{\pgfqpoint{2.235921in}{1.565602in}}%
\pgfpathlineto{\pgfqpoint{2.272957in}{1.539677in}}%
\pgfpathlineto{\pgfqpoint{2.309253in}{1.591527in}}%
\pgfpathclose%
\pgfusepath{stroke,fill}%
\end{pgfscope}%
\begin{pgfscope}%
\pgfpathrectangle{\pgfqpoint{0.500000in}{0.260000in}}{\pgfqpoint{3.100000in}{2.080000in}} %
\pgfusepath{clip}%
\pgfsetrectcap%
\pgfsetroundjoin%
\pgfsetlinewidth{1.003750pt}%
\definecolor{currentstroke}{rgb}{0.000000,0.000000,1.000000}%
\pgfsetstrokecolor{currentstroke}%
\pgfsetdash{}{0pt}%
\pgfpathmoveto{\pgfqpoint{0.758333in}{0.654167in}}%
\pgfpathlineto{\pgfqpoint{3.341667in}{0.654167in}}%
\pgfusepath{stroke}%
\end{pgfscope}%
\begin{pgfscope}%
\pgfpathrectangle{\pgfqpoint{0.500000in}{0.260000in}}{\pgfqpoint{3.100000in}{2.080000in}} %
\pgfusepath{clip}%
\pgfsetrectcap%
\pgfsetroundjoin%
\pgfsetlinewidth{1.003750pt}%
\definecolor{currentstroke}{rgb}{0.000000,0.000000,0.000000}%
\pgfsetstrokecolor{currentstroke}%
\pgfsetdash{}{0pt}%
\pgfpathmoveto{\pgfqpoint{2.050000in}{0.654167in}}%
\pgfpathlineto{\pgfqpoint{2.050000in}{1.300000in}}%
\pgfusepath{stroke}%
\end{pgfscope}%
\begin{pgfscope}%
\pgfpathrectangle{\pgfqpoint{0.500000in}{0.260000in}}{\pgfqpoint{3.100000in}{2.080000in}} %
\pgfusepath{clip}%
\pgfsetrectcap%
\pgfsetroundjoin%
\pgfsetlinewidth{1.003750pt}%
\definecolor{currentstroke}{rgb}{0.000000,0.000000,0.000000}%
\pgfsetstrokecolor{currentstroke}%
\pgfsetdash{}{0pt}%
\pgfpathmoveto{\pgfqpoint{2.139968in}{2.199679in}}%
\pgfpathlineto{\pgfqpoint{2.185809in}{2.193909in}}%
\pgfpathlineto{\pgfqpoint{2.231296in}{2.185804in}}%
\pgfpathlineto{\pgfqpoint{2.276309in}{2.175386in}}%
\pgfpathlineto{\pgfqpoint{2.320732in}{2.162683in}}%
\pgfpathlineto{\pgfqpoint{2.364447in}{2.147727in}}%
\pgfpathlineto{\pgfqpoint{2.407341in}{2.130557in}}%
\pgfpathlineto{\pgfqpoint{2.449302in}{2.111218in}}%
\pgfpathlineto{\pgfqpoint{2.490221in}{2.089761in}}%
\pgfpathlineto{\pgfqpoint{2.529990in}{2.066242in}}%
\pgfpathlineto{\pgfqpoint{2.568506in}{2.040722in}}%
\pgfusepath{stroke}%
\end{pgfscope}%
\begin{pgfscope}%
\pgfpathrectangle{\pgfqpoint{0.500000in}{0.260000in}}{\pgfqpoint{3.100000in}{2.080000in}} %
\pgfusepath{clip}%
\pgfsetrectcap%
\pgfsetroundjoin%
\pgfsetlinewidth{1.003750pt}%
\definecolor{currentstroke}{rgb}{0.000000,0.000000,0.000000}%
\pgfsetstrokecolor{currentstroke}%
\pgfsetdash{}{0pt}%
\pgfpathmoveto{\pgfqpoint{2.050000in}{1.622917in}}%
\pgfpathlineto{\pgfqpoint{2.063144in}{1.622649in}}%
\pgfpathlineto{\pgfqpoint{2.076266in}{1.621847in}}%
\pgfpathlineto{\pgfqpoint{2.089345in}{1.620511in}}%
\pgfpathlineto{\pgfqpoint{2.102358in}{1.618644in}}%
\pgfpathlineto{\pgfqpoint{2.115285in}{1.616248in}}%
\pgfpathlineto{\pgfqpoint{2.128103in}{1.613329in}}%
\pgfpathlineto{\pgfqpoint{2.140792in}{1.609890in}}%
\pgfpathlineto{\pgfqpoint{2.153331in}{1.605938in}}%
\pgfpathlineto{\pgfqpoint{2.165698in}{1.601478in}}%
\pgfpathlineto{\pgfqpoint{2.177873in}{1.596519in}}%
\pgfpathlineto{\pgfqpoint{2.189837in}{1.591068in}}%
\pgfpathlineto{\pgfqpoint{2.201568in}{1.585135in}}%
\pgfpathlineto{\pgfqpoint{2.213049in}{1.578730in}}%
\pgfpathlineto{\pgfqpoint{2.224259in}{1.571862in}}%
\pgfpathlineto{\pgfqpoint{2.235181in}{1.564544in}}%
\pgfusepath{stroke}%
\end{pgfscope}%
\begin{pgfscope}%
\pgfpathrectangle{\pgfqpoint{0.500000in}{0.260000in}}{\pgfqpoint{3.100000in}{2.080000in}} %
\pgfusepath{clip}%
\pgfsetrectcap%
\pgfsetroundjoin%
\pgfsetlinewidth{1.003750pt}%
\definecolor{currentstroke}{rgb}{0.000000,0.000000,0.000000}%
\pgfsetstrokecolor{currentstroke}%
\pgfsetdash{}{0pt}%
\pgfpathmoveto{\pgfqpoint{2.695833in}{1.300000in}}%
\pgfpathlineto{\pgfqpoint{2.687882in}{1.401031in}}%
\pgfpathlineto{\pgfqpoint{2.664224in}{1.499573in}}%
\pgfpathlineto{\pgfqpoint{2.625442in}{1.593202in}}%
\pgfpathlineto{\pgfqpoint{2.572490in}{1.679611in}}%
\pgfpathlineto{\pgfqpoint{2.506673in}{1.756673in}}%
\pgfpathlineto{\pgfqpoint{2.429611in}{1.822490in}}%
\pgfpathlineto{\pgfqpoint{2.343202in}{1.875442in}}%
\pgfpathlineto{\pgfqpoint{2.249573in}{1.914224in}}%
\pgfpathlineto{\pgfqpoint{2.151031in}{1.937882in}}%
\pgfpathlineto{\pgfqpoint{2.050000in}{1.945833in}}%
\pgfpathlineto{\pgfqpoint{1.948969in}{1.937882in}}%
\pgfpathlineto{\pgfqpoint{1.850427in}{1.914224in}}%
\pgfpathlineto{\pgfqpoint{1.756798in}{1.875442in}}%
\pgfpathlineto{\pgfqpoint{1.670389in}{1.822490in}}%
\pgfpathlineto{\pgfqpoint{1.593327in}{1.756673in}}%
\pgfpathlineto{\pgfqpoint{1.527510in}{1.679611in}}%
\pgfpathlineto{\pgfqpoint{1.474558in}{1.593202in}}%
\pgfpathlineto{\pgfqpoint{1.435776in}{1.499573in}}%
\pgfpathlineto{\pgfqpoint{1.412118in}{1.401031in}}%
\pgfpathlineto{\pgfqpoint{1.404167in}{1.300000in}}%
\pgfusepath{stroke}%
\end{pgfscope}%
\begin{pgfscope}%
\pgfpathrectangle{\pgfqpoint{0.500000in}{0.260000in}}{\pgfqpoint{3.100000in}{2.080000in}} %
\pgfusepath{clip}%
\pgfsetrectcap%
\pgfsetroundjoin%
\pgfsetlinewidth{1.003750pt}%
\definecolor{currentstroke}{rgb}{0.000000,0.000000,0.000000}%
\pgfsetstrokecolor{currentstroke}%
\pgfsetdash{}{0pt}%
\pgfpathmoveto{\pgfqpoint{1.404167in}{1.300000in}}%
\pgfpathlineto{\pgfqpoint{1.412118in}{1.198969in}}%
\pgfpathlineto{\pgfqpoint{1.435776in}{1.100427in}}%
\pgfpathlineto{\pgfqpoint{1.474558in}{1.006798in}}%
\pgfpathlineto{\pgfqpoint{1.527510in}{0.920389in}}%
\pgfpathlineto{\pgfqpoint{1.593327in}{0.843327in}}%
\pgfpathlineto{\pgfqpoint{1.670389in}{0.777510in}}%
\pgfpathlineto{\pgfqpoint{1.756798in}{0.724558in}}%
\pgfpathlineto{\pgfqpoint{1.850427in}{0.685776in}}%
\pgfpathlineto{\pgfqpoint{1.948969in}{0.662118in}}%
\pgfpathlineto{\pgfqpoint{2.050000in}{0.654167in}}%
\pgfpathlineto{\pgfqpoint{2.151031in}{0.662118in}}%
\pgfpathlineto{\pgfqpoint{2.249573in}{0.685776in}}%
\pgfpathlineto{\pgfqpoint{2.343202in}{0.724558in}}%
\pgfpathlineto{\pgfqpoint{2.429611in}{0.777510in}}%
\pgfpathlineto{\pgfqpoint{2.506673in}{0.843327in}}%
\pgfpathlineto{\pgfqpoint{2.572490in}{0.920389in}}%
\pgfpathlineto{\pgfqpoint{2.625442in}{1.006798in}}%
\pgfpathlineto{\pgfqpoint{2.664224in}{1.100427in}}%
\pgfpathlineto{\pgfqpoint{2.687882in}{1.198969in}}%
\pgfpathlineto{\pgfqpoint{2.695833in}{1.300000in}}%
\pgfusepath{stroke}%
\end{pgfscope}%
\begin{pgfscope}%
\pgfpathrectangle{\pgfqpoint{0.500000in}{0.260000in}}{\pgfqpoint{3.100000in}{2.080000in}} %
\pgfusepath{clip}%
\pgfsetbuttcap%
\pgfsetroundjoin%
\definecolor{currentfill}{rgb}{0.000000,0.000000,1.000000}%
\pgfsetfillcolor{currentfill}%
\pgfsetlinewidth{0.501875pt}%
\definecolor{currentstroke}{rgb}{0.000000,0.000000,0.000000}%
\pgfsetstrokecolor{currentstroke}%
\pgfsetdash{}{0pt}%
\pgfsys@defobject{currentmarker}{\pgfqpoint{-0.041667in}{-0.041667in}}{\pgfqpoint{0.041667in}{0.041667in}}{%
\pgfpathmoveto{\pgfqpoint{0.000000in}{-0.041667in}}%
\pgfpathcurveto{\pgfqpoint{0.011050in}{-0.041667in}}{\pgfqpoint{0.021649in}{-0.037276in}}{\pgfqpoint{0.029463in}{-0.029463in}}%
\pgfpathcurveto{\pgfqpoint{0.037276in}{-0.021649in}}{\pgfqpoint{0.041667in}{-0.011050in}}{\pgfqpoint{0.041667in}{0.000000in}}%
\pgfpathcurveto{\pgfqpoint{0.041667in}{0.011050in}}{\pgfqpoint{0.037276in}{0.021649in}}{\pgfqpoint{0.029463in}{0.029463in}}%
\pgfpathcurveto{\pgfqpoint{0.021649in}{0.037276in}}{\pgfqpoint{0.011050in}{0.041667in}}{\pgfqpoint{0.000000in}{0.041667in}}%
\pgfpathcurveto{\pgfqpoint{-0.011050in}{0.041667in}}{\pgfqpoint{-0.021649in}{0.037276in}}{\pgfqpoint{-0.029463in}{0.029463in}}%
\pgfpathcurveto{\pgfqpoint{-0.037276in}{0.021649in}}{\pgfqpoint{-0.041667in}{0.011050in}}{\pgfqpoint{-0.041667in}{0.000000in}}%
\pgfpathcurveto{\pgfqpoint{-0.041667in}{-0.011050in}}{\pgfqpoint{-0.037276in}{-0.021649in}}{\pgfqpoint{-0.029463in}{-0.029463in}}%
\pgfpathcurveto{\pgfqpoint{-0.021649in}{-0.037276in}}{\pgfqpoint{-0.011050in}{-0.041667in}}{\pgfqpoint{0.000000in}{-0.041667in}}%
\pgfpathclose%
\pgfusepath{stroke,fill}%
}%
\begin{pgfscope}%
\pgfsys@transformshift{2.050000in}{1.300000in}%
\pgfsys@useobject{currentmarker}{}%
\end{pgfscope}%
\end{pgfscope}%
\begin{pgfscope}%
\pgfpathrectangle{\pgfqpoint{0.500000in}{0.260000in}}{\pgfqpoint{3.100000in}{2.080000in}} %
\pgfusepath{clip}%
\pgfsetbuttcap%
\pgfsetroundjoin%
\definecolor{currentfill}{rgb}{0.000000,0.500000,0.000000}%
\pgfsetfillcolor{currentfill}%
\pgfsetlinewidth{0.501875pt}%
\definecolor{currentstroke}{rgb}{0.000000,0.000000,0.000000}%
\pgfsetstrokecolor{currentstroke}%
\pgfsetdash{}{0pt}%
\pgfsys@defobject{currentmarker}{\pgfqpoint{-0.041667in}{-0.041667in}}{\pgfqpoint{0.041667in}{0.041667in}}{%
\pgfpathmoveto{\pgfqpoint{0.000000in}{-0.041667in}}%
\pgfpathcurveto{\pgfqpoint{0.011050in}{-0.041667in}}{\pgfqpoint{0.021649in}{-0.037276in}}{\pgfqpoint{0.029463in}{-0.029463in}}%
\pgfpathcurveto{\pgfqpoint{0.037276in}{-0.021649in}}{\pgfqpoint{0.041667in}{-0.011050in}}{\pgfqpoint{0.041667in}{0.000000in}}%
\pgfpathcurveto{\pgfqpoint{0.041667in}{0.011050in}}{\pgfqpoint{0.037276in}{0.021649in}}{\pgfqpoint{0.029463in}{0.029463in}}%
\pgfpathcurveto{\pgfqpoint{0.021649in}{0.037276in}}{\pgfqpoint{0.011050in}{0.041667in}}{\pgfqpoint{0.000000in}{0.041667in}}%
\pgfpathcurveto{\pgfqpoint{-0.011050in}{0.041667in}}{\pgfqpoint{-0.021649in}{0.037276in}}{\pgfqpoint{-0.029463in}{0.029463in}}%
\pgfpathcurveto{\pgfqpoint{-0.037276in}{0.021649in}}{\pgfqpoint{-0.041667in}{0.011050in}}{\pgfqpoint{-0.041667in}{0.000000in}}%
\pgfpathcurveto{\pgfqpoint{-0.041667in}{-0.011050in}}{\pgfqpoint{-0.037276in}{-0.021649in}}{\pgfqpoint{-0.029463in}{-0.029463in}}%
\pgfpathcurveto{\pgfqpoint{-0.021649in}{-0.037276in}}{\pgfqpoint{-0.011050in}{-0.041667in}}{\pgfqpoint{0.000000in}{-0.041667in}}%
\pgfpathclose%
\pgfusepath{stroke,fill}%
}%
\begin{pgfscope}%
\pgfsys@transformshift{2.050000in}{0.654167in}%
\pgfsys@useobject{currentmarker}{}%
\end{pgfscope}%
\end{pgfscope}%
\begin{pgfscope}%
\pgftext[x=2.050000in,y=0.847917in,right,bottom]{{\sffamily\fontsize{14.000000}{16.800000}\selectfont \(\displaystyle R\)}}%
\end{pgfscope}%
\begin{pgfscope}%
\pgftext[x=2.308333in,y=2.204167in,left,bottom]{{\sffamily\fontsize{20.000000}{24.000000}\selectfont \(\displaystyle \alpha\)}}%
\end{pgfscope}%
\begin{pgfscope}%
\pgftext[x=2.230833in,y=1.558333in,left,bottom]{{\sffamily\fontsize{20.000000}{24.000000}\selectfont \(\displaystyle \tau\)}}%
\end{pgfscope}%
\begin{pgfscope}%
\pgftext[x=2.050000in,y=1.300000in,right,bottom]{{\sffamily\fontsize{14.000000}{16.800000}\selectfont \(\displaystyle CM\)}}%
\end{pgfscope}%
\begin{pgfscope}%
\pgftext[x=2.695833in,y=0.525000in,left,top]{{\sffamily\fontsize{20.000000}{24.000000}\selectfont \(\displaystyle F_{fr}\)}}%
\end{pgfscope}%
\begin{pgfscope}%
\definecolor{textcolor}{rgb}{1.000000,0.000000,0.000000}%
\pgfsetstrokecolor{textcolor}%
\pgfsetfillcolor{textcolor}%
\pgftext[x=2.911111in,y=1.170833in,left,top]{{\sffamily\fontsize{20.000000}{24.000000}\selectfont \(\displaystyle a\)}}%
\end{pgfscope}%
\end{pgfpicture}%
\makeatother%
\endgroup%

\end{case}

\newpage

\begin{case}
External forces not passing through CM pivot. (so there are both external
force and torque.)
\end{case}

%% Creator: Matplotlib, PGF backend
%%
%% To include the figure in your LaTeX document, write
%%   \input{<filename>.pgf}
%%
%% Make sure the required packages are loaded in your preamble
%%   \usepackage{pgf}
%%
%% Figures using additional raster images can only be included by \input if
%% they are in the same directory as the main LaTeX file. For loading figures
%% from other directories you can use the `import` package
%%   \usepackage{import}
%% and then include the figures with
%%   \import{<path to file>}{<filename>.pgf}
%%
%% Matplotlib used the following preamble
%%   \usepackage{fontspec}
%%   \setmainfont{Times New Roman}
%%   \setsansfont{Verdana}
%%   \setmonofont{Courier New}
%%
\begingroup%
\makeatletter%
\begin{pgfpicture}%
\pgfpathrectangle{\pgfpointorigin}{\pgfqpoint{3.000000in}{1.500000in}}%
\pgfusepath{use as bounding box}%
\begin{pgfscope}%
\pgfsetbuttcap%
\pgfsetroundjoin%
\definecolor{currentfill}{rgb}{1.000000,1.000000,1.000000}%
\pgfsetfillcolor{currentfill}%
\pgfsetlinewidth{0.000000pt}%
\definecolor{currentstroke}{rgb}{1.000000,1.000000,1.000000}%
\pgfsetstrokecolor{currentstroke}%
\pgfsetdash{}{0pt}%
\pgfpathmoveto{\pgfqpoint{0.000000in}{0.000000in}}%
\pgfpathlineto{\pgfqpoint{3.000000in}{0.000000in}}%
\pgfpathlineto{\pgfqpoint{3.000000in}{1.500000in}}%
\pgfpathlineto{\pgfqpoint{0.000000in}{1.500000in}}%
\pgfpathclose%
\pgfusepath{fill}%
\end{pgfscope}%
\begin{pgfscope}%
\pgfpathrectangle{\pgfqpoint{0.375000in}{0.150000in}}{\pgfqpoint{2.325000in}{1.200000in}} %
\pgfusepath{clip}%
\pgfsetbuttcap%
\pgfsetroundjoin%
\definecolor{currentfill}{rgb}{0.000000,0.000000,0.000000}%
\pgfsetfillcolor{currentfill}%
\pgfsetlinewidth{1.003750pt}%
\definecolor{currentstroke}{rgb}{0.000000,0.000000,0.000000}%
\pgfsetstrokecolor{currentstroke}%
\pgfsetdash{}{0pt}%
\pgfpathmoveto{\pgfqpoint{2.700000in}{1.292857in}}%
\pgfpathlineto{\pgfqpoint{2.632282in}{1.235714in}}%
\pgfpathlineto{\pgfqpoint{2.632282in}{1.291714in}}%
\pgfpathlineto{\pgfqpoint{1.503641in}{1.291714in}}%
\pgfpathlineto{\pgfqpoint{1.503641in}{1.294000in}}%
\pgfpathlineto{\pgfqpoint{2.632282in}{1.294000in}}%
\pgfpathlineto{\pgfqpoint{2.632282in}{1.350000in}}%
\pgfpathclose%
\pgfusepath{stroke,fill}%
\end{pgfscope}%
\begin{pgfscope}%
\pgfpathrectangle{\pgfqpoint{0.375000in}{0.150000in}}{\pgfqpoint{2.325000in}{1.200000in}} %
\pgfusepath{clip}%
\pgfsetrectcap%
\pgfsetroundjoin%
\pgfsetlinewidth{1.003750pt}%
\definecolor{currentstroke}{rgb}{0.000000,0.000000,1.000000}%
\pgfsetstrokecolor{currentstroke}%
\pgfsetdash{}{0pt}%
\pgfpathmoveto{\pgfqpoint{0.375000in}{0.150000in}}%
\pgfpathlineto{\pgfqpoint{2.632282in}{0.150000in}}%
\pgfusepath{stroke}%
\end{pgfscope}%
\begin{pgfscope}%
\pgfpathrectangle{\pgfqpoint{0.375000in}{0.150000in}}{\pgfqpoint{2.325000in}{1.200000in}} %
\pgfusepath{clip}%
\pgfsetrectcap%
\pgfsetroundjoin%
\pgfsetlinewidth{1.003750pt}%
\definecolor{currentstroke}{rgb}{0.000000,0.000000,0.000000}%
\pgfsetstrokecolor{currentstroke}%
\pgfsetdash{}{0pt}%
\pgfpathmoveto{\pgfqpoint{1.503641in}{0.150000in}}%
\pgfpathlineto{\pgfqpoint{1.503641in}{0.721429in}}%
\pgfusepath{stroke}%
\end{pgfscope}%
\begin{pgfscope}%
\pgfpathrectangle{\pgfqpoint{0.375000in}{0.150000in}}{\pgfqpoint{2.325000in}{1.200000in}} %
\pgfusepath{clip}%
\pgfsetrectcap%
\pgfsetroundjoin%
\pgfsetlinewidth{1.003750pt}%
\definecolor{currentstroke}{rgb}{0.000000,0.000000,0.000000}%
\pgfsetstrokecolor{currentstroke}%
\pgfsetdash{}{0pt}%
\pgfpathmoveto{\pgfqpoint{2.067961in}{0.721429in}}%
\pgfpathlineto{\pgfqpoint{2.061013in}{0.810820in}}%
\pgfpathlineto{\pgfqpoint{2.040341in}{0.898010in}}%
\pgfpathlineto{\pgfqpoint{2.006454in}{0.980852in}}%
\pgfpathlineto{\pgfqpoint{1.960186in}{1.057306in}}%
\pgfpathlineto{\pgfqpoint{1.902676in}{1.125490in}}%
\pgfpathlineto{\pgfqpoint{1.835340in}{1.183724in}}%
\pgfpathlineto{\pgfqpoint{1.759837in}{1.230575in}}%
\pgfpathlineto{\pgfqpoint{1.678025in}{1.264889in}}%
\pgfpathlineto{\pgfqpoint{1.591920in}{1.285822in}}%
\pgfpathlineto{\pgfqpoint{1.503641in}{1.292857in}}%
\pgfpathlineto{\pgfqpoint{1.415362in}{1.285822in}}%
\pgfpathlineto{\pgfqpoint{1.329256in}{1.264889in}}%
\pgfpathlineto{\pgfqpoint{1.247445in}{1.230575in}}%
\pgfpathlineto{\pgfqpoint{1.171942in}{1.183724in}}%
\pgfpathlineto{\pgfqpoint{1.104606in}{1.125490in}}%
\pgfpathlineto{\pgfqpoint{1.047096in}{1.057306in}}%
\pgfpathlineto{\pgfqpoint{1.000828in}{0.980852in}}%
\pgfpathlineto{\pgfqpoint{0.966940in}{0.898010in}}%
\pgfpathlineto{\pgfqpoint{0.946268in}{0.810820in}}%
\pgfpathlineto{\pgfqpoint{0.939320in}{0.721429in}}%
\pgfusepath{stroke}%
\end{pgfscope}%
\begin{pgfscope}%
\pgfpathrectangle{\pgfqpoint{0.375000in}{0.150000in}}{\pgfqpoint{2.325000in}{1.200000in}} %
\pgfusepath{clip}%
\pgfsetrectcap%
\pgfsetroundjoin%
\pgfsetlinewidth{1.003750pt}%
\definecolor{currentstroke}{rgb}{0.000000,0.000000,0.000000}%
\pgfsetstrokecolor{currentstroke}%
\pgfsetdash{}{0pt}%
\pgfpathmoveto{\pgfqpoint{0.939320in}{0.721429in}}%
\pgfpathlineto{\pgfqpoint{0.946268in}{0.632037in}}%
\pgfpathlineto{\pgfqpoint{0.966940in}{0.544847in}}%
\pgfpathlineto{\pgfqpoint{1.000828in}{0.462005in}}%
\pgfpathlineto{\pgfqpoint{1.047096in}{0.385551in}}%
\pgfpathlineto{\pgfqpoint{1.104606in}{0.317368in}}%
\pgfpathlineto{\pgfqpoint{1.171942in}{0.259133in}}%
\pgfpathlineto{\pgfqpoint{1.247445in}{0.212282in}}%
\pgfpathlineto{\pgfqpoint{1.329256in}{0.177968in}}%
\pgfpathlineto{\pgfqpoint{1.415362in}{0.157035in}}%
\pgfpathlineto{\pgfqpoint{1.503641in}{0.150000in}}%
\pgfpathlineto{\pgfqpoint{1.591920in}{0.157035in}}%
\pgfpathlineto{\pgfqpoint{1.678025in}{0.177968in}}%
\pgfpathlineto{\pgfqpoint{1.759837in}{0.212282in}}%
\pgfpathlineto{\pgfqpoint{1.835340in}{0.259133in}}%
\pgfpathlineto{\pgfqpoint{1.902676in}{0.317368in}}%
\pgfpathlineto{\pgfqpoint{1.960186in}{0.385551in}}%
\pgfpathlineto{\pgfqpoint{2.006454in}{0.462005in}}%
\pgfpathlineto{\pgfqpoint{2.040341in}{0.544847in}}%
\pgfpathlineto{\pgfqpoint{2.061013in}{0.632037in}}%
\pgfpathlineto{\pgfqpoint{2.067961in}{0.721429in}}%
\pgfusepath{stroke}%
\end{pgfscope}%
\begin{pgfscope}%
\pgfpathrectangle{\pgfqpoint{0.375000in}{0.150000in}}{\pgfqpoint{2.325000in}{1.200000in}} %
\pgfusepath{clip}%
\pgfsetbuttcap%
\pgfsetroundjoin%
\definecolor{currentfill}{rgb}{0.000000,0.000000,1.000000}%
\pgfsetfillcolor{currentfill}%
\pgfsetlinewidth{0.501875pt}%
\definecolor{currentstroke}{rgb}{0.000000,0.000000,0.000000}%
\pgfsetstrokecolor{currentstroke}%
\pgfsetdash{}{0pt}%
\pgfsys@defobject{currentmarker}{\pgfqpoint{-0.041667in}{-0.041667in}}{\pgfqpoint{0.041667in}{0.041667in}}{%
\pgfpathmoveto{\pgfqpoint{0.000000in}{-0.041667in}}%
\pgfpathcurveto{\pgfqpoint{0.011050in}{-0.041667in}}{\pgfqpoint{0.021649in}{-0.037276in}}{\pgfqpoint{0.029463in}{-0.029463in}}%
\pgfpathcurveto{\pgfqpoint{0.037276in}{-0.021649in}}{\pgfqpoint{0.041667in}{-0.011050in}}{\pgfqpoint{0.041667in}{0.000000in}}%
\pgfpathcurveto{\pgfqpoint{0.041667in}{0.011050in}}{\pgfqpoint{0.037276in}{0.021649in}}{\pgfqpoint{0.029463in}{0.029463in}}%
\pgfpathcurveto{\pgfqpoint{0.021649in}{0.037276in}}{\pgfqpoint{0.011050in}{0.041667in}}{\pgfqpoint{0.000000in}{0.041667in}}%
\pgfpathcurveto{\pgfqpoint{-0.011050in}{0.041667in}}{\pgfqpoint{-0.021649in}{0.037276in}}{\pgfqpoint{-0.029463in}{0.029463in}}%
\pgfpathcurveto{\pgfqpoint{-0.037276in}{0.021649in}}{\pgfqpoint{-0.041667in}{0.011050in}}{\pgfqpoint{-0.041667in}{0.000000in}}%
\pgfpathcurveto{\pgfqpoint{-0.041667in}{-0.011050in}}{\pgfqpoint{-0.037276in}{-0.021649in}}{\pgfqpoint{-0.029463in}{-0.029463in}}%
\pgfpathcurveto{\pgfqpoint{-0.021649in}{-0.037276in}}{\pgfqpoint{-0.011050in}{-0.041667in}}{\pgfqpoint{0.000000in}{-0.041667in}}%
\pgfpathclose%
\pgfusepath{stroke,fill}%
}%
\begin{pgfscope}%
\pgfsys@transformshift{1.503641in}{0.721429in}%
\pgfsys@useobject{currentmarker}{}%
\end{pgfscope}%
\end{pgfscope}%
\begin{pgfscope}%
\pgfpathrectangle{\pgfqpoint{0.375000in}{0.150000in}}{\pgfqpoint{2.325000in}{1.200000in}} %
\pgfusepath{clip}%
\pgfsetbuttcap%
\pgfsetroundjoin%
\definecolor{currentfill}{rgb}{0.000000,0.500000,0.000000}%
\pgfsetfillcolor{currentfill}%
\pgfsetlinewidth{0.501875pt}%
\definecolor{currentstroke}{rgb}{0.000000,0.000000,0.000000}%
\pgfsetstrokecolor{currentstroke}%
\pgfsetdash{}{0pt}%
\pgfsys@defobject{currentmarker}{\pgfqpoint{-0.041667in}{-0.041667in}}{\pgfqpoint{0.041667in}{0.041667in}}{%
\pgfpathmoveto{\pgfqpoint{0.000000in}{-0.041667in}}%
\pgfpathcurveto{\pgfqpoint{0.011050in}{-0.041667in}}{\pgfqpoint{0.021649in}{-0.037276in}}{\pgfqpoint{0.029463in}{-0.029463in}}%
\pgfpathcurveto{\pgfqpoint{0.037276in}{-0.021649in}}{\pgfqpoint{0.041667in}{-0.011050in}}{\pgfqpoint{0.041667in}{0.000000in}}%
\pgfpathcurveto{\pgfqpoint{0.041667in}{0.011050in}}{\pgfqpoint{0.037276in}{0.021649in}}{\pgfqpoint{0.029463in}{0.029463in}}%
\pgfpathcurveto{\pgfqpoint{0.021649in}{0.037276in}}{\pgfqpoint{0.011050in}{0.041667in}}{\pgfqpoint{0.000000in}{0.041667in}}%
\pgfpathcurveto{\pgfqpoint{-0.011050in}{0.041667in}}{\pgfqpoint{-0.021649in}{0.037276in}}{\pgfqpoint{-0.029463in}{0.029463in}}%
\pgfpathcurveto{\pgfqpoint{-0.037276in}{0.021649in}}{\pgfqpoint{-0.041667in}{0.011050in}}{\pgfqpoint{-0.041667in}{0.000000in}}%
\pgfpathcurveto{\pgfqpoint{-0.041667in}{-0.011050in}}{\pgfqpoint{-0.037276in}{-0.021649in}}{\pgfqpoint{-0.029463in}{-0.029463in}}%
\pgfpathcurveto{\pgfqpoint{-0.021649in}{-0.037276in}}{\pgfqpoint{-0.011050in}{-0.041667in}}{\pgfqpoint{0.000000in}{-0.041667in}}%
\pgfpathclose%
\pgfusepath{stroke,fill}%
}%
\begin{pgfscope}%
\pgfsys@transformshift{1.503641in}{0.150000in}%
\pgfsys@useobject{currentmarker}{}%
\end{pgfscope}%
\end{pgfscope}%
\begin{pgfscope}%
\pgftext[x=1.435922in,y=0.321429in,right,bottom]{{\sffamily\fontsize{12.000000}{14.400000}\selectfont \(\displaystyle R\)}}%
\end{pgfscope}%
\begin{pgfscope}%
\pgftext[x=1.503641in,y=0.721429in,right,bottom]{{\sffamily\fontsize{12.000000}{14.400000}\selectfont \(\displaystyle CM\)}}%
\end{pgfscope}%
\begin{pgfscope}%
\pgftext[x=2.632282in,y=1.178571in,left,top]{{\sffamily\fontsize{12.000000}{14.400000}\selectfont \(\displaystyle F_p\)}}%
\end{pgfscope}%
\end{pgfpicture}%
\makeatother%
\endgroup%


\bigskip

%TCIMACRO{\TeXButton{2columns}{\begin{multicols}{2}}}%
%BeginExpansion
\begin{multicols}{2}%
%EndExpansion

Let assume friction acting on the wheel goes right, then

%% Creator: Matplotlib, PGF backend
%%
%% To include the figure in your LaTeX document, write
%%   \input{<filename>.pgf}
%%
%% Make sure the required packages are loaded in your preamble
%%   \usepackage{pgf}
%%
%% Figures using additional raster images can only be included by \input if
%% they are in the same directory as the main LaTeX file. For loading figures
%% from other directories you can use the `import` package
%%   \usepackage{import}
%% and then include the figures with
%%   \import{<path to file>}{<filename>.pgf}
%%
%% Matplotlib used the following preamble
%%   \usepackage{fontspec}
%%   \setmainfont{Times New Roman}
%%   \setsansfont{Verdana}
%%   \setmonofont{Courier New}
%%
\begingroup%
\makeatletter%
\begin{pgfpicture}%
\pgfpathrectangle{\pgfpointorigin}{\pgfqpoint{3.000000in}{1.500000in}}%
\pgfusepath{use as bounding box}%
\begin{pgfscope}%
\pgfsetbuttcap%
\pgfsetroundjoin%
\definecolor{currentfill}{rgb}{1.000000,1.000000,1.000000}%
\pgfsetfillcolor{currentfill}%
\pgfsetlinewidth{0.000000pt}%
\definecolor{currentstroke}{rgb}{1.000000,1.000000,1.000000}%
\pgfsetstrokecolor{currentstroke}%
\pgfsetdash{}{0pt}%
\pgfpathmoveto{\pgfqpoint{0.000000in}{0.000000in}}%
\pgfpathlineto{\pgfqpoint{3.000000in}{0.000000in}}%
\pgfpathlineto{\pgfqpoint{3.000000in}{1.500000in}}%
\pgfpathlineto{\pgfqpoint{0.000000in}{1.500000in}}%
\pgfpathclose%
\pgfusepath{fill}%
\end{pgfscope}%
\begin{pgfscope}%
\pgfpathrectangle{\pgfqpoint{0.375000in}{0.150000in}}{\pgfqpoint{2.325000in}{1.200000in}} %
\pgfusepath{clip}%
\pgfsetbuttcap%
\pgfsetroundjoin%
\definecolor{currentfill}{rgb}{0.000000,0.000000,1.000000}%
\pgfsetfillcolor{currentfill}%
\pgfsetlinewidth{1.003750pt}%
\definecolor{currentstroke}{rgb}{0.000000,0.000000,0.000000}%
\pgfsetstrokecolor{currentstroke}%
\pgfsetdash{}{0pt}%
\pgfpathmoveto{\pgfqpoint{2.271117in}{0.234956in}}%
\pgfpathlineto{\pgfqpoint{2.067961in}{0.150000in}}%
\pgfpathlineto{\pgfqpoint{2.067961in}{0.215841in}}%
\pgfpathlineto{\pgfqpoint{1.503641in}{0.215841in}}%
\pgfpathlineto{\pgfqpoint{1.503641in}{0.254071in}}%
\pgfpathlineto{\pgfqpoint{2.067961in}{0.254071in}}%
\pgfpathlineto{\pgfqpoint{2.067961in}{0.319912in}}%
\pgfpathclose%
\pgfusepath{stroke,fill}%
\end{pgfscope}%
\begin{pgfscope}%
\pgfpathrectangle{\pgfqpoint{0.375000in}{0.150000in}}{\pgfqpoint{2.325000in}{1.200000in}} %
\pgfusepath{clip}%
\pgfsetbuttcap%
\pgfsetroundjoin%
\definecolor{currentfill}{rgb}{1.000000,0.000000,0.000000}%
\pgfsetfillcolor{currentfill}%
\pgfsetlinewidth{1.003750pt}%
\definecolor{currentstroke}{rgb}{1.000000,0.000000,0.000000}%
\pgfsetstrokecolor{currentstroke}%
\pgfsetdash{}{0pt}%
\pgfpathmoveto{\pgfqpoint{2.459223in}{0.765929in}}%
\pgfpathlineto{\pgfqpoint{2.256068in}{0.680973in}}%
\pgfpathlineto{\pgfqpoint{2.256068in}{0.746814in}}%
\pgfpathlineto{\pgfqpoint{1.503641in}{0.746814in}}%
\pgfpathlineto{\pgfqpoint{1.503641in}{0.785044in}}%
\pgfpathlineto{\pgfqpoint{2.256068in}{0.785044in}}%
\pgfpathlineto{\pgfqpoint{2.256068in}{0.850885in}}%
\pgfpathclose%
\pgfusepath{stroke,fill}%
\end{pgfscope}%
\begin{pgfscope}%
\pgfpathrectangle{\pgfqpoint{0.375000in}{0.150000in}}{\pgfqpoint{2.325000in}{1.200000in}} %
\pgfusepath{clip}%
\pgfsetbuttcap%
\pgfsetroundjoin%
\definecolor{currentfill}{rgb}{0.000000,0.000000,0.000000}%
\pgfsetfillcolor{currentfill}%
\pgfsetlinewidth{1.003750pt}%
\definecolor{currentstroke}{rgb}{0.000000,0.000000,0.000000}%
\pgfsetstrokecolor{currentstroke}%
\pgfsetdash{}{0pt}%
\pgfpathmoveto{\pgfqpoint{2.700000in}{1.296903in}}%
\pgfpathlineto{\pgfqpoint{2.632282in}{1.243805in}}%
\pgfpathlineto{\pgfqpoint{2.632282in}{1.295841in}}%
\pgfpathlineto{\pgfqpoint{1.503641in}{1.295841in}}%
\pgfpathlineto{\pgfqpoint{1.503641in}{1.297965in}}%
\pgfpathlineto{\pgfqpoint{2.632282in}{1.297965in}}%
\pgfpathlineto{\pgfqpoint{2.632282in}{1.350000in}}%
\pgfpathclose%
\pgfusepath{stroke,fill}%
\end{pgfscope}%
\begin{pgfscope}%
\pgfpathrectangle{\pgfqpoint{0.375000in}{0.150000in}}{\pgfqpoint{2.325000in}{1.200000in}} %
\pgfusepath{clip}%
\pgfsetbuttcap%
\pgfsetroundjoin%
\definecolor{currentfill}{rgb}{0.000000,0.000000,0.000000}%
\pgfsetfillcolor{currentfill}%
\pgfsetlinewidth{1.003750pt}%
\definecolor{currentstroke}{rgb}{0.000000,0.000000,0.000000}%
\pgfsetstrokecolor{currentstroke}%
\pgfsetdash{}{0pt}%
\pgfpathmoveto{\pgfqpoint{1.753288in}{0.925571in}}%
\pgfpathlineto{\pgfqpoint{1.665449in}{0.918611in}}%
\pgfpathlineto{\pgfqpoint{1.697163in}{0.961240in}}%
\pgfpathlineto{\pgfqpoint{1.664802in}{0.982554in}}%
\pgfpathlineto{\pgfqpoint{1.666096in}{0.984294in}}%
\pgfpathlineto{\pgfqpoint{1.698458in}{0.962980in}}%
\pgfpathlineto{\pgfqpoint{1.730172in}{1.005609in}}%
\pgfpathclose%
\pgfusepath{stroke,fill}%
\end{pgfscope}%
\begin{pgfscope}%
\pgfpathrectangle{\pgfqpoint{0.375000in}{0.150000in}}{\pgfqpoint{2.325000in}{1.200000in}} %
\pgfusepath{clip}%
\pgfsetrectcap%
\pgfsetroundjoin%
\pgfsetlinewidth{1.003750pt}%
\definecolor{currentstroke}{rgb}{0.000000,0.000000,1.000000}%
\pgfsetstrokecolor{currentstroke}%
\pgfsetdash{}{0pt}%
\pgfpathmoveto{\pgfqpoint{0.375000in}{0.234956in}}%
\pgfpathlineto{\pgfqpoint{2.632282in}{0.234956in}}%
\pgfusepath{stroke}%
\end{pgfscope}%
\begin{pgfscope}%
\pgfpathrectangle{\pgfqpoint{0.375000in}{0.150000in}}{\pgfqpoint{2.325000in}{1.200000in}} %
\pgfusepath{clip}%
\pgfsetrectcap%
\pgfsetroundjoin%
\pgfsetlinewidth{1.003750pt}%
\definecolor{currentstroke}{rgb}{0.000000,0.000000,0.000000}%
\pgfsetstrokecolor{currentstroke}%
\pgfsetdash{}{0pt}%
\pgfpathmoveto{\pgfqpoint{1.503641in}{0.234956in}}%
\pgfpathlineto{\pgfqpoint{1.503641in}{0.765929in}}%
\pgfusepath{stroke}%
\end{pgfscope}%
\begin{pgfscope}%
\pgfpathrectangle{\pgfqpoint{0.375000in}{0.150000in}}{\pgfqpoint{2.325000in}{1.200000in}} %
\pgfusepath{clip}%
\pgfsetrectcap%
\pgfsetroundjoin%
\pgfsetlinewidth{1.003750pt}%
\definecolor{currentstroke}{rgb}{0.000000,0.000000,0.000000}%
\pgfsetstrokecolor{currentstroke}%
\pgfsetdash{}{0pt}%
\pgfpathmoveto{\pgfqpoint{1.503641in}{1.031416in}}%
\pgfpathlineto{\pgfqpoint{1.515126in}{1.031196in}}%
\pgfpathlineto{\pgfqpoint{1.526592in}{1.030536in}}%
\pgfpathlineto{\pgfqpoint{1.538020in}{1.029438in}}%
\pgfpathlineto{\pgfqpoint{1.549391in}{1.027903in}}%
\pgfpathlineto{\pgfqpoint{1.560686in}{1.025934in}}%
\pgfpathlineto{\pgfqpoint{1.571886in}{1.023533in}}%
\pgfpathlineto{\pgfqpoint{1.582974in}{1.020706in}}%
\pgfpathlineto{\pgfqpoint{1.593930in}{1.017457in}}%
\pgfpathlineto{\pgfqpoint{1.604736in}{1.013790in}}%
\pgfpathlineto{\pgfqpoint{1.615375in}{1.009713in}}%
\pgfpathlineto{\pgfqpoint{1.625828in}{1.005232in}}%
\pgfpathlineto{\pgfqpoint{1.636079in}{1.000354in}}%
\pgfpathlineto{\pgfqpoint{1.646111in}{0.995087in}}%
\pgfpathlineto{\pgfqpoint{1.655906in}{0.989441in}}%
\pgfpathlineto{\pgfqpoint{1.665449in}{0.983424in}}%
\pgfusepath{stroke}%
\end{pgfscope}%
\begin{pgfscope}%
\pgfpathrectangle{\pgfqpoint{0.375000in}{0.150000in}}{\pgfqpoint{2.325000in}{1.200000in}} %
\pgfusepath{clip}%
\pgfsetrectcap%
\pgfsetroundjoin%
\pgfsetlinewidth{1.003750pt}%
\definecolor{currentstroke}{rgb}{0.000000,0.000000,0.000000}%
\pgfsetstrokecolor{currentstroke}%
\pgfsetdash{}{0pt}%
\pgfpathmoveto{\pgfqpoint{2.067961in}{0.765929in}}%
\pgfpathlineto{\pgfqpoint{2.061013in}{0.848992in}}%
\pgfpathlineto{\pgfqpoint{2.040341in}{0.930009in}}%
\pgfpathlineto{\pgfqpoint{2.006454in}{1.006986in}}%
\pgfpathlineto{\pgfqpoint{1.960186in}{1.078028in}}%
\pgfpathlineto{\pgfqpoint{1.902676in}{1.141384in}}%
\pgfpathlineto{\pgfqpoint{1.835340in}{1.195496in}}%
\pgfpathlineto{\pgfqpoint{1.759837in}{1.239030in}}%
\pgfpathlineto{\pgfqpoint{1.678025in}{1.270915in}}%
\pgfpathlineto{\pgfqpoint{1.591920in}{1.290365in}}%
\pgfpathlineto{\pgfqpoint{1.503641in}{1.296903in}}%
\pgfpathlineto{\pgfqpoint{1.415362in}{1.290365in}}%
\pgfpathlineto{\pgfqpoint{1.329256in}{1.270915in}}%
\pgfpathlineto{\pgfqpoint{1.247445in}{1.239030in}}%
\pgfpathlineto{\pgfqpoint{1.171942in}{1.195496in}}%
\pgfpathlineto{\pgfqpoint{1.104606in}{1.141384in}}%
\pgfpathlineto{\pgfqpoint{1.047096in}{1.078028in}}%
\pgfpathlineto{\pgfqpoint{1.000828in}{1.006986in}}%
\pgfpathlineto{\pgfqpoint{0.966940in}{0.930009in}}%
\pgfpathlineto{\pgfqpoint{0.946268in}{0.848992in}}%
\pgfpathlineto{\pgfqpoint{0.939320in}{0.765929in}}%
\pgfusepath{stroke}%
\end{pgfscope}%
\begin{pgfscope}%
\pgfpathrectangle{\pgfqpoint{0.375000in}{0.150000in}}{\pgfqpoint{2.325000in}{1.200000in}} %
\pgfusepath{clip}%
\pgfsetrectcap%
\pgfsetroundjoin%
\pgfsetlinewidth{1.003750pt}%
\definecolor{currentstroke}{rgb}{0.000000,0.000000,0.000000}%
\pgfsetstrokecolor{currentstroke}%
\pgfsetdash{}{0pt}%
\pgfpathmoveto{\pgfqpoint{0.939320in}{0.765929in}}%
\pgfpathlineto{\pgfqpoint{0.946268in}{0.682867in}}%
\pgfpathlineto{\pgfqpoint{0.966940in}{0.601849in}}%
\pgfpathlineto{\pgfqpoint{1.000828in}{0.524872in}}%
\pgfpathlineto{\pgfqpoint{1.047096in}{0.453831in}}%
\pgfpathlineto{\pgfqpoint{1.104606in}{0.390474in}}%
\pgfpathlineto{\pgfqpoint{1.171942in}{0.336363in}}%
\pgfpathlineto{\pgfqpoint{1.247445in}{0.292828in}}%
\pgfpathlineto{\pgfqpoint{1.329256in}{0.260943in}}%
\pgfpathlineto{\pgfqpoint{1.415362in}{0.241493in}}%
\pgfpathlineto{\pgfqpoint{1.503641in}{0.234956in}}%
\pgfpathlineto{\pgfqpoint{1.591920in}{0.241493in}}%
\pgfpathlineto{\pgfqpoint{1.678025in}{0.260943in}}%
\pgfpathlineto{\pgfqpoint{1.759837in}{0.292828in}}%
\pgfpathlineto{\pgfqpoint{1.835340in}{0.336363in}}%
\pgfpathlineto{\pgfqpoint{1.902676in}{0.390474in}}%
\pgfpathlineto{\pgfqpoint{1.960186in}{0.453831in}}%
\pgfpathlineto{\pgfqpoint{2.006454in}{0.524872in}}%
\pgfpathlineto{\pgfqpoint{2.040341in}{0.601849in}}%
\pgfpathlineto{\pgfqpoint{2.061013in}{0.682867in}}%
\pgfpathlineto{\pgfqpoint{2.067961in}{0.765929in}}%
\pgfusepath{stroke}%
\end{pgfscope}%
\begin{pgfscope}%
\pgfpathrectangle{\pgfqpoint{0.375000in}{0.150000in}}{\pgfqpoint{2.325000in}{1.200000in}} %
\pgfusepath{clip}%
\pgfsetbuttcap%
\pgfsetroundjoin%
\definecolor{currentfill}{rgb}{0.000000,0.000000,1.000000}%
\pgfsetfillcolor{currentfill}%
\pgfsetlinewidth{0.501875pt}%
\definecolor{currentstroke}{rgb}{0.000000,0.000000,0.000000}%
\pgfsetstrokecolor{currentstroke}%
\pgfsetdash{}{0pt}%
\pgfsys@defobject{currentmarker}{\pgfqpoint{-0.041667in}{-0.041667in}}{\pgfqpoint{0.041667in}{0.041667in}}{%
\pgfpathmoveto{\pgfqpoint{0.000000in}{-0.041667in}}%
\pgfpathcurveto{\pgfqpoint{0.011050in}{-0.041667in}}{\pgfqpoint{0.021649in}{-0.037276in}}{\pgfqpoint{0.029463in}{-0.029463in}}%
\pgfpathcurveto{\pgfqpoint{0.037276in}{-0.021649in}}{\pgfqpoint{0.041667in}{-0.011050in}}{\pgfqpoint{0.041667in}{0.000000in}}%
\pgfpathcurveto{\pgfqpoint{0.041667in}{0.011050in}}{\pgfqpoint{0.037276in}{0.021649in}}{\pgfqpoint{0.029463in}{0.029463in}}%
\pgfpathcurveto{\pgfqpoint{0.021649in}{0.037276in}}{\pgfqpoint{0.011050in}{0.041667in}}{\pgfqpoint{0.000000in}{0.041667in}}%
\pgfpathcurveto{\pgfqpoint{-0.011050in}{0.041667in}}{\pgfqpoint{-0.021649in}{0.037276in}}{\pgfqpoint{-0.029463in}{0.029463in}}%
\pgfpathcurveto{\pgfqpoint{-0.037276in}{0.021649in}}{\pgfqpoint{-0.041667in}{0.011050in}}{\pgfqpoint{-0.041667in}{0.000000in}}%
\pgfpathcurveto{\pgfqpoint{-0.041667in}{-0.011050in}}{\pgfqpoint{-0.037276in}{-0.021649in}}{\pgfqpoint{-0.029463in}{-0.029463in}}%
\pgfpathcurveto{\pgfqpoint{-0.021649in}{-0.037276in}}{\pgfqpoint{-0.011050in}{-0.041667in}}{\pgfqpoint{0.000000in}{-0.041667in}}%
\pgfpathclose%
\pgfusepath{stroke,fill}%
}%
\begin{pgfscope}%
\pgfsys@transformshift{1.503641in}{0.765929in}%
\pgfsys@useobject{currentmarker}{}%
\end{pgfscope}%
\end{pgfscope}%
\begin{pgfscope}%
\pgfpathrectangle{\pgfqpoint{0.375000in}{0.150000in}}{\pgfqpoint{2.325000in}{1.200000in}} %
\pgfusepath{clip}%
\pgfsetbuttcap%
\pgfsetroundjoin%
\definecolor{currentfill}{rgb}{0.000000,0.500000,0.000000}%
\pgfsetfillcolor{currentfill}%
\pgfsetlinewidth{0.501875pt}%
\definecolor{currentstroke}{rgb}{0.000000,0.000000,0.000000}%
\pgfsetstrokecolor{currentstroke}%
\pgfsetdash{}{0pt}%
\pgfsys@defobject{currentmarker}{\pgfqpoint{-0.041667in}{-0.041667in}}{\pgfqpoint{0.041667in}{0.041667in}}{%
\pgfpathmoveto{\pgfqpoint{0.000000in}{-0.041667in}}%
\pgfpathcurveto{\pgfqpoint{0.011050in}{-0.041667in}}{\pgfqpoint{0.021649in}{-0.037276in}}{\pgfqpoint{0.029463in}{-0.029463in}}%
\pgfpathcurveto{\pgfqpoint{0.037276in}{-0.021649in}}{\pgfqpoint{0.041667in}{-0.011050in}}{\pgfqpoint{0.041667in}{0.000000in}}%
\pgfpathcurveto{\pgfqpoint{0.041667in}{0.011050in}}{\pgfqpoint{0.037276in}{0.021649in}}{\pgfqpoint{0.029463in}{0.029463in}}%
\pgfpathcurveto{\pgfqpoint{0.021649in}{0.037276in}}{\pgfqpoint{0.011050in}{0.041667in}}{\pgfqpoint{0.000000in}{0.041667in}}%
\pgfpathcurveto{\pgfqpoint{-0.011050in}{0.041667in}}{\pgfqpoint{-0.021649in}{0.037276in}}{\pgfqpoint{-0.029463in}{0.029463in}}%
\pgfpathcurveto{\pgfqpoint{-0.037276in}{0.021649in}}{\pgfqpoint{-0.041667in}{0.011050in}}{\pgfqpoint{-0.041667in}{0.000000in}}%
\pgfpathcurveto{\pgfqpoint{-0.041667in}{-0.011050in}}{\pgfqpoint{-0.037276in}{-0.021649in}}{\pgfqpoint{-0.029463in}{-0.029463in}}%
\pgfpathcurveto{\pgfqpoint{-0.021649in}{-0.037276in}}{\pgfqpoint{-0.011050in}{-0.041667in}}{\pgfqpoint{0.000000in}{-0.041667in}}%
\pgfpathclose%
\pgfusepath{stroke,fill}%
}%
\begin{pgfscope}%
\pgfsys@transformshift{1.503641in}{0.234956in}%
\pgfsys@useobject{currentmarker}{}%
\end{pgfscope}%
\end{pgfscope}%
\begin{pgfscope}%
\pgftext[x=1.435922in,y=0.394248in,right,bottom]{{\sffamily\fontsize{12.000000}{14.400000}\selectfont \(\displaystyle R\)}}%
\end{pgfscope}%
\begin{pgfscope}%
\pgftext[x=1.661650in,y=1.020796in,left,bottom]{{\sffamily\fontsize{12.000000}{14.400000}\selectfont \(\displaystyle \alpha\)}}%
\end{pgfscope}%
\begin{pgfscope}%
\pgftext[x=1.503641in,y=0.765929in,right,bottom]{{\sffamily\fontsize{12.000000}{14.400000}\selectfont \(\displaystyle CM\)}}%
\end{pgfscope}%
\begin{pgfscope}%
\pgftext[x=2.180825in,y=0.234956in,left,bottom]{{\sffamily\fontsize{12.000000}{14.400000}\selectfont \(\displaystyle F_{fr}\)}}%
\end{pgfscope}%
\begin{pgfscope}%
\definecolor{textcolor}{rgb}{1.000000,0.000000,0.000000}%
\pgfsetstrokecolor{textcolor}%
\pgfsetfillcolor{textcolor}%
\pgftext[x=2.481796in,y=0.765929in,left,bottom]{{\sffamily\fontsize{12.000000}{14.400000}\selectfont \(\displaystyle a\)}}%
\end{pgfscope}%
\begin{pgfscope}%
\pgftext[x=2.632282in,y=1.190708in,left,top]{{\sffamily\fontsize{12.000000}{14.400000}\selectfont \(\displaystyle F_p\)}}%
\end{pgfscope}%
\end{pgfpicture}%
\makeatother%
\endgroup%


from 2nd law%
\begin{eqnarray*}
&&\left\{ 
\begin{array}{c}
F_{p}+F_{fr}=ma \\ 
\left( F_{p}-F_{fr}\right) R=I\alpha =I\frac{a}{R}%
\end{array}%
\right. \\
&\Rightarrow &F_{fr}=\frac{1}{2}\left( m-\frac{I}{R^{2}}\right) a
\end{eqnarray*}

So if $mR^{2}>I$, then $F_{fr}$ goes to the right!

\bigskip

If frction acting on the wheel goes to the left,

\input{../../../../Scripts/cordtrans/cases_fig_only/case5b.pgf}

\begin{eqnarray*}
&&\left\{ 
\begin{array}{c}
F_{p}-F_{fr}=ma \\ 
\left( F_{p}+F_{fr}\right) R=I\alpha =I\frac{a}{R}%
\end{array}%
\right. \\
&\Rightarrow &F_{fr}=\frac{1}{2}\left( \frac{I}{R^{2}}-m\right) a
\end{eqnarray*}

So if $mR^{2}<I$, then $F_{fr}$ goes to the left!

\bigskip

%TCIMACRO{\TeXButton{end2columns}{\end{multicols}}}%
%BeginExpansion
\end{multicols}%
%EndExpansion

\bigskip

In this case the direction of friction depends on the condition $mR^{2}>I$
or $mR^{2}<I$. We can also recgonize in the first senario if $mR^{2}<I$ then 
$F_{fr}$ is negative, so the friction should be opposite to the positive
direction we assume (to the right). This way we can save the time of working
on the 2nd case.

\end{document}
