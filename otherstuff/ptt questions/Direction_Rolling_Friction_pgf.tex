
\documentclass{article}
%%%%%%%%%%%%%%%%%%%%%%%%%%%%%%%%%%%%%%%%%%%%%%%%%%%%%%%%%%%%%%%%%%%%%%%%%%%%%%%%%%%%%%%%%%%%%%%%%%%%%%%%%%%%%%%%%%%%%%%%%%%%%%%%%%%%%%%%%%%%%%%%%%%%%%%%%%%%%%%%%%%%%%%%%%%%%%%%%%%%%%%%%%%%%%%%%%%%%%%%%%%%%%%%%%%%%%%%%%%%%%%%%%%%%%%%%%%%%%%%%%%%%%%%%%%%
\usepackage[ignoreall]{geometry}
\usepackage{graphicx}
\usepackage{pgf}
\usepackage{multicol}

%TCIDATA{OutputFilter=LATEX.DLL}
%TCIDATA{Version=5.00.0.2606}
%TCIDATA{<META NAME="SaveForMode" CONTENT="1">}
%TCIDATA{BibliographyScheme=Manual}
%TCIDATA{Created=Wednesday, April 06, 2016 11:13:15}
%TCIDATA{LastRevised=Monday, June 20, 2016 11:47:14}
%TCIDATA{<META NAME="GraphicsSave" CONTENT="32">}
%TCIDATA{<META NAME="DocumentShell" CONTENT="Standard LaTeX\Blank - Standard LaTeX Article">}
%TCIDATA{CSTFile=40 LaTeX article.cst}

\newtheorem{theorem}{Theorem}
\newtheorem{acknowledgement}[theorem]{Acknowledgement}
\newtheorem{algorithm}[theorem]{Algorithm}
\newtheorem{axiom}[theorem]{Axiom}
\newtheorem{case}[theorem]{Case}
\newtheorem{claim}[theorem]{Claim}
\newtheorem{conclusion}[theorem]{Conclusion}
\newtheorem{condition}[theorem]{Condition}
\newtheorem{conjecture}[theorem]{Conjecture}
\newtheorem{corollary}[theorem]{Corollary}
\newtheorem{criterion}[theorem]{Criterion}
\newtheorem{definition}[theorem]{Definition}
\newtheorem{example}[theorem]{Example}
\newtheorem{exercise}[theorem]{Exercise}
\newtheorem{lemma}[theorem]{Lemma}
\newtheorem{notation}[theorem]{Notation}
\newtheorem{problem}[theorem]{Problem}
\newtheorem{proposition}[theorem]{Proposition}
\newtheorem{remark}[theorem]{Remark}
\newtheorem{solution}[theorem]{Solution}
\newtheorem{summary}[theorem]{Summary}
\newenvironment{proof}[1][Proof]{\noindent\textbf{#1.} }{\ \rule{0.5em}{0.5em}}
% Macros for Scientific Word 4.0 documents saved with the LaTeX filter.
% Copyright (C) 2002 Mackichan Software, Inc.

\typeout{TCILATEX Macros for Scientific Word 5.0 <13 Feb 2003>.}
\typeout{NOTICE:  This macro file is NOT proprietary and may be 
freely copied and distributed.}
%
\makeatletter

%%%%%%%%%%%%%%%%%%%%%
% pdfTeX related.
\ifx\pdfoutput\relax\let\pdfoutput=\undefined\fi
\newcount\msipdfoutput
\ifx\pdfoutput\undefined
\else
 \ifcase\pdfoutput
 \else 
    \msipdfoutput=1
    \ifx\paperwidth\undefined
    \else
      \ifdim\paperheight=0pt\relax
      \else
        \pdfpageheight\paperheight
      \fi
      \ifdim\paperwidth=0pt\relax
      \else
        \pdfpagewidth\paperwidth
      \fi
    \fi
  \fi  
\fi

%%%%%%%%%%%%%%%%%%%%%
% FMTeXButton
% This is used for putting TeXButtons in the 
% frontmatter of a document. Add a line like
% \QTagDef{FMTeXButton}{101}{} to the filter 
% section of the cst being used. Also add a
% new section containing:
%     [f_101]
%     ALIAS=FMTexButton
%     TAG_TYPE=FIELD
%     TAG_LEADIN=TeX Button:
%
% It also works to put \defs in the preamble after 
% the \input tcilatex
\def\FMTeXButton#1{#1}
%
%%%%%%%%%%%%%%%%%%%%%%
% macros for time
\newcount\@hour\newcount\@minute\chardef\@x10\chardef\@xv60
\def\tcitime{
\def\@time{%
  \@minute\time\@hour\@minute\divide\@hour\@xv
  \ifnum\@hour<\@x 0\fi\the\@hour:%
  \multiply\@hour\@xv\advance\@minute-\@hour
  \ifnum\@minute<\@x 0\fi\the\@minute
  }}%

%%%%%%%%%%%%%%%%%%%%%%
% macro for hyperref and msihyperref
%\@ifundefined{hyperref}{\def\hyperref#1#2#3#4{#2\ref{#4}#3}}{}

\def\x@hyperref#1#2#3{%
   % Turn off various catcodes before reading parameter 4
   \catcode`\~ = 12
   \catcode`\$ = 12
   \catcode`\_ = 12
   \catcode`\# = 12
   \catcode`\& = 12
   \y@hyperref{#1}{#2}{#3}%
}

\def\y@hyperref#1#2#3#4{%
   #2\ref{#4}#3
   \catcode`\~ = 13
   \catcode`\$ = 3
   \catcode`\_ = 8
   \catcode`\# = 6
   \catcode`\& = 4
}

\@ifundefined{hyperref}{\let\hyperref\x@hyperref}{}
\@ifundefined{msihyperref}{\let\msihyperref\x@hyperref}{}




% macro for external program call
\@ifundefined{qExtProgCall}{\def\qExtProgCall#1#2#3#4#5#6{\relax}}{}
%%%%%%%%%%%%%%%%%%%%%%
%
% macros for graphics
%
\def\FILENAME#1{#1}%
%
\def\QCTOpt[#1]#2{%
  \def\QCTOptB{#1}
  \def\QCTOptA{#2}
}
\def\QCTNOpt#1{%
  \def\QCTOptA{#1}
  \let\QCTOptB\empty
}
\def\Qct{%
  \@ifnextchar[{%
    \QCTOpt}{\QCTNOpt}
}
\def\QCBOpt[#1]#2{%
  \def\QCBOptB{#1}%
  \def\QCBOptA{#2}%
}
\def\QCBNOpt#1{%
  \def\QCBOptA{#1}%
  \let\QCBOptB\empty
}
\def\Qcb{%
  \@ifnextchar[{%
    \QCBOpt}{\QCBNOpt}%
}
\def\PrepCapArgs{%
  \ifx\QCBOptA\empty
    \ifx\QCTOptA\empty
      {}%
    \else
      \ifx\QCTOptB\empty
        {\QCTOptA}%
      \else
        [\QCTOptB]{\QCTOptA}%
      \fi
    \fi
  \else
    \ifx\QCBOptA\empty
      {}%
    \else
      \ifx\QCBOptB\empty
        {\QCBOptA}%
      \else
        [\QCBOptB]{\QCBOptA}%
      \fi
    \fi
  \fi
}
\newcount\GRAPHICSTYPE
%\GRAPHICSTYPE 0 is for TurboTeX
%\GRAPHICSTYPE 1 is for DVIWindo (PostScript)
%%%(removed)%\GRAPHICSTYPE 2 is for psfig (PostScript)
\GRAPHICSTYPE=\z@
\def\GRAPHICSPS#1{%
 \ifcase\GRAPHICSTYPE%\GRAPHICSTYPE=0
   \special{ps: #1}%
 \or%\GRAPHICSTYPE=1
   \special{language "PS", include "#1"}%
%%%\or%\GRAPHICSTYPE=2
%%%  #1%
 \fi
}%
%
\def\GRAPHICSHP#1{\special{include #1}}%
%
% \graffile{ body }                                  %#1
%          { contentswidth (scalar)  }               %#2
%          { contentsheight (scalar) }               %#3
%          { vertical shift when in-line (scalar) }  %#4

\def\graffile#1#2#3#4{%
%%% \ifnum\GRAPHICSTYPE=\tw@
%%%  %Following if using psfig
%%%  \@ifundefined{psfig}{\input psfig.tex}{}%
%%%  \psfig{file=#1, height=#3, width=#2}%
%%% \else
  %Following for all others
  % JCS - added BOXTHEFRAME, see below
    \bgroup
	   \@inlabelfalse
       \leavevmode
       \@ifundefined{bbl@deactivate}{\def~{\string~}}{\activesoff}%
        \raise -#4 \BOXTHEFRAME{%
           \hbox to #2{\raise #3\hbox to #2{\null #1\hfil}}}%
    \egroup
}%
%
% A box for drafts
\def\draftbox#1#2#3#4{%
 \leavevmode\raise -#4 \hbox{%
  \frame{\rlap{\protect\tiny #1}\hbox to #2%
   {\vrule height#3 width\z@ depth\z@\hfil}%
  }%
 }%
}%
%
\newcount\@msidraft
\@msidraft=\z@
\let\nographics=\@msidraft
\newif\ifwasdraft
\wasdraftfalse

%  \GRAPHIC{ body }                                  %#1
%          { draft name }                            %#2
%          { contentswidth (scalar)  }               %#3
%          { contentsheight (scalar) }               %#4
%          { vertical shift when in-line (scalar) }  %#5
\def\GRAPHIC#1#2#3#4#5{%
   \ifnum\@msidraft=\@ne\draftbox{#2}{#3}{#4}{#5}%
   \else\graffile{#1}{#3}{#4}{#5}%
   \fi
}
%
\def\addtoLaTeXparams#1{%
    \edef\LaTeXparams{\LaTeXparams #1}}%
%
% JCS -  added a switch BoxFrame that can 
% be set by including X in the frame params.
% If set a box is drawn around the frame.

\newif\ifBoxFrame \BoxFramefalse
\newif\ifOverFrame \OverFramefalse
\newif\ifUnderFrame \UnderFramefalse

\def\BOXTHEFRAME#1{%
   \hbox{%
      \ifBoxFrame
         \frame{#1}%
      \else
         {#1}%
      \fi
   }%
}


\def\doFRAMEparams#1{\BoxFramefalse\OverFramefalse\UnderFramefalse\readFRAMEparams#1\end}%
\def\readFRAMEparams#1{%
 \ifx#1\end%
  \let\next=\relax
  \else
  \ifx#1i\dispkind=\z@\fi
  \ifx#1d\dispkind=\@ne\fi
  \ifx#1f\dispkind=\tw@\fi
  \ifx#1t\addtoLaTeXparams{t}\fi
  \ifx#1b\addtoLaTeXparams{b}\fi
  \ifx#1p\addtoLaTeXparams{p}\fi
  \ifx#1h\addtoLaTeXparams{h}\fi
  \ifx#1X\BoxFrametrue\fi
  \ifx#1O\OverFrametrue\fi
  \ifx#1U\UnderFrametrue\fi
  \ifx#1w
    \ifnum\@msidraft=1\wasdrafttrue\else\wasdraftfalse\fi
    \@msidraft=\@ne
  \fi
  \let\next=\readFRAMEparams
  \fi
 \next
 }%
%
%Macro for In-line graphics object
%   \IFRAME{ contentswidth (scalar)  }               %#1
%          { contentsheight (scalar) }               %#2
%          { vertical shift when in-line (scalar) }  %#3
%          { draft name }                            %#4
%          { body }                                  %#5
%          { caption}                                %#6


\def\IFRAME#1#2#3#4#5#6{%
      \bgroup
      \let\QCTOptA\empty
      \let\QCTOptB\empty
      \let\QCBOptA\empty
      \let\QCBOptB\empty
      #6%
      \parindent=0pt
      \leftskip=0pt
      \rightskip=0pt
      \setbox0=\hbox{\QCBOptA}%
      \@tempdima=#1\relax
      \ifOverFrame
          % Do this later
          \typeout{This is not implemented yet}%
          \show\HELP
      \else
         \ifdim\wd0>\@tempdima
            \advance\@tempdima by \@tempdima
            \ifdim\wd0 >\@tempdima
               \setbox1 =\vbox{%
                  \unskip\hbox to \@tempdima{\hfill\GRAPHIC{#5}{#4}{#1}{#2}{#3}\hfill}%
                  \unskip\hbox to \@tempdima{\parbox[b]{\@tempdima}{\QCBOptA}}%
               }%
               \wd1=\@tempdima
            \else
               \textwidth=\wd0
               \setbox1 =\vbox{%
                 \noindent\hbox to \wd0{\hfill\GRAPHIC{#5}{#4}{#1}{#2}{#3}\hfill}\\%
                 \noindent\hbox{\QCBOptA}%
               }%
               \wd1=\wd0
            \fi
         \else
            \ifdim\wd0>0pt
              \hsize=\@tempdima
              \setbox1=\vbox{%
                \unskip\GRAPHIC{#5}{#4}{#1}{#2}{0pt}%
                \break
                \unskip\hbox to \@tempdima{\hfill \QCBOptA\hfill}%
              }%
              \wd1=\@tempdima
           \else
              \hsize=\@tempdima
              \setbox1=\vbox{%
                \unskip\GRAPHIC{#5}{#4}{#1}{#2}{0pt}%
              }%
              \wd1=\@tempdima
           \fi
         \fi
         \@tempdimb=\ht1
         %\advance\@tempdimb by \dp1
         \advance\@tempdimb by -#2
         \advance\@tempdimb by #3
         \leavevmode
         \raise -\@tempdimb \hbox{\box1}%
      \fi
      \egroup%
}%
%
%Macro for Display graphics object
%   \DFRAME{ contentswidth (scalar)  }               %#1
%          { contentsheight (scalar) }               %#2
%          { draft label }                           %#3
%          { name }                                  %#4
%          { caption}                                %#5
\def\DFRAME#1#2#3#4#5{%
  \vspace\topsep
  \hfil\break
  \bgroup
     \leftskip\@flushglue
	 \rightskip\@flushglue
	 \parindent\z@
	 \parfillskip\z@skip
     \let\QCTOptA\empty
     \let\QCTOptB\empty
     \let\QCBOptA\empty
     \let\QCBOptB\empty
	 \vbox\bgroup
        \ifOverFrame 
           #5\QCTOptA\par
        \fi
        \GRAPHIC{#4}{#3}{#1}{#2}{\z@}%
        \ifUnderFrame 
           \break#5\QCBOptA
        \fi
	 \egroup
  \egroup
  \vspace\topsep
  \break
}%
%
%Macro for Floating graphic object
%   \FFRAME{ framedata f|i tbph x F|T }              %#1
%          { contentswidth (scalar)  }               %#2
%          { contentsheight (scalar) }               %#3
%          { caption }                               %#4
%          { label }                                 %#5
%          { draft name }                            %#6
%          { body }                                  %#7
\def\FFRAME#1#2#3#4#5#6#7{%
 %If float.sty loaded and float option is 'h', change to 'H'  (gp) 1998/09/05
  \@ifundefined{floatstyle}
    {%floatstyle undefined (and float.sty not present), no change
     \begin{figure}[#1]%
    }
    {%floatstyle DEFINED
	 \ifx#1h%Only the h parameter, change to H
      \begin{figure}[H]%
	 \else
      \begin{figure}[#1]%
	 \fi
	}
  \let\QCTOptA\empty
  \let\QCTOptB\empty
  \let\QCBOptA\empty
  \let\QCBOptB\empty
  \ifOverFrame
    #4
    \ifx\QCTOptA\empty
    \else
      \ifx\QCTOptB\empty
        \caption{\QCTOptA}%
      \else
        \caption[\QCTOptB]{\QCTOptA}%
      \fi
    \fi
    \ifUnderFrame\else
      \label{#5}%
    \fi
  \else
    \UnderFrametrue%
  \fi
  \begin{center}\GRAPHIC{#7}{#6}{#2}{#3}{\z@}\end{center}%
  \ifUnderFrame
    #4
    \ifx\QCBOptA\empty
      \caption{}%
    \else
      \ifx\QCBOptB\empty
        \caption{\QCBOptA}%
      \else
        \caption[\QCBOptB]{\QCBOptA}%
      \fi
    \fi
    \label{#5}%
  \fi
  \end{figure}%
 }%
%
%
%    \FRAME{ framedata f|i tbph x F|T }              %#1
%          { contentswidth (scalar)  }               %#2
%          { contentsheight (scalar) }               %#3
%          { vertical shift when in-line (scalar) }  %#4
%          { caption }                               %#5
%          { label }                                 %#6
%          { name }                                  %#7
%          { body }                                  %#8
%
%    framedata is a string which can contain the following
%    characters: idftbphxFT
%    Their meaning is as follows:
%             i, d or f : in-line, display, or floating
%             t,b,p,h   : LaTeX floating placement options
%             x         : fit contents box to contents
%             F or T    : Figure or Table. 
%                         Later this can expand
%                         to a more general float class.
%
%
\newcount\dispkind%

\def\makeactives{
  \catcode`\"=\active
  \catcode`\;=\active
  \catcode`\:=\active
  \catcode`\'=\active
  \catcode`\~=\active
}
\bgroup
   \makeactives
   \gdef\activesoff{%
      \def"{\string"}%
      \def;{\string;}%
      \def:{\string:}%
      \def'{\string'}%
      \def~{\string~}%
      %\bbl@deactivate{"}%
      %\bbl@deactivate{;}%
      %\bbl@deactivate{:}%
      %\bbl@deactivate{'}%
    }
\egroup

\def\FRAME#1#2#3#4#5#6#7#8{%
 \bgroup
 \ifnum\@msidraft=\@ne
   \wasdrafttrue
 \else
   \wasdraftfalse%
 \fi
 \def\LaTeXparams{}%
 \dispkind=\z@
 \def\LaTeXparams{}%
 \doFRAMEparams{#1}%
 \ifnum\dispkind=\z@\IFRAME{#2}{#3}{#4}{#7}{#8}{#5}\else
  \ifnum\dispkind=\@ne\DFRAME{#2}{#3}{#7}{#8}{#5}\else
   \ifnum\dispkind=\tw@
    \edef\@tempa{\noexpand\FFRAME{\LaTeXparams}}%
    \@tempa{#2}{#3}{#5}{#6}{#7}{#8}%
    \fi
   \fi
  \fi
  \ifwasdraft\@msidraft=1\else\@msidraft=0\fi{}%
  \egroup
 }%
%
% This macro added to let SW gobble a parameter that
% should not be passed on and expanded. 

\def\TEXUX#1{"texux"}

%
% Macros for text attributes:
%
\def\BF#1{{\bf {#1}}}%
\def\NEG#1{\leavevmode\hbox{\rlap{\thinspace/}{$#1$}}}%
%
%%%%%%%%%%%%%%%%%%%%%%%%%%%%%%%%%%%%%%%%%%%%%%%%%%%%%%%%%%%%%%%%%%%%%%%%
%
%
% macros for user - defined functions
\def\limfunc#1{\mathop{\rm #1}}%
\def\func#1{\mathop{\rm #1}\nolimits}%
% macro for unit names
\def\unit#1{\mathord{\thinspace\rm #1}}%

%
% miscellaneous 
\long\def\QQQ#1#2{%
     \long\expandafter\def\csname#1\endcsname{#2}}%
\@ifundefined{QTP}{\def\QTP#1{}}{}
\@ifundefined{QEXCLUDE}{\def\QEXCLUDE#1{}}{}
\@ifundefined{Qlb}{\def\Qlb#1{#1}}{}
\@ifundefined{Qlt}{\def\Qlt#1{#1}}{}
\def\QWE{}%
\long\def\QQA#1#2{}%
\def\QTR#1#2{{\csname#1\endcsname {#2}}}%
\long\def\TeXButton#1#2{#2}%
\long\def\QSubDoc#1#2{#2}%
\def\EXPAND#1[#2]#3{}%
\def\NOEXPAND#1[#2]#3{}%
\def\PROTECTED{}%
\def\LaTeXparent#1{}%
\def\ChildStyles#1{}%
\def\ChildDefaults#1{}%
\def\QTagDef#1#2#3{}%

% Constructs added with Scientific Notebook
\@ifundefined{correctchoice}{\def\correctchoice{\relax}}{}
\@ifundefined{HTML}{\def\HTML#1{\relax}}{}
\@ifundefined{TCIIcon}{\def\TCIIcon#1#2#3#4{\relax}}{}
\if@compatibility
  \typeout{Not defining UNICODE  U or CustomNote commands for LaTeX 2.09.}
\else
  \providecommand{\UNICODE}[2][]{\protect\rule{.1in}{.1in}}
  \providecommand{\U}[1]{\protect\rule{.1in}{.1in}}
  \providecommand{\CustomNote}[3][]{\marginpar{#3}}
\fi

\@ifundefined{lambdabar}{
      \def\lambdabar{\errmessage{You have used the lambdabar symbol. 
                      This is available for typesetting only in RevTeX styles.}}
   }{}

%
% Macros for style editor docs
\@ifundefined{StyleEditBeginDoc}{\def\StyleEditBeginDoc{\relax}}{}
%
% Macros for footnotes
\def\QQfnmark#1{\footnotemark}
\def\QQfntext#1#2{\addtocounter{footnote}{#1}\footnotetext{#2}}
%
% Macros for indexing.
%
\@ifundefined{TCIMAKEINDEX}{}{\makeindex}%
%
% Attempts to avoid problems with other styles
\@ifundefined{abstract}{%
 \def\abstract{%
  \if@twocolumn
   \section*{Abstract (Not appropriate in this style!)}%
   \else \small 
   \begin{center}{\bf Abstract\vspace{-.5em}\vspace{\z@}}\end{center}%
   \quotation 
   \fi
  }%
 }{%
 }%
\@ifundefined{endabstract}{\def\endabstract
  {\if@twocolumn\else\endquotation\fi}}{}%
\@ifundefined{maketitle}{\def\maketitle#1{}}{}%
\@ifundefined{affiliation}{\def\affiliation#1{}}{}%
\@ifundefined{proof}{\def\proof{\noindent{\bfseries Proof. }}}{}%
\@ifundefined{endproof}{\def\endproof{\mbox{\ \rule{.1in}{.1in}}}}{}%
\@ifundefined{newfield}{\def\newfield#1#2{}}{}%
\@ifundefined{chapter}{\def\chapter#1{\par(Chapter head:)#1\par }%
 \newcount\c@chapter}{}%
\@ifundefined{part}{\def\part#1{\par(Part head:)#1\par }}{}%
\@ifundefined{section}{\def\section#1{\par(Section head:)#1\par }}{}%
\@ifundefined{subsection}{\def\subsection#1%
 {\par(Subsection head:)#1\par }}{}%
\@ifundefined{subsubsection}{\def\subsubsection#1%
 {\par(Subsubsection head:)#1\par }}{}%
\@ifundefined{paragraph}{\def\paragraph#1%
 {\par(Subsubsubsection head:)#1\par }}{}%
\@ifundefined{subparagraph}{\def\subparagraph#1%
 {\par(Subsubsubsubsection head:)#1\par }}{}%
%%%%%%%%%%%%%%%%%%%%%%%%%%%%%%%%%%%%%%%%%%%%%%%%%%%%%%%%%%%%%%%%%%%%%%%%
% These symbols are not recognized by LaTeX
\@ifundefined{therefore}{\def\therefore{}}{}%
\@ifundefined{backepsilon}{\def\backepsilon{}}{}%
\@ifundefined{yen}{\def\yen{\hbox{\rm\rlap=Y}}}{}%
\@ifundefined{registered}{%
   \def\registered{\relax\ifmmode{}\r@gistered
                    \else$\m@th\r@gistered$\fi}%
 \def\r@gistered{^{\ooalign
  {\hfil\raise.07ex\hbox{$\scriptstyle\rm\text{R}$}\hfil\crcr
  \mathhexbox20D}}}}{}%
\@ifundefined{Eth}{\def\Eth{}}{}%
\@ifundefined{eth}{\def\eth{}}{}%
\@ifundefined{Thorn}{\def\Thorn{}}{}%
\@ifundefined{thorn}{\def\thorn{}}{}%
% A macro to allow any symbol that requires math to appear in text
\def\TEXTsymbol#1{\mbox{$#1$}}%
\@ifundefined{degree}{\def\degree{{}^{\circ}}}{}%
%
% macros for T3TeX files
\newdimen\theight
\@ifundefined{Column}{\def\Column{%
 \vadjust{\setbox\z@=\hbox{\scriptsize\quad\quad tcol}%
  \theight=\ht\z@\advance\theight by \dp\z@\advance\theight by \lineskip
  \kern -\theight \vbox to \theight{%
   \rightline{\rlap{\box\z@}}%
   \vss
   }%
  }%
 }}{}%
%
\@ifundefined{qed}{\def\qed{%
 \ifhmode\unskip\nobreak\fi\ifmmode\ifinner\else\hskip5\p@\fi\fi
 \hbox{\hskip5\p@\vrule width4\p@ height6\p@ depth1.5\p@\hskip\p@}%
 }}{}%
%
\@ifundefined{cents}{\def\cents{\hbox{\rm\rlap c/}}}{}%
\@ifundefined{tciLaplace}{\def\tciLaplace{\ensuremath{\mathcal{L}}}}{}%
\@ifundefined{tciFourier}{\def\tciFourier{\ensuremath{\mathcal{F}}}}{}%
\@ifundefined{textcurrency}{\def\textcurrency{\hbox{\rm\rlap xo}}}{}%
\@ifundefined{texteuro}{\def\texteuro{\hbox{\rm\rlap C=}}}{}%
\@ifundefined{euro}{\def\euro{\hbox{\rm\rlap C=}}}{}%
\@ifundefined{textfranc}{\def\textfranc{\hbox{\rm\rlap-F}}}{}%
\@ifundefined{textlira}{\def\textlira{\hbox{\rm\rlap L=}}}{}%
\@ifundefined{textpeseta}{\def\textpeseta{\hbox{\rm P\negthinspace s}}}{}%
%
\@ifundefined{miss}{\def\miss{\hbox{\vrule height2\p@ width 2\p@ depth\z@}}}{}%
%
\@ifundefined{vvert}{\def\vvert{\Vert}}{}%  %always translated to \left| or \right|
%
\@ifundefined{tcol}{\def\tcol#1{{\baselineskip=6\p@ \vcenter{#1}} \Column}}{}%
%
\@ifundefined{dB}{\def\dB{\hbox{{}}}}{}%        %dummy entry in column 
\@ifundefined{mB}{\def\mB#1{\hbox{$#1$}}}{}%   %column entry
\@ifundefined{nB}{\def\nB#1{\hbox{#1}}}{}%     %column entry (not math)
%
\@ifundefined{note}{\def\note{$^{\dag}}}{}%
%
\def\newfmtname{LaTeX2e}
% No longer load latexsym.  This is now handled by SWP, which uses amsfonts if necessary
%
\ifx\fmtname\newfmtname
  \DeclareOldFontCommand{\rm}{\normalfont\rmfamily}{\mathrm}
  \DeclareOldFontCommand{\sf}{\normalfont\sffamily}{\mathsf}
  \DeclareOldFontCommand{\tt}{\normalfont\ttfamily}{\mathtt}
  \DeclareOldFontCommand{\bf}{\normalfont\bfseries}{\mathbf}
  \DeclareOldFontCommand{\it}{\normalfont\itshape}{\mathit}
  \DeclareOldFontCommand{\sl}{\normalfont\slshape}{\@nomath\sl}
  \DeclareOldFontCommand{\sc}{\normalfont\scshape}{\@nomath\sc}
\fi

%
% Greek bold macros
% Redefine all of the math symbols 
% which might be bolded	 - there are 
% probably others to add to this list

\def\alpha{{\Greekmath 010B}}%
\def\beta{{\Greekmath 010C}}%
\def\gamma{{\Greekmath 010D}}%
\def\delta{{\Greekmath 010E}}%
\def\epsilon{{\Greekmath 010F}}%
\def\zeta{{\Greekmath 0110}}%
\def\eta{{\Greekmath 0111}}%
\def\theta{{\Greekmath 0112}}%
\def\iota{{\Greekmath 0113}}%
\def\kappa{{\Greekmath 0114}}%
\def\lambda{{\Greekmath 0115}}%
\def\mu{{\Greekmath 0116}}%
\def\nu{{\Greekmath 0117}}%
\def\xi{{\Greekmath 0118}}%
\def\pi{{\Greekmath 0119}}%
\def\rho{{\Greekmath 011A}}%
\def\sigma{{\Greekmath 011B}}%
\def\tau{{\Greekmath 011C}}%
\def\upsilon{{\Greekmath 011D}}%
\def\phi{{\Greekmath 011E}}%
\def\chi{{\Greekmath 011F}}%
\def\psi{{\Greekmath 0120}}%
\def\omega{{\Greekmath 0121}}%
\def\varepsilon{{\Greekmath 0122}}%
\def\vartheta{{\Greekmath 0123}}%
\def\varpi{{\Greekmath 0124}}%
\def\varrho{{\Greekmath 0125}}%
\def\varsigma{{\Greekmath 0126}}%
\def\varphi{{\Greekmath 0127}}%

\def\nabla{{\Greekmath 0272}}
\def\FindBoldGroup{%
   {\setbox0=\hbox{$\mathbf{x\global\edef\theboldgroup{\the\mathgroup}}$}}%
}

\def\Greekmath#1#2#3#4{%
    \if@compatibility
        \ifnum\mathgroup=\symbold
           \mathchoice{\mbox{\boldmath$\displaystyle\mathchar"#1#2#3#4$}}%
                      {\mbox{\boldmath$\textstyle\mathchar"#1#2#3#4$}}%
                      {\mbox{\boldmath$\scriptstyle\mathchar"#1#2#3#4$}}%
                      {\mbox{\boldmath$\scriptscriptstyle\mathchar"#1#2#3#4$}}%
        \else
           \mathchar"#1#2#3#4% 
        \fi 
    \else 
        \FindBoldGroup
        \ifnum\mathgroup=\theboldgroup % For 2e
           \mathchoice{\mbox{\boldmath$\displaystyle\mathchar"#1#2#3#4$}}%
                      {\mbox{\boldmath$\textstyle\mathchar"#1#2#3#4$}}%
                      {\mbox{\boldmath$\scriptstyle\mathchar"#1#2#3#4$}}%
                      {\mbox{\boldmath$\scriptscriptstyle\mathchar"#1#2#3#4$}}%
        \else
           \mathchar"#1#2#3#4% 
        \fi     	    
	  \fi}

\newif\ifGreekBold  \GreekBoldfalse
\let\SAVEPBF=\pbf
\def\pbf{\GreekBoldtrue\SAVEPBF}%
%

\@ifundefined{theorem}{\newtheorem{theorem}{Theorem}}{}
\@ifundefined{lemma}{\newtheorem{lemma}[theorem]{Lemma}}{}
\@ifundefined{corollary}{\newtheorem{corollary}[theorem]{Corollary}}{}
\@ifundefined{conjecture}{\newtheorem{conjecture}[theorem]{Conjecture}}{}
\@ifundefined{proposition}{\newtheorem{proposition}[theorem]{Proposition}}{}
\@ifundefined{axiom}{\newtheorem{axiom}{Axiom}}{}
\@ifundefined{remark}{\newtheorem{remark}{Remark}}{}
\@ifundefined{example}{\newtheorem{example}{Example}}{}
\@ifundefined{exercise}{\newtheorem{exercise}{Exercise}}{}
\@ifundefined{definition}{\newtheorem{definition}{Definition}}{}


\@ifundefined{mathletters}{%
  %\def\theequation{\arabic{equation}}
  \newcounter{equationnumber}  
  \def\mathletters{%
     \addtocounter{equation}{1}
     \edef\@currentlabel{\theequation}%
     \setcounter{equationnumber}{\c@equation}
     \setcounter{equation}{0}%
     \edef\theequation{\@currentlabel\noexpand\alph{equation}}%
  }
  \def\endmathletters{%
     \setcounter{equation}{\value{equationnumber}}%
  }
}{}

%Logos
\@ifundefined{BibTeX}{%
    \def\BibTeX{{\rm B\kern-.05em{\sc i\kern-.025em b}\kern-.08em
                 T\kern-.1667em\lower.7ex\hbox{E}\kern-.125emX}}}{}%
\@ifundefined{AmS}%
    {\def\AmS{{\protect\usefont{OMS}{cmsy}{m}{n}%
                A\kern-.1667em\lower.5ex\hbox{M}\kern-.125emS}}}{}%
\@ifundefined{AmSTeX}{\def\AmSTeX{\protect\AmS-\protect\TeX\@}}{}%
%

% This macro is a fix to eqnarray
\def\@@eqncr{\let\@tempa\relax
    \ifcase\@eqcnt \def\@tempa{& & &}\or \def\@tempa{& &}%
      \else \def\@tempa{&}\fi
     \@tempa
     \if@eqnsw
        \iftag@
           \@taggnum
        \else
           \@eqnnum\stepcounter{equation}%
        \fi
     \fi
     \global\tag@false
     \global\@eqnswtrue
     \global\@eqcnt\z@\cr}


\def\TCItag{\@ifnextchar*{\@TCItagstar}{\@TCItag}}
\def\@TCItag#1{%
    \global\tag@true
    \global\def\@taggnum{(#1)}%
    \global\def\@currentlabel{#1}}
\def\@TCItagstar*#1{%
    \global\tag@true
    \global\def\@taggnum{#1}%
    \global\def\@currentlabel{#1}}
%
%%%%%%%%%%%%%%%%%%%%%%%%%%%%%%%%%%%%%%%%%%%%%%%%%%%%%%%%%%%%%%%%%%%%%
%
\def\QATOP#1#2{{#1 \atop #2}}%
\def\QTATOP#1#2{{\textstyle {#1 \atop #2}}}%
\def\QDATOP#1#2{{\displaystyle {#1 \atop #2}}}%
\def\QABOVE#1#2#3{{#2 \above#1 #3}}%
\def\QTABOVE#1#2#3{{\textstyle {#2 \above#1 #3}}}%
\def\QDABOVE#1#2#3{{\displaystyle {#2 \above#1 #3}}}%
\def\QOVERD#1#2#3#4{{#3 \overwithdelims#1#2 #4}}%
\def\QTOVERD#1#2#3#4{{\textstyle {#3 \overwithdelims#1#2 #4}}}%
\def\QDOVERD#1#2#3#4{{\displaystyle {#3 \overwithdelims#1#2 #4}}}%
\def\QATOPD#1#2#3#4{{#3 \atopwithdelims#1#2 #4}}%
\def\QTATOPD#1#2#3#4{{\textstyle {#3 \atopwithdelims#1#2 #4}}}%
\def\QDATOPD#1#2#3#4{{\displaystyle {#3 \atopwithdelims#1#2 #4}}}%
\def\QABOVED#1#2#3#4#5{{#4 \abovewithdelims#1#2#3 #5}}%
\def\QTABOVED#1#2#3#4#5{{\textstyle 
   {#4 \abovewithdelims#1#2#3 #5}}}%
\def\QDABOVED#1#2#3#4#5{{\displaystyle 
   {#4 \abovewithdelims#1#2#3 #5}}}%
%
% Macros for text size operators:
%
\def\tint{\mathop{\textstyle \int}}%
\def\tiint{\mathop{\textstyle \iint }}%
\def\tiiint{\mathop{\textstyle \iiint }}%
\def\tiiiint{\mathop{\textstyle \iiiint }}%
\def\tidotsint{\mathop{\textstyle \idotsint }}%
\def\toint{\mathop{\textstyle \oint}}%
\def\tsum{\mathop{\textstyle \sum }}%
\def\tprod{\mathop{\textstyle \prod }}%
\def\tbigcap{\mathop{\textstyle \bigcap }}%
\def\tbigwedge{\mathop{\textstyle \bigwedge }}%
\def\tbigoplus{\mathop{\textstyle \bigoplus }}%
\def\tbigodot{\mathop{\textstyle \bigodot }}%
\def\tbigsqcup{\mathop{\textstyle \bigsqcup }}%
\def\tcoprod{\mathop{\textstyle \coprod }}%
\def\tbigcup{\mathop{\textstyle \bigcup }}%
\def\tbigvee{\mathop{\textstyle \bigvee }}%
\def\tbigotimes{\mathop{\textstyle \bigotimes }}%
\def\tbiguplus{\mathop{\textstyle \biguplus }}%
%
%
%Macros for display size operators:
%
\def\dint{\mathop{\displaystyle \int}}%
\def\diint{\mathop{\displaystyle \iint}}%
\def\diiint{\mathop{\displaystyle \iiint}}%
\def\diiiint{\mathop{\displaystyle \iiiint }}%
\def\didotsint{\mathop{\displaystyle \idotsint }}%
\def\doint{\mathop{\displaystyle \oint}}%
\def\dsum{\mathop{\displaystyle \sum }}%
\def\dprod{\mathop{\displaystyle \prod }}%
\def\dbigcap{\mathop{\displaystyle \bigcap }}%
\def\dbigwedge{\mathop{\displaystyle \bigwedge }}%
\def\dbigoplus{\mathop{\displaystyle \bigoplus }}%
\def\dbigodot{\mathop{\displaystyle \bigodot }}%
\def\dbigsqcup{\mathop{\displaystyle \bigsqcup }}%
\def\dcoprod{\mathop{\displaystyle \coprod }}%
\def\dbigcup{\mathop{\displaystyle \bigcup }}%
\def\dbigvee{\mathop{\displaystyle \bigvee }}%
\def\dbigotimes{\mathop{\displaystyle \bigotimes }}%
\def\dbiguplus{\mathop{\displaystyle \biguplus }}%

\if@compatibility\else
  % Always load amsmath in LaTeX2e mode
  \RequirePackage{amsmath}
\fi

\def\ExitTCILatex{\makeatother\endinput}

\bgroup
\ifx\ds@amstex\relax
   \message{amstex already loaded}\aftergroup\ExitTCILatex
\else
   \@ifpackageloaded{amsmath}%
      {\if@compatibility\message{amsmath already loaded}\fi\aftergroup\ExitTCILatex}
      {}
   \@ifpackageloaded{amstex}%
      {\if@compatibility\message{amstex already loaded}\fi\aftergroup\ExitTCILatex}
      {}
   \@ifpackageloaded{amsgen}%
      {\if@compatibility\message{amsgen already loaded}\fi\aftergroup\ExitTCILatex}
      {}
\fi
\egroup

%Exit if any of the AMS macros are already loaded.
%This is always the case for LaTeX2e mode.


%%%%%%%%%%%%%%%%%%%%%%%%%%%%%%%%%%%%%%%%%%%%%%%%%%%%%%%%%%%%%%%%%%%%%%%%%%
% NOTE: The rest of this file is read only if in LaTeX 2.09 compatibility
% mode. This section is used to define AMS-like constructs in the
% event they have not been defined.
%%%%%%%%%%%%%%%%%%%%%%%%%%%%%%%%%%%%%%%%%%%%%%%%%%%%%%%%%%%%%%%%%%%%%%%%%%
\typeout{TCILATEX defining AMS-like constructs in LaTeX 2.09 COMPATIBILITY MODE}
%%%%%%%%%%%%%%%%%%%%%%%%%%%%%%%%%%%%%%%%%%%%%%%%%%%%%%%%%%%%%%%%%%%%%%%%
%  Macros to define some AMS LaTeX constructs when 
%  AMS LaTeX has not been loaded
% 
% These macros are copied from the AMS-TeX package for doing
% multiple integrals.
%
\let\DOTSI\relax
\def\RIfM@{\relax\ifmmode}%
\def\FN@{\futurelet\next}%
\newcount\intno@
\def\iint{\DOTSI\intno@\tw@\FN@\ints@}%
\def\iiint{\DOTSI\intno@\thr@@\FN@\ints@}%
\def\iiiint{\DOTSI\intno@4 \FN@\ints@}%
\def\idotsint{\DOTSI\intno@\z@\FN@\ints@}%
\def\ints@{\findlimits@\ints@@}%
\newif\iflimtoken@
\newif\iflimits@
\def\findlimits@{\limtoken@true\ifx\next\limits\limits@true
 \else\ifx\next\nolimits\limits@false\else
 \limtoken@false\ifx\ilimits@\nolimits\limits@false\else
 \ifinner\limits@false\else\limits@true\fi\fi\fi\fi}%
\def\multint@{\int\ifnum\intno@=\z@\intdots@                          %1
 \else\intkern@\fi                                                    %2
 \ifnum\intno@>\tw@\int\intkern@\fi                                   %3
 \ifnum\intno@>\thr@@\int\intkern@\fi                                 %4
 \int}%                                                               %5
\def\multintlimits@{\intop\ifnum\intno@=\z@\intdots@\else\intkern@\fi
 \ifnum\intno@>\tw@\intop\intkern@\fi
 \ifnum\intno@>\thr@@\intop\intkern@\fi\intop}%
\def\intic@{%
    \mathchoice{\hskip.5em}{\hskip.4em}{\hskip.4em}{\hskip.4em}}%
\def\negintic@{\mathchoice
 {\hskip-.5em}{\hskip-.4em}{\hskip-.4em}{\hskip-.4em}}%
\def\ints@@{\iflimtoken@                                              %1
 \def\ints@@@{\iflimits@\negintic@
   \mathop{\intic@\multintlimits@}\limits                             %2
  \else\multint@\nolimits\fi                                          %3
  \eat@}%                                                             %4
 \else                                                                %5
 \def\ints@@@{\iflimits@\negintic@
  \mathop{\intic@\multintlimits@}\limits\else
  \multint@\nolimits\fi}\fi\ints@@@}%
\def\intkern@{\mathchoice{\!\!\!}{\!\!}{\!\!}{\!\!}}%
\def\plaincdots@{\mathinner{\cdotp\cdotp\cdotp}}%
\def\intdots@{\mathchoice{\plaincdots@}%
 {{\cdotp}\mkern1.5mu{\cdotp}\mkern1.5mu{\cdotp}}%
 {{\cdotp}\mkern1mu{\cdotp}\mkern1mu{\cdotp}}%
 {{\cdotp}\mkern1mu{\cdotp}\mkern1mu{\cdotp}}}%
%
%
%  These macros are for doing the AMS \text{} construct
%
\def\RIfM@{\relax\protect\ifmmode}
\def\text{\RIfM@\expandafter\text@\else\expandafter\mbox\fi}
\let\nfss@text\text
\def\text@#1{\mathchoice
   {\textdef@\displaystyle\f@size{#1}}%
   {\textdef@\textstyle\tf@size{\firstchoice@false #1}}%
   {\textdef@\textstyle\sf@size{\firstchoice@false #1}}%
   {\textdef@\textstyle \ssf@size{\firstchoice@false #1}}%
   \glb@settings}

\def\textdef@#1#2#3{\hbox{{%
                    \everymath{#1}%
                    \let\f@size#2\selectfont
                    #3}}}
\newif\iffirstchoice@
\firstchoice@true
%
%These are the AMS constructs for multiline limits.
%
\def\Let@{\relax\iffalse{\fi\let\\=\cr\iffalse}\fi}%
\def\vspace@{\def\vspace##1{\crcr\noalign{\vskip##1\relax}}}%
\def\multilimits@{\bgroup\vspace@\Let@
 \baselineskip\fontdimen10 \scriptfont\tw@
 \advance\baselineskip\fontdimen12 \scriptfont\tw@
 \lineskip\thr@@\fontdimen8 \scriptfont\thr@@
 \lineskiplimit\lineskip
 \vbox\bgroup\ialign\bgroup\hfil$\m@th\scriptstyle{##}$\hfil\crcr}%
\def\Sb{_\multilimits@}%
\def\endSb{\crcr\egroup\egroup\egroup}%
\def\Sp{^\multilimits@}%
\let\endSp\endSb
%
%
%These are AMS constructs for horizontal arrows
%
\newdimen\ex@
\ex@.2326ex
\def\rightarrowfill@#1{$#1\m@th\mathord-\mkern-6mu\cleaders
 \hbox{$#1\mkern-2mu\mathord-\mkern-2mu$}\hfill
 \mkern-6mu\mathord\rightarrow$}%
\def\leftarrowfill@#1{$#1\m@th\mathord\leftarrow\mkern-6mu\cleaders
 \hbox{$#1\mkern-2mu\mathord-\mkern-2mu$}\hfill\mkern-6mu\mathord-$}%
\def\leftrightarrowfill@#1{$#1\m@th\mathord\leftarrow
\mkern-6mu\cleaders
 \hbox{$#1\mkern-2mu\mathord-\mkern-2mu$}\hfill
 \mkern-6mu\mathord\rightarrow$}%
\def\overrightarrow{\mathpalette\overrightarrow@}%
\def\overrightarrow@#1#2{\vbox{\ialign{##\crcr\rightarrowfill@#1\crcr
 \noalign{\kern-\ex@\nointerlineskip}$\m@th\hfil#1#2\hfil$\crcr}}}%
\let\overarrow\overrightarrow
\def\overleftarrow{\mathpalette\overleftarrow@}%
\def\overleftarrow@#1#2{\vbox{\ialign{##\crcr\leftarrowfill@#1\crcr
 \noalign{\kern-\ex@\nointerlineskip}$\m@th\hfil#1#2\hfil$\crcr}}}%
\def\overleftrightarrow{\mathpalette\overleftrightarrow@}%
\def\overleftrightarrow@#1#2{\vbox{\ialign{##\crcr
   \leftrightarrowfill@#1\crcr
 \noalign{\kern-\ex@\nointerlineskip}$\m@th\hfil#1#2\hfil$\crcr}}}%
\def\underrightarrow{\mathpalette\underrightarrow@}%
\def\underrightarrow@#1#2{\vtop{\ialign{##\crcr$\m@th\hfil#1#2\hfil
  $\crcr\noalign{\nointerlineskip}\rightarrowfill@#1\crcr}}}%
\let\underarrow\underrightarrow
\def\underleftarrow{\mathpalette\underleftarrow@}%
\def\underleftarrow@#1#2{\vtop{\ialign{##\crcr$\m@th\hfil#1#2\hfil
  $\crcr\noalign{\nointerlineskip}\leftarrowfill@#1\crcr}}}%
\def\underleftrightarrow{\mathpalette\underleftrightarrow@}%
\def\underleftrightarrow@#1#2{\vtop{\ialign{##\crcr$\m@th
  \hfil#1#2\hfil$\crcr
 \noalign{\nointerlineskip}\leftrightarrowfill@#1\crcr}}}%
%%%%%%%%%%%%%%%%%%%%%

\def\qopnamewl@#1{\mathop{\operator@font#1}\nlimits@}
\let\nlimits@\displaylimits
\def\setboxz@h{\setbox\z@\hbox}


\def\varlim@#1#2{\mathop{\vtop{\ialign{##\crcr
 \hfil$#1\m@th\operator@font lim$\hfil\crcr
 \noalign{\nointerlineskip}#2#1\crcr
 \noalign{\nointerlineskip\kern-\ex@}\crcr}}}}

 \def\rightarrowfill@#1{\m@th\setboxz@h{$#1-$}\ht\z@\z@
  $#1\copy\z@\mkern-6mu\cleaders
  \hbox{$#1\mkern-2mu\box\z@\mkern-2mu$}\hfill
  \mkern-6mu\mathord\rightarrow$}
\def\leftarrowfill@#1{\m@th\setboxz@h{$#1-$}\ht\z@\z@
  $#1\mathord\leftarrow\mkern-6mu\cleaders
  \hbox{$#1\mkern-2mu\copy\z@\mkern-2mu$}\hfill
  \mkern-6mu\box\z@$}


\def\projlim{\qopnamewl@{proj\,lim}}
\def\injlim{\qopnamewl@{inj\,lim}}
\def\varinjlim{\mathpalette\varlim@\rightarrowfill@}
\def\varprojlim{\mathpalette\varlim@\leftarrowfill@}
\def\varliminf{\mathpalette\varliminf@{}}
\def\varliminf@#1{\mathop{\underline{\vrule\@depth.2\ex@\@width\z@
   \hbox{$#1\m@th\operator@font lim$}}}}
\def\varlimsup{\mathpalette\varlimsup@{}}
\def\varlimsup@#1{\mathop{\overline
  {\hbox{$#1\m@th\operator@font lim$}}}}

%
%Companion to stackrel
\def\stackunder#1#2{\mathrel{\mathop{#2}\limits_{#1}}}%
%
%
% These are AMS environments that will be defined to
% be verbatims if amstex has not actually been 
% loaded
%
%
\begingroup \catcode `|=0 \catcode `[= 1
\catcode`]=2 \catcode `\{=12 \catcode `\}=12
\catcode`\\=12 
|gdef|@alignverbatim#1\end{align}[#1|end[align]]
|gdef|@salignverbatim#1\end{align*}[#1|end[align*]]

|gdef|@alignatverbatim#1\end{alignat}[#1|end[alignat]]
|gdef|@salignatverbatim#1\end{alignat*}[#1|end[alignat*]]

|gdef|@xalignatverbatim#1\end{xalignat}[#1|end[xalignat]]
|gdef|@sxalignatverbatim#1\end{xalignat*}[#1|end[xalignat*]]

|gdef|@gatherverbatim#1\end{gather}[#1|end[gather]]
|gdef|@sgatherverbatim#1\end{gather*}[#1|end[gather*]]

|gdef|@gatherverbatim#1\end{gather}[#1|end[gather]]
|gdef|@sgatherverbatim#1\end{gather*}[#1|end[gather*]]


|gdef|@multilineverbatim#1\end{multiline}[#1|end[multiline]]
|gdef|@smultilineverbatim#1\end{multiline*}[#1|end[multiline*]]

|gdef|@arraxverbatim#1\end{arrax}[#1|end[arrax]]
|gdef|@sarraxverbatim#1\end{arrax*}[#1|end[arrax*]]

|gdef|@tabulaxverbatim#1\end{tabulax}[#1|end[tabulax]]
|gdef|@stabulaxverbatim#1\end{tabulax*}[#1|end[tabulax*]]


|endgroup
  

  
\def\align{\@verbatim \frenchspacing\@vobeyspaces \@alignverbatim
You are using the "align" environment in a style in which it is not defined.}
\let\endalign=\endtrivlist
 
\@namedef{align*}{\@verbatim\@salignverbatim
You are using the "align*" environment in a style in which it is not defined.}
\expandafter\let\csname endalign*\endcsname =\endtrivlist




\def\alignat{\@verbatim \frenchspacing\@vobeyspaces \@alignatverbatim
You are using the "alignat" environment in a style in which it is not defined.}
\let\endalignat=\endtrivlist
 
\@namedef{alignat*}{\@verbatim\@salignatverbatim
You are using the "alignat*" environment in a style in which it is not defined.}
\expandafter\let\csname endalignat*\endcsname =\endtrivlist




\def\xalignat{\@verbatim \frenchspacing\@vobeyspaces \@xalignatverbatim
You are using the "xalignat" environment in a style in which it is not defined.}
\let\endxalignat=\endtrivlist
 
\@namedef{xalignat*}{\@verbatim\@sxalignatverbatim
You are using the "xalignat*" environment in a style in which it is not defined.}
\expandafter\let\csname endxalignat*\endcsname =\endtrivlist




\def\gather{\@verbatim \frenchspacing\@vobeyspaces \@gatherverbatim
You are using the "gather" environment in a style in which it is not defined.}
\let\endgather=\endtrivlist
 
\@namedef{gather*}{\@verbatim\@sgatherverbatim
You are using the "gather*" environment in a style in which it is not defined.}
\expandafter\let\csname endgather*\endcsname =\endtrivlist


\def\multiline{\@verbatim \frenchspacing\@vobeyspaces \@multilineverbatim
You are using the "multiline" environment in a style in which it is not defined.}
\let\endmultiline=\endtrivlist
 
\@namedef{multiline*}{\@verbatim\@smultilineverbatim
You are using the "multiline*" environment in a style in which it is not defined.}
\expandafter\let\csname endmultiline*\endcsname =\endtrivlist


\def\arrax{\@verbatim \frenchspacing\@vobeyspaces \@arraxverbatim
You are using a type of "array" construct that is only allowed in AmS-LaTeX.}
\let\endarrax=\endtrivlist

\def\tabulax{\@verbatim \frenchspacing\@vobeyspaces \@tabulaxverbatim
You are using a type of "tabular" construct that is only allowed in AmS-LaTeX.}
\let\endtabulax=\endtrivlist

 
\@namedef{arrax*}{\@verbatim\@sarraxverbatim
You are using a type of "array*" construct that is only allowed in AmS-LaTeX.}
\expandafter\let\csname endarrax*\endcsname =\endtrivlist

\@namedef{tabulax*}{\@verbatim\@stabulaxverbatim
You are using a type of "tabular*" construct that is only allowed in AmS-LaTeX.}
\expandafter\let\csname endtabulax*\endcsname =\endtrivlist

% macro to simulate ams tag construct


% This macro is a fix to the equation environment
 \def\endequation{%
     \ifmmode\ifinner % FLEQN hack
      \iftag@
        \addtocounter{equation}{-1} % undo the increment made in the begin part
        $\hfil
           \displaywidth\linewidth\@taggnum\egroup \endtrivlist
        \global\tag@false
        \global\@ignoretrue   
      \else
        $\hfil
           \displaywidth\linewidth\@eqnnum\egroup \endtrivlist
        \global\tag@false
        \global\@ignoretrue 
      \fi
     \else   
      \iftag@
        \addtocounter{equation}{-1} % undo the increment made in the begin part
        \eqno \hbox{\@taggnum}
        \global\tag@false%
        $$\global\@ignoretrue
      \else
        \eqno \hbox{\@eqnnum}% $$ BRACE MATCHING HACK
        $$\global\@ignoretrue
      \fi
     \fi\fi
 } 

 \newif\iftag@ \tag@false
 
 \def\TCItag{\@ifnextchar*{\@TCItagstar}{\@TCItag}}
 \def\@TCItag#1{%
     \global\tag@true
     \global\def\@taggnum{(#1)}%
     \global\def\@currentlabel{#1}}
 \def\@TCItagstar*#1{%
     \global\tag@true
     \global\def\@taggnum{#1}%
     \global\def\@currentlabel{#1}}

  \@ifundefined{tag}{
     \def\tag{\@ifnextchar*{\@tagstar}{\@tag}}
     \def\@tag#1{%
         \global\tag@true
         \global\def\@taggnum{(#1)}}
     \def\@tagstar*#1{%
         \global\tag@true
         \global\def\@taggnum{#1}}
  }{}

\def\tfrac#1#2{{\textstyle {#1 \over #2}}}%
\def\dfrac#1#2{{\displaystyle {#1 \over #2}}}%
\def\binom#1#2{{#1 \choose #2}}%
\def\tbinom#1#2{{\textstyle {#1 \choose #2}}}%
\def\dbinom#1#2{{\displaystyle {#1 \choose #2}}}%

% Do not add anything to the end of this file.  
% The last section of the file is loaded only if 
% amstex has not been.
\makeatother
\endinput

\setlength{\columnseprule}{1pt}


\begin{document}


\part{The direction of rolling friction w/o slippling}

How to determine the direction of friction force acting on a rolling object?
This is important and is ensential to solving the dynamics of rolling
motions. A few easily-mistaken cases are reviewed here, and excercising
problems are provided.

\begin{case}
Round object freely rolls down the hill\newline
\newline
The only force that makes the object rotate is friction so friction has to
go up the hill. This friction force is exerted on the wheel by the slope.

\input{../../../../Scripts/cordtrans/cases_fig_only/whatever.pgf}
\end{case}

\newpage

\begin{case}
Object is forced to roll up the hill initially but external force is removed
once the object is going upward. We are considering the later part of the
motion when the external force is removed, so only gravitation is in place.
The wheel is still rolling up the hill.\newline
\newline
The rotation of the object slows down as it climbs up the hill. Friction is
the only force that produces a torque to slow down the rotation. So it needs
to go against the rotating direction. So the friction force acting on the
wheel is up the hill.

%% Creator: Matplotlib, PGF backend
%%
%% To include the figure in your LaTeX document, write
%%   \input{<filename>.pgf}
%%
%% Make sure the required packages are loaded in your preamble
%%   \usepackage{pgf}
%%
%% Figures using additional raster images can only be included by \input if
%% they are in the same directory as the main LaTeX file. For loading figures
%% from other directories you can use the `import` package
%%   \usepackage{import}
%% and then include the figures with
%%   \import{<path to file>}{<filename>.pgf}
%%
%% Matplotlib used the following preamble
%%   \usepackage{fontspec}
%%   \setmainfont{Times New Roman}
%%   \setsansfont{Verdana}
%%   \setmonofont{Courier New}
%%
\begingroup%
\makeatletter%
\begin{pgfpicture}%
\pgfpathrectangle{\pgfpointorigin}{\pgfqpoint{4.000000in}{4.000000in}}%
\pgfusepath{use as bounding box}%
\begin{pgfscope}%
\pgfsetbuttcap%
\pgfsetroundjoin%
\definecolor{currentfill}{rgb}{1.000000,1.000000,1.000000}%
\pgfsetfillcolor{currentfill}%
\pgfsetlinewidth{0.000000pt}%
\definecolor{currentstroke}{rgb}{1.000000,1.000000,1.000000}%
\pgfsetstrokecolor{currentstroke}%
\pgfsetdash{}{0pt}%
\pgfpathmoveto{\pgfqpoint{0.000000in}{0.000000in}}%
\pgfpathlineto{\pgfqpoint{4.000000in}{0.000000in}}%
\pgfpathlineto{\pgfqpoint{4.000000in}{4.000000in}}%
\pgfpathlineto{\pgfqpoint{0.000000in}{4.000000in}}%
\pgfpathclose%
\pgfusepath{fill}%
\end{pgfscope}%
\begin{pgfscope}%
\pgfpathrectangle{\pgfqpoint{0.500000in}{0.400000in}}{\pgfqpoint{3.100000in}{3.200000in}} %
\pgfusepath{clip}%
\pgfsetbuttcap%
\pgfsetroundjoin%
\definecolor{currentfill}{rgb}{0.000000,0.000000,1.000000}%
\pgfsetfillcolor{currentfill}%
\pgfsetlinewidth{1.003750pt}%
\definecolor{currentstroke}{rgb}{0.000000,0.000000,0.000000}%
\pgfsetstrokecolor{currentstroke}%
\pgfsetdash{}{0pt}%
\pgfpathmoveto{\pgfqpoint{2.871466in}{2.163611in}}%
\pgfpathlineto{\pgfqpoint{2.771677in}{2.026452in}}%
\pgfpathlineto{\pgfqpoint{2.744982in}{2.072688in}}%
\pgfpathlineto{\pgfqpoint{1.999238in}{1.642132in}}%
\pgfpathlineto{\pgfqpoint{1.983738in}{1.668979in}}%
\pgfpathlineto{\pgfqpoint{2.729482in}{2.099535in}}%
\pgfpathlineto{\pgfqpoint{2.702788in}{2.145771in}}%
\pgfpathclose%
\pgfusepath{stroke,fill}%
\end{pgfscope}%
\begin{pgfscope}%
\pgfpathrectangle{\pgfqpoint{0.500000in}{0.400000in}}{\pgfqpoint{3.100000in}{3.200000in}} %
\pgfusepath{clip}%
\pgfsetbuttcap%
\pgfsetroundjoin%
\definecolor{currentfill}{rgb}{0.000000,0.000000,1.000000}%
\pgfsetfillcolor{currentfill}%
\pgfsetlinewidth{1.003750pt}%
\definecolor{currentstroke}{rgb}{0.000000,0.000000,0.000000}%
\pgfsetstrokecolor{currentstroke}%
\pgfsetdash{}{0pt}%
\pgfpathmoveto{\pgfqpoint{1.647044in}{0.639444in}}%
\pgfpathlineto{\pgfqpoint{1.578155in}{0.794444in}}%
\pgfpathlineto{\pgfqpoint{1.631544in}{0.794444in}}%
\pgfpathlineto{\pgfqpoint{1.631544in}{2.252151in}}%
\pgfpathlineto{\pgfqpoint{1.662544in}{2.252151in}}%
\pgfpathlineto{\pgfqpoint{1.662544in}{0.794444in}}%
\pgfpathlineto{\pgfqpoint{1.715933in}{0.794444in}}%
\pgfpathclose%
\pgfusepath{stroke,fill}%
\end{pgfscope}%
\begin{pgfscope}%
\pgfpathrectangle{\pgfqpoint{0.500000in}{0.400000in}}{\pgfqpoint{3.100000in}{3.200000in}} %
\pgfusepath{clip}%
\pgfsetbuttcap%
\pgfsetroundjoin%
\definecolor{currentfill}{rgb}{1.000000,0.000000,0.000000}%
\pgfsetfillcolor{currentfill}%
\pgfsetlinewidth{1.003750pt}%
\definecolor{currentstroke}{rgb}{1.000000,0.000000,0.000000}%
\pgfsetstrokecolor{currentstroke}%
\pgfsetdash{}{0pt}%
\pgfpathmoveto{\pgfqpoint{3.317511in}{3.216595in}}%
\pgfpathlineto{\pgfqpoint{3.224643in}{2.964113in}}%
\pgfpathlineto{\pgfqpoint{3.160060in}{3.075975in}}%
\pgfpathlineto{\pgfqpoint{2.414316in}{2.645419in}}%
\pgfpathlineto{\pgfqpoint{2.371260in}{2.719994in}}%
\pgfpathlineto{\pgfqpoint{3.117004in}{3.150549in}}%
\pgfpathlineto{\pgfqpoint{3.052421in}{3.262411in}}%
\pgfpathclose%
\pgfusepath{stroke,fill}%
\end{pgfscope}%
\begin{pgfscope}%
\pgfpathrectangle{\pgfqpoint{0.500000in}{0.400000in}}{\pgfqpoint{3.100000in}{3.200000in}} %
\pgfusepath{clip}%
\pgfsetrectcap%
\pgfsetroundjoin%
\pgfsetlinewidth{1.003750pt}%
\definecolor{currentstroke}{rgb}{0.000000,0.000000,1.000000}%
\pgfsetstrokecolor{currentstroke}%
\pgfsetdash{}{0pt}%
\pgfpathmoveto{\pgfqpoint{0.500000in}{0.794444in}}%
\pgfpathlineto{\pgfqpoint{3.482976in}{0.794444in}}%
\pgfusepath{stroke}%
\end{pgfscope}%
\begin{pgfscope}%
\pgfpathrectangle{\pgfqpoint{0.500000in}{0.400000in}}{\pgfqpoint{3.100000in}{3.200000in}} %
\pgfusepath{clip}%
\pgfsetrectcap%
\pgfsetroundjoin%
\pgfsetlinewidth{1.003750pt}%
\definecolor{currentstroke}{rgb}{0.000000,0.000000,1.000000}%
\pgfsetstrokecolor{currentstroke}%
\pgfsetdash{}{0pt}%
\pgfpathmoveto{\pgfqpoint{3.482976in}{0.794444in}}%
\pgfpathlineto{\pgfqpoint{3.482976in}{2.516667in}}%
\pgfusepath{stroke}%
\end{pgfscope}%
\begin{pgfscope}%
\pgfpathrectangle{\pgfqpoint{0.500000in}{0.400000in}}{\pgfqpoint{3.100000in}{3.200000in}} %
\pgfusepath{clip}%
\pgfsetrectcap%
\pgfsetroundjoin%
\pgfsetlinewidth{1.003750pt}%
\definecolor{currentstroke}{rgb}{0.000000,0.000000,1.000000}%
\pgfsetstrokecolor{currentstroke}%
\pgfsetdash{}{0pt}%
\pgfpathmoveto{\pgfqpoint{0.500000in}{0.794444in}}%
\pgfpathlineto{\pgfqpoint{3.482976in}{2.516667in}}%
\pgfusepath{stroke}%
\end{pgfscope}%
\begin{pgfscope}%
\pgfpathrectangle{\pgfqpoint{0.500000in}{0.400000in}}{\pgfqpoint{3.100000in}{3.200000in}} %
\pgfusepath{clip}%
\pgfsetrectcap%
\pgfsetroundjoin%
\pgfsetlinewidth{1.003750pt}%
\definecolor{currentstroke}{rgb}{0.000000,0.000000,0.000000}%
\pgfsetstrokecolor{currentstroke}%
\pgfsetdash{}{0pt}%
\pgfpathmoveto{\pgfqpoint{0.672222in}{0.794444in}}%
\pgfpathlineto{\pgfqpoint{0.671986in}{0.803458in}}%
\pgfpathlineto{\pgfqpoint{0.671279in}{0.812447in}}%
\pgfpathlineto{\pgfqpoint{0.670102in}{0.821386in}}%
\pgfpathlineto{\pgfqpoint{0.668459in}{0.830251in}}%
\pgfpathlineto{\pgfqpoint{0.666354in}{0.839019in}}%
\pgfpathlineto{\pgfqpoint{0.663793in}{0.847664in}}%
\pgfpathlineto{\pgfqpoint{0.660783in}{0.856163in}}%
\pgfpathlineto{\pgfqpoint{0.657333in}{0.864494in}}%
\pgfpathlineto{\pgfqpoint{0.653451in}{0.872632in}}%
\pgfpathlineto{\pgfqpoint{0.649149in}{0.880556in}}%
\pgfusepath{stroke}%
\end{pgfscope}%
\begin{pgfscope}%
\pgfpathrectangle{\pgfqpoint{0.500000in}{0.400000in}}{\pgfqpoint{3.100000in}{3.200000in}} %
\pgfusepath{clip}%
\pgfsetrectcap%
\pgfsetroundjoin%
\pgfsetlinewidth{1.003750pt}%
\definecolor{currentstroke}{rgb}{0.000000,0.000000,0.000000}%
\pgfsetstrokecolor{currentstroke}%
\pgfsetdash{}{0pt}%
\pgfpathmoveto{\pgfqpoint{2.335933in}{2.252151in}}%
\pgfpathlineto{\pgfqpoint{2.327451in}{2.359917in}}%
\pgfpathlineto{\pgfqpoint{2.302216in}{2.465029in}}%
\pgfpathlineto{\pgfqpoint{2.260848in}{2.564900in}}%
\pgfpathlineto{\pgfqpoint{2.204367in}{2.657070in}}%
\pgfpathlineto{\pgfqpoint{2.134162in}{2.739269in}}%
\pgfpathlineto{\pgfqpoint{2.051962in}{2.809474in}}%
\pgfpathlineto{\pgfqpoint{1.959793in}{2.865955in}}%
\pgfpathlineto{\pgfqpoint{1.859922in}{2.907323in}}%
\pgfpathlineto{\pgfqpoint{1.754810in}{2.932558in}}%
\pgfpathlineto{\pgfqpoint{1.647044in}{2.941040in}}%
\pgfpathlineto{\pgfqpoint{1.539278in}{2.932558in}}%
\pgfpathlineto{\pgfqpoint{1.434165in}{2.907323in}}%
\pgfpathlineto{\pgfqpoint{1.334295in}{2.865955in}}%
\pgfpathlineto{\pgfqpoint{1.242125in}{2.809474in}}%
\pgfpathlineto{\pgfqpoint{1.159926in}{2.739269in}}%
\pgfpathlineto{\pgfqpoint{1.089721in}{2.657070in}}%
\pgfpathlineto{\pgfqpoint{1.033239in}{2.564900in}}%
\pgfpathlineto{\pgfqpoint{0.991871in}{2.465029in}}%
\pgfpathlineto{\pgfqpoint{0.966636in}{2.359917in}}%
\pgfpathlineto{\pgfqpoint{0.958155in}{2.252151in}}%
\pgfusepath{stroke}%
\end{pgfscope}%
\begin{pgfscope}%
\pgfpathrectangle{\pgfqpoint{0.500000in}{0.400000in}}{\pgfqpoint{3.100000in}{3.200000in}} %
\pgfusepath{clip}%
\pgfsetrectcap%
\pgfsetroundjoin%
\pgfsetlinewidth{1.003750pt}%
\definecolor{currentstroke}{rgb}{0.000000,0.000000,0.000000}%
\pgfsetstrokecolor{currentstroke}%
\pgfsetdash{}{0pt}%
\pgfpathmoveto{\pgfqpoint{0.958155in}{2.252151in}}%
\pgfpathlineto{\pgfqpoint{0.966636in}{2.144385in}}%
\pgfpathlineto{\pgfqpoint{0.991871in}{2.039272in}}%
\pgfpathlineto{\pgfqpoint{1.033239in}{1.939402in}}%
\pgfpathlineto{\pgfqpoint{1.089721in}{1.847232in}}%
\pgfpathlineto{\pgfqpoint{1.159926in}{1.765033in}}%
\pgfpathlineto{\pgfqpoint{1.242125in}{1.694828in}}%
\pgfpathlineto{\pgfqpoint{1.334295in}{1.638346in}}%
\pgfpathlineto{\pgfqpoint{1.434165in}{1.596979in}}%
\pgfpathlineto{\pgfqpoint{1.539278in}{1.571743in}}%
\pgfpathlineto{\pgfqpoint{1.647044in}{1.563262in}}%
\pgfpathlineto{\pgfqpoint{1.754810in}{1.571743in}}%
\pgfpathlineto{\pgfqpoint{1.859922in}{1.596979in}}%
\pgfpathlineto{\pgfqpoint{1.959793in}{1.638346in}}%
\pgfpathlineto{\pgfqpoint{2.051962in}{1.694828in}}%
\pgfpathlineto{\pgfqpoint{2.134162in}{1.765033in}}%
\pgfpathlineto{\pgfqpoint{2.204367in}{1.847232in}}%
\pgfpathlineto{\pgfqpoint{2.260848in}{1.939402in}}%
\pgfpathlineto{\pgfqpoint{2.302216in}{2.039272in}}%
\pgfpathlineto{\pgfqpoint{2.327451in}{2.144385in}}%
\pgfpathlineto{\pgfqpoint{2.335933in}{2.252151in}}%
\pgfusepath{stroke}%
\end{pgfscope}%
\begin{pgfscope}%
\pgfpathrectangle{\pgfqpoint{0.500000in}{0.400000in}}{\pgfqpoint{3.100000in}{3.200000in}} %
\pgfusepath{clip}%
\pgfsetbuttcap%
\pgfsetroundjoin%
\definecolor{currentfill}{rgb}{0.000000,0.000000,1.000000}%
\pgfsetfillcolor{currentfill}%
\pgfsetlinewidth{0.501875pt}%
\definecolor{currentstroke}{rgb}{0.000000,0.000000,0.000000}%
\pgfsetstrokecolor{currentstroke}%
\pgfsetdash{}{0pt}%
\pgfsys@defobject{currentmarker}{\pgfqpoint{-0.041667in}{-0.041667in}}{\pgfqpoint{0.041667in}{0.041667in}}{%
\pgfpathmoveto{\pgfqpoint{0.000000in}{-0.041667in}}%
\pgfpathcurveto{\pgfqpoint{0.011050in}{-0.041667in}}{\pgfqpoint{0.021649in}{-0.037276in}}{\pgfqpoint{0.029463in}{-0.029463in}}%
\pgfpathcurveto{\pgfqpoint{0.037276in}{-0.021649in}}{\pgfqpoint{0.041667in}{-0.011050in}}{\pgfqpoint{0.041667in}{0.000000in}}%
\pgfpathcurveto{\pgfqpoint{0.041667in}{0.011050in}}{\pgfqpoint{0.037276in}{0.021649in}}{\pgfqpoint{0.029463in}{0.029463in}}%
\pgfpathcurveto{\pgfqpoint{0.021649in}{0.037276in}}{\pgfqpoint{0.011050in}{0.041667in}}{\pgfqpoint{0.000000in}{0.041667in}}%
\pgfpathcurveto{\pgfqpoint{-0.011050in}{0.041667in}}{\pgfqpoint{-0.021649in}{0.037276in}}{\pgfqpoint{-0.029463in}{0.029463in}}%
\pgfpathcurveto{\pgfqpoint{-0.037276in}{0.021649in}}{\pgfqpoint{-0.041667in}{0.011050in}}{\pgfqpoint{-0.041667in}{0.000000in}}%
\pgfpathcurveto{\pgfqpoint{-0.041667in}{-0.011050in}}{\pgfqpoint{-0.037276in}{-0.021649in}}{\pgfqpoint{-0.029463in}{-0.029463in}}%
\pgfpathcurveto{\pgfqpoint{-0.021649in}{-0.037276in}}{\pgfqpoint{-0.011050in}{-0.041667in}}{\pgfqpoint{0.000000in}{-0.041667in}}%
\pgfpathclose%
\pgfusepath{stroke,fill}%
}%
\begin{pgfscope}%
\pgfsys@transformshift{1.647044in}{2.252151in}%
\pgfsys@useobject{currentmarker}{}%
\end{pgfscope}%
\end{pgfscope}%
\begin{pgfscope}%
\pgfpathrectangle{\pgfqpoint{0.500000in}{0.400000in}}{\pgfqpoint{3.100000in}{3.200000in}} %
\pgfusepath{clip}%
\pgfsetbuttcap%
\pgfsetroundjoin%
\definecolor{currentfill}{rgb}{0.000000,0.500000,0.000000}%
\pgfsetfillcolor{currentfill}%
\pgfsetlinewidth{0.501875pt}%
\definecolor{currentstroke}{rgb}{0.000000,0.000000,0.000000}%
\pgfsetstrokecolor{currentstroke}%
\pgfsetdash{}{0pt}%
\pgfsys@defobject{currentmarker}{\pgfqpoint{-0.041667in}{-0.041667in}}{\pgfqpoint{0.041667in}{0.041667in}}{%
\pgfpathmoveto{\pgfqpoint{0.000000in}{-0.041667in}}%
\pgfpathcurveto{\pgfqpoint{0.011050in}{-0.041667in}}{\pgfqpoint{0.021649in}{-0.037276in}}{\pgfqpoint{0.029463in}{-0.029463in}}%
\pgfpathcurveto{\pgfqpoint{0.037276in}{-0.021649in}}{\pgfqpoint{0.041667in}{-0.011050in}}{\pgfqpoint{0.041667in}{0.000000in}}%
\pgfpathcurveto{\pgfqpoint{0.041667in}{0.011050in}}{\pgfqpoint{0.037276in}{0.021649in}}{\pgfqpoint{0.029463in}{0.029463in}}%
\pgfpathcurveto{\pgfqpoint{0.021649in}{0.037276in}}{\pgfqpoint{0.011050in}{0.041667in}}{\pgfqpoint{0.000000in}{0.041667in}}%
\pgfpathcurveto{\pgfqpoint{-0.011050in}{0.041667in}}{\pgfqpoint{-0.021649in}{0.037276in}}{\pgfqpoint{-0.029463in}{0.029463in}}%
\pgfpathcurveto{\pgfqpoint{-0.037276in}{0.021649in}}{\pgfqpoint{-0.041667in}{0.011050in}}{\pgfqpoint{-0.041667in}{0.000000in}}%
\pgfpathcurveto{\pgfqpoint{-0.041667in}{-0.011050in}}{\pgfqpoint{-0.037276in}{-0.021649in}}{\pgfqpoint{-0.029463in}{-0.029463in}}%
\pgfpathcurveto{\pgfqpoint{-0.021649in}{-0.037276in}}{\pgfqpoint{-0.011050in}{-0.041667in}}{\pgfqpoint{0.000000in}{-0.041667in}}%
\pgfpathclose%
\pgfusepath{stroke,fill}%
}%
\begin{pgfscope}%
\pgfsys@transformshift{1.991488in}{1.655556in}%
\pgfsys@useobject{currentmarker}{}%
\end{pgfscope}%
\end{pgfscope}%
\begin{pgfscope}%
\pgftext[x=0.844444in,y=0.794444in,left,bottom]{{\sffamily\fontsize{20.000000}{24.000000}\selectfont \(\displaystyle \theta\)}}%
\end{pgfscope}%
\begin{pgfscope}%
\pgftext[x=1.647044in,y=2.252151in,right,bottom]{{\sffamily\fontsize{20.000000}{24.000000}\selectfont \(\displaystyle CM\)}}%
\end{pgfscope}%
\begin{pgfscope}%
\pgftext[x=2.737232in,y=2.086111in,left,top]{{\sffamily\fontsize{20.000000}{24.000000}\selectfont \(\displaystyle F_{fr}\)}}%
\end{pgfscope}%
\begin{pgfscope}%
\pgftext[x=1.647044in,y=0.794444in,left,top]{{\sffamily\fontsize{20.000000}{24.000000}\selectfont \(\displaystyle mg\)}}%
\end{pgfscope}%
\begin{pgfscope}%
\pgftext[x=3.138532in,y=3.285484in,left,bottom]{{\sffamily\fontsize{20.000000}{24.000000}\selectfont \(\displaystyle v\)}}%
\end{pgfscope}%
\end{pgfpicture}%
\makeatother%
\endgroup%

\end{case}

\newpage

\begin{case}
External force $F_{p}$ passing through CM point\newline
%% Creator: Matplotlib, PGF backend
%%
%% To include the figure in your LaTeX document, write
%%   \input{<filename>.pgf}
%%
%% Make sure the required packages are loaded in your preamble
%%   \usepackage{pgf}
%%
%% Figures using additional raster images can only be included by \input if
%% they are in the same directory as the main LaTeX file. For loading figures
%% from other directories you can use the `import` package
%%   \usepackage{import}
%% and then include the figures with
%%   \import{<path to file>}{<filename>.pgf}
%%
%% Matplotlib used the following preamble
%%   \usepackage{fontspec}
%%   \setmainfont{Times New Roman}
%%   \setsansfont{Verdana}
%%   \setmonofont{Courier New}
%%
\begingroup%
\makeatletter%
\begin{pgfpicture}%
\pgfpathrectangle{\pgfpointorigin}{\pgfqpoint{3.000000in}{2.000000in}}%
\pgfusepath{use as bounding box}%
\begin{pgfscope}%
\pgfsetbuttcap%
\pgfsetroundjoin%
\definecolor{currentfill}{rgb}{1.000000,1.000000,1.000000}%
\pgfsetfillcolor{currentfill}%
\pgfsetlinewidth{0.000000pt}%
\definecolor{currentstroke}{rgb}{1.000000,1.000000,1.000000}%
\pgfsetstrokecolor{currentstroke}%
\pgfsetdash{}{0pt}%
\pgfpathmoveto{\pgfqpoint{0.000000in}{0.000000in}}%
\pgfpathlineto{\pgfqpoint{3.000000in}{0.000000in}}%
\pgfpathlineto{\pgfqpoint{3.000000in}{2.000000in}}%
\pgfpathlineto{\pgfqpoint{0.000000in}{2.000000in}}%
\pgfpathclose%
\pgfusepath{fill}%
\end{pgfscope}%
\begin{pgfscope}%
\pgfpathrectangle{\pgfqpoint{0.375000in}{0.200000in}}{\pgfqpoint{2.325000in}{1.600000in}} %
\pgfusepath{clip}%
\pgfsetbuttcap%
\pgfsetroundjoin%
\definecolor{currentfill}{rgb}{1.000000,0.000000,0.000000}%
\pgfsetfillcolor{currentfill}%
\pgfsetlinewidth{1.003750pt}%
\definecolor{currentstroke}{rgb}{1.000000,0.000000,0.000000}%
\pgfsetstrokecolor{currentstroke}%
\pgfsetdash{}{0pt}%
\pgfpathmoveto{\pgfqpoint{2.357708in}{1.000000in}}%
\pgfpathlineto{\pgfqpoint{2.183333in}{0.922500in}}%
\pgfpathlineto{\pgfqpoint{2.183333in}{0.982563in}}%
\pgfpathlineto{\pgfqpoint{1.537500in}{0.982563in}}%
\pgfpathlineto{\pgfqpoint{1.537500in}{1.017437in}}%
\pgfpathlineto{\pgfqpoint{2.183333in}{1.017437in}}%
\pgfpathlineto{\pgfqpoint{2.183333in}{1.077500in}}%
\pgfpathclose%
\pgfusepath{stroke,fill}%
\end{pgfscope}%
\begin{pgfscope}%
\pgfpathrectangle{\pgfqpoint{0.375000in}{0.200000in}}{\pgfqpoint{2.325000in}{1.600000in}} %
\pgfusepath{clip}%
\pgfsetrectcap%
\pgfsetroundjoin%
\pgfsetlinewidth{1.003750pt}%
\definecolor{currentstroke}{rgb}{0.000000,0.000000,1.000000}%
\pgfsetstrokecolor{currentstroke}%
\pgfsetdash{}{0pt}%
\pgfpathmoveto{\pgfqpoint{0.568750in}{0.515625in}}%
\pgfpathlineto{\pgfqpoint{2.506250in}{0.515625in}}%
\pgfusepath{stroke}%
\end{pgfscope}%
\begin{pgfscope}%
\pgfpathrectangle{\pgfqpoint{0.375000in}{0.200000in}}{\pgfqpoint{2.325000in}{1.600000in}} %
\pgfusepath{clip}%
\pgfsetrectcap%
\pgfsetroundjoin%
\pgfsetlinewidth{1.003750pt}%
\definecolor{currentstroke}{rgb}{0.000000,0.000000,0.000000}%
\pgfsetstrokecolor{currentstroke}%
\pgfsetdash{}{0pt}%
\pgfpathmoveto{\pgfqpoint{2.021875in}{1.000000in}}%
\pgfpathlineto{\pgfqpoint{2.015912in}{1.075773in}}%
\pgfpathlineto{\pgfqpoint{1.998168in}{1.149680in}}%
\pgfpathlineto{\pgfqpoint{1.969081in}{1.219902in}}%
\pgfpathlineto{\pgfqpoint{1.929368in}{1.284708in}}%
\pgfpathlineto{\pgfqpoint{1.880005in}{1.342505in}}%
\pgfpathlineto{\pgfqpoint{1.822208in}{1.391868in}}%
\pgfpathlineto{\pgfqpoint{1.757402in}{1.431581in}}%
\pgfpathlineto{\pgfqpoint{1.687180in}{1.460668in}}%
\pgfpathlineto{\pgfqpoint{1.613273in}{1.478412in}}%
\pgfpathlineto{\pgfqpoint{1.537500in}{1.484375in}}%
\pgfpathlineto{\pgfqpoint{1.461727in}{1.478412in}}%
\pgfpathlineto{\pgfqpoint{1.387820in}{1.460668in}}%
\pgfpathlineto{\pgfqpoint{1.317598in}{1.431581in}}%
\pgfpathlineto{\pgfqpoint{1.252792in}{1.391868in}}%
\pgfpathlineto{\pgfqpoint{1.194995in}{1.342505in}}%
\pgfpathlineto{\pgfqpoint{1.145632in}{1.284708in}}%
\pgfpathlineto{\pgfqpoint{1.105919in}{1.219902in}}%
\pgfpathlineto{\pgfqpoint{1.076832in}{1.149680in}}%
\pgfpathlineto{\pgfqpoint{1.059088in}{1.075773in}}%
\pgfpathlineto{\pgfqpoint{1.053125in}{1.000000in}}%
\pgfusepath{stroke}%
\end{pgfscope}%
\begin{pgfscope}%
\pgfpathrectangle{\pgfqpoint{0.375000in}{0.200000in}}{\pgfqpoint{2.325000in}{1.600000in}} %
\pgfusepath{clip}%
\pgfsetrectcap%
\pgfsetroundjoin%
\pgfsetlinewidth{1.003750pt}%
\definecolor{currentstroke}{rgb}{0.000000,0.000000,0.000000}%
\pgfsetstrokecolor{currentstroke}%
\pgfsetdash{}{0pt}%
\pgfpathmoveto{\pgfqpoint{1.053125in}{1.000000in}}%
\pgfpathlineto{\pgfqpoint{1.059088in}{0.924227in}}%
\pgfpathlineto{\pgfqpoint{1.076832in}{0.850320in}}%
\pgfpathlineto{\pgfqpoint{1.105919in}{0.780098in}}%
\pgfpathlineto{\pgfqpoint{1.145632in}{0.715292in}}%
\pgfpathlineto{\pgfqpoint{1.194995in}{0.657495in}}%
\pgfpathlineto{\pgfqpoint{1.252792in}{0.608132in}}%
\pgfpathlineto{\pgfqpoint{1.317598in}{0.568419in}}%
\pgfpathlineto{\pgfqpoint{1.387820in}{0.539332in}}%
\pgfpathlineto{\pgfqpoint{1.461727in}{0.521588in}}%
\pgfpathlineto{\pgfqpoint{1.537500in}{0.515625in}}%
\pgfpathlineto{\pgfqpoint{1.613273in}{0.521588in}}%
\pgfpathlineto{\pgfqpoint{1.687180in}{0.539332in}}%
\pgfpathlineto{\pgfqpoint{1.757402in}{0.568419in}}%
\pgfpathlineto{\pgfqpoint{1.822208in}{0.608132in}}%
\pgfpathlineto{\pgfqpoint{1.880005in}{0.657495in}}%
\pgfpathlineto{\pgfqpoint{1.929368in}{0.715292in}}%
\pgfpathlineto{\pgfqpoint{1.969081in}{0.780098in}}%
\pgfpathlineto{\pgfqpoint{1.998168in}{0.850320in}}%
\pgfpathlineto{\pgfqpoint{2.015912in}{0.924227in}}%
\pgfpathlineto{\pgfqpoint{2.021875in}{1.000000in}}%
\pgfusepath{stroke}%
\end{pgfscope}%
\begin{pgfscope}%
\pgfpathrectangle{\pgfqpoint{0.375000in}{0.200000in}}{\pgfqpoint{2.325000in}{1.600000in}} %
\pgfusepath{clip}%
\pgfsetbuttcap%
\pgfsetroundjoin%
\definecolor{currentfill}{rgb}{0.000000,0.000000,1.000000}%
\pgfsetfillcolor{currentfill}%
\pgfsetlinewidth{0.501875pt}%
\definecolor{currentstroke}{rgb}{0.000000,0.000000,0.000000}%
\pgfsetstrokecolor{currentstroke}%
\pgfsetdash{}{0pt}%
\pgfsys@defobject{currentmarker}{\pgfqpoint{-0.041667in}{-0.041667in}}{\pgfqpoint{0.041667in}{0.041667in}}{%
\pgfpathmoveto{\pgfqpoint{0.000000in}{-0.041667in}}%
\pgfpathcurveto{\pgfqpoint{0.011050in}{-0.041667in}}{\pgfqpoint{0.021649in}{-0.037276in}}{\pgfqpoint{0.029463in}{-0.029463in}}%
\pgfpathcurveto{\pgfqpoint{0.037276in}{-0.021649in}}{\pgfqpoint{0.041667in}{-0.011050in}}{\pgfqpoint{0.041667in}{0.000000in}}%
\pgfpathcurveto{\pgfqpoint{0.041667in}{0.011050in}}{\pgfqpoint{0.037276in}{0.021649in}}{\pgfqpoint{0.029463in}{0.029463in}}%
\pgfpathcurveto{\pgfqpoint{0.021649in}{0.037276in}}{\pgfqpoint{0.011050in}{0.041667in}}{\pgfqpoint{0.000000in}{0.041667in}}%
\pgfpathcurveto{\pgfqpoint{-0.011050in}{0.041667in}}{\pgfqpoint{-0.021649in}{0.037276in}}{\pgfqpoint{-0.029463in}{0.029463in}}%
\pgfpathcurveto{\pgfqpoint{-0.037276in}{0.021649in}}{\pgfqpoint{-0.041667in}{0.011050in}}{\pgfqpoint{-0.041667in}{0.000000in}}%
\pgfpathcurveto{\pgfqpoint{-0.041667in}{-0.011050in}}{\pgfqpoint{-0.037276in}{-0.021649in}}{\pgfqpoint{-0.029463in}{-0.029463in}}%
\pgfpathcurveto{\pgfqpoint{-0.021649in}{-0.037276in}}{\pgfqpoint{-0.011050in}{-0.041667in}}{\pgfqpoint{0.000000in}{-0.041667in}}%
\pgfpathclose%
\pgfusepath{stroke,fill}%
}%
\begin{pgfscope}%
\pgfsys@transformshift{1.537500in}{1.000000in}%
\pgfsys@useobject{currentmarker}{}%
\end{pgfscope}%
\end{pgfscope}%
\begin{pgfscope}%
\pgfpathrectangle{\pgfqpoint{0.375000in}{0.200000in}}{\pgfqpoint{2.325000in}{1.600000in}} %
\pgfusepath{clip}%
\pgfsetbuttcap%
\pgfsetroundjoin%
\definecolor{currentfill}{rgb}{0.000000,0.500000,0.000000}%
\pgfsetfillcolor{currentfill}%
\pgfsetlinewidth{0.501875pt}%
\definecolor{currentstroke}{rgb}{0.000000,0.000000,0.000000}%
\pgfsetstrokecolor{currentstroke}%
\pgfsetdash{}{0pt}%
\pgfsys@defobject{currentmarker}{\pgfqpoint{-0.041667in}{-0.041667in}}{\pgfqpoint{0.041667in}{0.041667in}}{%
\pgfpathmoveto{\pgfqpoint{0.000000in}{-0.041667in}}%
\pgfpathcurveto{\pgfqpoint{0.011050in}{-0.041667in}}{\pgfqpoint{0.021649in}{-0.037276in}}{\pgfqpoint{0.029463in}{-0.029463in}}%
\pgfpathcurveto{\pgfqpoint{0.037276in}{-0.021649in}}{\pgfqpoint{0.041667in}{-0.011050in}}{\pgfqpoint{0.041667in}{0.000000in}}%
\pgfpathcurveto{\pgfqpoint{0.041667in}{0.011050in}}{\pgfqpoint{0.037276in}{0.021649in}}{\pgfqpoint{0.029463in}{0.029463in}}%
\pgfpathcurveto{\pgfqpoint{0.021649in}{0.037276in}}{\pgfqpoint{0.011050in}{0.041667in}}{\pgfqpoint{0.000000in}{0.041667in}}%
\pgfpathcurveto{\pgfqpoint{-0.011050in}{0.041667in}}{\pgfqpoint{-0.021649in}{0.037276in}}{\pgfqpoint{-0.029463in}{0.029463in}}%
\pgfpathcurveto{\pgfqpoint{-0.037276in}{0.021649in}}{\pgfqpoint{-0.041667in}{0.011050in}}{\pgfqpoint{-0.041667in}{0.000000in}}%
\pgfpathcurveto{\pgfqpoint{-0.041667in}{-0.011050in}}{\pgfqpoint{-0.037276in}{-0.021649in}}{\pgfqpoint{-0.029463in}{-0.029463in}}%
\pgfpathcurveto{\pgfqpoint{-0.021649in}{-0.037276in}}{\pgfqpoint{-0.011050in}{-0.041667in}}{\pgfqpoint{0.000000in}{-0.041667in}}%
\pgfpathclose%
\pgfusepath{stroke,fill}%
}%
\begin{pgfscope}%
\pgfsys@transformshift{1.537500in}{0.515625in}%
\pgfsys@useobject{currentmarker}{}%
\end{pgfscope}%
\end{pgfscope}%
\begin{pgfscope}%
\pgftext[x=1.537500in,y=1.000000in,right,bottom]{{\sffamily\fontsize{14.000000}{16.800000}\selectfont \(\displaystyle CM\)}}%
\end{pgfscope}%
\begin{pgfscope}%
\pgftext[x=2.183333in,y=0.903125in,left,top]{{\sffamily\fontsize{14.000000}{16.800000}\selectfont \(\displaystyle F_{p}\)}}%
\end{pgfscope}%
\end{pgfpicture}%
\makeatother%
\endgroup%
\newline
Friction is still the only force that produces an angular acceleration, so
friction should have the same direction as angular acceleration. If the
object is moving to the right and rotating faster, friction acting on the
object goes towards left. If the object is moving to the left and rotation
slow down because of the external force slowing it down, the friction force
goes towards the left, to produce a torque and a angular acceleration to
slow down the rotation.\newline
%% Creator: Matplotlib, PGF backend
%%
%% To include the figure in your LaTeX document, write
%%   \input{<filename>.pgf}
%%
%% Make sure the required packages are loaded in your preamble
%%   \usepackage{pgf}
%%
%% Figures using additional raster images can only be included by \input if
%% they are in the same directory as the main LaTeX file. For loading figures
%% from other directories you can use the `import` package
%%   \usepackage{import}
%% and then include the figures with
%%   \import{<path to file>}{<filename>.pgf}
%%
%% Matplotlib used the following preamble
%%   \usepackage{fontspec}
%%   \setmainfont{Times New Roman}
%%   \setsansfont{Verdana}
%%   \setmonofont{Courier New}
%%
\begingroup%
\makeatletter%
\begin{pgfpicture}%
\pgfpathrectangle{\pgfpointorigin}{\pgfqpoint{4.000000in}{3.970000in}}%
\pgfusepath{use as bounding box}%
\begin{pgfscope}%
\pgfsetbuttcap%
\pgfsetroundjoin%
\definecolor{currentfill}{rgb}{1.000000,1.000000,1.000000}%
\pgfsetfillcolor{currentfill}%
\pgfsetlinewidth{0.000000pt}%
\definecolor{currentstroke}{rgb}{1.000000,1.000000,1.000000}%
\pgfsetstrokecolor{currentstroke}%
\pgfsetdash{}{0pt}%
\pgfpathmoveto{\pgfqpoint{0.000000in}{0.000000in}}%
\pgfpathlineto{\pgfqpoint{4.000000in}{0.000000in}}%
\pgfpathlineto{\pgfqpoint{4.000000in}{3.970000in}}%
\pgfpathlineto{\pgfqpoint{0.000000in}{3.970000in}}%
\pgfpathclose%
\pgfusepath{fill}%
\end{pgfscope}%
\begin{pgfscope}%
\pgfpathrectangle{\pgfqpoint{0.500000in}{0.397000in}}{\pgfqpoint{3.100000in}{3.176000in}} %
\pgfusepath{clip}%
\pgfsetbuttcap%
\pgfsetroundjoin%
\definecolor{currentfill}{rgb}{0.000000,0.000000,1.000000}%
\pgfsetfillcolor{currentfill}%
\pgfsetlinewidth{1.003750pt}%
\definecolor{currentstroke}{rgb}{0.000000,0.000000,0.000000}%
\pgfsetstrokecolor{currentstroke}%
\pgfsetdash{}{0pt}%
\pgfpathmoveto{\pgfqpoint{1.171667in}{1.339167in}}%
\pgfpathlineto{\pgfqpoint{1.404167in}{1.442500in}}%
\pgfpathlineto{\pgfqpoint{1.404167in}{1.362417in}}%
\pgfpathlineto{\pgfqpoint{2.050000in}{1.362417in}}%
\pgfpathlineto{\pgfqpoint{2.050000in}{1.315917in}}%
\pgfpathlineto{\pgfqpoint{1.404167in}{1.315917in}}%
\pgfpathlineto{\pgfqpoint{1.404167in}{1.235833in}}%
\pgfpathclose%
\pgfusepath{stroke,fill}%
\end{pgfscope}%
\begin{pgfscope}%
\pgfpathrectangle{\pgfqpoint{0.500000in}{0.397000in}}{\pgfqpoint{3.100000in}{3.176000in}} %
\pgfusepath{clip}%
\pgfsetbuttcap%
\pgfsetroundjoin%
\definecolor{currentfill}{rgb}{1.000000,0.000000,0.000000}%
\pgfsetfillcolor{currentfill}%
\pgfsetlinewidth{1.003750pt}%
\definecolor{currentstroke}{rgb}{1.000000,0.000000,0.000000}%
\pgfsetstrokecolor{currentstroke}%
\pgfsetdash{}{0pt}%
\pgfpathmoveto{\pgfqpoint{3.143611in}{1.985000in}}%
\pgfpathlineto{\pgfqpoint{2.911111in}{1.881667in}}%
\pgfpathlineto{\pgfqpoint{2.911111in}{1.961750in}}%
\pgfpathlineto{\pgfqpoint{2.050000in}{1.961750in}}%
\pgfpathlineto{\pgfqpoint{2.050000in}{2.008250in}}%
\pgfpathlineto{\pgfqpoint{2.911111in}{2.008250in}}%
\pgfpathlineto{\pgfqpoint{2.911111in}{2.088333in}}%
\pgfpathclose%
\pgfusepath{stroke,fill}%
\end{pgfscope}%
\begin{pgfscope}%
\pgfpathrectangle{\pgfqpoint{0.500000in}{0.397000in}}{\pgfqpoint{3.100000in}{3.176000in}} %
\pgfusepath{clip}%
\pgfsetbuttcap%
\pgfsetroundjoin%
\definecolor{currentfill}{rgb}{0.000000,0.000000,0.000000}%
\pgfsetfillcolor{currentfill}%
\pgfsetlinewidth{1.003750pt}%
\definecolor{currentstroke}{rgb}{0.000000,0.000000,0.000000}%
\pgfsetstrokecolor{currentstroke}%
\pgfsetdash{}{0pt}%
\pgfpathmoveto{\pgfqpoint{2.669032in}{2.655354in}}%
\pgfpathlineto{\pgfqpoint{2.568506in}{2.646888in}}%
\pgfpathlineto{\pgfqpoint{2.604801in}{2.698739in}}%
\pgfpathlineto{\pgfqpoint{2.567765in}{2.724664in}}%
\pgfpathlineto{\pgfqpoint{2.569246in}{2.726780in}}%
\pgfpathlineto{\pgfqpoint{2.606282in}{2.700855in}}%
\pgfpathlineto{\pgfqpoint{2.642578in}{2.752706in}}%
\pgfpathclose%
\pgfusepath{stroke,fill}%
\end{pgfscope}%
\begin{pgfscope}%
\pgfpathrectangle{\pgfqpoint{0.500000in}{0.397000in}}{\pgfqpoint{3.100000in}{3.176000in}} %
\pgfusepath{clip}%
\pgfsetbuttcap%
\pgfsetroundjoin%
\definecolor{currentfill}{rgb}{0.000000,0.000000,1.000000}%
\pgfsetfillcolor{currentfill}%
\pgfsetlinewidth{1.003750pt}%
\definecolor{currentstroke}{rgb}{0.000000,0.000000,0.000000}%
\pgfsetstrokecolor{currentstroke}%
\pgfsetdash{}{0pt}%
\pgfpathmoveto{\pgfqpoint{1.171667in}{1.339167in}}%
\pgfpathlineto{\pgfqpoint{1.404167in}{1.442500in}}%
\pgfpathlineto{\pgfqpoint{1.404167in}{1.362417in}}%
\pgfpathlineto{\pgfqpoint{2.050000in}{1.362417in}}%
\pgfpathlineto{\pgfqpoint{2.050000in}{1.315917in}}%
\pgfpathlineto{\pgfqpoint{1.404167in}{1.315917in}}%
\pgfpathlineto{\pgfqpoint{1.404167in}{1.235833in}}%
\pgfpathclose%
\pgfusepath{stroke,fill}%
\end{pgfscope}%
\begin{pgfscope}%
\pgfpathrectangle{\pgfqpoint{0.500000in}{0.397000in}}{\pgfqpoint{3.100000in}{3.176000in}} %
\pgfusepath{clip}%
\pgfsetbuttcap%
\pgfsetroundjoin%
\definecolor{currentfill}{rgb}{1.000000,0.000000,0.000000}%
\pgfsetfillcolor{currentfill}%
\pgfsetlinewidth{1.003750pt}%
\definecolor{currentstroke}{rgb}{1.000000,0.000000,0.000000}%
\pgfsetstrokecolor{currentstroke}%
\pgfsetdash{}{0pt}%
\pgfpathmoveto{\pgfqpoint{3.143611in}{1.985000in}}%
\pgfpathlineto{\pgfqpoint{2.911111in}{1.881667in}}%
\pgfpathlineto{\pgfqpoint{2.911111in}{1.961750in}}%
\pgfpathlineto{\pgfqpoint{2.050000in}{1.961750in}}%
\pgfpathlineto{\pgfqpoint{2.050000in}{2.008250in}}%
\pgfpathlineto{\pgfqpoint{2.911111in}{2.008250in}}%
\pgfpathlineto{\pgfqpoint{2.911111in}{2.088333in}}%
\pgfpathclose%
\pgfusepath{stroke,fill}%
\end{pgfscope}%
\begin{pgfscope}%
\pgfpathrectangle{\pgfqpoint{0.500000in}{0.397000in}}{\pgfqpoint{3.100000in}{3.176000in}} %
\pgfusepath{clip}%
\pgfsetbuttcap%
\pgfsetroundjoin%
\definecolor{currentfill}{rgb}{0.000000,0.000000,0.000000}%
\pgfsetfillcolor{currentfill}%
\pgfsetlinewidth{1.003750pt}%
\definecolor{currentstroke}{rgb}{0.000000,0.000000,0.000000}%
\pgfsetstrokecolor{currentstroke}%
\pgfsetdash{}{0pt}%
\pgfpathmoveto{\pgfqpoint{2.669032in}{2.655354in}}%
\pgfpathlineto{\pgfqpoint{2.568506in}{2.646888in}}%
\pgfpathlineto{\pgfqpoint{2.604801in}{2.698739in}}%
\pgfpathlineto{\pgfqpoint{2.567765in}{2.724664in}}%
\pgfpathlineto{\pgfqpoint{2.569246in}{2.726780in}}%
\pgfpathlineto{\pgfqpoint{2.606282in}{2.700855in}}%
\pgfpathlineto{\pgfqpoint{2.642578in}{2.752706in}}%
\pgfpathclose%
\pgfusepath{stroke,fill}%
\end{pgfscope}%
\begin{pgfscope}%
\pgfpathrectangle{\pgfqpoint{0.500000in}{0.397000in}}{\pgfqpoint{3.100000in}{3.176000in}} %
\pgfusepath{clip}%
\pgfsetrectcap%
\pgfsetroundjoin%
\pgfsetlinewidth{1.003750pt}%
\definecolor{currentstroke}{rgb}{0.000000,0.000000,1.000000}%
\pgfsetstrokecolor{currentstroke}%
\pgfsetdash{}{0pt}%
\pgfpathmoveto{\pgfqpoint{0.758333in}{1.339167in}}%
\pgfpathlineto{\pgfqpoint{3.341667in}{1.339167in}}%
\pgfusepath{stroke}%
\end{pgfscope}%
\begin{pgfscope}%
\pgfpathrectangle{\pgfqpoint{0.500000in}{0.397000in}}{\pgfqpoint{3.100000in}{3.176000in}} %
\pgfusepath{clip}%
\pgfsetrectcap%
\pgfsetroundjoin%
\pgfsetlinewidth{1.003750pt}%
\definecolor{currentstroke}{rgb}{0.000000,0.000000,0.000000}%
\pgfsetstrokecolor{currentstroke}%
\pgfsetdash{}{0pt}%
\pgfpathmoveto{\pgfqpoint{2.050000in}{2.889167in}}%
\pgfpathlineto{\pgfqpoint{2.105185in}{2.887481in}}%
\pgfpathlineto{\pgfqpoint{2.160165in}{2.882430in}}%
\pgfpathlineto{\pgfqpoint{2.214734in}{2.874033in}}%
\pgfpathlineto{\pgfqpoint{2.268689in}{2.862321in}}%
\pgfpathlineto{\pgfqpoint{2.321828in}{2.847338in}}%
\pgfpathlineto{\pgfqpoint{2.373954in}{2.829139in}}%
\pgfpathlineto{\pgfqpoint{2.424872in}{2.807793in}}%
\pgfpathlineto{\pgfqpoint{2.474392in}{2.783379in}}%
\pgfpathlineto{\pgfqpoint{2.522329in}{2.755988in}}%
\pgfpathlineto{\pgfqpoint{2.568506in}{2.725722in}}%
\pgfusepath{stroke}%
\end{pgfscope}%
\begin{pgfscope}%
\pgfpathrectangle{\pgfqpoint{0.500000in}{0.397000in}}{\pgfqpoint{3.100000in}{3.176000in}} %
\pgfusepath{clip}%
\pgfsetrectcap%
\pgfsetroundjoin%
\pgfsetlinewidth{1.003750pt}%
\definecolor{currentstroke}{rgb}{0.000000,0.000000,0.000000}%
\pgfsetstrokecolor{currentstroke}%
\pgfsetdash{}{0pt}%
\pgfpathmoveto{\pgfqpoint{2.695833in}{1.985000in}}%
\pgfpathlineto{\pgfqpoint{2.687882in}{2.086031in}}%
\pgfpathlineto{\pgfqpoint{2.664224in}{2.184573in}}%
\pgfpathlineto{\pgfqpoint{2.625442in}{2.278202in}}%
\pgfpathlineto{\pgfqpoint{2.572490in}{2.364611in}}%
\pgfpathlineto{\pgfqpoint{2.506673in}{2.441673in}}%
\pgfpathlineto{\pgfqpoint{2.429611in}{2.507490in}}%
\pgfpathlineto{\pgfqpoint{2.343202in}{2.560442in}}%
\pgfpathlineto{\pgfqpoint{2.249573in}{2.599224in}}%
\pgfpathlineto{\pgfqpoint{2.151031in}{2.622882in}}%
\pgfpathlineto{\pgfqpoint{2.050000in}{2.630833in}}%
\pgfpathlineto{\pgfqpoint{1.948969in}{2.622882in}}%
\pgfpathlineto{\pgfqpoint{1.850427in}{2.599224in}}%
\pgfpathlineto{\pgfqpoint{1.756798in}{2.560442in}}%
\pgfpathlineto{\pgfqpoint{1.670389in}{2.507490in}}%
\pgfpathlineto{\pgfqpoint{1.593327in}{2.441673in}}%
\pgfpathlineto{\pgfqpoint{1.527510in}{2.364611in}}%
\pgfpathlineto{\pgfqpoint{1.474558in}{2.278202in}}%
\pgfpathlineto{\pgfqpoint{1.435776in}{2.184573in}}%
\pgfpathlineto{\pgfqpoint{1.412118in}{2.086031in}}%
\pgfpathlineto{\pgfqpoint{1.404167in}{1.985000in}}%
\pgfusepath{stroke}%
\end{pgfscope}%
\begin{pgfscope}%
\pgfpathrectangle{\pgfqpoint{0.500000in}{0.397000in}}{\pgfqpoint{3.100000in}{3.176000in}} %
\pgfusepath{clip}%
\pgfsetrectcap%
\pgfsetroundjoin%
\pgfsetlinewidth{1.003750pt}%
\definecolor{currentstroke}{rgb}{0.000000,0.000000,0.000000}%
\pgfsetstrokecolor{currentstroke}%
\pgfsetdash{}{0pt}%
\pgfpathmoveto{\pgfqpoint{1.404167in}{1.985000in}}%
\pgfpathlineto{\pgfqpoint{1.412118in}{1.883969in}}%
\pgfpathlineto{\pgfqpoint{1.435776in}{1.785427in}}%
\pgfpathlineto{\pgfqpoint{1.474558in}{1.691798in}}%
\pgfpathlineto{\pgfqpoint{1.527510in}{1.605389in}}%
\pgfpathlineto{\pgfqpoint{1.593327in}{1.528327in}}%
\pgfpathlineto{\pgfqpoint{1.670389in}{1.462510in}}%
\pgfpathlineto{\pgfqpoint{1.756798in}{1.409558in}}%
\pgfpathlineto{\pgfqpoint{1.850427in}{1.370776in}}%
\pgfpathlineto{\pgfqpoint{1.948969in}{1.347118in}}%
\pgfpathlineto{\pgfqpoint{2.050000in}{1.339167in}}%
\pgfpathlineto{\pgfqpoint{2.151031in}{1.347118in}}%
\pgfpathlineto{\pgfqpoint{2.249573in}{1.370776in}}%
\pgfpathlineto{\pgfqpoint{2.343202in}{1.409558in}}%
\pgfpathlineto{\pgfqpoint{2.429611in}{1.462510in}}%
\pgfpathlineto{\pgfqpoint{2.506673in}{1.528327in}}%
\pgfpathlineto{\pgfqpoint{2.572490in}{1.605389in}}%
\pgfpathlineto{\pgfqpoint{2.625442in}{1.691798in}}%
\pgfpathlineto{\pgfqpoint{2.664224in}{1.785427in}}%
\pgfpathlineto{\pgfqpoint{2.687882in}{1.883969in}}%
\pgfpathlineto{\pgfqpoint{2.695833in}{1.985000in}}%
\pgfusepath{stroke}%
\end{pgfscope}%
\begin{pgfscope}%
\pgfpathrectangle{\pgfqpoint{0.500000in}{0.397000in}}{\pgfqpoint{3.100000in}{3.176000in}} %
\pgfusepath{clip}%
\pgfsetbuttcap%
\pgfsetroundjoin%
\definecolor{currentfill}{rgb}{0.000000,0.000000,1.000000}%
\pgfsetfillcolor{currentfill}%
\pgfsetlinewidth{0.501875pt}%
\definecolor{currentstroke}{rgb}{0.000000,0.000000,0.000000}%
\pgfsetstrokecolor{currentstroke}%
\pgfsetdash{}{0pt}%
\pgfsys@defobject{currentmarker}{\pgfqpoint{-0.041667in}{-0.041667in}}{\pgfqpoint{0.041667in}{0.041667in}}{%
\pgfpathmoveto{\pgfqpoint{0.000000in}{-0.041667in}}%
\pgfpathcurveto{\pgfqpoint{0.011050in}{-0.041667in}}{\pgfqpoint{0.021649in}{-0.037276in}}{\pgfqpoint{0.029463in}{-0.029463in}}%
\pgfpathcurveto{\pgfqpoint{0.037276in}{-0.021649in}}{\pgfqpoint{0.041667in}{-0.011050in}}{\pgfqpoint{0.041667in}{0.000000in}}%
\pgfpathcurveto{\pgfqpoint{0.041667in}{0.011050in}}{\pgfqpoint{0.037276in}{0.021649in}}{\pgfqpoint{0.029463in}{0.029463in}}%
\pgfpathcurveto{\pgfqpoint{0.021649in}{0.037276in}}{\pgfqpoint{0.011050in}{0.041667in}}{\pgfqpoint{0.000000in}{0.041667in}}%
\pgfpathcurveto{\pgfqpoint{-0.011050in}{0.041667in}}{\pgfqpoint{-0.021649in}{0.037276in}}{\pgfqpoint{-0.029463in}{0.029463in}}%
\pgfpathcurveto{\pgfqpoint{-0.037276in}{0.021649in}}{\pgfqpoint{-0.041667in}{0.011050in}}{\pgfqpoint{-0.041667in}{0.000000in}}%
\pgfpathcurveto{\pgfqpoint{-0.041667in}{-0.011050in}}{\pgfqpoint{-0.037276in}{-0.021649in}}{\pgfqpoint{-0.029463in}{-0.029463in}}%
\pgfpathcurveto{\pgfqpoint{-0.021649in}{-0.037276in}}{\pgfqpoint{-0.011050in}{-0.041667in}}{\pgfqpoint{0.000000in}{-0.041667in}}%
\pgfpathclose%
\pgfusepath{stroke,fill}%
}%
\begin{pgfscope}%
\pgfsys@transformshift{2.050000in}{1.985000in}%
\pgfsys@useobject{currentmarker}{}%
\end{pgfscope}%
\end{pgfscope}%
\begin{pgfscope}%
\pgfpathrectangle{\pgfqpoint{0.500000in}{0.397000in}}{\pgfqpoint{3.100000in}{3.176000in}} %
\pgfusepath{clip}%
\pgfsetbuttcap%
\pgfsetroundjoin%
\definecolor{currentfill}{rgb}{0.000000,0.500000,0.000000}%
\pgfsetfillcolor{currentfill}%
\pgfsetlinewidth{0.501875pt}%
\definecolor{currentstroke}{rgb}{0.000000,0.000000,0.000000}%
\pgfsetstrokecolor{currentstroke}%
\pgfsetdash{}{0pt}%
\pgfsys@defobject{currentmarker}{\pgfqpoint{-0.041667in}{-0.041667in}}{\pgfqpoint{0.041667in}{0.041667in}}{%
\pgfpathmoveto{\pgfqpoint{0.000000in}{-0.041667in}}%
\pgfpathcurveto{\pgfqpoint{0.011050in}{-0.041667in}}{\pgfqpoint{0.021649in}{-0.037276in}}{\pgfqpoint{0.029463in}{-0.029463in}}%
\pgfpathcurveto{\pgfqpoint{0.037276in}{-0.021649in}}{\pgfqpoint{0.041667in}{-0.011050in}}{\pgfqpoint{0.041667in}{0.000000in}}%
\pgfpathcurveto{\pgfqpoint{0.041667in}{0.011050in}}{\pgfqpoint{0.037276in}{0.021649in}}{\pgfqpoint{0.029463in}{0.029463in}}%
\pgfpathcurveto{\pgfqpoint{0.021649in}{0.037276in}}{\pgfqpoint{0.011050in}{0.041667in}}{\pgfqpoint{0.000000in}{0.041667in}}%
\pgfpathcurveto{\pgfqpoint{-0.011050in}{0.041667in}}{\pgfqpoint{-0.021649in}{0.037276in}}{\pgfqpoint{-0.029463in}{0.029463in}}%
\pgfpathcurveto{\pgfqpoint{-0.037276in}{0.021649in}}{\pgfqpoint{-0.041667in}{0.011050in}}{\pgfqpoint{-0.041667in}{0.000000in}}%
\pgfpathcurveto{\pgfqpoint{-0.041667in}{-0.011050in}}{\pgfqpoint{-0.037276in}{-0.021649in}}{\pgfqpoint{-0.029463in}{-0.029463in}}%
\pgfpathcurveto{\pgfqpoint{-0.021649in}{-0.037276in}}{\pgfqpoint{-0.011050in}{-0.041667in}}{\pgfqpoint{0.000000in}{-0.041667in}}%
\pgfpathclose%
\pgfusepath{stroke,fill}%
}%
\begin{pgfscope}%
\pgfsys@transformshift{2.050000in}{1.339167in}%
\pgfsys@useobject{currentmarker}{}%
\end{pgfscope}%
\end{pgfscope}%
\begin{pgfscope}%
\pgfpathrectangle{\pgfqpoint{0.500000in}{0.397000in}}{\pgfqpoint{3.100000in}{3.176000in}} %
\pgfusepath{clip}%
\pgfsetrectcap%
\pgfsetroundjoin%
\pgfsetlinewidth{1.003750pt}%
\definecolor{currentstroke}{rgb}{0.000000,0.000000,1.000000}%
\pgfsetstrokecolor{currentstroke}%
\pgfsetdash{}{0pt}%
\pgfpathmoveto{\pgfqpoint{0.758333in}{1.339167in}}%
\pgfpathlineto{\pgfqpoint{3.341667in}{1.339167in}}%
\pgfusepath{stroke}%
\end{pgfscope}%
\begin{pgfscope}%
\pgfpathrectangle{\pgfqpoint{0.500000in}{0.397000in}}{\pgfqpoint{3.100000in}{3.176000in}} %
\pgfusepath{clip}%
\pgfsetrectcap%
\pgfsetroundjoin%
\pgfsetlinewidth{1.003750pt}%
\definecolor{currentstroke}{rgb}{0.000000,0.000000,0.000000}%
\pgfsetstrokecolor{currentstroke}%
\pgfsetdash{}{0pt}%
\pgfpathmoveto{\pgfqpoint{2.050000in}{2.889167in}}%
\pgfpathlineto{\pgfqpoint{2.105185in}{2.887481in}}%
\pgfpathlineto{\pgfqpoint{2.160165in}{2.882430in}}%
\pgfpathlineto{\pgfqpoint{2.214734in}{2.874033in}}%
\pgfpathlineto{\pgfqpoint{2.268689in}{2.862321in}}%
\pgfpathlineto{\pgfqpoint{2.321828in}{2.847338in}}%
\pgfpathlineto{\pgfqpoint{2.373954in}{2.829139in}}%
\pgfpathlineto{\pgfqpoint{2.424872in}{2.807793in}}%
\pgfpathlineto{\pgfqpoint{2.474392in}{2.783379in}}%
\pgfpathlineto{\pgfqpoint{2.522329in}{2.755988in}}%
\pgfpathlineto{\pgfqpoint{2.568506in}{2.725722in}}%
\pgfusepath{stroke}%
\end{pgfscope}%
\begin{pgfscope}%
\pgfpathrectangle{\pgfqpoint{0.500000in}{0.397000in}}{\pgfqpoint{3.100000in}{3.176000in}} %
\pgfusepath{clip}%
\pgfsetrectcap%
\pgfsetroundjoin%
\pgfsetlinewidth{1.003750pt}%
\definecolor{currentstroke}{rgb}{0.000000,0.000000,0.000000}%
\pgfsetstrokecolor{currentstroke}%
\pgfsetdash{}{0pt}%
\pgfpathmoveto{\pgfqpoint{2.695833in}{1.985000in}}%
\pgfpathlineto{\pgfqpoint{2.687882in}{2.086031in}}%
\pgfpathlineto{\pgfqpoint{2.664224in}{2.184573in}}%
\pgfpathlineto{\pgfqpoint{2.625442in}{2.278202in}}%
\pgfpathlineto{\pgfqpoint{2.572490in}{2.364611in}}%
\pgfpathlineto{\pgfqpoint{2.506673in}{2.441673in}}%
\pgfpathlineto{\pgfqpoint{2.429611in}{2.507490in}}%
\pgfpathlineto{\pgfqpoint{2.343202in}{2.560442in}}%
\pgfpathlineto{\pgfqpoint{2.249573in}{2.599224in}}%
\pgfpathlineto{\pgfqpoint{2.151031in}{2.622882in}}%
\pgfpathlineto{\pgfqpoint{2.050000in}{2.630833in}}%
\pgfpathlineto{\pgfqpoint{1.948969in}{2.622882in}}%
\pgfpathlineto{\pgfqpoint{1.850427in}{2.599224in}}%
\pgfpathlineto{\pgfqpoint{1.756798in}{2.560442in}}%
\pgfpathlineto{\pgfqpoint{1.670389in}{2.507490in}}%
\pgfpathlineto{\pgfqpoint{1.593327in}{2.441673in}}%
\pgfpathlineto{\pgfqpoint{1.527510in}{2.364611in}}%
\pgfpathlineto{\pgfqpoint{1.474558in}{2.278202in}}%
\pgfpathlineto{\pgfqpoint{1.435776in}{2.184573in}}%
\pgfpathlineto{\pgfqpoint{1.412118in}{2.086031in}}%
\pgfpathlineto{\pgfqpoint{1.404167in}{1.985000in}}%
\pgfusepath{stroke}%
\end{pgfscope}%
\begin{pgfscope}%
\pgfpathrectangle{\pgfqpoint{0.500000in}{0.397000in}}{\pgfqpoint{3.100000in}{3.176000in}} %
\pgfusepath{clip}%
\pgfsetrectcap%
\pgfsetroundjoin%
\pgfsetlinewidth{1.003750pt}%
\definecolor{currentstroke}{rgb}{0.000000,0.000000,0.000000}%
\pgfsetstrokecolor{currentstroke}%
\pgfsetdash{}{0pt}%
\pgfpathmoveto{\pgfqpoint{1.404167in}{1.985000in}}%
\pgfpathlineto{\pgfqpoint{1.412118in}{1.883969in}}%
\pgfpathlineto{\pgfqpoint{1.435776in}{1.785427in}}%
\pgfpathlineto{\pgfqpoint{1.474558in}{1.691798in}}%
\pgfpathlineto{\pgfqpoint{1.527510in}{1.605389in}}%
\pgfpathlineto{\pgfqpoint{1.593327in}{1.528327in}}%
\pgfpathlineto{\pgfqpoint{1.670389in}{1.462510in}}%
\pgfpathlineto{\pgfqpoint{1.756798in}{1.409558in}}%
\pgfpathlineto{\pgfqpoint{1.850427in}{1.370776in}}%
\pgfpathlineto{\pgfqpoint{1.948969in}{1.347118in}}%
\pgfpathlineto{\pgfqpoint{2.050000in}{1.339167in}}%
\pgfpathlineto{\pgfqpoint{2.151031in}{1.347118in}}%
\pgfpathlineto{\pgfqpoint{2.249573in}{1.370776in}}%
\pgfpathlineto{\pgfqpoint{2.343202in}{1.409558in}}%
\pgfpathlineto{\pgfqpoint{2.429611in}{1.462510in}}%
\pgfpathlineto{\pgfqpoint{2.506673in}{1.528327in}}%
\pgfpathlineto{\pgfqpoint{2.572490in}{1.605389in}}%
\pgfpathlineto{\pgfqpoint{2.625442in}{1.691798in}}%
\pgfpathlineto{\pgfqpoint{2.664224in}{1.785427in}}%
\pgfpathlineto{\pgfqpoint{2.687882in}{1.883969in}}%
\pgfpathlineto{\pgfqpoint{2.695833in}{1.985000in}}%
\pgfusepath{stroke}%
\end{pgfscope}%
\begin{pgfscope}%
\pgfpathrectangle{\pgfqpoint{0.500000in}{0.397000in}}{\pgfqpoint{3.100000in}{3.176000in}} %
\pgfusepath{clip}%
\pgfsetbuttcap%
\pgfsetroundjoin%
\definecolor{currentfill}{rgb}{1.000000,0.000000,0.000000}%
\pgfsetfillcolor{currentfill}%
\pgfsetlinewidth{0.501875pt}%
\definecolor{currentstroke}{rgb}{0.000000,0.000000,0.000000}%
\pgfsetstrokecolor{currentstroke}%
\pgfsetdash{}{0pt}%
\pgfsys@defobject{currentmarker}{\pgfqpoint{-0.041667in}{-0.041667in}}{\pgfqpoint{0.041667in}{0.041667in}}{%
\pgfpathmoveto{\pgfqpoint{0.000000in}{-0.041667in}}%
\pgfpathcurveto{\pgfqpoint{0.011050in}{-0.041667in}}{\pgfqpoint{0.021649in}{-0.037276in}}{\pgfqpoint{0.029463in}{-0.029463in}}%
\pgfpathcurveto{\pgfqpoint{0.037276in}{-0.021649in}}{\pgfqpoint{0.041667in}{-0.011050in}}{\pgfqpoint{0.041667in}{0.000000in}}%
\pgfpathcurveto{\pgfqpoint{0.041667in}{0.011050in}}{\pgfqpoint{0.037276in}{0.021649in}}{\pgfqpoint{0.029463in}{0.029463in}}%
\pgfpathcurveto{\pgfqpoint{0.021649in}{0.037276in}}{\pgfqpoint{0.011050in}{0.041667in}}{\pgfqpoint{0.000000in}{0.041667in}}%
\pgfpathcurveto{\pgfqpoint{-0.011050in}{0.041667in}}{\pgfqpoint{-0.021649in}{0.037276in}}{\pgfqpoint{-0.029463in}{0.029463in}}%
\pgfpathcurveto{\pgfqpoint{-0.037276in}{0.021649in}}{\pgfqpoint{-0.041667in}{0.011050in}}{\pgfqpoint{-0.041667in}{0.000000in}}%
\pgfpathcurveto{\pgfqpoint{-0.041667in}{-0.011050in}}{\pgfqpoint{-0.037276in}{-0.021649in}}{\pgfqpoint{-0.029463in}{-0.029463in}}%
\pgfpathcurveto{\pgfqpoint{-0.021649in}{-0.037276in}}{\pgfqpoint{-0.011050in}{-0.041667in}}{\pgfqpoint{0.000000in}{-0.041667in}}%
\pgfpathclose%
\pgfusepath{stroke,fill}%
}%
\begin{pgfscope}%
\pgfsys@transformshift{2.050000in}{1.985000in}%
\pgfsys@useobject{currentmarker}{}%
\end{pgfscope}%
\end{pgfscope}%
\begin{pgfscope}%
\pgfpathrectangle{\pgfqpoint{0.500000in}{0.397000in}}{\pgfqpoint{3.100000in}{3.176000in}} %
\pgfusepath{clip}%
\pgfsetbuttcap%
\pgfsetroundjoin%
\definecolor{currentfill}{rgb}{0.000000,0.750000,0.750000}%
\pgfsetfillcolor{currentfill}%
\pgfsetlinewidth{0.501875pt}%
\definecolor{currentstroke}{rgb}{0.000000,0.000000,0.000000}%
\pgfsetstrokecolor{currentstroke}%
\pgfsetdash{}{0pt}%
\pgfsys@defobject{currentmarker}{\pgfqpoint{-0.041667in}{-0.041667in}}{\pgfqpoint{0.041667in}{0.041667in}}{%
\pgfpathmoveto{\pgfqpoint{0.000000in}{-0.041667in}}%
\pgfpathcurveto{\pgfqpoint{0.011050in}{-0.041667in}}{\pgfqpoint{0.021649in}{-0.037276in}}{\pgfqpoint{0.029463in}{-0.029463in}}%
\pgfpathcurveto{\pgfqpoint{0.037276in}{-0.021649in}}{\pgfqpoint{0.041667in}{-0.011050in}}{\pgfqpoint{0.041667in}{0.000000in}}%
\pgfpathcurveto{\pgfqpoint{0.041667in}{0.011050in}}{\pgfqpoint{0.037276in}{0.021649in}}{\pgfqpoint{0.029463in}{0.029463in}}%
\pgfpathcurveto{\pgfqpoint{0.021649in}{0.037276in}}{\pgfqpoint{0.011050in}{0.041667in}}{\pgfqpoint{0.000000in}{0.041667in}}%
\pgfpathcurveto{\pgfqpoint{-0.011050in}{0.041667in}}{\pgfqpoint{-0.021649in}{0.037276in}}{\pgfqpoint{-0.029463in}{0.029463in}}%
\pgfpathcurveto{\pgfqpoint{-0.037276in}{0.021649in}}{\pgfqpoint{-0.041667in}{0.011050in}}{\pgfqpoint{-0.041667in}{0.000000in}}%
\pgfpathcurveto{\pgfqpoint{-0.041667in}{-0.011050in}}{\pgfqpoint{-0.037276in}{-0.021649in}}{\pgfqpoint{-0.029463in}{-0.029463in}}%
\pgfpathcurveto{\pgfqpoint{-0.021649in}{-0.037276in}}{\pgfqpoint{-0.011050in}{-0.041667in}}{\pgfqpoint{0.000000in}{-0.041667in}}%
\pgfpathclose%
\pgfusepath{stroke,fill}%
}%
\begin{pgfscope}%
\pgfsys@transformshift{2.050000in}{1.339167in}%
\pgfsys@useobject{currentmarker}{}%
\end{pgfscope}%
\end{pgfscope}%
\begin{pgfscope}%
\pgftext[x=2.308333in,y=2.889167in,left,bottom]{{\sffamily\fontsize{20.000000}{24.000000}\selectfont \(\displaystyle \alpha\)}}%
\end{pgfscope}%
\begin{pgfscope}%
\pgftext[x=2.050000in,y=1.985000in,right,bottom]{{\sffamily\fontsize{20.000000}{24.000000}\selectfont \(\displaystyle CM\)}}%
\end{pgfscope}%
\begin{pgfscope}%
\pgftext[x=1.404167in,y=1.210000in,left,top]{{\sffamily\fontsize{20.000000}{24.000000}\selectfont \(\displaystyle F_{fr}\)}}%
\end{pgfscope}%
\begin{pgfscope}%
\pgftext[x=2.911111in,y=1.855833in,left,top]{{\sffamily\fontsize{20.000000}{24.000000}\selectfont \(\displaystyle F_{p}\)}}%
\end{pgfscope}%
\begin{pgfscope}%
\pgftext[x=2.308333in,y=2.889167in,left,bottom]{{\sffamily\fontsize{20.000000}{24.000000}\selectfont \(\displaystyle \alpha\)}}%
\end{pgfscope}%
\begin{pgfscope}%
\pgftext[x=2.050000in,y=1.985000in,right,bottom]{{\sffamily\fontsize{20.000000}{24.000000}\selectfont \(\displaystyle CM\)}}%
\end{pgfscope}%
\begin{pgfscope}%
\pgftext[x=1.404167in,y=1.210000in,left,top]{{\sffamily\fontsize{20.000000}{24.000000}\selectfont \(\displaystyle F_{fr}\)}}%
\end{pgfscope}%
\begin{pgfscope}%
\pgftext[x=2.911111in,y=1.855833in,left,top]{{\sffamily\fontsize{20.000000}{24.000000}\selectfont \(\displaystyle F_{p}\)}}%
\end{pgfscope}%
\end{pgfpicture}%
\makeatother%
\endgroup%
%
\[
\left\{ 
\begin{array}{c}
F_{p}-F_{fr}=ma \\ 
F_{fr}\cdot R=I\alpha =I\frac{a}{R}%
\end{array}%
\right. 
\]
\end{case}

\newpage

\begin{case}
An external torque (not force) acting around the axial, like on car tire.%
\newline
%% Creator: Matplotlib, PGF backend
%%
%% To include the figure in your LaTeX document, write
%%   \input{<filename>.pgf}
%%
%% Make sure the required packages are loaded in your preamble
%%   \usepackage{pgf}
%%
%% Figures using additional raster images can only be included by \input if
%% they are in the same directory as the main LaTeX file. For loading figures
%% from other directories you can use the `import` package
%%   \usepackage{import}
%% and then include the figures with
%%   \import{<path to file>}{<filename>.pgf}
%%
%% Matplotlib used the following preamble
%%   \usepackage{fontspec}
%%   \setmainfont{Times New Roman}
%%   \setsansfont{Verdana}
%%   \setmonofont{Courier New}
%%
\begingroup%
\makeatletter%
\begin{pgfpicture}%
\pgfpathrectangle{\pgfpointorigin}{\pgfqpoint{3.000000in}{1.500000in}}%
\pgfusepath{use as bounding box}%
\begin{pgfscope}%
\pgfsetbuttcap%
\pgfsetroundjoin%
\definecolor{currentfill}{rgb}{1.000000,1.000000,1.000000}%
\pgfsetfillcolor{currentfill}%
\pgfsetlinewidth{0.000000pt}%
\definecolor{currentstroke}{rgb}{1.000000,1.000000,1.000000}%
\pgfsetstrokecolor{currentstroke}%
\pgfsetdash{}{0pt}%
\pgfpathmoveto{\pgfqpoint{0.000000in}{0.000000in}}%
\pgfpathlineto{\pgfqpoint{3.000000in}{0.000000in}}%
\pgfpathlineto{\pgfqpoint{3.000000in}{1.500000in}}%
\pgfpathlineto{\pgfqpoint{0.000000in}{1.500000in}}%
\pgfpathclose%
\pgfusepath{fill}%
\end{pgfscope}%
\begin{pgfscope}%
\pgfpathrectangle{\pgfqpoint{0.375000in}{0.150000in}}{\pgfqpoint{2.325000in}{1.200000in}} %
\pgfusepath{clip}%
\pgfsetbuttcap%
\pgfsetroundjoin%
\definecolor{currentfill}{rgb}{0.000000,0.000000,0.000000}%
\pgfsetfillcolor{currentfill}%
\pgfsetlinewidth{1.003750pt}%
\definecolor{currentstroke}{rgb}{0.000000,0.000000,0.000000}%
\pgfsetstrokecolor{currentstroke}%
\pgfsetdash{}{0pt}%
\pgfpathmoveto{\pgfqpoint{1.733924in}{0.908131in}}%
\pgfpathlineto{\pgfqpoint{1.658529in}{0.901782in}}%
\pgfpathlineto{\pgfqpoint{1.685750in}{0.940670in}}%
\pgfpathlineto{\pgfqpoint{1.675830in}{0.947614in}}%
\pgfpathlineto{\pgfqpoint{1.676941in}{0.949201in}}%
\pgfpathlineto{\pgfqpoint{1.686861in}{0.942257in}}%
\pgfpathlineto{\pgfqpoint{1.714083in}{0.981145in}}%
\pgfpathclose%
\pgfusepath{stroke,fill}%
\end{pgfscope}%
\begin{pgfscope}%
\pgfpathrectangle{\pgfqpoint{0.375000in}{0.150000in}}{\pgfqpoint{2.325000in}{1.200000in}} %
\pgfusepath{clip}%
\pgfsetrectcap%
\pgfsetroundjoin%
\pgfsetlinewidth{1.003750pt}%
\definecolor{currentstroke}{rgb}{0.000000,0.000000,1.000000}%
\pgfsetstrokecolor{currentstroke}%
\pgfsetdash{}{0pt}%
\pgfpathmoveto{\pgfqpoint{0.568750in}{0.265625in}}%
\pgfpathlineto{\pgfqpoint{2.506250in}{0.265625in}}%
\pgfusepath{stroke}%
\end{pgfscope}%
\begin{pgfscope}%
\pgfpathrectangle{\pgfqpoint{0.375000in}{0.150000in}}{\pgfqpoint{2.325000in}{1.200000in}} %
\pgfusepath{clip}%
\pgfsetrectcap%
\pgfsetroundjoin%
\pgfsetlinewidth{1.003750pt}%
\definecolor{currentstroke}{rgb}{0.000000,0.000000,0.000000}%
\pgfsetstrokecolor{currentstroke}%
\pgfsetdash{}{0pt}%
\pgfpathmoveto{\pgfqpoint{1.320881in}{0.858310in}}%
\pgfpathlineto{\pgfqpoint{1.334677in}{0.882354in}}%
\pgfpathlineto{\pgfqpoint{1.351130in}{0.904665in}}%
\pgfpathlineto{\pgfqpoint{1.370026in}{0.924949in}}%
\pgfpathlineto{\pgfqpoint{1.391115in}{0.942941in}}%
\pgfpathlineto{\pgfqpoint{1.414122in}{0.958405in}}%
\pgfpathlineto{\pgfqpoint{1.438746in}{0.971139in}}%
\pgfpathlineto{\pgfqpoint{1.464663in}{0.980975in}}%
\pgfpathlineto{\pgfqpoint{1.491535in}{0.987786in}}%
\pgfpathlineto{\pgfqpoint{1.519009in}{0.991481in}}%
\pgfpathlineto{\pgfqpoint{1.546725in}{0.992012in}}%
\pgfpathlineto{\pgfqpoint{1.574321in}{0.989372in}}%
\pgfpathlineto{\pgfqpoint{1.601434in}{0.983596in}}%
\pgfpathlineto{\pgfqpoint{1.627709in}{0.974760in}}%
\pgfpathlineto{\pgfqpoint{1.652803in}{0.962979in}}%
\pgfpathlineto{\pgfqpoint{1.676385in}{0.948408in}}%
\pgfusepath{stroke}%
\end{pgfscope}%
\begin{pgfscope}%
\pgfpathrectangle{\pgfqpoint{0.375000in}{0.150000in}}{\pgfqpoint{2.325000in}{1.200000in}} %
\pgfusepath{clip}%
\pgfsetrectcap%
\pgfsetroundjoin%
\pgfsetlinewidth{1.003750pt}%
\definecolor{currentstroke}{rgb}{0.000000,0.000000,0.000000}%
\pgfsetstrokecolor{currentstroke}%
\pgfsetdash{}{0pt}%
\pgfpathmoveto{\pgfqpoint{2.021875in}{0.750000in}}%
\pgfpathlineto{\pgfqpoint{2.015912in}{0.825773in}}%
\pgfpathlineto{\pgfqpoint{1.998168in}{0.899680in}}%
\pgfpathlineto{\pgfqpoint{1.969081in}{0.969902in}}%
\pgfpathlineto{\pgfqpoint{1.929368in}{1.034708in}}%
\pgfpathlineto{\pgfqpoint{1.880005in}{1.092505in}}%
\pgfpathlineto{\pgfqpoint{1.822208in}{1.141868in}}%
\pgfpathlineto{\pgfqpoint{1.757402in}{1.181581in}}%
\pgfpathlineto{\pgfqpoint{1.687180in}{1.210668in}}%
\pgfpathlineto{\pgfqpoint{1.613273in}{1.228412in}}%
\pgfpathlineto{\pgfqpoint{1.537500in}{1.234375in}}%
\pgfpathlineto{\pgfqpoint{1.461727in}{1.228412in}}%
\pgfpathlineto{\pgfqpoint{1.387820in}{1.210668in}}%
\pgfpathlineto{\pgfqpoint{1.317598in}{1.181581in}}%
\pgfpathlineto{\pgfqpoint{1.252792in}{1.141868in}}%
\pgfpathlineto{\pgfqpoint{1.194995in}{1.092505in}}%
\pgfpathlineto{\pgfqpoint{1.145632in}{1.034708in}}%
\pgfpathlineto{\pgfqpoint{1.105919in}{0.969902in}}%
\pgfpathlineto{\pgfqpoint{1.076832in}{0.899680in}}%
\pgfpathlineto{\pgfqpoint{1.059088in}{0.825773in}}%
\pgfpathlineto{\pgfqpoint{1.053125in}{0.750000in}}%
\pgfusepath{stroke}%
\end{pgfscope}%
\begin{pgfscope}%
\pgfpathrectangle{\pgfqpoint{0.375000in}{0.150000in}}{\pgfqpoint{2.325000in}{1.200000in}} %
\pgfusepath{clip}%
\pgfsetrectcap%
\pgfsetroundjoin%
\pgfsetlinewidth{1.003750pt}%
\definecolor{currentstroke}{rgb}{0.000000,0.000000,0.000000}%
\pgfsetstrokecolor{currentstroke}%
\pgfsetdash{}{0pt}%
\pgfpathmoveto{\pgfqpoint{1.053125in}{0.750000in}}%
\pgfpathlineto{\pgfqpoint{1.059088in}{0.674227in}}%
\pgfpathlineto{\pgfqpoint{1.076832in}{0.600320in}}%
\pgfpathlineto{\pgfqpoint{1.105919in}{0.530098in}}%
\pgfpathlineto{\pgfqpoint{1.145632in}{0.465292in}}%
\pgfpathlineto{\pgfqpoint{1.194995in}{0.407495in}}%
\pgfpathlineto{\pgfqpoint{1.252792in}{0.358132in}}%
\pgfpathlineto{\pgfqpoint{1.317598in}{0.318419in}}%
\pgfpathlineto{\pgfqpoint{1.387820in}{0.289332in}}%
\pgfpathlineto{\pgfqpoint{1.461727in}{0.271588in}}%
\pgfpathlineto{\pgfqpoint{1.537500in}{0.265625in}}%
\pgfpathlineto{\pgfqpoint{1.613273in}{0.271588in}}%
\pgfpathlineto{\pgfqpoint{1.687180in}{0.289332in}}%
\pgfpathlineto{\pgfqpoint{1.757402in}{0.318419in}}%
\pgfpathlineto{\pgfqpoint{1.822208in}{0.358132in}}%
\pgfpathlineto{\pgfqpoint{1.880005in}{0.407495in}}%
\pgfpathlineto{\pgfqpoint{1.929368in}{0.465292in}}%
\pgfpathlineto{\pgfqpoint{1.969081in}{0.530098in}}%
\pgfpathlineto{\pgfqpoint{1.998168in}{0.600320in}}%
\pgfpathlineto{\pgfqpoint{2.015912in}{0.674227in}}%
\pgfpathlineto{\pgfqpoint{2.021875in}{0.750000in}}%
\pgfusepath{stroke}%
\end{pgfscope}%
\begin{pgfscope}%
\pgfpathrectangle{\pgfqpoint{0.375000in}{0.150000in}}{\pgfqpoint{2.325000in}{1.200000in}} %
\pgfusepath{clip}%
\pgfsetbuttcap%
\pgfsetroundjoin%
\definecolor{currentfill}{rgb}{0.000000,0.000000,1.000000}%
\pgfsetfillcolor{currentfill}%
\pgfsetlinewidth{0.501875pt}%
\definecolor{currentstroke}{rgb}{0.000000,0.000000,0.000000}%
\pgfsetstrokecolor{currentstroke}%
\pgfsetdash{}{0pt}%
\pgfsys@defobject{currentmarker}{\pgfqpoint{-0.041667in}{-0.041667in}}{\pgfqpoint{0.041667in}{0.041667in}}{%
\pgfpathmoveto{\pgfqpoint{0.000000in}{-0.041667in}}%
\pgfpathcurveto{\pgfqpoint{0.011050in}{-0.041667in}}{\pgfqpoint{0.021649in}{-0.037276in}}{\pgfqpoint{0.029463in}{-0.029463in}}%
\pgfpathcurveto{\pgfqpoint{0.037276in}{-0.021649in}}{\pgfqpoint{0.041667in}{-0.011050in}}{\pgfqpoint{0.041667in}{0.000000in}}%
\pgfpathcurveto{\pgfqpoint{0.041667in}{0.011050in}}{\pgfqpoint{0.037276in}{0.021649in}}{\pgfqpoint{0.029463in}{0.029463in}}%
\pgfpathcurveto{\pgfqpoint{0.021649in}{0.037276in}}{\pgfqpoint{0.011050in}{0.041667in}}{\pgfqpoint{0.000000in}{0.041667in}}%
\pgfpathcurveto{\pgfqpoint{-0.011050in}{0.041667in}}{\pgfqpoint{-0.021649in}{0.037276in}}{\pgfqpoint{-0.029463in}{0.029463in}}%
\pgfpathcurveto{\pgfqpoint{-0.037276in}{0.021649in}}{\pgfqpoint{-0.041667in}{0.011050in}}{\pgfqpoint{-0.041667in}{0.000000in}}%
\pgfpathcurveto{\pgfqpoint{-0.041667in}{-0.011050in}}{\pgfqpoint{-0.037276in}{-0.021649in}}{\pgfqpoint{-0.029463in}{-0.029463in}}%
\pgfpathcurveto{\pgfqpoint{-0.021649in}{-0.037276in}}{\pgfqpoint{-0.011050in}{-0.041667in}}{\pgfqpoint{0.000000in}{-0.041667in}}%
\pgfpathclose%
\pgfusepath{stroke,fill}%
}%
\begin{pgfscope}%
\pgfsys@transformshift{1.537500in}{0.750000in}%
\pgfsys@useobject{currentmarker}{}%
\end{pgfscope}%
\end{pgfscope}%
\begin{pgfscope}%
\pgfpathrectangle{\pgfqpoint{0.375000in}{0.150000in}}{\pgfqpoint{2.325000in}{1.200000in}} %
\pgfusepath{clip}%
\pgfsetbuttcap%
\pgfsetroundjoin%
\definecolor{currentfill}{rgb}{0.000000,0.500000,0.000000}%
\pgfsetfillcolor{currentfill}%
\pgfsetlinewidth{0.501875pt}%
\definecolor{currentstroke}{rgb}{0.000000,0.000000,0.000000}%
\pgfsetstrokecolor{currentstroke}%
\pgfsetdash{}{0pt}%
\pgfsys@defobject{currentmarker}{\pgfqpoint{-0.041667in}{-0.041667in}}{\pgfqpoint{0.041667in}{0.041667in}}{%
\pgfpathmoveto{\pgfqpoint{0.000000in}{-0.041667in}}%
\pgfpathcurveto{\pgfqpoint{0.011050in}{-0.041667in}}{\pgfqpoint{0.021649in}{-0.037276in}}{\pgfqpoint{0.029463in}{-0.029463in}}%
\pgfpathcurveto{\pgfqpoint{0.037276in}{-0.021649in}}{\pgfqpoint{0.041667in}{-0.011050in}}{\pgfqpoint{0.041667in}{0.000000in}}%
\pgfpathcurveto{\pgfqpoint{0.041667in}{0.011050in}}{\pgfqpoint{0.037276in}{0.021649in}}{\pgfqpoint{0.029463in}{0.029463in}}%
\pgfpathcurveto{\pgfqpoint{0.021649in}{0.037276in}}{\pgfqpoint{0.011050in}{0.041667in}}{\pgfqpoint{0.000000in}{0.041667in}}%
\pgfpathcurveto{\pgfqpoint{-0.011050in}{0.041667in}}{\pgfqpoint{-0.021649in}{0.037276in}}{\pgfqpoint{-0.029463in}{0.029463in}}%
\pgfpathcurveto{\pgfqpoint{-0.037276in}{0.021649in}}{\pgfqpoint{-0.041667in}{0.011050in}}{\pgfqpoint{-0.041667in}{0.000000in}}%
\pgfpathcurveto{\pgfqpoint{-0.041667in}{-0.011050in}}{\pgfqpoint{-0.037276in}{-0.021649in}}{\pgfqpoint{-0.029463in}{-0.029463in}}%
\pgfpathcurveto{\pgfqpoint{-0.021649in}{-0.037276in}}{\pgfqpoint{-0.011050in}{-0.041667in}}{\pgfqpoint{0.000000in}{-0.041667in}}%
\pgfpathclose%
\pgfusepath{stroke,fill}%
}%
\begin{pgfscope}%
\pgfsys@transformshift{1.537500in}{0.265625in}%
\pgfsys@useobject{currentmarker}{}%
\end{pgfscope}%
\end{pgfscope}%
\begin{pgfscope}%
\pgftext[x=1.673125in,y=0.943750in,left,bottom]{{\sffamily\fontsize{20.000000}{24.000000}\selectfont \(\displaystyle \tau\)}}%
\end{pgfscope}%
\begin{pgfscope}%
\pgftext[x=1.537500in,y=0.750000in,right,top]{{\sffamily\fontsize{14.000000}{16.800000}\selectfont \(\displaystyle CM\)}}%
\end{pgfscope}%
\end{pgfpicture}%
\makeatother%
\endgroup%
\newline
In this case, the rule of friction is not to cause rotation; the rotation
about the center is done by the external torque. From 2nd law, in order to
have no slipping, a translational force is needed to accelerate the CM of
the object to move. Otherwise the object will slip. Friction in this case
acts as this \emph{force}, to give the CM of the object an acceleration to
the right. One can judge from the relative movement between the object and
the ground too. Friction prevents the object from skidding at the contact.
So friction acting on the object goes to the right, while the mutual
friction force acting on the ground goes to the left.%
\begin{eqnarray*}
&&\left\{ 
\begin{array}{c}
F_{fr}=ma \\ 
\tau -F_{fr}\cdot R=I\alpha =I\frac{a}{R}%
\end{array}%
\right.  \\
&\Rightarrow &a=\frac{\tau }{mR+\frac{I}{R}}
\end{eqnarray*}%
%% Creator: Matplotlib, PGF backend
%%
%% To include the figure in your LaTeX document, write
%%   \input{<filename>.pgf}
%%
%% Make sure the required packages are loaded in your preamble
%%   \usepackage{pgf}
%%
%% Figures using additional raster images can only be included by \input if
%% they are in the same directory as the main LaTeX file. For loading figures
%% from other directories you can use the `import` package
%%   \usepackage{import}
%% and then include the figures with
%%   \import{<path to file>}{<filename>.pgf}
%%
%% Matplotlib used the following preamble
%%   \usepackage{fontspec}
%%   \setmainfont{Times New Roman}
%%   \setsansfont{Verdana}
%%   \setmonofont{Courier New}
%%
\begingroup%
\makeatletter%
\begin{pgfpicture}%
\pgfpathrectangle{\pgfpointorigin}{\pgfqpoint{4.000000in}{2.600000in}}%
\pgfusepath{use as bounding box}%
\begin{pgfscope}%
\pgfsetbuttcap%
\pgfsetroundjoin%
\definecolor{currentfill}{rgb}{1.000000,1.000000,1.000000}%
\pgfsetfillcolor{currentfill}%
\pgfsetlinewidth{0.000000pt}%
\definecolor{currentstroke}{rgb}{1.000000,1.000000,1.000000}%
\pgfsetstrokecolor{currentstroke}%
\pgfsetdash{}{0pt}%
\pgfpathmoveto{\pgfqpoint{0.000000in}{0.000000in}}%
\pgfpathlineto{\pgfqpoint{4.000000in}{0.000000in}}%
\pgfpathlineto{\pgfqpoint{4.000000in}{2.600000in}}%
\pgfpathlineto{\pgfqpoint{0.000000in}{2.600000in}}%
\pgfpathclose%
\pgfusepath{fill}%
\end{pgfscope}%
\begin{pgfscope}%
\pgfpathrectangle{\pgfqpoint{0.500000in}{0.260000in}}{\pgfqpoint{3.100000in}{2.080000in}} %
\pgfusepath{clip}%
\pgfsetbuttcap%
\pgfsetroundjoin%
\definecolor{currentfill}{rgb}{0.000000,0.000000,1.000000}%
\pgfsetfillcolor{currentfill}%
\pgfsetlinewidth{1.003750pt}%
\definecolor{currentstroke}{rgb}{0.000000,0.000000,0.000000}%
\pgfsetstrokecolor{currentstroke}%
\pgfsetdash{}{0pt}%
\pgfpathmoveto{\pgfqpoint{2.928333in}{0.654167in}}%
\pgfpathlineto{\pgfqpoint{2.695833in}{0.550833in}}%
\pgfpathlineto{\pgfqpoint{2.695833in}{0.630917in}}%
\pgfpathlineto{\pgfqpoint{2.050000in}{0.630917in}}%
\pgfpathlineto{\pgfqpoint{2.050000in}{0.677417in}}%
\pgfpathlineto{\pgfqpoint{2.695833in}{0.677417in}}%
\pgfpathlineto{\pgfqpoint{2.695833in}{0.757500in}}%
\pgfpathclose%
\pgfusepath{stroke,fill}%
\end{pgfscope}%
\begin{pgfscope}%
\pgfpathrectangle{\pgfqpoint{0.500000in}{0.260000in}}{\pgfqpoint{3.100000in}{2.080000in}} %
\pgfusepath{clip}%
\pgfsetbuttcap%
\pgfsetroundjoin%
\definecolor{currentfill}{rgb}{1.000000,0.000000,0.000000}%
\pgfsetfillcolor{currentfill}%
\pgfsetlinewidth{1.003750pt}%
\definecolor{currentstroke}{rgb}{1.000000,0.000000,0.000000}%
\pgfsetstrokecolor{currentstroke}%
\pgfsetdash{}{0pt}%
\pgfpathmoveto{\pgfqpoint{3.143611in}{1.300000in}}%
\pgfpathlineto{\pgfqpoint{2.911111in}{1.196667in}}%
\pgfpathlineto{\pgfqpoint{2.911111in}{1.276750in}}%
\pgfpathlineto{\pgfqpoint{2.050000in}{1.276750in}}%
\pgfpathlineto{\pgfqpoint{2.050000in}{1.323250in}}%
\pgfpathlineto{\pgfqpoint{2.911111in}{1.323250in}}%
\pgfpathlineto{\pgfqpoint{2.911111in}{1.403333in}}%
\pgfpathclose%
\pgfusepath{stroke,fill}%
\end{pgfscope}%
\begin{pgfscope}%
\pgfpathrectangle{\pgfqpoint{0.500000in}{0.260000in}}{\pgfqpoint{3.100000in}{2.080000in}} %
\pgfusepath{clip}%
\pgfsetbuttcap%
\pgfsetroundjoin%
\definecolor{currentfill}{rgb}{0.000000,0.000000,0.000000}%
\pgfsetfillcolor{currentfill}%
\pgfsetlinewidth{1.003750pt}%
\definecolor{currentstroke}{rgb}{0.000000,0.000000,0.000000}%
\pgfsetstrokecolor{currentstroke}%
\pgfsetdash{}{0pt}%
\pgfpathmoveto{\pgfqpoint{2.669032in}{1.970354in}}%
\pgfpathlineto{\pgfqpoint{2.568506in}{1.961888in}}%
\pgfpathlineto{\pgfqpoint{2.604801in}{2.013739in}}%
\pgfpathlineto{\pgfqpoint{2.567765in}{2.039664in}}%
\pgfpathlineto{\pgfqpoint{2.569246in}{2.041780in}}%
\pgfpathlineto{\pgfqpoint{2.606282in}{2.015855in}}%
\pgfpathlineto{\pgfqpoint{2.642578in}{2.067706in}}%
\pgfpathclose%
\pgfusepath{stroke,fill}%
\end{pgfscope}%
\begin{pgfscope}%
\pgfpathrectangle{\pgfqpoint{0.500000in}{0.260000in}}{\pgfqpoint{3.100000in}{2.080000in}} %
\pgfusepath{clip}%
\pgfsetbuttcap%
\pgfsetroundjoin%
\definecolor{currentfill}{rgb}{0.000000,0.000000,0.000000}%
\pgfsetfillcolor{currentfill}%
\pgfsetlinewidth{1.003750pt}%
\definecolor{currentstroke}{rgb}{0.000000,0.000000,0.000000}%
\pgfsetstrokecolor{currentstroke}%
\pgfsetdash{}{0pt}%
\pgfpathmoveto{\pgfqpoint{2.335707in}{1.494175in}}%
\pgfpathlineto{\pgfqpoint{2.235181in}{1.485710in}}%
\pgfpathlineto{\pgfqpoint{2.271476in}{1.537560in}}%
\pgfpathlineto{\pgfqpoint{2.234440in}{1.563485in}}%
\pgfpathlineto{\pgfqpoint{2.235921in}{1.565602in}}%
\pgfpathlineto{\pgfqpoint{2.272957in}{1.539677in}}%
\pgfpathlineto{\pgfqpoint{2.309253in}{1.591527in}}%
\pgfpathclose%
\pgfusepath{stroke,fill}%
\end{pgfscope}%
\begin{pgfscope}%
\pgfpathrectangle{\pgfqpoint{0.500000in}{0.260000in}}{\pgfqpoint{3.100000in}{2.080000in}} %
\pgfusepath{clip}%
\pgfsetrectcap%
\pgfsetroundjoin%
\pgfsetlinewidth{1.003750pt}%
\definecolor{currentstroke}{rgb}{0.000000,0.000000,1.000000}%
\pgfsetstrokecolor{currentstroke}%
\pgfsetdash{}{0pt}%
\pgfpathmoveto{\pgfqpoint{0.758333in}{0.654167in}}%
\pgfpathlineto{\pgfqpoint{3.341667in}{0.654167in}}%
\pgfusepath{stroke}%
\end{pgfscope}%
\begin{pgfscope}%
\pgfpathrectangle{\pgfqpoint{0.500000in}{0.260000in}}{\pgfqpoint{3.100000in}{2.080000in}} %
\pgfusepath{clip}%
\pgfsetrectcap%
\pgfsetroundjoin%
\pgfsetlinewidth{1.003750pt}%
\definecolor{currentstroke}{rgb}{0.000000,0.000000,0.000000}%
\pgfsetstrokecolor{currentstroke}%
\pgfsetdash{}{0pt}%
\pgfpathmoveto{\pgfqpoint{2.050000in}{0.654167in}}%
\pgfpathlineto{\pgfqpoint{2.050000in}{1.300000in}}%
\pgfusepath{stroke}%
\end{pgfscope}%
\begin{pgfscope}%
\pgfpathrectangle{\pgfqpoint{0.500000in}{0.260000in}}{\pgfqpoint{3.100000in}{2.080000in}} %
\pgfusepath{clip}%
\pgfsetrectcap%
\pgfsetroundjoin%
\pgfsetlinewidth{1.003750pt}%
\definecolor{currentstroke}{rgb}{0.000000,0.000000,0.000000}%
\pgfsetstrokecolor{currentstroke}%
\pgfsetdash{}{0pt}%
\pgfpathmoveto{\pgfqpoint{2.139968in}{2.199679in}}%
\pgfpathlineto{\pgfqpoint{2.185809in}{2.193909in}}%
\pgfpathlineto{\pgfqpoint{2.231296in}{2.185804in}}%
\pgfpathlineto{\pgfqpoint{2.276309in}{2.175386in}}%
\pgfpathlineto{\pgfqpoint{2.320732in}{2.162683in}}%
\pgfpathlineto{\pgfqpoint{2.364447in}{2.147727in}}%
\pgfpathlineto{\pgfqpoint{2.407341in}{2.130557in}}%
\pgfpathlineto{\pgfqpoint{2.449302in}{2.111218in}}%
\pgfpathlineto{\pgfqpoint{2.490221in}{2.089761in}}%
\pgfpathlineto{\pgfqpoint{2.529990in}{2.066242in}}%
\pgfpathlineto{\pgfqpoint{2.568506in}{2.040722in}}%
\pgfusepath{stroke}%
\end{pgfscope}%
\begin{pgfscope}%
\pgfpathrectangle{\pgfqpoint{0.500000in}{0.260000in}}{\pgfqpoint{3.100000in}{2.080000in}} %
\pgfusepath{clip}%
\pgfsetrectcap%
\pgfsetroundjoin%
\pgfsetlinewidth{1.003750pt}%
\definecolor{currentstroke}{rgb}{0.000000,0.000000,0.000000}%
\pgfsetstrokecolor{currentstroke}%
\pgfsetdash{}{0pt}%
\pgfpathmoveto{\pgfqpoint{2.050000in}{1.622917in}}%
\pgfpathlineto{\pgfqpoint{2.063144in}{1.622649in}}%
\pgfpathlineto{\pgfqpoint{2.076266in}{1.621847in}}%
\pgfpathlineto{\pgfqpoint{2.089345in}{1.620511in}}%
\pgfpathlineto{\pgfqpoint{2.102358in}{1.618644in}}%
\pgfpathlineto{\pgfqpoint{2.115285in}{1.616248in}}%
\pgfpathlineto{\pgfqpoint{2.128103in}{1.613329in}}%
\pgfpathlineto{\pgfqpoint{2.140792in}{1.609890in}}%
\pgfpathlineto{\pgfqpoint{2.153331in}{1.605938in}}%
\pgfpathlineto{\pgfqpoint{2.165698in}{1.601478in}}%
\pgfpathlineto{\pgfqpoint{2.177873in}{1.596519in}}%
\pgfpathlineto{\pgfqpoint{2.189837in}{1.591068in}}%
\pgfpathlineto{\pgfqpoint{2.201568in}{1.585135in}}%
\pgfpathlineto{\pgfqpoint{2.213049in}{1.578730in}}%
\pgfpathlineto{\pgfqpoint{2.224259in}{1.571862in}}%
\pgfpathlineto{\pgfqpoint{2.235181in}{1.564544in}}%
\pgfusepath{stroke}%
\end{pgfscope}%
\begin{pgfscope}%
\pgfpathrectangle{\pgfqpoint{0.500000in}{0.260000in}}{\pgfqpoint{3.100000in}{2.080000in}} %
\pgfusepath{clip}%
\pgfsetrectcap%
\pgfsetroundjoin%
\pgfsetlinewidth{1.003750pt}%
\definecolor{currentstroke}{rgb}{0.000000,0.000000,0.000000}%
\pgfsetstrokecolor{currentstroke}%
\pgfsetdash{}{0pt}%
\pgfpathmoveto{\pgfqpoint{2.695833in}{1.300000in}}%
\pgfpathlineto{\pgfqpoint{2.687882in}{1.401031in}}%
\pgfpathlineto{\pgfqpoint{2.664224in}{1.499573in}}%
\pgfpathlineto{\pgfqpoint{2.625442in}{1.593202in}}%
\pgfpathlineto{\pgfqpoint{2.572490in}{1.679611in}}%
\pgfpathlineto{\pgfqpoint{2.506673in}{1.756673in}}%
\pgfpathlineto{\pgfqpoint{2.429611in}{1.822490in}}%
\pgfpathlineto{\pgfqpoint{2.343202in}{1.875442in}}%
\pgfpathlineto{\pgfqpoint{2.249573in}{1.914224in}}%
\pgfpathlineto{\pgfqpoint{2.151031in}{1.937882in}}%
\pgfpathlineto{\pgfqpoint{2.050000in}{1.945833in}}%
\pgfpathlineto{\pgfqpoint{1.948969in}{1.937882in}}%
\pgfpathlineto{\pgfqpoint{1.850427in}{1.914224in}}%
\pgfpathlineto{\pgfqpoint{1.756798in}{1.875442in}}%
\pgfpathlineto{\pgfqpoint{1.670389in}{1.822490in}}%
\pgfpathlineto{\pgfqpoint{1.593327in}{1.756673in}}%
\pgfpathlineto{\pgfqpoint{1.527510in}{1.679611in}}%
\pgfpathlineto{\pgfqpoint{1.474558in}{1.593202in}}%
\pgfpathlineto{\pgfqpoint{1.435776in}{1.499573in}}%
\pgfpathlineto{\pgfqpoint{1.412118in}{1.401031in}}%
\pgfpathlineto{\pgfqpoint{1.404167in}{1.300000in}}%
\pgfusepath{stroke}%
\end{pgfscope}%
\begin{pgfscope}%
\pgfpathrectangle{\pgfqpoint{0.500000in}{0.260000in}}{\pgfqpoint{3.100000in}{2.080000in}} %
\pgfusepath{clip}%
\pgfsetrectcap%
\pgfsetroundjoin%
\pgfsetlinewidth{1.003750pt}%
\definecolor{currentstroke}{rgb}{0.000000,0.000000,0.000000}%
\pgfsetstrokecolor{currentstroke}%
\pgfsetdash{}{0pt}%
\pgfpathmoveto{\pgfqpoint{1.404167in}{1.300000in}}%
\pgfpathlineto{\pgfqpoint{1.412118in}{1.198969in}}%
\pgfpathlineto{\pgfqpoint{1.435776in}{1.100427in}}%
\pgfpathlineto{\pgfqpoint{1.474558in}{1.006798in}}%
\pgfpathlineto{\pgfqpoint{1.527510in}{0.920389in}}%
\pgfpathlineto{\pgfqpoint{1.593327in}{0.843327in}}%
\pgfpathlineto{\pgfqpoint{1.670389in}{0.777510in}}%
\pgfpathlineto{\pgfqpoint{1.756798in}{0.724558in}}%
\pgfpathlineto{\pgfqpoint{1.850427in}{0.685776in}}%
\pgfpathlineto{\pgfqpoint{1.948969in}{0.662118in}}%
\pgfpathlineto{\pgfqpoint{2.050000in}{0.654167in}}%
\pgfpathlineto{\pgfqpoint{2.151031in}{0.662118in}}%
\pgfpathlineto{\pgfqpoint{2.249573in}{0.685776in}}%
\pgfpathlineto{\pgfqpoint{2.343202in}{0.724558in}}%
\pgfpathlineto{\pgfqpoint{2.429611in}{0.777510in}}%
\pgfpathlineto{\pgfqpoint{2.506673in}{0.843327in}}%
\pgfpathlineto{\pgfqpoint{2.572490in}{0.920389in}}%
\pgfpathlineto{\pgfqpoint{2.625442in}{1.006798in}}%
\pgfpathlineto{\pgfqpoint{2.664224in}{1.100427in}}%
\pgfpathlineto{\pgfqpoint{2.687882in}{1.198969in}}%
\pgfpathlineto{\pgfqpoint{2.695833in}{1.300000in}}%
\pgfusepath{stroke}%
\end{pgfscope}%
\begin{pgfscope}%
\pgfpathrectangle{\pgfqpoint{0.500000in}{0.260000in}}{\pgfqpoint{3.100000in}{2.080000in}} %
\pgfusepath{clip}%
\pgfsetbuttcap%
\pgfsetroundjoin%
\definecolor{currentfill}{rgb}{0.000000,0.000000,1.000000}%
\pgfsetfillcolor{currentfill}%
\pgfsetlinewidth{0.501875pt}%
\definecolor{currentstroke}{rgb}{0.000000,0.000000,0.000000}%
\pgfsetstrokecolor{currentstroke}%
\pgfsetdash{}{0pt}%
\pgfsys@defobject{currentmarker}{\pgfqpoint{-0.041667in}{-0.041667in}}{\pgfqpoint{0.041667in}{0.041667in}}{%
\pgfpathmoveto{\pgfqpoint{0.000000in}{-0.041667in}}%
\pgfpathcurveto{\pgfqpoint{0.011050in}{-0.041667in}}{\pgfqpoint{0.021649in}{-0.037276in}}{\pgfqpoint{0.029463in}{-0.029463in}}%
\pgfpathcurveto{\pgfqpoint{0.037276in}{-0.021649in}}{\pgfqpoint{0.041667in}{-0.011050in}}{\pgfqpoint{0.041667in}{0.000000in}}%
\pgfpathcurveto{\pgfqpoint{0.041667in}{0.011050in}}{\pgfqpoint{0.037276in}{0.021649in}}{\pgfqpoint{0.029463in}{0.029463in}}%
\pgfpathcurveto{\pgfqpoint{0.021649in}{0.037276in}}{\pgfqpoint{0.011050in}{0.041667in}}{\pgfqpoint{0.000000in}{0.041667in}}%
\pgfpathcurveto{\pgfqpoint{-0.011050in}{0.041667in}}{\pgfqpoint{-0.021649in}{0.037276in}}{\pgfqpoint{-0.029463in}{0.029463in}}%
\pgfpathcurveto{\pgfqpoint{-0.037276in}{0.021649in}}{\pgfqpoint{-0.041667in}{0.011050in}}{\pgfqpoint{-0.041667in}{0.000000in}}%
\pgfpathcurveto{\pgfqpoint{-0.041667in}{-0.011050in}}{\pgfqpoint{-0.037276in}{-0.021649in}}{\pgfqpoint{-0.029463in}{-0.029463in}}%
\pgfpathcurveto{\pgfqpoint{-0.021649in}{-0.037276in}}{\pgfqpoint{-0.011050in}{-0.041667in}}{\pgfqpoint{0.000000in}{-0.041667in}}%
\pgfpathclose%
\pgfusepath{stroke,fill}%
}%
\begin{pgfscope}%
\pgfsys@transformshift{2.050000in}{1.300000in}%
\pgfsys@useobject{currentmarker}{}%
\end{pgfscope}%
\end{pgfscope}%
\begin{pgfscope}%
\pgfpathrectangle{\pgfqpoint{0.500000in}{0.260000in}}{\pgfqpoint{3.100000in}{2.080000in}} %
\pgfusepath{clip}%
\pgfsetbuttcap%
\pgfsetroundjoin%
\definecolor{currentfill}{rgb}{0.000000,0.500000,0.000000}%
\pgfsetfillcolor{currentfill}%
\pgfsetlinewidth{0.501875pt}%
\definecolor{currentstroke}{rgb}{0.000000,0.000000,0.000000}%
\pgfsetstrokecolor{currentstroke}%
\pgfsetdash{}{0pt}%
\pgfsys@defobject{currentmarker}{\pgfqpoint{-0.041667in}{-0.041667in}}{\pgfqpoint{0.041667in}{0.041667in}}{%
\pgfpathmoveto{\pgfqpoint{0.000000in}{-0.041667in}}%
\pgfpathcurveto{\pgfqpoint{0.011050in}{-0.041667in}}{\pgfqpoint{0.021649in}{-0.037276in}}{\pgfqpoint{0.029463in}{-0.029463in}}%
\pgfpathcurveto{\pgfqpoint{0.037276in}{-0.021649in}}{\pgfqpoint{0.041667in}{-0.011050in}}{\pgfqpoint{0.041667in}{0.000000in}}%
\pgfpathcurveto{\pgfqpoint{0.041667in}{0.011050in}}{\pgfqpoint{0.037276in}{0.021649in}}{\pgfqpoint{0.029463in}{0.029463in}}%
\pgfpathcurveto{\pgfqpoint{0.021649in}{0.037276in}}{\pgfqpoint{0.011050in}{0.041667in}}{\pgfqpoint{0.000000in}{0.041667in}}%
\pgfpathcurveto{\pgfqpoint{-0.011050in}{0.041667in}}{\pgfqpoint{-0.021649in}{0.037276in}}{\pgfqpoint{-0.029463in}{0.029463in}}%
\pgfpathcurveto{\pgfqpoint{-0.037276in}{0.021649in}}{\pgfqpoint{-0.041667in}{0.011050in}}{\pgfqpoint{-0.041667in}{0.000000in}}%
\pgfpathcurveto{\pgfqpoint{-0.041667in}{-0.011050in}}{\pgfqpoint{-0.037276in}{-0.021649in}}{\pgfqpoint{-0.029463in}{-0.029463in}}%
\pgfpathcurveto{\pgfqpoint{-0.021649in}{-0.037276in}}{\pgfqpoint{-0.011050in}{-0.041667in}}{\pgfqpoint{0.000000in}{-0.041667in}}%
\pgfpathclose%
\pgfusepath{stroke,fill}%
}%
\begin{pgfscope}%
\pgfsys@transformshift{2.050000in}{0.654167in}%
\pgfsys@useobject{currentmarker}{}%
\end{pgfscope}%
\end{pgfscope}%
\begin{pgfscope}%
\pgftext[x=2.050000in,y=0.847917in,right,bottom]{{\sffamily\fontsize{14.000000}{16.800000}\selectfont \(\displaystyle R\)}}%
\end{pgfscope}%
\begin{pgfscope}%
\pgftext[x=2.308333in,y=2.204167in,left,bottom]{{\sffamily\fontsize{20.000000}{24.000000}\selectfont \(\displaystyle \alpha\)}}%
\end{pgfscope}%
\begin{pgfscope}%
\pgftext[x=2.230833in,y=1.558333in,left,bottom]{{\sffamily\fontsize{20.000000}{24.000000}\selectfont \(\displaystyle \tau\)}}%
\end{pgfscope}%
\begin{pgfscope}%
\pgftext[x=2.050000in,y=1.300000in,right,bottom]{{\sffamily\fontsize{14.000000}{16.800000}\selectfont \(\displaystyle CM\)}}%
\end{pgfscope}%
\begin{pgfscope}%
\pgftext[x=2.695833in,y=0.525000in,left,top]{{\sffamily\fontsize{20.000000}{24.000000}\selectfont \(\displaystyle F_{fr}\)}}%
\end{pgfscope}%
\begin{pgfscope}%
\definecolor{textcolor}{rgb}{1.000000,0.000000,0.000000}%
\pgfsetstrokecolor{textcolor}%
\pgfsetfillcolor{textcolor}%
\pgftext[x=2.911111in,y=1.170833in,left,top]{{\sffamily\fontsize{20.000000}{24.000000}\selectfont \(\displaystyle a\)}}%
\end{pgfscope}%
\end{pgfpicture}%
\makeatother%
\endgroup%
\newline
There is an interesting conclusion we can draw from the result of our
calculation. In a real car race, the beginning part or the start off part of
the race is very important. A car driver want to accelerate as fast as they
can, but at the same time not to cause the tire to slip and loose grip. How
to control the acceleration throttle and peddle is a hard problem. Well not
so hard if you have studied this example. From the last equation we arrived,
we know the acceleration of the tire $a$, which is essentially the same as
the car's acceleration, is directly proportional to the torque $\tau $. So
before you may think you want to start with a smooth peddle control and put
a little bit of throttle in the beginning so as not to loose grip. Wrong.
Once you give the tire a torque $\tau $, you acceleration $a$ will be
proportional to $\tau $, what you want is the highest throttle right at the
beginning! Of course not so high as to loose grip. So next time in a formula
race, when the red light off and green light lid, put down the highest
throttle that will not cause slip immediately. Give it a try. Easier to say
than done! Ideally this is what you want to do. In reality, when you start
off, the weight will tranfer from the front tires to the rear tires (because
of the accleralation!), so there will be some delay before your rear tire
get maximal down force and maximal grip. And drivers need to take this delay
into account.
\end{case}

\newpage

\begin{case}
External forces not passing through CM pivot. (so there are both external
force and torque.)
\end{case}

%% Creator: Matplotlib, PGF backend
%%
%% To include the figure in your LaTeX document, write
%%   \input{<filename>.pgf}
%%
%% Make sure the required packages are loaded in your preamble
%%   \usepackage{pgf}
%%
%% Figures using additional raster images can only be included by \input if
%% they are in the same directory as the main LaTeX file. For loading figures
%% from other directories you can use the `import` package
%%   \usepackage{import}
%% and then include the figures with
%%   \import{<path to file>}{<filename>.pgf}
%%
%% Matplotlib used the following preamble
%%   \usepackage{fontspec}
%%   \setmainfont{Times New Roman}
%%   \setsansfont{Verdana}
%%   \setmonofont{Courier New}
%%
\begingroup%
\makeatletter%
\begin{pgfpicture}%
\pgfpathrectangle{\pgfpointorigin}{\pgfqpoint{3.000000in}{1.500000in}}%
\pgfusepath{use as bounding box}%
\begin{pgfscope}%
\pgfsetbuttcap%
\pgfsetroundjoin%
\definecolor{currentfill}{rgb}{1.000000,1.000000,1.000000}%
\pgfsetfillcolor{currentfill}%
\pgfsetlinewidth{0.000000pt}%
\definecolor{currentstroke}{rgb}{1.000000,1.000000,1.000000}%
\pgfsetstrokecolor{currentstroke}%
\pgfsetdash{}{0pt}%
\pgfpathmoveto{\pgfqpoint{0.000000in}{0.000000in}}%
\pgfpathlineto{\pgfqpoint{3.000000in}{0.000000in}}%
\pgfpathlineto{\pgfqpoint{3.000000in}{1.500000in}}%
\pgfpathlineto{\pgfqpoint{0.000000in}{1.500000in}}%
\pgfpathclose%
\pgfusepath{fill}%
\end{pgfscope}%
\begin{pgfscope}%
\pgfpathrectangle{\pgfqpoint{0.375000in}{0.150000in}}{\pgfqpoint{2.325000in}{1.200000in}} %
\pgfusepath{clip}%
\pgfsetbuttcap%
\pgfsetroundjoin%
\definecolor{currentfill}{rgb}{0.000000,0.000000,0.000000}%
\pgfsetfillcolor{currentfill}%
\pgfsetlinewidth{1.003750pt}%
\definecolor{currentstroke}{rgb}{0.000000,0.000000,0.000000}%
\pgfsetstrokecolor{currentstroke}%
\pgfsetdash{}{0pt}%
\pgfpathmoveto{\pgfqpoint{2.700000in}{1.292857in}}%
\pgfpathlineto{\pgfqpoint{2.632282in}{1.235714in}}%
\pgfpathlineto{\pgfqpoint{2.632282in}{1.291714in}}%
\pgfpathlineto{\pgfqpoint{1.503641in}{1.291714in}}%
\pgfpathlineto{\pgfqpoint{1.503641in}{1.294000in}}%
\pgfpathlineto{\pgfqpoint{2.632282in}{1.294000in}}%
\pgfpathlineto{\pgfqpoint{2.632282in}{1.350000in}}%
\pgfpathclose%
\pgfusepath{stroke,fill}%
\end{pgfscope}%
\begin{pgfscope}%
\pgfpathrectangle{\pgfqpoint{0.375000in}{0.150000in}}{\pgfqpoint{2.325000in}{1.200000in}} %
\pgfusepath{clip}%
\pgfsetrectcap%
\pgfsetroundjoin%
\pgfsetlinewidth{1.003750pt}%
\definecolor{currentstroke}{rgb}{0.000000,0.000000,1.000000}%
\pgfsetstrokecolor{currentstroke}%
\pgfsetdash{}{0pt}%
\pgfpathmoveto{\pgfqpoint{0.375000in}{0.150000in}}%
\pgfpathlineto{\pgfqpoint{2.632282in}{0.150000in}}%
\pgfusepath{stroke}%
\end{pgfscope}%
\begin{pgfscope}%
\pgfpathrectangle{\pgfqpoint{0.375000in}{0.150000in}}{\pgfqpoint{2.325000in}{1.200000in}} %
\pgfusepath{clip}%
\pgfsetrectcap%
\pgfsetroundjoin%
\pgfsetlinewidth{1.003750pt}%
\definecolor{currentstroke}{rgb}{0.000000,0.000000,0.000000}%
\pgfsetstrokecolor{currentstroke}%
\pgfsetdash{}{0pt}%
\pgfpathmoveto{\pgfqpoint{1.503641in}{0.150000in}}%
\pgfpathlineto{\pgfqpoint{1.503641in}{0.721429in}}%
\pgfusepath{stroke}%
\end{pgfscope}%
\begin{pgfscope}%
\pgfpathrectangle{\pgfqpoint{0.375000in}{0.150000in}}{\pgfqpoint{2.325000in}{1.200000in}} %
\pgfusepath{clip}%
\pgfsetrectcap%
\pgfsetroundjoin%
\pgfsetlinewidth{1.003750pt}%
\definecolor{currentstroke}{rgb}{0.000000,0.000000,0.000000}%
\pgfsetstrokecolor{currentstroke}%
\pgfsetdash{}{0pt}%
\pgfpathmoveto{\pgfqpoint{2.067961in}{0.721429in}}%
\pgfpathlineto{\pgfqpoint{2.061013in}{0.810820in}}%
\pgfpathlineto{\pgfqpoint{2.040341in}{0.898010in}}%
\pgfpathlineto{\pgfqpoint{2.006454in}{0.980852in}}%
\pgfpathlineto{\pgfqpoint{1.960186in}{1.057306in}}%
\pgfpathlineto{\pgfqpoint{1.902676in}{1.125490in}}%
\pgfpathlineto{\pgfqpoint{1.835340in}{1.183724in}}%
\pgfpathlineto{\pgfqpoint{1.759837in}{1.230575in}}%
\pgfpathlineto{\pgfqpoint{1.678025in}{1.264889in}}%
\pgfpathlineto{\pgfqpoint{1.591920in}{1.285822in}}%
\pgfpathlineto{\pgfqpoint{1.503641in}{1.292857in}}%
\pgfpathlineto{\pgfqpoint{1.415362in}{1.285822in}}%
\pgfpathlineto{\pgfqpoint{1.329256in}{1.264889in}}%
\pgfpathlineto{\pgfqpoint{1.247445in}{1.230575in}}%
\pgfpathlineto{\pgfqpoint{1.171942in}{1.183724in}}%
\pgfpathlineto{\pgfqpoint{1.104606in}{1.125490in}}%
\pgfpathlineto{\pgfqpoint{1.047096in}{1.057306in}}%
\pgfpathlineto{\pgfqpoint{1.000828in}{0.980852in}}%
\pgfpathlineto{\pgfqpoint{0.966940in}{0.898010in}}%
\pgfpathlineto{\pgfqpoint{0.946268in}{0.810820in}}%
\pgfpathlineto{\pgfqpoint{0.939320in}{0.721429in}}%
\pgfusepath{stroke}%
\end{pgfscope}%
\begin{pgfscope}%
\pgfpathrectangle{\pgfqpoint{0.375000in}{0.150000in}}{\pgfqpoint{2.325000in}{1.200000in}} %
\pgfusepath{clip}%
\pgfsetrectcap%
\pgfsetroundjoin%
\pgfsetlinewidth{1.003750pt}%
\definecolor{currentstroke}{rgb}{0.000000,0.000000,0.000000}%
\pgfsetstrokecolor{currentstroke}%
\pgfsetdash{}{0pt}%
\pgfpathmoveto{\pgfqpoint{0.939320in}{0.721429in}}%
\pgfpathlineto{\pgfqpoint{0.946268in}{0.632037in}}%
\pgfpathlineto{\pgfqpoint{0.966940in}{0.544847in}}%
\pgfpathlineto{\pgfqpoint{1.000828in}{0.462005in}}%
\pgfpathlineto{\pgfqpoint{1.047096in}{0.385551in}}%
\pgfpathlineto{\pgfqpoint{1.104606in}{0.317368in}}%
\pgfpathlineto{\pgfqpoint{1.171942in}{0.259133in}}%
\pgfpathlineto{\pgfqpoint{1.247445in}{0.212282in}}%
\pgfpathlineto{\pgfqpoint{1.329256in}{0.177968in}}%
\pgfpathlineto{\pgfqpoint{1.415362in}{0.157035in}}%
\pgfpathlineto{\pgfqpoint{1.503641in}{0.150000in}}%
\pgfpathlineto{\pgfqpoint{1.591920in}{0.157035in}}%
\pgfpathlineto{\pgfqpoint{1.678025in}{0.177968in}}%
\pgfpathlineto{\pgfqpoint{1.759837in}{0.212282in}}%
\pgfpathlineto{\pgfqpoint{1.835340in}{0.259133in}}%
\pgfpathlineto{\pgfqpoint{1.902676in}{0.317368in}}%
\pgfpathlineto{\pgfqpoint{1.960186in}{0.385551in}}%
\pgfpathlineto{\pgfqpoint{2.006454in}{0.462005in}}%
\pgfpathlineto{\pgfqpoint{2.040341in}{0.544847in}}%
\pgfpathlineto{\pgfqpoint{2.061013in}{0.632037in}}%
\pgfpathlineto{\pgfqpoint{2.067961in}{0.721429in}}%
\pgfusepath{stroke}%
\end{pgfscope}%
\begin{pgfscope}%
\pgfpathrectangle{\pgfqpoint{0.375000in}{0.150000in}}{\pgfqpoint{2.325000in}{1.200000in}} %
\pgfusepath{clip}%
\pgfsetbuttcap%
\pgfsetroundjoin%
\definecolor{currentfill}{rgb}{0.000000,0.000000,1.000000}%
\pgfsetfillcolor{currentfill}%
\pgfsetlinewidth{0.501875pt}%
\definecolor{currentstroke}{rgb}{0.000000,0.000000,0.000000}%
\pgfsetstrokecolor{currentstroke}%
\pgfsetdash{}{0pt}%
\pgfsys@defobject{currentmarker}{\pgfqpoint{-0.041667in}{-0.041667in}}{\pgfqpoint{0.041667in}{0.041667in}}{%
\pgfpathmoveto{\pgfqpoint{0.000000in}{-0.041667in}}%
\pgfpathcurveto{\pgfqpoint{0.011050in}{-0.041667in}}{\pgfqpoint{0.021649in}{-0.037276in}}{\pgfqpoint{0.029463in}{-0.029463in}}%
\pgfpathcurveto{\pgfqpoint{0.037276in}{-0.021649in}}{\pgfqpoint{0.041667in}{-0.011050in}}{\pgfqpoint{0.041667in}{0.000000in}}%
\pgfpathcurveto{\pgfqpoint{0.041667in}{0.011050in}}{\pgfqpoint{0.037276in}{0.021649in}}{\pgfqpoint{0.029463in}{0.029463in}}%
\pgfpathcurveto{\pgfqpoint{0.021649in}{0.037276in}}{\pgfqpoint{0.011050in}{0.041667in}}{\pgfqpoint{0.000000in}{0.041667in}}%
\pgfpathcurveto{\pgfqpoint{-0.011050in}{0.041667in}}{\pgfqpoint{-0.021649in}{0.037276in}}{\pgfqpoint{-0.029463in}{0.029463in}}%
\pgfpathcurveto{\pgfqpoint{-0.037276in}{0.021649in}}{\pgfqpoint{-0.041667in}{0.011050in}}{\pgfqpoint{-0.041667in}{0.000000in}}%
\pgfpathcurveto{\pgfqpoint{-0.041667in}{-0.011050in}}{\pgfqpoint{-0.037276in}{-0.021649in}}{\pgfqpoint{-0.029463in}{-0.029463in}}%
\pgfpathcurveto{\pgfqpoint{-0.021649in}{-0.037276in}}{\pgfqpoint{-0.011050in}{-0.041667in}}{\pgfqpoint{0.000000in}{-0.041667in}}%
\pgfpathclose%
\pgfusepath{stroke,fill}%
}%
\begin{pgfscope}%
\pgfsys@transformshift{1.503641in}{0.721429in}%
\pgfsys@useobject{currentmarker}{}%
\end{pgfscope}%
\end{pgfscope}%
\begin{pgfscope}%
\pgfpathrectangle{\pgfqpoint{0.375000in}{0.150000in}}{\pgfqpoint{2.325000in}{1.200000in}} %
\pgfusepath{clip}%
\pgfsetbuttcap%
\pgfsetroundjoin%
\definecolor{currentfill}{rgb}{0.000000,0.500000,0.000000}%
\pgfsetfillcolor{currentfill}%
\pgfsetlinewidth{0.501875pt}%
\definecolor{currentstroke}{rgb}{0.000000,0.000000,0.000000}%
\pgfsetstrokecolor{currentstroke}%
\pgfsetdash{}{0pt}%
\pgfsys@defobject{currentmarker}{\pgfqpoint{-0.041667in}{-0.041667in}}{\pgfqpoint{0.041667in}{0.041667in}}{%
\pgfpathmoveto{\pgfqpoint{0.000000in}{-0.041667in}}%
\pgfpathcurveto{\pgfqpoint{0.011050in}{-0.041667in}}{\pgfqpoint{0.021649in}{-0.037276in}}{\pgfqpoint{0.029463in}{-0.029463in}}%
\pgfpathcurveto{\pgfqpoint{0.037276in}{-0.021649in}}{\pgfqpoint{0.041667in}{-0.011050in}}{\pgfqpoint{0.041667in}{0.000000in}}%
\pgfpathcurveto{\pgfqpoint{0.041667in}{0.011050in}}{\pgfqpoint{0.037276in}{0.021649in}}{\pgfqpoint{0.029463in}{0.029463in}}%
\pgfpathcurveto{\pgfqpoint{0.021649in}{0.037276in}}{\pgfqpoint{0.011050in}{0.041667in}}{\pgfqpoint{0.000000in}{0.041667in}}%
\pgfpathcurveto{\pgfqpoint{-0.011050in}{0.041667in}}{\pgfqpoint{-0.021649in}{0.037276in}}{\pgfqpoint{-0.029463in}{0.029463in}}%
\pgfpathcurveto{\pgfqpoint{-0.037276in}{0.021649in}}{\pgfqpoint{-0.041667in}{0.011050in}}{\pgfqpoint{-0.041667in}{0.000000in}}%
\pgfpathcurveto{\pgfqpoint{-0.041667in}{-0.011050in}}{\pgfqpoint{-0.037276in}{-0.021649in}}{\pgfqpoint{-0.029463in}{-0.029463in}}%
\pgfpathcurveto{\pgfqpoint{-0.021649in}{-0.037276in}}{\pgfqpoint{-0.011050in}{-0.041667in}}{\pgfqpoint{0.000000in}{-0.041667in}}%
\pgfpathclose%
\pgfusepath{stroke,fill}%
}%
\begin{pgfscope}%
\pgfsys@transformshift{1.503641in}{0.150000in}%
\pgfsys@useobject{currentmarker}{}%
\end{pgfscope}%
\end{pgfscope}%
\begin{pgfscope}%
\pgftext[x=1.435922in,y=0.321429in,right,bottom]{{\sffamily\fontsize{12.000000}{14.400000}\selectfont \(\displaystyle R\)}}%
\end{pgfscope}%
\begin{pgfscope}%
\pgftext[x=1.503641in,y=0.721429in,right,bottom]{{\sffamily\fontsize{12.000000}{14.400000}\selectfont \(\displaystyle CM\)}}%
\end{pgfscope}%
\begin{pgfscope}%
\pgftext[x=2.632282in,y=1.178571in,left,top]{{\sffamily\fontsize{12.000000}{14.400000}\selectfont \(\displaystyle F_p\)}}%
\end{pgfscope}%
\end{pgfpicture}%
\makeatother%
\endgroup%


\bigskip

%TCIMACRO{\TeXButton{2columns}{\begin{multicols}{2}}}%
%BeginExpansion
\begin{multicols}{2}%
%EndExpansion

Let assume friction acting on the wheel goes right, then

%% Creator: Matplotlib, PGF backend
%%
%% To include the figure in your LaTeX document, write
%%   \input{<filename>.pgf}
%%
%% Make sure the required packages are loaded in your preamble
%%   \usepackage{pgf}
%%
%% Figures using additional raster images can only be included by \input if
%% they are in the same directory as the main LaTeX file. For loading figures
%% from other directories you can use the `import` package
%%   \usepackage{import}
%% and then include the figures with
%%   \import{<path to file>}{<filename>.pgf}
%%
%% Matplotlib used the following preamble
%%   \usepackage{fontspec}
%%   \setmainfont{Times New Roman}
%%   \setsansfont{Verdana}
%%   \setmonofont{Courier New}
%%
\begingroup%
\makeatletter%
\begin{pgfpicture}%
\pgfpathrectangle{\pgfpointorigin}{\pgfqpoint{3.000000in}{1.500000in}}%
\pgfusepath{use as bounding box}%
\begin{pgfscope}%
\pgfsetbuttcap%
\pgfsetroundjoin%
\definecolor{currentfill}{rgb}{1.000000,1.000000,1.000000}%
\pgfsetfillcolor{currentfill}%
\pgfsetlinewidth{0.000000pt}%
\definecolor{currentstroke}{rgb}{1.000000,1.000000,1.000000}%
\pgfsetstrokecolor{currentstroke}%
\pgfsetdash{}{0pt}%
\pgfpathmoveto{\pgfqpoint{0.000000in}{0.000000in}}%
\pgfpathlineto{\pgfqpoint{3.000000in}{0.000000in}}%
\pgfpathlineto{\pgfqpoint{3.000000in}{1.500000in}}%
\pgfpathlineto{\pgfqpoint{0.000000in}{1.500000in}}%
\pgfpathclose%
\pgfusepath{fill}%
\end{pgfscope}%
\begin{pgfscope}%
\pgfpathrectangle{\pgfqpoint{0.375000in}{0.150000in}}{\pgfqpoint{2.325000in}{1.200000in}} %
\pgfusepath{clip}%
\pgfsetbuttcap%
\pgfsetroundjoin%
\definecolor{currentfill}{rgb}{0.000000,0.000000,1.000000}%
\pgfsetfillcolor{currentfill}%
\pgfsetlinewidth{1.003750pt}%
\definecolor{currentstroke}{rgb}{0.000000,0.000000,0.000000}%
\pgfsetstrokecolor{currentstroke}%
\pgfsetdash{}{0pt}%
\pgfpathmoveto{\pgfqpoint{2.271117in}{0.234956in}}%
\pgfpathlineto{\pgfqpoint{2.067961in}{0.150000in}}%
\pgfpathlineto{\pgfqpoint{2.067961in}{0.215841in}}%
\pgfpathlineto{\pgfqpoint{1.503641in}{0.215841in}}%
\pgfpathlineto{\pgfqpoint{1.503641in}{0.254071in}}%
\pgfpathlineto{\pgfqpoint{2.067961in}{0.254071in}}%
\pgfpathlineto{\pgfqpoint{2.067961in}{0.319912in}}%
\pgfpathclose%
\pgfusepath{stroke,fill}%
\end{pgfscope}%
\begin{pgfscope}%
\pgfpathrectangle{\pgfqpoint{0.375000in}{0.150000in}}{\pgfqpoint{2.325000in}{1.200000in}} %
\pgfusepath{clip}%
\pgfsetbuttcap%
\pgfsetroundjoin%
\definecolor{currentfill}{rgb}{1.000000,0.000000,0.000000}%
\pgfsetfillcolor{currentfill}%
\pgfsetlinewidth{1.003750pt}%
\definecolor{currentstroke}{rgb}{1.000000,0.000000,0.000000}%
\pgfsetstrokecolor{currentstroke}%
\pgfsetdash{}{0pt}%
\pgfpathmoveto{\pgfqpoint{2.459223in}{0.765929in}}%
\pgfpathlineto{\pgfqpoint{2.256068in}{0.680973in}}%
\pgfpathlineto{\pgfqpoint{2.256068in}{0.746814in}}%
\pgfpathlineto{\pgfqpoint{1.503641in}{0.746814in}}%
\pgfpathlineto{\pgfqpoint{1.503641in}{0.785044in}}%
\pgfpathlineto{\pgfqpoint{2.256068in}{0.785044in}}%
\pgfpathlineto{\pgfqpoint{2.256068in}{0.850885in}}%
\pgfpathclose%
\pgfusepath{stroke,fill}%
\end{pgfscope}%
\begin{pgfscope}%
\pgfpathrectangle{\pgfqpoint{0.375000in}{0.150000in}}{\pgfqpoint{2.325000in}{1.200000in}} %
\pgfusepath{clip}%
\pgfsetbuttcap%
\pgfsetroundjoin%
\definecolor{currentfill}{rgb}{0.000000,0.000000,0.000000}%
\pgfsetfillcolor{currentfill}%
\pgfsetlinewidth{1.003750pt}%
\definecolor{currentstroke}{rgb}{0.000000,0.000000,0.000000}%
\pgfsetstrokecolor{currentstroke}%
\pgfsetdash{}{0pt}%
\pgfpathmoveto{\pgfqpoint{2.700000in}{1.296903in}}%
\pgfpathlineto{\pgfqpoint{2.632282in}{1.243805in}}%
\pgfpathlineto{\pgfqpoint{2.632282in}{1.295841in}}%
\pgfpathlineto{\pgfqpoint{1.503641in}{1.295841in}}%
\pgfpathlineto{\pgfqpoint{1.503641in}{1.297965in}}%
\pgfpathlineto{\pgfqpoint{2.632282in}{1.297965in}}%
\pgfpathlineto{\pgfqpoint{2.632282in}{1.350000in}}%
\pgfpathclose%
\pgfusepath{stroke,fill}%
\end{pgfscope}%
\begin{pgfscope}%
\pgfpathrectangle{\pgfqpoint{0.375000in}{0.150000in}}{\pgfqpoint{2.325000in}{1.200000in}} %
\pgfusepath{clip}%
\pgfsetbuttcap%
\pgfsetroundjoin%
\definecolor{currentfill}{rgb}{0.000000,0.000000,0.000000}%
\pgfsetfillcolor{currentfill}%
\pgfsetlinewidth{1.003750pt}%
\definecolor{currentstroke}{rgb}{0.000000,0.000000,0.000000}%
\pgfsetstrokecolor{currentstroke}%
\pgfsetdash{}{0pt}%
\pgfpathmoveto{\pgfqpoint{1.753288in}{0.925571in}}%
\pgfpathlineto{\pgfqpoint{1.665449in}{0.918611in}}%
\pgfpathlineto{\pgfqpoint{1.697163in}{0.961240in}}%
\pgfpathlineto{\pgfqpoint{1.664802in}{0.982554in}}%
\pgfpathlineto{\pgfqpoint{1.666096in}{0.984294in}}%
\pgfpathlineto{\pgfqpoint{1.698458in}{0.962980in}}%
\pgfpathlineto{\pgfqpoint{1.730172in}{1.005609in}}%
\pgfpathclose%
\pgfusepath{stroke,fill}%
\end{pgfscope}%
\begin{pgfscope}%
\pgfpathrectangle{\pgfqpoint{0.375000in}{0.150000in}}{\pgfqpoint{2.325000in}{1.200000in}} %
\pgfusepath{clip}%
\pgfsetrectcap%
\pgfsetroundjoin%
\pgfsetlinewidth{1.003750pt}%
\definecolor{currentstroke}{rgb}{0.000000,0.000000,1.000000}%
\pgfsetstrokecolor{currentstroke}%
\pgfsetdash{}{0pt}%
\pgfpathmoveto{\pgfqpoint{0.375000in}{0.234956in}}%
\pgfpathlineto{\pgfqpoint{2.632282in}{0.234956in}}%
\pgfusepath{stroke}%
\end{pgfscope}%
\begin{pgfscope}%
\pgfpathrectangle{\pgfqpoint{0.375000in}{0.150000in}}{\pgfqpoint{2.325000in}{1.200000in}} %
\pgfusepath{clip}%
\pgfsetrectcap%
\pgfsetroundjoin%
\pgfsetlinewidth{1.003750pt}%
\definecolor{currentstroke}{rgb}{0.000000,0.000000,0.000000}%
\pgfsetstrokecolor{currentstroke}%
\pgfsetdash{}{0pt}%
\pgfpathmoveto{\pgfqpoint{1.503641in}{0.234956in}}%
\pgfpathlineto{\pgfqpoint{1.503641in}{0.765929in}}%
\pgfusepath{stroke}%
\end{pgfscope}%
\begin{pgfscope}%
\pgfpathrectangle{\pgfqpoint{0.375000in}{0.150000in}}{\pgfqpoint{2.325000in}{1.200000in}} %
\pgfusepath{clip}%
\pgfsetrectcap%
\pgfsetroundjoin%
\pgfsetlinewidth{1.003750pt}%
\definecolor{currentstroke}{rgb}{0.000000,0.000000,0.000000}%
\pgfsetstrokecolor{currentstroke}%
\pgfsetdash{}{0pt}%
\pgfpathmoveto{\pgfqpoint{1.503641in}{1.031416in}}%
\pgfpathlineto{\pgfqpoint{1.515126in}{1.031196in}}%
\pgfpathlineto{\pgfqpoint{1.526592in}{1.030536in}}%
\pgfpathlineto{\pgfqpoint{1.538020in}{1.029438in}}%
\pgfpathlineto{\pgfqpoint{1.549391in}{1.027903in}}%
\pgfpathlineto{\pgfqpoint{1.560686in}{1.025934in}}%
\pgfpathlineto{\pgfqpoint{1.571886in}{1.023533in}}%
\pgfpathlineto{\pgfqpoint{1.582974in}{1.020706in}}%
\pgfpathlineto{\pgfqpoint{1.593930in}{1.017457in}}%
\pgfpathlineto{\pgfqpoint{1.604736in}{1.013790in}}%
\pgfpathlineto{\pgfqpoint{1.615375in}{1.009713in}}%
\pgfpathlineto{\pgfqpoint{1.625828in}{1.005232in}}%
\pgfpathlineto{\pgfqpoint{1.636079in}{1.000354in}}%
\pgfpathlineto{\pgfqpoint{1.646111in}{0.995087in}}%
\pgfpathlineto{\pgfqpoint{1.655906in}{0.989441in}}%
\pgfpathlineto{\pgfqpoint{1.665449in}{0.983424in}}%
\pgfusepath{stroke}%
\end{pgfscope}%
\begin{pgfscope}%
\pgfpathrectangle{\pgfqpoint{0.375000in}{0.150000in}}{\pgfqpoint{2.325000in}{1.200000in}} %
\pgfusepath{clip}%
\pgfsetrectcap%
\pgfsetroundjoin%
\pgfsetlinewidth{1.003750pt}%
\definecolor{currentstroke}{rgb}{0.000000,0.000000,0.000000}%
\pgfsetstrokecolor{currentstroke}%
\pgfsetdash{}{0pt}%
\pgfpathmoveto{\pgfqpoint{2.067961in}{0.765929in}}%
\pgfpathlineto{\pgfqpoint{2.061013in}{0.848992in}}%
\pgfpathlineto{\pgfqpoint{2.040341in}{0.930009in}}%
\pgfpathlineto{\pgfqpoint{2.006454in}{1.006986in}}%
\pgfpathlineto{\pgfqpoint{1.960186in}{1.078028in}}%
\pgfpathlineto{\pgfqpoint{1.902676in}{1.141384in}}%
\pgfpathlineto{\pgfqpoint{1.835340in}{1.195496in}}%
\pgfpathlineto{\pgfqpoint{1.759837in}{1.239030in}}%
\pgfpathlineto{\pgfqpoint{1.678025in}{1.270915in}}%
\pgfpathlineto{\pgfqpoint{1.591920in}{1.290365in}}%
\pgfpathlineto{\pgfqpoint{1.503641in}{1.296903in}}%
\pgfpathlineto{\pgfqpoint{1.415362in}{1.290365in}}%
\pgfpathlineto{\pgfqpoint{1.329256in}{1.270915in}}%
\pgfpathlineto{\pgfqpoint{1.247445in}{1.239030in}}%
\pgfpathlineto{\pgfqpoint{1.171942in}{1.195496in}}%
\pgfpathlineto{\pgfqpoint{1.104606in}{1.141384in}}%
\pgfpathlineto{\pgfqpoint{1.047096in}{1.078028in}}%
\pgfpathlineto{\pgfqpoint{1.000828in}{1.006986in}}%
\pgfpathlineto{\pgfqpoint{0.966940in}{0.930009in}}%
\pgfpathlineto{\pgfqpoint{0.946268in}{0.848992in}}%
\pgfpathlineto{\pgfqpoint{0.939320in}{0.765929in}}%
\pgfusepath{stroke}%
\end{pgfscope}%
\begin{pgfscope}%
\pgfpathrectangle{\pgfqpoint{0.375000in}{0.150000in}}{\pgfqpoint{2.325000in}{1.200000in}} %
\pgfusepath{clip}%
\pgfsetrectcap%
\pgfsetroundjoin%
\pgfsetlinewidth{1.003750pt}%
\definecolor{currentstroke}{rgb}{0.000000,0.000000,0.000000}%
\pgfsetstrokecolor{currentstroke}%
\pgfsetdash{}{0pt}%
\pgfpathmoveto{\pgfqpoint{0.939320in}{0.765929in}}%
\pgfpathlineto{\pgfqpoint{0.946268in}{0.682867in}}%
\pgfpathlineto{\pgfqpoint{0.966940in}{0.601849in}}%
\pgfpathlineto{\pgfqpoint{1.000828in}{0.524872in}}%
\pgfpathlineto{\pgfqpoint{1.047096in}{0.453831in}}%
\pgfpathlineto{\pgfqpoint{1.104606in}{0.390474in}}%
\pgfpathlineto{\pgfqpoint{1.171942in}{0.336363in}}%
\pgfpathlineto{\pgfqpoint{1.247445in}{0.292828in}}%
\pgfpathlineto{\pgfqpoint{1.329256in}{0.260943in}}%
\pgfpathlineto{\pgfqpoint{1.415362in}{0.241493in}}%
\pgfpathlineto{\pgfqpoint{1.503641in}{0.234956in}}%
\pgfpathlineto{\pgfqpoint{1.591920in}{0.241493in}}%
\pgfpathlineto{\pgfqpoint{1.678025in}{0.260943in}}%
\pgfpathlineto{\pgfqpoint{1.759837in}{0.292828in}}%
\pgfpathlineto{\pgfqpoint{1.835340in}{0.336363in}}%
\pgfpathlineto{\pgfqpoint{1.902676in}{0.390474in}}%
\pgfpathlineto{\pgfqpoint{1.960186in}{0.453831in}}%
\pgfpathlineto{\pgfqpoint{2.006454in}{0.524872in}}%
\pgfpathlineto{\pgfqpoint{2.040341in}{0.601849in}}%
\pgfpathlineto{\pgfqpoint{2.061013in}{0.682867in}}%
\pgfpathlineto{\pgfqpoint{2.067961in}{0.765929in}}%
\pgfusepath{stroke}%
\end{pgfscope}%
\begin{pgfscope}%
\pgfpathrectangle{\pgfqpoint{0.375000in}{0.150000in}}{\pgfqpoint{2.325000in}{1.200000in}} %
\pgfusepath{clip}%
\pgfsetbuttcap%
\pgfsetroundjoin%
\definecolor{currentfill}{rgb}{0.000000,0.000000,1.000000}%
\pgfsetfillcolor{currentfill}%
\pgfsetlinewidth{0.501875pt}%
\definecolor{currentstroke}{rgb}{0.000000,0.000000,0.000000}%
\pgfsetstrokecolor{currentstroke}%
\pgfsetdash{}{0pt}%
\pgfsys@defobject{currentmarker}{\pgfqpoint{-0.041667in}{-0.041667in}}{\pgfqpoint{0.041667in}{0.041667in}}{%
\pgfpathmoveto{\pgfqpoint{0.000000in}{-0.041667in}}%
\pgfpathcurveto{\pgfqpoint{0.011050in}{-0.041667in}}{\pgfqpoint{0.021649in}{-0.037276in}}{\pgfqpoint{0.029463in}{-0.029463in}}%
\pgfpathcurveto{\pgfqpoint{0.037276in}{-0.021649in}}{\pgfqpoint{0.041667in}{-0.011050in}}{\pgfqpoint{0.041667in}{0.000000in}}%
\pgfpathcurveto{\pgfqpoint{0.041667in}{0.011050in}}{\pgfqpoint{0.037276in}{0.021649in}}{\pgfqpoint{0.029463in}{0.029463in}}%
\pgfpathcurveto{\pgfqpoint{0.021649in}{0.037276in}}{\pgfqpoint{0.011050in}{0.041667in}}{\pgfqpoint{0.000000in}{0.041667in}}%
\pgfpathcurveto{\pgfqpoint{-0.011050in}{0.041667in}}{\pgfqpoint{-0.021649in}{0.037276in}}{\pgfqpoint{-0.029463in}{0.029463in}}%
\pgfpathcurveto{\pgfqpoint{-0.037276in}{0.021649in}}{\pgfqpoint{-0.041667in}{0.011050in}}{\pgfqpoint{-0.041667in}{0.000000in}}%
\pgfpathcurveto{\pgfqpoint{-0.041667in}{-0.011050in}}{\pgfqpoint{-0.037276in}{-0.021649in}}{\pgfqpoint{-0.029463in}{-0.029463in}}%
\pgfpathcurveto{\pgfqpoint{-0.021649in}{-0.037276in}}{\pgfqpoint{-0.011050in}{-0.041667in}}{\pgfqpoint{0.000000in}{-0.041667in}}%
\pgfpathclose%
\pgfusepath{stroke,fill}%
}%
\begin{pgfscope}%
\pgfsys@transformshift{1.503641in}{0.765929in}%
\pgfsys@useobject{currentmarker}{}%
\end{pgfscope}%
\end{pgfscope}%
\begin{pgfscope}%
\pgfpathrectangle{\pgfqpoint{0.375000in}{0.150000in}}{\pgfqpoint{2.325000in}{1.200000in}} %
\pgfusepath{clip}%
\pgfsetbuttcap%
\pgfsetroundjoin%
\definecolor{currentfill}{rgb}{0.000000,0.500000,0.000000}%
\pgfsetfillcolor{currentfill}%
\pgfsetlinewidth{0.501875pt}%
\definecolor{currentstroke}{rgb}{0.000000,0.000000,0.000000}%
\pgfsetstrokecolor{currentstroke}%
\pgfsetdash{}{0pt}%
\pgfsys@defobject{currentmarker}{\pgfqpoint{-0.041667in}{-0.041667in}}{\pgfqpoint{0.041667in}{0.041667in}}{%
\pgfpathmoveto{\pgfqpoint{0.000000in}{-0.041667in}}%
\pgfpathcurveto{\pgfqpoint{0.011050in}{-0.041667in}}{\pgfqpoint{0.021649in}{-0.037276in}}{\pgfqpoint{0.029463in}{-0.029463in}}%
\pgfpathcurveto{\pgfqpoint{0.037276in}{-0.021649in}}{\pgfqpoint{0.041667in}{-0.011050in}}{\pgfqpoint{0.041667in}{0.000000in}}%
\pgfpathcurveto{\pgfqpoint{0.041667in}{0.011050in}}{\pgfqpoint{0.037276in}{0.021649in}}{\pgfqpoint{0.029463in}{0.029463in}}%
\pgfpathcurveto{\pgfqpoint{0.021649in}{0.037276in}}{\pgfqpoint{0.011050in}{0.041667in}}{\pgfqpoint{0.000000in}{0.041667in}}%
\pgfpathcurveto{\pgfqpoint{-0.011050in}{0.041667in}}{\pgfqpoint{-0.021649in}{0.037276in}}{\pgfqpoint{-0.029463in}{0.029463in}}%
\pgfpathcurveto{\pgfqpoint{-0.037276in}{0.021649in}}{\pgfqpoint{-0.041667in}{0.011050in}}{\pgfqpoint{-0.041667in}{0.000000in}}%
\pgfpathcurveto{\pgfqpoint{-0.041667in}{-0.011050in}}{\pgfqpoint{-0.037276in}{-0.021649in}}{\pgfqpoint{-0.029463in}{-0.029463in}}%
\pgfpathcurveto{\pgfqpoint{-0.021649in}{-0.037276in}}{\pgfqpoint{-0.011050in}{-0.041667in}}{\pgfqpoint{0.000000in}{-0.041667in}}%
\pgfpathclose%
\pgfusepath{stroke,fill}%
}%
\begin{pgfscope}%
\pgfsys@transformshift{1.503641in}{0.234956in}%
\pgfsys@useobject{currentmarker}{}%
\end{pgfscope}%
\end{pgfscope}%
\begin{pgfscope}%
\pgftext[x=1.435922in,y=0.394248in,right,bottom]{{\sffamily\fontsize{12.000000}{14.400000}\selectfont \(\displaystyle R\)}}%
\end{pgfscope}%
\begin{pgfscope}%
\pgftext[x=1.661650in,y=1.020796in,left,bottom]{{\sffamily\fontsize{12.000000}{14.400000}\selectfont \(\displaystyle \alpha\)}}%
\end{pgfscope}%
\begin{pgfscope}%
\pgftext[x=1.503641in,y=0.765929in,right,bottom]{{\sffamily\fontsize{12.000000}{14.400000}\selectfont \(\displaystyle CM\)}}%
\end{pgfscope}%
\begin{pgfscope}%
\pgftext[x=2.180825in,y=0.234956in,left,bottom]{{\sffamily\fontsize{12.000000}{14.400000}\selectfont \(\displaystyle F_{fr}\)}}%
\end{pgfscope}%
\begin{pgfscope}%
\definecolor{textcolor}{rgb}{1.000000,0.000000,0.000000}%
\pgfsetstrokecolor{textcolor}%
\pgfsetfillcolor{textcolor}%
\pgftext[x=2.481796in,y=0.765929in,left,bottom]{{\sffamily\fontsize{12.000000}{14.400000}\selectfont \(\displaystyle a\)}}%
\end{pgfscope}%
\begin{pgfscope}%
\pgftext[x=2.632282in,y=1.190708in,left,top]{{\sffamily\fontsize{12.000000}{14.400000}\selectfont \(\displaystyle F_p\)}}%
\end{pgfscope}%
\end{pgfpicture}%
\makeatother%
\endgroup%


from 2nd law%
\begin{eqnarray*}
&&\left\{ 
\begin{array}{c}
F_{p}+F_{fr}=ma \\ 
\left( F_{p}-F_{fr}\right) R=I\alpha =I\frac{a}{R}%
\end{array}%
\right. \\
&\Rightarrow &F_{fr}=\frac{1}{2}\left( m-\frac{I}{R^{2}}\right) a
\end{eqnarray*}

So if $mR^{2}>I$, then $F_{fr}$ goes to the right!

\bigskip

If frction acting on the wheel goes to the left,

%% Creator: Matplotlib, PGF backend
%%
%% To include the figure in your LaTeX document, write
%%   \input{<filename>.pgf}
%%
%% Make sure the required packages are loaded in your preamble
%%   \usepackage{pgf}
%%
%% Figures using additional raster images can only be included by \input if
%% they are in the same directory as the main LaTeX file. For loading figures
%% from other directories you can use the `import` package
%%   \usepackage{import}
%% and then include the figures with
%%   \import{<path to file>}{<filename>.pgf}
%%
%% Matplotlib used the following preamble
%%   \usepackage{fontspec}
%%   \setmainfont{Times New Roman}
%%   \setsansfont{Verdana}
%%   \setmonofont{Courier New}
%%
\begingroup%
\makeatletter%
\begin{pgfpicture}%
\pgfpathrectangle{\pgfpointorigin}{\pgfqpoint{3.000000in}{1.500000in}}%
\pgfusepath{use as bounding box}%
\begin{pgfscope}%
\pgfsetbuttcap%
\pgfsetroundjoin%
\definecolor{currentfill}{rgb}{1.000000,1.000000,1.000000}%
\pgfsetfillcolor{currentfill}%
\pgfsetlinewidth{0.000000pt}%
\definecolor{currentstroke}{rgb}{1.000000,1.000000,1.000000}%
\pgfsetstrokecolor{currentstroke}%
\pgfsetdash{}{0pt}%
\pgfpathmoveto{\pgfqpoint{0.000000in}{0.000000in}}%
\pgfpathlineto{\pgfqpoint{3.000000in}{0.000000in}}%
\pgfpathlineto{\pgfqpoint{3.000000in}{1.500000in}}%
\pgfpathlineto{\pgfqpoint{0.000000in}{1.500000in}}%
\pgfpathclose%
\pgfusepath{fill}%
\end{pgfscope}%
\begin{pgfscope}%
\pgfpathrectangle{\pgfqpoint{0.375000in}{0.150000in}}{\pgfqpoint{2.325000in}{1.200000in}} %
\pgfusepath{clip}%
\pgfsetbuttcap%
\pgfsetroundjoin%
\definecolor{currentfill}{rgb}{0.000000,0.000000,1.000000}%
\pgfsetfillcolor{currentfill}%
\pgfsetlinewidth{1.003750pt}%
\definecolor{currentstroke}{rgb}{0.000000,0.000000,0.000000}%
\pgfsetstrokecolor{currentstroke}%
\pgfsetdash{}{0pt}%
\pgfpathmoveto{\pgfqpoint{0.736165in}{0.234956in}}%
\pgfpathlineto{\pgfqpoint{0.939320in}{0.319912in}}%
\pgfpathlineto{\pgfqpoint{0.939320in}{0.254071in}}%
\pgfpathlineto{\pgfqpoint{1.503641in}{0.254071in}}%
\pgfpathlineto{\pgfqpoint{1.503641in}{0.215841in}}%
\pgfpathlineto{\pgfqpoint{0.939320in}{0.215841in}}%
\pgfpathlineto{\pgfqpoint{0.939320in}{0.150000in}}%
\pgfpathclose%
\pgfusepath{stroke,fill}%
\end{pgfscope}%
\begin{pgfscope}%
\pgfpathrectangle{\pgfqpoint{0.375000in}{0.150000in}}{\pgfqpoint{2.325000in}{1.200000in}} %
\pgfusepath{clip}%
\pgfsetbuttcap%
\pgfsetroundjoin%
\definecolor{currentfill}{rgb}{1.000000,0.000000,0.000000}%
\pgfsetfillcolor{currentfill}%
\pgfsetlinewidth{1.003750pt}%
\definecolor{currentstroke}{rgb}{1.000000,0.000000,0.000000}%
\pgfsetstrokecolor{currentstroke}%
\pgfsetdash{}{0pt}%
\pgfpathmoveto{\pgfqpoint{2.459223in}{0.765929in}}%
\pgfpathlineto{\pgfqpoint{2.256068in}{0.680973in}}%
\pgfpathlineto{\pgfqpoint{2.256068in}{0.746814in}}%
\pgfpathlineto{\pgfqpoint{1.503641in}{0.746814in}}%
\pgfpathlineto{\pgfqpoint{1.503641in}{0.785044in}}%
\pgfpathlineto{\pgfqpoint{2.256068in}{0.785044in}}%
\pgfpathlineto{\pgfqpoint{2.256068in}{0.850885in}}%
\pgfpathclose%
\pgfusepath{stroke,fill}%
\end{pgfscope}%
\begin{pgfscope}%
\pgfpathrectangle{\pgfqpoint{0.375000in}{0.150000in}}{\pgfqpoint{2.325000in}{1.200000in}} %
\pgfusepath{clip}%
\pgfsetbuttcap%
\pgfsetroundjoin%
\definecolor{currentfill}{rgb}{0.000000,0.000000,0.000000}%
\pgfsetfillcolor{currentfill}%
\pgfsetlinewidth{1.003750pt}%
\definecolor{currentstroke}{rgb}{0.000000,0.000000,0.000000}%
\pgfsetstrokecolor{currentstroke}%
\pgfsetdash{}{0pt}%
\pgfpathmoveto{\pgfqpoint{2.700000in}{1.296903in}}%
\pgfpathlineto{\pgfqpoint{2.632282in}{1.243805in}}%
\pgfpathlineto{\pgfqpoint{2.632282in}{1.295841in}}%
\pgfpathlineto{\pgfqpoint{1.503641in}{1.295841in}}%
\pgfpathlineto{\pgfqpoint{1.503641in}{1.297965in}}%
\pgfpathlineto{\pgfqpoint{2.632282in}{1.297965in}}%
\pgfpathlineto{\pgfqpoint{2.632282in}{1.350000in}}%
\pgfpathclose%
\pgfusepath{stroke,fill}%
\end{pgfscope}%
\begin{pgfscope}%
\pgfpathrectangle{\pgfqpoint{0.375000in}{0.150000in}}{\pgfqpoint{2.325000in}{1.200000in}} %
\pgfusepath{clip}%
\pgfsetbuttcap%
\pgfsetroundjoin%
\definecolor{currentfill}{rgb}{0.000000,0.000000,0.000000}%
\pgfsetfillcolor{currentfill}%
\pgfsetlinewidth{1.003750pt}%
\definecolor{currentstroke}{rgb}{0.000000,0.000000,0.000000}%
\pgfsetstrokecolor{currentstroke}%
\pgfsetdash{}{0pt}%
\pgfpathmoveto{\pgfqpoint{1.753288in}{0.925571in}}%
\pgfpathlineto{\pgfqpoint{1.665449in}{0.918611in}}%
\pgfpathlineto{\pgfqpoint{1.697163in}{0.961240in}}%
\pgfpathlineto{\pgfqpoint{1.664802in}{0.982554in}}%
\pgfpathlineto{\pgfqpoint{1.666096in}{0.984294in}}%
\pgfpathlineto{\pgfqpoint{1.698458in}{0.962980in}}%
\pgfpathlineto{\pgfqpoint{1.730172in}{1.005609in}}%
\pgfpathclose%
\pgfusepath{stroke,fill}%
\end{pgfscope}%
\begin{pgfscope}%
\pgfpathrectangle{\pgfqpoint{0.375000in}{0.150000in}}{\pgfqpoint{2.325000in}{1.200000in}} %
\pgfusepath{clip}%
\pgfsetrectcap%
\pgfsetroundjoin%
\pgfsetlinewidth{1.003750pt}%
\definecolor{currentstroke}{rgb}{0.000000,0.000000,1.000000}%
\pgfsetstrokecolor{currentstroke}%
\pgfsetdash{}{0pt}%
\pgfpathmoveto{\pgfqpoint{0.375000in}{0.234956in}}%
\pgfpathlineto{\pgfqpoint{2.632282in}{0.234956in}}%
\pgfusepath{stroke}%
\end{pgfscope}%
\begin{pgfscope}%
\pgfpathrectangle{\pgfqpoint{0.375000in}{0.150000in}}{\pgfqpoint{2.325000in}{1.200000in}} %
\pgfusepath{clip}%
\pgfsetrectcap%
\pgfsetroundjoin%
\pgfsetlinewidth{1.003750pt}%
\definecolor{currentstroke}{rgb}{0.000000,0.000000,0.000000}%
\pgfsetstrokecolor{currentstroke}%
\pgfsetdash{}{0pt}%
\pgfpathmoveto{\pgfqpoint{1.503641in}{0.234956in}}%
\pgfpathlineto{\pgfqpoint{1.503641in}{0.765929in}}%
\pgfusepath{stroke}%
\end{pgfscope}%
\begin{pgfscope}%
\pgfpathrectangle{\pgfqpoint{0.375000in}{0.150000in}}{\pgfqpoint{2.325000in}{1.200000in}} %
\pgfusepath{clip}%
\pgfsetrectcap%
\pgfsetroundjoin%
\pgfsetlinewidth{1.003750pt}%
\definecolor{currentstroke}{rgb}{0.000000,0.000000,0.000000}%
\pgfsetstrokecolor{currentstroke}%
\pgfsetdash{}{0pt}%
\pgfpathmoveto{\pgfqpoint{1.503641in}{1.031416in}}%
\pgfpathlineto{\pgfqpoint{1.515126in}{1.031196in}}%
\pgfpathlineto{\pgfqpoint{1.526592in}{1.030536in}}%
\pgfpathlineto{\pgfqpoint{1.538020in}{1.029438in}}%
\pgfpathlineto{\pgfqpoint{1.549391in}{1.027903in}}%
\pgfpathlineto{\pgfqpoint{1.560686in}{1.025934in}}%
\pgfpathlineto{\pgfqpoint{1.571886in}{1.023533in}}%
\pgfpathlineto{\pgfqpoint{1.582974in}{1.020706in}}%
\pgfpathlineto{\pgfqpoint{1.593930in}{1.017457in}}%
\pgfpathlineto{\pgfqpoint{1.604736in}{1.013790in}}%
\pgfpathlineto{\pgfqpoint{1.615375in}{1.009713in}}%
\pgfpathlineto{\pgfqpoint{1.625828in}{1.005232in}}%
\pgfpathlineto{\pgfqpoint{1.636079in}{1.000354in}}%
\pgfpathlineto{\pgfqpoint{1.646111in}{0.995087in}}%
\pgfpathlineto{\pgfqpoint{1.655906in}{0.989441in}}%
\pgfpathlineto{\pgfqpoint{1.665449in}{0.983424in}}%
\pgfusepath{stroke}%
\end{pgfscope}%
\begin{pgfscope}%
\pgfpathrectangle{\pgfqpoint{0.375000in}{0.150000in}}{\pgfqpoint{2.325000in}{1.200000in}} %
\pgfusepath{clip}%
\pgfsetrectcap%
\pgfsetroundjoin%
\pgfsetlinewidth{1.003750pt}%
\definecolor{currentstroke}{rgb}{0.000000,0.000000,0.000000}%
\pgfsetstrokecolor{currentstroke}%
\pgfsetdash{}{0pt}%
\pgfpathmoveto{\pgfqpoint{2.067961in}{0.765929in}}%
\pgfpathlineto{\pgfqpoint{2.061013in}{0.848992in}}%
\pgfpathlineto{\pgfqpoint{2.040341in}{0.930009in}}%
\pgfpathlineto{\pgfqpoint{2.006454in}{1.006986in}}%
\pgfpathlineto{\pgfqpoint{1.960186in}{1.078028in}}%
\pgfpathlineto{\pgfqpoint{1.902676in}{1.141384in}}%
\pgfpathlineto{\pgfqpoint{1.835340in}{1.195496in}}%
\pgfpathlineto{\pgfqpoint{1.759837in}{1.239030in}}%
\pgfpathlineto{\pgfqpoint{1.678025in}{1.270915in}}%
\pgfpathlineto{\pgfqpoint{1.591920in}{1.290365in}}%
\pgfpathlineto{\pgfqpoint{1.503641in}{1.296903in}}%
\pgfpathlineto{\pgfqpoint{1.415362in}{1.290365in}}%
\pgfpathlineto{\pgfqpoint{1.329256in}{1.270915in}}%
\pgfpathlineto{\pgfqpoint{1.247445in}{1.239030in}}%
\pgfpathlineto{\pgfqpoint{1.171942in}{1.195496in}}%
\pgfpathlineto{\pgfqpoint{1.104606in}{1.141384in}}%
\pgfpathlineto{\pgfqpoint{1.047096in}{1.078028in}}%
\pgfpathlineto{\pgfqpoint{1.000828in}{1.006986in}}%
\pgfpathlineto{\pgfqpoint{0.966940in}{0.930009in}}%
\pgfpathlineto{\pgfqpoint{0.946268in}{0.848992in}}%
\pgfpathlineto{\pgfqpoint{0.939320in}{0.765929in}}%
\pgfusepath{stroke}%
\end{pgfscope}%
\begin{pgfscope}%
\pgfpathrectangle{\pgfqpoint{0.375000in}{0.150000in}}{\pgfqpoint{2.325000in}{1.200000in}} %
\pgfusepath{clip}%
\pgfsetrectcap%
\pgfsetroundjoin%
\pgfsetlinewidth{1.003750pt}%
\definecolor{currentstroke}{rgb}{0.000000,0.000000,0.000000}%
\pgfsetstrokecolor{currentstroke}%
\pgfsetdash{}{0pt}%
\pgfpathmoveto{\pgfqpoint{0.939320in}{0.765929in}}%
\pgfpathlineto{\pgfqpoint{0.946268in}{0.682867in}}%
\pgfpathlineto{\pgfqpoint{0.966940in}{0.601849in}}%
\pgfpathlineto{\pgfqpoint{1.000828in}{0.524872in}}%
\pgfpathlineto{\pgfqpoint{1.047096in}{0.453831in}}%
\pgfpathlineto{\pgfqpoint{1.104606in}{0.390474in}}%
\pgfpathlineto{\pgfqpoint{1.171942in}{0.336363in}}%
\pgfpathlineto{\pgfqpoint{1.247445in}{0.292828in}}%
\pgfpathlineto{\pgfqpoint{1.329256in}{0.260943in}}%
\pgfpathlineto{\pgfqpoint{1.415362in}{0.241493in}}%
\pgfpathlineto{\pgfqpoint{1.503641in}{0.234956in}}%
\pgfpathlineto{\pgfqpoint{1.591920in}{0.241493in}}%
\pgfpathlineto{\pgfqpoint{1.678025in}{0.260943in}}%
\pgfpathlineto{\pgfqpoint{1.759837in}{0.292828in}}%
\pgfpathlineto{\pgfqpoint{1.835340in}{0.336363in}}%
\pgfpathlineto{\pgfqpoint{1.902676in}{0.390474in}}%
\pgfpathlineto{\pgfqpoint{1.960186in}{0.453831in}}%
\pgfpathlineto{\pgfqpoint{2.006454in}{0.524872in}}%
\pgfpathlineto{\pgfqpoint{2.040341in}{0.601849in}}%
\pgfpathlineto{\pgfqpoint{2.061013in}{0.682867in}}%
\pgfpathlineto{\pgfqpoint{2.067961in}{0.765929in}}%
\pgfusepath{stroke}%
\end{pgfscope}%
\begin{pgfscope}%
\pgfpathrectangle{\pgfqpoint{0.375000in}{0.150000in}}{\pgfqpoint{2.325000in}{1.200000in}} %
\pgfusepath{clip}%
\pgfsetbuttcap%
\pgfsetroundjoin%
\definecolor{currentfill}{rgb}{0.000000,0.000000,1.000000}%
\pgfsetfillcolor{currentfill}%
\pgfsetlinewidth{0.501875pt}%
\definecolor{currentstroke}{rgb}{0.000000,0.000000,0.000000}%
\pgfsetstrokecolor{currentstroke}%
\pgfsetdash{}{0pt}%
\pgfsys@defobject{currentmarker}{\pgfqpoint{-0.041667in}{-0.041667in}}{\pgfqpoint{0.041667in}{0.041667in}}{%
\pgfpathmoveto{\pgfqpoint{0.000000in}{-0.041667in}}%
\pgfpathcurveto{\pgfqpoint{0.011050in}{-0.041667in}}{\pgfqpoint{0.021649in}{-0.037276in}}{\pgfqpoint{0.029463in}{-0.029463in}}%
\pgfpathcurveto{\pgfqpoint{0.037276in}{-0.021649in}}{\pgfqpoint{0.041667in}{-0.011050in}}{\pgfqpoint{0.041667in}{0.000000in}}%
\pgfpathcurveto{\pgfqpoint{0.041667in}{0.011050in}}{\pgfqpoint{0.037276in}{0.021649in}}{\pgfqpoint{0.029463in}{0.029463in}}%
\pgfpathcurveto{\pgfqpoint{0.021649in}{0.037276in}}{\pgfqpoint{0.011050in}{0.041667in}}{\pgfqpoint{0.000000in}{0.041667in}}%
\pgfpathcurveto{\pgfqpoint{-0.011050in}{0.041667in}}{\pgfqpoint{-0.021649in}{0.037276in}}{\pgfqpoint{-0.029463in}{0.029463in}}%
\pgfpathcurveto{\pgfqpoint{-0.037276in}{0.021649in}}{\pgfqpoint{-0.041667in}{0.011050in}}{\pgfqpoint{-0.041667in}{0.000000in}}%
\pgfpathcurveto{\pgfqpoint{-0.041667in}{-0.011050in}}{\pgfqpoint{-0.037276in}{-0.021649in}}{\pgfqpoint{-0.029463in}{-0.029463in}}%
\pgfpathcurveto{\pgfqpoint{-0.021649in}{-0.037276in}}{\pgfqpoint{-0.011050in}{-0.041667in}}{\pgfqpoint{0.000000in}{-0.041667in}}%
\pgfpathclose%
\pgfusepath{stroke,fill}%
}%
\begin{pgfscope}%
\pgfsys@transformshift{1.503641in}{0.765929in}%
\pgfsys@useobject{currentmarker}{}%
\end{pgfscope}%
\end{pgfscope}%
\begin{pgfscope}%
\pgfpathrectangle{\pgfqpoint{0.375000in}{0.150000in}}{\pgfqpoint{2.325000in}{1.200000in}} %
\pgfusepath{clip}%
\pgfsetbuttcap%
\pgfsetroundjoin%
\definecolor{currentfill}{rgb}{0.000000,0.500000,0.000000}%
\pgfsetfillcolor{currentfill}%
\pgfsetlinewidth{0.501875pt}%
\definecolor{currentstroke}{rgb}{0.000000,0.000000,0.000000}%
\pgfsetstrokecolor{currentstroke}%
\pgfsetdash{}{0pt}%
\pgfsys@defobject{currentmarker}{\pgfqpoint{-0.041667in}{-0.041667in}}{\pgfqpoint{0.041667in}{0.041667in}}{%
\pgfpathmoveto{\pgfqpoint{0.000000in}{-0.041667in}}%
\pgfpathcurveto{\pgfqpoint{0.011050in}{-0.041667in}}{\pgfqpoint{0.021649in}{-0.037276in}}{\pgfqpoint{0.029463in}{-0.029463in}}%
\pgfpathcurveto{\pgfqpoint{0.037276in}{-0.021649in}}{\pgfqpoint{0.041667in}{-0.011050in}}{\pgfqpoint{0.041667in}{0.000000in}}%
\pgfpathcurveto{\pgfqpoint{0.041667in}{0.011050in}}{\pgfqpoint{0.037276in}{0.021649in}}{\pgfqpoint{0.029463in}{0.029463in}}%
\pgfpathcurveto{\pgfqpoint{0.021649in}{0.037276in}}{\pgfqpoint{0.011050in}{0.041667in}}{\pgfqpoint{0.000000in}{0.041667in}}%
\pgfpathcurveto{\pgfqpoint{-0.011050in}{0.041667in}}{\pgfqpoint{-0.021649in}{0.037276in}}{\pgfqpoint{-0.029463in}{0.029463in}}%
\pgfpathcurveto{\pgfqpoint{-0.037276in}{0.021649in}}{\pgfqpoint{-0.041667in}{0.011050in}}{\pgfqpoint{-0.041667in}{0.000000in}}%
\pgfpathcurveto{\pgfqpoint{-0.041667in}{-0.011050in}}{\pgfqpoint{-0.037276in}{-0.021649in}}{\pgfqpoint{-0.029463in}{-0.029463in}}%
\pgfpathcurveto{\pgfqpoint{-0.021649in}{-0.037276in}}{\pgfqpoint{-0.011050in}{-0.041667in}}{\pgfqpoint{0.000000in}{-0.041667in}}%
\pgfpathclose%
\pgfusepath{stroke,fill}%
}%
\begin{pgfscope}%
\pgfsys@transformshift{1.503641in}{0.234956in}%
\pgfsys@useobject{currentmarker}{}%
\end{pgfscope}%
\end{pgfscope}%
\begin{pgfscope}%
\pgftext[x=1.435922in,y=0.394248in,right,bottom]{{\sffamily\fontsize{12.000000}{14.400000}\selectfont \(\displaystyle R\)}}%
\end{pgfscope}%
\begin{pgfscope}%
\pgftext[x=1.661650in,y=1.020796in,left,bottom]{{\sffamily\fontsize{12.000000}{14.400000}\selectfont \(\displaystyle \alpha\)}}%
\end{pgfscope}%
\begin{pgfscope}%
\pgftext[x=1.503641in,y=0.765929in,right,bottom]{{\sffamily\fontsize{12.000000}{14.400000}\selectfont \(\displaystyle CM\)}}%
\end{pgfscope}%
\begin{pgfscope}%
\pgftext[x=0.713592in,y=0.298673in,left,bottom]{{\sffamily\fontsize{12.000000}{14.400000}\selectfont \(\displaystyle F_{fr}\)}}%
\end{pgfscope}%
\begin{pgfscope}%
\definecolor{textcolor}{rgb}{1.000000,0.000000,0.000000}%
\pgfsetstrokecolor{textcolor}%
\pgfsetfillcolor{textcolor}%
\pgftext[x=2.481796in,y=0.765929in,left,bottom]{{\sffamily\fontsize{12.000000}{14.400000}\selectfont \(\displaystyle a\)}}%
\end{pgfscope}%
\begin{pgfscope}%
\pgftext[x=2.632282in,y=1.190708in,left,top]{{\sffamily\fontsize{12.000000}{14.400000}\selectfont \(\displaystyle F_p\)}}%
\end{pgfscope}%
\end{pgfpicture}%
\makeatother%
\endgroup%


\begin{eqnarray*}
&&\left\{ 
\begin{array}{c}
F_{p}-F_{fr}=ma \\ 
\left( F_{p}+F_{fr}\right) R=I\alpha =I\frac{a}{R}%
\end{array}%
\right. \\
&\Rightarrow &F_{fr}=\frac{1}{2}\left( \frac{I}{R^{2}}-m\right) a
\end{eqnarray*}

So if $mR^{2}<I$, then $F_{fr}$ goes to the left!

\bigskip

%TCIMACRO{\TeXButton{end2columns}{\end{multicols}}}%
%BeginExpansion
\end{multicols}%
%EndExpansion

\bigskip

In this case the direction of friction depends on the condition $mR^{2}>I$
or $mR^{2}<I$. We can also recgonize in the first senario if $mR^{2}<I$ then 
$F_{fr}$ is negative, so the friction should be opposite to the positive
direction we assume (to the right). This way we can save the time of working
on the 2nd case.

\bigskip

\newpage

\subsection{Study Cases}

Determine the direction of friction force acting on the roller. (Assume
rolling without slipping.)

\begin{problem}
An applied force $F_{p}$ passing through CM acts on the roller. The
magnitude of $F_{p}$ is larger than $mg\sin \theta $, so the roller is
accelerating up the hill.
\end{problem}

%% Creator: Matplotlib, PGF backend
%%
%% To include the figure in your LaTeX document, write
%%   \input{<filename>.pgf}
%%
%% Make sure the required packages are loaded in your preamble
%%   \usepackage{pgf}
%%
%% Figures using additional raster images can only be included by \input if
%% they are in the same directory as the main LaTeX file. For loading figures
%% from other directories you can use the `import` package
%%   \usepackage{import}
%% and then include the figures with
%%   \import{<path to file>}{<filename>.pgf}
%%
%% Matplotlib used the following preamble
%%   \usepackage{fontspec}
%%   \setmainfont{Times New Roman}
%%   \setsansfont{Verdana}
%%   \setmonofont{Courier New}
%%
\begingroup%
\makeatletter%
\begin{pgfpicture}%
\pgfpathrectangle{\pgfpointorigin}{\pgfqpoint{3.000000in}{3.000000in}}%
\pgfusepath{use as bounding box}%
\begin{pgfscope}%
\pgfsetbuttcap%
\pgfsetroundjoin%
\definecolor{currentfill}{rgb}{1.000000,1.000000,1.000000}%
\pgfsetfillcolor{currentfill}%
\pgfsetlinewidth{0.000000pt}%
\definecolor{currentstroke}{rgb}{1.000000,1.000000,1.000000}%
\pgfsetstrokecolor{currentstroke}%
\pgfsetdash{}{0pt}%
\pgfpathmoveto{\pgfqpoint{0.000000in}{0.000000in}}%
\pgfpathlineto{\pgfqpoint{3.000000in}{0.000000in}}%
\pgfpathlineto{\pgfqpoint{3.000000in}{3.000000in}}%
\pgfpathlineto{\pgfqpoint{0.000000in}{3.000000in}}%
\pgfpathclose%
\pgfusepath{fill}%
\end{pgfscope}%
\begin{pgfscope}%
\pgfpathrectangle{\pgfqpoint{0.375000in}{0.300000in}}{\pgfqpoint{2.325000in}{2.400000in}} %
\pgfusepath{clip}%
\pgfsetbuttcap%
\pgfsetroundjoin%
\definecolor{currentfill}{rgb}{0.000000,0.000000,1.000000}%
\pgfsetfillcolor{currentfill}%
\pgfsetlinewidth{1.003750pt}%
\definecolor{currentstroke}{rgb}{0.000000,0.000000,0.000000}%
\pgfsetstrokecolor{currentstroke}%
\pgfsetdash{}{0pt}%
\pgfpathmoveto{\pgfqpoint{1.235283in}{0.822849in}}%
\pgfpathlineto{\pgfqpoint{1.183616in}{0.942849in}}%
\pgfpathlineto{\pgfqpoint{1.223658in}{0.942849in}}%
\pgfpathlineto{\pgfqpoint{1.223658in}{1.695214in}}%
\pgfpathlineto{\pgfqpoint{1.246908in}{1.695214in}}%
\pgfpathlineto{\pgfqpoint{1.246908in}{0.942849in}}%
\pgfpathlineto{\pgfqpoint{1.286949in}{0.942849in}}%
\pgfpathclose%
\pgfusepath{stroke,fill}%
\end{pgfscope}%
\begin{pgfscope}%
\pgfpathrectangle{\pgfqpoint{0.375000in}{0.300000in}}{\pgfqpoint{2.325000in}{2.400000in}} %
\pgfusepath{clip}%
\pgfsetbuttcap%
\pgfsetroundjoin%
\definecolor{currentfill}{rgb}{1.000000,0.000000,0.000000}%
\pgfsetfillcolor{currentfill}%
\pgfsetlinewidth{1.003750pt}%
\definecolor{currentstroke}{rgb}{1.000000,0.000000,0.000000}%
\pgfsetstrokecolor{currentstroke}%
\pgfsetdash{}{0pt}%
\pgfpathmoveto{\pgfqpoint{1.605908in}{2.685897in}}%
\pgfpathlineto{\pgfqpoint{1.536258in}{2.490427in}}%
\pgfpathlineto{\pgfqpoint{1.487820in}{2.577030in}}%
\pgfpathlineto{\pgfqpoint{0.928512in}{2.243696in}}%
\pgfpathlineto{\pgfqpoint{0.896220in}{2.301431in}}%
\pgfpathlineto{\pgfqpoint{1.455528in}{2.634765in}}%
\pgfpathlineto{\pgfqpoint{1.413449in}{2.710000in}}%
\pgfpathclose%
\pgfusepath{stroke,fill}%
\end{pgfscope}%
\begin{pgfscope}%
\pgfpathrectangle{\pgfqpoint{0.375000in}{0.300000in}}{\pgfqpoint{2.325000in}{2.400000in}} %
\pgfusepath{clip}%
\pgfsetbuttcap%
\pgfsetroundjoin%
\definecolor{currentfill}{rgb}{0.000000,0.000000,1.000000}%
\pgfsetfillcolor{currentfill}%
\pgfsetlinewidth{1.003750pt}%
\definecolor{currentstroke}{rgb}{0.000000,0.000000,0.000000}%
\pgfsetstrokecolor{currentstroke}%
\pgfsetdash{}{0pt}%
\pgfpathmoveto{\pgfqpoint{1.895266in}{2.088547in}}%
\pgfpathlineto{\pgfqpoint{1.820424in}{1.982359in}}%
\pgfpathlineto{\pgfqpoint{1.800403in}{2.018155in}}%
\pgfpathlineto{\pgfqpoint{1.241095in}{1.684821in}}%
\pgfpathlineto{\pgfqpoint{1.229470in}{1.705606in}}%
\pgfpathlineto{\pgfqpoint{1.788778in}{2.038939in}}%
\pgfpathlineto{\pgfqpoint{1.768758in}{2.074735in}}%
\pgfpathclose%
\pgfusepath{stroke,fill}%
\end{pgfscope}%
\begin{pgfscope}%
\pgfpathrectangle{\pgfqpoint{0.375000in}{0.300000in}}{\pgfqpoint{2.325000in}{2.400000in}} %
\pgfusepath{clip}%
\pgfsetrectcap%
\pgfsetroundjoin%
\pgfsetlinewidth{1.003750pt}%
\definecolor{currentstroke}{rgb}{0.000000,0.000000,1.000000}%
\pgfsetstrokecolor{currentstroke}%
\pgfsetdash{}{0pt}%
\pgfpathmoveto{\pgfqpoint{0.375000in}{0.566667in}}%
\pgfpathlineto{\pgfqpoint{2.612232in}{0.566667in}}%
\pgfusepath{stroke}%
\end{pgfscope}%
\begin{pgfscope}%
\pgfpathrectangle{\pgfqpoint{0.375000in}{0.300000in}}{\pgfqpoint{2.325000in}{2.400000in}} %
\pgfusepath{clip}%
\pgfsetrectcap%
\pgfsetroundjoin%
\pgfsetlinewidth{1.003750pt}%
\definecolor{currentstroke}{rgb}{0.000000,0.000000,1.000000}%
\pgfsetstrokecolor{currentstroke}%
\pgfsetdash{}{0pt}%
\pgfpathmoveto{\pgfqpoint{2.612232in}{0.566667in}}%
\pgfpathlineto{\pgfqpoint{2.612232in}{1.900000in}}%
\pgfusepath{stroke}%
\end{pgfscope}%
\begin{pgfscope}%
\pgfpathrectangle{\pgfqpoint{0.375000in}{0.300000in}}{\pgfqpoint{2.325000in}{2.400000in}} %
\pgfusepath{clip}%
\pgfsetrectcap%
\pgfsetroundjoin%
\pgfsetlinewidth{1.003750pt}%
\definecolor{currentstroke}{rgb}{0.000000,0.000000,1.000000}%
\pgfsetstrokecolor{currentstroke}%
\pgfsetdash{}{0pt}%
\pgfpathmoveto{\pgfqpoint{0.375000in}{0.566667in}}%
\pgfpathlineto{\pgfqpoint{2.612232in}{1.900000in}}%
\pgfusepath{stroke}%
\end{pgfscope}%
\begin{pgfscope}%
\pgfpathrectangle{\pgfqpoint{0.375000in}{0.300000in}}{\pgfqpoint{2.325000in}{2.400000in}} %
\pgfusepath{clip}%
\pgfsetrectcap%
\pgfsetroundjoin%
\pgfsetlinewidth{1.003750pt}%
\definecolor{currentstroke}{rgb}{0.000000,0.000000,0.000000}%
\pgfsetstrokecolor{currentstroke}%
\pgfsetdash{}{0pt}%
\pgfpathmoveto{\pgfqpoint{0.504167in}{0.566667in}}%
\pgfpathlineto{\pgfqpoint{0.503990in}{0.573645in}}%
\pgfpathlineto{\pgfqpoint{0.503459in}{0.580604in}}%
\pgfpathlineto{\pgfqpoint{0.502576in}{0.587525in}}%
\pgfpathlineto{\pgfqpoint{0.501344in}{0.594388in}}%
\pgfpathlineto{\pgfqpoint{0.499765in}{0.601176in}}%
\pgfpathlineto{\pgfqpoint{0.497845in}{0.607869in}}%
\pgfpathlineto{\pgfqpoint{0.495587in}{0.614449in}}%
\pgfpathlineto{\pgfqpoint{0.493000in}{0.620898in}}%
\pgfpathlineto{\pgfqpoint{0.490088in}{0.627199in}}%
\pgfpathlineto{\pgfqpoint{0.486862in}{0.633333in}}%
\pgfusepath{stroke}%
\end{pgfscope}%
\begin{pgfscope}%
\pgfpathrectangle{\pgfqpoint{0.375000in}{0.300000in}}{\pgfqpoint{2.325000in}{2.400000in}} %
\pgfusepath{clip}%
\pgfsetrectcap%
\pgfsetroundjoin%
\pgfsetlinewidth{1.003750pt}%
\definecolor{currentstroke}{rgb}{0.000000,0.000000,0.000000}%
\pgfsetstrokecolor{currentstroke}%
\pgfsetdash{}{0pt}%
\pgfpathmoveto{\pgfqpoint{1.751949in}{1.695214in}}%
\pgfpathlineto{\pgfqpoint{1.745588in}{1.778645in}}%
\pgfpathlineto{\pgfqpoint{1.726662in}{1.860023in}}%
\pgfpathlineto{\pgfqpoint{1.695636in}{1.937342in}}%
\pgfpathlineto{\pgfqpoint{1.653275in}{2.008699in}}%
\pgfpathlineto{\pgfqpoint{1.600621in}{2.072337in}}%
\pgfpathlineto{\pgfqpoint{1.538972in}{2.126689in}}%
\pgfpathlineto{\pgfqpoint{1.469845in}{2.170417in}}%
\pgfpathlineto{\pgfqpoint{1.394942in}{2.202444in}}%
\pgfpathlineto{\pgfqpoint{1.316107in}{2.221981in}}%
\pgfpathlineto{\pgfqpoint{1.235283in}{2.228547in}}%
\pgfpathlineto{\pgfqpoint{1.154458in}{2.221981in}}%
\pgfpathlineto{\pgfqpoint{1.075624in}{2.202444in}}%
\pgfpathlineto{\pgfqpoint{1.000721in}{2.170417in}}%
\pgfpathlineto{\pgfqpoint{0.931594in}{2.126689in}}%
\pgfpathlineto{\pgfqpoint{0.869944in}{2.072337in}}%
\pgfpathlineto{\pgfqpoint{0.817291in}{2.008699in}}%
\pgfpathlineto{\pgfqpoint{0.774929in}{1.937342in}}%
\pgfpathlineto{\pgfqpoint{0.743904in}{1.860023in}}%
\pgfpathlineto{\pgfqpoint{0.724977in}{1.778645in}}%
\pgfpathlineto{\pgfqpoint{0.718616in}{1.695214in}}%
\pgfusepath{stroke}%
\end{pgfscope}%
\begin{pgfscope}%
\pgfpathrectangle{\pgfqpoint{0.375000in}{0.300000in}}{\pgfqpoint{2.325000in}{2.400000in}} %
\pgfusepath{clip}%
\pgfsetrectcap%
\pgfsetroundjoin%
\pgfsetlinewidth{1.003750pt}%
\definecolor{currentstroke}{rgb}{0.000000,0.000000,0.000000}%
\pgfsetstrokecolor{currentstroke}%
\pgfsetdash{}{0pt}%
\pgfpathmoveto{\pgfqpoint{0.718616in}{1.695214in}}%
\pgfpathlineto{\pgfqpoint{0.724977in}{1.611782in}}%
\pgfpathlineto{\pgfqpoint{0.743904in}{1.530404in}}%
\pgfpathlineto{\pgfqpoint{0.774929in}{1.453085in}}%
\pgfpathlineto{\pgfqpoint{0.817291in}{1.381728in}}%
\pgfpathlineto{\pgfqpoint{0.869944in}{1.318090in}}%
\pgfpathlineto{\pgfqpoint{0.931594in}{1.263738in}}%
\pgfpathlineto{\pgfqpoint{1.000721in}{1.220010in}}%
\pgfpathlineto{\pgfqpoint{1.075624in}{1.187983in}}%
\pgfpathlineto{\pgfqpoint{1.154458in}{1.168446in}}%
\pgfpathlineto{\pgfqpoint{1.235283in}{1.161880in}}%
\pgfpathlineto{\pgfqpoint{1.316107in}{1.168446in}}%
\pgfpathlineto{\pgfqpoint{1.394942in}{1.187983in}}%
\pgfpathlineto{\pgfqpoint{1.469845in}{1.220010in}}%
\pgfpathlineto{\pgfqpoint{1.538972in}{1.263738in}}%
\pgfpathlineto{\pgfqpoint{1.600621in}{1.318090in}}%
\pgfpathlineto{\pgfqpoint{1.653275in}{1.381728in}}%
\pgfpathlineto{\pgfqpoint{1.695636in}{1.453085in}}%
\pgfpathlineto{\pgfqpoint{1.726662in}{1.530404in}}%
\pgfpathlineto{\pgfqpoint{1.745588in}{1.611782in}}%
\pgfpathlineto{\pgfqpoint{1.751949in}{1.695214in}}%
\pgfusepath{stroke}%
\end{pgfscope}%
\begin{pgfscope}%
\pgfpathrectangle{\pgfqpoint{0.375000in}{0.300000in}}{\pgfqpoint{2.325000in}{2.400000in}} %
\pgfusepath{clip}%
\pgfsetbuttcap%
\pgfsetroundjoin%
\definecolor{currentfill}{rgb}{0.000000,0.000000,1.000000}%
\pgfsetfillcolor{currentfill}%
\pgfsetlinewidth{0.501875pt}%
\definecolor{currentstroke}{rgb}{0.000000,0.000000,0.000000}%
\pgfsetstrokecolor{currentstroke}%
\pgfsetdash{}{0pt}%
\pgfsys@defobject{currentmarker}{\pgfqpoint{-0.041667in}{-0.041667in}}{\pgfqpoint{0.041667in}{0.041667in}}{%
\pgfpathmoveto{\pgfqpoint{0.000000in}{-0.041667in}}%
\pgfpathcurveto{\pgfqpoint{0.011050in}{-0.041667in}}{\pgfqpoint{0.021649in}{-0.037276in}}{\pgfqpoint{0.029463in}{-0.029463in}}%
\pgfpathcurveto{\pgfqpoint{0.037276in}{-0.021649in}}{\pgfqpoint{0.041667in}{-0.011050in}}{\pgfqpoint{0.041667in}{0.000000in}}%
\pgfpathcurveto{\pgfqpoint{0.041667in}{0.011050in}}{\pgfqpoint{0.037276in}{0.021649in}}{\pgfqpoint{0.029463in}{0.029463in}}%
\pgfpathcurveto{\pgfqpoint{0.021649in}{0.037276in}}{\pgfqpoint{0.011050in}{0.041667in}}{\pgfqpoint{0.000000in}{0.041667in}}%
\pgfpathcurveto{\pgfqpoint{-0.011050in}{0.041667in}}{\pgfqpoint{-0.021649in}{0.037276in}}{\pgfqpoint{-0.029463in}{0.029463in}}%
\pgfpathcurveto{\pgfqpoint{-0.037276in}{0.021649in}}{\pgfqpoint{-0.041667in}{0.011050in}}{\pgfqpoint{-0.041667in}{0.000000in}}%
\pgfpathcurveto{\pgfqpoint{-0.041667in}{-0.011050in}}{\pgfqpoint{-0.037276in}{-0.021649in}}{\pgfqpoint{-0.029463in}{-0.029463in}}%
\pgfpathcurveto{\pgfqpoint{-0.021649in}{-0.037276in}}{\pgfqpoint{-0.011050in}{-0.041667in}}{\pgfqpoint{0.000000in}{-0.041667in}}%
\pgfpathclose%
\pgfusepath{stroke,fill}%
}%
\begin{pgfscope}%
\pgfsys@transformshift{1.235283in}{1.695214in}%
\pgfsys@useobject{currentmarker}{}%
\end{pgfscope}%
\end{pgfscope}%
\begin{pgfscope}%
\pgfpathrectangle{\pgfqpoint{0.375000in}{0.300000in}}{\pgfqpoint{2.325000in}{2.400000in}} %
\pgfusepath{clip}%
\pgfsetbuttcap%
\pgfsetroundjoin%
\definecolor{currentfill}{rgb}{0.000000,0.500000,0.000000}%
\pgfsetfillcolor{currentfill}%
\pgfsetlinewidth{0.501875pt}%
\definecolor{currentstroke}{rgb}{0.000000,0.000000,0.000000}%
\pgfsetstrokecolor{currentstroke}%
\pgfsetdash{}{0pt}%
\pgfsys@defobject{currentmarker}{\pgfqpoint{-0.041667in}{-0.041667in}}{\pgfqpoint{0.041667in}{0.041667in}}{%
\pgfpathmoveto{\pgfqpoint{0.000000in}{-0.041667in}}%
\pgfpathcurveto{\pgfqpoint{0.011050in}{-0.041667in}}{\pgfqpoint{0.021649in}{-0.037276in}}{\pgfqpoint{0.029463in}{-0.029463in}}%
\pgfpathcurveto{\pgfqpoint{0.037276in}{-0.021649in}}{\pgfqpoint{0.041667in}{-0.011050in}}{\pgfqpoint{0.041667in}{0.000000in}}%
\pgfpathcurveto{\pgfqpoint{0.041667in}{0.011050in}}{\pgfqpoint{0.037276in}{0.021649in}}{\pgfqpoint{0.029463in}{0.029463in}}%
\pgfpathcurveto{\pgfqpoint{0.021649in}{0.037276in}}{\pgfqpoint{0.011050in}{0.041667in}}{\pgfqpoint{0.000000in}{0.041667in}}%
\pgfpathcurveto{\pgfqpoint{-0.011050in}{0.041667in}}{\pgfqpoint{-0.021649in}{0.037276in}}{\pgfqpoint{-0.029463in}{0.029463in}}%
\pgfpathcurveto{\pgfqpoint{-0.037276in}{0.021649in}}{\pgfqpoint{-0.041667in}{0.011050in}}{\pgfqpoint{-0.041667in}{0.000000in}}%
\pgfpathcurveto{\pgfqpoint{-0.041667in}{-0.011050in}}{\pgfqpoint{-0.037276in}{-0.021649in}}{\pgfqpoint{-0.029463in}{-0.029463in}}%
\pgfpathcurveto{\pgfqpoint{-0.021649in}{-0.037276in}}{\pgfqpoint{-0.011050in}{-0.041667in}}{\pgfqpoint{0.000000in}{-0.041667in}}%
\pgfpathclose%
\pgfusepath{stroke,fill}%
}%
\begin{pgfscope}%
\pgfsys@transformshift{1.493616in}{1.233333in}%
\pgfsys@useobject{currentmarker}{}%
\end{pgfscope}%
\end{pgfscope}%
\begin{pgfscope}%
\pgftext[x=0.633333in,y=0.566667in,left,bottom]{{\sffamily\fontsize{14.000000}{16.800000}\selectfont \(\displaystyle \theta\)}}%
\end{pgfscope}%
\begin{pgfscope}%
\pgftext[x=1.235283in,y=1.695214in,right,bottom]{{\sffamily\fontsize{14.000000}{16.800000}\selectfont \(\displaystyle CM\)}}%
\end{pgfscope}%
\begin{pgfscope}%
\pgftext[x=1.235283in,y=0.942849in,left,top]{{\sffamily\fontsize{14.000000}{16.800000}\selectfont \(\displaystyle mg\)}}%
\end{pgfscope}%
\begin{pgfscope}%
\pgftext[x=1.471674in,y=2.739230in,left,bottom]{{\sffamily\fontsize{14.000000}{16.800000}\selectfont \(\displaystyle a\)}}%
\end{pgfscope}%
\begin{pgfscope}%
\pgftext[x=1.794591in,y=2.028547in,left,top]{{\sffamily\fontsize{14.000000}{16.800000}\selectfont \(\displaystyle F_{p}\)}}%
\end{pgfscope}%
\end{pgfpicture}%
\makeatother%
\endgroup%


%% Creator: Matplotlib, PGF backend
%%
%% To include the figure in your LaTeX document, write
%%   \input{<filename>.pgf}
%%
%% Make sure the required packages are loaded in your preamble
%%   \usepackage{pgf}
%%
%% Figures using additional raster images can only be included by \input if
%% they are in the same directory as the main LaTeX file. For loading figures
%% from other directories you can use the `import` package
%%   \usepackage{import}
%% and then include the figures with
%%   \import{<path to file>}{<filename>.pgf}
%%
%% Matplotlib used the following preamble
%%   \usepackage{fontspec}
%%   \setmainfont{Times New Roman}
%%   \setsansfont{Verdana}
%%   \setmonofont{Courier New}
%%
\begingroup%
\makeatletter%
\begin{pgfpicture}%
\pgfpathrectangle{\pgfpointorigin}{\pgfqpoint{3.000000in}{3.000000in}}%
\pgfusepath{use as bounding box}%
\begin{pgfscope}%
\pgfsetbuttcap%
\pgfsetroundjoin%
\definecolor{currentfill}{rgb}{1.000000,1.000000,1.000000}%
\pgfsetfillcolor{currentfill}%
\pgfsetlinewidth{0.000000pt}%
\definecolor{currentstroke}{rgb}{1.000000,1.000000,1.000000}%
\pgfsetstrokecolor{currentstroke}%
\pgfsetdash{}{0pt}%
\pgfpathmoveto{\pgfqpoint{0.000000in}{0.000000in}}%
\pgfpathlineto{\pgfqpoint{3.000000in}{0.000000in}}%
\pgfpathlineto{\pgfqpoint{3.000000in}{3.000000in}}%
\pgfpathlineto{\pgfqpoint{0.000000in}{3.000000in}}%
\pgfpathclose%
\pgfusepath{fill}%
\end{pgfscope}%
\begin{pgfscope}%
\pgfpathrectangle{\pgfqpoint{0.375000in}{0.300000in}}{\pgfqpoint{2.325000in}{2.400000in}} %
\pgfusepath{clip}%
\pgfsetbuttcap%
\pgfsetroundjoin%
\definecolor{currentfill}{rgb}{0.000000,0.000000,1.000000}%
\pgfsetfillcolor{currentfill}%
\pgfsetlinewidth{1.003750pt}%
\definecolor{currentstroke}{rgb}{0.000000,0.000000,0.000000}%
\pgfsetstrokecolor{currentstroke}%
\pgfsetdash{}{0pt}%
\pgfpathmoveto{\pgfqpoint{0.833633in}{0.840000in}}%
\pgfpathlineto{\pgfqpoint{0.908475in}{0.946188in}}%
\pgfpathlineto{\pgfqpoint{0.928496in}{0.910392in}}%
\pgfpathlineto{\pgfqpoint{1.487804in}{1.243726in}}%
\pgfpathlineto{\pgfqpoint{1.499429in}{1.222941in}}%
\pgfpathlineto{\pgfqpoint{0.940121in}{0.889608in}}%
\pgfpathlineto{\pgfqpoint{0.960141in}{0.853812in}}%
\pgfpathclose%
\pgfusepath{stroke,fill}%
\end{pgfscope}%
\begin{pgfscope}%
\pgfpathrectangle{\pgfqpoint{0.375000in}{0.300000in}}{\pgfqpoint{2.325000in}{2.400000in}} %
\pgfusepath{clip}%
\pgfsetbuttcap%
\pgfsetroundjoin%
\definecolor{currentfill}{rgb}{0.000000,0.000000,1.000000}%
\pgfsetfillcolor{currentfill}%
\pgfsetlinewidth{1.003750pt}%
\definecolor{currentstroke}{rgb}{0.000000,0.000000,0.000000}%
\pgfsetstrokecolor{currentstroke}%
\pgfsetdash{}{0pt}%
\pgfpathmoveto{\pgfqpoint{1.235283in}{0.822849in}}%
\pgfpathlineto{\pgfqpoint{1.183616in}{0.942849in}}%
\pgfpathlineto{\pgfqpoint{1.223658in}{0.942849in}}%
\pgfpathlineto{\pgfqpoint{1.223658in}{1.695214in}}%
\pgfpathlineto{\pgfqpoint{1.246908in}{1.695214in}}%
\pgfpathlineto{\pgfqpoint{1.246908in}{0.942849in}}%
\pgfpathlineto{\pgfqpoint{1.286949in}{0.942849in}}%
\pgfpathclose%
\pgfusepath{stroke,fill}%
\end{pgfscope}%
\begin{pgfscope}%
\pgfpathrectangle{\pgfqpoint{0.375000in}{0.300000in}}{\pgfqpoint{2.325000in}{2.400000in}} %
\pgfusepath{clip}%
\pgfsetbuttcap%
\pgfsetroundjoin%
\definecolor{currentfill}{rgb}{1.000000,0.000000,0.000000}%
\pgfsetfillcolor{currentfill}%
\pgfsetlinewidth{1.003750pt}%
\definecolor{currentstroke}{rgb}{1.000000,0.000000,0.000000}%
\pgfsetstrokecolor{currentstroke}%
\pgfsetdash{}{0pt}%
\pgfpathmoveto{\pgfqpoint{1.605908in}{2.685897in}}%
\pgfpathlineto{\pgfqpoint{1.536258in}{2.490427in}}%
\pgfpathlineto{\pgfqpoint{1.487820in}{2.577030in}}%
\pgfpathlineto{\pgfqpoint{0.928512in}{2.243696in}}%
\pgfpathlineto{\pgfqpoint{0.896220in}{2.301431in}}%
\pgfpathlineto{\pgfqpoint{1.455528in}{2.634765in}}%
\pgfpathlineto{\pgfqpoint{1.413449in}{2.710000in}}%
\pgfpathclose%
\pgfusepath{stroke,fill}%
\end{pgfscope}%
\begin{pgfscope}%
\pgfpathrectangle{\pgfqpoint{0.375000in}{0.300000in}}{\pgfqpoint{2.325000in}{2.400000in}} %
\pgfusepath{clip}%
\pgfsetbuttcap%
\pgfsetroundjoin%
\definecolor{currentfill}{rgb}{0.000000,0.000000,1.000000}%
\pgfsetfillcolor{currentfill}%
\pgfsetlinewidth{1.003750pt}%
\definecolor{currentstroke}{rgb}{0.000000,0.000000,0.000000}%
\pgfsetstrokecolor{currentstroke}%
\pgfsetdash{}{0pt}%
\pgfpathmoveto{\pgfqpoint{1.895266in}{2.088547in}}%
\pgfpathlineto{\pgfqpoint{1.820424in}{1.982359in}}%
\pgfpathlineto{\pgfqpoint{1.800403in}{2.018155in}}%
\pgfpathlineto{\pgfqpoint{1.241095in}{1.684821in}}%
\pgfpathlineto{\pgfqpoint{1.229470in}{1.705606in}}%
\pgfpathlineto{\pgfqpoint{1.788778in}{2.038939in}}%
\pgfpathlineto{\pgfqpoint{1.768758in}{2.074735in}}%
\pgfpathclose%
\pgfusepath{stroke,fill}%
\end{pgfscope}%
\begin{pgfscope}%
\pgfpathrectangle{\pgfqpoint{0.375000in}{0.300000in}}{\pgfqpoint{2.325000in}{2.400000in}} %
\pgfusepath{clip}%
\pgfsetrectcap%
\pgfsetroundjoin%
\pgfsetlinewidth{1.003750pt}%
\definecolor{currentstroke}{rgb}{0.000000,0.000000,1.000000}%
\pgfsetstrokecolor{currentstroke}%
\pgfsetdash{}{0pt}%
\pgfpathmoveto{\pgfqpoint{0.375000in}{0.566667in}}%
\pgfpathlineto{\pgfqpoint{2.612232in}{0.566667in}}%
\pgfusepath{stroke}%
\end{pgfscope}%
\begin{pgfscope}%
\pgfpathrectangle{\pgfqpoint{0.375000in}{0.300000in}}{\pgfqpoint{2.325000in}{2.400000in}} %
\pgfusepath{clip}%
\pgfsetrectcap%
\pgfsetroundjoin%
\pgfsetlinewidth{1.003750pt}%
\definecolor{currentstroke}{rgb}{0.000000,0.000000,1.000000}%
\pgfsetstrokecolor{currentstroke}%
\pgfsetdash{}{0pt}%
\pgfpathmoveto{\pgfqpoint{2.612232in}{0.566667in}}%
\pgfpathlineto{\pgfqpoint{2.612232in}{1.900000in}}%
\pgfusepath{stroke}%
\end{pgfscope}%
\begin{pgfscope}%
\pgfpathrectangle{\pgfqpoint{0.375000in}{0.300000in}}{\pgfqpoint{2.325000in}{2.400000in}} %
\pgfusepath{clip}%
\pgfsetrectcap%
\pgfsetroundjoin%
\pgfsetlinewidth{1.003750pt}%
\definecolor{currentstroke}{rgb}{0.000000,0.000000,1.000000}%
\pgfsetstrokecolor{currentstroke}%
\pgfsetdash{}{0pt}%
\pgfpathmoveto{\pgfqpoint{0.375000in}{0.566667in}}%
\pgfpathlineto{\pgfqpoint{2.612232in}{1.900000in}}%
\pgfusepath{stroke}%
\end{pgfscope}%
\begin{pgfscope}%
\pgfpathrectangle{\pgfqpoint{0.375000in}{0.300000in}}{\pgfqpoint{2.325000in}{2.400000in}} %
\pgfusepath{clip}%
\pgfsetrectcap%
\pgfsetroundjoin%
\pgfsetlinewidth{1.003750pt}%
\definecolor{currentstroke}{rgb}{0.000000,0.000000,0.000000}%
\pgfsetstrokecolor{currentstroke}%
\pgfsetdash{}{0pt}%
\pgfpathmoveto{\pgfqpoint{0.504167in}{0.566667in}}%
\pgfpathlineto{\pgfqpoint{0.503990in}{0.573645in}}%
\pgfpathlineto{\pgfqpoint{0.503459in}{0.580604in}}%
\pgfpathlineto{\pgfqpoint{0.502576in}{0.587525in}}%
\pgfpathlineto{\pgfqpoint{0.501344in}{0.594388in}}%
\pgfpathlineto{\pgfqpoint{0.499765in}{0.601176in}}%
\pgfpathlineto{\pgfqpoint{0.497845in}{0.607869in}}%
\pgfpathlineto{\pgfqpoint{0.495587in}{0.614449in}}%
\pgfpathlineto{\pgfqpoint{0.493000in}{0.620898in}}%
\pgfpathlineto{\pgfqpoint{0.490088in}{0.627199in}}%
\pgfpathlineto{\pgfqpoint{0.486862in}{0.633333in}}%
\pgfusepath{stroke}%
\end{pgfscope}%
\begin{pgfscope}%
\pgfpathrectangle{\pgfqpoint{0.375000in}{0.300000in}}{\pgfqpoint{2.325000in}{2.400000in}} %
\pgfusepath{clip}%
\pgfsetrectcap%
\pgfsetroundjoin%
\pgfsetlinewidth{1.003750pt}%
\definecolor{currentstroke}{rgb}{0.000000,0.000000,0.000000}%
\pgfsetstrokecolor{currentstroke}%
\pgfsetdash{}{0pt}%
\pgfpathmoveto{\pgfqpoint{1.751949in}{1.695214in}}%
\pgfpathlineto{\pgfqpoint{1.745588in}{1.778645in}}%
\pgfpathlineto{\pgfqpoint{1.726662in}{1.860023in}}%
\pgfpathlineto{\pgfqpoint{1.695636in}{1.937342in}}%
\pgfpathlineto{\pgfqpoint{1.653275in}{2.008699in}}%
\pgfpathlineto{\pgfqpoint{1.600621in}{2.072337in}}%
\pgfpathlineto{\pgfqpoint{1.538972in}{2.126689in}}%
\pgfpathlineto{\pgfqpoint{1.469845in}{2.170417in}}%
\pgfpathlineto{\pgfqpoint{1.394942in}{2.202444in}}%
\pgfpathlineto{\pgfqpoint{1.316107in}{2.221981in}}%
\pgfpathlineto{\pgfqpoint{1.235283in}{2.228547in}}%
\pgfpathlineto{\pgfqpoint{1.154458in}{2.221981in}}%
\pgfpathlineto{\pgfqpoint{1.075624in}{2.202444in}}%
\pgfpathlineto{\pgfqpoint{1.000721in}{2.170417in}}%
\pgfpathlineto{\pgfqpoint{0.931594in}{2.126689in}}%
\pgfpathlineto{\pgfqpoint{0.869944in}{2.072337in}}%
\pgfpathlineto{\pgfqpoint{0.817291in}{2.008699in}}%
\pgfpathlineto{\pgfqpoint{0.774929in}{1.937342in}}%
\pgfpathlineto{\pgfqpoint{0.743904in}{1.860023in}}%
\pgfpathlineto{\pgfqpoint{0.724977in}{1.778645in}}%
\pgfpathlineto{\pgfqpoint{0.718616in}{1.695214in}}%
\pgfusepath{stroke}%
\end{pgfscope}%
\begin{pgfscope}%
\pgfpathrectangle{\pgfqpoint{0.375000in}{0.300000in}}{\pgfqpoint{2.325000in}{2.400000in}} %
\pgfusepath{clip}%
\pgfsetrectcap%
\pgfsetroundjoin%
\pgfsetlinewidth{1.003750pt}%
\definecolor{currentstroke}{rgb}{0.000000,0.000000,0.000000}%
\pgfsetstrokecolor{currentstroke}%
\pgfsetdash{}{0pt}%
\pgfpathmoveto{\pgfqpoint{0.718616in}{1.695214in}}%
\pgfpathlineto{\pgfqpoint{0.724977in}{1.611782in}}%
\pgfpathlineto{\pgfqpoint{0.743904in}{1.530404in}}%
\pgfpathlineto{\pgfqpoint{0.774929in}{1.453085in}}%
\pgfpathlineto{\pgfqpoint{0.817291in}{1.381728in}}%
\pgfpathlineto{\pgfqpoint{0.869944in}{1.318090in}}%
\pgfpathlineto{\pgfqpoint{0.931594in}{1.263738in}}%
\pgfpathlineto{\pgfqpoint{1.000721in}{1.220010in}}%
\pgfpathlineto{\pgfqpoint{1.075624in}{1.187983in}}%
\pgfpathlineto{\pgfqpoint{1.154458in}{1.168446in}}%
\pgfpathlineto{\pgfqpoint{1.235283in}{1.161880in}}%
\pgfpathlineto{\pgfqpoint{1.316107in}{1.168446in}}%
\pgfpathlineto{\pgfqpoint{1.394942in}{1.187983in}}%
\pgfpathlineto{\pgfqpoint{1.469845in}{1.220010in}}%
\pgfpathlineto{\pgfqpoint{1.538972in}{1.263738in}}%
\pgfpathlineto{\pgfqpoint{1.600621in}{1.318090in}}%
\pgfpathlineto{\pgfqpoint{1.653275in}{1.381728in}}%
\pgfpathlineto{\pgfqpoint{1.695636in}{1.453085in}}%
\pgfpathlineto{\pgfqpoint{1.726662in}{1.530404in}}%
\pgfpathlineto{\pgfqpoint{1.745588in}{1.611782in}}%
\pgfpathlineto{\pgfqpoint{1.751949in}{1.695214in}}%
\pgfusepath{stroke}%
\end{pgfscope}%
\begin{pgfscope}%
\pgfpathrectangle{\pgfqpoint{0.375000in}{0.300000in}}{\pgfqpoint{2.325000in}{2.400000in}} %
\pgfusepath{clip}%
\pgfsetbuttcap%
\pgfsetroundjoin%
\definecolor{currentfill}{rgb}{0.000000,0.000000,1.000000}%
\pgfsetfillcolor{currentfill}%
\pgfsetlinewidth{0.501875pt}%
\definecolor{currentstroke}{rgb}{0.000000,0.000000,0.000000}%
\pgfsetstrokecolor{currentstroke}%
\pgfsetdash{}{0pt}%
\pgfsys@defobject{currentmarker}{\pgfqpoint{-0.041667in}{-0.041667in}}{\pgfqpoint{0.041667in}{0.041667in}}{%
\pgfpathmoveto{\pgfqpoint{0.000000in}{-0.041667in}}%
\pgfpathcurveto{\pgfqpoint{0.011050in}{-0.041667in}}{\pgfqpoint{0.021649in}{-0.037276in}}{\pgfqpoint{0.029463in}{-0.029463in}}%
\pgfpathcurveto{\pgfqpoint{0.037276in}{-0.021649in}}{\pgfqpoint{0.041667in}{-0.011050in}}{\pgfqpoint{0.041667in}{0.000000in}}%
\pgfpathcurveto{\pgfqpoint{0.041667in}{0.011050in}}{\pgfqpoint{0.037276in}{0.021649in}}{\pgfqpoint{0.029463in}{0.029463in}}%
\pgfpathcurveto{\pgfqpoint{0.021649in}{0.037276in}}{\pgfqpoint{0.011050in}{0.041667in}}{\pgfqpoint{0.000000in}{0.041667in}}%
\pgfpathcurveto{\pgfqpoint{-0.011050in}{0.041667in}}{\pgfqpoint{-0.021649in}{0.037276in}}{\pgfqpoint{-0.029463in}{0.029463in}}%
\pgfpathcurveto{\pgfqpoint{-0.037276in}{0.021649in}}{\pgfqpoint{-0.041667in}{0.011050in}}{\pgfqpoint{-0.041667in}{0.000000in}}%
\pgfpathcurveto{\pgfqpoint{-0.041667in}{-0.011050in}}{\pgfqpoint{-0.037276in}{-0.021649in}}{\pgfqpoint{-0.029463in}{-0.029463in}}%
\pgfpathcurveto{\pgfqpoint{-0.021649in}{-0.037276in}}{\pgfqpoint{-0.011050in}{-0.041667in}}{\pgfqpoint{0.000000in}{-0.041667in}}%
\pgfpathclose%
\pgfusepath{stroke,fill}%
}%
\begin{pgfscope}%
\pgfsys@transformshift{1.235283in}{1.695214in}%
\pgfsys@useobject{currentmarker}{}%
\end{pgfscope}%
\end{pgfscope}%
\begin{pgfscope}%
\pgfpathrectangle{\pgfqpoint{0.375000in}{0.300000in}}{\pgfqpoint{2.325000in}{2.400000in}} %
\pgfusepath{clip}%
\pgfsetbuttcap%
\pgfsetroundjoin%
\definecolor{currentfill}{rgb}{0.000000,0.500000,0.000000}%
\pgfsetfillcolor{currentfill}%
\pgfsetlinewidth{0.501875pt}%
\definecolor{currentstroke}{rgb}{0.000000,0.000000,0.000000}%
\pgfsetstrokecolor{currentstroke}%
\pgfsetdash{}{0pt}%
\pgfsys@defobject{currentmarker}{\pgfqpoint{-0.041667in}{-0.041667in}}{\pgfqpoint{0.041667in}{0.041667in}}{%
\pgfpathmoveto{\pgfqpoint{0.000000in}{-0.041667in}}%
\pgfpathcurveto{\pgfqpoint{0.011050in}{-0.041667in}}{\pgfqpoint{0.021649in}{-0.037276in}}{\pgfqpoint{0.029463in}{-0.029463in}}%
\pgfpathcurveto{\pgfqpoint{0.037276in}{-0.021649in}}{\pgfqpoint{0.041667in}{-0.011050in}}{\pgfqpoint{0.041667in}{0.000000in}}%
\pgfpathcurveto{\pgfqpoint{0.041667in}{0.011050in}}{\pgfqpoint{0.037276in}{0.021649in}}{\pgfqpoint{0.029463in}{0.029463in}}%
\pgfpathcurveto{\pgfqpoint{0.021649in}{0.037276in}}{\pgfqpoint{0.011050in}{0.041667in}}{\pgfqpoint{0.000000in}{0.041667in}}%
\pgfpathcurveto{\pgfqpoint{-0.011050in}{0.041667in}}{\pgfqpoint{-0.021649in}{0.037276in}}{\pgfqpoint{-0.029463in}{0.029463in}}%
\pgfpathcurveto{\pgfqpoint{-0.037276in}{0.021649in}}{\pgfqpoint{-0.041667in}{0.011050in}}{\pgfqpoint{-0.041667in}{0.000000in}}%
\pgfpathcurveto{\pgfqpoint{-0.041667in}{-0.011050in}}{\pgfqpoint{-0.037276in}{-0.021649in}}{\pgfqpoint{-0.029463in}{-0.029463in}}%
\pgfpathcurveto{\pgfqpoint{-0.021649in}{-0.037276in}}{\pgfqpoint{-0.011050in}{-0.041667in}}{\pgfqpoint{0.000000in}{-0.041667in}}%
\pgfpathclose%
\pgfusepath{stroke,fill}%
}%
\begin{pgfscope}%
\pgfsys@transformshift{1.493616in}{1.233333in}%
\pgfsys@useobject{currentmarker}{}%
\end{pgfscope}%
\end{pgfscope}%
\begin{pgfscope}%
\pgftext[x=0.633333in,y=0.566667in,left,bottom]{{\sffamily\fontsize{14.000000}{16.800000}\selectfont \(\displaystyle \theta\)}}%
\end{pgfscope}%
\begin{pgfscope}%
\pgftext[x=1.235283in,y=1.695214in,right,bottom]{{\sffamily\fontsize{14.000000}{16.800000}\selectfont \(\displaystyle CM\)}}%
\end{pgfscope}%
\begin{pgfscope}%
\pgftext[x=0.934308in,y=0.900000in,right,bottom]{{\sffamily\fontsize{14.000000}{16.800000}\selectfont \(\displaystyle F_{fr}\)}}%
\end{pgfscope}%
\begin{pgfscope}%
\pgftext[x=1.235283in,y=0.942849in,left,top]{{\sffamily\fontsize{14.000000}{16.800000}\selectfont \(\displaystyle mg\)}}%
\end{pgfscope}%
\begin{pgfscope}%
\pgftext[x=1.471674in,y=2.739230in,left,bottom]{{\sffamily\fontsize{14.000000}{16.800000}\selectfont \(\displaystyle a\)}}%
\end{pgfscope}%
\begin{pgfscope}%
\pgftext[x=1.794591in,y=2.028547in,left,top]{{\sffamily\fontsize{14.000000}{16.800000}\selectfont \(\displaystyle F_{p}\)}}%
\end{pgfscope}%
\end{pgfpicture}%
\makeatother%
\endgroup%


\newpage

\begin{problem}
An applied force $F_{p}$ passing through CM acts on the roller. The
magnitude of $F_{p}$ is smaller than $mg\sin \theta $. The roller is on its
way to the highest point it can reach, and it is already slowing down.
\end{problem}

%% Creator: Matplotlib, PGF backend
%%
%% To include the figure in your LaTeX document, write
%%   \input{<filename>.pgf}
%%
%% Make sure the required packages are loaded in your preamble
%%   \usepackage{pgf}
%%
%% Figures using additional raster images can only be included by \input if
%% they are in the same directory as the main LaTeX file. For loading figures
%% from other directories you can use the `import` package
%%   \usepackage{import}
%% and then include the figures with
%%   \import{<path to file>}{<filename>.pgf}
%%
%% Matplotlib used the following preamble
%%   \usepackage{fontspec}
%%   \setmainfont{Times New Roman}
%%   \setsansfont{Verdana}
%%   \setmonofont{Courier New}
%%
\begingroup%
\makeatletter%
\begin{pgfpicture}%
\pgfpathrectangle{\pgfpointorigin}{\pgfqpoint{3.000000in}{3.000000in}}%
\pgfusepath{use as bounding box}%
\begin{pgfscope}%
\pgfsetbuttcap%
\pgfsetroundjoin%
\definecolor{currentfill}{rgb}{1.000000,1.000000,1.000000}%
\pgfsetfillcolor{currentfill}%
\pgfsetlinewidth{0.000000pt}%
\definecolor{currentstroke}{rgb}{1.000000,1.000000,1.000000}%
\pgfsetstrokecolor{currentstroke}%
\pgfsetdash{}{0pt}%
\pgfpathmoveto{\pgfqpoint{0.000000in}{0.000000in}}%
\pgfpathlineto{\pgfqpoint{3.000000in}{0.000000in}}%
\pgfpathlineto{\pgfqpoint{3.000000in}{3.000000in}}%
\pgfpathlineto{\pgfqpoint{0.000000in}{3.000000in}}%
\pgfpathclose%
\pgfusepath{fill}%
\end{pgfscope}%
\begin{pgfscope}%
\pgfpathrectangle{\pgfqpoint{0.375000in}{0.300000in}}{\pgfqpoint{2.325000in}{2.400000in}} %
\pgfusepath{clip}%
\pgfsetbuttcap%
\pgfsetroundjoin%
\definecolor{currentfill}{rgb}{0.000000,0.000000,1.000000}%
\pgfsetfillcolor{currentfill}%
\pgfsetlinewidth{1.003750pt}%
\definecolor{currentstroke}{rgb}{0.000000,0.000000,0.000000}%
\pgfsetstrokecolor{currentstroke}%
\pgfsetdash{}{0pt}%
\pgfpathmoveto{\pgfqpoint{1.235283in}{0.822849in}}%
\pgfpathlineto{\pgfqpoint{1.183616in}{0.942849in}}%
\pgfpathlineto{\pgfqpoint{1.223658in}{0.942849in}}%
\pgfpathlineto{\pgfqpoint{1.223658in}{1.695214in}}%
\pgfpathlineto{\pgfqpoint{1.246908in}{1.695214in}}%
\pgfpathlineto{\pgfqpoint{1.246908in}{0.942849in}}%
\pgfpathlineto{\pgfqpoint{1.286949in}{0.942849in}}%
\pgfpathclose%
\pgfusepath{stroke,fill}%
\end{pgfscope}%
\begin{pgfscope}%
\pgfpathrectangle{\pgfqpoint{0.375000in}{0.300000in}}{\pgfqpoint{2.325000in}{2.400000in}} %
\pgfusepath{clip}%
\pgfsetbuttcap%
\pgfsetroundjoin%
\definecolor{currentfill}{rgb}{1.000000,0.000000,0.000000}%
\pgfsetfillcolor{currentfill}%
\pgfsetlinewidth{1.003750pt}%
\definecolor{currentstroke}{rgb}{1.000000,0.000000,0.000000}%
\pgfsetstrokecolor{currentstroke}%
\pgfsetdash{}{0pt}%
\pgfpathmoveto{\pgfqpoint{0.627119in}{2.102564in}}%
\pgfpathlineto{\pgfqpoint{0.696770in}{2.298034in}}%
\pgfpathlineto{\pgfqpoint{0.748436in}{2.205658in}}%
\pgfpathlineto{\pgfqpoint{0.899449in}{2.295658in}}%
\pgfpathlineto{\pgfqpoint{0.925283in}{2.249470in}}%
\pgfpathlineto{\pgfqpoint{0.774270in}{2.159470in}}%
\pgfpathlineto{\pgfqpoint{0.825936in}{2.067094in}}%
\pgfpathclose%
\pgfusepath{stroke,fill}%
\end{pgfscope}%
\begin{pgfscope}%
\pgfpathrectangle{\pgfqpoint{0.375000in}{0.300000in}}{\pgfqpoint{2.325000in}{2.400000in}} %
\pgfusepath{clip}%
\pgfsetbuttcap%
\pgfsetroundjoin%
\definecolor{currentfill}{rgb}{0.000000,0.000000,0.000000}%
\pgfsetfillcolor{currentfill}%
\pgfsetlinewidth{1.003750pt}%
\definecolor{currentstroke}{rgb}{0.000000,0.000000,0.000000}%
\pgfsetstrokecolor{currentstroke}%
\pgfsetdash{}{0pt}%
\pgfpathmoveto{\pgfqpoint{1.348626in}{2.532564in}}%
\pgfpathlineto{\pgfqpoint{1.278976in}{2.337094in}}%
\pgfpathlineto{\pgfqpoint{1.227309in}{2.429470in}}%
\pgfpathlineto{\pgfqpoint{0.925283in}{2.249470in}}%
\pgfpathlineto{\pgfqpoint{0.899449in}{2.295658in}}%
\pgfpathlineto{\pgfqpoint{1.201476in}{2.475658in}}%
\pgfpathlineto{\pgfqpoint{1.149809in}{2.568034in}}%
\pgfpathclose%
\pgfusepath{stroke,fill}%
\end{pgfscope}%
\begin{pgfscope}%
\pgfpathrectangle{\pgfqpoint{0.375000in}{0.300000in}}{\pgfqpoint{2.325000in}{2.400000in}} %
\pgfusepath{clip}%
\pgfsetbuttcap%
\pgfsetroundjoin%
\definecolor{currentfill}{rgb}{0.000000,0.000000,1.000000}%
\pgfsetfillcolor{currentfill}%
\pgfsetlinewidth{1.003750pt}%
\definecolor{currentstroke}{rgb}{0.000000,0.000000,0.000000}%
\pgfsetstrokecolor{currentstroke}%
\pgfsetdash{}{0pt}%
\pgfpathmoveto{\pgfqpoint{1.637985in}{1.935214in}}%
\pgfpathlineto{\pgfqpoint{1.563143in}{1.829026in}}%
\pgfpathlineto{\pgfqpoint{1.543122in}{1.864821in}}%
\pgfpathlineto{\pgfqpoint{1.241095in}{1.684821in}}%
\pgfpathlineto{\pgfqpoint{1.229470in}{1.705606in}}%
\pgfpathlineto{\pgfqpoint{1.531497in}{1.885606in}}%
\pgfpathlineto{\pgfqpoint{1.511476in}{1.921402in}}%
\pgfpathclose%
\pgfusepath{stroke,fill}%
\end{pgfscope}%
\begin{pgfscope}%
\pgfpathrectangle{\pgfqpoint{0.375000in}{0.300000in}}{\pgfqpoint{2.325000in}{2.400000in}} %
\pgfusepath{clip}%
\pgfsetrectcap%
\pgfsetroundjoin%
\pgfsetlinewidth{1.003750pt}%
\definecolor{currentstroke}{rgb}{0.000000,0.000000,1.000000}%
\pgfsetstrokecolor{currentstroke}%
\pgfsetdash{}{0pt}%
\pgfpathmoveto{\pgfqpoint{0.375000in}{0.566667in}}%
\pgfpathlineto{\pgfqpoint{2.612232in}{0.566667in}}%
\pgfusepath{stroke}%
\end{pgfscope}%
\begin{pgfscope}%
\pgfpathrectangle{\pgfqpoint{0.375000in}{0.300000in}}{\pgfqpoint{2.325000in}{2.400000in}} %
\pgfusepath{clip}%
\pgfsetrectcap%
\pgfsetroundjoin%
\pgfsetlinewidth{1.003750pt}%
\definecolor{currentstroke}{rgb}{0.000000,0.000000,1.000000}%
\pgfsetstrokecolor{currentstroke}%
\pgfsetdash{}{0pt}%
\pgfpathmoveto{\pgfqpoint{2.612232in}{0.566667in}}%
\pgfpathlineto{\pgfqpoint{2.612232in}{1.900000in}}%
\pgfusepath{stroke}%
\end{pgfscope}%
\begin{pgfscope}%
\pgfpathrectangle{\pgfqpoint{0.375000in}{0.300000in}}{\pgfqpoint{2.325000in}{2.400000in}} %
\pgfusepath{clip}%
\pgfsetrectcap%
\pgfsetroundjoin%
\pgfsetlinewidth{1.003750pt}%
\definecolor{currentstroke}{rgb}{0.000000,0.000000,1.000000}%
\pgfsetstrokecolor{currentstroke}%
\pgfsetdash{}{0pt}%
\pgfpathmoveto{\pgfqpoint{0.375000in}{0.566667in}}%
\pgfpathlineto{\pgfqpoint{2.612232in}{1.900000in}}%
\pgfusepath{stroke}%
\end{pgfscope}%
\begin{pgfscope}%
\pgfpathrectangle{\pgfqpoint{0.375000in}{0.300000in}}{\pgfqpoint{2.325000in}{2.400000in}} %
\pgfusepath{clip}%
\pgfsetrectcap%
\pgfsetroundjoin%
\pgfsetlinewidth{1.003750pt}%
\definecolor{currentstroke}{rgb}{0.000000,0.000000,0.000000}%
\pgfsetstrokecolor{currentstroke}%
\pgfsetdash{}{0pt}%
\pgfpathmoveto{\pgfqpoint{0.504167in}{0.566667in}}%
\pgfpathlineto{\pgfqpoint{0.503990in}{0.573645in}}%
\pgfpathlineto{\pgfqpoint{0.503459in}{0.580604in}}%
\pgfpathlineto{\pgfqpoint{0.502576in}{0.587525in}}%
\pgfpathlineto{\pgfqpoint{0.501344in}{0.594388in}}%
\pgfpathlineto{\pgfqpoint{0.499765in}{0.601176in}}%
\pgfpathlineto{\pgfqpoint{0.497845in}{0.607869in}}%
\pgfpathlineto{\pgfqpoint{0.495587in}{0.614449in}}%
\pgfpathlineto{\pgfqpoint{0.493000in}{0.620898in}}%
\pgfpathlineto{\pgfqpoint{0.490088in}{0.627199in}}%
\pgfpathlineto{\pgfqpoint{0.486862in}{0.633333in}}%
\pgfusepath{stroke}%
\end{pgfscope}%
\begin{pgfscope}%
\pgfpathrectangle{\pgfqpoint{0.375000in}{0.300000in}}{\pgfqpoint{2.325000in}{2.400000in}} %
\pgfusepath{clip}%
\pgfsetrectcap%
\pgfsetroundjoin%
\pgfsetlinewidth{1.003750pt}%
\definecolor{currentstroke}{rgb}{0.000000,0.000000,0.000000}%
\pgfsetstrokecolor{currentstroke}%
\pgfsetdash{}{0pt}%
\pgfpathmoveto{\pgfqpoint{1.751949in}{1.695214in}}%
\pgfpathlineto{\pgfqpoint{1.745588in}{1.778645in}}%
\pgfpathlineto{\pgfqpoint{1.726662in}{1.860023in}}%
\pgfpathlineto{\pgfqpoint{1.695636in}{1.937342in}}%
\pgfpathlineto{\pgfqpoint{1.653275in}{2.008699in}}%
\pgfpathlineto{\pgfqpoint{1.600621in}{2.072337in}}%
\pgfpathlineto{\pgfqpoint{1.538972in}{2.126689in}}%
\pgfpathlineto{\pgfqpoint{1.469845in}{2.170417in}}%
\pgfpathlineto{\pgfqpoint{1.394942in}{2.202444in}}%
\pgfpathlineto{\pgfqpoint{1.316107in}{2.221981in}}%
\pgfpathlineto{\pgfqpoint{1.235283in}{2.228547in}}%
\pgfpathlineto{\pgfqpoint{1.154458in}{2.221981in}}%
\pgfpathlineto{\pgfqpoint{1.075624in}{2.202444in}}%
\pgfpathlineto{\pgfqpoint{1.000721in}{2.170417in}}%
\pgfpathlineto{\pgfqpoint{0.931594in}{2.126689in}}%
\pgfpathlineto{\pgfqpoint{0.869944in}{2.072337in}}%
\pgfpathlineto{\pgfqpoint{0.817291in}{2.008699in}}%
\pgfpathlineto{\pgfqpoint{0.774929in}{1.937342in}}%
\pgfpathlineto{\pgfqpoint{0.743904in}{1.860023in}}%
\pgfpathlineto{\pgfqpoint{0.724977in}{1.778645in}}%
\pgfpathlineto{\pgfqpoint{0.718616in}{1.695214in}}%
\pgfusepath{stroke}%
\end{pgfscope}%
\begin{pgfscope}%
\pgfpathrectangle{\pgfqpoint{0.375000in}{0.300000in}}{\pgfqpoint{2.325000in}{2.400000in}} %
\pgfusepath{clip}%
\pgfsetrectcap%
\pgfsetroundjoin%
\pgfsetlinewidth{1.003750pt}%
\definecolor{currentstroke}{rgb}{0.000000,0.000000,0.000000}%
\pgfsetstrokecolor{currentstroke}%
\pgfsetdash{}{0pt}%
\pgfpathmoveto{\pgfqpoint{0.718616in}{1.695214in}}%
\pgfpathlineto{\pgfqpoint{0.724977in}{1.611782in}}%
\pgfpathlineto{\pgfqpoint{0.743904in}{1.530404in}}%
\pgfpathlineto{\pgfqpoint{0.774929in}{1.453085in}}%
\pgfpathlineto{\pgfqpoint{0.817291in}{1.381728in}}%
\pgfpathlineto{\pgfqpoint{0.869944in}{1.318090in}}%
\pgfpathlineto{\pgfqpoint{0.931594in}{1.263738in}}%
\pgfpathlineto{\pgfqpoint{1.000721in}{1.220010in}}%
\pgfpathlineto{\pgfqpoint{1.075624in}{1.187983in}}%
\pgfpathlineto{\pgfqpoint{1.154458in}{1.168446in}}%
\pgfpathlineto{\pgfqpoint{1.235283in}{1.161880in}}%
\pgfpathlineto{\pgfqpoint{1.316107in}{1.168446in}}%
\pgfpathlineto{\pgfqpoint{1.394942in}{1.187983in}}%
\pgfpathlineto{\pgfqpoint{1.469845in}{1.220010in}}%
\pgfpathlineto{\pgfqpoint{1.538972in}{1.263738in}}%
\pgfpathlineto{\pgfqpoint{1.600621in}{1.318090in}}%
\pgfpathlineto{\pgfqpoint{1.653275in}{1.381728in}}%
\pgfpathlineto{\pgfqpoint{1.695636in}{1.453085in}}%
\pgfpathlineto{\pgfqpoint{1.726662in}{1.530404in}}%
\pgfpathlineto{\pgfqpoint{1.745588in}{1.611782in}}%
\pgfpathlineto{\pgfqpoint{1.751949in}{1.695214in}}%
\pgfusepath{stroke}%
\end{pgfscope}%
\begin{pgfscope}%
\pgfpathrectangle{\pgfqpoint{0.375000in}{0.300000in}}{\pgfqpoint{2.325000in}{2.400000in}} %
\pgfusepath{clip}%
\pgfsetbuttcap%
\pgfsetroundjoin%
\definecolor{currentfill}{rgb}{0.000000,0.000000,1.000000}%
\pgfsetfillcolor{currentfill}%
\pgfsetlinewidth{0.501875pt}%
\definecolor{currentstroke}{rgb}{0.000000,0.000000,0.000000}%
\pgfsetstrokecolor{currentstroke}%
\pgfsetdash{}{0pt}%
\pgfsys@defobject{currentmarker}{\pgfqpoint{-0.041667in}{-0.041667in}}{\pgfqpoint{0.041667in}{0.041667in}}{%
\pgfpathmoveto{\pgfqpoint{0.000000in}{-0.041667in}}%
\pgfpathcurveto{\pgfqpoint{0.011050in}{-0.041667in}}{\pgfqpoint{0.021649in}{-0.037276in}}{\pgfqpoint{0.029463in}{-0.029463in}}%
\pgfpathcurveto{\pgfqpoint{0.037276in}{-0.021649in}}{\pgfqpoint{0.041667in}{-0.011050in}}{\pgfqpoint{0.041667in}{0.000000in}}%
\pgfpathcurveto{\pgfqpoint{0.041667in}{0.011050in}}{\pgfqpoint{0.037276in}{0.021649in}}{\pgfqpoint{0.029463in}{0.029463in}}%
\pgfpathcurveto{\pgfqpoint{0.021649in}{0.037276in}}{\pgfqpoint{0.011050in}{0.041667in}}{\pgfqpoint{0.000000in}{0.041667in}}%
\pgfpathcurveto{\pgfqpoint{-0.011050in}{0.041667in}}{\pgfqpoint{-0.021649in}{0.037276in}}{\pgfqpoint{-0.029463in}{0.029463in}}%
\pgfpathcurveto{\pgfqpoint{-0.037276in}{0.021649in}}{\pgfqpoint{-0.041667in}{0.011050in}}{\pgfqpoint{-0.041667in}{0.000000in}}%
\pgfpathcurveto{\pgfqpoint{-0.041667in}{-0.011050in}}{\pgfqpoint{-0.037276in}{-0.021649in}}{\pgfqpoint{-0.029463in}{-0.029463in}}%
\pgfpathcurveto{\pgfqpoint{-0.021649in}{-0.037276in}}{\pgfqpoint{-0.011050in}{-0.041667in}}{\pgfqpoint{0.000000in}{-0.041667in}}%
\pgfpathclose%
\pgfusepath{stroke,fill}%
}%
\begin{pgfscope}%
\pgfsys@transformshift{1.235283in}{1.695214in}%
\pgfsys@useobject{currentmarker}{}%
\end{pgfscope}%
\end{pgfscope}%
\begin{pgfscope}%
\pgfpathrectangle{\pgfqpoint{0.375000in}{0.300000in}}{\pgfqpoint{2.325000in}{2.400000in}} %
\pgfusepath{clip}%
\pgfsetbuttcap%
\pgfsetroundjoin%
\definecolor{currentfill}{rgb}{0.000000,0.500000,0.000000}%
\pgfsetfillcolor{currentfill}%
\pgfsetlinewidth{0.501875pt}%
\definecolor{currentstroke}{rgb}{0.000000,0.000000,0.000000}%
\pgfsetstrokecolor{currentstroke}%
\pgfsetdash{}{0pt}%
\pgfsys@defobject{currentmarker}{\pgfqpoint{-0.041667in}{-0.041667in}}{\pgfqpoint{0.041667in}{0.041667in}}{%
\pgfpathmoveto{\pgfqpoint{0.000000in}{-0.041667in}}%
\pgfpathcurveto{\pgfqpoint{0.011050in}{-0.041667in}}{\pgfqpoint{0.021649in}{-0.037276in}}{\pgfqpoint{0.029463in}{-0.029463in}}%
\pgfpathcurveto{\pgfqpoint{0.037276in}{-0.021649in}}{\pgfqpoint{0.041667in}{-0.011050in}}{\pgfqpoint{0.041667in}{0.000000in}}%
\pgfpathcurveto{\pgfqpoint{0.041667in}{0.011050in}}{\pgfqpoint{0.037276in}{0.021649in}}{\pgfqpoint{0.029463in}{0.029463in}}%
\pgfpathcurveto{\pgfqpoint{0.021649in}{0.037276in}}{\pgfqpoint{0.011050in}{0.041667in}}{\pgfqpoint{0.000000in}{0.041667in}}%
\pgfpathcurveto{\pgfqpoint{-0.011050in}{0.041667in}}{\pgfqpoint{-0.021649in}{0.037276in}}{\pgfqpoint{-0.029463in}{0.029463in}}%
\pgfpathcurveto{\pgfqpoint{-0.037276in}{0.021649in}}{\pgfqpoint{-0.041667in}{0.011050in}}{\pgfqpoint{-0.041667in}{0.000000in}}%
\pgfpathcurveto{\pgfqpoint{-0.041667in}{-0.011050in}}{\pgfqpoint{-0.037276in}{-0.021649in}}{\pgfqpoint{-0.029463in}{-0.029463in}}%
\pgfpathcurveto{\pgfqpoint{-0.021649in}{-0.037276in}}{\pgfqpoint{-0.011050in}{-0.041667in}}{\pgfqpoint{0.000000in}{-0.041667in}}%
\pgfpathclose%
\pgfusepath{stroke,fill}%
}%
\begin{pgfscope}%
\pgfsys@transformshift{1.493616in}{1.233333in}%
\pgfsys@useobject{currentmarker}{}%
\end{pgfscope}%
\end{pgfscope}%
\begin{pgfscope}%
\pgftext[x=0.633333in,y=0.566667in,left,bottom]{{\sffamily\fontsize{14.000000}{16.800000}\selectfont \(\displaystyle \theta\)}}%
\end{pgfscope}%
\begin{pgfscope}%
\pgftext[x=1.235283in,y=1.695214in,right,bottom]{{\sffamily\fontsize{14.000000}{16.800000}\selectfont \(\displaystyle CM\)}}%
\end{pgfscope}%
\begin{pgfscope}%
\pgftext[x=1.235283in,y=0.942849in,left,top]{{\sffamily\fontsize{14.000000}{16.800000}\selectfont \(\displaystyle mg\)}}%
\end{pgfscope}%
\begin{pgfscope}%
\pgftext[x=0.761353in,y=2.315897in,left,bottom]{{\sffamily\fontsize{14.000000}{16.800000}\selectfont \(\displaystyle a\)}}%
\end{pgfscope}%
\begin{pgfscope}%
\pgftext[x=1.214393in,y=2.585897in,left,bottom]{{\sffamily\fontsize{14.000000}{16.800000}\selectfont \(\displaystyle v\)}}%
\end{pgfscope}%
\begin{pgfscope}%
\pgftext[x=1.537309in,y=1.875214in,left,top]{{\sffamily\fontsize{14.000000}{16.800000}\selectfont \(\displaystyle F_{p}\)}}%
\end{pgfscope}%
\end{pgfpicture}%
\makeatother%
\endgroup%


%% Creator: Matplotlib, PGF backend
%%
%% To include the figure in your LaTeX document, write
%%   \input{<filename>.pgf}
%%
%% Make sure the required packages are loaded in your preamble
%%   \usepackage{pgf}
%%
%% Figures using additional raster images can only be included by \input if
%% they are in the same directory as the main LaTeX file. For loading figures
%% from other directories you can use the `import` package
%%   \usepackage{import}
%% and then include the figures with
%%   \import{<path to file>}{<filename>.pgf}
%%
%% Matplotlib used the following preamble
%%   \usepackage{fontspec}
%%   \setmainfont{Times New Roman}
%%   \setsansfont{Verdana}
%%   \setmonofont{Courier New}
%%
\begingroup%
\makeatletter%
\begin{pgfpicture}%
\pgfpathrectangle{\pgfpointorigin}{\pgfqpoint{3.000000in}{3.000000in}}%
\pgfusepath{use as bounding box}%
\begin{pgfscope}%
\pgfsetbuttcap%
\pgfsetroundjoin%
\definecolor{currentfill}{rgb}{1.000000,1.000000,1.000000}%
\pgfsetfillcolor{currentfill}%
\pgfsetlinewidth{0.000000pt}%
\definecolor{currentstroke}{rgb}{1.000000,1.000000,1.000000}%
\pgfsetstrokecolor{currentstroke}%
\pgfsetdash{}{0pt}%
\pgfpathmoveto{\pgfqpoint{0.000000in}{0.000000in}}%
\pgfpathlineto{\pgfqpoint{3.000000in}{0.000000in}}%
\pgfpathlineto{\pgfqpoint{3.000000in}{3.000000in}}%
\pgfpathlineto{\pgfqpoint{0.000000in}{3.000000in}}%
\pgfpathclose%
\pgfusepath{fill}%
\end{pgfscope}%
\begin{pgfscope}%
\pgfpathrectangle{\pgfqpoint{0.375000in}{0.300000in}}{\pgfqpoint{2.325000in}{2.400000in}} %
\pgfusepath{clip}%
\pgfsetbuttcap%
\pgfsetroundjoin%
\definecolor{currentfill}{rgb}{0.000000,0.000000,1.000000}%
\pgfsetfillcolor{currentfill}%
\pgfsetlinewidth{1.003750pt}%
\definecolor{currentstroke}{rgb}{0.000000,0.000000,0.000000}%
\pgfsetstrokecolor{currentstroke}%
\pgfsetdash{}{0pt}%
\pgfpathmoveto{\pgfqpoint{1.896318in}{1.473333in}}%
\pgfpathlineto{\pgfqpoint{1.821476in}{1.367145in}}%
\pgfpathlineto{\pgfqpoint{1.801455in}{1.402941in}}%
\pgfpathlineto{\pgfqpoint{1.499429in}{1.222941in}}%
\pgfpathlineto{\pgfqpoint{1.487804in}{1.243726in}}%
\pgfpathlineto{\pgfqpoint{1.789830in}{1.423726in}}%
\pgfpathlineto{\pgfqpoint{1.769809in}{1.459521in}}%
\pgfpathclose%
\pgfusepath{stroke,fill}%
\end{pgfscope}%
\begin{pgfscope}%
\pgfpathrectangle{\pgfqpoint{0.375000in}{0.300000in}}{\pgfqpoint{2.325000in}{2.400000in}} %
\pgfusepath{clip}%
\pgfsetbuttcap%
\pgfsetroundjoin%
\definecolor{currentfill}{rgb}{0.000000,0.000000,1.000000}%
\pgfsetfillcolor{currentfill}%
\pgfsetlinewidth{1.003750pt}%
\definecolor{currentstroke}{rgb}{0.000000,0.000000,0.000000}%
\pgfsetstrokecolor{currentstroke}%
\pgfsetdash{}{0pt}%
\pgfpathmoveto{\pgfqpoint{1.235283in}{0.822849in}}%
\pgfpathlineto{\pgfqpoint{1.183616in}{0.942849in}}%
\pgfpathlineto{\pgfqpoint{1.223658in}{0.942849in}}%
\pgfpathlineto{\pgfqpoint{1.223658in}{1.695214in}}%
\pgfpathlineto{\pgfqpoint{1.246908in}{1.695214in}}%
\pgfpathlineto{\pgfqpoint{1.246908in}{0.942849in}}%
\pgfpathlineto{\pgfqpoint{1.286949in}{0.942849in}}%
\pgfpathclose%
\pgfusepath{stroke,fill}%
\end{pgfscope}%
\begin{pgfscope}%
\pgfpathrectangle{\pgfqpoint{0.375000in}{0.300000in}}{\pgfqpoint{2.325000in}{2.400000in}} %
\pgfusepath{clip}%
\pgfsetbuttcap%
\pgfsetroundjoin%
\definecolor{currentfill}{rgb}{1.000000,0.000000,0.000000}%
\pgfsetfillcolor{currentfill}%
\pgfsetlinewidth{1.003750pt}%
\definecolor{currentstroke}{rgb}{1.000000,0.000000,0.000000}%
\pgfsetstrokecolor{currentstroke}%
\pgfsetdash{}{0pt}%
\pgfpathmoveto{\pgfqpoint{0.627119in}{2.102564in}}%
\pgfpathlineto{\pgfqpoint{0.696770in}{2.298034in}}%
\pgfpathlineto{\pgfqpoint{0.748436in}{2.205658in}}%
\pgfpathlineto{\pgfqpoint{0.899449in}{2.295658in}}%
\pgfpathlineto{\pgfqpoint{0.925283in}{2.249470in}}%
\pgfpathlineto{\pgfqpoint{0.774270in}{2.159470in}}%
\pgfpathlineto{\pgfqpoint{0.825936in}{2.067094in}}%
\pgfpathclose%
\pgfusepath{stroke,fill}%
\end{pgfscope}%
\begin{pgfscope}%
\pgfpathrectangle{\pgfqpoint{0.375000in}{0.300000in}}{\pgfqpoint{2.325000in}{2.400000in}} %
\pgfusepath{clip}%
\pgfsetbuttcap%
\pgfsetroundjoin%
\definecolor{currentfill}{rgb}{0.000000,0.000000,0.000000}%
\pgfsetfillcolor{currentfill}%
\pgfsetlinewidth{1.003750pt}%
\definecolor{currentstroke}{rgb}{0.000000,0.000000,0.000000}%
\pgfsetstrokecolor{currentstroke}%
\pgfsetdash{}{0pt}%
\pgfpathmoveto{\pgfqpoint{1.348626in}{2.532564in}}%
\pgfpathlineto{\pgfqpoint{1.278976in}{2.337094in}}%
\pgfpathlineto{\pgfqpoint{1.227309in}{2.429470in}}%
\pgfpathlineto{\pgfqpoint{0.925283in}{2.249470in}}%
\pgfpathlineto{\pgfqpoint{0.899449in}{2.295658in}}%
\pgfpathlineto{\pgfqpoint{1.201476in}{2.475658in}}%
\pgfpathlineto{\pgfqpoint{1.149809in}{2.568034in}}%
\pgfpathclose%
\pgfusepath{stroke,fill}%
\end{pgfscope}%
\begin{pgfscope}%
\pgfpathrectangle{\pgfqpoint{0.375000in}{0.300000in}}{\pgfqpoint{2.325000in}{2.400000in}} %
\pgfusepath{clip}%
\pgfsetbuttcap%
\pgfsetroundjoin%
\definecolor{currentfill}{rgb}{0.000000,0.000000,1.000000}%
\pgfsetfillcolor{currentfill}%
\pgfsetlinewidth{1.003750pt}%
\definecolor{currentstroke}{rgb}{0.000000,0.000000,0.000000}%
\pgfsetstrokecolor{currentstroke}%
\pgfsetdash{}{0pt}%
\pgfpathmoveto{\pgfqpoint{1.637985in}{1.935214in}}%
\pgfpathlineto{\pgfqpoint{1.563143in}{1.829026in}}%
\pgfpathlineto{\pgfqpoint{1.543122in}{1.864821in}}%
\pgfpathlineto{\pgfqpoint{1.241095in}{1.684821in}}%
\pgfpathlineto{\pgfqpoint{1.229470in}{1.705606in}}%
\pgfpathlineto{\pgfqpoint{1.531497in}{1.885606in}}%
\pgfpathlineto{\pgfqpoint{1.511476in}{1.921402in}}%
\pgfpathclose%
\pgfusepath{stroke,fill}%
\end{pgfscope}%
\begin{pgfscope}%
\pgfpathrectangle{\pgfqpoint{0.375000in}{0.300000in}}{\pgfqpoint{2.325000in}{2.400000in}} %
\pgfusepath{clip}%
\pgfsetrectcap%
\pgfsetroundjoin%
\pgfsetlinewidth{1.003750pt}%
\definecolor{currentstroke}{rgb}{0.000000,0.000000,1.000000}%
\pgfsetstrokecolor{currentstroke}%
\pgfsetdash{}{0pt}%
\pgfpathmoveto{\pgfqpoint{0.375000in}{0.566667in}}%
\pgfpathlineto{\pgfqpoint{2.612232in}{0.566667in}}%
\pgfusepath{stroke}%
\end{pgfscope}%
\begin{pgfscope}%
\pgfpathrectangle{\pgfqpoint{0.375000in}{0.300000in}}{\pgfqpoint{2.325000in}{2.400000in}} %
\pgfusepath{clip}%
\pgfsetrectcap%
\pgfsetroundjoin%
\pgfsetlinewidth{1.003750pt}%
\definecolor{currentstroke}{rgb}{0.000000,0.000000,1.000000}%
\pgfsetstrokecolor{currentstroke}%
\pgfsetdash{}{0pt}%
\pgfpathmoveto{\pgfqpoint{2.612232in}{0.566667in}}%
\pgfpathlineto{\pgfqpoint{2.612232in}{1.900000in}}%
\pgfusepath{stroke}%
\end{pgfscope}%
\begin{pgfscope}%
\pgfpathrectangle{\pgfqpoint{0.375000in}{0.300000in}}{\pgfqpoint{2.325000in}{2.400000in}} %
\pgfusepath{clip}%
\pgfsetrectcap%
\pgfsetroundjoin%
\pgfsetlinewidth{1.003750pt}%
\definecolor{currentstroke}{rgb}{0.000000,0.000000,1.000000}%
\pgfsetstrokecolor{currentstroke}%
\pgfsetdash{}{0pt}%
\pgfpathmoveto{\pgfqpoint{0.375000in}{0.566667in}}%
\pgfpathlineto{\pgfqpoint{2.612232in}{1.900000in}}%
\pgfusepath{stroke}%
\end{pgfscope}%
\begin{pgfscope}%
\pgfpathrectangle{\pgfqpoint{0.375000in}{0.300000in}}{\pgfqpoint{2.325000in}{2.400000in}} %
\pgfusepath{clip}%
\pgfsetrectcap%
\pgfsetroundjoin%
\pgfsetlinewidth{1.003750pt}%
\definecolor{currentstroke}{rgb}{0.000000,0.000000,0.000000}%
\pgfsetstrokecolor{currentstroke}%
\pgfsetdash{}{0pt}%
\pgfpathmoveto{\pgfqpoint{0.504167in}{0.566667in}}%
\pgfpathlineto{\pgfqpoint{0.503990in}{0.573645in}}%
\pgfpathlineto{\pgfqpoint{0.503459in}{0.580604in}}%
\pgfpathlineto{\pgfqpoint{0.502576in}{0.587525in}}%
\pgfpathlineto{\pgfqpoint{0.501344in}{0.594388in}}%
\pgfpathlineto{\pgfqpoint{0.499765in}{0.601176in}}%
\pgfpathlineto{\pgfqpoint{0.497845in}{0.607869in}}%
\pgfpathlineto{\pgfqpoint{0.495587in}{0.614449in}}%
\pgfpathlineto{\pgfqpoint{0.493000in}{0.620898in}}%
\pgfpathlineto{\pgfqpoint{0.490088in}{0.627199in}}%
\pgfpathlineto{\pgfqpoint{0.486862in}{0.633333in}}%
\pgfusepath{stroke}%
\end{pgfscope}%
\begin{pgfscope}%
\pgfpathrectangle{\pgfqpoint{0.375000in}{0.300000in}}{\pgfqpoint{2.325000in}{2.400000in}} %
\pgfusepath{clip}%
\pgfsetrectcap%
\pgfsetroundjoin%
\pgfsetlinewidth{1.003750pt}%
\definecolor{currentstroke}{rgb}{0.000000,0.000000,0.000000}%
\pgfsetstrokecolor{currentstroke}%
\pgfsetdash{}{0pt}%
\pgfpathmoveto{\pgfqpoint{1.751949in}{1.695214in}}%
\pgfpathlineto{\pgfqpoint{1.745588in}{1.778645in}}%
\pgfpathlineto{\pgfqpoint{1.726662in}{1.860023in}}%
\pgfpathlineto{\pgfqpoint{1.695636in}{1.937342in}}%
\pgfpathlineto{\pgfqpoint{1.653275in}{2.008699in}}%
\pgfpathlineto{\pgfqpoint{1.600621in}{2.072337in}}%
\pgfpathlineto{\pgfqpoint{1.538972in}{2.126689in}}%
\pgfpathlineto{\pgfqpoint{1.469845in}{2.170417in}}%
\pgfpathlineto{\pgfqpoint{1.394942in}{2.202444in}}%
\pgfpathlineto{\pgfqpoint{1.316107in}{2.221981in}}%
\pgfpathlineto{\pgfqpoint{1.235283in}{2.228547in}}%
\pgfpathlineto{\pgfqpoint{1.154458in}{2.221981in}}%
\pgfpathlineto{\pgfqpoint{1.075624in}{2.202444in}}%
\pgfpathlineto{\pgfqpoint{1.000721in}{2.170417in}}%
\pgfpathlineto{\pgfqpoint{0.931594in}{2.126689in}}%
\pgfpathlineto{\pgfqpoint{0.869944in}{2.072337in}}%
\pgfpathlineto{\pgfqpoint{0.817291in}{2.008699in}}%
\pgfpathlineto{\pgfqpoint{0.774929in}{1.937342in}}%
\pgfpathlineto{\pgfqpoint{0.743904in}{1.860023in}}%
\pgfpathlineto{\pgfqpoint{0.724977in}{1.778645in}}%
\pgfpathlineto{\pgfqpoint{0.718616in}{1.695214in}}%
\pgfusepath{stroke}%
\end{pgfscope}%
\begin{pgfscope}%
\pgfpathrectangle{\pgfqpoint{0.375000in}{0.300000in}}{\pgfqpoint{2.325000in}{2.400000in}} %
\pgfusepath{clip}%
\pgfsetrectcap%
\pgfsetroundjoin%
\pgfsetlinewidth{1.003750pt}%
\definecolor{currentstroke}{rgb}{0.000000,0.000000,0.000000}%
\pgfsetstrokecolor{currentstroke}%
\pgfsetdash{}{0pt}%
\pgfpathmoveto{\pgfqpoint{0.718616in}{1.695214in}}%
\pgfpathlineto{\pgfqpoint{0.724977in}{1.611782in}}%
\pgfpathlineto{\pgfqpoint{0.743904in}{1.530404in}}%
\pgfpathlineto{\pgfqpoint{0.774929in}{1.453085in}}%
\pgfpathlineto{\pgfqpoint{0.817291in}{1.381728in}}%
\pgfpathlineto{\pgfqpoint{0.869944in}{1.318090in}}%
\pgfpathlineto{\pgfqpoint{0.931594in}{1.263738in}}%
\pgfpathlineto{\pgfqpoint{1.000721in}{1.220010in}}%
\pgfpathlineto{\pgfqpoint{1.075624in}{1.187983in}}%
\pgfpathlineto{\pgfqpoint{1.154458in}{1.168446in}}%
\pgfpathlineto{\pgfqpoint{1.235283in}{1.161880in}}%
\pgfpathlineto{\pgfqpoint{1.316107in}{1.168446in}}%
\pgfpathlineto{\pgfqpoint{1.394942in}{1.187983in}}%
\pgfpathlineto{\pgfqpoint{1.469845in}{1.220010in}}%
\pgfpathlineto{\pgfqpoint{1.538972in}{1.263738in}}%
\pgfpathlineto{\pgfqpoint{1.600621in}{1.318090in}}%
\pgfpathlineto{\pgfqpoint{1.653275in}{1.381728in}}%
\pgfpathlineto{\pgfqpoint{1.695636in}{1.453085in}}%
\pgfpathlineto{\pgfqpoint{1.726662in}{1.530404in}}%
\pgfpathlineto{\pgfqpoint{1.745588in}{1.611782in}}%
\pgfpathlineto{\pgfqpoint{1.751949in}{1.695214in}}%
\pgfusepath{stroke}%
\end{pgfscope}%
\begin{pgfscope}%
\pgfpathrectangle{\pgfqpoint{0.375000in}{0.300000in}}{\pgfqpoint{2.325000in}{2.400000in}} %
\pgfusepath{clip}%
\pgfsetbuttcap%
\pgfsetroundjoin%
\definecolor{currentfill}{rgb}{0.000000,0.000000,1.000000}%
\pgfsetfillcolor{currentfill}%
\pgfsetlinewidth{0.501875pt}%
\definecolor{currentstroke}{rgb}{0.000000,0.000000,0.000000}%
\pgfsetstrokecolor{currentstroke}%
\pgfsetdash{}{0pt}%
\pgfsys@defobject{currentmarker}{\pgfqpoint{-0.041667in}{-0.041667in}}{\pgfqpoint{0.041667in}{0.041667in}}{%
\pgfpathmoveto{\pgfqpoint{0.000000in}{-0.041667in}}%
\pgfpathcurveto{\pgfqpoint{0.011050in}{-0.041667in}}{\pgfqpoint{0.021649in}{-0.037276in}}{\pgfqpoint{0.029463in}{-0.029463in}}%
\pgfpathcurveto{\pgfqpoint{0.037276in}{-0.021649in}}{\pgfqpoint{0.041667in}{-0.011050in}}{\pgfqpoint{0.041667in}{0.000000in}}%
\pgfpathcurveto{\pgfqpoint{0.041667in}{0.011050in}}{\pgfqpoint{0.037276in}{0.021649in}}{\pgfqpoint{0.029463in}{0.029463in}}%
\pgfpathcurveto{\pgfqpoint{0.021649in}{0.037276in}}{\pgfqpoint{0.011050in}{0.041667in}}{\pgfqpoint{0.000000in}{0.041667in}}%
\pgfpathcurveto{\pgfqpoint{-0.011050in}{0.041667in}}{\pgfqpoint{-0.021649in}{0.037276in}}{\pgfqpoint{-0.029463in}{0.029463in}}%
\pgfpathcurveto{\pgfqpoint{-0.037276in}{0.021649in}}{\pgfqpoint{-0.041667in}{0.011050in}}{\pgfqpoint{-0.041667in}{0.000000in}}%
\pgfpathcurveto{\pgfqpoint{-0.041667in}{-0.011050in}}{\pgfqpoint{-0.037276in}{-0.021649in}}{\pgfqpoint{-0.029463in}{-0.029463in}}%
\pgfpathcurveto{\pgfqpoint{-0.021649in}{-0.037276in}}{\pgfqpoint{-0.011050in}{-0.041667in}}{\pgfqpoint{0.000000in}{-0.041667in}}%
\pgfpathclose%
\pgfusepath{stroke,fill}%
}%
\begin{pgfscope}%
\pgfsys@transformshift{1.235283in}{1.695214in}%
\pgfsys@useobject{currentmarker}{}%
\end{pgfscope}%
\end{pgfscope}%
\begin{pgfscope}%
\pgfpathrectangle{\pgfqpoint{0.375000in}{0.300000in}}{\pgfqpoint{2.325000in}{2.400000in}} %
\pgfusepath{clip}%
\pgfsetbuttcap%
\pgfsetroundjoin%
\definecolor{currentfill}{rgb}{0.000000,0.500000,0.000000}%
\pgfsetfillcolor{currentfill}%
\pgfsetlinewidth{0.501875pt}%
\definecolor{currentstroke}{rgb}{0.000000,0.000000,0.000000}%
\pgfsetstrokecolor{currentstroke}%
\pgfsetdash{}{0pt}%
\pgfsys@defobject{currentmarker}{\pgfqpoint{-0.041667in}{-0.041667in}}{\pgfqpoint{0.041667in}{0.041667in}}{%
\pgfpathmoveto{\pgfqpoint{0.000000in}{-0.041667in}}%
\pgfpathcurveto{\pgfqpoint{0.011050in}{-0.041667in}}{\pgfqpoint{0.021649in}{-0.037276in}}{\pgfqpoint{0.029463in}{-0.029463in}}%
\pgfpathcurveto{\pgfqpoint{0.037276in}{-0.021649in}}{\pgfqpoint{0.041667in}{-0.011050in}}{\pgfqpoint{0.041667in}{0.000000in}}%
\pgfpathcurveto{\pgfqpoint{0.041667in}{0.011050in}}{\pgfqpoint{0.037276in}{0.021649in}}{\pgfqpoint{0.029463in}{0.029463in}}%
\pgfpathcurveto{\pgfqpoint{0.021649in}{0.037276in}}{\pgfqpoint{0.011050in}{0.041667in}}{\pgfqpoint{0.000000in}{0.041667in}}%
\pgfpathcurveto{\pgfqpoint{-0.011050in}{0.041667in}}{\pgfqpoint{-0.021649in}{0.037276in}}{\pgfqpoint{-0.029463in}{0.029463in}}%
\pgfpathcurveto{\pgfqpoint{-0.037276in}{0.021649in}}{\pgfqpoint{-0.041667in}{0.011050in}}{\pgfqpoint{-0.041667in}{0.000000in}}%
\pgfpathcurveto{\pgfqpoint{-0.041667in}{-0.011050in}}{\pgfqpoint{-0.037276in}{-0.021649in}}{\pgfqpoint{-0.029463in}{-0.029463in}}%
\pgfpathcurveto{\pgfqpoint{-0.021649in}{-0.037276in}}{\pgfqpoint{-0.011050in}{-0.041667in}}{\pgfqpoint{0.000000in}{-0.041667in}}%
\pgfpathclose%
\pgfusepath{stroke,fill}%
}%
\begin{pgfscope}%
\pgfsys@transformshift{1.493616in}{1.233333in}%
\pgfsys@useobject{currentmarker}{}%
\end{pgfscope}%
\end{pgfscope}%
\begin{pgfscope}%
\pgftext[x=0.633333in,y=0.566667in,left,bottom]{{\sffamily\fontsize{14.000000}{16.800000}\selectfont \(\displaystyle \theta\)}}%
\end{pgfscope}%
\begin{pgfscope}%
\pgftext[x=1.235283in,y=1.695214in,right,bottom]{{\sffamily\fontsize{14.000000}{16.800000}\selectfont \(\displaystyle CM\)}}%
\end{pgfscope}%
\begin{pgfscope}%
\pgftext[x=1.821476in,y=1.413333in,left,top]{{\sffamily\fontsize{14.000000}{16.800000}\selectfont \(\displaystyle F_{fr}\)}}%
\end{pgfscope}%
\begin{pgfscope}%
\pgftext[x=1.235283in,y=0.942849in,left,top]{{\sffamily\fontsize{14.000000}{16.800000}\selectfont \(\displaystyle mg\)}}%
\end{pgfscope}%
\begin{pgfscope}%
\pgftext[x=0.761353in,y=2.315897in,left,bottom]{{\sffamily\fontsize{14.000000}{16.800000}\selectfont \(\displaystyle a\)}}%
\end{pgfscope}%
\begin{pgfscope}%
\pgftext[x=1.214393in,y=2.585897in,left,bottom]{{\sffamily\fontsize{14.000000}{16.800000}\selectfont \(\displaystyle v\)}}%
\end{pgfscope}%
\begin{pgfscope}%
\pgftext[x=1.537309in,y=1.875214in,left,top]{{\sffamily\fontsize{14.000000}{16.800000}\selectfont \(\displaystyle F_{p}\)}}%
\end{pgfscope}%
\end{pgfpicture}%
\makeatother%
\endgroup%


\newpage

\begin{problem}
An applied force $F_{p}$ passing through the top edge of the roller, up the
hill. The roller is accelerating up the hill. Determine the direction of
friction acting on the roller.
\end{problem}

%% Creator: Matplotlib, PGF backend
%%
%% To include the figure in your LaTeX document, write
%%   \input{<filename>.pgf}
%%
%% Make sure the required packages are loaded in your preamble
%%   \usepackage{pgf}
%%
%% Figures using additional raster images can only be included by \input if
%% they are in the same directory as the main LaTeX file. For loading figures
%% from other directories you can use the `import` package
%%   \usepackage{import}
%% and then include the figures with
%%   \import{<path to file>}{<filename>.pgf}
%%
%% Matplotlib used the following preamble
%%   \usepackage{fontspec}
%%   \setmainfont{Times New Roman}
%%   \setsansfont{Verdana}
%%   \setmonofont{Courier New}
%%
\begingroup%
\makeatletter%
\begin{pgfpicture}%
\pgfpathrectangle{\pgfpointorigin}{\pgfqpoint{3.000000in}{3.000000in}}%
\pgfusepath{use as bounding box}%
\begin{pgfscope}%
\pgfsetbuttcap%
\pgfsetroundjoin%
\definecolor{currentfill}{rgb}{1.000000,1.000000,1.000000}%
\pgfsetfillcolor{currentfill}%
\pgfsetlinewidth{0.000000pt}%
\definecolor{currentstroke}{rgb}{1.000000,1.000000,1.000000}%
\pgfsetstrokecolor{currentstroke}%
\pgfsetdash{}{0pt}%
\pgfpathmoveto{\pgfqpoint{0.000000in}{0.000000in}}%
\pgfpathlineto{\pgfqpoint{3.000000in}{0.000000in}}%
\pgfpathlineto{\pgfqpoint{3.000000in}{3.000000in}}%
\pgfpathlineto{\pgfqpoint{0.000000in}{3.000000in}}%
\pgfpathclose%
\pgfusepath{fill}%
\end{pgfscope}%
\begin{pgfscope}%
\pgfpathrectangle{\pgfqpoint{0.375000in}{0.300000in}}{\pgfqpoint{2.325000in}{2.400000in}} %
\pgfusepath{clip}%
\pgfsetbuttcap%
\pgfsetroundjoin%
\definecolor{currentfill}{rgb}{0.000000,0.000000,1.000000}%
\pgfsetfillcolor{currentfill}%
\pgfsetlinewidth{1.003750pt}%
\definecolor{currentstroke}{rgb}{0.000000,0.000000,0.000000}%
\pgfsetstrokecolor{currentstroke}%
\pgfsetdash{}{0pt}%
\pgfpathmoveto{\pgfqpoint{1.235283in}{0.822849in}}%
\pgfpathlineto{\pgfqpoint{1.183616in}{0.942849in}}%
\pgfpathlineto{\pgfqpoint{1.223658in}{0.942849in}}%
\pgfpathlineto{\pgfqpoint{1.223658in}{1.695214in}}%
\pgfpathlineto{\pgfqpoint{1.246908in}{1.695214in}}%
\pgfpathlineto{\pgfqpoint{1.246908in}{0.942849in}}%
\pgfpathlineto{\pgfqpoint{1.286949in}{0.942849in}}%
\pgfpathclose%
\pgfusepath{stroke,fill}%
\end{pgfscope}%
\begin{pgfscope}%
\pgfpathrectangle{\pgfqpoint{0.375000in}{0.300000in}}{\pgfqpoint{2.325000in}{2.400000in}} %
\pgfusepath{clip}%
\pgfsetbuttcap%
\pgfsetroundjoin%
\definecolor{currentfill}{rgb}{1.000000,0.000000,0.000000}%
\pgfsetfillcolor{currentfill}%
\pgfsetlinewidth{1.003750pt}%
\definecolor{currentstroke}{rgb}{1.000000,0.000000,0.000000}%
\pgfsetstrokecolor{currentstroke}%
\pgfsetdash{}{0pt}%
\pgfpathmoveto{\pgfqpoint{2.376271in}{2.375214in}}%
\pgfpathlineto{\pgfqpoint{2.306621in}{2.179743in}}%
\pgfpathlineto{\pgfqpoint{2.254954in}{2.272120in}}%
\pgfpathlineto{\pgfqpoint{1.919369in}{2.072120in}}%
\pgfpathlineto{\pgfqpoint{1.893536in}{2.118308in}}%
\pgfpathlineto{\pgfqpoint{2.229121in}{2.318308in}}%
\pgfpathlineto{\pgfqpoint{2.177454in}{2.410684in}}%
\pgfpathclose%
\pgfusepath{stroke,fill}%
\end{pgfscope}%
\begin{pgfscope}%
\pgfpathrectangle{\pgfqpoint{0.375000in}{0.300000in}}{\pgfqpoint{2.325000in}{2.400000in}} %
\pgfusepath{clip}%
\pgfsetbuttcap%
\pgfsetroundjoin%
\definecolor{currentfill}{rgb}{0.000000,0.000000,1.000000}%
\pgfsetfillcolor{currentfill}%
\pgfsetlinewidth{1.003750pt}%
\definecolor{currentstroke}{rgb}{0.000000,0.000000,0.000000}%
\pgfsetstrokecolor{currentstroke}%
\pgfsetdash{}{0pt}%
\pgfpathmoveto{\pgfqpoint{1.748795in}{2.617094in}}%
\pgfpathlineto{\pgfqpoint{1.673953in}{2.510906in}}%
\pgfpathlineto{\pgfqpoint{1.653932in}{2.546701in}}%
\pgfpathlineto{\pgfqpoint{0.982762in}{2.146701in}}%
\pgfpathlineto{\pgfqpoint{0.971137in}{2.167486in}}%
\pgfpathlineto{\pgfqpoint{1.642307in}{2.567486in}}%
\pgfpathlineto{\pgfqpoint{1.622286in}{2.603282in}}%
\pgfpathclose%
\pgfusepath{stroke,fill}%
\end{pgfscope}%
\begin{pgfscope}%
\pgfpathrectangle{\pgfqpoint{0.375000in}{0.300000in}}{\pgfqpoint{2.325000in}{2.400000in}} %
\pgfusepath{clip}%
\pgfsetrectcap%
\pgfsetroundjoin%
\pgfsetlinewidth{1.003750pt}%
\definecolor{currentstroke}{rgb}{0.000000,0.000000,1.000000}%
\pgfsetstrokecolor{currentstroke}%
\pgfsetdash{}{0pt}%
\pgfpathmoveto{\pgfqpoint{0.375000in}{0.566667in}}%
\pgfpathlineto{\pgfqpoint{2.612232in}{0.566667in}}%
\pgfusepath{stroke}%
\end{pgfscope}%
\begin{pgfscope}%
\pgfpathrectangle{\pgfqpoint{0.375000in}{0.300000in}}{\pgfqpoint{2.325000in}{2.400000in}} %
\pgfusepath{clip}%
\pgfsetrectcap%
\pgfsetroundjoin%
\pgfsetlinewidth{1.003750pt}%
\definecolor{currentstroke}{rgb}{0.000000,0.000000,1.000000}%
\pgfsetstrokecolor{currentstroke}%
\pgfsetdash{}{0pt}%
\pgfpathmoveto{\pgfqpoint{2.612232in}{0.566667in}}%
\pgfpathlineto{\pgfqpoint{2.612232in}{1.900000in}}%
\pgfusepath{stroke}%
\end{pgfscope}%
\begin{pgfscope}%
\pgfpathrectangle{\pgfqpoint{0.375000in}{0.300000in}}{\pgfqpoint{2.325000in}{2.400000in}} %
\pgfusepath{clip}%
\pgfsetrectcap%
\pgfsetroundjoin%
\pgfsetlinewidth{1.003750pt}%
\definecolor{currentstroke}{rgb}{0.000000,0.000000,1.000000}%
\pgfsetstrokecolor{currentstroke}%
\pgfsetdash{}{0pt}%
\pgfpathmoveto{\pgfqpoint{0.375000in}{0.566667in}}%
\pgfpathlineto{\pgfqpoint{2.612232in}{1.900000in}}%
\pgfusepath{stroke}%
\end{pgfscope}%
\begin{pgfscope}%
\pgfpathrectangle{\pgfqpoint{0.375000in}{0.300000in}}{\pgfqpoint{2.325000in}{2.400000in}} %
\pgfusepath{clip}%
\pgfsetrectcap%
\pgfsetroundjoin%
\pgfsetlinewidth{1.003750pt}%
\definecolor{currentstroke}{rgb}{0.000000,0.000000,0.000000}%
\pgfsetstrokecolor{currentstroke}%
\pgfsetdash{}{0pt}%
\pgfpathmoveto{\pgfqpoint{0.504167in}{0.566667in}}%
\pgfpathlineto{\pgfqpoint{0.503990in}{0.573645in}}%
\pgfpathlineto{\pgfqpoint{0.503459in}{0.580604in}}%
\pgfpathlineto{\pgfqpoint{0.502576in}{0.587525in}}%
\pgfpathlineto{\pgfqpoint{0.501344in}{0.594388in}}%
\pgfpathlineto{\pgfqpoint{0.499765in}{0.601176in}}%
\pgfpathlineto{\pgfqpoint{0.497845in}{0.607869in}}%
\pgfpathlineto{\pgfqpoint{0.495587in}{0.614449in}}%
\pgfpathlineto{\pgfqpoint{0.493000in}{0.620898in}}%
\pgfpathlineto{\pgfqpoint{0.490088in}{0.627199in}}%
\pgfpathlineto{\pgfqpoint{0.486862in}{0.633333in}}%
\pgfusepath{stroke}%
\end{pgfscope}%
\begin{pgfscope}%
\pgfpathrectangle{\pgfqpoint{0.375000in}{0.300000in}}{\pgfqpoint{2.325000in}{2.400000in}} %
\pgfusepath{clip}%
\pgfsetrectcap%
\pgfsetroundjoin%
\pgfsetlinewidth{1.003750pt}%
\definecolor{currentstroke}{rgb}{0.000000,0.000000,0.000000}%
\pgfsetstrokecolor{currentstroke}%
\pgfsetdash{}{0pt}%
\pgfpathmoveto{\pgfqpoint{1.751949in}{1.695214in}}%
\pgfpathlineto{\pgfqpoint{1.745588in}{1.778645in}}%
\pgfpathlineto{\pgfqpoint{1.726662in}{1.860023in}}%
\pgfpathlineto{\pgfqpoint{1.695636in}{1.937342in}}%
\pgfpathlineto{\pgfqpoint{1.653275in}{2.008699in}}%
\pgfpathlineto{\pgfqpoint{1.600621in}{2.072337in}}%
\pgfpathlineto{\pgfqpoint{1.538972in}{2.126689in}}%
\pgfpathlineto{\pgfqpoint{1.469845in}{2.170417in}}%
\pgfpathlineto{\pgfqpoint{1.394942in}{2.202444in}}%
\pgfpathlineto{\pgfqpoint{1.316107in}{2.221981in}}%
\pgfpathlineto{\pgfqpoint{1.235283in}{2.228547in}}%
\pgfpathlineto{\pgfqpoint{1.154458in}{2.221981in}}%
\pgfpathlineto{\pgfqpoint{1.075624in}{2.202444in}}%
\pgfpathlineto{\pgfqpoint{1.000721in}{2.170417in}}%
\pgfpathlineto{\pgfqpoint{0.931594in}{2.126689in}}%
\pgfpathlineto{\pgfqpoint{0.869944in}{2.072337in}}%
\pgfpathlineto{\pgfqpoint{0.817291in}{2.008699in}}%
\pgfpathlineto{\pgfqpoint{0.774929in}{1.937342in}}%
\pgfpathlineto{\pgfqpoint{0.743904in}{1.860023in}}%
\pgfpathlineto{\pgfqpoint{0.724977in}{1.778645in}}%
\pgfpathlineto{\pgfqpoint{0.718616in}{1.695214in}}%
\pgfusepath{stroke}%
\end{pgfscope}%
\begin{pgfscope}%
\pgfpathrectangle{\pgfqpoint{0.375000in}{0.300000in}}{\pgfqpoint{2.325000in}{2.400000in}} %
\pgfusepath{clip}%
\pgfsetrectcap%
\pgfsetroundjoin%
\pgfsetlinewidth{1.003750pt}%
\definecolor{currentstroke}{rgb}{0.000000,0.000000,0.000000}%
\pgfsetstrokecolor{currentstroke}%
\pgfsetdash{}{0pt}%
\pgfpathmoveto{\pgfqpoint{0.718616in}{1.695214in}}%
\pgfpathlineto{\pgfqpoint{0.724977in}{1.611782in}}%
\pgfpathlineto{\pgfqpoint{0.743904in}{1.530404in}}%
\pgfpathlineto{\pgfqpoint{0.774929in}{1.453085in}}%
\pgfpathlineto{\pgfqpoint{0.817291in}{1.381728in}}%
\pgfpathlineto{\pgfqpoint{0.869944in}{1.318090in}}%
\pgfpathlineto{\pgfqpoint{0.931594in}{1.263738in}}%
\pgfpathlineto{\pgfqpoint{1.000721in}{1.220010in}}%
\pgfpathlineto{\pgfqpoint{1.075624in}{1.187983in}}%
\pgfpathlineto{\pgfqpoint{1.154458in}{1.168446in}}%
\pgfpathlineto{\pgfqpoint{1.235283in}{1.161880in}}%
\pgfpathlineto{\pgfqpoint{1.316107in}{1.168446in}}%
\pgfpathlineto{\pgfqpoint{1.394942in}{1.187983in}}%
\pgfpathlineto{\pgfqpoint{1.469845in}{1.220010in}}%
\pgfpathlineto{\pgfqpoint{1.538972in}{1.263738in}}%
\pgfpathlineto{\pgfqpoint{1.600621in}{1.318090in}}%
\pgfpathlineto{\pgfqpoint{1.653275in}{1.381728in}}%
\pgfpathlineto{\pgfqpoint{1.695636in}{1.453085in}}%
\pgfpathlineto{\pgfqpoint{1.726662in}{1.530404in}}%
\pgfpathlineto{\pgfqpoint{1.745588in}{1.611782in}}%
\pgfpathlineto{\pgfqpoint{1.751949in}{1.695214in}}%
\pgfusepath{stroke}%
\end{pgfscope}%
\begin{pgfscope}%
\pgfpathrectangle{\pgfqpoint{0.375000in}{0.300000in}}{\pgfqpoint{2.325000in}{2.400000in}} %
\pgfusepath{clip}%
\pgfsetbuttcap%
\pgfsetroundjoin%
\definecolor{currentfill}{rgb}{0.000000,0.000000,1.000000}%
\pgfsetfillcolor{currentfill}%
\pgfsetlinewidth{0.501875pt}%
\definecolor{currentstroke}{rgb}{0.000000,0.000000,0.000000}%
\pgfsetstrokecolor{currentstroke}%
\pgfsetdash{}{0pt}%
\pgfsys@defobject{currentmarker}{\pgfqpoint{-0.041667in}{-0.041667in}}{\pgfqpoint{0.041667in}{0.041667in}}{%
\pgfpathmoveto{\pgfqpoint{0.000000in}{-0.041667in}}%
\pgfpathcurveto{\pgfqpoint{0.011050in}{-0.041667in}}{\pgfqpoint{0.021649in}{-0.037276in}}{\pgfqpoint{0.029463in}{-0.029463in}}%
\pgfpathcurveto{\pgfqpoint{0.037276in}{-0.021649in}}{\pgfqpoint{0.041667in}{-0.011050in}}{\pgfqpoint{0.041667in}{0.000000in}}%
\pgfpathcurveto{\pgfqpoint{0.041667in}{0.011050in}}{\pgfqpoint{0.037276in}{0.021649in}}{\pgfqpoint{0.029463in}{0.029463in}}%
\pgfpathcurveto{\pgfqpoint{0.021649in}{0.037276in}}{\pgfqpoint{0.011050in}{0.041667in}}{\pgfqpoint{0.000000in}{0.041667in}}%
\pgfpathcurveto{\pgfqpoint{-0.011050in}{0.041667in}}{\pgfqpoint{-0.021649in}{0.037276in}}{\pgfqpoint{-0.029463in}{0.029463in}}%
\pgfpathcurveto{\pgfqpoint{-0.037276in}{0.021649in}}{\pgfqpoint{-0.041667in}{0.011050in}}{\pgfqpoint{-0.041667in}{0.000000in}}%
\pgfpathcurveto{\pgfqpoint{-0.041667in}{-0.011050in}}{\pgfqpoint{-0.037276in}{-0.021649in}}{\pgfqpoint{-0.029463in}{-0.029463in}}%
\pgfpathcurveto{\pgfqpoint{-0.021649in}{-0.037276in}}{\pgfqpoint{-0.011050in}{-0.041667in}}{\pgfqpoint{0.000000in}{-0.041667in}}%
\pgfpathclose%
\pgfusepath{stroke,fill}%
}%
\begin{pgfscope}%
\pgfsys@transformshift{1.235283in}{1.695214in}%
\pgfsys@useobject{currentmarker}{}%
\end{pgfscope}%
\end{pgfscope}%
\begin{pgfscope}%
\pgfpathrectangle{\pgfqpoint{0.375000in}{0.300000in}}{\pgfqpoint{2.325000in}{2.400000in}} %
\pgfusepath{clip}%
\pgfsetbuttcap%
\pgfsetroundjoin%
\definecolor{currentfill}{rgb}{0.000000,0.500000,0.000000}%
\pgfsetfillcolor{currentfill}%
\pgfsetlinewidth{0.501875pt}%
\definecolor{currentstroke}{rgb}{0.000000,0.000000,0.000000}%
\pgfsetstrokecolor{currentstroke}%
\pgfsetdash{}{0pt}%
\pgfsys@defobject{currentmarker}{\pgfqpoint{-0.041667in}{-0.041667in}}{\pgfqpoint{0.041667in}{0.041667in}}{%
\pgfpathmoveto{\pgfqpoint{0.000000in}{-0.041667in}}%
\pgfpathcurveto{\pgfqpoint{0.011050in}{-0.041667in}}{\pgfqpoint{0.021649in}{-0.037276in}}{\pgfqpoint{0.029463in}{-0.029463in}}%
\pgfpathcurveto{\pgfqpoint{0.037276in}{-0.021649in}}{\pgfqpoint{0.041667in}{-0.011050in}}{\pgfqpoint{0.041667in}{0.000000in}}%
\pgfpathcurveto{\pgfqpoint{0.041667in}{0.011050in}}{\pgfqpoint{0.037276in}{0.021649in}}{\pgfqpoint{0.029463in}{0.029463in}}%
\pgfpathcurveto{\pgfqpoint{0.021649in}{0.037276in}}{\pgfqpoint{0.011050in}{0.041667in}}{\pgfqpoint{0.000000in}{0.041667in}}%
\pgfpathcurveto{\pgfqpoint{-0.011050in}{0.041667in}}{\pgfqpoint{-0.021649in}{0.037276in}}{\pgfqpoint{-0.029463in}{0.029463in}}%
\pgfpathcurveto{\pgfqpoint{-0.037276in}{0.021649in}}{\pgfqpoint{-0.041667in}{0.011050in}}{\pgfqpoint{-0.041667in}{0.000000in}}%
\pgfpathcurveto{\pgfqpoint{-0.041667in}{-0.011050in}}{\pgfqpoint{-0.037276in}{-0.021649in}}{\pgfqpoint{-0.029463in}{-0.029463in}}%
\pgfpathcurveto{\pgfqpoint{-0.021649in}{-0.037276in}}{\pgfqpoint{-0.011050in}{-0.041667in}}{\pgfqpoint{0.000000in}{-0.041667in}}%
\pgfpathclose%
\pgfusepath{stroke,fill}%
}%
\begin{pgfscope}%
\pgfsys@transformshift{1.493616in}{1.233333in}%
\pgfsys@useobject{currentmarker}{}%
\end{pgfscope}%
\end{pgfscope}%
\begin{pgfscope}%
\pgfpathrectangle{\pgfqpoint{0.375000in}{0.300000in}}{\pgfqpoint{2.325000in}{2.400000in}} %
\pgfusepath{clip}%
\pgfsetbuttcap%
\pgfsetroundjoin%
\definecolor{currentfill}{rgb}{1.000000,0.000000,0.000000}%
\pgfsetfillcolor{currentfill}%
\pgfsetlinewidth{0.501875pt}%
\definecolor{currentstroke}{rgb}{0.000000,0.000000,0.000000}%
\pgfsetstrokecolor{currentstroke}%
\pgfsetdash{}{0pt}%
\pgfsys@defobject{currentmarker}{\pgfqpoint{-0.041667in}{-0.041667in}}{\pgfqpoint{0.041667in}{0.041667in}}{%
\pgfpathmoveto{\pgfqpoint{0.000000in}{-0.041667in}}%
\pgfpathcurveto{\pgfqpoint{0.011050in}{-0.041667in}}{\pgfqpoint{0.021649in}{-0.037276in}}{\pgfqpoint{0.029463in}{-0.029463in}}%
\pgfpathcurveto{\pgfqpoint{0.037276in}{-0.021649in}}{\pgfqpoint{0.041667in}{-0.011050in}}{\pgfqpoint{0.041667in}{0.000000in}}%
\pgfpathcurveto{\pgfqpoint{0.041667in}{0.011050in}}{\pgfqpoint{0.037276in}{0.021649in}}{\pgfqpoint{0.029463in}{0.029463in}}%
\pgfpathcurveto{\pgfqpoint{0.021649in}{0.037276in}}{\pgfqpoint{0.011050in}{0.041667in}}{\pgfqpoint{0.000000in}{0.041667in}}%
\pgfpathcurveto{\pgfqpoint{-0.011050in}{0.041667in}}{\pgfqpoint{-0.021649in}{0.037276in}}{\pgfqpoint{-0.029463in}{0.029463in}}%
\pgfpathcurveto{\pgfqpoint{-0.037276in}{0.021649in}}{\pgfqpoint{-0.041667in}{0.011050in}}{\pgfqpoint{-0.041667in}{0.000000in}}%
\pgfpathcurveto{\pgfqpoint{-0.041667in}{-0.011050in}}{\pgfqpoint{-0.037276in}{-0.021649in}}{\pgfqpoint{-0.029463in}{-0.029463in}}%
\pgfpathcurveto{\pgfqpoint{-0.021649in}{-0.037276in}}{\pgfqpoint{-0.011050in}{-0.041667in}}{\pgfqpoint{0.000000in}{-0.041667in}}%
\pgfpathclose%
\pgfusepath{stroke,fill}%
}%
\begin{pgfscope}%
\pgfsys@transformshift{0.976949in}{2.157094in}%
\pgfsys@useobject{currentmarker}{}%
\end{pgfscope}%
\end{pgfscope}%
\begin{pgfscope}%
\pgftext[x=0.633333in,y=0.566667in,left,bottom]{{\sffamily\fontsize{14.000000}{16.800000}\selectfont \(\displaystyle \theta\)}}%
\end{pgfscope}%
\begin{pgfscope}%
\pgftext[x=1.235283in,y=1.695214in,right,bottom]{{\sffamily\fontsize{14.000000}{16.800000}\selectfont \(\displaystyle CM\)}}%
\end{pgfscope}%
\begin{pgfscope}%
\pgftext[x=1.235283in,y=0.942849in,left,top]{{\sffamily\fontsize{14.000000}{16.800000}\selectfont \(\displaystyle mg\)}}%
\end{pgfscope}%
\begin{pgfscope}%
\pgftext[x=2.242037in,y=2.428547in,left,bottom]{{\sffamily\fontsize{14.000000}{16.800000}\selectfont \(\displaystyle a\)}}%
\end{pgfscope}%
\begin{pgfscope}%
\pgftext[x=1.648119in,y=2.557094in,left,top]{{\sffamily\fontsize{14.000000}{16.800000}\selectfont \(\displaystyle F_{p}\)}}%
\end{pgfscope}%
\end{pgfpicture}%
\makeatother%
\endgroup%


\bigskip

Friction can go up or down the hill, depending on the condition. Similar to
case 5.

%% Creator: Matplotlib, PGF backend
%%
%% To include the figure in your LaTeX document, write
%%   \input{<filename>.pgf}
%%
%% Make sure the required packages are loaded in your preamble
%%   \usepackage{pgf}
%%
%% Figures using additional raster images can only be included by \input if
%% they are in the same directory as the main LaTeX file. For loading figures
%% from other directories you can use the `import` package
%%   \usepackage{import}
%% and then include the figures with
%%   \import{<path to file>}{<filename>.pgf}
%%
%% Matplotlib used the following preamble
%%   \usepackage{fontspec}
%%   \setmainfont{Times New Roman}
%%   \setsansfont{Verdana}
%%   \setmonofont{Courier New}
%%
\begingroup%
\makeatletter%
\begin{pgfpicture}%
\pgfpathrectangle{\pgfpointorigin}{\pgfqpoint{3.000000in}{3.000000in}}%
\pgfusepath{use as bounding box}%
\begin{pgfscope}%
\pgfsetbuttcap%
\pgfsetroundjoin%
\definecolor{currentfill}{rgb}{1.000000,1.000000,1.000000}%
\pgfsetfillcolor{currentfill}%
\pgfsetlinewidth{0.000000pt}%
\definecolor{currentstroke}{rgb}{1.000000,1.000000,1.000000}%
\pgfsetstrokecolor{currentstroke}%
\pgfsetdash{}{0pt}%
\pgfpathmoveto{\pgfqpoint{0.000000in}{0.000000in}}%
\pgfpathlineto{\pgfqpoint{3.000000in}{0.000000in}}%
\pgfpathlineto{\pgfqpoint{3.000000in}{3.000000in}}%
\pgfpathlineto{\pgfqpoint{0.000000in}{3.000000in}}%
\pgfpathclose%
\pgfusepath{fill}%
\end{pgfscope}%
\begin{pgfscope}%
\pgfpathrectangle{\pgfqpoint{0.375000in}{0.300000in}}{\pgfqpoint{2.325000in}{2.400000in}} %
\pgfusepath{clip}%
\pgfsetbuttcap%
\pgfsetroundjoin%
\definecolor{currentfill}{rgb}{0.000000,0.000000,1.000000}%
\pgfsetfillcolor{currentfill}%
\pgfsetlinewidth{1.003750pt}%
\definecolor{currentstroke}{rgb}{0.000000,0.000000,0.000000}%
\pgfsetstrokecolor{currentstroke}%
\pgfsetdash{}{0pt}%
\pgfpathmoveto{\pgfqpoint{2.265461in}{1.693333in}}%
\pgfpathlineto{\pgfqpoint{2.190619in}{1.587145in}}%
\pgfpathlineto{\pgfqpoint{2.170598in}{1.622941in}}%
\pgfpathlineto{\pgfqpoint{1.499429in}{1.222941in}}%
\pgfpathlineto{\pgfqpoint{1.487804in}{1.243726in}}%
\pgfpathlineto{\pgfqpoint{2.158973in}{1.643726in}}%
\pgfpathlineto{\pgfqpoint{2.138953in}{1.679521in}}%
\pgfpathclose%
\pgfusepath{stroke,fill}%
\end{pgfscope}%
\begin{pgfscope}%
\pgfpathrectangle{\pgfqpoint{0.375000in}{0.300000in}}{\pgfqpoint{2.325000in}{2.400000in}} %
\pgfusepath{clip}%
\pgfsetbuttcap%
\pgfsetroundjoin%
\definecolor{currentfill}{rgb}{0.000000,0.000000,1.000000}%
\pgfsetfillcolor{currentfill}%
\pgfsetlinewidth{1.003750pt}%
\definecolor{currentstroke}{rgb}{0.000000,0.000000,0.000000}%
\pgfsetstrokecolor{currentstroke}%
\pgfsetdash{}{0pt}%
\pgfpathmoveto{\pgfqpoint{1.235283in}{0.822849in}}%
\pgfpathlineto{\pgfqpoint{1.183616in}{0.942849in}}%
\pgfpathlineto{\pgfqpoint{1.223658in}{0.942849in}}%
\pgfpathlineto{\pgfqpoint{1.223658in}{1.695214in}}%
\pgfpathlineto{\pgfqpoint{1.246908in}{1.695214in}}%
\pgfpathlineto{\pgfqpoint{1.246908in}{0.942849in}}%
\pgfpathlineto{\pgfqpoint{1.286949in}{0.942849in}}%
\pgfpathclose%
\pgfusepath{stroke,fill}%
\end{pgfscope}%
\begin{pgfscope}%
\pgfpathrectangle{\pgfqpoint{0.375000in}{0.300000in}}{\pgfqpoint{2.325000in}{2.400000in}} %
\pgfusepath{clip}%
\pgfsetbuttcap%
\pgfsetroundjoin%
\definecolor{currentfill}{rgb}{1.000000,0.000000,0.000000}%
\pgfsetfillcolor{currentfill}%
\pgfsetlinewidth{1.003750pt}%
\definecolor{currentstroke}{rgb}{1.000000,0.000000,0.000000}%
\pgfsetstrokecolor{currentstroke}%
\pgfsetdash{}{0pt}%
\pgfpathmoveto{\pgfqpoint{2.376271in}{2.375214in}}%
\pgfpathlineto{\pgfqpoint{2.306621in}{2.179743in}}%
\pgfpathlineto{\pgfqpoint{2.254954in}{2.272120in}}%
\pgfpathlineto{\pgfqpoint{1.919369in}{2.072120in}}%
\pgfpathlineto{\pgfqpoint{1.893536in}{2.118308in}}%
\pgfpathlineto{\pgfqpoint{2.229121in}{2.318308in}}%
\pgfpathlineto{\pgfqpoint{2.177454in}{2.410684in}}%
\pgfpathclose%
\pgfusepath{stroke,fill}%
\end{pgfscope}%
\begin{pgfscope}%
\pgfpathrectangle{\pgfqpoint{0.375000in}{0.300000in}}{\pgfqpoint{2.325000in}{2.400000in}} %
\pgfusepath{clip}%
\pgfsetbuttcap%
\pgfsetroundjoin%
\definecolor{currentfill}{rgb}{0.000000,0.000000,1.000000}%
\pgfsetfillcolor{currentfill}%
\pgfsetlinewidth{1.003750pt}%
\definecolor{currentstroke}{rgb}{0.000000,0.000000,0.000000}%
\pgfsetstrokecolor{currentstroke}%
\pgfsetdash{}{0pt}%
\pgfpathmoveto{\pgfqpoint{1.748795in}{2.617094in}}%
\pgfpathlineto{\pgfqpoint{1.673953in}{2.510906in}}%
\pgfpathlineto{\pgfqpoint{1.653932in}{2.546701in}}%
\pgfpathlineto{\pgfqpoint{0.982762in}{2.146701in}}%
\pgfpathlineto{\pgfqpoint{0.971137in}{2.167486in}}%
\pgfpathlineto{\pgfqpoint{1.642307in}{2.567486in}}%
\pgfpathlineto{\pgfqpoint{1.622286in}{2.603282in}}%
\pgfpathclose%
\pgfusepath{stroke,fill}%
\end{pgfscope}%
\begin{pgfscope}%
\pgfpathrectangle{\pgfqpoint{0.375000in}{0.300000in}}{\pgfqpoint{2.325000in}{2.400000in}} %
\pgfusepath{clip}%
\pgfsetbuttcap%
\pgfsetroundjoin%
\definecolor{currentfill}{rgb}{0.000000,0.000000,1.000000}%
\pgfsetfillcolor{currentfill}%
\pgfsetlinewidth{1.003750pt}%
\definecolor{currentstroke}{rgb}{0.000000,0.000000,0.000000}%
\pgfsetstrokecolor{currentstroke}%
\pgfsetdash{}{0pt}%
\pgfpathmoveto{\pgfqpoint{0.721771in}{0.773333in}}%
\pgfpathlineto{\pgfqpoint{0.796613in}{0.879521in}}%
\pgfpathlineto{\pgfqpoint{0.816634in}{0.843726in}}%
\pgfpathlineto{\pgfqpoint{1.487804in}{1.243726in}}%
\pgfpathlineto{\pgfqpoint{1.499429in}{1.222941in}}%
\pgfpathlineto{\pgfqpoint{0.828259in}{0.822941in}}%
\pgfpathlineto{\pgfqpoint{0.848280in}{0.787145in}}%
\pgfpathclose%
\pgfusepath{stroke,fill}%
\end{pgfscope}%
\begin{pgfscope}%
\pgfpathrectangle{\pgfqpoint{0.375000in}{0.300000in}}{\pgfqpoint{2.325000in}{2.400000in}} %
\pgfusepath{clip}%
\pgfsetrectcap%
\pgfsetroundjoin%
\pgfsetlinewidth{1.003750pt}%
\definecolor{currentstroke}{rgb}{0.000000,0.000000,1.000000}%
\pgfsetstrokecolor{currentstroke}%
\pgfsetdash{}{0pt}%
\pgfpathmoveto{\pgfqpoint{0.375000in}{0.566667in}}%
\pgfpathlineto{\pgfqpoint{2.612232in}{0.566667in}}%
\pgfusepath{stroke}%
\end{pgfscope}%
\begin{pgfscope}%
\pgfpathrectangle{\pgfqpoint{0.375000in}{0.300000in}}{\pgfqpoint{2.325000in}{2.400000in}} %
\pgfusepath{clip}%
\pgfsetrectcap%
\pgfsetroundjoin%
\pgfsetlinewidth{1.003750pt}%
\definecolor{currentstroke}{rgb}{0.000000,0.000000,1.000000}%
\pgfsetstrokecolor{currentstroke}%
\pgfsetdash{}{0pt}%
\pgfpathmoveto{\pgfqpoint{2.612232in}{0.566667in}}%
\pgfpathlineto{\pgfqpoint{2.612232in}{1.900000in}}%
\pgfusepath{stroke}%
\end{pgfscope}%
\begin{pgfscope}%
\pgfpathrectangle{\pgfqpoint{0.375000in}{0.300000in}}{\pgfqpoint{2.325000in}{2.400000in}} %
\pgfusepath{clip}%
\pgfsetrectcap%
\pgfsetroundjoin%
\pgfsetlinewidth{1.003750pt}%
\definecolor{currentstroke}{rgb}{0.000000,0.000000,1.000000}%
\pgfsetstrokecolor{currentstroke}%
\pgfsetdash{}{0pt}%
\pgfpathmoveto{\pgfqpoint{0.375000in}{0.566667in}}%
\pgfpathlineto{\pgfqpoint{2.612232in}{1.900000in}}%
\pgfusepath{stroke}%
\end{pgfscope}%
\begin{pgfscope}%
\pgfpathrectangle{\pgfqpoint{0.375000in}{0.300000in}}{\pgfqpoint{2.325000in}{2.400000in}} %
\pgfusepath{clip}%
\pgfsetrectcap%
\pgfsetroundjoin%
\pgfsetlinewidth{1.003750pt}%
\definecolor{currentstroke}{rgb}{0.000000,0.000000,0.000000}%
\pgfsetstrokecolor{currentstroke}%
\pgfsetdash{}{0pt}%
\pgfpathmoveto{\pgfqpoint{0.504167in}{0.566667in}}%
\pgfpathlineto{\pgfqpoint{0.503990in}{0.573645in}}%
\pgfpathlineto{\pgfqpoint{0.503459in}{0.580604in}}%
\pgfpathlineto{\pgfqpoint{0.502576in}{0.587525in}}%
\pgfpathlineto{\pgfqpoint{0.501344in}{0.594388in}}%
\pgfpathlineto{\pgfqpoint{0.499765in}{0.601176in}}%
\pgfpathlineto{\pgfqpoint{0.497845in}{0.607869in}}%
\pgfpathlineto{\pgfqpoint{0.495587in}{0.614449in}}%
\pgfpathlineto{\pgfqpoint{0.493000in}{0.620898in}}%
\pgfpathlineto{\pgfqpoint{0.490088in}{0.627199in}}%
\pgfpathlineto{\pgfqpoint{0.486862in}{0.633333in}}%
\pgfusepath{stroke}%
\end{pgfscope}%
\begin{pgfscope}%
\pgfpathrectangle{\pgfqpoint{0.375000in}{0.300000in}}{\pgfqpoint{2.325000in}{2.400000in}} %
\pgfusepath{clip}%
\pgfsetrectcap%
\pgfsetroundjoin%
\pgfsetlinewidth{1.003750pt}%
\definecolor{currentstroke}{rgb}{0.000000,0.000000,0.000000}%
\pgfsetstrokecolor{currentstroke}%
\pgfsetdash{}{0pt}%
\pgfpathmoveto{\pgfqpoint{1.751949in}{1.695214in}}%
\pgfpathlineto{\pgfqpoint{1.745588in}{1.778645in}}%
\pgfpathlineto{\pgfqpoint{1.726662in}{1.860023in}}%
\pgfpathlineto{\pgfqpoint{1.695636in}{1.937342in}}%
\pgfpathlineto{\pgfqpoint{1.653275in}{2.008699in}}%
\pgfpathlineto{\pgfqpoint{1.600621in}{2.072337in}}%
\pgfpathlineto{\pgfqpoint{1.538972in}{2.126689in}}%
\pgfpathlineto{\pgfqpoint{1.469845in}{2.170417in}}%
\pgfpathlineto{\pgfqpoint{1.394942in}{2.202444in}}%
\pgfpathlineto{\pgfqpoint{1.316107in}{2.221981in}}%
\pgfpathlineto{\pgfqpoint{1.235283in}{2.228547in}}%
\pgfpathlineto{\pgfqpoint{1.154458in}{2.221981in}}%
\pgfpathlineto{\pgfqpoint{1.075624in}{2.202444in}}%
\pgfpathlineto{\pgfqpoint{1.000721in}{2.170417in}}%
\pgfpathlineto{\pgfqpoint{0.931594in}{2.126689in}}%
\pgfpathlineto{\pgfqpoint{0.869944in}{2.072337in}}%
\pgfpathlineto{\pgfqpoint{0.817291in}{2.008699in}}%
\pgfpathlineto{\pgfqpoint{0.774929in}{1.937342in}}%
\pgfpathlineto{\pgfqpoint{0.743904in}{1.860023in}}%
\pgfpathlineto{\pgfqpoint{0.724977in}{1.778645in}}%
\pgfpathlineto{\pgfqpoint{0.718616in}{1.695214in}}%
\pgfusepath{stroke}%
\end{pgfscope}%
\begin{pgfscope}%
\pgfpathrectangle{\pgfqpoint{0.375000in}{0.300000in}}{\pgfqpoint{2.325000in}{2.400000in}} %
\pgfusepath{clip}%
\pgfsetrectcap%
\pgfsetroundjoin%
\pgfsetlinewidth{1.003750pt}%
\definecolor{currentstroke}{rgb}{0.000000,0.000000,0.000000}%
\pgfsetstrokecolor{currentstroke}%
\pgfsetdash{}{0pt}%
\pgfpathmoveto{\pgfqpoint{0.718616in}{1.695214in}}%
\pgfpathlineto{\pgfqpoint{0.724977in}{1.611782in}}%
\pgfpathlineto{\pgfqpoint{0.743904in}{1.530404in}}%
\pgfpathlineto{\pgfqpoint{0.774929in}{1.453085in}}%
\pgfpathlineto{\pgfqpoint{0.817291in}{1.381728in}}%
\pgfpathlineto{\pgfqpoint{0.869944in}{1.318090in}}%
\pgfpathlineto{\pgfqpoint{0.931594in}{1.263738in}}%
\pgfpathlineto{\pgfqpoint{1.000721in}{1.220010in}}%
\pgfpathlineto{\pgfqpoint{1.075624in}{1.187983in}}%
\pgfpathlineto{\pgfqpoint{1.154458in}{1.168446in}}%
\pgfpathlineto{\pgfqpoint{1.235283in}{1.161880in}}%
\pgfpathlineto{\pgfqpoint{1.316107in}{1.168446in}}%
\pgfpathlineto{\pgfqpoint{1.394942in}{1.187983in}}%
\pgfpathlineto{\pgfqpoint{1.469845in}{1.220010in}}%
\pgfpathlineto{\pgfqpoint{1.538972in}{1.263738in}}%
\pgfpathlineto{\pgfqpoint{1.600621in}{1.318090in}}%
\pgfpathlineto{\pgfqpoint{1.653275in}{1.381728in}}%
\pgfpathlineto{\pgfqpoint{1.695636in}{1.453085in}}%
\pgfpathlineto{\pgfqpoint{1.726662in}{1.530404in}}%
\pgfpathlineto{\pgfqpoint{1.745588in}{1.611782in}}%
\pgfpathlineto{\pgfqpoint{1.751949in}{1.695214in}}%
\pgfusepath{stroke}%
\end{pgfscope}%
\begin{pgfscope}%
\pgfpathrectangle{\pgfqpoint{0.375000in}{0.300000in}}{\pgfqpoint{2.325000in}{2.400000in}} %
\pgfusepath{clip}%
\pgfsetbuttcap%
\pgfsetroundjoin%
\definecolor{currentfill}{rgb}{0.000000,0.000000,1.000000}%
\pgfsetfillcolor{currentfill}%
\pgfsetlinewidth{0.501875pt}%
\definecolor{currentstroke}{rgb}{0.000000,0.000000,0.000000}%
\pgfsetstrokecolor{currentstroke}%
\pgfsetdash{}{0pt}%
\pgfsys@defobject{currentmarker}{\pgfqpoint{-0.041667in}{-0.041667in}}{\pgfqpoint{0.041667in}{0.041667in}}{%
\pgfpathmoveto{\pgfqpoint{0.000000in}{-0.041667in}}%
\pgfpathcurveto{\pgfqpoint{0.011050in}{-0.041667in}}{\pgfqpoint{0.021649in}{-0.037276in}}{\pgfqpoint{0.029463in}{-0.029463in}}%
\pgfpathcurveto{\pgfqpoint{0.037276in}{-0.021649in}}{\pgfqpoint{0.041667in}{-0.011050in}}{\pgfqpoint{0.041667in}{0.000000in}}%
\pgfpathcurveto{\pgfqpoint{0.041667in}{0.011050in}}{\pgfqpoint{0.037276in}{0.021649in}}{\pgfqpoint{0.029463in}{0.029463in}}%
\pgfpathcurveto{\pgfqpoint{0.021649in}{0.037276in}}{\pgfqpoint{0.011050in}{0.041667in}}{\pgfqpoint{0.000000in}{0.041667in}}%
\pgfpathcurveto{\pgfqpoint{-0.011050in}{0.041667in}}{\pgfqpoint{-0.021649in}{0.037276in}}{\pgfqpoint{-0.029463in}{0.029463in}}%
\pgfpathcurveto{\pgfqpoint{-0.037276in}{0.021649in}}{\pgfqpoint{-0.041667in}{0.011050in}}{\pgfqpoint{-0.041667in}{0.000000in}}%
\pgfpathcurveto{\pgfqpoint{-0.041667in}{-0.011050in}}{\pgfqpoint{-0.037276in}{-0.021649in}}{\pgfqpoint{-0.029463in}{-0.029463in}}%
\pgfpathcurveto{\pgfqpoint{-0.021649in}{-0.037276in}}{\pgfqpoint{-0.011050in}{-0.041667in}}{\pgfqpoint{0.000000in}{-0.041667in}}%
\pgfpathclose%
\pgfusepath{stroke,fill}%
}%
\begin{pgfscope}%
\pgfsys@transformshift{1.235283in}{1.695214in}%
\pgfsys@useobject{currentmarker}{}%
\end{pgfscope}%
\end{pgfscope}%
\begin{pgfscope}%
\pgfpathrectangle{\pgfqpoint{0.375000in}{0.300000in}}{\pgfqpoint{2.325000in}{2.400000in}} %
\pgfusepath{clip}%
\pgfsetbuttcap%
\pgfsetroundjoin%
\definecolor{currentfill}{rgb}{0.000000,0.500000,0.000000}%
\pgfsetfillcolor{currentfill}%
\pgfsetlinewidth{0.501875pt}%
\definecolor{currentstroke}{rgb}{0.000000,0.000000,0.000000}%
\pgfsetstrokecolor{currentstroke}%
\pgfsetdash{}{0pt}%
\pgfsys@defobject{currentmarker}{\pgfqpoint{-0.041667in}{-0.041667in}}{\pgfqpoint{0.041667in}{0.041667in}}{%
\pgfpathmoveto{\pgfqpoint{0.000000in}{-0.041667in}}%
\pgfpathcurveto{\pgfqpoint{0.011050in}{-0.041667in}}{\pgfqpoint{0.021649in}{-0.037276in}}{\pgfqpoint{0.029463in}{-0.029463in}}%
\pgfpathcurveto{\pgfqpoint{0.037276in}{-0.021649in}}{\pgfqpoint{0.041667in}{-0.011050in}}{\pgfqpoint{0.041667in}{0.000000in}}%
\pgfpathcurveto{\pgfqpoint{0.041667in}{0.011050in}}{\pgfqpoint{0.037276in}{0.021649in}}{\pgfqpoint{0.029463in}{0.029463in}}%
\pgfpathcurveto{\pgfqpoint{0.021649in}{0.037276in}}{\pgfqpoint{0.011050in}{0.041667in}}{\pgfqpoint{0.000000in}{0.041667in}}%
\pgfpathcurveto{\pgfqpoint{-0.011050in}{0.041667in}}{\pgfqpoint{-0.021649in}{0.037276in}}{\pgfqpoint{-0.029463in}{0.029463in}}%
\pgfpathcurveto{\pgfqpoint{-0.037276in}{0.021649in}}{\pgfqpoint{-0.041667in}{0.011050in}}{\pgfqpoint{-0.041667in}{0.000000in}}%
\pgfpathcurveto{\pgfqpoint{-0.041667in}{-0.011050in}}{\pgfqpoint{-0.037276in}{-0.021649in}}{\pgfqpoint{-0.029463in}{-0.029463in}}%
\pgfpathcurveto{\pgfqpoint{-0.021649in}{-0.037276in}}{\pgfqpoint{-0.011050in}{-0.041667in}}{\pgfqpoint{0.000000in}{-0.041667in}}%
\pgfpathclose%
\pgfusepath{stroke,fill}%
}%
\begin{pgfscope}%
\pgfsys@transformshift{1.493616in}{1.233333in}%
\pgfsys@useobject{currentmarker}{}%
\end{pgfscope}%
\end{pgfscope}%
\begin{pgfscope}%
\pgfpathrectangle{\pgfqpoint{0.375000in}{0.300000in}}{\pgfqpoint{2.325000in}{2.400000in}} %
\pgfusepath{clip}%
\pgfsetbuttcap%
\pgfsetroundjoin%
\definecolor{currentfill}{rgb}{1.000000,0.000000,0.000000}%
\pgfsetfillcolor{currentfill}%
\pgfsetlinewidth{0.501875pt}%
\definecolor{currentstroke}{rgb}{0.000000,0.000000,0.000000}%
\pgfsetstrokecolor{currentstroke}%
\pgfsetdash{}{0pt}%
\pgfsys@defobject{currentmarker}{\pgfqpoint{-0.041667in}{-0.041667in}}{\pgfqpoint{0.041667in}{0.041667in}}{%
\pgfpathmoveto{\pgfqpoint{0.000000in}{-0.041667in}}%
\pgfpathcurveto{\pgfqpoint{0.011050in}{-0.041667in}}{\pgfqpoint{0.021649in}{-0.037276in}}{\pgfqpoint{0.029463in}{-0.029463in}}%
\pgfpathcurveto{\pgfqpoint{0.037276in}{-0.021649in}}{\pgfqpoint{0.041667in}{-0.011050in}}{\pgfqpoint{0.041667in}{0.000000in}}%
\pgfpathcurveto{\pgfqpoint{0.041667in}{0.011050in}}{\pgfqpoint{0.037276in}{0.021649in}}{\pgfqpoint{0.029463in}{0.029463in}}%
\pgfpathcurveto{\pgfqpoint{0.021649in}{0.037276in}}{\pgfqpoint{0.011050in}{0.041667in}}{\pgfqpoint{0.000000in}{0.041667in}}%
\pgfpathcurveto{\pgfqpoint{-0.011050in}{0.041667in}}{\pgfqpoint{-0.021649in}{0.037276in}}{\pgfqpoint{-0.029463in}{0.029463in}}%
\pgfpathcurveto{\pgfqpoint{-0.037276in}{0.021649in}}{\pgfqpoint{-0.041667in}{0.011050in}}{\pgfqpoint{-0.041667in}{0.000000in}}%
\pgfpathcurveto{\pgfqpoint{-0.041667in}{-0.011050in}}{\pgfqpoint{-0.037276in}{-0.021649in}}{\pgfqpoint{-0.029463in}{-0.029463in}}%
\pgfpathcurveto{\pgfqpoint{-0.021649in}{-0.037276in}}{\pgfqpoint{-0.011050in}{-0.041667in}}{\pgfqpoint{0.000000in}{-0.041667in}}%
\pgfpathclose%
\pgfusepath{stroke,fill}%
}%
\begin{pgfscope}%
\pgfsys@transformshift{0.976949in}{2.157094in}%
\pgfsys@useobject{currentmarker}{}%
\end{pgfscope}%
\end{pgfscope}%
\begin{pgfscope}%
\pgftext[x=0.633333in,y=0.566667in,left,bottom]{{\sffamily\fontsize{14.000000}{16.800000}\selectfont \(\displaystyle \theta\)}}%
\end{pgfscope}%
\begin{pgfscope}%
\pgftext[x=1.235283in,y=1.695214in,right,bottom]{{\sffamily\fontsize{14.000000}{16.800000}\selectfont \(\displaystyle CM\)}}%
\end{pgfscope}%
\begin{pgfscope}%
\pgftext[x=2.190619in,y=1.633333in,left,top]{{\sffamily\fontsize{14.000000}{16.800000}\selectfont \(\displaystyle F_{fr}\)}}%
\end{pgfscope}%
\begin{pgfscope}%
\pgftext[x=1.235283in,y=0.942849in,left,top]{{\sffamily\fontsize{14.000000}{16.800000}\selectfont \(\displaystyle mg\)}}%
\end{pgfscope}%
\begin{pgfscope}%
\pgftext[x=2.242037in,y=2.428547in,left,bottom]{{\sffamily\fontsize{14.000000}{16.800000}\selectfont \(\displaystyle a\)}}%
\end{pgfscope}%
\begin{pgfscope}%
\pgftext[x=1.648119in,y=2.557094in,left,top]{{\sffamily\fontsize{14.000000}{16.800000}\selectfont \(\displaystyle F_{p}\)}}%
\end{pgfscope}%
\begin{pgfscope}%
\pgftext[x=0.796613in,y=0.833333in,right,bottom]{{\sffamily\fontsize{14.000000}{16.800000}\selectfont \(\displaystyle F_{fr}\)}}%
\end{pgfscope}%
\end{pgfpicture}%
\makeatother%
\endgroup%


\end{document}
