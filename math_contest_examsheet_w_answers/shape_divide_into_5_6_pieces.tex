\documentclass{article}%[12pt,twoside]
%\usepackage[inner=1in,outer=0.6in,top=0.7in,bottom=1in]{geometry}
\usepackage{xeCJK}
\setmainfont{Times New Roman}
\setsansfont{Verdana}
\setmonofont{Courier New}                    % tt
\setCJKmainfont{微軟正黑體}
\setCJKfamilyfont{kai}{標楷體}		% for changing the title font in title.pgf -> have to manually 
\usepackage{pstricks}
\usepackage{eso-pic}
\usepackage{graphicx}


\begin{document}

\AddToShipoutPictureBG{  \AtTextLowerLeft{  \resizebox{\textwidth}{!}{\rotatebox{60}{{\color{cyan} whymranderson 製版}}}
   }}%


2007國際小學數學及自然科學奧林匹亞IMSO
英文版試題
數學探索題第四題
\vspace{5mm}

\includegraphics[scale=0.5]{divide_5_6.png}


Analysis:

The provided graph and the arangement in this problem is intended to distract and steer the mindset of the students to think that this is a complicated geometric graph problem, but in fact it is really simple. The left graph serves as a decoy example of complicated geometric rotation and symmetry, and this tricks the students into thinking of more complicated geometric shapes in order to fill the same area with 5 and more pieces.

However, it is not your fault to fall into the trap, especially in a dilerberate one like this, shit happens all the time. This is a great example of learning how to pull yourself out of the trap. I was tricked into the complicated graph arrangment. But I stay calm. Forget about the problem. If I have a square, how to split it up into 5 identical pieces. If the area is 1 by 1 in size, then I am looking for an shape of an area of size equal to 0.2 (for 5 pieces). And to keep things simple I randomly choose an rectangle of size 1 by 0.2. This is a strip. And suddenly I realized 5 of this rectangle can make up for an square. And geometric translation suddenly strick my head, after realizing the shape in the problem has translational symmetry. The rest is simple.

Solution is the following.

\begin{pspicture}(1in,0.6in)
%\psgrid

\multirput(0,0)(0.2,0){5}{\psline(0.2,0)(0,0)(0,0.5)(0.3,0.75)(0,1)(0.2,1)}
\psline(1,0)(1,0.5)(1.3,0.75)(1,1)

\multirput(2,0)(0.1667,0){6}{\psline(0.1667,0)(0,0)(0,0.5)(0.3,0.75)(0,1)(0.1667,1)}
\psline(3,0)(3,0.5)(3.3,0.75)(3,1)


\end{pspicture}


\end{document}