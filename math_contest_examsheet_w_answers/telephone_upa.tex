\documentclass{article}
\usepackage{fontspec}
\usepackage{xeCJK}
\setmainfont{Times New Roman}
\setsansfont{Verdana}
\setmonofont{Courier New}
\setCJKmainfont{微軟正黑體}
\usepackage{eso-pic}
\usepackage{graphicx}
\usepackage{lipsum}
\usepackage{color}

\begin{document}


%TCIMACRO{%
%\TeXButton{watermark}{\AddToShipoutPictureBG{  \AtTextLowerLeft{  \resizebox{\textwidth}{!}{\rotatebox{60}{{\color{cyan} whymranderson 製版}}}
%   }}}}%
%BeginExpansion
\AddToShipoutPictureBG{  \AtTextLowerLeft{  \resizebox{\textwidth}{!}{\rotatebox{60}{{\color{cyan} whymranderson 製版}}}
   }}%
%EndExpansion

\begin{description}
\item[2012小學數學競賽初賽%
選拔賽第二試應用題%
第十二題] 小明家的電%
話號碼原為六位數,%
因號碼不夠用,電信%
公司在所有電話號碼%
的首位數與第二位數%
之間加上一個數碼1而%
成為一個七位數的電%
話號碼。數年後,電%
信公司發現號碼仍不%
敷使用,因此再將所%
有電話號碼的首位數%
前加上一個數碼2而成%
為一個八位數的電話%
號碼。小明發現經過%
這兩次更改後,家中%
最新的八位數電話號%
碼為原先六位數電話%
號碼的97倍。請問小明%
家最新的八位數電話%
號碼是什麼?
\end{description}

\bigskip

假設原六位數為$XYZABC$,%
後來的八位數為$2X1YZABC$,%
條件是%
\[
XYZABC\times 97=2X1YZABC
\]%
。寫成十進位制即為%
,%
\begin{eqnarray*}
&&\left( 10000X+10000Y+1000Z+100A+10B+C\right) \times 97 \\
&=&20000000+1000000X+100000+10000Y+1000Z+100A+10B+C
\end{eqnarray*}%
為了不要寫那麼多個%
零,我們消去三個零%
變成(這一步可以不需%
要做)%
\begin{eqnarray*}
&&\left( 100X+10Y+Z+0.1A+0.01B+0.001C\right) \times 97 \\
&=&20000+1000X+100+10Y+Z+0.1A+0.01B+0.001C
\end{eqnarray*}%
整理一下我們得到%
\begin{equation}
8700X+960Y+96Z+9.6A+0.96B+0.096C=20100  \label{fineX}
\end{equation}%
現在我們知道$YZABC$最小%
是0,最大是9%
\[
0\leq 960Y+96Z+9.6A+0.96B+0.096C\leq 960\times 9+96\times 9+\cdots 
\]%
也就是%
\[
10500\leq 8700X\leq 20100
\]%
這樣我們會發現%
\[
1.2\leq X\leq 2.3
\]%
所以$X$只能等於2。代入%
\ref{fineX}式中得到%
\[
960Y+96Z+9.6A+0.96B+0.096C=20100-8700X=2700
\]%
用同樣的方法,因為$ZABC
$最小零最大九%
\[
1740\leq 960Y=2700-\left( 96Z+9.6A+0.96B+0.096C\right) \leq 2700
\]%
這樣$Y$只能為2。再代入%
\ref{fineX}式中整理%
\[
96Z+9.6A+0.96B+0.096C=20100-8700\times 2-960\times 2
\]%
所以%
\[
684\leq 96Z=780-\left( 9.6A+0.96B+0.096C\right) \leq 780
\]%
這樣$Z$等於8。再代入\ref%
{fineX}式中整理%
\[
9.6A+0.96B+0.096C=20100-8700\times 2-960\times 2-96\times 8
\]%
因此%
\[
2.49\leq 9.6A=12-\left( 0.96B+0.096C\right) \leq 12
\]%
所以$A=1$。再代入\ref{fineX}式%
中整理%
\[
0.96B+0.096C=20100-8700\times 2-960\times 2-96\times 8-9.6\times 1=2.4
\]%
這樣%
\[
1.6\leq 0.96B\leq 2.4
\]%
所以$B=2$。再代入\ref{fineX}式%
最後可求得C%
\[
0.096C=20100-8700\times 2-960\times 2-96\times 8-9.6\times 1-0.96\times 2
\]%
得到$C=5$。所以原六位數%
為228125,八位數為22128125。

\end{document}
