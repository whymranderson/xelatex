\documentclass{article}
\usepackage[inner=0.5 in,outer=0.5in]{geometry}
\usepackage{fontspec}
\usepackage{xeCJK}
\setmainfont{Times New Roman}
\setsansfont{Verdana}
\setmonofont{Courier New}
\setCJKmainfont{微軟正黑體}
\usepackage{eso-pic}
\usepackage{graphicx}
\usepackage{lipsum}
\usepackage{color}

\begin{document}


%TCIMACRO{%
%\TeXButton{watermark}{\AddToShipoutPictureBG{  \AtTextLowerLeft{  \resizebox{\textwidth}{!}{\rotatebox{60}{{\color{cyan} whymranderson 製版}}}
%}}}}%
%BeginExpansion
\AddToShipoutPictureBG{  \AtTextLowerLeft{  \resizebox{\textwidth}{!}{\rotatebox{45}{{\color{cyan} whymranderson 製版}}}
}}%
%EndExpansion

\begin{description}
\item[2007 國際小學數學自然%
科學奧林匹亞ISMO-數學基%
本題第一題] The 400-digit number
12345678901234567890\ldots 890 is given.

\item Step 1: Cross out all the digits in odd-numbered places.

\item Step 2: Cross out all the digits in odd-numbered places of the
remaining digits.

\item \ldots 

\item Continue until no digits remain. What is the last digit to be crossed
out?
\end{description}

\vspace{2cm}
Solution:
\vspace{1cm}

%\documentclass[a4paper]{article}
%\usepackage[inner=0.5 in,outer=0.5in,top=0.7in,bottom=1in]{geometry}
%\usepackage{fontspec}
%\usepackage{xeCJK}
%\setmainfont{Times New Roman}
%\setsansfont{Verdana}
%\setmonofont{Courier New}                    % tt
%\setCJKmainfont{微軟正黑體}
%
%\begin{document}



\begin{tabular}[b]{c | c c c c c c c c c c c c c c c c c c c c }
position&1st&2nd&3rd&&&&&&&&&&13th\\ \hline
total num of digits: 400&1&2&3&4&5&6&7&8&9&0&1&2&3&4&5&6&7&8&9&0\\
200& &2&&4&&6&&8&&0&&2&&4&&6&&8&&0\\
100& &&&4&&&&8&&&&2&&&&6&&&&0\\
50& &&&&&&&8&&&&&&&&6&&&&\\
25& &&&&&&&&&&&&&&&6&&&&\\
12\\
6\\
3\\
1\\
\end{tabular}



%\end{document}\bigskip 

After the first operation, 200 digits will remain and the positions of these
remains are $2\times N$, where $N$ is an interger. After the second
operation, 100 digits will remain and the position of them are $2^{2}\times
N $. 50 digits remain after the third operation and positions of them are $%
2^{3}\times N$. So in the last operation only 1 digit remains and the
position of this digit is $2^{8}\times N$. But the position has to been
smaller than 400, so $N$ can only be 1. So the position is $2^{8}\times
1=256 $. Because the digit sequence has a repetive period of 10 digits. 256
position is $25\times 10+6$, so it is the 6th position in the sequence,
which has a number 6. QED

\end{document}
