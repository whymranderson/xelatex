
\documentclass{article}
\usepackage{fontspec}
\usepackage{xeCJK}
\setmainfont{Times New Roman}
\setsansfont{Verdana}
\setmonofont{Courier New}
\setCJKmainfont{微軟正黑體}
\usepackage{eso-pic}
\usepackage{graphicx}
\usepackage{lipsum}
\usepackage{color}

\newtheorem{theorem}{Theorem}
\newtheorem{acknowledgement}[theorem]{Acknowledgement}
\newtheorem{algorithm}[theorem]{Algorithm}
\newtheorem{axiom}[theorem]{Axiom}
\newtheorem{case}[theorem]{Case}
\newtheorem{claim}[theorem]{Claim}
\newtheorem{conclusion}[theorem]{Conclusion}
\newtheorem{condition}[theorem]{Condition}
\newtheorem{conjecture}[theorem]{Conjecture}
\newtheorem{corollary}[theorem]{Corollary}
\newtheorem{criterion}[theorem]{Criterion}
\newtheorem{definition}[theorem]{Definition}
\newtheorem{example}[theorem]{Example}
\newtheorem{exercise}[theorem]{Exercise}
\newtheorem{lemma}[theorem]{Lemma}
\newtheorem{notation}[theorem]{Notation}
\newtheorem{problem}[theorem]{Problem}
\newtheorem{proposition}[theorem]{Proposition}
\newtheorem{remark}[theorem]{Remark}
\newtheorem{solution}[theorem]{Solution}
\newtheorem{summary}[theorem]{Summary}
\newenvironment{proof}[1][Proof]{\noindent\textbf{#1.} }{\ \rule{0.5em}{0.5em}}

\begin{document}


%TCIMACRO{%
%\TeXButton{donimal graph1}{\begin{center}
%\includegraphics[scale=0.3]{donimal_graph1.png}
%\end{center}}}%
%BeginExpansion
\begin{center}
\includegraphics[scale=0.3]{donimal_graph1.png}
\end{center}%
%EndExpansion

\begin{description}
\item[2006小學數學競賽選拔%
賽初賽第二試應用題%
第八題] 一套多明諾骨%
牌共有以上的28張。小%
華依照相鄰兩張牌相%
連接處的點數必須相%
同的規定,將這套多%
明諾骨牌排成如下圖%
的一長列,直到28張牌%
排完為止。請問排完%
後,最右邊小方格內%
的點數是幾點?
\end{description}

%TCIMACRO{%
%\TeXButton{donimal graph 2}{\begin{center}
%\includegraphics[scale=0.3]{donimal_graph2.png}
%\end{center}}}%
%BeginExpansion
\begin{center}
\includegraphics[scale=0.3]{donimal_graph2.png}
\end{center}%
%EndExpansion

\bigskip 

28張多明諾骨牌中的每%
一個數字0\symbol{126}8都有八%
個,也就是偶數個。%
在相接骨牌時,由於%
數字必須相同,代表%
相接的數字必須成雙%
成對,也就是有相接%
的數字出現的次數都%
是偶數。但是第一個%
字母沒有與其他相接%
,由於28張骨牌最後都%
排完,代表每個數字%
都要出現偶數次在骨%
牌中,但是第一個五%
沒有與其他相接,是%
奇數,而中間成雙成%
對的五一定都是偶數%
個,所以最後一個數%
字必須是五來達成每%
個字母都要出現偶數%
次的要求。

Q.E.D.

\AddToShipoutPictureBG*{% Add picture to current page
  \AtTextLowerLeft{% Add picture to lower-left corner of paper stock
  \resizebox{\textwidth}{!}{\rotatebox{60}{{\color{yellow} whymranderson 製版}}}
    }% http://latex.tug.org/texlive/devsrc/Master/texmf-dist/doc/generic/pstricks/images/tiger.eps
}

\end{document}
