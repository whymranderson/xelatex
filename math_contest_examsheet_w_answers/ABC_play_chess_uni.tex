
\documentclass{article}
\usepackage{fontspec}
\usepackage{xeCJK}
\setmainfont{Times New Roman}
\setsansfont{Verdana}
\setmonofont{Courier New}
\setCJKmainfont{微軟正黑體}
\usepackage{eso-pic}
\usepackage{graphicx}
\usepackage{lipsum}
\usepackage{color}


\newtheorem{theorem}{Theorem}
\newtheorem{acknowledgement}[theorem]{Acknowledgement}
\newtheorem{algorithm}[theorem]{Algorithm}
\newtheorem{axiom}[theorem]{Axiom}
\newtheorem{case}[theorem]{Case}
\newtheorem{claim}[theorem]{Claim}
\newtheorem{conclusion}[theorem]{Conclusion}
\newtheorem{condition}[theorem]{Condition}
\newtheorem{conjecture}[theorem]{Conjecture}
\newtheorem{corollary}[theorem]{Corollary}
\newtheorem{criterion}[theorem]{Criterion}
\newtheorem{definition}[theorem]{Definition}
\newtheorem{example}[theorem]{Example}
\newtheorem{exercise}[theorem]{Exercise}
\newtheorem{lemma}[theorem]{Lemma}
\newtheorem{notation}[theorem]{Notation}
\newtheorem{problem}[theorem]{Problem}
\newtheorem{proposition}[theorem]{Proposition}
\newtheorem{remark}[theorem]{Remark}
\newtheorem{solution}[theorem]{Solution}
\newtheorem{summary}[theorem]{Summary}
\newenvironment{proof}[1][Proof]{\noindent\textbf{#1.} }{\ \rule{0.5em}{0.5em}}
%\input{tcilatex}

\begin{document}

\begin{description}
\item[2006小學數學競賽選拔%
賽初賽第二試應用題%
第三題] ABC三個人玩棋,%
沒有和棋,輸得換人%
,贏的繼續玩。只知%
道A最後共玩了10盤,B共%
玩了21盤,請問第九盤%
是誰跟誰玩?
\end{description}

\bigskip

先畫張表,1代表玩,0%
代表沒玩。ABC代表三個%
人。

%TCIMACRO{%
%\TeXButton{Table}{\begin{center}
%%
%\documentclass{article}
%\usepackage{fontspec}
%\usepackage{xeCJK}
%\setmainfont{Times New Roman}
%\setsansfont{Verdana}
%\setmonofont{Courier New}
%\setCJKmainfont{微軟正黑體}
%
%\begin{document}

\begin{tabular}{| c| l | l | l |}
  %\multicolumn{2}{|c|}{RGCordTransV11.py} \\
   \hline
  	 & A& B &C \\%\\ \cline{2-2}
   \hline
   1       & 1 & 1 	&0 \\
   2	&  1&0	&1\\
   3	& 0 &1	&1\\
   4 	& 1 &1	&0\\
   5	& 0 &1	&1\\
   6	& 1 &0	&1\\
  \vdots &\vdots  &\vdots	&\vdots\\
 第X盤 & 0 &1	&1\\
        \hline
   
\end{tabular}



%\end{document}
%\end{center}}}%
%BeginExpansion
\begin{center}
%
%\documentclass{article}
%\usepackage{fontspec}
%\usepackage{xeCJK}
%\setmainfont{Times New Roman}
%\setsansfont{Verdana}
%\setmonofont{Courier New}
%\setCJKmainfont{微軟正黑體}
%
%\begin{document}

\begin{tabular}{| c| l | l | l |}
  %\multicolumn{2}{|c|}{RGCordTransV11.py} \\
   \hline
  	 & A& B &C \\%\\ \cline{2-2}
   \hline
   1       & 1 & 1 	&0 \\
   2	&  1&0	&1\\
   3	& 0 &1	&1\\
   4 	& 1 &1	&0\\
   5	& 0 &1	&1\\
   6	& 1 &0	&1\\
  \vdots &\vdots  &\vdots	&\vdots\\
 第X盤 & 0 &1	&1\\
        \hline
   
\end{tabular}



%\end{document}
\end{center}%
%EndExpansion
\bigskip 

輸了就要換人,代表%
每個人不能連玩,也%
就是不能有兩個零是%
上下相連。

\bigskip

$A_{0}$代表$A$輸的總次數,$%
A_{1}$代表$A$贏的總次數。%
相同的我們也定義$%
B_{0}B_{1}C_{0}C_{1}$。我們假設總共%
有x盤,這樣的話$A_{0}+A_{1}=x$%
,$B_{0}+B_{1}=x$,$C_{0}+C_{1}=x$。因最%
後我們知道$A$玩了10盤,%
$B$玩了21盤,所以$A_{1}=10$,$%
B_{1}=21$。所以$A_{0}=x-10$,$B_{0}=x-21$,%
且我們知道$A_{0}$、$B_{0}$必%
須為大於等於零的整%
數,所以$x\geq 21$。

\bigskip 

但因為每個人的0不能%
相臨,代表$A_{0}$最多只%
能為$\frac{x}{2}$(若x為偶數),%
或$\frac{x+1}{2}($若x為奇數)。也%
就是說$A_{0}$不管如何都%
小於等於$\frac{x+1}{2}$。不過%
我們又知道$A_{0}=x-10$,這樣%
我們得到$x-10\leq \frac{x+1}{2}$,整%
理一下這樣$x\leq 21, $但是%
前面我們知道$x$必須大%
於等於21,所以$x$只能為%
21。這樣的話代表總共%
有21盤,然後$A_{0}=11$,代表%
21盤中A沒玩11盤,但是因%
不會連著沒玩,所以%
一定是一盤沒玩一盤%
玩,且A第一盤一定是0%
是輸,這樣21盤中才可%
給出11個0。這樣也代表A%
奇數盤都是0,所以第%
九盤A是0,也就是A沒玩%
,所以是BC玩。Q.E.D.

\bigskip

\begin{exercise}
衍伸題。若A玩12盤,B玩9%
盤,那C最多玩幾盤?
\end{exercise}

\AddToShipoutPictureBG*{% Add picture to current page
  \AtTextLowerLeft{% Add picture to lower-left corner of paper stock
  \resizebox{\textwidth}{!}{\rotatebox{60}{{\color{cyan} whymranderson 製版}}}
    }% http://latex.tug.org/texlive/devsrc/Master/texmf-dist/doc/generic/pstricks/images/tiger.eps
}

\end{document}
