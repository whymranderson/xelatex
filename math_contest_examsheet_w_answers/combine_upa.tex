\documentclass[12pt]{article}
\usepackage[inner=0.75 in,outer=0.75in]{geometry}
\usepackage{xeCJK}
\setmainfont{Times New Roman}
\setsansfont{Verdana}
\setmonofont{Courier New}
\setCJKmainfont{微軟正黑體}
\usepackage{eso-pic}
\usepackage{graphicx}
\usepackage{lipsum}
\usepackage{color}
\usepackage{pstricks}

\begin{document}


%TCIMACRO{%
%\TeXButton{watermark}{\AddToShipoutPictureBG{  \AtTextLowerLeft{  \resizebox{\textwidth}{!}{\rotatebox{45}{{\color{cyan} whymranderson 製版}}}
%}}}}%
%BeginExpansion
\AddToShipoutPictureBG{  \AtTextLowerLeft{  \resizebox{\textwidth}{!}{\rotatebox{45}{{\color{cyan} whymranderson 製版}}}
}}%
%EndExpansion

\begin{description}
\item[2007 國際小學數學自然%
科學奧林匹亞ISMO-數學基%
本題第一題] The 400-digit number 
\[
12345678901234567890\ldots 890 
\]
is given.
\end{description}

Step 1: Cross out all the digits in odd-numbered places.

Step 2: Cross out all the digits in odd-numbered places of the remaining
digits.

\ldots

Continue until no digits remain. What is the last digit to be crossed out?

%\documentclass[a4paper]{article}
%\usepackage[inner=0.5 in,outer=0.5in,top=0.7in,bottom=1in]{geometry}
%\usepackage{fontspec}
%\usepackage{xeCJK}
%\setmainfont{Times New Roman}
%\setsansfont{Verdana}
%\setmonofont{Courier New}                    % tt
%\setCJKmainfont{微軟正黑體}
%
%\begin{document}



\begin{tabular}[b]{c | c c c c c c c c c c c c c c c c c c c c }
position&1st&2nd&3rd&&&&&&&&&&13th\\ \hline
total num of digits: 400&1&2&3&4&5&6&7&8&9&0&1&2&3&4&5&6&7&8&9&0\\
200& &2&&4&&6&&8&&0&&2&&4&&6&&8&&0\\
100& &&&4&&&&8&&&&2&&&&6&&&&0\\
50& &&&&&&&8&&&&&&&&6&&&&\\
25& &&&&&&&&&&&&&&&6&&&&\\
12\\
6\\
3\\
1\\
\end{tabular}



%\end{document}\bigskip

After the first operation, 200 digits will remain and the positions of these
remains are $2\times N$, where $N$ is an interger. After the second
operation, 100 digits will remain and the position of them are $2^{2}\times
N $. 50 digits remain after the third operation and positions of them are $%
2^{3}\times N$. So in the last operation only 1 digit remains and the
position of this digit is $2^{8}\times N$. But the position has to been
smaller than 400, so $N$ can only be 1. So the position is $2^{8}\times
1=256 $. Because the digit sequence has a repetive period of 10 digits. 256
position is $25\times 10+6$, so it is the 6th position in the sequence,
which has a number 6. QED

\newpage

\begin{description}
\item[網路題目] 一個多位%
數,此多位數的開頭%
前幾位數可以是零,%
如$00358$。把最後一位數%
提到前面就變成原來%
那位數的兩倍,求最%
小的此多位數。
\end{description}

\bigskip

\bigskip

設此多位數最後一位%
數為$b\left( 0<b<9\right) $,其餘位%
數為$a$,$a$為整數,因%
此此多位數為$10a+b$%
\[
\left( 10a+b\right) \times 2=10^{n}\times b+a 
\]%
\[
\rightarrow 19a=\left( 10^{n}-2\right) b 
\]%
\[
\rightarrow a=\frac{10^{n}-2}{19}\times b 
\]

接下來我們提供兩個%
做法,我的方法是用%
電腦去解,有點作弊%
。這裡感謝張廷宇醫%
師提供了一個更正確%
的作法,可以不需要%
用到電腦。我們會先%
解釋張醫師的作法。

\bigskip

通常$1$除以質數的結果%
會是循環小數,而$\frac{1}{19}
$也是如此。而我們知%
道循環小數是可以分%
數化的。我們要找到%
符合上面式子的$a$與$b$%
,可以朝$\frac{1}{19}$循環小%
數分數化的方向去試%
試看。說不定會找到%
合適的$a$與$b$。

\bigskip

我們先算出$\frac{1}{19}=0.\overline{%
052631578947368421}$,後面$18$位數會%
無限循環,依照循環%
小數分數化的公式,%
我們可以重寫成%
\[
\frac{1}{19}=\frac{052631578947368421}{10^{18}-1}=\frac{k}{10^{18}-1} 
\]%
其中$k=052631578947368421$。朝上面%
的目標式整理%
\[
\frac{10^{18}-1}{19}-1=k-1 
\]%
\[
\Rightarrow 10\times \frac{10^{17}-2}{19}=k-1=052631578947368420 
\]%
\[
\Rightarrow 1\times \frac{10^{17}-2}{19}=05263157894736842 
\]%
因此我們就找到符合%
目標式子的$a$與$b$了。%
注意我們刻意保留$%
05263157894736842$的首位數$0$,因為%
這樣才會有$18$位循環小%
數,分數化公式才會%
成立。這樣我們找到$%
a=05263157894736842$,$b=1$。所以此數%
字為$052631578947368421$。驗證一下$%
2\times 052631578947368421=105263157894736842$沒錯!

\bigskip

那如何證明這是最小%
滿足條件的數目呢?%
因為$k$是$\frac{1}{19}$的循環位%
數,而$\frac{1}{19}$的數值也%
只有一個,這代表$k$是%
唯一的循環位數,因%
此也就會是最小的循%
環位數。這樣子$\frac{k-1}{10}%
=05263157894736842$,也就是$a$,就%
是最小滿足條件式的%
數值。而$b=1$也是最小,%
因此$10a+b$,就是最小的%
滿足條件的數字。

\bigskip

(我的作弊方法)由於$0<b<9$%
,因此$b$的公因數不可%
能有$19$,所以由於$a$是%
一整數,代表$\frac{10^{n}-2}{19}$必%
須是一整數。也就是%
我們要用方法去找到%
一個$n$讓$10^{n}-2$有最大質因%
數$19$,我能想到的方法%
只能用電腦一個一個%
去代入$n$然後用電腦求%
餘數mod找出第一個找到$%
\frac{10^{n}-2}{19}$餘數為零的$n$。%
電腦程式如下
\begin{verbatim}
n = 0
while (10**n-2) % 19 != 0:
    n=n+1
print n
\end{verbatim}

找到最小的$n$為$17$,與%
前面方法相同。因此$a$%
為$5263157894736842$,$b$為$1$,所以%
此數為$52631578947368421$。驗算過%
後發現我們必須在$a$首%
位加個零,這樣尾數%
提到前面才會是$%
2a=105263157894736842$。因此滿足題%
目最小的此位數為$%
052631578947368421$。

\bigskip

Remark: 這裡也提出一個問%
題,我在算$\frac{1}{19}$的時%
候遇到一個問題,就%
是電腦無法顯示足夠%
的位數,計算機更不%
用說,因此我是查網%
路才知道$\frac{1}{19}$的結果%
與他有幾位旋環小數%
。這是電腦中很常見%
的問題。其實電腦應%
該有指令可以顯示出$64$%
位元的浮點數float,因此%
我還要去查一下。

\newpage

\begin{description}
\item[2006小學數學競賽選拔%
賽初賽第二試應用題%
第三題] ABC三個人玩棋,%
沒有和棋,輸得換人%
,贏的繼續玩。只知%
道A最後共玩了10盤,B共%
玩了21盤,請問第九盤%
是誰跟誰玩?
\end{description}

\bigskip

先畫張表,1代表玩,0%
代表沒玩。ABC代表三個%
人。

%TCIMACRO{%
%\TeXButton{Table}{\begin{center}
%%
%\documentclass{article}
%\usepackage{fontspec}
%\usepackage{xeCJK}
%\setmainfont{Times New Roman}
%\setsansfont{Verdana}
%\setmonofont{Courier New}
%\setCJKmainfont{微軟正黑體}
%
%\begin{document}

\begin{tabular}{| c| l | l | l |}
  %\multicolumn{2}{|c|}{RGCordTransV11.py} \\
   \hline
  	 & A& B &C \\%\\ \cline{2-2}
   \hline
   1       & 1 & 1 	&0 \\
   2	&  1&0	&1\\
   3	& 0 &1	&1\\
   4 	& 1 &1	&0\\
   5	& 0 &1	&1\\
   6	& 1 &0	&1\\
  \vdots &\vdots  &\vdots	&\vdots\\
 第X盤 & 0 &1	&1\\
        \hline
   
\end{tabular}



%\end{document}
%\end{center}}}%
%BeginExpansion
\begin{center}
%
%\documentclass{article}
%\usepackage{fontspec}
%\usepackage{xeCJK}
%\setmainfont{Times New Roman}
%\setsansfont{Verdana}
%\setmonofont{Courier New}
%\setCJKmainfont{微軟正黑體}
%
%\begin{document}

\begin{tabular}{| c| l | l | l |}
  %\multicolumn{2}{|c|}{RGCordTransV11.py} \\
   \hline
  	 & A& B &C \\%\\ \cline{2-2}
   \hline
   1       & 1 & 1 	&0 \\
   2	&  1&0	&1\\
   3	& 0 &1	&1\\
   4 	& 1 &1	&0\\
   5	& 0 &1	&1\\
   6	& 1 &0	&1\\
  \vdots &\vdots  &\vdots	&\vdots\\
 第X盤 & 0 &1	&1\\
        \hline
   
\end{tabular}



%\end{document}
\end{center}%
%EndExpansion
\bigskip

輸了就要換人,代表%
每個人不能連玩,也%
就是不能有兩個零是%
上下相連。

\bigskip

$A_{0}$代表$A$輸的總次數,$%
A_{1}$代表$A$贏的總次數。%
相同的我們也定義$%
B_{0}B_{1}C_{0}C_{1}$。我們假設總共%
有x盤,這樣的話$A_{0}+A_{1}=x$%
,$B_{0}+B_{1}=x$,$C_{0}+C_{1}=x$。因最%
後我們知道$A$玩了10盤,%
$B$玩了21盤,所以$A_{1}=10$,$%
B_{1}=21$。所以$A_{0}=x-10$,$B_{0}=x-21$,%
且我們知道$A_{0}$、$B_{0}$必%
須為大於等於零的整%
數,所以$x\geq 21$。

\bigskip

但因為每個人的0不能%
相臨,代表$A_{0}$最多只%
能為$\frac{x}{2}$(若x為偶數),%
或$\frac{x+1}{2}($若x為奇數)。也%
就是說$A_{0}$不管如何都%
小於等於$\frac{x+1}{2}$。不過%
我們又知道$A_{0}=x-10$,這樣%
我們得到$x-10\leq \frac{x+1}{2}$,整%
理一下這樣$x\leq 21, $但是%
前面我們知道$x$必須大%
於等於21,所以$x$只能為%
21。這樣的話代表總共%
有21盤,然後$A_{0}=11$,代表%
21盤中A沒玩11盤,但是因%
不會連著沒玩,所以%
一定是一盤沒玩一盤%
玩,且A第一盤一定是0%
是輸,這樣21盤中才可%
給出11個0。這樣也代表A%
奇數盤都是0,所以第%
九盤A是0,也就是A沒玩%
,所以是BC玩。Q.E.D.

\bigskip

衍伸題。若A玩12盤,B玩9%
盤,那C最多玩幾盤?

\newpage

%TCIMACRO{%
%\TeXButton{donimal graph1}{\begin{center}
%\includegraphics[scale=0.3]{donimal_graph1.png}
%\end{center}}}%
%BeginExpansion
\begin{center}
\includegraphics[scale=0.3]{donimal_graph1.png}
\end{center}%
%EndExpansion

\begin{description}
\item[2006小學數學競賽選拔%
賽初賽第二試應用題%
第八題] 一套多明諾骨%
牌共有以上的28張。小%
華依照相鄰兩張牌相%
連接處的點數必須相%
同的規定,將這套多%
明諾骨牌排成如下圖%
的一長列,直到28張牌%
排完為止。請問排完%
後,最右邊小方格內%
的點數是幾點?
\end{description}

%TCIMACRO{%
%\TeXButton{donimal graph 2}{\begin{center}
%\includegraphics[scale=0.3]{donimal_graph2.png}
%\end{center}}}%
%BeginExpansion
\begin{center}
\includegraphics[scale=0.3]{donimal_graph2.png}
\end{center}%
%EndExpansion

\bigskip

28張多明諾骨牌中的每%
一個數字0\symbol{126}8都有八%
個,也就是偶數個。%
在相接骨牌時,由於%
數字必須相同,代表%
相接的數字必須成雙%
成對,也就是有相接%
的數字出現的次數都%
是偶數。但是第一個%
字母沒有與其他相接%
,由於28張骨牌最後都%
排完,代表每個數字%
都要出現偶數次在骨%
牌中,但是第一個五%
沒有與其他相接,是%
奇數,而中間成雙成%
對的五一定都是偶數%
個,所以最後一個數%
字必須是五來達成每%
個字母都要出現偶數%
次的要求。

\newpage

\begin{description}
\item[2007國際小學數學及自%
然科學奧林匹亞IMSO英文%
版試題數學探索題第%
四題] 
\end{description}

%TCIMACRO{%
%\TeXButton{graphics}{\includegraphics[scale=0.5]{divide_5_6.png}}}%
%BeginExpansion
\includegraphics[scale=0.5]{divide_5_6.png}%
%EndExpansion

\bigskip

Analysis:

The provided graph and the arangement in this problem is intended to
distract and steer the mindset of the students to think that this is a
complicated geometric graph problem, but in fact it is really simple. The
left graph serves as a decoy example of complicated geometric rotation and
symmetry, and this tricks the students into thinking of more complicated
geometric shapes in order to fill the same area with 5 and more pieces.

\bigskip

However, it is not your fault to fall into the trap, especially in a
dilerberate one like this, shit happens all the time. This is a great
example of learning how to pull yourself out of the trap. I was tricked into
the complicated graph arrangment. But I stay calm. Forget about the problem.
If I have a square, how to split it up into 5 identical pieces. If the area
is 1 by 1 in size, then I am looking for an shape of an area of size equal
to 0.2 (for 5 pieces). And to keep things simple I randomly choose an
rectangle of size 1 by 0.2. This is a strip. And suddenly I realized 5 of
this rectangle can make up for an square. And geometric translation suddenly
strick my head, after realizing the shape in the problem has translational
symmetry. The rest is simple.

Solution is the following.

%TCIMACRO{%
%\TeXButton{solution}{\begin{pspicture}(1in,0.6in)
%
%\multirput(0,0)(0.2,0){5}{\psline(0.2,0)(0,0)(0,0.5)(0.3,0.75)(0,1)(0.2,1)}
%\psline(1,0)(1,0.5)(1.3,0.75)(1,1)
%
%\multirput(2,0)(0.1667,0){6}{\psline(0.1667,0)(0,0)(0,0.5)(0.3,0.75)(0,1)(0.1667,1)}
%\psline(3,0)(3,0.5)(3.3,0.75)(3,1)
%
%
%\end{pspicture}
%
%}}%
%BeginExpansion
\begin{pspicture}(1in,0.6in)

\multirput(0,0)(0.2,0){5}{\psline(0.2,0)(0,0)(0,0.5)(0.3,0.75)(0,1)(0.2,1)}
\psline(1,0)(1,0.5)(1.3,0.75)(1,1)

\multirput(2,0)(0.1667,0){6}{\psline(0.1667,0)(0,0)(0,0.5)(0.3,0.75)(0,1)(0.1667,1)}
\psline(3,0)(3,0.5)(3.3,0.75)(3,1)


\end{pspicture}

%
%EndExpansion

\newpage

\begin{description}
\item[2012小學數學競賽初賽%
選拔賽第二試應用題%
第十二題] 小明家的電%
話號碼原為六位數,%
因號碼不夠用,電信%
公司在所有電話號碼%
的首位數與第二位數%
之間加上一個數碼1而%
成為一個七位數的電%
話號碼。數年後,電%
信公司發現號碼仍不%
敷使用,因此再將所%
有電話號碼的首位數%
前加上一個數碼2而成%
為一個八位數的電話%
號碼。小明發現經過%
這兩次更改後,家中%
最新的八位數電話號%
碼為原先六位數電話%
號碼的97倍。請問小明%
家最新的八位數電話%
號碼是什麼?
\end{description}

\bigskip

假設原六位數為$XYZABC$,%
後來的八位數為$2X1YZABC$,%
條件是%
\[
XYZABC\times 97=2X1YZABC
\]%
。寫成十進位制即為%
,%
\begin{eqnarray*}
&&\left( 10000X+10000Y+1000Z+100A+10B+C\right) \times 97 \\
&=&20000000+1000000X+100000+10000Y+1000Z+100A+10B+C
\end{eqnarray*}%
為了不要寫那麼多個%
零,我們消去三個零%
變成(這一步可以不需%
要做)%
\begin{eqnarray*}
&&\left( 100X+10Y+Z+0.1A+0.01B+0.001C\right) \times 97 \\
&=&20000+1000X+100+10Y+Z+0.1A+0.01B+0.001C
\end{eqnarray*}%
整理一下我們得到%
\begin{equation}
8700X+960Y+96Z+9.6A+0.96B+0.096C=20100  \label{fineX}
\end{equation}%
現在我們知道$YZABC$最小%
是0,最大是9%
\[
0\leq 960Y+96Z+9.6A+0.96B+0.096C\leq 960\times 9+96\times 9+\cdots 
\]%
也就是%
\[
10500\leq 8700X\leq 20100
\]%
這樣我們會發現%
\[
1.2\leq X\leq 2.3
\]%
所以$X$只能等於2。代入%
\ref{fineX}式中得到%
\[
960Y+96Z+9.6A+0.96B+0.096C=20100-8700X=2700
\]%
用同樣的方法,因為$ZABC
$最小零最大九%
\[
1740\leq 960Y=2700-\left( 96Z+9.6A+0.96B+0.096C\right) \leq 2700
\]%
這樣$Y$只能為2。再代入%
\ref{fineX}式中整理%
\[
96Z+9.6A+0.96B+0.096C=20100-8700\times 2-960\times 2
\]%
所以%
\[
684\leq 96Z=780-\left( 9.6A+0.96B+0.096C\right) \leq 780
\]%
這樣$Z$等於8。再代入\ref%
{fineX}式中整理%
\[
9.6A+0.96B+0.096C=20100-8700\times 2-960\times 2-96\times 8
\]%
因此%
\[
2.49\leq 9.6A=12-\left( 0.96B+0.096C\right) \leq 12
\]%
所以$A=1$。再代入\ref{fineX}式%
中整理%
\[
0.96B+0.096C=20100-8700\times 2-960\times 2-96\times 8-9.6\times 1=2.4
\]%
這樣%
\[
1.6\leq 0.96B\leq 2.4
\]%
所以$B=2$。再代入\ref{fineX}式%
最後可求得C%
\[
0.096C=20100-8700\times 2-960\times 2-96\times 8-9.6\times 1-0.96\times 2
\]%
得到$C=5$。所以原六位數%
為228125,八位數為22128125。

\newpage

國中有一道機率的問%
題,我覺得值得我們%
聊聊,因為我們可以%
從這問題看到我們國%
高中教育與大學教育%
的銜接落差。而這主%
要是題目出得不好,%
是老師的問題。但就%
這麼一個簡單的出題%
的缺失,會造成非常%
多學生絞盡腦汁,花%
費且浪費非常多的時%
間,甚至錯失一個理%
解非常基本觀念的機%
會,你說這值不值得%
討論一下?

原題目在這裡%
\[
\text{https://www.youtube.com/watch?v=DefzS7\_OD74}
\]

\bigskip

題目是:甲有兩個小%
孩,若看到了甲其中%
一個小孩是女生,請%
問兩個小孩都是女生%
的機率是多少?

\bigskip

好,我們已經看到了%
一個小孩是女生了,%
所以題目問兩個小孩%
都是女生的機率,其%
實就是在問另一個小%
孩也是女生的機率是%
多少,那應該就是1/2摟%
?

\bigskip

這時候老師說話了。%
如果你列出所有可能%
,假設有AB兩個小孩,%
則所有可能為A男B男、A%
女B男、A男B女,A女B女。%
其中一個小孩是女生%
,所以第一種可能性%
去除,剩下三種可能%
性,而三種可能性之%
中只有一種是兩個女%
生,所以答案應該是1//3%
。

\bigskip

但是,我們來更仔細%
的審視原來的題目。%
我們重新把問題問的%
方式改變一下,但是%
還是一樣的問題。

\begin{enumerate}
\item 假設今天我們看到%
其中一個小孩名子叫Emily%
,她是女生,那麼另%
一個小孩是女生的機%
率是多少?

\item 假設今天有一個房%
間,裡面有兩個人,%
今天走出來一個人他%
告訴我們他叫emily,我們%
看到是女生,那請問%
房間裡面另一個人是%
女生的機率是多少?

\item 又,再換一個方式%
問,假設房間裡面有%
兩個人,我們只知道%
裡面至少有一個人是%
女生,我們並沒有看%
到,那麼請問房間裡%
面兩個人都是女生的%
機率是多少?
\end{enumerate}

\bigskip

以上所嘗試告訴大家%
的這個觀念其實是一%
個機率中非常重要的%
觀念,排列 與 組合。%
第一題的答案是1/2,因%
為第一個人我們已經%
看到是女生,這是第%
一件事,第二件事第%
二個人是男是女我們%
關心的是與第一件事%
情無關的事,兩件事%
情只是照著次序先後%
問,這種類型的概念%
我們稱作排列。第二%
題跟第一題一模一樣%
。第三題,因為我們%
只知道裡面有一個是%
女生,但是裡面AB兩人%
,我們不知道誰是女%
生,所以確實若是把%
所以可能性AB男女組合%
起來會有四種狀況,%
然後去掉兩個人都是%
男生的情況,剩三種%
狀況,而只有一種兩%
個人都是女生,所以%
是1/3。這種類型的概念%
,稱作組合。

\bigskip

我們再回來看原題目%
,"若看到了甲其中一%
個小孩是女生",這其%
實是有點給出了排列%
的概念,因為我們已%
經看到特定的人了,%
然後"請問兩個小孩都%
是女生的機率是多少%
?",這就奇怪了,這%
其實有點隱含了組合%
的意味在裡面了,有%
點在問組合的概念,%
這樣就題目就有點不%
清楚,到底是要問排%
列,還是組合的問題%
?

\bigskip

如果問題是"若看到了%
甲其中一個小孩是女%
生,另一個小孩是女%
生的機率是多少",那%
答案是1/2。這是非常恰%
當的排列問題。如果%
問題是"若知道甲至少%
有一個小孩是女生,%
那兩個小孩都是女生%
的機率是多少?",那%
答案是1/3。這樣就是非%
常好的組合問題。

\bigskip

為什麼這個在國中階%
段就很重要?因為在%
高中教到排列組合的%
時候,就用上了這個%
概念。舉個例子為什%
麼高中會用上這個例%
子。一個袋子裡面有%
黑白兩種球,黑白數%
目一樣多,拿出球看%
完顏色後要放回去。%
第一次從裡面拿出一%
顆白球,放回去,請%
問第二次是白球的機%
率是多少?是1/2。是排%
列問題。

\bigskip

若是一次拿出兩顆球%
,請問兩顆球都是白%
色的機率是多少,是1/4%
。是組合問題。若知%
道拿出的兩顆球其中%
至少有一顆球是白色%
,但不知道是哪一顆%
,請問兩顆球都是白%
色的機率是多少,是1/3%
。也是組合問題。

\bigskip

所以高中的排列組合%
的概念從國中就開始%
介紹了,但是看介紹%
得好不好。大學呢?%
從排列組合出發的應%
用性呢,那更多了。%
熱力學中的氣體分子%
動力論,分子的可分%
辨性與不可分辨姓,%
都是排列組合的原理%
。統計力學所用到的%
機率論也是以排列組%
合為其重要根基。

\bigskip

所以若是能在國中階%
段就將此觀念介紹清%
楚,那真的事會對未%
來事半功倍,而且從%
題目上的設計就很重%
要,若是題目設計得%
不好,那麼真的會事%
倍功半,這樣事半功%
倍與事倍功半的差距%
,可是四倍的差距阿%
。並且,如果考試中%
只有一題是這樣也就%
算了。要是所有題目%
有八成的題目都是這%
樣,那學生就很累很%
辛苦了,大概也根本%
不會想學了。不過學%
生或家長也不必太緊%
張,學習本來就是靠%
從錯誤中學,本來就%
是要靠摔倒來學習怎%
麼跑步。並不是說題%
目設計的好,學生就%
一定學得會,有太多%
的因素在裡面了。一%
步一步來,有穩定的%
進步才比較重要。

\end{document}
