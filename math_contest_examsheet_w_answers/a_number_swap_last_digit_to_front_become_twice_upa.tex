\documentclass{article}
\usepackage{xeCJK}
\setmainfont{Times New Roman}
\setsansfont{Verdana}
\setmonofont{Courier New}
\setCJKmainfont{微軟正黑體}
\usepackage{eso-pic}
\usepackage{graphicx}
\usepackage{lipsum}
\usepackage{color}

\begin{document}


%TCIMACRO{%
%\TeXButton{watermark}{\AddToShipoutPictureBG{  \AtTextLowerLeft{  \resizebox{\textwidth}{!}{\rotatebox{60}{{\color{cyan} whymranderson 製版}}}
%}}}}%
%BeginExpansion
\AddToShipoutPictureBG{  \AtTextLowerLeft{  \resizebox{\textwidth}{!}{\rotatebox{60}{{\color{cyan} whymranderson 製版}}}
}}%
%EndExpansion

\begin{itemize}
\item 一個多位數,此多%
位數的開頭前幾位數%
可以是零,如$00358$。把%
最後一位數提到前面%
就變成原來那位數的%
兩倍,求最小的此多%
位數。
\end{itemize}

\bigskip

\bigskip

設此多位數最後一位%
數為$b\left( 0<b<9\right) $,其餘位%
數為$a$,$a$為整數,因%
此此多位數為$10a+b$%
\[
\left( 10a+b\right) \times 2=10^{n}\times b+a 
\]%
\[
\rightarrow 19a=\left( 10^{n}-2\right) b 
\]%
\[
\rightarrow a=\frac{10^{n}-2}{19}\times b 
\]

接下來我們提供兩個%
做法,我的方法是用%
電腦去解,有點作弊%
。這裡感謝張廷宇醫%
師提供了一個更正確%
的作法,可以不需要%
用到電腦。我們會先%
解釋張醫師的作法。

\begin{enumerate}
\item 通常$1$除以質數的結%
果會是循環小數,而$%
\frac{1}{19}$也是如此。而我們%
知道循環小數是可以%
分數化的。我們要找%
到符合上面式子的$a$與$b
$,可以朝$\frac{1}{19}$循環小%
數分數化的方向去試%
試看。說不定會找到%
合適的$a$與$b$。\newline
我們先算出$\frac{1}{19}=0.\overline{%
052631578947368421}$,後面$18$位數會%
無限循環,依照循環%
小數分數化的公式,%
我們可以重寫成%
\[
\frac{1}{19}=\frac{052631578947368421}{10^{18}-1}=\frac{k}{10^{18}-1}
\]%
其中$k=052631578947368421$。朝上面%
的目標式整理%
\[
\frac{10^{18}-1}{19}-1=k-1
\]%
\[
\Rightarrow 10\times \frac{10^{17}-2}{19}=k-1=052631578947368420
\]%
\[
\Rightarrow 1\times \frac{10^{17}-2}{19}=05263157894736842
\]%
因此我們就找到符合%
目標式子的$a$與$b$了。%
注意我們刻意保留$%
05263157894736842$的首位數$0$,因為%
這樣才會有$18$位循環小%
數,分數化公式才會%
成立。這樣我們找到$%
a=05263157894736842$,$b=1$。所以此數%
字為$052631578947368421$。驗證一下$%
2\times 052631578947368421=105263157894736842$沒錯!%
\newline
那如何證明這是最小%
滿足條件的數目呢?%
因為$k$是$\frac{1}{19}$的循環位%
數,而$\frac{1}{19}$的數值也%
只有一個,這代表$k$是%
唯一的循環位數,因%
此也就會是最小的循%
環位數。這樣子$\frac{k-1}{10}%
=05263157894736842$,也就是$a$,就%
是最小滿足條件式的%
數值。而$b=1$也是最小,%
因此$10a+b$,就是最小的%
滿足條件的數字。

\item (我的作弊方法)由於$%
0<b<9$,因此$b$的公因數不%
可能有$19$,所以由於$a$%
是一整數,代表$\frac{10^{n}-2}{19}$%
必須是一整數。也就%
是我們要用方法去找%
到一個$n$讓$10^{n}-2$有最大質%
因數$19$,我能想到的方%
法只能用電腦一個一%
個去代入$n$然後用電腦%
求餘數mod找出第一個找%
到$\frac{10^{n}-2}{19}$餘數為零的$n$%
。\newline
電腦程式如下\newline
\texttt{n = 0\newline
while (10**n-2) \% 19 != 0:\newline
\quad n=n+1\newline
print n}\newline
找到最小的$n$為$17$,與%
前面方法相同。因此$a$%
為$5263157894736842$,$b$為$1$,所以%
此數為$52631578947368421$。驗算過%
後發現我們必須在$a$首%
位加個零,這樣尾數%
提到前面才會是$%
2a=105263157894736842$。因此滿足題%
目最小的此位數為$%
052631578947368421$。
\end{enumerate}

\bigskip

Remark: 這裡也提出一個問%
題,我在算$\frac{1}{19}$的時%
候遇到一個問題,就%
是電腦無法顯示足夠%
的位數,計算機更不%
用說,因此我是查網%
路才知道$\frac{1}{19}$的結果%
與他有幾位旋環小數%
。這是電腦中很常見%
的問題。其實電腦應%
該有指令可以顯示出$64$%
位元的浮點數$float$,因%
此我還要去查一下。

\end{document}
