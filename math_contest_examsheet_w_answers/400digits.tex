
\documentclass{article}
%%%%%%%%%%%%%%%%%%%%%%%%%%%%%%%%%%%%%%%%%%%%%%%%%%%%%%%%%%%%%%%%%%%%%%%%%%%%%%%%%%%%%%%%%%%%%%%%%%%%%%%%%%%%%%%%%%%%%%%%%%%%%%%%%%%%%%%%%%%%%%%%%%%%%%%%%%%%%%%%%%%%%%%%%%%%%%%%%%%%%%%%%%%%%%%%%%%%%%%%%%%%%%%%%%%%%%%%%%%%%%%%%%%%%%%%%%%%%%%%%%%%%%%%%%%%
%TCIDATA{OutputFilter=LATEX.DLL}
%TCIDATA{Version=5.00.0.2606}
%TCIDATA{<META NAME="SaveForMode" CONTENT="1">}
%TCIDATA{BibliographyScheme=Manual}
%TCIDATA{Created=Sunday, May 17, 2015 12:37:39}
%TCIDATA{LastRevised=Saturday, May 20, 2017 11:30:28}
%TCIDATA{<META NAME="GraphicsSave" CONTENT="32">}
%TCIDATA{<META NAME="DocumentShell" CONTENT="Standard LaTeX\Blank - Standard LaTeX Article">}
%TCIDATA{CSTFile=40 LaTeX article.cst}

\newtheorem{theorem}{Theorem}
\newtheorem{acknowledgement}[theorem]{Acknowledgement}
\newtheorem{algorithm}[theorem]{Algorithm}
\newtheorem{axiom}[theorem]{Axiom}
\newtheorem{case}[theorem]{Case}
\newtheorem{claim}[theorem]{Claim}
\newtheorem{conclusion}[theorem]{Conclusion}
\newtheorem{condition}[theorem]{Condition}
\newtheorem{conjecture}[theorem]{Conjecture}
\newtheorem{corollary}[theorem]{Corollary}
\newtheorem{criterion}[theorem]{Criterion}
\newtheorem{definition}[theorem]{Definition}
\newtheorem{example}[theorem]{Example}
\newtheorem{exercise}[theorem]{Exercise}
\newtheorem{lemma}[theorem]{Lemma}
\newtheorem{notation}[theorem]{Notation}
\newtheorem{problem}[theorem]{Problem}
\newtheorem{proposition}[theorem]{Proposition}
\newtheorem{remark}[theorem]{Remark}
\newtheorem{solution}[theorem]{Solution}
\newtheorem{summary}[theorem]{Summary}
\newenvironment{proof}[1][Proof]{\noindent\textbf{#1.} }{\ \rule{0.5em}{0.5em}}
\input{tcilatex}

\begin{document}


%TCIMACRO{%
%\TeXButton{watermark}{\AddToShipoutPictureBG{  \AtTextLowerLeft{  \resizebox{\textwidth}{!}{\rotatebox{60}{{\color{cyan} whymranderson \U{88fd}\U{7248}}}}
%}}}}%
%BeginExpansion
\AddToShipoutPictureBG{  \AtTextLowerLeft{  \resizebox{\textwidth}{!}{\rotatebox{60}{{\color{cyan} whymranderson \U{88fd}\U{7248}}}}
}}%
%EndExpansion

\begin{description}
\item[2007 \U{570b}\U{969b}\U{5c0f}\U{5b78}\U{6578}\U{5b78}\U{81ea}\U{7136}%
\U{79d1}\U{5b78}\U{5967}\U{6797}\U{5339}\U{4e9e}ISMO-\U{6578}\U{5b78}\U{57fa}%
\U{672c}\U{984c}\U{7b2c}\U{4e00}\U{984c}] The 400-digit number 
\[
12345678901234567890\ldots 890
\]
is given.

\item Step 1: Cross out all the digits in odd-numbered places.

\item Step 2: Cross out all the digits in odd-numbered places of the
remaining digits.

\item \ldots

\item Continue until no digits remain. What is the last digit to be crossed
out?
\end{description}

%\documentclass[a4paper]{article}
%\usepackage[inner=0.5 in,outer=0.5in,top=0.7in,bottom=1in]{geometry}
%\usepackage{fontspec}
%\usepackage{xeCJK}
%\setmainfont{Times New Roman}
%\setsansfont{Verdana}
%\setmonofont{Courier New}                    % tt
%\setCJKmainfont{微軟正黑體}
%
%\begin{document}



\begin{tabular}[b]{c | c c c c c c c c c c c c c c c c c c c c }
position&1st&2nd&3rd&&&&&&&&&&13th\\ \hline
total num of digits: 400&1&2&3&4&5&6&7&8&9&0&1&2&3&4&5&6&7&8&9&0\\
200& &2&&4&&6&&8&&0&&2&&4&&6&&8&&0\\
100& &&&4&&&&8&&&&2&&&&6&&&&0\\
50& &&&&&&&8&&&&&&&&6&&&&\\
25& &&&&&&&&&&&&&&&6&&&&\\
12\\
6\\
3\\
1\\
\end{tabular}



%\end{document}\bigskip

After the first operation, 200 digits will remain and the positions of these
remains are $2\times N$, where $N$ is an interger. After the second
operation, 100 digits will remain and the position of them are $2^{2}\times
N $. 50 digits remain after the third operation and positions of them are $%
2^{3}\times N$. So in the last operation only 1 digit remains and the
position of this digit is $2^{8}\times N$. But the position has to been
smaller than 400, so $N$ can only be 1. So the position is $2^{8}\times
1=256 $. Because the digit sequence has a repetive period of 10 digits. 256
position is $25\times 10+6$, so it is the 6th position in the sequence,
which has a number 6. QED

\end{document}
