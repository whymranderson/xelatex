
\documentclass[a5paper,twocolumn]{article}
\usepackage{fontspec}
\usepackage{xeCJK}
\setmainfont{Times New Roman}
\setsansfont{Verdana}
\setmonofont{Courier New}
\setCJKmainfont{微軟正黑體}

%TCIDATA{OutputFilter=LATEX.DLL}
%TCIDATA{Version=5.00.0.2606}
%TCIDATA{<META NAME="SaveForMode" CONTENT="1">}
%TCIDATA{BibliographyScheme=Manual}
%TCIDATA{Created=Tuesday, December 16, 2014 11:56:54}
%TCIDATA{LastRevised=Friday, December 19, 2014 11:09:43}
%TCIDATA{<META NAME="GraphicsSave" CONTENT="32">}
%TCIDATA{<META NAME="DocumentShell" CONTENT="Standard LaTeX\Blank - Standard LaTeX Article">}
%TCIDATA{CSTFile=40 LaTeX article.cst}

\newtheorem{theorem}{Theorem}
\newtheorem{acknowledgement}[theorem]{Acknowledgement}
\newtheorem{algorithm}[theorem]{Algorithm}
\newtheorem{axiom}[theorem]{Axiom}
\newtheorem{case}[theorem]{Case}
\newtheorem{claim}[theorem]{Claim}
\newtheorem{conclusion}[theorem]{Conclusion}
\newtheorem{condition}[theorem]{Condition}
\newtheorem{conjecture}[theorem]{Conjecture}
\newtheorem{corollary}[theorem]{Corollary}
\newtheorem{criterion}[theorem]{Criterion}
\newtheorem{definition}[theorem]{Definition}
\newtheorem{example}[theorem]{Example}
\newtheorem{exercise}[theorem]{Exercise}
\newtheorem{lemma}[theorem]{Lemma}
\newtheorem{notation}[theorem]{Notation}
\newtheorem{problem}[theorem]{Problem}
\newtheorem{proposition}[theorem]{Proposition}
\newtheorem{remark}[theorem]{Remark}
\newtheorem{solution}[theorem]{Solution}
\newtheorem{summary}[theorem]{Summary}
\newenvironment{proof}[1][Proof]{\noindent\textbf{#1.} }{\ \rule{0.5em}{0.5em}}
\input{tcilatex}

\begin{document}


\begin{description}
\item[\textbf{阿} (a)] prefix;kinship term to indicate familiarity. 
\underline{阿}。例:$\underset{\text{a-b\u{u}}}{%
\text{阿母}}$ mom;$\underset{\text{a-pah}}{\text{阿%
爸}}$ dad;$\underset{\text{a-b\u{u}}}{\text{阿母}}$
mom;$\underset{\text{a-pah}}{\text{阿爸}}$ dad。

\item[\textbf{阿衿 }(a-k\`{\i}m)] aunt(maternal uncle's wife). 
\underline{舅媽}。例:$\underset{\text{gun
a-ku kap a-kim teh tso sing-li}}{\text{阮阿舅佮%
阿妗咧做生理}}$ My uncle and aunt are in
business.。

\item[抑是 (ah-si; iah-si; a-si; iah-shi)] or; maybe; either. 
\underline{或是};\underline{或者}。Ex: $%
\underset{\text{ho ah-si m ho}}{\text{好抑是毋%
好}}$; $\underset{\text{li ho}}{\text{你好.}}$
\item[\textbf{阿} (a)] prefix;kinship term to indicate familiarity. 
\underline{阿}。例:$\underset{\text{a-b\u{u}}}{%
\text{阿母}}$ mom;$\underset{\text{a-pah}}{\text{阿%
爸}}$ dad;$\underset{\text{a-b\u{u}}}{\text{阿母}}$
mom;$\underset{\text{a-pah}}{\text{阿爸}}$ dad。

\item[\textbf{阿衿 }(a-k\`{\i}m)] aunt(maternal uncle's wife). 
\underline{舅媽}。例:$\underset{\text{gun
a-ku kap a-kim teh tso sing-li}}{\text{阮阿舅佮%
阿妗咧做生理}}$ My uncle and aunt are in
business.。

\item[抑是 (ah-si; iah-si; a-si; iah-shi)] or; maybe; either. 
\underline{或是};\underline{或者}。Ex: $%
\underset{\text{ho ah-si m ho}}{\text{好抑是毋%
好}}$; $\underset{\text{li ho}}{\text{你好.}}$

\item[\textbf{阿} (a)] prefix;kinship term to indicate familiarity. 
\underline{阿}。例:$\underset{\text{a-b\u{u}}}{%
\text{阿母}}$ mom;$\underset{\text{a-pah}}{\text{阿%
爸}}$ dad;$\underset{\text{a-b\u{u}}}{\text{阿母}}$
mom;$\underset{\text{a-pah}}{\text{阿爸}}$ dad。

\item[\textbf{阿衿 }(a-k\`{\i}m)] aunt(maternal uncle's wife). 
\underline{舅媽}。例:$\underset{\text{gun
a-ku kap a-kim teh tso sing-li}}{\text{阮阿舅佮%
阿妗咧做生理}}$ My uncle and aunt are in
business.。

\item[抑是 (ah-si; iah-si; a-si; iah-shi)] or; maybe; either. 
\underline{或是};\underline{或者}。Ex: $%
\underset{\text{ho ah-si m ho}}{\text{好抑是毋%
好}}$; $\underset{\text{li ho}}{\text{你好.}}$

\item[\textbf{阿} (a)] prefix;kinship term to indicate familiarity. 
\underline{阿}。例:$\underset{\text{a-b\u{u}}}{%
\text{阿母}}$ mom;$\underset{\text{a-pah}}{\text{阿%
爸}}$ dad;$\underset{\text{a-b\u{u}}}{\text{阿母}}$
mom;$\underset{\text{a-pah}}{\text{阿爸}}$ dad。

\item[\textbf{阿衿 }(a-k\`{\i}m)] aunt(maternal uncle's wife). 
\underline{舅媽}。例:$\underset{\text{gun
a-ku kap a-kim teh tso sing-li}}{\text{阮阿舅佮%
阿妗咧做生理}}$ My uncle and aunt are in
business.。

\item[抑是 (ah-si; iah-si; a-si; iah-shi)] or; maybe; either. 
\underline{或是};\underline{或者}。Ex: $%
\underset{\text{ho ah-si m ho}}{\text{好抑是毋%
好}}$; $\underset{\text{li ho}}{\text{你好.}}$

\end{description}

\end{document}
