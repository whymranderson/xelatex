
\documentclass{article}
\usepackage{fontspec}
\usepackage{xeCJK}
\setmainfont{Times New Roman}
\setsansfont{Verdana}
\setmonofont{Courier New}
\setCJKmainfont{微軟正黑體}

\begin{document}

\begin{center}
\begin{tabular}{| l | l | l | l |}
  %\multicolumn{2}{|c|}{RGCordTransV11.py} \\
   \hline
   & 優點 & 缺點 & 適合用途 \\%\\ \cline{2-2}
   \hline
   B法  & 模擬時間可以很長  & 需知道力矩項 &  \\
   \hline
   A法 & 計算量小,快 & 模擬時間短 & 電腦遊戲引擎 \\
          & 程式容易          & 誤差較大    & 教學軟體 \\
   \hline
   C法 & 不須知道力矩項  & 誤差最大 & 陀螺儀感測器的姿態演算解算\\
   \hline
   
\end{tabular}
\end{center}



\end{document}