
\documentclass[12pt,a4paper]{article}
%%%%%%%%%%%%%%%%%%%%%%%%%%%%%%%%%%%%%%%%%%%%%%%%%%%%%%%%%%%%%%%%%%%%%%%%%%%%%%%%%%%%%%%%%%%%%%%%%%%%%%%%%%%%%%%%%%%%%%%%%%%%%%%%%%%%%%%%%%%%%%%%%%%%%%%%%%%%%%%%%%%%%%%%%%%%%%%%%%%%%%%%%%%%%%%%%%%%%%%%%%%%%%%%%%%%%%%%%%%%%%%%%%%%%%%%%%%%%%%%%%%%%%%%%%%%
\usepackage{amsmath}
\usepackage{fontspec}
\usepackage{xeCJK}
\setmainfont[Mapping=tex-text]{Times New Roman} % rm
\setsansfont[Mapping=tex-text]{Arial}           % sf
\setmonofont{Courier New}                       % tt
\setCJKmainfont{微軟正黑體} 
\usepackage[left=0.95in,right=0.95in,top=2cm,bottom=2.54cm]{geometry}
\usepackage{unicode-math}
\usepackage{graphicx}
\usepackage[hidelinks]{hyperref}

\setcounter{MaxMatrixCols}{10}
%TCIDATA{OutputFilter=LATEX.DLL}
%TCIDATA{Version=5.00.0.2606}
%TCIDATA{<META NAME="SaveForMode" CONTENT="1">}
%TCIDATA{BibliographyScheme=Manual}
%TCIDATA{Created=Monday, January 13, 2014 11:43:31}
%TCIDATA{LastRevised=Saturday, May 03, 2014 15:01:01}
%TCIDATA{<META NAME="GraphicsSave" CONTENT="32">}
%TCIDATA{<META NAME="DocumentShell" CONTENT="International\Traditional Chinese Article">}
%TCIDATA{CSTFile=Traditional Chinese.cst}

\newtheorem{theorem}{Theorem}
\newtheorem{acknowledgement}[theorem]{Acknowledgement}
\newtheorem{algorithm}[theorem]{Algorithm}
\newtheorem{axiom}[theorem]{Axiom}
\newtheorem{case}[theorem]{Case}
\newtheorem{claim}[theorem]{Claim}
\newtheorem{conclusion}[theorem]{Conclusion}
\newtheorem{condition}[theorem]{Condition}
\newtheorem{conjecture}[theorem]{Conjecture}
\newtheorem{corollary}[theorem]{Corollary}
\newtheorem{criterion}[theorem]{Criterion}
\newtheorem{definition}[theorem]{Definition}
\newtheorem{example}[theorem]{Example}
\newtheorem{exercise}[theorem]{Exercise}
\newtheorem{lemma}[theorem]{Lemma}
\newtheorem{notation}[theorem]{Notation}
\newtheorem{problem}[theorem]{Problem}
\newtheorem{proposition}[theorem]{Proposition}
\newtheorem{remark}[theorem]{Remark}
\newtheorem{solution}[theorem]{Solution}
\newtheorem{summary}[theorem]{Summary}
\newenvironment{proof}[1][Proof]{\noindent\textbf{#1.} }{\ \rule{0.5em}{0.5em}}
\input{tcilatex}

\begin{document}


\begin{center}
\textbf{剛體轉動尤拉方程%
(rigid body rotation Euler's equation)的詳細推%
導及其非Lagrangian的快速數%
值模擬}
\end{center}

我們知道剛體轉動中%
從牛頓定律出發而得%
到的尤拉運動方程給%
出的是角速度在body座標%
的分量,一般力學的%
書上說明到角速度的%
尤拉運動方程,就會%
轉而求諸euler angle來得到Lagrangian%
,接著如果是陀螺的%
例子就以elliptical integral解出解%
析解,然後就可以模%
擬其運動。這樣做可%
以得到最完整的解析%
解。不過若只想要模%
擬其運動,並不是那%
麼在意有沒有得到解%
析解,我這邊分享一%
個只要得到body分量角速%
度的尤拉運動方程加%
上給定初始值,就可%
以數值模擬陀螺轉動%
,雖然無法得到解析%
解,但以此方法寫出%
的python模擬程式相較於%
Lagrangian方法簡單許多。此%
模擬可以廣泛的運用%
於任何剛體運動的尤%
拉方程,並且此處所%
涉及的body角速度的理解%
還可應用上其他用到%
轉動矩陣的相關領域%
如電腦視覺及物理模%
擬引擎。這邊我們呈%
現以此方法數值模擬%
完整的陀螺運動。

這裡以陀螺為例子,%
將陀螺的運動分解成%
為很多個t到t+dt時間的微%
小轉動,此作法有了%
兩個目的:

\begin{enumerate}
\item 證明Euler equation中的陀螺角%
速度在body的分量就是t到%
t+dt時間陀螺的body frame在space frame%
的轉動角速度,以此%
可建立space跟body frames間的主%
動與被動轉動矩陣。

\item 在近似$\vec{\omega}\cdot dt$的角位%
移時,t到t+dt時間的微小%
座標運動將可以非常%
準確地以在t+dt時間的body%
角速度來近似,即以$%
\vec{\omega}_{b}\left( t+dt\right) $來建立dt時%
間內的轉動矩陣$\footnote{%
The path order intergral of $\omega _{b}$ from time t to t+dt. This will be
discussed more in the text.}$,這裡將展示%
以此近似加上用簡單%
的四階Runge Kutta就可以解出%
很精確的陀螺運動。(%
目前一般陀螺運動都%
需要蠻高階的數值計%
算來避免其中一些守%
恆的運動量退化。此%
方法將以非常簡單的%
四階Runge Kutta來得到非常理%
想的結果。)
\end{enumerate}

\begin{summary}
一般認為解出陀螺的%
角速度在body上的分量並%
無太大用處,這裡我%
們證明了只要對角速%
度及微小轉動矩陣的%
線性代數性質瞭解正%
確透徹,依然能夠用%
非euler angle或非Lagrangian的方法模%
擬三維空間陀螺的運%
動。
\end{summary}

首先我們先討論向量%
變化量在不同觀測座%
標中的關係。由於當%
我實際在解這問題時%
我發現Goldstein classical mechanics書中描%
述的還不是那麼完全%
清楚,因此這邊寫上%
我認為可以補充書上%
的推導證明。

\begin{equation}
\left( \frac{d\vec{L}}{dt}\right) _{s}=\left( \frac{d\vec{L}}{dt}\right)
_{b}+\vec{\omega}\times \vec{L}
\end{equation}

此公式如何而來?此%
公式為一隨時間變動%
的向量在恆定座標與%
非恆定座標(此例為轉%
動中座標)之間線性變%
換的結果。

\begin{figure}[th]
\caption{}
\begin{center}
\fbox{\includegraphics{cordtrans.JPG}}
\end{center}
\end{figure}
\bigskip 

首先考慮一恆定座標%
S(space),一轉動座標b(body),%
為了方便討論座標軸%
的主被動性與座標轉%
換的左右手法則,我%
們這邊方便的先假設$%
\hat{S}_{x}, \hat{b}_{x}$兩軸重合,%
因此圖中顯示了body frame沿%
著$+\hat{S}_{x}$遵守右手定則%
逆時針轉了$\Omega $角度,%
依右手定則此角位移%
向量$\hat{\Omega}$會在$+\hat{S}_{x}$方向%
。但是接下來的推導%
以及所有公式都適用%
任意的座標旋轉,這%
邊是為了方便討論矩%
陣的主動被動的方向%
性,以及在之後的推%
導方便我們追蹤正負%
號以及矩陣主動被動%
意義的改變,因此在%
圖中做了一個方便我%
們思考的情形。另外%
,大部分書上在討論%
座標轉換時有時候給%
的公式是遵守左手定%
則,但這與物理定律%
所採納的右手定則相%
反,因此這邊我寫下%
完整的右手定則的推%
導,希望之後的人不%
需要像我一樣花了大%
半時間在轉換不同公%
式間左手右手定則帶%
來的正負號的改變。

\bigskip 依照圖??所示,我們%
可以寫下$\vec{A}$向量在S,b座%
標間的關係%
\begin{equation*}
\left( \vec{A}\right) _{b}=\underset{\text{passive, r.h.}}{\Omega }\left( 
\vec{A}\right) _{s}
\end{equation*}%
其中$\Omega $是s frame到b frame的座%
標轉換矩陣,因為是%
轉換座標軸,因此矩%
陣取被動含意,並且%
我們採用右手定則,%
因此逆時針方向為正%
方向。接下來只要有%
用到矩陣的運算我都%
會標明主被動及左右%
手(r.h. right-hand or l.h. left-hand),這對接%
下來的推倒很重要。

若我們考慮$\Omega $的角度%
很小$\Omega \rightarrow d\Omega $(infinitesimal rotation),%
則$d\Omega $矩陣與unity matrix相去%
不遠,可以寫成$1$(unity matrix) +$%
\epsilon $(infinitesimal matrix),$\epsilon $具有%
antisymmetric matrix的特性[??],帶入%
上式%
\begin{equation*}
\left( \vec{A}\right) _{b}=\underset{\text{passive, r.h.}}{\left( 1+\epsilon
\right) }\left( \vec{A}\right) _{s}
\end{equation*}%
infinitesimal matrix有個特性,很容%
易自行驗證,%
\begin{equation*}
\underset{\text{r.h., passive or active}}{\epsilon }=\left[ 
\begin{array}{ccc}
0 & \epsilon _{3}\geq 0 & -\epsilon _{2}\leq 0 \\ 
-\epsilon _{3} & 0 & \epsilon _{1}\geq 0 \\ 
\epsilon _{2} & -\epsilon _{1} & 0%
\end{array}%
\right] \text{, }\underset{\text{l.h., passive or active}}{\epsilon }=\left[ 
\begin{array}{ccc}
0 & -\epsilon _{3}\leq 0 & \epsilon _{2}\geq 0 \\ 
+\epsilon _{3} & 0 & -\epsilon _{1}\leq 0 \\ 
-\epsilon _{2} & \epsilon _{1} & 0%
\end{array}%
\right]
\end{equation*}

\bigskip 現在我們考慮$\vec{A}$是$+%
\hat{b}_{y}$軸的狀況,並且考%
慮矩陣$\left( 1+\epsilon \right) $的主動%
特性,也就是主動轉%
向量,這樣的話轉動%
方向會與原本的方向%
相反,變左手定則,%
我們會得到%
\begin{equation*}
\left( \hat{S}_{y}\right) _{s}=\underset{\text{active, l.h.}}{\left(
1+\epsilon \right) }\times \left( \hat{b}_{y}\right) _{s}
\end{equation*}%
整理一下%
\begin{equation*}
\left( \hat{b}_{y}\right) _{s}=\underset{\text{active, r.h.}}{\underbrace{%
\left[ \left( 1+\epsilon \right) \right] ^{T}}}\times \left( \hat{S}%
_{y}\right) _{s}=\underset{\text{active, l.h.}}{\left( 1-\epsilon \right) }%
\times \left( \hat{S}_{y}\right) _{s}
\end{equation*}%
代入上面r.h. $\epsilon $的公式(%
因$\epsilon $還是原本的矩陣)%
,整理一下%
\begin{equation*}
\left( \hat{b}_{y}\right) _{s}-\left( \hat{S}_{y}\right) _{s}=-\left[ 
\begin{array}{ccc}
0 & \epsilon _{3}\geq 0 & -\epsilon _{2}\leq 0 \\ 
-\epsilon _{3} & 0 & \epsilon _{1}\geq 0 \\ 
\epsilon _{2} & -\epsilon _{1} & 0%
\end{array}%
\right] \times \left( \hat{S}_{y}\right) _{s}
\end{equation*}%
利用向量外積,上式%
也可寫成%
\begin{equation*}
\left( \hat{b}_{y}\right) _{s}-\left( \hat{S}_{y}\right) _{s}=\left( \vec{%
\epsilon}\right) _{s}\times \left( \hat{S}_{y}\right) _{s}
\end{equation*}%
其中$\vec{\epsilon}=\left[ 
\begin{array}{c}
\epsilon _{1} \\ 
\epsilon _{2} \\ 
\epsilon _{3}%
\end{array}%
\right] _{s}$為一向量,在S frame中%
的分量為$\epsilon _{1}$,$\epsilon _{2}$%
,$\epsilon _{3}$。

現在我們將上式跟微%
小轉動公式Rodrigues rotation formula比%
較%
\begin{equation*}
\vec{r}^{\prime }-\vec{r}=d\vec{\Omega}\times \vec{r}
\end{equation*}%
$d\vec{\Omega}$是r到r'的r.h.角位移%
向量\thinspace ,因此我們得%
到$\vec{\epsilon}=d\vec{\Omega}$,$\vec{\epsilon}$就%
是s frame到b frame的角位移向%
量(follow r.h. rule)%
\begin{equation*}
\left( \hat{b}_{y}\right) _{s}-\left( \hat{S}_{y}\right) _{s}=\left( d\vec{%
\Omega}\right) _{s}\times \left( \hat{S}_{y}\right) _{s}
\end{equation*}%
這一點很重要,因為%
我們將證明此$\left( d\vec{\Omega}\right)
_{s}$跟接下來我們要推導%
的尤拉公式中的貼體%
角速度$\vec{\omega}$有直接相%
關性,並且以此來做%
我們模擬剛體轉動的%
基礎。

\bigskip

以上的討論是考慮$\vec{A}$%
向量不隨時間變動的%
情況,接下來我們必%
須討論$\vec{A}$以及b frame皆隨%
時間變動的狀況。

\bigskip 
\begin{figure}[th]
\caption{Rate change of a vector observed in a inertial and non-inertial
frame.}
\begin{center}
\fbox{\includegraphics{vecratechange.JPG}}
\end{center}
\label{ratevecfig}
\end{figure}

\bigskip 在時間t時我們令S與b
frame重合,過了dt時間原%
本的$\vec{A}$向量有了一改%
變量$d\vec{A}$,並且b frame依右%
手定則轉動了一微小%
角度(infinitisemal rotation),在此前%
提下,t時間的向量$\vec{A}$%
符合%
\begin{equation*}
\left( \vec{A}\left( t\right) \right) _{s}=\left( \vec{A}\left( t\right)
\right) _{b}
\end{equation*}%
接著,在t+dt時間$\vec{A}+d\vec{A}$%
向量在s與b frame間的關係%
為%
\begin{equation*}
\left( \vec{A}+d\vec{A}\right) _{b}=\underset{\text{passive, r.h.}}{\Omega }%
\left( \vec{A}+d\vec{A}\right) _{s}
\end{equation*}%
$\Omega $為s, b frame轉動矩陣(passive r.h.)%
,此$\Omega $矩陣與上一段A%
不變動的情況的$\Omega $矩%
陣完全相同,我們取s%
到b frame的轉動為微小量%
,$\Omega \rightarrow d\Omega $,上式依之%
前所述的原理可寫成%
\begin{equation*}
\left( \vec{A}+d\vec{A}\right) _{b}=\underset{\text{passive, r.h.}}{\left(
1+\epsilon \right) }\left( \vec{A}+d\vec{A}\right) _{s}
\end{equation*}%
要強調這邊的$\epsilon $矩陣%
跟之前上一段的$\epsilon $矩%
陣是完全相同的,展%
開上式%
\begin{equation*}
\left( \vec{A}\right) _{b}+\left( d\vec{A}\right) _{b}=\left( \vec{A}\right)
_{s}+\left( d\vec{A}\right) _{s}+\epsilon \left( \vec{A}\right)
_{s}+\epsilon \left( d\vec{A}\right) _{s}
\end{equation*}%
利用之前得到的$\left( \vec{A}%
\left( t\right) \right) _{s}=\left( \vec{A}\left( t\right) \right) _{b}$%
,以及忽略高階項$\epsilon
\left( d\vec{A}\right) _{s}$,重新整理成%
\begin{equation*}
\left( d\vec{A}\right) _{s}=\left( d\vec{A}\right) _{b}-\underset{\text{r.h.}%
}{\epsilon }\left( \vec{A}\right) _{s}
\end{equation*}%
依之前所述原理代入r.h. 
$\epsilon $的公式,並且利用%
向量外積%
\begin{eqnarray*}
\left( d\vec{A}\right) _{s} &=&\left( d\vec{A}\right) _{b}-\left[ 
\begin{array}{ccc}
0 & \epsilon _{3}\geq 0 & -\epsilon _{2}\leq 0 \\ 
-\epsilon _{3} & 0 & \epsilon _{1}\geq 0 \\ 
\epsilon _{2} & -\epsilon _{1} & 0%
\end{array}%
\right] \left( \vec{A}\right) _{s} \\
&=&\left( d\vec{A}\right) _{b}-\left( \vec{A}\right) _{s}\times \left( d\vec{%
\Omega}\right) _{s} \\
&=&\left( d\vec{A}\right) _{b}+\left( d\vec{\Omega}\right) _{s}\times \left( 
\vec{A}\right) _{s}
\end{eqnarray*}%
因為這裡的$\epsilon $矩陣與%
上一段的$\epsilon $矩陣是一%
樣的,因此我們也可%
以用上之前轉動公式%
所推導的微小轉動矩%
陣$\epsilon $所對應的轉動向%
量$\left( d\vec{\Omega}\right) $,這樣我%
們就得到了rate of change of a vector in
rotating frame公式%
\begin{equation}
\left( d\vec{A}\right) _{s}=\left( d\vec{A}\right) _{b}+\left( d\vec{\Omega}%
\right) _{s}\times \left( \vec{A}\right) _{s}
\end{equation}%
這邊要強調,因為這%
裡的$\epsilon $矩陣與上一段%
的$\epsilon $矩陣是一樣的,%
所以證明了$d\vec{\Omega}$所對%
應的向量就是s frame轉到b
frame的角位移向量(r.h.),這%
樣強調的目的是,我%
們會以此特性模擬剛%
體轉動。另外要注意%
的是$\vec{A}$與$d\vec{\Omega}$都是在t%
時間的s frame的投影量。%
這邊值得一提的是,%
傳統公式大多寫成%
\begin{equation*}
\left( d\vec{A}\right) _{s}=\left( d\vec{A}\right) _{b}+\left( d\vec{\Omega}%
\right) _{b}\times \left( \vec{A}\right) _{b}
\end{equation*}%
因為我們知道$\left( \vec{A}\left(
t\right) \right) _{s}=\left( \vec{A}\left( t\right) \right) _{b}$,%
若考慮$\left( \vec{A}\left( t\right) \right) _{b}$那%
我們的微小矩陣是作%
用在body frame的$\vec{A}$上面,因%
此在利用外積特性來%
指定轉動向量時我們%
也考慮$\left( d\vec{\Omega}\right) _{b}$在b frame%
的投影$, $因此事實上$%
\vec{A}$與$d\vec{\Omega}$取s或b frame分量%
都是可以的,只要矩%
陣過後出來的結果是%
一樣的就可以。\footnote{這%
也是為什麼Goldstein也說明%
沿著space跟body取分量都是%
可以的。Page ??, version 2.}

\bigskip 上式取微分即得到%
一般常見的形式%
\begin{equation}
\left( \frac{d\vec{A}}{dt}\right) _{s}=\left( \frac{d\vec{A}}{dt}\right)
_{b}+\left( \vec{\omega}\right) _{s}\times \left( \vec{A}\right) _{s}
\label{rateofchange}
\end{equation}%
其中$\left( \vec{\omega}\right) _{b}$為s frame到b
frame的瞬時角速度。

嚴謹的定義了$d\vec{\Omega}$後%
,我們接著可以利用%
Calvin Klein parameter來得到原本的%
轉動矩陣(也就是$1+\epsilon $%
矩陣),這邊我們給他%
一個新代號$CK(d\vec{\Omega})$,當%
然,接下來只要是矩%
陣運算我們都會寫上$CK$%
的主被動及左右手性%
質。%
\begin{eqnarray*}
\underset{\text{r.h.}}{CK(d\vec{\Omega})} &=&\left[ 
\begin{array}{ccc}
a^{2}+b^{2}-c^{2}-d^{2} &  &  \\ 
&  &  \\ 
&  & 
\end{array}%
\right] \text{,} \\
\text{with }a &=&\cos \left( \frac{\left\vert d\vec{\Omega}\right\vert }{2}%
\right) \text{, b, c, d = component of }d\hat{\Omega}\cdot \sin \left( \frac{%
\left\vert d\vec{\Omega}\right\vert }{2}\right)
\end{eqnarray*}%
現在,我們一再強調$d%
\vec{\Omega}$所對應的是s frame轉動%
到b frame,因此我們建立%
的$CK(d\vec{\Omega})$矩陣具有以下%
的特性%
\begin{eqnarray*}
\left( \vec{A}\right) _{b} &=&\underset{\text{passive, r.h.}}{CK(d\vec{\Omega%
})}\left( \vec{A}\right) _{s} \\
\left( \hat{S}_{y}\right) _{s} &=&\underset{\text{active, l.h.}}{CK(d\vec{%
\Omega})}\left( \hat{b}_{y}\right) _{s}
\end{eqnarray*}%
或者%
\begin{eqnarray}
\left( \vec{A}\right) _{s} &=&\underset{\text{active, l.h.}}{\underbrace{%
\left[ CK(d\vec{\Omega})\right] ^{T}}}\left( \vec{A}\right) _{b}
\label{frametrans} \\
\left( \hat{b}_{y}\right) _{s} &=&\underset{\text{active, r.h.}}{\underbrace{%
\left[ CK(d\vec{\Omega})\right] ^{T}}}\left( \hat{S}_{y}\right) _{s}
\label{vecrot}
\end{eqnarray}%
若我們知道的是$\vec{\omega}$%
則可帶入$CK(\vec{\omega}\cdot dt)$來得%
到矩陣。以上兩式就%
是模擬或追蹤剛體的body
frame的x,y,z軸轉動的基礎。

\begin{figure}[th]
\caption{How to apply rate-of-change-of-a-vector equation to a real rotation.
}
\begin{center}
\fbox{\includegraphics{szsbtdt.JPG}}
\end{center}
\label{szsbtdtfig}
\end{figure}

在我們進一步討論\ref%
{frametrans}及\ref{vecrot}式前,我們%
必須先說明我們如何%
應用上\ref{rateofchange}式來解剛%
體轉動。我們會把剛%
體轉動分解為很多的%
微小轉動,每一小段%
的微小轉動我們都會%
運用上圖??的原理,現%
在我們需要另外設定%
一個Lab frame\thinspace ,此為真正%
的world frame(觀測者所處在的%
inertial frame)。考慮任意一段%
微小轉動t到t+dt,在t時%
刻時我們將剛體的principle
axes設定為S frame,再將t+dt時%
刻剛體的principle axes設定為b frame%
,這樣代表s frame到b frame就%
是剛體t到t+dt的轉動。將%
\ref{rateofchange}式應用上這一段t%
到t+dt的微小轉動,並且%
考慮$\vec{A}$為剛體角動量$%
\vec{L}$,則我們得到%
\begin{equation*}
\left( \Gamma \right) _{s}=\left( \frac{d\vec{L}}{dt}\right) _{s}=\left( 
\frac{d\vec{L}}{dt}\right) _{b}+\left( \vec{\omega}\right) _{s}\times \left( 
\vec{L}\right) _{s}
\end{equation*}%
這裡第一等號也用上%
牛頓定律。因為b frame是%
沿著body principle axes而取,因此%
\begin{equation*}
\left( \vec{L}\right) _{s}=\left[ 
\begin{array}{ccc}
I_{xx} &  &  \\ 
& I_{yy} &  \\ 
&  & I_{zz}%
\end{array}%
\right] \left( 
\begin{array}{c}
\omega _{x} \\ 
\omega _{y} \\ 
\omega _{z}%
\end{array}%
\right) _{s}
\end{equation*}%
另外注意$\left( \Gamma \right) _{s}$與$\left( 
\vec{\omega}\right) _{s}$是沿著t時刻的%
剛體特徵軸(也就是s frame)%
取的投影,並不是Lab frame%
的投影,這點要特別%
注意,基本上這代表%
,力矩也必須從lab frame轉%
換到t時間的s frame。代入$%
\vec{L}$並展開上上式,我%
們就得到所謂的尤拉%
公式(Euler's equation)%
\begin{eqnarray*}
\Gamma _{x}(t) &=&I_{x}\dot{\omega}_{x}+(I_{z}-I_{y})\omega _{y}\left(
t\right) \omega _{z}\left( t\right)  \\
\Gamma _{y}(t) &=&I_{y}\dot{\omega}_{y}+(I_{x}-I_{z})\omega _{x}\omega _{z}
\\
\Gamma _{z}(t) &=&I_{z}\dot{\omega}_{z}+(I_{x}-I_{y})\omega _{x}\omega _{y}
\end{eqnarray*}%
注意$\vec{\Gamma}$及$\vec{\omega}$的x,y,z分%
量都是沿著t時刻的剛%
體特徵軸s frame取的分量%
,這點必須要強調。%
之後數值模擬的時候%
這點是重要的。

\bigskip 若以

我們將在陀螺的每一%
段t到t+dt的分解運動運用%
每個t時S,b座標重合的%
事實,也就是運用上\ref%
{vectorrateofchange}式。也就是我們%
將持續地改變space座標來%
符合這個條件,但是%
當然我們會跟蹤space座標%
每一個變換的位置來%
最終模擬陀螺運動‧(%
注意\ref{vectorrateofchange}式中$\Omega ,d\Omega $%
都是s到b的轉動矩陣)。

另外,在t時s,b重和情%
況下若我們加上考慮$%
\vec{A}$向量fixed在b座標中不%
變動(如body的xyz軸,在body座%
標中為固定值),並且%
考慮轉動矩陣的主動%
特性,則\ref{finiterotmatrix}式變成%
\begin{equation}
\left( \hat{z}_{s}\right) _{s}=\underset{active}{\left( 1+d\Omega \right) }%
\times \left( \hat{z}_{b}\right) _{s}  \label{baxesrot}
\end{equation}%
首先\ref{baxesrot}式很重要的%
一個特性是,當我們%
應用上轉動矩陣的主%
動特性,從\ref{baxesrot}式可%
以知道轉動矩陣$1+d\Omega $主%
動地把body的z軸$\left( \hat{z}_{b}\right) _{s}$%
轉到space的z軸$\left( \hat{z}_{s}\right) _{s}$%
,這也代表轉動矩陣$%
1+d\Omega $取其主動特性時其%
所屬的轉動矩陣作用%
在body z軸等同於body z 軸在space%
空間的轉動,也就是$%
\frac{d\Omega }{dt}$為body z軸在空間的%
角速度!(在時間為t到%
t+dt的時候),此事實對fixed%
在body上的任意向量都成%
立,所以body x, y軸也成立%
。另一個看法是,因%
為 space 與 body frame在t時重合,%
因此space的被動轉動矩陣%
$d\Omega $ = body的主動轉動矩陣$%
d\Omega $,稍後我們將會應%
用上這一重要的特性%
。這裡的結論是,主%
動特性的$d\Omega $等同於body%
在space空間的微小轉動。

\begin{remark}
Goldstein中$d\mathbf{\Omega }\times \mathbf{G}$中的\textbf{%
G}是在space座標,但因t時s,b%
兩座標重和,用s或body是%
都可以的。這也是為%
什麼goldstein p176(version 2)說明,\ref%
{vectorrateofchange}式中$\vec{A}$或$\vec{b}$向%
量沿著space或沿著body方向%
取分量都是可以的。%
而Goldstein後來用上陀螺後%
,是取body方向的分量沒%
錯。
\end{remark}

\begin{remark}
但我們必須強調,任%
意情況下,角速度$\left( \vec{%
\omega}\right) $在body轉動座標下的%
投影並不是body座標上觀%
察到的角速度!這裡%
我們是有條件的考慮t%
到t+dt時刻的t時刻s,b座標%
重和。
\end{remark}

\begin{remark}
角速度與Lie algebra相關連,%
SO(3) tangent,$\dot{\Omega}^{T}=-\dot{\Omega}$,$e^{\Omega }$%
是orthogonal matrix,wikipedia(orthogonal transformation)上%
有些線索。
\end{remark}

接下來用上陀螺,取%
陀螺的自旋軸為body z軸%
,取$\vec{A}$向量為陀螺角%
動量$\vec{L}$,以及對\ref%
{vectorrateofchange}式取微分即得到

\begin{equation}
\left( \vec{\Gamma}\right) _{s}=\left( \frac{d\vec{L}}{dt}\right)
_{s}=\left( \frac{d\vec{L}}{dt}\right) _{b}+d\dot{\Omega}\times \left( \vec{L%
}\right) _{b}  \label{newton1}
\end{equation}%
其中$\vec{\Gamma}$為Torque,$d\dot{\Omega}$我%
們已經證明是body xyz軸相%
對於space frame的轉動角速度%
,因euler theorem知轉動矩陣$%
d\Omega $可看成一向量,因%
此此向量 $d\Omega $即等於 $\left( 
\vec{\omega}\right) _{b}\cdot dt$,重新整理%
後得%
\begin{equation}
\left( \vec{\Gamma}\right) _{s}=\left( \frac{dI\vec{\omega}}{dt}\right) _{b}+%
\vec{\omega}\times \left( I\vec{\omega}\right) _{b}  \label{newton2}
\end{equation}%
(也可從微小轉動(infinitesimal
rotation)的向量相加特性考%
慮,我們可以考慮$d\Omega $%
的微小轉動是body x y z軸分%
別的微小轉動而組成%
,這代表我們可以寫$%
d\Omega =(\omega _{x}\hat{x}_{b}+\omega _{y}\hat{y}_{b}+\omega _{z}\hat{z}%
_{b})dt$,$d\Omega $即等於 $\left( \vec{\omega}%
\right) _{b}\cdot dt$)。另外在 body axis中$I$%
是 diagonal的,因此$\left( \vec{L}\right)
_{b}=I\times $ 角速度在body的分量$%
\vec{\omega}$,因此$(\vec{L})_{b}=I\times \vec{\omega}$%
。若加上考慮陀螺的%
條件 $I_{x}=I_{y}\neq I_{z}$,Eq.(\ref{newton2})可%
以寫成%
\begin{eqnarray}
\Gamma _{x} &=&I_{x}\dot{\omega}_{x}+(I_{z}-I_{y})\omega _{y}\omega _{z} \\
\Gamma _{y} &=&I_{y}\dot{\omega}_{y}+(I_{x}-I_{z})\omega _{x}\omega _{z} \\
\Gamma _{z} &=&I_{z}\dot{\omega}_{z}=0
\end{eqnarray}%
重新整理得%
\begin{eqnarray}
\dot{\omega}_{x} &=&-\frac{I_{z}-I_{y}}{I_{x}}\omega _{y}\omega _{z}+\frac{%
\Gamma _{x}}{I_{x}} \\
\dot{\omega}_{y} &=&-\frac{I_{x}-I_{z}}{I_{y}}\omega _{x}\omega _{z}+\frac{%
\Gamma _{y}}{I_{y}} \\
\dot{\omega}_{z} &=&0
\end{eqnarray}%
此方程組也可寫成%
\begin{equation}
\frac{d}{dt}\left[ 
\begin{array}{c}
\omega _{x} \\ 
\omega _{y} \\ 
\omega _{z}%
\end{array}%
\right] =\left[ 
\begin{array}{ccc}
0 & -\frac{I_{z}-I_{y}}{I_{x}} & 0 \\ 
-\frac{I_{x}-I_{z}}{I_{y}} & 0 & 0 \\ 
0 & 0 & 0%
\end{array}%
\right] \left[ 
\begin{array}{c}
\omega _{x} \\ 
\omega _{y} \\ 
\omega _{z}%
\end{array}%
\right] +\left[ 
\begin{array}{c}
\frac{\Gamma _{x}}{I_{x}} \\ 
\frac{\Gamma _{y}}{I_{y}} \\ 
\frac{\Gamma _{z}}{I_{z}}%
\end{array}%
\right]
\end{equation}%
以上的微分方程組可%
以用Ruge Kutta求出$\vec{\omega}(t)$,另%
外,此角速度雖然是%
沿著body軸的分量,但我%
們也證明了此角速度%
在任意一段t到t+dt的時間%
中代表了body座標軸在space%
座標上的轉動,接下%
來我們就說明如何以%
這個特性重構出陀螺%
在三維空間的運動。%
\bigskip

\begin{figure}[th]
\caption{Boby軸在每一分段t到t+dt%
的追蹤示意圖}
\begin{center}
\fbox{\includegraphics{top.eps}}
\end{center}
\end{figure}

由之前討論知道,雖%
然$\left( \vec{\omega}(t)\right) _{b}$是角速度%
沿著body軸的分量,並不%
代表是body座標中觀察到%
的角速度,但我們知%
道若只考慮t到t+dt時間,%
在t時間時s與b重和,在%
此條件下我們證明了$%
\left( \vec{\omega}(t)\right) _{b}$在t到t+dt時是body%
軸在space座標的轉動矩陣%
,也就是body軸的轉動速%
度,因此只要給定初%
始條件,接著一步一%
步的算出body軸在space frame中%
的位置,即可得到三%
維陀螺運動。\bigskip

\begin{figure}[th]
\caption{陀螺的初始值設定}
\begin{center}
\fbox{\includegraphics{initialsetup.eps}}
\end{center}
\end{figure}

假設陀螺起始位置已%
知$\left( z_{s}(t_{0})\right) $,此為body軸z%
軸且為單位向量,下%
標s代表的是觀測的frame%
。所以$z_{s}(t_{0})$代表時間%
是t$_{0}$的時候,z軸在space frame%
中的位置向量,$z_{s}(t_{0})$%
與$z_{s}$不一樣,$z_{s}$為恆%
定座標space frame z axis,見圖,%
我們可藉由一轉動矩%
陣$\mathbf{\Omega (orien)}$輕易改變陀%
螺的初始位置$z_{s}(t_{0})=\mathbf{\Omega
(}orien\mathbf{)}\times z_{s}$,orien為轉動向%
量。$z_{0}(t_{1})$則代表時間%
是t$_{1}$的時候z軸在t$_{0}$時%
的座標軸觀察到的位%
置向量,而$z_{0}(t_{0})=z_{1}(t_{1})=\left[ 
\begin{array}{ccc}
0 & 0 & 1%
\end{array}%
\right] $都是該時間上的座%
標軸上的單位軸向量%
。因此我們要得到的%
是$z_{s}(t_{1\symbol{126}N})$,即z軸時間%
上的變化在space中的向量%
值。在我們以數值法%
求得$\left( \vec{\omega}(t_{i})\right) _{b}$之後%
,接著可以用以下方%
法來求得$z_{s}(t_{i})$。首先%
先定義一些符號,這%
裡$\overset{active}{\overbrace{\mathbf{[}t_{i},t_{i-1}\mathbf{]}}}$%
代表t$_{i-1}$時的座標軸到t$%
_{i}$時的座標軸的轉動矩%
陣,取矩陣的主動特%
性,也就是$z_{i-1}(t_{i})=\overset{active}{%
\overbrace{\mathbf{[}t_{i},t_{i-1}\mathbf{]}}}\times z_{i-1}(t_{i-1})=%
\overset{active}{\overbrace{\mathbf{[}t_{i},t_{i-1}\mathbf{]}}}\times \left[ 
\begin{array}{ccc}
0 & 0 & 1%
\end{array}%
\right] $,當上面標示改成%
passive時,$\overset{passive}{\overbrace{\mathbf{[}t_{i},t_{i-1}%
\mathbf{]}}}$時取轉動矩陣的被%
動性質,也就是轉座%
標軸,因此$\overset{passive}{\overbrace{%
\mathbf{[}t_{i-1},t_{i-2}\mathbf{]}}}\times z_{i-1}(t_{i})=z_{i-2}(t_{i})$%
,因此%
\begin{eqnarray}
z_{s}(t_{i}) &=&\overset{passive}{\overbrace{\mathbf{[}t_{0},s\mathbf{]}}}%
\mathbf{\times }\overset{passive}{\overbrace{\mathbf{[}t_{1},t_{0}\mathbf{]}}%
}\cdots \\
&&\times \overset{passive}{\overbrace{\mathbf{[}t_{i-2},t_{i-3}\mathbf{]}}}%
\mathbf{\times }\underset{\text{t}_{i}\text{時的z在t}_{i-2}%
\text{時的座標的觀察值}}{%
\underbrace{\overset{passive}{\overbrace{\mathbf{[}t_{i-1},t_{i-2}\mathbf{]}}%
}\times \underset{\text{t}_{i}\text{時的z在t}_{i-1}\text{%
時的座標的觀察值}}{%
\underbrace{\overset{active}{\overbrace{\mathbf{[}t_{i},t_{i-1}\mathbf{]}}}%
\times \underset{\text{t}_{i-1}\text{時的z在t}_{i-1}\text{%
時的座標的觀察值,%
顯然為}\left[ 
\begin{array}{ccc}
0 & 0 & 1%
\end{array}%
\right] }{\underbrace{z_{i-1}(t_{i-1})}}}}}}  \label{bodytracking}
\end{eqnarray}%
以這方法我們可以求%
得所有$z_{s}(t_{i\text{, i=1}\sim N})$。但%
怎麼知道$\mathbf{[}t_{i},t_{i-1}\mathbf{]}$的%
轉動矩陣?這裡就是%
第二點個重點,我們%
近似t$_{i-1}$到t$_{i}$的座標運%
動為t$_{i-1}$的座標軸以$\vec{\omega%
}(t_{i})$的速度轉動了dt時間%
,因此,若$\mathbf{\Omega }(\vec{A})$代%
表一轉動矩陣\textbf{其轉%
軸在}$\vec{A}$方向且轉動角%
量值為$\left\vert \vec{A}\right\vert $\textbf{,%
則\ref{bodytracking}公式可寫成}%
\begin{equation}
z_{s}(t_{i})=\overset{passive}{\overbrace{\mathbf{\Omega (orien)}}}\mathbf{%
\times }\overset{passive}{\overbrace{\mathbf{\Omega }(\vec{\omega}%
(t_{1})\cdot dt)}}\cdots \overset{passive}{\overbrace{\mathbf{\Omega }(\vec{%
\omega}(t_{i-2})\cdot dt)}}\mathbf{\times }\overset{passive}{\overbrace{%
\mathbf{\Omega }(\vec{\omega}(t_{i-1})\cdot dt)}}\times \overset{active}{%
\overbrace{\mathbf{\Omega }(\vec{\omega}(t_{i})\cdot dt)}}\times
z_{i-1}(t_{i-1})
\end{equation}%
\bigskip

可以看出上面所有passive%
的矩陣的作用只是再%
把坐標軸從body frame轉回到%
space frame。因此若i=1$\sim N$我們%
可以以此求得$z_{s}(t_{i\text{, i=1}\sim
N})$,這樣我們就用上了%
之前求得的$\left( \vec{\omega}(t_{i})\right)
_{b}$。以此同方法可求得%
x與y軸的運動。以上以$%
\omega (t_{i})$來近似t到t+dt的轉動%
事實上包含更深的物%
理含意,我們是不是%
可以用$\omega (t_{i-1})$來做近似%
?以下將作解釋。

由於t到t+dt時的s,b座標重%
合,因此body軸從t到t+dt的%
轉動可以如下近似,%
由於s,b重合,$\Omega _{b}(t)=\Omega _{s}(t)$%
,我們先考慮陀螺沿%
著$\Omega _{b}(t)$轉了$exp(\Omega _{b}(t))$,但%
在t+dt時,$\Omega _{b}(t+dt)\neq \Omega _{b}(t)$,%
這代表從t到t+dt時,轉動%
向量在body座標上有變化%
,也因此我們不能夠%
單只考慮陀螺轉了$\Omega
_{b}(t)$而已,此額外向量%
的變化在t時space座標的%
向量表達式為$\Omega _{b}(t+dt)-\Omega
_{b}(t)=\Omega _{b}(t)+d\Omega _{b}(dt)-\Omega _{b}(t)=d\Omega
_{b}(dt)=d\Omega _{s}(dt)$,也是一個轉%
動向量,所以space空間中%
總共的轉動可以考慮%
成兩步,第一步轉$\Omega
_{s}(t)$,第二步轉$d\Omega _{s}(dt)$,%
寫成轉動矩陣%
\begin{equation}
\exp (\Omega _{b}(t))\times \exp (d\Omega _{b}(dt))=\exp (\Omega
_{b}(t)+d\Omega _{b}(dt))=\exp (\Omega _{b}(t+dt))
\end{equation}%
這代表我們只要考慮%
陀螺從t到t+dt的時候是轉%
了$\Omega _{b}(t+dt)$而不只是$\Omega _{b}(t)$%
,因此考慮$\Omega _{b}(t+dt)$我們%
就更準確的近似了這%
個轉動,以下的Python程%
式模擬會證明,考慮%
了$\Omega _{b}(t+dt)$給出的結果幾%
乎是完美的。

\begin{remark}
這裡要注意torque $\left( \Gamma _{s}\right) $%
雖然是力矩在space座標上%
的觀察值,但因為我%
們推導eq.\ref{newton2}時是利用%
上該方程在t到t+dt時,假%
設 s, b在 t時重合,這代%
表了我們考慮的space frame是%
必須與body frame在t時重和。%
這也代表了當考慮下%
一個$t^{\prime }$到$t^{\prime }+dt$時候,%
方程考慮的space frame會必須%
是在$t^{\prime }$時的body frame,也%
就是space frame必須持續的改%
變,這代表,若要考%
慮任意t到t+dt時刻,那再%
算$\vec{\omega}(t_{i+1})$時候torque應該要%
用$\left( \Gamma \right) _{b}$而不是用$\left(
\Gamma \right) _{s}$,因為我們的假%
設導致space frame必須持續的%
改變來符合eq.\ref{newton2}的假%
設。不過事實上這樣%
的考慮正是我們想要%
的結果,也就是eq.\ref{newton2}%
完全可以寫成在body frame的%
分量%
\begin{equation}
\left( \vec{\Gamma}\right) _{b}=\left( \frac{dI\vec{\omega}}{dt}\right) _{b}+%
\vec{\omega}\times \left( I\vec{\omega}\right) _{b}
\end{equation}%
,正是這樣才使的該%
方程容易求解。另外%
注意$\vec{L}$並不在body z軸的%
方向,這點可以從最%
後的模擬動畫看出來%
。
\end{remark}

\begin{remark}
要陀螺具有Precession and Nutation的%
動作,L/$\Delta L$必須要大,%
如果L小於$\Delta L$,則只會%
有陀螺質量受重力影%
響往下倒下的運動(不%
過這對檢查程式有沒%
有錯誤很有幫助!),理%
想上L至少要大於$\Delta L$,%
最好L大大於$\Delta L$。化成%
數值上的比較:這代%
表%
\begin{equation}
L\gg \Delta L\Rightarrow I\cdot 2\pi f\gg \vec{\Gamma}\Delta t\Rightarrow
I\cdot 2\pi f\gg \vec{r}\times \vec{F}\cdot 1/f\Rightarrow f\gg \sqrt{\frac{%
arm\cdot Mg\cdot \sin (\theta )}{2\pi I\cdot G}}
\end{equation}%
where $\theta $ is gyro's tilt angle and G is moment of inertial geometry
factor. 考慮$\Delta t$的量級大約%
是陀螺轉幾圈的時間%
(characteristic time),量級上約是$\sim
1/f$,若假設arm是10 cm, M = 1kg, g=10 m/s$^{2}$%
, I = 0.5M(0.05)$^{2}$,則f最少要10 Hertz以%
上。因此我們將以這%
些參數比較f = 1, 10, 50 Hertz所給%
出的陀螺運動。
\end{remark}

\begin{remark}
力矩給出的角速度是%
遵守右手定則(counterclockwise),%
所以rotation formula必須使用其%
active counterclockwise sense才能描述座標%
轉動,要小心,因大%
部分書上(如Goldstein)給的公%
式都是active clockwise(follow左手定%
則)(舉例如書上的Caley Klein
parameter rotation matrix),因此差一個%
負號。\bigskip
\end{remark}

以下將上述方法寫成%
python程式,並且畫圖模%
擬其xyz軸運動。

\begin{figure}[th]
\caption{尖點運動}
\begin{center}
\fbox{\includegraphics[scale=0.6]{figure_xy.eps}}
\end{center}
\end{figure}
\bigskip

\begin{figure}[th]
\caption{有環運動}
\begin{center}
\fbox{\includegraphics[scale=0.6]{figure_x.eps}}
\end{center}
\end{figure}

\begin{figure}[th]
\caption{無環運動}
\begin{center}
\fbox{\includegraphics[scale=0.6]{figure_y.eps}}
\end{center}
\end{figure}

\begin{figure}[th]
\caption{等周速運動}
\label{figure_uniform}
\begin{center}
\fbox{\includegraphics[scale=0.6]{figure_uniform.eps}}
\end{center}
\end{figure}

陀螺等周速運動(Figure \ref%
{figure_uniform})的初始值條件如%
何計算呢?等周速的條%
件在Goldstein第二版5-77式給出%
\begin{equation}
Mgl=\dot{\phi}\left( I_{3}\omega _{3}-I_{1}\dot{\phi}\cos \theta _{0}\right)
\end{equation}%
,不過此式是由尤拉%
角(euler angles)給出,但我們%
需要的是anguler velocity along body的初%
始值,因此我們必須%
轉換尤拉角到anguler velocity along body%
,方法如下。上式中$%
\omega _{3}$即為我們的$\left( \omega
_{z}\right) _{b}$,這裡是20 Hz,$\theta _{0}$%
即為我們之前的orien向量%
所定,此模擬中是取45%
度角,由上式可求出%
兩組$\dot{\phi}(t_{0})$。另外尤拉%
角跟anugler velocity along body的關係式%
在Goldstein 4-125式給出%
\begin{eqnarray}
(\omega _{x})_{b} &=&\dot{\phi}\sin \theta \sin \psi +\dot{\theta}\cos \psi
\\
(\omega _{y})_{b} &=&\dot{\phi}\sin \theta \cos \psi -\dot{\theta}\sin \psi
\\
(\omega _{z})_{b} &=&\dot{\phi}\cos \theta +\dot{\psi}
\end{eqnarray}%
知道$\dot{\phi}(t_{0})$、$\theta _{0}$、$(\omega
_{z})_{b}$,我們由第三條求%
出$\dot{\psi}(t_{0})$,再把$\dot{\psi}(t_{0})$%
帶到第一二條後就可%
得到$(\omega _{x}(t_{0}),\omega _{y}(t_{0}))_{b}$,這%
樣我們就得到anguler velocity along body%
的初始值。因為$\dot{\phi}(t_{0})$%
有兩組,因此解出的%
貼體角速度也會有兩%
組,兩組的物理意義%
分別如下,一種情況%
是fast top,這個狀況相當%
於重力的影響遠小於%
總角動量$L$,因此這個%
特別的例子基本上相%
當於忽略重力,而陀%
螺基本上會像一個free top%
一樣進行precession。另一種%
狀況是slow top,也就是上%
面模擬結果中第四種%
的狀況,這裡提供的%
python程式所有情況都可%
以模擬。另外一個特%
殊的情況是在fast top的情%
形下,如果初始值$\theta
_{0}=0$,也就是陀螺z軸的%
起始狀態是垂直於水%
平面的,這樣的話陀%
螺幾乎會像靜止不動%
一樣,我們也叫這情%
況做sleeping top。

\href{https://drive.google.com/file/d/0B96HmLH-SQVmekx0a0RoSVFzWFE/edit?usp=sharing%
}{\underline{\color{blue}\smash{Python code can be found here.}}}

\href{http://tinypic.com/r/10cw9yf/8}{\underline{\color{blue}%
\smash{3D
animation.}}}

This document is prepared with Scientific Workplace 5.0 and typeset with Tex
Live 2013 (Xelatex). Date 4/2/2014

\end{document}
