\documentclass[12pt,twoside]{article}
%\documentclass[12pt,a4paper]{article}
%%%%%%%%%%%%%%%%%%%%%%%%%%%%%%%%%%%%%%%%%%%%%%%%%%%%%%%%%%%%%%%%%%%%%%%%%%%%%%%%%%%%%%%%%%%%%%%%%%%%%%%%%%%%%%%%%%%%%%%%%%%%%%%%%%%%%%%%%%%%%%%%%%%%%%%%%%%%%%%%%%%%%%%%%%%%%%%%%%%%%%%%%%%%%%%%%%%%%%%%%%%%%%%%%%%%%%%%%%%%%%%%%%%%%%%%%%%%%%%%%%%%%%%%%%%%
\usepackage{amsmath}
\usepackage{fontspec}
\usepackage{xeCJK}
\setmainfont{Times New Roman}
\setsansfont{Verdana}
\setmonofont{Courier New}                    % tt
\setCJKmainfont{微軟正黑體}
\setCJKfamilyfont{kai}{標楷體}		% for changing the title font in title.pgf -> have to manually % add {\kai } in the pgf file
\newcommand*{\kai}{\CJKfamily{kai}}
\usepackage[inner=1in,outer=0.6in,top=0.7in,bottom=1in]{geometry}
\usepackage{unicode-math}
\usepackage{graphicx}
\usepackage[hidelinks]{hyperref}
\usepackage{pgf}

\setcounter{MaxMatrixCols}{10}
%TCIDATA{OutputFilter=LATEX.DLL}
%TCIDATA{Version=5.00.0.2606}
%TCIDATA{<META NAME="SaveForMode" CONTENT="1">}
%TCIDATA{BibliographyScheme=Manual}
%TCIDATA{Created=Monday, January 13, 2014 11:43:31}
%TCIDATA{LastRevised=Monday, September 22, 2014 11:18:24}
%TCIDATA{<META NAME="GraphicsSave" CONTENT="32">}
%TCIDATA{<META NAME="DocumentShell" CONTENT="International\Traditional Chinese Article">}
%TCIDATA{CSTFile=Traditional Chinese.cst}

\newtheorem{theorem}{Theorem}
\newtheorem{acknowledgement}[theorem]{Acknowledgement}
\newtheorem{algorithm}[theorem]{Algorithm}
\newtheorem{axiom}[theorem]{Axiom}
\newtheorem{case}[theorem]{Case}
\newtheorem{claim}[theorem]{Claim}
\newtheorem{conclusion}[theorem]{Conclusion}
\newtheorem{condition}[theorem]{Condition}
\newtheorem{conjecture}[theorem]{Conjecture}
\newtheorem{corollary}[theorem]{Corollary}
\newtheorem{criterion}[theorem]{Criterion}
\newtheorem{definition}[theorem]{Definition}
\newtheorem{example}[theorem]{Example}
\newtheorem{exercise}[theorem]{Exercise}
\newtheorem{lemma}[theorem]{Lemma}
\newtheorem{notation}[theorem]{Notation}
\newtheorem{problem}[theorem]{Problem}
\newtheorem{proposition}[theorem]{Proposition}
\newtheorem{remark}[theorem]{Remark}
\newtheorem{solution}[theorem]{Solution}
\newtheorem{summary}[theorem]{Summary}
\newenvironment{proof}[1][Proof]{\noindent\textbf{#1.} }{\ \rule{0.5em}{0.5em}}
\input{tcilatex}

\begin{document}


%% Creator: Matplotlib, PGF backend
%%
%% To include the figure in your LaTeX document, write
%%   \input{<filename>.pgf}
%%
%% Make sure the required packages are loaded in your preamble
%%   \usepackage{pgf}
%%
%% Figures using additional raster images can only be included by \input if
%% they are in the same directory as the main LaTeX file. For loading figures
%% from other directories you can use the `import` package
%%   \usepackage{import}
%% and then include the figures with
%%   \import{<path to file>}{<filename>.pgf}
%%
%% Matplotlib used the following preamble
%%   \usepackage{fontspec}
%%   \setmainfont{Times New Roman}
%%   \setsansfont{Verdana}
%%   \setmonofont{Courier New}
%%
\begingroup%
\makeatletter%
\begin{pgfpicture}%
\pgfpathrectangle{\pgfpointorigin}{\pgfqpoint{1.637500in}{2.000000in}}%
\pgfusepath{use as bounding box}%
\begin{pgfscope}%
\pgfsetbuttcap%
\pgfsetroundjoin%
\definecolor{currentfill}{rgb}{1.000000,1.000000,1.000000}%
\pgfsetfillcolor{currentfill}%
\pgfsetlinewidth{0.000000pt}%
\definecolor{currentstroke}{rgb}{1.000000,1.000000,1.000000}%
\pgfsetstrokecolor{currentstroke}%
\pgfsetdash{}{0pt}%
\pgfpathmoveto{\pgfqpoint{0.000000in}{0.000000in}}%
\pgfpathlineto{\pgfqpoint{1.637500in}{0.000000in}}%
\pgfpathlineto{\pgfqpoint{1.637500in}{2.000000in}}%
\pgfpathlineto{\pgfqpoint{0.000000in}{2.000000in}}%
\pgfpathclose%
\pgfusepath{fill}%
\end{pgfscope}%
\begin{pgfscope}%
\pgfsetbuttcap%
\pgfsetroundjoin%
\definecolor{currentfill}{rgb}{1.000000,1.000000,1.000000}%
\pgfsetfillcolor{currentfill}%
\pgfsetlinewidth{0.000000pt}%
\definecolor{currentstroke}{rgb}{0.000000,0.000000,0.000000}%
\pgfsetstrokecolor{currentstroke}%
\pgfsetstrokeopacity{0.000000}%
\pgfsetdash{}{0pt}%
\pgfpathmoveto{\pgfqpoint{0.000000in}{0.000000in}}%
\pgfpathlineto{\pgfqpoint{1.637500in}{0.000000in}}%
\pgfpathlineto{\pgfqpoint{1.637500in}{2.000000in}}%
\pgfpathlineto{\pgfqpoint{0.000000in}{2.000000in}}%
\pgfpathclose%
\pgfusepath{fill}%
\end{pgfscope}%
\begin{pgfscope}%
\pgfpathrectangle{\pgfqpoint{0.000000in}{0.000000in}}{\pgfqpoint{1.637500in}{2.000000in}} %
\pgfusepath{clip}%
\pgfsetrectcap%
\pgfsetroundjoin%
\pgfsetlinewidth{4.015000pt}%
\definecolor{currentstroke}{rgb}{0.000000,0.000000,0.000000}%
\pgfsetstrokecolor{currentstroke}%
\pgfsetdash{}{0pt}%
\pgfpathmoveto{\pgfqpoint{0.199808in}{0.793543in}}%
\pgfpathlineto{\pgfqpoint{0.198157in}{0.782929in}}%
\pgfpathlineto{\pgfqpoint{0.206076in}{0.767035in}}%
\pgfpathlineto{\pgfqpoint{0.223553in}{0.746161in}}%
\pgfpathlineto{\pgfqpoint{0.250314in}{0.720756in}}%
\pgfpathlineto{\pgfqpoint{0.285813in}{0.691414in}}%
\pgfpathlineto{\pgfqpoint{0.329233in}{0.658861in}}%
\pgfpathlineto{\pgfqpoint{0.379501in}{0.623938in}}%
\pgfpathlineto{\pgfqpoint{0.435311in}{0.587579in}}%
\pgfpathlineto{\pgfqpoint{0.495162in}{0.550776in}}%
\pgfpathlineto{\pgfqpoint{0.557414in}{0.514551in}}%
\pgfpathlineto{\pgfqpoint{0.620337in}{0.479912in}}%
\pgfpathlineto{\pgfqpoint{0.682186in}{0.447816in}}%
\pgfpathlineto{\pgfqpoint{0.741255in}{0.419140in}}%
\pgfpathlineto{\pgfqpoint{0.795948in}{0.394640in}}%
\pgfpathlineto{\pgfqpoint{0.844826in}{0.374939in}}%
\pgfpathlineto{\pgfqpoint{0.886658in}{0.360500in}}%
\pgfpathlineto{\pgfqpoint{0.920446in}{0.351623in}}%
\pgfpathlineto{\pgfqpoint{0.945452in}{0.348442in}}%
\pgfpathlineto{\pgfqpoint{0.961201in}{0.350934in}}%
\pgfpathlineto{\pgfqpoint{0.967479in}{0.358926in}}%
\pgfusepath{stroke}%
\end{pgfscope}%
\begin{pgfscope}%
\pgfpathrectangle{\pgfqpoint{0.000000in}{0.000000in}}{\pgfqpoint{1.637500in}{2.000000in}} %
\pgfusepath{clip}%
\pgfsetrectcap%
\pgfsetroundjoin%
\pgfsetlinewidth{4.015000pt}%
\definecolor{currentstroke}{rgb}{0.000000,0.000000,0.000000}%
\pgfsetstrokecolor{currentstroke}%
\pgfsetdash{}{0pt}%
\pgfpathmoveto{\pgfqpoint{0.416734in}{0.980212in}}%
\pgfpathlineto{\pgfqpoint{0.416580in}{0.972798in}}%
\pgfpathlineto{\pgfqpoint{0.423985in}{0.961213in}}%
\pgfpathlineto{\pgfqpoint{0.438879in}{0.945686in}}%
\pgfpathlineto{\pgfqpoint{0.460990in}{0.926559in}}%
\pgfpathlineto{\pgfqpoint{0.489842in}{0.904278in}}%
\pgfpathlineto{\pgfqpoint{0.524758in}{0.879393in}}%
\pgfpathlineto{\pgfqpoint{0.564871in}{0.852538in}}%
\pgfpathlineto{\pgfqpoint{0.609151in}{0.824418in}}%
\pgfpathlineto{\pgfqpoint{0.656427in}{0.795783in}}%
\pgfpathlineto{\pgfqpoint{0.705432in}{0.767407in}}%
\pgfpathlineto{\pgfqpoint{0.754838in}{0.740059in}}%
\pgfpathlineto{\pgfqpoint{0.803307in}{0.714479in}}%
\pgfpathlineto{\pgfqpoint{0.849536in}{0.691349in}}%
\pgfpathlineto{\pgfqpoint{0.892300in}{0.671275in}}%
\pgfpathlineto{\pgfqpoint{0.930490in}{0.654764in}}%
\pgfpathlineto{\pgfqpoint{0.963149in}{0.642213in}}%
\pgfpathlineto{\pgfqpoint{0.989495in}{0.633901in}}%
\pgfpathlineto{\pgfqpoint{1.008937in}{0.629984in}}%
\pgfpathlineto{\pgfqpoint{1.021085in}{0.630495in}}%
\pgfpathlineto{\pgfqpoint{1.025748in}{0.635354in}}%
\pgfusepath{stroke}%
\end{pgfscope}%
\begin{pgfscope}%
\pgfpathrectangle{\pgfqpoint{0.000000in}{0.000000in}}{\pgfqpoint{1.637500in}{2.000000in}} %
\pgfusepath{clip}%
\pgfsetrectcap%
\pgfsetroundjoin%
\pgfsetlinewidth{5.018750pt}%
\definecolor{currentstroke}{rgb}{0.000000,0.000000,0.000000}%
\pgfsetstrokecolor{currentstroke}%
\pgfsetdash{}{0pt}%
\pgfpathmoveto{\pgfqpoint{0.310673in}{0.116783in}}%
\pgfpathlineto{\pgfqpoint{0.331470in}{0.151808in}}%
\pgfpathlineto{\pgfqpoint{0.351852in}{0.187102in}}%
\pgfpathlineto{\pgfqpoint{0.369725in}{0.223850in}}%
\pgfpathlineto{\pgfqpoint{0.381232in}{0.264231in}}%
\pgfpathlineto{\pgfqpoint{0.382494in}{0.310441in}}%
\pgfpathlineto{\pgfqpoint{0.371478in}{0.363629in}}%
\pgfpathlineto{\pgfqpoint{0.348876in}{0.423406in}}%
\pgfpathlineto{\pgfqpoint{0.317866in}{0.487975in}}%
\pgfpathlineto{\pgfqpoint{0.283182in}{0.554660in}}%
\pgfpathlineto{\pgfqpoint{0.249989in}{0.620539in}}%
\pgfpathlineto{\pgfqpoint{0.222927in}{0.682989in}}%
\pgfpathlineto{\pgfqpoint{0.205488in}{0.740034in}}%
\pgfpathlineto{\pgfqpoint{0.199752in}{0.790496in}}%
\pgfpathlineto{\pgfqpoint{0.206414in}{0.833981in}}%
\pgfpathlineto{\pgfqpoint{0.225001in}{0.870755in}}%
\pgfpathlineto{\pgfqpoint{0.254188in}{0.901568in}}%
\pgfpathlineto{\pgfqpoint{0.292120in}{0.927467in}}%
\pgfpathlineto{\pgfqpoint{0.336714in}{0.949631in}}%
\pgfpathlineto{\pgfqpoint{0.385888in}{0.969238in}}%
\pgfpathlineto{\pgfqpoint{0.437730in}{0.987368in}}%
\pgfpathlineto{\pgfqpoint{0.490599in}{1.004951in}}%
\pgfpathlineto{\pgfqpoint{0.543170in}{1.022736in}}%
\pgfpathlineto{\pgfqpoint{0.594441in}{1.041290in}}%
\pgfpathlineto{\pgfqpoint{0.643711in}{1.061011in}}%
\pgfpathlineto{\pgfqpoint{0.690542in}{1.082145in}}%
\pgfpathlineto{\pgfqpoint{0.734711in}{1.104821in}}%
\pgfpathlineto{\pgfqpoint{0.776163in}{1.129069in}}%
\pgfpathlineto{\pgfqpoint{0.814971in}{1.154848in}}%
\pgfpathlineto{\pgfqpoint{0.851289in}{1.182071in}}%
\pgfpathlineto{\pgfqpoint{0.885328in}{1.210620in}}%
\pgfpathlineto{\pgfqpoint{0.917327in}{1.240358in}}%
\pgfpathlineto{\pgfqpoint{0.947530in}{1.271148in}}%
\pgfpathlineto{\pgfqpoint{0.976182in}{1.302852in}}%
\pgfpathlineto{\pgfqpoint{1.003511in}{1.335342in}}%
\pgfpathlineto{\pgfqpoint{1.029724in}{1.368498in}}%
\pgfpathlineto{\pgfqpoint{1.055008in}{1.402215in}}%
\pgfpathlineto{\pgfqpoint{1.079527in}{1.436402in}}%
\pgfpathlineto{\pgfqpoint{1.103422in}{1.470979in}}%
\pgfpathlineto{\pgfqpoint{1.126811in}{1.505877in}}%
\pgfpathlineto{\pgfqpoint{1.149796in}{1.541041in}}%
\pgfpathlineto{\pgfqpoint{1.172459in}{1.576423in}}%
\pgfpathlineto{\pgfqpoint{1.194869in}{1.611984in}}%
\pgfpathlineto{\pgfqpoint{1.217081in}{1.647693in}}%
\pgfpathlineto{\pgfqpoint{1.239142in}{1.683525in}}%
\pgfpathlineto{\pgfqpoint{1.261088in}{1.719458in}}%
\pgfpathlineto{\pgfqpoint{1.282946in}{1.755477in}}%
\pgfpathlineto{\pgfqpoint{1.304741in}{1.791568in}}%
\pgfpathlineto{\pgfqpoint{1.326490in}{1.827722in}}%
\pgfpathlineto{\pgfqpoint{1.348207in}{1.863930in}}%
\pgfpathlineto{\pgfqpoint{1.349130in}{1.863407in}}%
\pgfpathlineto{\pgfqpoint{1.327716in}{1.827027in}}%
\pgfpathlineto{\pgfqpoint{1.306366in}{1.790647in}}%
\pgfpathlineto{\pgfqpoint{1.285094in}{1.754260in}}%
\pgfpathlineto{\pgfqpoint{1.263918in}{1.717854in}}%
\pgfpathlineto{\pgfqpoint{1.242861in}{1.681418in}}%
\pgfpathlineto{\pgfqpoint{1.221952in}{1.644934in}}%
\pgfpathlineto{\pgfqpoint{1.201228in}{1.608382in}}%
\pgfpathlineto{\pgfqpoint{1.180732in}{1.571736in}}%
\pgfpathlineto{\pgfqpoint{1.160522in}{1.534965in}}%
\pgfpathlineto{\pgfqpoint{1.140665in}{1.498029in}}%
\pgfpathlineto{\pgfqpoint{1.121246in}{1.460883in}}%
\pgfpathlineto{\pgfqpoint{1.102363in}{1.423468in}}%
\pgfpathlineto{\pgfqpoint{1.084137in}{1.385718in}}%
\pgfpathlineto{\pgfqpoint{1.066709in}{1.347552in}}%
\pgfpathlineto{\pgfqpoint{1.050240in}{1.308879in}}%
\pgfpathlineto{\pgfqpoint{1.034919in}{1.269592in}}%
\pgfpathlineto{\pgfqpoint{1.020953in}{1.229574in}}%
\pgfpathlineto{\pgfqpoint{1.008571in}{1.188696in}}%
\pgfpathlineto{\pgfqpoint{0.998014in}{1.146820in}}%
\pgfpathlineto{\pgfqpoint{0.989529in}{1.103808in}}%
\pgfpathlineto{\pgfqpoint{0.983354in}{1.059524in}}%
\pgfpathlineto{\pgfqpoint{0.979699in}{1.013851in}}%
\pgfpathlineto{\pgfqpoint{0.978719in}{0.966700in}}%
\pgfpathlineto{\pgfqpoint{0.980485in}{0.918031in}}%
\pgfpathlineto{\pgfqpoint{0.984942in}{0.867876in}}%
\pgfpathlineto{\pgfqpoint{0.991867in}{0.816361in}}%
\pgfpathlineto{\pgfqpoint{1.000822in}{0.763735in}}%
\pgfpathlineto{\pgfqpoint{1.011105in}{0.710395in}}%
\pgfpathlineto{\pgfqpoint{1.021714in}{0.656906in}}%
\pgfpathlineto{\pgfqpoint{1.031327in}{0.604019in}}%
\pgfpathlineto{\pgfqpoint{1.038302in}{0.552661in}}%
\pgfpathlineto{\pgfqpoint{1.040731in}{0.503912in}}%
\pgfpathlineto{\pgfqpoint{1.036535in}{0.458948in}}%
\pgfpathlineto{\pgfqpoint{1.023633in}{0.418946in}}%
\pgfpathlineto{\pgfqpoint{1.000173in}{0.384951in}}%
\pgfpathlineto{\pgfqpoint{0.964830in}{0.357714in}}%
\pgfpathlineto{\pgfqpoint{0.917131in}{0.337499in}}%
\pgfpathlineto{\pgfqpoint{0.857768in}{0.323916in}}%
\pgfpathlineto{\pgfqpoint{0.788816in}{0.315788in}}%
\pgfpathlineto{\pgfqpoint{0.713762in}{0.311144in}}%
\pgfpathlineto{\pgfqpoint{0.637240in}{0.307361in}}%
\pgfpathlineto{\pgfqpoint{0.564415in}{0.301520in}}%
\pgfpathlineto{\pgfqpoint{0.500021in}{0.290942in}}%
\pgfpathlineto{\pgfqpoint{0.447239in}{0.273832in}}%
\pgfpathlineto{\pgfqpoint{0.406763in}{0.249796in}}%
\pgfpathlineto{\pgfqpoint{0.376562in}{0.219984in}}%
\pgfpathlineto{\pgfqpoint{0.352756in}{0.186591in}}%
\pgfpathlineto{\pgfqpoint{0.331492in}{0.151796in}}%
\pgfpathlineto{\pgfqpoint{0.310673in}{0.116783in}}%
\pgfusepath{stroke}%
\end{pgfscope}%
\end{pgfpicture}%
\makeatother%
\endgroup%
 \input{title1.pgf}

\part{導言\protect\bigskip}

這裡將姿態﹝orientation or attitude%
﹞動力學與姿態估測%
應用在剛體轉動的尤%
拉運動方程上,以姿%
態估測中的方向餘弦%
法來積分尤拉方程的%
貼體角速度轉動向量%
,以此達到模擬剛體%
的轉動,並且應用在%
陀螺的三維運動上。%
以陀螺的三維運動而%
言,幾乎所有古典力%
學的書上推導出貼體%
角速度的尤拉方程之%
後,就會求諸Euler angles來得%
到Lagrangian,接著用elliptical integral解%
出解析解。再者,以%
數值方法解其尤拉角%
的尤拉方程ODE,然後再%
以尤拉角模擬其運動%
。不過,若對尤拉方%
程透徹理解,只要有%
貼體角速度,也可以%
利用轉動向量﹝rotation vector%
﹞,用姿態估測的方%
法直接簡單地數值模%
擬剛體轉動,這樣就%
不需要用到Lagrangian或Euler angles%
。這個方法屬於姿態%
估測學中的方向餘弦%
遞推法﹝iteration of direction cosine matrix, DCM%
﹞。這邊不只介紹公%
式,還以淺顯易懂的%
方式給出姿態的運動%
方程,所以只要有基%
礎的線性代數矩陣知%
識,就可以掌握此方%
法。這裡詳述的這些%
理論在物理書籍上比%
較少見,反而在航空%
太空領域或電機的姿%
態控制領域才有講解%
,但大多也缺乏清楚%
的解釋,只給出複雜%
的公式,這樣在應用%
層面的時候會發生比%
較多不必要的試誤與%
嘗試。這裡我將這觀%
念連結錯綜複雜的東%
西清楚地寫下來,以%
供自己以及別人參考%
。

這裡詳述的方法可以%
廣泛的運用於任何剛%
體轉動或其尤拉方程%
。因此,此處所涉及%
的方向餘弦遞推的完%
整理解還可應用上其%
他相關領域。舉例來%
說,這裡在作陀螺模%
擬的姿態演算程式可%
以用在陀螺儀這類型%
的綁附式慣性感測器\cite%
[Ch 3.6.4]{titterton}上,積分測得的%
貼體角速度來作姿態%
估測,這將在第四章%
作介紹。仿間姿態演%
算法林立,但真能跑%
得起來的沒半個,因%
為那些程式是沒有經%
過驗證的。這邊從根%
本的推導開始,一步%
步進階到實用的程式%
碼,每一步推導出來%
的公式都有標明轉動%
方向是遵守左手還是%
右手定則,以及標明%
轉動矩陣是取其主動%
性轉向量或是被動性%
轉坐標軸,這一點是%
所有書上所共同缺乏%
的,而這在實際應用%
的時候尤其重要,這%
樣子推導出來的公式%
我們可以很方便的檢%
查其正確性,也可以%
很放心地應用在其他%
地方。另外,這邊提%
供與第三方程式碼的%
比較,驗證我們公式%
的正確性。這裡介紹%
的尤拉方程的數值化%
也可以應用上剛體物%
理模擬﹝simulation of rigid body﹞,%
這在電腦物理引擎或%
電腦3D圖學\cite[Ch 2.3]{pixarnote}中也%
是重要的。這邊我們%
呈現以這些方法數值%
模擬高精度的三維陀%
螺運動,並且模擬結%
果也與其它文獻\cite{hasbun}作%
比較驗證了其正確性%
。這邊提供的詳細解%
說也適合當作大學物%
理系或資工系或航太%
系的例題。

首先,這篇文章的架%
構的三個大重點是:

\begin{enumerate}
\item 第一部分為最完整%
的剛體轉動運動方程%
尤拉方程的推導證明%
。因為要做轉動軸的%
轉動向量數值積分必%
須要對尤拉方程有最%
正確的理解。這邊補%
充了Goldstein Classical Mechanics\cite{goldstein}中的%
證明觀念跳來跳去的%
缺失,以及大多數的%
書上推導尤拉方程解%
釋模糊的地方。轉動%
理論在航太科系書籍%
較有教導,但可惜講%
述通常過於複雜,要%
不就是過於簡化,缺%
乏與基本原理的結合%
,並且大部分也無提%
供實際實例操作這重%
要的一環,這裡則提%
供完整的陀螺模擬實%
例與完整公開的code供練%
習。

\item 接著藉由第一段尤%
拉方程的推導來嚴謹%
的證明Euler equation中的貼體%
角速度﹝angular velocity along body frame﹞%
可直接用於建立剛體%
特徵軸與lab frame間的主動%
與被動轉動矩陣\thinspace ,%
並以此推導出方向餘%
弦法的主要遞推原理%
來積分剛體轉動,追%
蹤每一時刻的剛體特%
徵軸在lab frame的位置。並%
以python程式編寫演算法%
,並以python繪圖庫matplotlib來%
作動畫。

\item 接著說明了以貼體%
轉動向量來近似t到t+dt時%
間的微小轉動時我們%
將以在t+dt時間的貼體角%
速度來近似,即以$\vec{\omega}%
_{b}\left( t+dt\right) $來建立dt時間內%
的轉動矩陣$\footnote{%
The path order exponential of $\vec{\omega}_{b}$ from time t to t+dt. This
will be discussed more in the text.}$。這裡也給%
出不同近似的結果來%
證明此優化所帶來精%
度上的提升。最後將%
以上方法以python程式寫%
出,並且模擬陀螺的%
三維運動,模擬結果%
將與文獻\cite{hasbun}做比較。
\end{enumerate}

\part{\protect\bigskip 尤拉方程與姿%
態動力方程的推導}

首先我們先討論向量%
變化量在不同觀測座%
標中的關係。由於當%
我實際在解這問題時%
我發現Goldstein classical mechanics書中還%
有幾點證明還不清楚%
,因此這邊寫上我認%
為可以補充書上的推%
導證明。首先我們從%
以下公式開始

\begin{equation}
\left( \frac{d\vec{L}}{dt}\right) _{s}=\left( \frac{d\vec{L}}{dt}\right)
_{b}+\vec{\omega}\times \vec{L}
\end{equation}%
對此公式的理解將是%
整篇文章最重要的基%
礎。此公式如何而來%
?此公式為一隨時間%
變動的向量在恆定座%
標與非恆定座標(此例%
為轉動中座標)之間線%
性變換的結果。以下%
是此公式的推導。

\begin{figure}[th]
\caption{{}}
\label{firstfig}
\begin{center}
\fbox{\input{rateofchange.pgf}}
\end{center}
\end{figure}
\bigskip

首先考慮一恆定座標%
S(space),一轉動座標b(body),%
為了方便討論矩陣轉%
換的主被動性與座標%
轉換的左右手法則,%
我們這邊方便的先假%
設$\hat{S}_{x}, \hat{b}_{x}$兩軸重合%
,因此圖中顯示了body frame%
沿著$+\hat{S}_{x}$遵守右手定%
則逆時針轉了$\Omega $角度%
,依右手定則此角位%
移向量$\hat{\Omega}$會在$+\hat{S}_{x}$方%
向。但是接下來的推%
導以及所有公式都適%
用任意的座標旋轉,%
這邊是為了方便討論%
矩陣的主動被動的方%
向性,以及在之後的%
推導方便我們追蹤正%
負號以及矩陣主動被%
動意義的改變,因此%
在圖中做了一個方便%
我們思考的情形。另%
外,大部分書上在討%
論座標轉換時有時候%
給的公式是遵守左手%
定則,但這與物理定%
律所採納的右手定則%
相反,因此這邊我寫%
下完整的右手定則的%
推導,希望之後的人%
不需要像我一樣花了%
大半時間在轉換不同%
公式間左手右手定則%
帶來的正負號的改變%
。

\bigskip 依照圖\ref{firstfig}所示,%
我們可以寫下$\vec{A}$向量%
在S,b座標間的關係%
\begin{equation*}
\left( \vec{A}\right) _{b}=\underset{\text{passive, r.h.}}{\Omega }\left( 
\vec{A}\right) _{s}
\end{equation*}%
其中$\Omega $是s frame到b frame的座%
標轉換矩陣,因為是%
轉換座標軸,因此矩%
陣取被動含意,並且%
我們採用右手定則,%
因此逆時針方向為正%
方向。接下來只要有%
用到矩陣的運算我都%
會標明主被動及左右%
手(r.h. right-hand or l.h. left-hand),這樣我%
們可以很快速對照圖%
表來理解轉動方向,%
這很重要。

若我們考慮$\Omega $的角度%
很小$\Omega \rightarrow d\Omega $(infinitesimal rotation),%
則$d\Omega $矩陣與unity matrix相去%
不遠,可以寫成$1$(unity matrix) +$%
\epsilon $(infinitesimal matrix),$\epsilon $具有%
antisymmetric matrix的特性\cite[p. 169]{goldstein},%
帶入上式%
\begin{equation*}
\left( \vec{A}\right) _{b}=\underset{\text{passive, r.h.}}{\left( 1+\epsilon
\right) }\left( \vec{A}\right) _{s}
\end{equation*}%
infinitesimal matrix有個特性,很容%
易自行驗證,%
\begin{equation*}
\underset{\text{r.h., passive or active}}{\epsilon }=\left[ 
\begin{array}{ccc}
0 & \epsilon _{3}\geq 0 & -\epsilon _{2}\leq 0 \\ 
-\epsilon _{3} & 0 & \epsilon _{1}\geq 0 \\ 
\epsilon _{2} & -\epsilon _{1} & 0%
\end{array}%
\right] \text{, }\underset{\text{l.h., passive or active}}{\epsilon }=\left[ 
\begin{array}{ccc}
0 & -\epsilon _{3}\leq 0 & \epsilon _{2}\geq 0 \\ 
+\epsilon _{3} & 0 & -\epsilon _{1}\leq 0 \\ 
-\epsilon _{2} & \epsilon _{1} & 0%
\end{array}%
\right]
\end{equation*}

\bigskip 現在我們考慮$\vec{A}$是$+%
\hat{b}_{y}$軸的狀況,不過考%
慮相同矩陣$\left( 1+\epsilon \right) $的%
主動特性,也就是主%
動轉向量,這樣的話%
轉動方向會與原本的%
方向相反,變左手定%
則,我們會得到%
\begin{equation*}
\left( \hat{S}_{y}\right) _{s}=\underset{\text{active, l.h.}}{\left(
1+\epsilon \right) }\times \left( \hat{b}_{y}\right) _{s}
\end{equation*}%
整理一下%
\begin{equation*}
\left( \hat{b}_{y}\right) _{s}=\underset{\text{active, r.h.}}{\underbrace{%
\left[ \left( 1+\epsilon \right) \right] ^{T}}}\times \left( \hat{S}%
_{y}\right) _{s}=\underset{\text{active, l.h.}}{\left( 1-\epsilon \right) }%
\times \left( \hat{S}_{y}\right) _{s}
\end{equation*}%
代入上面r.h. $\epsilon $的公式(%
因$\epsilon $還是原本的矩陣)%
,整理一下%
\begin{equation*}
\left( \hat{b}_{y}\right) _{s}-\left( \hat{S}_{y}\right) _{s}=-\left[ 
\begin{array}{ccc}
0 & \epsilon _{3}\geq 0 & -\epsilon _{2}\leq 0 \\ 
-\epsilon _{3} & 0 & \epsilon _{1}\geq 0 \\ 
\epsilon _{2} & -\epsilon _{1} & 0%
\end{array}%
\right] \times \left( \hat{S}_{y}\right) _{s}
\end{equation*}%
利用向量外積,上式%
也可寫成%
\begin{equation*}
\left( \hat{b}_{y}\right) _{s}-\left( \hat{S}_{y}\right) _{s}=\left( \vec{%
\epsilon}\right) _{s}\times \left( \hat{S}_{y}\right) _{s}
\end{equation*}%
其中$\vec{\epsilon}=\left[ 
\begin{array}{c}
\epsilon _{1} \\ 
\epsilon _{2} \\ 
\epsilon _{3}%
\end{array}%
\right] _{s}$為一向量,在S frame中%
的分量為$\epsilon _{1}$,$\epsilon _{2}$%
,$\epsilon _{3}$。

現在我們將上式跟微%
小轉動公式Rodrigues rotation formula比%
較%
\begin{equation*}
\vec{r}^{\prime }-\vec{r}=d\vec{\Omega}\times \vec{r}
\end{equation*}%
$d\vec{\Omega}$是r到r'的r.h.角位移%
向量\thinspace ,因此我們得%
到$\vec{\epsilon}=d\vec{\Omega}$,$d\vec{\Omega}$就是%
s frame到b frame的角位移向量%
(follow r.h. rule)%
\begin{equation*}
\left( \hat{b}_{y}\right) _{s}-\left( \hat{S}_{y}\right) _{s}=\left( d\vec{%
\Omega}\right) _{s}\times \left( \hat{S}_{y}\right) _{s}
\end{equation*}%
這一點很重要,因為%
我們將證明此$\left( d\vec{\Omega}\right)
_{s}$跟接下來我們要推導%
的尤拉公式中的貼體%
角速度$\vec{\omega}$有直接相%
關性,並且以此來做%
我們模擬剛體轉動的%
基礎。

以上的討論是考慮$\vec{A}$%
向量不隨時間變動的%
情況,接下來我們必%
須討論$\vec{A}$以及b frame皆隨%
時間變動的狀況。

\bigskip 
\begin{figure}[th]
\caption{Rate change of a vector observed in a inertial and non-inertial
frame.}
\label{ratevecfig}
\begin{center}
\fbox{\input{rateofchanget2tdt.pgf}}
\end{center}
\end{figure}

\bigskip 在時間t時我們令S與b
frame重合,過了dt時間原%
本的$\vec{A}$向量加了一改%
變量$d\vec{A}$變成$\vec{A}^{\prime }$,並%
且b frame依右手定則轉動%
了一微小角度(infinitisemal rotation)%
,在此前提下,向量$%
\vec{A}$在t時間符合%
\begin{equation}
\left( \vec{A}\right) _{s(t)}=\left( \vec{A}\right) _{b(t)}  \label{roc1}
\end{equation}%
接著,在t+dt時間$\vec{A}+d\vec{A}$%
向量在s與b frame間的關係%
為%
\begin{equation*}
\left( \vec{A}^{\prime }\right) _{b(t+dt)}=\underset{\text{passive, r.h.}}{%
\Omega }\left( \vec{A}^{\prime }\right) _{s(t+dt)}
\end{equation*}%
$\Omega $為s, b frame轉動矩陣(passive r.h.)%
,此$\Omega $矩陣與上一段$%
\vec{A}$不變動的情況的$\Omega $%
矩陣完全相同,我們%
取s到b frame的轉動為微小%
量,$\Omega \rightarrow d\Omega $,上式依%
之前所述的原理可寫%
成%
\begin{equation*}
\left( \vec{A}^{\prime }\right) _{b(t+dt)}=\underset{\text{passive, r.h.}}{%
\left( 1+\epsilon \right) }\left( \vec{A}^{\prime }\right) _{s(t+dt)}
\end{equation*}%
要強調這邊的$\epsilon $矩陣%
跟之前上一段的$\epsilon $矩%
陣是完全相同的,重%
新整理上式%
\begin{equation}
\left( \vec{A}^{\prime }\right) _{s(t+dt)}=\left( \vec{A}^{\prime }\right)
_{b(t+dt)}-\epsilon \left( \vec{A}^{\prime }\right) _{s(t+dt)}  \label{roc2}
\end{equation}%
接著我們用\ref{roc2}式減去%
\ref{roc1}式,%
\begin{eqnarray}
\left( \vec{A}^{\prime }\right) _{s(t+dt)}-\left( \vec{A}\right) _{s(t)}
&=&\left( d\vec{A}\right) _{s}\text{ , (the change in observable A in space
frame)}  \notag \\
&=&\left( \left( \vec{A}^{\prime }\right) _{b(t+dt)}-\left( \vec{A}\right)
_{b(t)}\right) -\epsilon \left( \vec{A}^{\prime }\right) _{s(t+dt)}  \notag
\\
&=&\left( d\vec{A}\right) _{body}-\epsilon \left( \vec{A}^{\prime }\right)
_{s(t+dt)}  \notag \\
&=&\left( d\vec{A}\right) _{body}-\epsilon \left( \vec{A}+d\vec{A}\right)
_{s(t+dt)}  \label{roc4}
\end{eqnarray}%
注意,由於s frame式恆定%
座標因此s frame不變動,$%
s(t+dt)=s(t)$。忽略高階項$\epsilon
\left( d\vec{A}\right) _{s(t+dt)}$,重新整理%
成%
\begin{equation*}
\left( d\vec{A}\right) _{s}=\left( d\vec{A}\right) _{b}-\underset{\text{r.h.}%
}{\epsilon }\left( \vec{A}\right) _{s(t)}
\end{equation*}%
接下來我們只要記得%
我們的下標b frame總是在t+dt%
時的frame,s frame總是指在t時%
間的frame,我們將不再寫%
出frame所對應的時間。依%
之前所述原理代入r.h. $%
\epsilon $的公式,並且利用%
向量外積%
\begin{eqnarray}
\left( d\vec{A}\right) _{s} &=&\left( d\vec{A}\right) _{b}-\left[ 
\begin{array}{ccc}
0 & \epsilon _{3}\geq 0 & -\epsilon _{2}\leq 0 \\ 
-\epsilon _{3} & 0 & \epsilon _{1}\geq 0 \\ 
\epsilon _{2} & -\epsilon _{1} & 0%
\end{array}%
\right] \left( \vec{A}\right) _{s}  \notag \\
&=&\left( d\vec{A}\right) _{b}-\left( \vec{A}\right) _{s}\times \left( d\vec{%
\Omega}\right) _{s}  \label{roc3} \\
&=&\left( d\vec{A}\right) _{b}+\left( d\vec{\Omega}\right) _{s}\times \left( 
\vec{A}\right) _{s}  \notag
\end{eqnarray}%
因為這裡的$\epsilon $矩陣與%
上一段的$\epsilon $矩陣是一%
樣的,因此我們也可%
以用上之前轉動公式%
所推導的微小轉動矩%
陣$\epsilon $所對應的轉動向%
量$\left( d\vec{\Omega}\right) $,這樣我%
們就得到了rate of change of a
vector/observable in a rotating frame公式%
\begin{equation}
\left( d\vec{A}\right) _{s}=\left( d\vec{A}\right) _{b}+\left( d\vec{\Omega}%
\right) _{s}\times \left( \vec{A}\right) _{s}  \label{rateofdomega}
\end{equation}%
我們連結了不同觀測%
座標觀測到的物理變%
化量,並且所用到都%
是已知的物理量$\left( d\vec{\Omega}%
\right) _{s}$與$\left( \vec{A}\right) _{s}$。這邊%
要強調,因為這裡的$%
\epsilon $矩陣與上一段的$\epsilon $%
矩陣是一樣的,所以%
證明了$d\vec{\Omega}$所對應的%
向量就是s frame轉到b frame的%
角位移向量(r.h.),這樣%
強調的目的是,接下%
來$d\vec{\Omega}$所導出的貼體%
角速度,就是s frame轉到b
frame的角速度,因為這跟%
一般我們對貼體角速%
度的定義與認知並不%
一樣,這裡再次強調%
,我們必須考慮了t時%
間s與b frame重合,才能得%
到這結果。接下來會%
說明,也是因為如此%
,我們才能利用貼體%
角速度來作座標軸的%
轉動追蹤,因此此觀%
念至關重要。另外要%
注意的是$\vec{A}$與$d\vec{\Omega}$是%
沿著t時間的s frame取的投%
影量。這邊值得一提%
的是,傳統公式大多%
寫成%
\begin{equation*}
\left( d\vec{A}\right) _{s}=\left( d\vec{A}\right) _{b}+\left( d\vec{\Omega}%
\right) _{b}\times \left( \vec{A}\right) _{b}
\end{equation*}%
為什麼這邊的$d\vec{\Omega}$是%
沿著body取分量呢?注意%
\ref{roc3}式中當我們將微小%
轉動矩陣$\epsilon $寫成向量$d%
\vec{\Omega}$時,我們並沒有侷%
限此向量是定義在哪%
一個觀測座標,因此%
沿著s或b frame取都是可以%
的。另外\ref{roc4}式中我們%
忽略了高階項$\epsilon \left( d\vec{A}%
\right) _{s(t+dt)}$,因此留下了$\epsilon
\left( \vec{A}\right) _{s(t+dt)}=\epsilon \left( \vec{A}\right) _{s(t)}$%
,但其實我們也可以%
取另一個近似%
\begin{equation*}
-\epsilon \left( \vec{A}+d\vec{A}\right) _{s(t+dt)}=-\epsilon \left( \vec{A}%
\right) _{b(t+dt)}
\end{equation*}%
若我們考慮s到b frame的轉%
動非常的微小,我們%
只是做了一個不一樣%
的忽略方法。這樣的%
話,我們就得到傳統%
公式。這裡也是Goldstein\cite%
{goldstein}裡面的附註說明$\left( d%
\vec{\Omega}\right) \times \left( \vec{A}\right) $沿著s或b
frame取分量都是可以的,%
只要外積矩陣運算後%
出來的結果是一樣的%
就可以,不過他並沒%
有給出背後的原因。

\bigskip \ref{rateofdomega}式取微分即得%
到一般常見的形式,%
這邊我們取沿s frame給出%
的公式,方便我們之%
後作數值模擬%
\begin{equation}
\left( \frac{d\vec{A}}{dt}\right) _{s}=\left( \frac{d\vec{A}}{dt}\right)
_{b}+\left( \vec{\omega}\right) _{s}\times \left( \vec{A}\right) _{s}
\label{rateofchange}
\end{equation}%
其中$\left( \vec{\omega}\right) _{s}$為$\left( \frac{d\vec{%
\Omega}}{dt}\right) _{s}$,根據我們之%
前對$d\vec{\Omega}$的定義與強%
調,我們知道$\left( \vec{\omega}\right)
_{s}$即為s frame到b frame的瞬時角%
速度。

嚴謹的定義了$d\vec{\Omega}$與$%
\left( \vec{\omega}\right) _{s}$後,我們接%
著需要討論如何從$\left( \vec{%
\omega}\right) _{s}$求回相對應的轉%
動矩陣,這邊你會認%
為,不是將$\left( \vec{\omega}\right) _{s}$%
的xyz分量帶入之前$1+\epsilon $%
矩陣中的$\epsilon _{1}\epsilon _{2}\epsilon _{3}$%
就可以了嗎,這樣是%
不行的,因為從之前%
微小轉動的推導可以%
看出,$\epsilon _{1}\epsilon _{2}\epsilon _{3}$是%
符合特定的antisymmetric matrix properties%
的,但任意的角速度%
向量$\left( \vec{\omega}\right) _{s}$可不然%
。這邊我們利用Calvin Klein
parameter來近似原本的轉動%
矩陣﹝CK parameters矩陣基本上%
與轉動公式Rodrigues rotation formula同%
源\cite{goldstein}﹞,這邊我們%
給他一個新代號$CK(d\vec{\Omega})$%
,當然,接下來只要%
是矩陣運算我們都會%
寫上$CK$的主被動及左右%
手性質,方便我們與%
圖對照與思考\footnote{力矩%
給出的角速度是遵守%
右手定則(counterclockwise),所以CK%
矩陣必須使用其active
counterclockwise sense才能描述正確%
向量轉動,要小心,%
因大部分書上(如Goldstein)給%
的公式都是active clockwise(follow左%
手定則)(舉例如書上的%
Caley Klein parameter rotation matrix),因此差%
一個負號,這裡我花%
了許多時間把文獻上%
所有公式轉成了正確%
的右手定則。}。%
\begin{eqnarray*}
\underset{\text{r.h.}}{CK(d\vec{\Omega})} &=&\left[ 
\begin{array}{ccc}
a^{2}+b^{2}-c^{2}-d^{2} & 2(bc-ad) & 2(bd+ac) \\ 
2(bc+ad) & a^{2}+c^{2}-b^{2}-d^{2} & 2(cd-ab) \\ 
2(bd-ac) & 2(cd+ab) & a^{2}+d^{2}-b^{2}-c^{2}%
\end{array}%
\right] \text{,} \\
\text{with }a &=&\cos \left( \frac{\left\vert d\vec{\Omega}\right\vert }{2}%
\right) \text{, b, c, d = component of }d\hat{\Omega}\cdot \sin \left( \frac{%
\left\vert d\vec{\Omega}\right\vert }{2}\right)
\end{eqnarray*}%
現在,我們一再強調$d%
\vec{\Omega}$所對應的是s frame轉動%
到b frame,因此我們建立%
的$CK(d\vec{\Omega})$矩陣具有以下%
的特性,根據圖\ref{ratevecfig}%
,%
\begin{eqnarray*}
\left( \vec{A}\right) _{b} &=&\underset{\text{passive, r.h.}}{CK(d\vec{\Omega%
})}\left( \vec{A}\right) _{s} \\
\left( \hat{S}_{y}\right) _{s} &=&\underset{\text{active, l.h.}}{CK(d\vec{%
\Omega})}\left( \hat{b}_{y}\right) _{s}
\end{eqnarray*}%
或者也可寫成%
\begin{eqnarray}
\left( \vec{A}\right) _{s} &=&\underset{\text{active, l.h.}}{\underbrace{%
\left[ CK(d\vec{\Omega})\right] ^{T}}}\left( \vec{A}\right) _{b}
\label{frametrans} \\
\left( \hat{b}_{y}\right) _{s} &=&\underset{\text{active, r.h.}}{\underbrace{%
\left[ CK(d\vec{\Omega})\right] ^{T}}}\left( \hat{S}_{y}\right) _{s}
\label{vecrot}
\end{eqnarray}%
這代表了,若我們知%
道s frame到b frame的轉動角度%
,我們就可以求出t+dt時%
間的xyz軸在t時間xyz軸的%
投影量。若我們知道%
的是s到b frame的瞬時角速%
度$\left( \vec{\omega}\right) _{s}$則可帶入$%
CK(\left( \vec{\omega}\right) _{s}\cdot dt)^{T}$來得到%
轉矩矩陣。以上兩式%
就是模擬或追蹤剛體%
的body frame的x,y,z軸轉動的基%
礎。

\begin{figure}[th]
\caption{How to apply rate-of-change-of-a-vector equation to numerically
simulate a true rotation. Here $S_{x}S_{y}b_{x}b_{y}$ are not shown.}
\label{szsbtdtfig}
\begin{center}
\fbox{\input{SzBz.pgf}}
\end{center}
\end{figure}

上述的微小轉動是只%
考慮t到t+dt時間內的變化%
,現在我們將用遞推%
並且discrete的方式,求出body
frame在實驗者處在的靜止%
座標的變化。因此現%
在我們設定一個真正%
的靜止座標Lab frame\thinspace ,見%
圖\ref{szsbtdtfig},此為真正的%
觀測者所處在的inertial frame%
。考慮任意一段微小%
轉動t到t+dt,在t時刻時%
我們將剛體的principle axes設%
定為S frame,再將t+dt時刻剛%
體的principle axes設定為b frame,這%
樣代表s frame到b frame就是剛%
體t到t+dt的轉動,並且s%
到b frame的瞬時角速度也%
是剛體轉動的瞬時角%
速度。我們重新將\ref{vecrot}%
式寫成%
\begin{equation}
\hat{y}_{s}\left( t+dt\right) =\underset{\text{active, r.h.}}{\underbrace{%
\left[ CK\left( \vec{\omega}_{s}\left( t\right) dt\right) \right] ^{T}}}\hat{%
y}_{s}\left( t\right)  \label{vecrot05}
\end{equation}%
$\hat{y}_{s}\left( t\right) $現在為t時間%
剛體特徵軸$\hat{y}$在s frame(也%
是t時刻)的投影,這代%
表$\hat{y}_{s}\left( t\right) $即為單位向%
量$\left[ 
\begin{array}{ccc}
0 & 1 & 0%
\end{array}%
\right] $。現在我們再將上%
式寫成%
\begin{equation}
\hat{y}_{s}\left( t_{i+1}\right) =\underset{\text{active, r.h.}}{\underbrace{%
\left[ CK\left( \vec{\omega}_{s}\left( t_{i}\right) \Delta t\right) \right]
^{T}}}\times \left[ 
\begin{array}{ccc}
0 & 1 & 0%
\end{array}%
\right]  \label{vecrot1}
\end{equation}

\begin{figure}[th]
\caption{Boby軸在每一分段t到t+dt%
的追蹤示意圖。}
\begin{center}
\fbox{\input{Zt0Zt1.pgf}}
\end{center}
\end{figure}

\begin{figure}[th]
\caption{陀螺的初始值設定%
。}
\begin{center}
\fbox{\input{orien.pgf}}
\end{center}
\end{figure}

假設陀螺特徵軸在lab frame%
的起始位置已知$\hat{x}\hat{y}\hat{z%
}_{lab}(t_{0})$,初始貼體角速%
度已知$\vec{\omega}_{s}(t_{0})$,依照%
以往下標代表的是觀%
測的frame。首先我們將s frame%
放在$\left( \hat{x}\hat{y}\hat{z}_{lab}(t_{0})\right) $,%
這樣依照圖\ref{szsbtdtfig}及其%
所述原理,b frame的軸就%
是我們要求的$\hat{x}\hat{y}\hat{z}%
_{lab}(t_{1})$。我們先看$\hat{z}$軸%
,\ref{vecrot1}式告訴我們%
\begin{equation*}
\underset{\text{active, r.h.}}{\hat{z}_{0}\left( t_{1}\right) =\underbrace{%
\left[ CK\left( \vec{\omega}_{s}\left( t_{0}\right) dt\right) \right] ^{T}}}%
\times \hat{z}_{0}\left( t_{0}\right) \underset{\text{active, r.h.}}{=%
\underbrace{\left[ CK\left( \vec{\omega}_{s}\left( t_{0}\right) dt\right) %
\right] ^{T}}}\times \left[ 
\begin{array}{ccc}
0 & 0 & 1%
\end{array}%
\right]
\end{equation*}%
其中$\hat{z}_{0}\left( t_{0})\text{,}\hat{z}%
_{0}(t_{1}\right) $代表時間為$t_{0}$與$%
t_{1}$的\^{z}軸在$t_{0}$時間的座%
標軸(也就是s frame)的投影%
,因此$\hat{z}_{0}\left( t_{0}\right) $為單%
位向量$\left[ 
\begin{array}{ccc}
0 & 0 & 1%
\end{array}%
\right] $。這樣子我們求得%
下一個z軸的位置在$t_{0}$%
的投影,不過我們得%
轉回lab frame,我們假設lab frame%
的$xyz$軸到陀螺初始位置%
$\hat{x}\hat{y}\hat{z}_{lab}(t_{0})$的轉動向量%
是$\vec{\Omega}_{0}, $這樣我們可%
以用$\vec{\Omega}_{0}$輕易的改變%
陀螺初始位置,運用%
上\ref{frametrans}式%
\begin{equation*}
\hat{z}_{lab}\left( t_{1}\right) =\underset{\text{passive, l.h.}}{%
\underbrace{\left[ CK\left( \vec{\Omega}_{0}\right) \right] ^{T}}}\times 
\hat{z}_{0}\left( t_{1}\right)
\end{equation*}%
注意這邊矩陣就取被%
動含意,結合以上兩%
式得到%
\begin{eqnarray*}
\hat{z}_{lab}\left( t_{1}\right) &=&\underset{\text{passive, l.h.}}{%
\underbrace{\left[ CK\left( \vec{\Omega}_{0}\right) \right] ^{T}}}\underset{%
\text{active, r.h.}}{\underbrace{\left[ CK\left( \vec{\omega}_{s}\left(
t_{0}\right) dt\right) \right] ^{T}}}\times \hat{z}_{0}\left( t_{0}\right) \\
&=&\underset{\text{passive, l.h.}}{\underbrace{\left[ CK\left( \vec{\Omega}%
_{0}\right) \right] ^{T}}}\underset{\text{active, r.h.}}{\underbrace{\left[
CK\left( \vec{\omega}_{s}\left( t_{0}\right) dt\right) \right] ^{T}}}\times %
\left[ 
\begin{array}{ccc}
0 & 0 & 1%
\end{array}%
\right]
\end{eqnarray*}%
這樣我們就從t$_{0}$時間%
得到t$_{1}$時間陀螺z軸的%
位置。

接著若我們知道$\hat{z}_{lab}\left(
t_{i}\right) $,再從尤拉方程%
數值法解出的$\vec{\omega}_{s}\left(
t_{0},t_{1},\cdots ,t_{i}\right) $(接下來會說%
明),我們同樣可以求%
得$\hat{z}_{lab}\left( t_{i+1}\right) $,首先用%
\ref{vecrot}式%
\begin{eqnarray*}
\hat{z}_{i}\left( t_{i+1}\right) &=&\underset{\text{active, r.h.}}{%
\underbrace{\left[ CK\left( \vec{\omega}_{s}\left( t_{i}\right) dt\right) %
\right] ^{T}}}\times \hat{z}_{i}\left( t_{i}\right) \\
&=&\underset{\text{active, r.h.}}{\underbrace{\left[ CK\left( \vec{\omega}%
_{s}\left( t_{i}\right) dt\right) \right] ^{T}}}\times \left[ 
\begin{array}{ccc}
0 & 0 & 1%
\end{array}%
\right]
\end{eqnarray*}%
再用\ref{frametrans}式轉回到lab frame%
\begin{eqnarray}
\hat{z}_{lab}\left( t_{i+1}\right) &=&\underset{\text{passive, l.h.}}{%
\underbrace{\left[ CK\left( lab\rightarrow t_{i}\right) \right] ^{T}}}\times 
\hat{z}_{i}\left( t_{i+1}\right)  \notag \\
&=&\underset{\text{passive, l.h.}}{\underbrace{\left[ CK\left( \vec{\Omega}%
_{0}\right) \cdot CK\left( \vec{\omega}_{s}\left( t_{0}\right) dt\right)
\cdot CK\left( \vec{\omega}_{s}\left( t_{1}\right) dt\right) \cdot \cdots
\cdot CK\left( \vec{\omega}_{s}\left( t_{i-1}\right) dt\right) \right] ^{T}}}%
\times  \notag \\
&&\hat{z}_{i}\left( t_{i+1}\right)  \notag \\
&=&\underset{\text{passive, l.h.}}{\underbrace{\left[ CK\left( \vec{\Omega}%
_{0}\right) \cdot CK\left( \vec{\omega}_{s}\left( t_{0}\right) dt\right)
\cdot CK\left( \vec{\omega}_{s}\left( t_{1}\right) dt\right) \cdot \cdots
\cdot CK\left( \vec{\omega}_{s}\left( t_{i-1}\right) dt\right) \right] ^{T}}}%
\times  \notag \\
&&\underset{\text{active, r.h.}}{\underbrace{\left[ CK\left( \vec{\omega}%
_{s}\left( t_{i}\right) dt\right) \right] ^{T}}}\times \left[ 
\begin{array}{ccc}
0 & 0 & 1%
\end{array}%
\right]  \label{iterateeq0}
\end{eqnarray}%
這裡用上不同時間微%
小轉動矩陣的commutive性質$%
\left( AB\right) C=A\left( BC\right) $。這樣我%
們就得到了所有時刻$%
\left( t_{0},t_{1},\cdots ,t_{i+1}\right) $z軸在lab frame位%
置的公式,同樣方法%
可求得x,y軸。可以看出%
上面所有passive的矩陣的%
作用只是再把坐標軸%
從body frame轉回到lab frame。若我%
們再用上轉動矩陣相%
乘即等於轉動向量相%
加的事實\cite{goldstein},上式%
可在寫成%
\begin{equation}
\hat{z}_{lab}\left( t_{i+1}\right) ==\left[ CK\left( \vec{\Omega}%
_{0}+\sum\limits_{m=0}^{i}\vec{\omega}_{s}\left( t_{m}\right) \right) \right]
^{T}\times \left[ 
\begin{array}{ccc}
0 & 0 & 1%
\end{array}%
\right]  \label{iterateeq2}
\end{equation}%
不過要注意,只有CK內%
包含的所有轉動向量%
都是要微小轉動向量%
此條公式才會成立。%
﹝在我們的Python程式中%
也包含有使用此公式%
的選項,讀者可以自%
由切換比較。用此公%
式的好處是,矩陣的%
相乘轉換成了矩陣的%
相加,在速度及記憶%
體的佔用上此近似會%
有優勢。不過,程式%
預設是使用\ref{iterateeq0}公式%
,因$\vec{\Omega}_{0}$量值不小。%
﹞另外這邊我們就省%
略了矩陣的主被動左%
右手的標示,但注意%
,我們的CK矩陣是定義%
右手定則,遵守物理%
定律。這裡若不強調%
會產生問題,因為很%
多書上給的是CK是左手%
公式。並且,若要了%
解公式,主被動及左%
右手的標示是必須保%
留並且對照圖表才有%
辦法了解公式如何而%
來以及轉動方向是如%
何,而這也是大部分%
書上所缺乏的。這樣%
,我們就得到了剛體%
特徵軸的追蹤公式,%
並且只要知道"貼體"角%
速度﹝注意並不是角%
速度在lab frame的觀察值,%
當然如果知道這個,%
我就不用寫這麼多了%
。﹞就可以求得。同%
理可得$\hat{x}_{lab}\left( t_{i+1}\right) $,$\hat{y}%
_{lab}\left( t_{i+1}\right) $。

\bigskip 若將$\hat{x}_{lab}\left( t_{i+1}\right) $,$\hat{y%
}_{lab}\left( t_{i+1}\right) $,$\hat{z}_{lab}\left( t_{i+1}\right) $%
的三個行組成一矩陣C%
,我們會發現\ref{iterateeq2}式%
即為導航書籍上常看%
到的方向餘弦遞推理%
論的遞推公式\cite{titterton}%
\begin{equation}
C(t_{i+1})=C(t_{i})\exp (\int \omega dt)  \label{iterationC}
\end{equation}%
不過很少有書上會把%
此公式解釋的清楚,%
比如說,\ref{iterationCK}式中的%
transpose若不經過我們每一%
步轉動矩陣都紀錄並%
寫下主被動與左右手%
性質,實際應用的時%
候我們將不知道要套%
用transpose矩陣,轉動向量%
的主動還是坐標系的%
被動特性以及轉動方%
向遵守是左手還是右%
手定則,這樣將造成%
極大的困擾。因此我%
們知道,只給出\ref{iterationC}%
公式只是一小步,離%
實作層面還有很大一%
段距離。此式中$\exp (\int \omega
dt)$為path order exponential,關於此,%
以及要如何證明$\exp (\int \omega
dt)$是我們的CK矩陣,可%
以在\cite[Page 49]{tong}找到,並注%
意連結與$\sin $,$\cos $的泰%
勒展開的關係。

以上為轉動坐標系中%
的向量變化量與恆定%
坐標系中的向量變化%
量與轉動坐標系的角%
速度的關係推導,不%
過我們發現我們必須%
要知道所有時刻的貼%
體角速度$\vec{\omega}_{s}\left( t_{0}\cdots
t_{i}\right) $。若是在慣性感%
測器的應用,由於strap-down(%
捷聯式)慣性感測器測%
量得的就是貼體角速%
度,因此我們可以直%
接帶入我們的公式,%
這就是捷聯式)慣性感%
測器姿態估測演算法%
的一種演算法。不過%
接下來我們說明如何%
從尤拉方程數值求出$%
\vec{\omega}_{s}\left( t_{0}\cdots t_{i}\right) $,並以%
此模擬陀螺運動。

將\ref{rateofchange}式應用上一段s%
到b frame,t到t+dt的微小轉動%
,並且考慮$\vec{A}$為剛體%
角動量$\vec{L}$,則我們得%
到%
\begin{equation}
\left( \Gamma \right) _{s}=\left( \frac{d\vec{L}}{dt}\right) _{s}=\left( 
\frac{d\vec{L}}{dt}\right) _{b}+\left( \vec{\omega}\right) _{s}\times \left( 
\vec{L}\right) _{s}  \label{liw}
\end{equation}%
這裡第一等號也用上%
牛頓定律。現在我們%
從\ref{rateofdomega}式知道$\left( d\vec{\Omega}%
\right) _{s}$是s到b frame的角位移,%
而經由我們之前剛體%
特徵軸的設定,s到b frame%
正是我們剛體特徵軸%
從t到t+dt的角位移,因此%
$\left( \frac{d\vec{\Omega}}{dt}\right) _{s}=\left( \vec{\omega}\right) _{s}$%
正是剛體的瞬時角速%
度(沿著t時間s frame取分量)%
,接著,因為s, b frame都是%
沿著body principle axes而取,因此%
沿s frame的角動量$\left( \vec{L}\right) _{s}$%
可以寫成%
\begin{equation*}
\left( \vec{L}\right) _{s}=\left[ 
\begin{array}{ccc}
I_{xx} & 0 & 0 \\ 
0 & I_{yy} & 0 \\ 
0 & 0 & I_{zz}%
\end{array}%
\right] \times \left( \vec{\omega}\right) _{s}
\end{equation*}%
並且沿著b frame的$\left( \frac{d\vec{L}}{dt}%
\right) _{b}$項也可寫成%
\begin{equation*}
\left( \frac{d\vec{L}}{dt}\right) _{b}=\left[ 
\begin{array}{ccc}
I_{xx} & 0 & 0 \\ 
0 & I_{yy} & 0 \\ 
0 & 0 & I_{zz}%
\end{array}%
\right] \times \frac{d\left( \vec{\omega}\right) _{b}}{dt}
\end{equation*}%
其中$I_{xx}$、$I_{yy}$、$I_{zz}$都不%
隨時間變動。再次注%
意$\left( \Gamma \right) _{s}$與$\left( \vec{\omega}\right) _{s}$%
與$\left( \vec{L}\right) _{s}$都是沿著t時%
刻的剛體特徵軸(也就%
是s frame)取的投影,並不%
是Lab frame的投影,這點要%
特別注意,基本上這%
代表,$\left( \Gamma \right) _{s}$是貼體%
的角動量!這裡大部%
分的書上都沒有給出%
恰當的原因,如果用lab
frame的$\Gamma $那麼就無法成%
功的數值化喔\footnote{注意%
因為s frame會持續的改變%
所以$\left( \vec{\Gamma}\right) _{s}$不可取$%
\left( \vec{\Gamma}\right) _{lab}$的值,同理$%
\left( \vec{\omega}\right) _{s}$也不是$\left( \vec{\omega}%
\right) _{lab}$,兩者都必須經%
過轉換從lab轉到t時刻s frame%
。}。這邊我們證明了\ref%
{liw}式最後那一項中的兩%
個$\vec{\omega}$是相同的\footnote{但%
我們必須強調,任意%
情況下,角速度$\left( \vec{\omega}%
\right) $在body轉動座標下的投%
影並不是body座標上觀察%
到的角速度!這是很%
常見的錯誤,這裡我%
們是有條件的考慮t到t+dt%
時刻的t時刻s,b座標重和%
。}。代入$\vec{L}$並展開\ref{liw}%
式,我們就得到所謂%
的尤拉公式(Euler's equation)%
\begin{eqnarray}
\Gamma _{x}(t) &=&I_{x}\dot{\omega}_{x}+(I_{z}-I_{y})\omega _{y}\left(
t\right) \omega _{z}\left( t\right)  \notag \\
\Gamma _{y}(t) &=&I_{y}\dot{\omega}_{y}+(I_{x}-I_{z})\omega _{x}\omega _{z}
\label{eulereqbody} \\
\Gamma _{z}(t) &=&I_{z}\dot{\omega}_{z}+(I_{x}-I_{y})\omega _{x}\omega _{y} 
\notag
\end{eqnarray}%
注意$\vec{\Gamma}$及$\vec{\omega}$的x,y,z分%
量都是沿著t時刻的剛%
體特徵軸s frame取的分量%
,這點必須要強調。%
之後數值模擬的時候%
這點是非常重要的。

接下來應用上陀螺,%
若考慮陀螺的條件 $%
I_{x}=I_{y}\neq I_{z}$,\ref{eulereqbody}式可寫%
成%
\begin{equation}
\frac{d}{dt}\left[ 
\begin{array}{c}
\omega _{x} \\ 
\omega _{y} \\ 
\omega _{z}%
\end{array}%
\right] =\left[ 
\begin{array}{ccc}
0 & -\frac{I_{z}-I_{y}}{I_{x}}\omega _{z} & 0 \\ 
-\frac{I_{x}-I_{z}}{I_{y}}\omega _{z} & 0 & 0 \\ 
0 & 0 & 0%
\end{array}%
\right] \left[ 
\begin{array}{c}
\omega _{x} \\ 
\omega _{y} \\ 
\omega _{z}%
\end{array}%
\right] +\left[ 
\begin{array}{c}
\frac{\Gamma _{x}}{I_{x}} \\ 
\frac{\Gamma _{y}}{I_{y}} \\ 
\frac{\Gamma _{z}}{I_{z}}%
\end{array}%
\right]
\end{equation}%
如之前所強調,右邊%
所有項都是在時間為t%
時刻的s frame取得值,也%
因此以上的微分方程%
組可以用普通數值由%
拉法或四階Ruge Kutta求出左%
側$\omega _{x,y,z}(t+dt)$,也就是從$\vec{%
\omega}_{s}(t)$求得$\vec{\omega}_{s}(t+dt)$。不%
過對於任意的剛體轉%
動系統,只要能從\ref%
{eulereqbody}式右側$\vec{\omega}_{s}(t)$求得%
左側$\vec{\omega}_{s}(t+dt)$,都還是%
能夠適用接下來的模%
擬方法,也因此這裡%
描述的方法是具有任%
意一般性的,可以應%
用在任何的剛體轉動%
。有不少的數值方法%
可以解一般的非線性%
一階ODE尤拉方程\cite{matlab}。%
這裡我以Ruge Kutta四階法求%
解上式,來得到$\vec{\omega}%
_{s}\left( t_{0}\cdots t_{i}\right) $,並且寫成%
python程式,程式將在下%
一章介紹。

\begin{figure}[th]
\caption{陀螺的對稱軸的單%
位向量的模擬軌跡圖%
。以$\protect\omega (t_{i})$或$\protect\omega (t_{i+1})$%
轉動向量來近似$t_{i}$到$%
t_{i+1}$時間的轉動的模擬%
結果,並與Hasbun教授的%
正確解比較。$\protect\omega (t_{i+1})$%
的結果與Hasbun教授的結%
果在此圖形中幾乎重%
合,這代表以$\protect\omega (t_{i+1})$%
來近似比$\protect\omega (t_{i})$好很%
多。}
\label{wtiwti1}
\begin{center}
\fbox{\input{wti_wtiplus1.pgf}}
\end{center}
\end{figure}

\section{\protect\bigskip 數值偏差﹝numerical
drift﹞}

以上\ref{vecrot05}式中以$CK\left( \vec{\omega}%
_{s}(t_{i})dt\right) $來近似t$_{i}$到t$_{i+1}$%
的轉動事實上還不夠%
好,圖\ref{wtiwti1}顯示,以$\omega
(t_{i})$轉動向量近似的結%
果與Hasbun的正確結果偏%
差不少。這邊我提出%
以$CK\left( \vec{\omega}_{s}(t_{i+1})dt\right) $來近%
似t$_{i}$到t$_{i+1}$的轉動,圖\ref%
{wtiwti1}顯示模擬結果幾乎%
與Hasbun的正確解重合,%
至少此圖表上無法看%
出任何差別。以下我%
也嘗試提供物理解釋%
。這裡我們暫時假設$%
\vec{\omega}_{s}(t_{i})dt=\vec{\Omega}_{s}(t_{i})$,我們%
知道轉動向量在$t_{i+1}$跟$%
t_{i}$時刻在body frame中的向量%
值一般不會一樣,也%
就是$\vec{\Omega}_{i+1}(t_{i+1})\neq \vec{\Omega}_{i}(t_{i})$%
,這代表從$t_{i}$到$t_{i+1}$時%
,轉動向量在body座標上%
有變化,也因此我們%
不能夠單只考慮陀螺%
轉了$\vec{\Omega}_{s}(t_{i})$而已,此%
額外轉動向量的變化%
在$t_{i}$時s frame的向量值為$%
\Omega _{i+1}(t_{i+1})-\Omega _{i}(t_{i})=\Omega _{i}(t_{i})+d\Omega
_{i}(dt)-\Omega _{i}(t_{i})=d\Omega _{s}(dt)$,也是一%
個轉動向量,所以space空%
間中總共的轉動可以%
考慮成兩步,第一步%
轉$\Omega _{s}(t_{i})$,第二步轉$d\Omega
_{s}(dt)$,寫成轉動矩陣%
\begin{equation}
CK(\Omega _{s}(t_{i}))\times CK(d\Omega _{s}(dt))=CK(\Omega
_{s}(t_{i})+d\Omega _{s}(dt))=CK(\Omega _{i+1}(t_{i+1}))
\end{equation}%
這代表我們只要考慮%
陀螺從t到t+dt的時候是轉%
了$\Omega _{s}(t+dt)$而不只是$\Omega _{s}(t)$%
,因此考慮$\Omega _{s}(t+dt)$我們%
就更準確的近似了這%
個轉動,以下的Python程%
式模擬會證明,考慮%
了$\Omega _{s}(t+dt)$給出的結果比$%
\Omega _{s}(t)$好非常多。若如%
此考慮則\ref{iterateeq0}式變成%
\begin{eqnarray}
\hat{z}_{lab}\left( t_{i+1}\right) &=&\left[ CK\left( \vec{\Omega}%
_{0}\right) \cdot CK\left( \vec{\omega}_{s}\left( t_{1}\right) dt\right)
\cdot CK\left( \vec{\omega}_{s}\left( t_{2}\right) dt\right) \cdot \cdots
\cdot CK\left( \vec{\omega}_{s}\left( t_{i+1}\right) dt\right) \right]
^{T}\times  \notag \\
&&\left[ 
\begin{array}{ccc}
0 & 0 & 1%
\end{array}%
\right]  \label{iterationCK}
\end{eqnarray}%
此公式即為我Python程式%
中使用的DCM遞推的公式%
。相同方法可求得另%
外兩軸x,y的運動。

以下將上述方法寫成%
python程式,以尤拉方程%
數值解出貼體角速度%
,接著用遞推公式\ref%
{iterateeq2}畫圖模擬其xyz軸運%
動。這邊劃出四種陀%
螺的經典運動。

\begin{figure}[th]
\caption{尖點運動}
\begin{center}
\fbox{\includegraphics[scale=0.6]{figure_xy.eps}}
\end{center}
\end{figure}

\begin{figure}[th]
\caption{有環運動}
\begin{center}
\fbox{\includegraphics[scale=0.6]{figure_x.eps}}
\end{center}
\end{figure}

\begin{figure}[th]
\caption{無環運動}
\begin{center}
\fbox{\includegraphics[scale=0.6]{figure_y.eps}}
\end{center}
\end{figure}

\begin{figure}[th]
\caption{等周速運動}
\label{figure_uniform}
\begin{center}
\fbox{\includegraphics[scale=0.6]{figure_uniform.eps}}
\end{center}
\end{figure}

\part{\protect\bigskip 方向餘弦演算%
法Python程式碼之說明}

這裡提供的Python程式碼%
可以適用於任意的陀%
螺狀態,因此不會遇%
到以尤拉角表達的方%
法所遇到在九十度角%
所遭遇的發散問題。%
這個程式碼也包含了%
Hasbun教授在他的著作中\cite%
{hasbun}提供的Matlab程式碼轉%
成的Python程式碼,由於%
我希望能夠作系統性%
的比較,所以花了一%
些時間將Hasbun教授的Matlab程%
式碼重寫成Python程式碼%
。因此在我的Python程式%
碼中也包含了Hasbun教授%
的程式,此程式函數%
叫做\texttt{HasbunMatlabTop()},其定義%
在\texttt{top.py}檔案中。Hasbun教授%
的程式碼是尤拉角所%
寫出來的,所以會有%
發散問題,不過尤拉%
角法解出的是解析解%
,所以只要不發散,%
結果會是絕對正確的%
。由於姿態的方向餘%
弦遞推是一種數值近%
似,因此會有誤差,%
也因此我們有跟解析%
解比較的必要。當然%
,方向餘弦法適用任%
意的剛體轉動方程,%
可直接套用,而尤拉%
角法還得要看轉動尤%
拉方程的複雜程度,%
必須要能寫成一階ODE微%
分方程組,才能做數%
值積分,因此各有其%
優缺點。我們這邊比%
較方向餘弦法與Hasbun教%
授尤拉法所算出來的%
陀螺對稱軸﹝body z軸﹞%
之間的夾角角度\thinspace ,%
見圖\ref{compare}。可以看出來%
模擬結果此夾角角度%
在任意時刻都小於0.1 degree%
。這代表我們的方法%
是可行的。 
\begin{figure}[th]
\caption{{}模擬條件(SI units): I=0.002; Is=0.0008;
g=9.8; M=1 ; arm=.04; spin freq= 20 Hz; Initial angle from vertical 54.57
degree; Sampling rate 2500 Hz.}
\begin{center}
\fbox{\input{compare.pgf}}
\end{center}
\end{figure}

使用$\omega (t+dt)$來做t時間的%
轉動近似讓我們的程%
式碼變得非常簡單,%
比書中\cite[Page 301, Equation 10.24]{titterton}給%
出的轉動向量近似公%
式簡單太多了。簡單%
的程式碼代表之後在%
做速度優化,或轉寫%
成速度快的編譯式語%
言上,會簡單非常多%
。主程式叫做\texttt{RGCordTrans.py}%
。我使用的Python版本為Python
2.7.5 - Anaconda 1.8.0 (32-bit),NumPy 1.7.1,SciPy 0.13.0%
,Matplotlib 1.3.1。不同版本的%
Python﹝如Python 3.x﹞會有一些%
指令的不同需要做小%
修改,可以聯絡我我%
可以幫做修正。\texttt{RGCordTrans%
主程式可調整的}參數%
為

\begin{quotation}
M = 1 \#mass in kg

\#R = 0.025 \#radius of disk in m

\#L = 0.01 \#width of disk in m

arm = 0.04 \#level arm of Center of mass in m

Iy= 0.002 \#moment of inertial, substitute disk formula if needed
Iy=0.25*M*R**2 + 1/12*M*L**2 +M*arm**2

Iz= 0.0008 \#Iz = 0.5*M*R**2

Ix= 0.002

g = 9.8 \#gravity m/s\symbol{94}2

F = np.array([0,0,-M*g]) \#gravity f

freq = 20 \#top rotating angular speed in hertz

tn=1.2*1 \# end of simulation time, take longer time if tn \TEXTsymbol{>} 10
seconds

t0=0 \# start of time zero

samplerate = 2000 \# rate of iteration in Hz

N = (tn-t0)*samplerate \#number of steps

h = (tn-t0)/N \#time step interval
\end{quotation}

主程式以四階Rugge Kutta (\texttt{%
topRK(wi,torquei)})直接數值解陀螺%
尤拉方程\texttt{topEOM(wi,torquei)}得到%
貼體角速度,接著用%
遞推公式\ref{iterateeq0}來得到%
陀螺特徵軸xyz軸的單位%
向量\texttt{cordvec}對時間的變%
化。cordvec陣列的維度\texttt{%
shape=(N,3,7)。其中維度為7的%
方向的首要三個陣列%
就是xyz軸。}

DCM遞推的程式為以下五%
行

\begin{quotation}
for i in range(1,N+1):

w[i,:]=topRK(w[i-1,:],Tau\_body\_temp)

TrackMat\_temp = np.dot(TrackMat\_temp,CK(w[i,:]*h)) \qquad \# wi here,

for j in range(3):

cordvec[i,:,j]=np.dot(TrackMat\_temp,np.eye(3)[j,:])
\end{quotation}

注意力矩我們是用沿%
著body軸的Tau\_body\_temp。

\part{\protect\bigskip \protect\bigskip 在綁缚式%
慣性感測器姿態演算%
的應用}

上面的五行遞推DCM的程%
式可以獨立出來應用%
在感測器的姿態估測%
上。由於DCM是考慮貼體%
角速度$\vec{\omega}$,因此主%
要應用在綁缚式慣性%
感測元件﹝strap-down IMU﹞上%
,此類包含範圍涵蓋%
陀螺儀角速度感測器%
,微機電角速度感測%
器,等諸如此類的元%
件。只要把元件讀到%
的角速度資料輸出,%
然後正確的輸入到我%
們這邊的DCM遞推程式,%
就可以得到物件的姿%
態。這邊大部分的書%
籍都無法從原理說起%
,都只有提供公式,%
而且也很少詳細的說%
明公式如何使用。不%
過要補充的是,我這%
邊並沒有噪音過濾的%
功能,因為任何的量%
測儀器都會有噪音,%
因此在輸入DCM遞推前,%
必須先經過過濾,一%
般常見的過濾方法是%
卡爾曼濾波,這裡不%
作噪音濾波的討論。

這邊似乎是第一個結%
合尤拉方程與姿態估%
測的實際應用的例子%
。另外這裡的實際應%
用例子,也提供了一%
個,以尤拉方程來驗%
證姿態估測演算法的%
一個實例。

\part{\protect\bigskip 補充}

陀螺等周速運動(Figure \ref%
{figure_uniform})的初始值條件如%
何計算呢?等周速的條%
件在Goldstein第二版5-77式給出%
\begin{equation}
Mgl=\dot{\phi}\left( I_{3}\omega _{3}-I_{1}\dot{\phi}\cos \theta _{0}\right)
\end{equation}%
,不過此式是由尤拉%
角(euler angles)給出,但我們%
需要的是anguler velocity along body的初%
始值,因此我們必須%
轉換尤拉角到anguler velocity along body%
,方法如下。上式中$%
\omega _{3}$即為我們的$\left( \omega
_{z}\right) _{b}$,這裡是20 Hz,$\theta _{0}$%
即為我們之前的orien向量%
所定,此模擬中是取45%
度角,由上式可求出%
兩組$\dot{\phi}(t_{0})$。另外尤拉%
角跟anugler velocity along body的關係式%
在Goldstein 4-125式給出%
\begin{eqnarray}
(\omega _{x})_{b} &=&\dot{\phi}\sin \theta \sin \psi +\dot{\theta}\cos \psi
\\
(\omega _{y})_{b} &=&\dot{\phi}\sin \theta \cos \psi -\dot{\theta}\sin \psi
\\
(\omega _{z})_{b} &=&\dot{\phi}\cos \theta +\dot{\psi}
\end{eqnarray}%
知道$\dot{\phi}(t_{0})$、$\theta _{0}$、$(\omega
_{z})_{b}$,我們由第三條求%
出$\dot{\psi}(t_{0})$,再把$\dot{\psi}(t_{0})$%
帶到第一二條後就可%
得到$(\omega _{x}(t_{0}),\omega _{y}(t_{0}))_{b}$,這%
樣我們就得到anguler velocity along body%
的初始值。因為$\dot{\phi}(t_{0})$%
有兩組,因此解出的%
貼體角速度也會有兩%
組,兩組的物理意義%
分別如下,一種情況%
是fast top,這個狀況相當%
於重力的影響遠小於%
總角動量$L$,因此這個%
特別的例子基本上相%
當於忽略重力,而陀%
螺基本上會像一個free top%
一樣進行precession。另一種%
狀況是slow top,也就是上%
面模擬結果中第四種%
的狀況,這裡提供的%
python程式所有情況都可%
以模擬。另外一個特%
殊的情況是在fast top的情%
形下,如果初始值$\theta
_{0}=0$,也就是陀螺z軸的%
起始狀態是垂直於水%
平面的,這樣的話陀%
螺幾乎會像靜止不動%
一樣,我們也叫這情%
況做sleeping top。

\begin{remark}
要陀螺具有Precession and Nutation的%
動作,L/$\Delta L$必須要大,%
如果L小於$\Delta L$,則只會%
有陀螺質量受重力影%
響往下倒下的運動(不%
過這對檢查程式有沒%
有錯誤很有幫助!),理%
想上L至少要大於$\Delta L$,%
最好L大大於$\Delta L$。化成%
數值上的比較:這代%
表%
\begin{equation}
L\gg \Delta L\Rightarrow I\cdot 2\pi f\gg \vec{\Gamma}\Delta t\Rightarrow
I\cdot 2\pi f\gg \vec{r}\times \vec{F}\cdot 1/f\Rightarrow f\gg \sqrt{\frac{%
arm\cdot Mg\cdot \sin (\theta )}{2\pi I\cdot G}}
\end{equation}%
where $\theta $ is gyro's tilt angle and G is moment of inertial geometry
factor. 考慮$\Delta t$的量級大約%
是陀螺轉幾圈的時間%
(characteristic time),量級上約是$\sim
1/f$,若假設arm是10 cm, M = 1kg, g=10 m/s$^{2}$%
, I = 0.5M(0.05)$^{2}$,則f最少要10 Hertz以%
上。因此我們將以這%
些參數比較f = 1, 10, 50 Hertz所給%
出的陀螺運動。
\end{remark}

\href{https://drive.google.com/file/d/0B96HmLH-SQVmM1dvYlFiQm9ESGM/edit?usp=sharing%
}{\underline{\color{blue}\smash{Python code can be found here.}}}

\href{http://tinypic.com/r/10cw9yf/8}{\underline{\color{blue}%
\smash{3D
animation.}}}

This document is prepared with Scientific Workplace 5.0 and typeset with Tex
Live 2013 (Xelatex). \href{http://whymranderson.blogspot.tw/2014/03/how-to-convert-swp-50-special-unicode.html%
}{\underline{\color{blue}\smash{Here is how.}}}

If you feel like supporting this work, you can \href{https://www.paypal.com/us/cgi-bin/webscr?cmd=_flow&SESSION=WlwN4JPJMnohiYq8N9IuRiEIHqDEyNxauM_sF1u1Qh3M5sQBsssTjYFi4yu&dispatch=5885d80a13c0db1f8e263663d3faee8d5402c249c5a2cfd4a145d37ec05e9a5e%
}{\underline{\color{blue}\smash{donate here}}}.

\begin{thebibliography}{9}
\bibitem{goldstein} Herbert Goldstein, \emph{Classical Mechanics}. Addison
Wesley, Massachusetts, 2nd Edition, 1980

\bibitem{tong} David Tong, \emph{Classical Dynamics University of Cambridge
Part II Mathematical Tripos.} Cambridge UK, 2004-2005, (Course note,
available on the web)

\bibitem{matlab} \href{http://www.mathworks.com/help/matlab/ordinary-differential-equations.html%
}{\underline{\color{blue}%
\smash{Matlab online documentation - Ordinary
differential equations.}}}, Matlab R2014a

\bibitem{徐鐘} 徐小明 钟万勰,%
\textit{刚体动力学的四元%
数表示及保辛积分},%
《应用数学和力学》 2014%
, 35(1): 111

\bibitem{hasbun} Javier E. Hasbun, \emph{Classical Mechanics with Matlab
Appications.} Jones and Bartlett Publishers, London UK, 2009

\bibitem{titterton} D.H. Titterton and J.L. Weston, \textit{Strapdown
inertial navigation technology}, Peter Peregrinus Ltd., London UK, 1997

\bibitem{pixarnote} David Baraff\textit{, }\href{http://graphics.cs.cmu.edu/courses/15-869-F08/lec/14/notesg.pdf%
}{\underline{\color{blue}%
\smash{Physically Based Modeling - Rigid Body
Simulation}}}, Pixar Animation Studios notes
\end{thebibliography}

\end{document}
