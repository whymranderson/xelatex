
\documentclass[12pt,a4paper]{article}
%%%%%%%%%%%%%%%%%%%%%%%%%%%%%%%%%%%%%%%%%%%%%%%%%%%%%%%%%%%%%%%%%%%%%%%%%%%%%%%%%%%%%%%%%%%%%%%%%%%%%%%%%%%%%%%%%%%%%%%%%%%%%%%%%%%%%%%%%%%%%%%%%%%%%%%%%%%%%%%%%%%%%%%%%%%%%%%%%%%%%%%%%%%%%%%%%%%%%%%%%%%%%%%%%%%%%%%%%%%%%%%%%%%%%%%%%%%%%%%%%%%%%%%%%%%%
\usepackage{amsmath}
\usepackage{fontspec}
\usepackage{xeCJK}
\setmainfont{Times New Roman}
\setsansfont{Verdana}
\setmonofont{Courier New}                    % tt
\setCJKmainfont{微軟正黑體} 
\usepackage[left=0.95in,right=0.95in,top=2cm,bottom=2.54cm]{geometry}
\usepackage{unicode-math}
\usepackage{graphicx}
\usepackage[hidelinks]{hyperref}
\usepackage{pgf}

\setcounter{MaxMatrixCols}{10}
%TCIDATA{OutputFilter=LATEX.DLL}
%TCIDATA{Version=5.00.0.2606}
%TCIDATA{<META NAME="SaveForMode" CONTENT="1">}
%TCIDATA{BibliographyScheme=Manual}
%TCIDATA{Created=Monday, January 13, 2014 11:43:31}
%TCIDATA{LastRevised=Monday, August 04, 2014 16:28:32}
%TCIDATA{<META NAME="GraphicsSave" CONTENT="32">}
%TCIDATA{<META NAME="DocumentShell" CONTENT="International\Traditional Chinese Article">}
%TCIDATA{CSTFile=Traditional Chinese.cst}

\newtheorem{theorem}{Theorem}
\newtheorem{acknowledgement}[theorem]{Acknowledgement}
\newtheorem{algorithm}[theorem]{Algorithm}
\newtheorem{axiom}[theorem]{Axiom}
\newtheorem{case}[theorem]{Case}
\newtheorem{claim}[theorem]{Claim}
\newtheorem{conclusion}[theorem]{Conclusion}
\newtheorem{condition}[theorem]{Condition}
\newtheorem{conjecture}[theorem]{Conjecture}
\newtheorem{corollary}[theorem]{Corollary}
\newtheorem{criterion}[theorem]{Criterion}
\newtheorem{definition}[theorem]{Definition}
\newtheorem{example}[theorem]{Example}
\newtheorem{exercise}[theorem]{Exercise}
\newtheorem{lemma}[theorem]{Lemma}
\newtheorem{notation}[theorem]{Notation}
\newtheorem{problem}[theorem]{Problem}
\newtheorem{proposition}[theorem]{Proposition}
\newtheorem{remark}[theorem]{Remark}
\newtheorem{solution}[theorem]{Solution}
\newtheorem{summary}[theorem]{Summary}
\newenvironment{proof}[1][Proof]{\noindent\textbf{#1.} }{\ \rule{0.5em}{0.5em}}
\input{tcilatex}

\begin{document}

\title{\bigskip \textbf{以尤拉方程與%
方向餘弦遞推來實現%
剛體的轉動數值模擬%
與姿態估測}}
\author{}
\maketitle

\begin{abstract}
第一部分為完整清楚%
的剛體轉動尤拉方程%
的詳細推導,第二部%
分以此基礎嚴謹並且%
淺顯自然地導出方向%
餘弦遞推公式,第三%
段結合以上所有方法%
將之應用上陀螺三維%
運動的轉動數值模擬%
與姿態估測的實際例%
子上,並且與其他方%
法比較確認其正確性%
。
\end{abstract}

由於古典力學中的剛%
體轉動已經是很古老%
的題目,少有書籍會%
在尤拉轉動運動方程%
的推導詳細著墨,導%
致大部分書籍上的推%
導已經趨於簡化,這%
造成我在數值模擬的%
時候遇到非常非常多%
不清楚而且觀念模糊%
的地方,讓我非常受%
挫,追根究柢似乎是%
沒有人願意花時間去%
全面清楚地解釋。也%
由於找不到清楚解釋%
的書,導致我決定自%
己推導,也因此花了%
不少時間,這邊我將%
全面完整且清楚的推%
導寫下來﹝我自己認%
為沒有漏失任何一個%
不清楚的環節,不過%
還盼各界指教﹞,一%
來我可以確認我自己%
全盤了解,二來也可%
以讓別人省下不少不%
必要的摸索嘗試。舉%
陀螺之例來說,幾乎%
所有古典力學的書上%
說明貼體角速度的尤%
拉轉動運動方程之後%
,就會轉而求諸Euler angles來%
得到Lagrangian,接著用elliptical integral%
解出解析解。或者,%
以數值方法解其Lagrangian,%
然後以Euler angles模擬其運動%
。不過,若對尤拉方%
程透徹理解,我們這%
邊展示,只要給定初%
始值,就可以利用轉%
動向量(rotation vector)直接簡單%
地數值模擬剛體特徵%
軸的轉動。這裡的方%
法屬於姿態估測學中%
的方向餘弦遞推法(iteration
of direction cosine matrix, DCM),不過如同%
前所述,這邊不只介%
紹公式,還以淺顯易%
懂、清楚的方式給出%
原理。因此,只要有%
基礎的線性代數矩陣%
知識,就可以掌握此%
方法。這樣子學習到%
的才是最精華的部分%
,也因為學習到的是%
最基礎的原理,以後%
在應用上也會更加的%
正確以及順暢。這裡%
詳述的這些理論基本%
上屬於物理古典力學%
,不過反而在航空太%
空領域中的慣性導航%
才有較多實作實務的%
講解,但也缺乏基礎%
原理的講解,只給出%
公式,這樣在應用層%
面的時候會發生比較%
多不必要的試誤的嘗%
試。也由於我實在找%
不太到解釋完整的書%
籍或文獻,我發現是%
值得去花時間整理不%
同領域且觀念連結錯%
綜複雜的東西。我也%
找不到轉動運動這個%
複雜的觀念上實作與%
理論兼顧的清楚解釋%
,就算是英文書籍也%
是非常艱澀難懂,觀%
念跳來跳去,因此這%
邊嘗試先以中文寫下%
實作與理論兼顧下最%
清楚最淺顯的推導,%
之後有時間再翻成英%
文。

這裡詳述的方法可以%
廣泛的運用於任何剛%
體轉動或其尤拉方程%
,因此,此處所涉及%
的方向餘弦遞推的完%
整理解還可應用上其%
他相關領域如綁附式%
慣性感測\cite[Ch 3.6.4]{titterton},姿%
態估測,電腦3D圖學\cite[Ch
2.3]{pixarnote}及電腦剛體模擬%
(simulation of rigid body)。這邊我們呈%
現以此方法數值模擬%
完整的三維陀螺運動%
,並且模擬結果也與%
其它文獻\cite{hasbun}作比較確%
認了其正確性,也因%
為如此,這邊提供的%
Python程式已經可以直接%
應用上以上所說的幾%
點應用領域。並且,%
這邊提供的詳細解說%
也適合當作大學教材%
中尤拉方程與方向餘%
弦遞推的教材例題。

這裡將剛體轉動分解%
成為很多個t到t+dt時間的%
微小轉動,這邊有三%
個大重點:

\begin{enumerate}
\item 第一部分為最完整%
的剛體尤拉方程的推%
導證明。因為要做轉%
動軸的轉動數值積分%
必須要對尤拉方程的%
貼體角速度有最正確%
的理解。這邊補充了%
Goldstein Classical Mechanics\cite{goldstein}中證明觀%
念跳來跳去的缺失,%
以及大部分古典力學%
書上解釋非常模糊的%
地方。轉動理論在大%
部分航太導航書籍較%
有教導,但可惜講述%
的非常的複雜,缺乏%
與轉動原理尤拉方程%
的結合,並且大部分%
也無提供實際實例操%
作這重要的一環,這%
裡則提供完整的陀螺%
模擬實例。

\item 接著藉由第一段尤%
拉方程的推導來嚴謹%
的證明Euler equation中的貼體%
角速度(angular velocity along body frame)可直%
接用於建立剛體特徵%
軸與lab frame間的主動與被%
動轉動矩陣\thinspace ,並以%
此推導出方向餘弦法%
的主要原理來積分剛%
體轉動,追蹤每一時%
刻的剛體特徵軸在lab frame%
的位置。

\item 接著說明了在近似t%
到t+dt時間的微小特徵軸%
轉動時我們以在t+dt時間%
的body角速度來近似,即%
以$\vec{\omega}_{b}\left( t+dt\right) $來建立dt%
時間內的轉動矩陣$\footnote{%
The path order exponential of $\vec{\omega}_{b}$ from time t to t+dt. This
will be discussed more in the text.}$。最後將以%
上方法以python程式寫出%
,並且跟文獻\cite{hasbun}做比%
較。
\end{enumerate}

\bigskip

首先我們先討論向量%
變化量在不同觀測座%
標中的關係。由於當%
我實際在解這問題時%
我發現Goldstein classical mechanics書中還%
有幾點證明還不清楚%
,因此這邊寫上我認%
為可以補充書上的推%
導證明。

\begin{equation}
\left( \frac{d\vec{L}}{dt}\right) _{s}=\left( \frac{d\vec{L}}{dt}\right)
_{b}+\vec{\omega}\times \vec{L}
\end{equation}

此公式如何而來?此%
公式為一隨時間變動%
的向量在恆定座標與%
非恆定座標(此例為轉%
動中座標)之間線性變%
換的結果。

\begin{figure}[th]
\caption{{}}
\begin{center}
\fbox{\input{rateofchange.pgf}}
\end{center}
\label{firstfig}
\end{figure}
\bigskip 

首先考慮一恆定座標%
S(space),一轉動座標b(body),%
為了方便討論座標軸%
的主被動性與座標轉%
換的左右手法則,我%
們這邊方便的先假設$%
\hat{S}_{x}, \hat{b}_{x}$兩軸重合,%
因此圖中顯示了body frame沿%
著$+\hat{S}_{x}$遵守右手定則%
逆時針轉了$\Omega $角度,%
依右手定則此角位移%
向量$\hat{\Omega}$會在$+\hat{S}_{x}$方向%
。但是接下來的推導%
以及所有公式都適用%
任意的座標旋轉,這%
邊是為了方便討論矩%
陣的主動被動的方向%
性,以及在之後的推%
導方便我們追蹤正負%
號以及矩陣主動被動%
意義的改變,因此在%
圖中做了一個方便我%
們思考的情形。另外%
,大部分書上在討論%
座標轉換時有時候給%
的公式是遵守左手定%
則,但這與物理定律%
所採納的右手定則相%
反,因此這邊我寫下%
完整的右手定則的推%
導,希望之後的人不%
需要像我一樣花了大%
半時間在轉換不同公%
式間左手右手定則帶%
來的正負號的改變。

\bigskip 依照圖\ref{firstfig}所示,%
我們可以寫下$\vec{A}$向量%
在S,b座標間的關係%
\begin{equation*}
\left( \vec{A}\right) _{b}=\underset{\text{passive, r.h.}}{\Omega }\left( 
\vec{A}\right) _{s}
\end{equation*}%
其中$\Omega $是s frame到b frame的座%
標轉換矩陣,因為是%
轉換座標軸,因此矩%
陣取被動含意,並且%
我們採用右手定則,%
因此逆時針方向為正%
方向。接下來只要有%
用到矩陣的運算我都%
會標明主被動及左右%
手(r.h. right-hand or l.h. left-hand),這對接%
下來的推倒很重要。

若我們考慮$\Omega $的角度%
很小$\Omega \rightarrow d\Omega $(infinitesimal rotation),%
則$d\Omega $矩陣與unity matrix相去%
不遠,可以寫成$1$(unity matrix) +$%
\epsilon $(infinitesimal matrix),$\epsilon $具有%
antisymmetric matrix的特性\cite[p. 169]{goldstein},%
帶入上式%
\begin{equation*}
\left( \vec{A}\right) _{b}=\underset{\text{passive, r.h.}}{\left( 1+\epsilon
\right) }\left( \vec{A}\right) _{s}
\end{equation*}%
infinitesimal matrix有個特性,很容%
易自行驗證,%
\begin{equation*}
\underset{\text{r.h., passive or active}}{\epsilon }=\left[ 
\begin{array}{ccc}
0 & \epsilon _{3}\geq 0 & -\epsilon _{2}\leq 0 \\ 
-\epsilon _{3} & 0 & \epsilon _{1}\geq 0 \\ 
\epsilon _{2} & -\epsilon _{1} & 0%
\end{array}%
\right] \text{, }\underset{\text{l.h., passive or active}}{\epsilon }=\left[ 
\begin{array}{ccc}
0 & -\epsilon _{3}\leq 0 & \epsilon _{2}\geq 0 \\ 
+\epsilon _{3} & 0 & -\epsilon _{1}\leq 0 \\ 
-\epsilon _{2} & \epsilon _{1} & 0%
\end{array}%
\right]
\end{equation*}

\bigskip 現在我們考慮$\vec{A}$是$+%
\hat{b}_{y}$軸的狀況,不過考%
慮相同矩陣$\left( 1+\epsilon \right) $的%
主動特性,也就是主%
動轉向量,這樣的話%
轉動方向會與原本的%
方向相反,變左手定%
則,我們會得到%
\begin{equation*}
\left( \hat{S}_{y}\right) _{s}=\underset{\text{active, l.h.}}{\left(
1+\epsilon \right) }\times \left( \hat{b}_{y}\right) _{s}
\end{equation*}%
整理一下%
\begin{equation*}
\left( \hat{b}_{y}\right) _{s}=\underset{\text{active, r.h.}}{\underbrace{%
\left[ \left( 1+\epsilon \right) \right] ^{T}}}\times \left( \hat{S}%
_{y}\right) _{s}=\underset{\text{active, l.h.}}{\left( 1-\epsilon \right) }%
\times \left( \hat{S}_{y}\right) _{s}
\end{equation*}%
代入上面r.h. $\epsilon $的公式(%
因$\epsilon $還是原本的矩陣)%
,整理一下%
\begin{equation*}
\left( \hat{b}_{y}\right) _{s}-\left( \hat{S}_{y}\right) _{s}=-\left[ 
\begin{array}{ccc}
0 & \epsilon _{3}\geq 0 & -\epsilon _{2}\leq 0 \\ 
-\epsilon _{3} & 0 & \epsilon _{1}\geq 0 \\ 
\epsilon _{2} & -\epsilon _{1} & 0%
\end{array}%
\right] \times \left( \hat{S}_{y}\right) _{s}
\end{equation*}%
利用向量外積,上式%
也可寫成%
\begin{equation*}
\left( \hat{b}_{y}\right) _{s}-\left( \hat{S}_{y}\right) _{s}=\left( \vec{%
\epsilon}\right) _{s}\times \left( \hat{S}_{y}\right) _{s}
\end{equation*}%
其中$\vec{\epsilon}=\left[ 
\begin{array}{c}
\epsilon _{1} \\ 
\epsilon _{2} \\ 
\epsilon _{3}%
\end{array}%
\right] _{s}$為一向量,在S frame中%
的分量為$\epsilon _{1}$,$\epsilon _{2}$%
,$\epsilon _{3}$。

現在我們將上式跟微%
小轉動公式Rodrigues rotation formula比%
較%
\begin{equation*}
\vec{r}^{\prime }-\vec{r}=d\vec{\Omega}\times \vec{r}
\end{equation*}%
$d\vec{\Omega}$是r到r'的r.h.角位移%
向量\thinspace ,因此我們得%
到$\vec{\epsilon}=d\vec{\Omega}$,$d\vec{\Omega}$就是%
s frame到b frame的角位移向量%
(follow r.h. rule)%
\begin{equation*}
\left( \hat{b}_{y}\right) _{s}-\left( \hat{S}_{y}\right) _{s}=\left( d\vec{%
\Omega}\right) _{s}\times \left( \hat{S}_{y}\right) _{s}
\end{equation*}%
這一點很重要,因為%
我們將證明此$\left( d\vec{\Omega}\right)
_{s}$跟接下來我們要推導%
的尤拉公式中的貼體%
角速度$\vec{\omega}$有直接相%
關性,並且以此來做%
我們模擬剛體轉動的%
基礎。

以上的討論是考慮$\vec{A}$%
向量不隨時間變動的%
情況,接下來我們必%
須討論$\vec{A}$以及b frame皆隨%
時間變動的狀況。

\bigskip 
\begin{figure}[th]
\caption{Rate change of a vector observed in a inertial and non-inertial
frame.}
\begin{center}
\fbox{\input{rateofchanget2tdt.pgf}}
\end{center}
\label{ratevecfig}
\end{figure}

\bigskip 在時間t時我們令S與b
frame重合,過了dt時間原%
本的$\vec{A}$向量加了一改%
變量$d\vec{A}$,並且b frame依右%
手定則轉動了一微小%
角度(infinitisemal rotation),這邊考%
慮相同的方便性我們%
假定$S_{x}$與$b_{x}$重和,但%
所有推導均考慮最一%
般性。在此前提下,%
向量$\vec{A}$在t到t+dt時間皆%
符合%
\begin{equation*}
\left( \vec{A}\right) _{s}=\left( \vec{A}\right) _{b}
\end{equation*}%
接著,在t+dt時間$\vec{A}+d\vec{A}$%
向量在s與b frame間的關係%
為%
\begin{equation*}
\left( \vec{A}+d\vec{A}\right) _{b}=\underset{\text{passive, r.h.}}{\Omega }%
\left( \vec{A}+d\vec{A}\right) _{s}
\end{equation*}%
$\Omega $為s, b frame轉動矩陣(passive r.h.)%
,此$\Omega $矩陣與上一段$%
\vec{A}$不變動的情況的$\Omega $%
矩陣完全相同,我們%
取s到b frame的轉動為微小%
量,$\Omega \rightarrow d\Omega $,上式依%
之前所述的原理可寫%
成%
\begin{equation*}
\left( \vec{A}+d\vec{A}\right) _{b}=\underset{\text{passive, r.h.}}{\left(
1+\epsilon \right) }\left( \vec{A}+d\vec{A}\right) _{s}
\end{equation*}%
要強調這邊的$\epsilon $矩陣%
跟之前上一段的$\epsilon $矩%
陣是完全相同的,展%
開上式%
\begin{equation*}
\left( \vec{A}\right) _{b}+\left( d\vec{A}\right) _{b}=\left( \vec{A}\right)
_{s}+\left( d\vec{A}\right) _{s}+\epsilon \left( \vec{A}\right)
_{s}+\epsilon \left( d\vec{A}\right) _{s}
\end{equation*}%
利用之前知道的$\left( \vec{A}%
\right) _{s}=\left( \vec{A}\right) _{b}$,以及忽%
略高階項$\epsilon \left( d\vec{A}\right) _{s}$%
,重新整理成%
\begin{equation*}
\left( d\vec{A}\right) _{s}=\left( d\vec{A}\right) _{b}-\underset{\text{r.h.}%
}{\epsilon }\left( \vec{A}\right) _{s}
\end{equation*}%
依之前所述原理代入r.h. 
$\epsilon $的公式,並且利用%
向量外積%
\begin{eqnarray*}
\left( d\vec{A}\right) _{s} &=&\left( d\vec{A}\right) _{b}-\left[ 
\begin{array}{ccc}
0 & \epsilon _{3}\geq 0 & -\epsilon _{2}\leq 0 \\ 
-\epsilon _{3} & 0 & \epsilon _{1}\geq 0 \\ 
\epsilon _{2} & -\epsilon _{1} & 0%
\end{array}%
\right] \left( \vec{A}\right) _{s} \\
&=&\left( d\vec{A}\right) _{b}-\left( \vec{A}\right) _{s}\times \left( d\vec{%
\Omega}\right) _{s} \\
&=&\left( d\vec{A}\right) _{b}+\left( d\vec{\Omega}\right) _{s}\times \left( 
\vec{A}\right) _{s}
\end{eqnarray*}%
因為這裡的$\epsilon $矩陣與%
上一段的$\epsilon $矩陣是一%
樣的,因此我們也可%
以用上之前轉動公式%
所推導的微小轉動矩%
陣$\epsilon $所對應的轉動向%
量$\left( d\vec{\Omega}\right) $,這樣我%
們就得到了rate of change of a vector in
rotating frame公式%
\begin{equation}
\left( d\vec{A}\right) _{s}=\left( d\vec{A}\right) _{b}+\left( d\vec{\Omega}%
\right) _{s}\times \left( \vec{A}\right) _{s}  \label{rateofdomega}
\end{equation}%
這邊要強調,因為這%
裡的$\epsilon $矩陣與上一段%
的$\epsilon $矩陣是一樣的,%
所以證明了$d\vec{\Omega}$所對%
應的向量就是s frame轉到b
frame的角位移向量(r.h.),這%
樣強調的目的是,我%
們會以此特性模擬剛%
體轉動。另外要注意%
的是$\vec{A}$與$d\vec{\Omega}$是沿著t%
時間的s frame取的投影量%
。這邊值得一提的是%
,傳統公式大多寫成%
\begin{equation*}
\left( d\vec{A}\right) _{s}=\left( d\vec{A}\right) _{b}+\left( d\vec{\Omega}%
\right) _{b}\times \left( \vec{A}\right) _{b}
\end{equation*}%
這邊$\vec{A}$與$d\vec{\Omega}$則是沿%
著t+dt時間的b frame取的投影%
量,因若考慮$\left( \vec{A}\right) _{b}$%
那我們的微小矩陣是%
作用在body frame的$\vec{A}$上面,%
因此在利用外積特性%
來指定轉動向量時我%
們也考慮$\left( d\vec{\Omega}\right) _{b}$在b
frame的投影$, $因此事實%
上$\vec{A}$與$d\vec{\Omega}$取s或b frame分%
量都是可以的,只要%
矩陣運算後出來的結%
果是一樣的就可以。%
\footnote{%
Goldstein在書上在這部分的%
說明比較少,不過他%
也提到只要在微分取%
完後,向量沿space或body取%
分量都是可以的\cite[p. 176]%
{goldstein}。}

\bigskip 上式取微分即得到%
一般常見的形式%
\begin{equation}
\left( \frac{d\vec{A}}{dt}\right) _{s}=\left( \frac{d\vec{A}}{dt}\right)
_{b}+\left( \vec{\omega}\right) _{s}\times \left( \vec{A}\right) _{s}
\label{rateofchange}
\end{equation}%
其中$\left( \vec{\omega}\right) _{s}$為s frame到b
frame的瞬時角速度。

嚴謹的定義了$d\vec{\Omega}$與$%
\left( \vec{\omega}\right) _{s}$後,我們接%
著需要討論如何從$\left( \vec{%
\omega}\right) _{s}$求回相對應的轉%
動矩陣,這邊你會認%
為,不是將$\left( \vec{\omega}\right) _{s}$%
的xyz分量帶入之前$1+\epsilon $%
矩陣中的$\epsilon _{1}\epsilon _{2}\epsilon _{3}$%
就可以了嗎,這樣是%
不行的,因為從之前%
微小轉動的推導可以%
看出,$\epsilon _{1}\epsilon _{2}\epsilon _{3}$是%
符合特定的antisymmetric matrix properties%
的,但任意的角速度%
向量$\left( \vec{\omega}\right) _{s}$可不然%
。這邊我們利用Calvin Klein
parameter來近似原本的轉動%
矩陣﹝CK parameters矩陣基本上%
與轉動公式Rodrigues rotation formula同%
源\cite{goldstein}﹞,這邊我們%
給他一個新代號$CK(d\vec{\Omega})$%
,當然,接下來只要%
是矩陣運算我們都會%
寫上$CK$的主被動及左右%
手性質\footnote{力矩給出的%
角速度是遵守右手定%
則(counterclockwise),所以CK矩陣必%
須使用其active counterclockwise sense才能%
描述正確向量轉動,%
要小心,因大部分書%
上(如Goldstein)給的公式都是%
active clockwise(follow左手定則)(舉例%
如書上的Caley Klein parameter rotation matrix)%
,因此差一個負號,%
這裡我花了許多時間%
把文獻上所有公式轉%
成了正確的右手定則%
。}。%
\begin{eqnarray*}
\underset{\text{r.h.}}{CK(d\vec{\Omega})} &=&\left[ 
\begin{array}{ccc}
a^{2}+b^{2}-c^{2}-d^{2} & 2(bc-ad) & 2(bd+ac) \\ 
2(bc+ad) & a^{2}+c^{2}-b^{2}-d^{2} & 2(cd-ab) \\ 
2(bd-ac) & 2(cd+ab) & a^{2}+d^{2}-b^{2}-c^{2}%
\end{array}%
\right] \text{,} \\
\text{with }a &=&\cos \left( \frac{\left\vert d\vec{\Omega}\right\vert }{2}%
\right) \text{, b, c, d = component of }d\hat{\Omega}\cdot \sin \left( \frac{%
\left\vert d\vec{\Omega}\right\vert }{2}\right)
\end{eqnarray*}%
現在,我們一再強調$d%
\vec{\Omega}$所對應的是s frame轉動%
到b frame,因此我們建立%
的$CK(d\vec{\Omega})$矩陣具有以下%
的特性,根據圖\ref{ratevecfig}%
,%
\begin{eqnarray*}
\left( \vec{A}\right) _{b} &=&\underset{\text{passive, r.h.}}{CK(d\vec{\Omega%
})}\left( \vec{A}\right) _{s} \\
\left( \hat{S}_{y}\right) _{s} &=&\underset{\text{active, l.h.}}{CK(d\vec{%
\Omega})}\left( \hat{b}_{y}\right) _{s}
\end{eqnarray*}%
或者也可寫成%
\begin{eqnarray}
\left( \vec{A}\right) _{s} &=&\underset{\text{active, l.h.}}{\underbrace{%
\left[ CK(d\vec{\Omega})\right] ^{T}}}\left( \vec{A}\right) _{b}
\label{frametrans} \\
\left( \hat{b}_{y}\right) _{s} &=&\underset{\text{active, r.h.}}{\underbrace{%
\left[ CK(d\vec{\Omega})\right] ^{T}}}\left( \hat{S}_{y}\right) _{s}
\label{vecrot}
\end{eqnarray}%
\ref{ratevecfig}若我們知道的是$%
\left( \vec{\omega}\right) _{s}$則可帶入$CK(\left( 
\vec{\omega}\right) _{s}\cdot dt)$來得到矩陣%
。以上兩式就是模擬%
或追蹤剛體的body frame的x,y,z%
軸轉動的基礎。

\begin{figure}[th]
\caption{How to apply rate-of-change-of-a-vector equation to a real
rotation. }
\begin{center}
\fbox{\input{SzBz.pgf}}
\end{center}
\label{szsbtdtfig}
\end{figure}

在我們進一步討論\ref%
{frametrans}及\ref{vecrot}式前,我們%
必須先說明我們如何%
應用上\ref{rateofchange}式來解剛%
體轉動。我們會把剛%
體轉動分解為很多的%
微小轉動,每一小段%
的微小轉動我們都會%
運用上圖\ref{ratevecfig}的原理%
,現在我們需要另外%
設定一個Lab frame\thinspace ,見圖%
\ref{szsbtdtfig},此為真正的觀%
測者所處在的inertial frame。%
考慮任意一段微小轉%
動t到t+dt,在t時刻時我%
們將剛體的principle axes設定%
為S frame,再將t+dt時刻剛體%
的principle axes設定為b frame,這樣%
代表s frame到b frame就是剛體t%
到t+dt的轉動。將\ref{rateofchange}%
式應用上這一段t到t+dt的%
微小轉動,並且考慮$%
\vec{A}$為剛體角動量$\vec{L}$,%
則我們得到%
\begin{equation}
\left( \Gamma \right) _{s}=\left( \frac{d\vec{L}}{dt}\right) _{s}=\left( 
\frac{d\vec{L}}{dt}\right) _{b}+\left( \vec{\omega}\right) _{s}\times \left( 
\vec{L}\right) _{s}  \label{liw}
\end{equation}%
這裡第一等號也用上%
牛頓定律。現在我們%
從\ref{rateofdomega}式知道$\left( d\vec{\Omega}%
\right) _{s}$是s到b frame的角位移,%
而這邊經由我們的設%
定,s到b frame正是我們剛%
體特徵軸從t到t+dt的角位%
移,因此$\left( \frac{d\vec{\Omega}}{dt}\right)
_{s}=\left( \vec{\omega}\right) _{s}$正是剛體的%
瞬時角速度(沿著t時間s
frame取分量),接著,因%
為s, b frame都是沿著body principle axes而%
取,因此沿s frame的角動%
量$\left( \vec{L}\right) _{s}$可以寫成%
\begin{equation*}
\left( \vec{L}\right) _{s}=\left[ 
\begin{array}{ccc}
I_{xx} & 0 & 0 \\ 
0 & I_{yy} & 0 \\ 
0 & 0 & I_{zz}%
\end{array}%
\right] \times \left( \vec{\omega}\right) _{s}
\end{equation*}%
再次注意$\left( \Gamma \right) _{s}$與$\left( 
\vec{\omega}\right) _{s}$與$\left( \vec{L}\right) _{s}$都是%
沿著t時刻的剛體特徵%
軸(也就是s frame)取的投影%
,並不是Lab frame的投影,%
這點要特別注意,基%
本上這代表,$\left( \vec{\omega}\right)
_{s}$就是貼體角速度!並%
且$\left( \Gamma \right) _{s}$是貼體的角%
動量!這裡大部分的%
書上都沒有給出恰當%
的原因\footnote{注意因為s frame%
會持續的改變所以$\left( \vec{%
\Gamma}\right) _{s}$不可取$\left( \vec{\Gamma}\right)
_{lab}$的值,同理$\left( \vec{\omega}\right)
_{s}$也不是$\left( \vec{\omega}\right) _{lab}$,%
兩者都必須經過轉換%
從lab轉到t時刻s frame。}。這%
邊我們證明了\ref{liw}式最%
後那一項中的兩個$\vec{\omega}
$是相同的\footnote{但我們必%
須強調,任意情況下%
,角速度$\left( \vec{\omega}\right) $在body%
轉動座標下的投影並%
不是body座標上觀察到的%
角速度!這是很常見%
的錯誤,這裡我們是%
有條件的考慮t到t+dt時刻%
的t時刻s,b座標重和。}。%
並且,力矩也必須從lab
frame轉換到t時間的s frame。代%
入$\vec{L}$並展開\ref{liw}式,我%
們就得到所謂的尤拉%
公式(Euler's equation)%
\begin{eqnarray}
\Gamma _{x}(t) &=&I_{x}\dot{\omega}_{x}+(I_{z}-I_{y})\omega _{y}\left(
t\right) \omega _{z}\left( t\right)  \notag \\
\Gamma _{y}(t) &=&I_{y}\dot{\omega}_{y}+(I_{x}-I_{z})\omega _{x}\omega _{z}
\label{eulereqbody} \\
\Gamma _{z}(t) &=&I_{z}\dot{\omega}_{z}+(I_{x}-I_{y})\omega _{x}\omega _{y} 
\notag
\end{eqnarray}%
注意$\vec{\Gamma}$及$\vec{\omega}$的x,y,z分%
量都是沿著t時刻的剛%
體特徵軸s frame取的分量%
,這點必須要強調。%
之後數值模擬的時候%
這點是重要的。

我們將在剛體特徵軸%
的每一段t到t+dt的分解運%
動運用上s frame轉到b frame的%
尤拉公式,也就是運%
用上\ref{eulereqbody}式。這也代%
表我們將持續地改變%
space frame來符合這個條件,%
但是當然我們會跟蹤%
space frame每一個變換的位置%
來達到模擬轉動運動%
,而這時候就會用上\ref%
{frametrans}及\ref{vecrot}式。

接下來應用上陀螺,%
若考慮陀螺的條件 $%
I_{x}=I_{y}\neq I_{z}$,\ref{eulereqbody}式可寫%
成%
\begin{equation}
\frac{d}{dt}\left[ 
\begin{array}{c}
\omega _{x} \\ 
\omega _{y} \\ 
\omega _{z}%
\end{array}%
\right] =\left[ 
\begin{array}{ccc}
0 & -\frac{I_{z}-I_{y}}{I_{x}} & 0 \\ 
-\frac{I_{x}-I_{z}}{I_{y}} & 0 & 0 \\ 
0 & 0 & 0%
\end{array}%
\right] \left[ 
\begin{array}{c}
\omega _{x} \\ 
\omega _{y} \\ 
\omega _{z}%
\end{array}%
\right] +\left[ 
\begin{array}{c}
\frac{\Gamma _{x}}{I_{x}} \\ 
\frac{\Gamma _{y}}{I_{y}} \\ 
\frac{\Gamma _{z}}{I_{z}}%
\end{array}%
\right]
\end{equation}%
如之前所強調,右邊%
所有項都是在時間為t%
時刻的s frame取得值,也%
因此以上的微分方程%
組可以用普通數值由%
拉法或四階Ruge Kutta求出左%
側$\omega _{x,y,z}(t+dt)$,也就是從$\vec{%
\omega}_{s}(t)$求得$\vec{\omega}_{s}(t+dt)$。不%
過對於任意的剛體轉%
動系統,只要能從\ref%
{eulereqbody}式右側$\vec{\omega}_{s}(t)$求得%
左側$\vec{\omega}_{s}(t+dt)$,都還是%
能夠適用接下來的模%
擬方法,也因此這裡%
描述的方法是具有任%
意一般性的,可以應%
用在任何的剛體轉動%
。有不少的數值方法%
可以解一般的非線性%
一階ODE尤拉方程\cite{matlab}。

現在我們都設定好了%
,我們也證明了貼體%
角速度在任意一段t到t+dt%
的時間中代表的意義%
,接下來我們就說明%
如何以這個特性及利%
用\ref{frametrans}及\ref{vecrot}式來追蹤%
s frame在各個微小轉動的%
位置,也就是我們要%
求陀螺特徵軸(s frame) $\hat{x}\hat{y}%
\hat{z}_{lab}(t_{i}$, i=1\symbol{126}N$)$ 在lab frame中的%
各個時間的位置。\bigskip

\begin{figure}[th]
\caption{Boby軸在每一分段t到t+dt%
的追蹤示意圖。}
\begin{center}
\fbox{\input{Zt0Zt1.pgf}}
\end{center}
\end{figure}

\begin{figure}[th]
\caption{陀螺的初始值設定%
。}
\begin{center}
\fbox{\input{orien.pgf}}
\end{center}
\end{figure}

假設陀螺特徵軸在lab frame%
的起始位置已知$\hat{x}\hat{y}\hat{z%
}_{lab}(t_{0})$,初始貼體角速%
度已知$\vec{\omega}_{s}(t_{0})$,下標%
代表的是觀測的frame。首%
先我們將s frame放在$\left( \hat{x}\hat{y}%
\hat{z}_{lab}(t_{0})\right) $,這樣依照圖%
\ref{szsbtdtfig}及其所述原理,b
frame的軸就是我們要求的%
$\hat{x}\hat{y}\hat{z}_{lab}(t_{1})$,我們先看$%
\hat{z}$軸,\ref{vecrot}式告訴我們%
\begin{equation*}
\underset{\text{active, r.h.}}{\underbrace{\left[ CK\left( \vec{\omega}%
_{s}\left( t_{0}\right) dt\right) \right] ^{T}}}\times \hat{z}_{0}\left(
t_{0}\right) =\hat{z}_{0}\left( t_{1}\right)
\end{equation*}%
其中$\hat{z}_{0}\left( t_{0})\text{,}\hat{z}%
_{0}(t_{1}\right) $代表時間為$t_{0}$與$%
t_{1}$的\^{z}軸在$t_{0}$時間的座%
標軸(也就是s frame)的投影%
,因此$\hat{z}_{0}\left( t_{0}\right) $為單%
位向量$\left[ 
\begin{array}{ccc}
0 & 0 & 1%
\end{array}%
\right] $。這樣子我們求得%
下一個z軸的位置在$t_{0}$%
的投影,不過我們得%
轉回lab frame,我們假設lab frame%
的$xyz$軸到陀螺初始位置%
$\hat{x}\hat{y}\hat{z}_{lab}(t_{0})$的轉動向量%
是$\vec{\Omega}_{0}, $這樣我們可%
以用$\vec{\Omega}_{0}$輕易的改變%
陀螺初始位置,運用%
上\ref{frametrans}式%
\begin{equation*}
\hat{z}_{lab}\left( t_{1}\right) =\underset{\text{passive, l.h.}}{%
\underbrace{\left[ CK\left( \vec{\Omega}_{0}\right) \right] ^{T}}}\times 
\hat{z}_{0}\left( t_{1}\right)
\end{equation*}%
注意這邊矩陣就取被%
動含意,結合以上兩%
式得到%
\begin{eqnarray*}
\hat{z}_{lab}\left( t_{1}\right) &=&\underset{\text{passive, l.h.}}{%
\underbrace{\left[ CK\left( \vec{\Omega}_{0}\right) \right] ^{T}}}\underset{%
\text{active, r.h.}}{\underbrace{\left[ CK\left( \vec{\omega}_{s}\left(
t_{0}\right) dt\right) \right] ^{T}}}\times \hat{z}_{0}\left( t_{0}\right) \\
&=&\underset{\text{passive, l.h.}}{\underbrace{\left[ CK\left( \vec{\Omega}%
_{0}\right) \right] ^{T}}}\underset{\text{active, r.h.}}{\underbrace{\left[
CK\left( \vec{\omega}_{s}\left( t_{0}\right) dt\right) \right] ^{T}}}\times %
\left[ 
\begin{array}{ccc}
0 & 0 & 1%
\end{array}%
\right]
\end{eqnarray*}%
這樣我們就從t$_{0}$時間%
得到t$_{1}$時間陀螺z軸的%
位置。

接著若我們知道$\hat{z}_{lab}\left(
t_{i}\right) $,以及從尤拉公%
式數值法解出的$\vec{\omega}%
_{s}\left( t_{0},t_{1},\cdots ,t_{i}\right) $,我們同%
樣可以求得$\hat{z}_{lab}\left( t_{i+1}\right) $%
,首先用\ref{vecrot}式%
\begin{eqnarray}
\hat{z}_{i}\left( t_{i+1}\right) &=&\underset{\text{active, r.h.}}{%
\underbrace{\left[ CK\left( \vec{\omega}_{s}\left( t_{i}\right) dt\right) %
\right] ^{T}}}\times \hat{z}_{i}\left( t_{i}\right)  \label{iterateeq} \\
&=&\underset{\text{active, r.h.}}{\underbrace{\left[ CK\left( \vec{\omega}%
_{s}\left( t_{i}\right) dt\right) \right] ^{T}}}\times \left[ 
\begin{array}{ccc}
0 & 0 & 1%
\end{array}%
\right]  \notag
\end{eqnarray}%
再用\ref{frametrans}式轉回到lab frame%
\begin{eqnarray*}
\hat{z}_{lab}\left( t_{i+1}\right) &=&\underset{\text{passive, l.h.}}{%
\underbrace{\left[ CK\left( lab\rightarrow t_{i}\right) \right] ^{T}}}\times 
\hat{z}_{i}\left( t_{i+1}\right) \\
&=&\underset{\text{passive, l.h.}}{\underbrace{\left[ CK\left( \vec{\Omega}%
_{0}\right) \cdot CK\left( \vec{\omega}_{s}\left( t_{0}\right) dt\right)
\cdot CK\left( \vec{\omega}_{s}\left( t_{1}\right) dt\right) \cdot \cdots
\cdot CK\left( \vec{\omega}_{s}\left( t_{i-1}\right) dt\right) \right] ^{T}}}%
\times \\
&&\hat{z}_{i}\left( t_{i+1}\right) \\
&=&\underset{\text{passive, l.h.}}{\underbrace{\left[ CK\left( \vec{\Omega}%
_{0}\right) \cdot CK\left( \vec{\omega}_{s}\left( t_{0}\right) dt\right)
\cdot CK\left( \vec{\omega}_{s}\left( t_{1}\right) dt\right) \cdot \cdots
\cdot CK\left( \vec{\omega}_{s}\left( t_{i-1}\right) dt\right) \right] ^{T}}}%
\times \\
&&\underset{\text{active, r.h.}}{\underbrace{\left[ CK\left( \vec{\omega}%
_{s}\left( t_{i}\right) dt\right) \right] ^{T}}}\times \left[ 
\begin{array}{ccc}
0 & 0 & 1%
\end{array}%
\right]
\end{eqnarray*}%
這裡用上不同時間微%
小轉動矩陣的commutive性質$%
\left( AB\right) C=A\left( BC\right) $,及微小%
轉動向量的可相加性%
。這樣我們就得到了$%
\left( t_{0},t_{1},\cdots ,t_{i+1}\right) $時刻z軸在%
lab frame位置的公式,同樣%
方法可求得x,y軸。可以%
看出上面所有passive的矩%
陣的作用只是再把坐%
標軸從body frame轉回到lab frame。

以上以$CK\left( \vec{\omega}_{s}(t_{i})dt\right) $來%
近似t$_{i}$到t$_{i+1}$的轉動事%
實上還不夠好,數值%
模擬結果會發現陀螺%
總能量退化的很快,%
陀螺進動高度不應該%
下降但卻下降了。這%
邊我提出以$CK\left( \vec{\omega}%
_{s}(t_{i+1})dt\right) $來近似t$_{i}$到t$_{i+1}$%
的轉動,因為模擬結%
果更好,以下我也嘗%
試提供物理解釋。這%
裡我們暫時假設$\vec{\omega}%
_{s}(t_{i})dt=\vec{\Omega}_{s}(t_{i})$,我們知%
道轉動向量在$t_{i+1}$跟$t_{i}$%
時刻在body frame中的向量值%
一般不會一樣,也就%
是$\vec{\Omega}_{i+1}(t_{i+1})\neq \vec{\Omega}_{i}(t_{i})$,%
這代表從$t_{i}$到$t_{i+1}$時,%
轉動向量在body座標上有%
變化,也因此我們不%
能夠單只考慮陀螺轉%
了$\vec{\Omega}_{s}(t_{i})$而已,此額%
外轉動向量的變化在$%
t_{i}$時s frame的向量值為$\Omega
_{i+1}(t_{i+1})-\Omega _{i}(t_{i})=\Omega _{i}(t_{i})+d\Omega
_{i}(dt)-\Omega _{i}(t_{i})=d\Omega _{s}(dt)$,也是一%
個轉動向量,所以space空%
間中總共的轉動可以%
考慮成兩步,第一步%
轉$\Omega _{s}(t_{i})$,第二步轉$d\Omega
_{s}(dt)$,寫成轉動矩陣%
\begin{equation}
CK(\Omega _{s}(t_{i}))\times CK(d\Omega _{s}(dt))=CK(\Omega
_{s}(t_{i})+d\Omega _{s}(dt))=CK(\Omega _{i+1}(t_{i+1}))
\end{equation}%
這代表我們只要考慮%
陀螺從t到t+dt的時候是轉%
了$\Omega _{s}(t+dt)$而不只是$\Omega _{s}(t)$%
,因此考慮$\Omega _{s}(t+dt)$我們%
就更準確的近似了這%
個轉動,以下的Python程%
式模擬會證明,考慮%
了$\Omega _{s}(t+dt)$給出的結果比$%
\Omega _{s}(t)$好非常多。若如%
此考慮則上式%
\begin{eqnarray}
\hat{z}_{lab}\left( t_{i+1}\right) &=&\left[ CK\left( \vec{\Omega}%
_{0}\right) \cdot CK\left( \vec{\omega}_{s}\left( t_{1}\right) dt\right)
\cdot CK\left( \vec{\omega}_{s}\left( t_{2}\right) dt\right) \cdot \cdots
\cdot CK\left( \vec{\omega}_{s}\left( t_{i+1}\right) dt\right) \right]
^{T}\times  \notag \\
&&\left[ 
\begin{array}{ccc}
0 & 0 & 1%
\end{array}%
\right]  \label{iterationCK}
\end{eqnarray}%
此公式極為剛體特徵%
軸轉動的公式。相同%
方法可求得另外兩軸x,y%
的轉動。

若將$\hat{x}_{lab}\left( t_{i+1}\right) $,$\hat{y}%
_{lab}\left( t_{i+1}\right) $,$\hat{z}_{lab}\left( t_{i+1}\right) $%
的三個行組成一矩陣C%
,我們會發現\ref{iterateeq}式%
即為方向餘弦遞推理%
論的遞推公式\cite{titterton}%
\begin{equation}
C(t_{i+1})=C(t_{i})\exp (\int \omega dt)  \label{iterationC}
\end{equation}%
不過很少有書上會把%
此公式解釋的清楚,%
比如說,\ref{iterationCK}式中的%
transpose若不經過我們每一%
步轉動矩陣都紀錄並%
寫下主被動與左右手%
性質,實際應用的時%
候我們將不知道要套%
用矩陣的主動還是被%
動特性以及轉動方向%
遵守是左手還是右手%
定則,這樣將造成極%
大的困擾,因此我們%
知道,只給出\ref{iterationC}公%
式只是一小步,是不%
夠的,離實作層面還%
有一段距離。

以下將上述方法寫成%
python程式,並且畫圖模%
擬其xyz軸運動。

\begin{figure}[th]
\caption{尖點運動}
\begin{center}
\fbox{\includegraphics[scale=0.6]{figure_xy.eps}}
\end{center}
\end{figure}

\begin{figure}[th]
\caption{有環運動}
\begin{center}
\fbox{\includegraphics[scale=0.6]{figure_x.eps}}
\end{center}
\end{figure}

\begin{figure}[th]
\caption{無環運動}
\begin{center}
\fbox{\includegraphics[scale=0.6]{figure_y.eps}}
\end{center}
\end{figure}

\begin{figure}[th]
\caption{等周速運動}
\label{figure_uniform}
\begin{center}
\fbox{\includegraphics[scale=0.6]{figure_uniform.eps}}
\end{center}
\end{figure}

這裡將模擬結果與文%
獻中\cite{hasbun}裡Hasbun教授寫的%
Matlab code做比較,圖\ref{compare}顯%
示了這裡的code與Hasbun的code%
的body z軸與unit sphere的交叉點%
。圖\ref{compare2} 顯示body z軸與unit
sphere的交叉點的距離差%
距。 
\begin{figure}[th]
\caption{{}模擬條件(SI units): I=0.002; Is=0.0008;
g=9.8; M=1 ; arm=.04; spin freq= 20 Hz; Initial angle from vertical 54.57
degree; Simulation time 3.2 seconds with 4000 steps. This figure shows both
the last 1600 data points from the two simulation codes. The first 2400
points are omitted to avoid overlap.}
\label{compare}
\begin{center}
\fbox{\includegraphics[scale=0.8]{locus4000pts3d.png}}
\end{center}
\end{figure}
\begin{figure}[th]
\caption{{}}
\label{compare2}
\begin{center}
\fbox{\includegraphics[scale=0.8]{locus4000pts.png}}
\end{center}
\end{figure}

陀螺等周速運動(Figure \ref%
{figure_uniform})的初始值條件如%
何計算呢?等周速的條%
件在Goldstein第二版5-77式給出%
\begin{equation}
Mgl=\dot{\phi}\left( I_{3}\omega _{3}-I_{1}\dot{\phi}\cos \theta _{0}\right)
\end{equation}%
,不過此式是由尤拉%
角(euler angles)給出,但我們%
需要的是anguler velocity along body的初%
始值,因此我們必須%
轉換尤拉角到anguler velocity along body%
,方法如下。上式中$%
\omega _{3}$即為我們的$\left( \omega
_{z}\right) _{b}$,這裡是20 Hz,$\theta _{0}$%
即為我們之前的orien向量%
所定,此模擬中是取45%
度角,由上式可求出%
兩組$\dot{\phi}(t_{0})$。另外尤拉%
角跟anugler velocity along body的關係式%
在Goldstein 4-125式給出%
\begin{eqnarray}
(\omega _{x})_{b} &=&\dot{\phi}\sin \theta \sin \psi +\dot{\theta}\cos \psi
\\
(\omega _{y})_{b} &=&\dot{\phi}\sin \theta \cos \psi -\dot{\theta}\sin \psi
\\
(\omega _{z})_{b} &=&\dot{\phi}\cos \theta +\dot{\psi}
\end{eqnarray}%
知道$\dot{\phi}(t_{0})$、$\theta _{0}$、$(\omega
_{z})_{b}$,我們由第三條求%
出$\dot{\psi}(t_{0})$,再把$\dot{\psi}(t_{0})$%
帶到第一二條後就可%
得到$(\omega _{x}(t_{0}),\omega _{y}(t_{0}))_{b}$,這%
樣我們就得到anguler velocity along body%
的初始值。因為$\dot{\phi}(t_{0})$%
有兩組,因此解出的%
貼體角速度也會有兩%
組,兩組的物理意義%
分別如下,一種情況%
是fast top,這個狀況相當%
於重力的影響遠小於%
總角動量$L$,因此這個%
特別的例子基本上相%
當於忽略重力,而陀%
螺基本上會像一個free top%
一樣進行precession。另一種%
狀況是slow top,也就是上%
面模擬結果中第四種%
的狀況,這裡提供的%
python程式所有情況都可%
以模擬。另外一個特%
殊的情況是在fast top的情%
形下,如果初始值$\theta
_{0}=0$,也就是陀螺z軸的%
起始狀態是垂直於水%
平面的,這樣的話陀%
螺幾乎會像靜止不動%
一樣,我們也叫這情%
況做sleeping top。

\begin{remark}
要陀螺具有Precession and Nutation的%
動作,L/$\Delta L$必須要大,%
如果L小於$\Delta L$,則只會%
有陀螺質量受重力影%
響往下倒下的運動(不%
過這對檢查程式有沒%
有錯誤很有幫助!),理%
想上L至少要大於$\Delta L$,%
最好L大大於$\Delta L$。化成%
數值上的比較:這代%
表%
\begin{equation}
L\gg \Delta L\Rightarrow I\cdot 2\pi f\gg \vec{\Gamma}\Delta t\Rightarrow
I\cdot 2\pi f\gg \vec{r}\times \vec{F}\cdot 1/f\Rightarrow f\gg \sqrt{\frac{%
arm\cdot Mg\cdot \sin (\theta )}{2\pi I\cdot G}}
\end{equation}%
where $\theta $ is gyro's tilt angle and G is moment of inertial geometry
factor. 考慮$\Delta t$的量級大約%
是陀螺轉幾圈的時間%
(characteristic time),量級上約是$\sim
1/f$,若假設arm是10 cm, M = 1kg, g=10 m/s$^{2}$%
, I = 0.5M(0.05)$^{2}$,則f最少要10 Hertz以%
上。因此我們將以這%
些參數比較f = 1, 10, 50 Hertz所給%
出的陀螺運動。
\end{remark}

\href{https://drive.google.com/file/d/0B96HmLH-SQVmM1dvYlFiQm9ESGM/edit?usp=sharing%
}{\underline{\color{blue}\smash{Python code can be found here.}}}

\href{http://tinypic.com/r/10cw9yf/8}{\underline{\color{blue}%
\smash{3D
animation.}}}

This document is prepared with Scientific Workplace 5.0 and typeset with Tex
Live 2013 (Xelatex). \href{http://whymranderson.blogspot.tw/2014/03/how-to-convert-swp-50-special-unicode.html%
}{\underline{\color{blue}\smash{Here is how.}}}

If you feel like supporting this work, you can \href{https://www.paypal.com/us/cgi-bin/webscr?cmd=_flow&SESSION=WlwN4JPJMnohiYq8N9IuRiEIHqDEyNxauM_sF1u1Qh3M5sQBsssTjYFi4yu&dispatch=5885d80a13c0db1f8e263663d3faee8d5402c249c5a2cfd4a145d37ec05e9a5e%
}{\underline{\color{blue}\smash{donate here}}}.

\begin{thebibliography}{9}
\bibitem{goldstein} Herbert Goldstein, \emph{Classical Mechanics}. Addison
Wesley, Massachusetts, 2nd Edition, 1980

\bibitem{tong} David Tong, \emph{Classical Dynamics University of Cambridge
Part II Mathematical Tripos.} Cambridge UK, 2004-2005, (Course note,
available on the web)

\bibitem{matlab} \href{http://www.mathworks.com/help/matlab/ordinary-differential-equations.html%
}{\underline{\color{blue}%
\smash{Matlab online documentation - Ordinary
differential equations.}}}, Matlab R2014a

\bibitem{徐鐘} 徐小明 钟万勰,%
\textit{刚体动力学的四元%
数表示及保辛积分},%
《应用数学和力学》 2014%
, 35(1): 111

\bibitem{hasbun} Javier E. Hasbun, \emph{Classical Mechanics with Matlab
Appications.} Jones and Bartlett Publishers, London UK, 2009

\bibitem{titterton} D.H. Titterton and J.L. Weston, \textit{Strapdown
inertial navigation technology}, Peter Peregrinus Ltd., London UK, 1997

\bibitem{pixarnote} David Baraff\textit{, }\href{http://graphics.cs.cmu.edu/courses/15-869-F08/lec/14/notesg.pdf%
}{\underline{\color{blue}%
\smash{Physically Based Modeling - Rigid Body
Simulation}}}, Pixar Animation Studios notes
\end{thebibliography}

\end{document}
