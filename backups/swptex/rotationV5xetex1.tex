
\documentclass[12pt]{article}
%%%%%%%%%%%%%%%%%%%%%%%%%%%%%%%%%%%%%%%%%%%%%%%%%%%%%%%%%%%%%%%%%%%%%%%%%%%%%%%%%%%%%%%%%%%%%%%%%%%%%%%%%%%%%%%%%%%%%%%%%%%%%%%%%%%%%%%%%%%%%%%%%%%%%%%%%%%%%%%%%%%%%%%%%%%%%%%%%%%%%%%%%%%%%%%%%%%%%%%%%%%%%%%%%%%%%%%%%%%%%%%%%%%%%%%%%%%%%%%%%%%%%%%%%%%%
\usepackage{amsmath}
\usepackage{amsfonts}
\usepackage{sw2unicode}
\usepackage[UT1]{fontenc}
\usepackage{pmingliu}
\usepackage[left=0.95in,right=0.95in,top=2cm,bottom=2.54cm]{geometry}

\setcounter{MaxMatrixCols}{10}
%TCIDATA{OutputFilter=LATEX.DLL}
%TCIDATA{Version=5.00.0.2606}
%TCIDATA{<META NAME="SaveForMode" CONTENT="1">}
%TCIDATA{BibliographyScheme=Manual}
%TCIDATA{Created=Monday, January 13, 2014 11:43:31}
%TCIDATA{LastRevised=Monday, April 14, 2014 17:18:48}
%TCIDATA{<META NAME="GraphicsSave" CONTENT="32">}
%TCIDATA{<META NAME="DocumentShell" CONTENT="International\Traditional Chinese Article">}
%TCIDATA{CSTFile=Traditional Chinese.cst}

\newtheorem{theorem}{Theorem}
\newtheorem{acknowledgement}[theorem]{Acknowledgement}
\newtheorem{algorithm}[theorem]{Algorithm}
\newtheorem{axiom}[theorem]{Axiom}
\newtheorem{case}[theorem]{Case}
\newtheorem{claim}[theorem]{Claim}
\newtheorem{conclusion}[theorem]{Conclusion}
\newtheorem{condition}[theorem]{Condition}
\newtheorem{conjecture}[theorem]{Conjecture}
\newtheorem{corollary}[theorem]{Corollary}
\newtheorem{criterion}[theorem]{Criterion}
\newtheorem{definition}[theorem]{Definition}
\newtheorem{example}[theorem]{Example}
\newtheorem{exercise}[theorem]{Exercise}
\newtheorem{lemma}[theorem]{Lemma}
\newtheorem{notation}[theorem]{Notation}
\newtheorem{problem}[theorem]{Problem}
\newtheorem{proposition}[theorem]{Proposition}
\newtheorem{remark}[theorem]{Remark}
\newtheorem{solution}[theorem]{Solution}
\newtheorem{summary}[theorem]{Summary}
\newenvironment{proof}[1][Proof]{\noindent\textbf{#1.} }{\ \rule{0.5em}{0.5em}}
% Macros for Scientific Word 4.0 documents saved with the LaTeX filter.
% Copyright (C) 2002 Mackichan Software, Inc.

\typeout{TCILATEX Macros for Scientific Word 5.0 <13 Feb 2003>.}
\typeout{NOTICE:  This macro file is NOT proprietary and may be 
freely copied and distributed.}
%
\makeatletter

%%%%%%%%%%%%%%%%%%%%%
% pdfTeX related.
\ifx\pdfoutput\relax\let\pdfoutput=\undefined\fi
\newcount\msipdfoutput
\ifx\pdfoutput\undefined
\else
 \ifcase\pdfoutput
 \else 
    \msipdfoutput=1
    \ifx\paperwidth\undefined
    \else
      \ifdim\paperheight=0pt\relax
      \else
        \pdfpageheight\paperheight
      \fi
      \ifdim\paperwidth=0pt\relax
      \else
        \pdfpagewidth\paperwidth
      \fi
    \fi
  \fi  
\fi

%%%%%%%%%%%%%%%%%%%%%
% FMTeXButton
% This is used for putting TeXButtons in the 
% frontmatter of a document. Add a line like
% \QTagDef{FMTeXButton}{101}{} to the filter 
% section of the cst being used. Also add a
% new section containing:
%     [f_101]
%     ALIAS=FMTexButton
%     TAG_TYPE=FIELD
%     TAG_LEADIN=TeX Button:
%
% It also works to put \defs in the preamble after 
% the \input tcilatex
\def\FMTeXButton#1{#1}
%
%%%%%%%%%%%%%%%%%%%%%%
% macros for time
\newcount\@hour\newcount\@minute\chardef\@x10\chardef\@xv60
\def\tcitime{
\def\@time{%
  \@minute\time\@hour\@minute\divide\@hour\@xv
  \ifnum\@hour<\@x 0\fi\the\@hour:%
  \multiply\@hour\@xv\advance\@minute-\@hour
  \ifnum\@minute<\@x 0\fi\the\@minute
  }}%

%%%%%%%%%%%%%%%%%%%%%%
% macro for hyperref and msihyperref
%\@ifundefined{hyperref}{\def\hyperref#1#2#3#4{#2\ref{#4}#3}}{}

\def\x@hyperref#1#2#3{%
   % Turn off various catcodes before reading parameter 4
   \catcode`\~ = 12
   \catcode`\$ = 12
   \catcode`\_ = 12
   \catcode`\# = 12
   \catcode`\& = 12
   \y@hyperref{#1}{#2}{#3}%
}

\def\y@hyperref#1#2#3#4{%
   #2\ref{#4}#3
   \catcode`\~ = 13
   \catcode`\$ = 3
   \catcode`\_ = 8
   \catcode`\# = 6
   \catcode`\& = 4
}

\@ifundefined{hyperref}{\let\hyperref\x@hyperref}{}
\@ifundefined{msihyperref}{\let\msihyperref\x@hyperref}{}




% macro for external program call
\@ifundefined{qExtProgCall}{\def\qExtProgCall#1#2#3#4#5#6{\relax}}{}
%%%%%%%%%%%%%%%%%%%%%%
%
% macros for graphics
%
\def\FILENAME#1{#1}%
%
\def\QCTOpt[#1]#2{%
  \def\QCTOptB{#1}
  \def\QCTOptA{#2}
}
\def\QCTNOpt#1{%
  \def\QCTOptA{#1}
  \let\QCTOptB\empty
}
\def\Qct{%
  \@ifnextchar[{%
    \QCTOpt}{\QCTNOpt}
}
\def\QCBOpt[#1]#2{%
  \def\QCBOptB{#1}%
  \def\QCBOptA{#2}%
}
\def\QCBNOpt#1{%
  \def\QCBOptA{#1}%
  \let\QCBOptB\empty
}
\def\Qcb{%
  \@ifnextchar[{%
    \QCBOpt}{\QCBNOpt}%
}
\def\PrepCapArgs{%
  \ifx\QCBOptA\empty
    \ifx\QCTOptA\empty
      {}%
    \else
      \ifx\QCTOptB\empty
        {\QCTOptA}%
      \else
        [\QCTOptB]{\QCTOptA}%
      \fi
    \fi
  \else
    \ifx\QCBOptA\empty
      {}%
    \else
      \ifx\QCBOptB\empty
        {\QCBOptA}%
      \else
        [\QCBOptB]{\QCBOptA}%
      \fi
    \fi
  \fi
}
\newcount\GRAPHICSTYPE
%\GRAPHICSTYPE 0 is for TurboTeX
%\GRAPHICSTYPE 1 is for DVIWindo (PostScript)
%%%(removed)%\GRAPHICSTYPE 2 is for psfig (PostScript)
\GRAPHICSTYPE=\z@
\def\GRAPHICSPS#1{%
 \ifcase\GRAPHICSTYPE%\GRAPHICSTYPE=0
   \special{ps: #1}%
 \or%\GRAPHICSTYPE=1
   \special{language "PS", include "#1"}%
%%%\or%\GRAPHICSTYPE=2
%%%  #1%
 \fi
}%
%
\def\GRAPHICSHP#1{\special{include #1}}%
%
% \graffile{ body }                                  %#1
%          { contentswidth (scalar)  }               %#2
%          { contentsheight (scalar) }               %#3
%          { vertical shift when in-line (scalar) }  %#4

\def\graffile#1#2#3#4{%
%%% \ifnum\GRAPHICSTYPE=\tw@
%%%  %Following if using psfig
%%%  \@ifundefined{psfig}{\input psfig.tex}{}%
%%%  \psfig{file=#1, height=#3, width=#2}%
%%% \else
  %Following for all others
  % JCS - added BOXTHEFRAME, see below
    \bgroup
	   \@inlabelfalse
       \leavevmode
       \@ifundefined{bbl@deactivate}{\def~{\string~}}{\activesoff}%
        \raise -#4 \BOXTHEFRAME{%
           \hbox to #2{\raise #3\hbox to #2{\null #1\hfil}}}%
    \egroup
}%
%
% A box for drafts
\def\draftbox#1#2#3#4{%
 \leavevmode\raise -#4 \hbox{%
  \frame{\rlap{\protect\tiny #1}\hbox to #2%
   {\vrule height#3 width\z@ depth\z@\hfil}%
  }%
 }%
}%
%
\newcount\@msidraft
\@msidraft=\z@
\let\nographics=\@msidraft
\newif\ifwasdraft
\wasdraftfalse

%  \GRAPHIC{ body }                                  %#1
%          { draft name }                            %#2
%          { contentswidth (scalar)  }               %#3
%          { contentsheight (scalar) }               %#4
%          { vertical shift when in-line (scalar) }  %#5
\def\GRAPHIC#1#2#3#4#5{%
   \ifnum\@msidraft=\@ne\draftbox{#2}{#3}{#4}{#5}%
   \else\graffile{#1}{#3}{#4}{#5}%
   \fi
}
%
\def\addtoLaTeXparams#1{%
    \edef\LaTeXparams{\LaTeXparams #1}}%
%
% JCS -  added a switch BoxFrame that can 
% be set by including X in the frame params.
% If set a box is drawn around the frame.

\newif\ifBoxFrame \BoxFramefalse
\newif\ifOverFrame \OverFramefalse
\newif\ifUnderFrame \UnderFramefalse

\def\BOXTHEFRAME#1{%
   \hbox{%
      \ifBoxFrame
         \frame{#1}%
      \else
         {#1}%
      \fi
   }%
}


\def\doFRAMEparams#1{\BoxFramefalse\OverFramefalse\UnderFramefalse\readFRAMEparams#1\end}%
\def\readFRAMEparams#1{%
 \ifx#1\end%
  \let\next=\relax
  \else
  \ifx#1i\dispkind=\z@\fi
  \ifx#1d\dispkind=\@ne\fi
  \ifx#1f\dispkind=\tw@\fi
  \ifx#1t\addtoLaTeXparams{t}\fi
  \ifx#1b\addtoLaTeXparams{b}\fi
  \ifx#1p\addtoLaTeXparams{p}\fi
  \ifx#1h\addtoLaTeXparams{h}\fi
  \ifx#1X\BoxFrametrue\fi
  \ifx#1O\OverFrametrue\fi
  \ifx#1U\UnderFrametrue\fi
  \ifx#1w
    \ifnum\@msidraft=1\wasdrafttrue\else\wasdraftfalse\fi
    \@msidraft=\@ne
  \fi
  \let\next=\readFRAMEparams
  \fi
 \next
 }%
%
%Macro for In-line graphics object
%   \IFRAME{ contentswidth (scalar)  }               %#1
%          { contentsheight (scalar) }               %#2
%          { vertical shift when in-line (scalar) }  %#3
%          { draft name }                            %#4
%          { body }                                  %#5
%          { caption}                                %#6


\def\IFRAME#1#2#3#4#5#6{%
      \bgroup
      \let\QCTOptA\empty
      \let\QCTOptB\empty
      \let\QCBOptA\empty
      \let\QCBOptB\empty
      #6%
      \parindent=0pt
      \leftskip=0pt
      \rightskip=0pt
      \setbox0=\hbox{\QCBOptA}%
      \@tempdima=#1\relax
      \ifOverFrame
          % Do this later
          \typeout{This is not implemented yet}%
          \show\HELP
      \else
         \ifdim\wd0>\@tempdima
            \advance\@tempdima by \@tempdima
            \ifdim\wd0 >\@tempdima
               \setbox1 =\vbox{%
                  \unskip\hbox to \@tempdima{\hfill\GRAPHIC{#5}{#4}{#1}{#2}{#3}\hfill}%
                  \unskip\hbox to \@tempdima{\parbox[b]{\@tempdima}{\QCBOptA}}%
               }%
               \wd1=\@tempdima
            \else
               \textwidth=\wd0
               \setbox1 =\vbox{%
                 \noindent\hbox to \wd0{\hfill\GRAPHIC{#5}{#4}{#1}{#2}{#3}\hfill}\\%
                 \noindent\hbox{\QCBOptA}%
               }%
               \wd1=\wd0
            \fi
         \else
            \ifdim\wd0>0pt
              \hsize=\@tempdima
              \setbox1=\vbox{%
                \unskip\GRAPHIC{#5}{#4}{#1}{#2}{0pt}%
                \break
                \unskip\hbox to \@tempdima{\hfill \QCBOptA\hfill}%
              }%
              \wd1=\@tempdima
           \else
              \hsize=\@tempdima
              \setbox1=\vbox{%
                \unskip\GRAPHIC{#5}{#4}{#1}{#2}{0pt}%
              }%
              \wd1=\@tempdima
           \fi
         \fi
         \@tempdimb=\ht1
         %\advance\@tempdimb by \dp1
         \advance\@tempdimb by -#2
         \advance\@tempdimb by #3
         \leavevmode
         \raise -\@tempdimb \hbox{\box1}%
      \fi
      \egroup%
}%
%
%Macro for Display graphics object
%   \DFRAME{ contentswidth (scalar)  }               %#1
%          { contentsheight (scalar) }               %#2
%          { draft label }                           %#3
%          { name }                                  %#4
%          { caption}                                %#5
\def\DFRAME#1#2#3#4#5{%
  \vspace\topsep
  \hfil\break
  \bgroup
     \leftskip\@flushglue
	 \rightskip\@flushglue
	 \parindent\z@
	 \parfillskip\z@skip
     \let\QCTOptA\empty
     \let\QCTOptB\empty
     \let\QCBOptA\empty
     \let\QCBOptB\empty
	 \vbox\bgroup
        \ifOverFrame 
           #5\QCTOptA\par
        \fi
        \GRAPHIC{#4}{#3}{#1}{#2}{\z@}%
        \ifUnderFrame 
           \break#5\QCBOptA
        \fi
	 \egroup
  \egroup
  \vspace\topsep
  \break
}%
%
%Macro for Floating graphic object
%   \FFRAME{ framedata f|i tbph x F|T }              %#1
%          { contentswidth (scalar)  }               %#2
%          { contentsheight (scalar) }               %#3
%          { caption }                               %#4
%          { label }                                 %#5
%          { draft name }                            %#6
%          { body }                                  %#7
\def\FFRAME#1#2#3#4#5#6#7{%
 %If float.sty loaded and float option is 'h', change to 'H'  (gp) 1998/09/05
  \@ifundefined{floatstyle}
    {%floatstyle undefined (and float.sty not present), no change
     \begin{figure}[#1]%
    }
    {%floatstyle DEFINED
	 \ifx#1h%Only the h parameter, change to H
      \begin{figure}[H]%
	 \else
      \begin{figure}[#1]%
	 \fi
	}
  \let\QCTOptA\empty
  \let\QCTOptB\empty
  \let\QCBOptA\empty
  \let\QCBOptB\empty
  \ifOverFrame
    #4
    \ifx\QCTOptA\empty
    \else
      \ifx\QCTOptB\empty
        \caption{\QCTOptA}%
      \else
        \caption[\QCTOptB]{\QCTOptA}%
      \fi
    \fi
    \ifUnderFrame\else
      \label{#5}%
    \fi
  \else
    \UnderFrametrue%
  \fi
  \begin{center}\GRAPHIC{#7}{#6}{#2}{#3}{\z@}\end{center}%
  \ifUnderFrame
    #4
    \ifx\QCBOptA\empty
      \caption{}%
    \else
      \ifx\QCBOptB\empty
        \caption{\QCBOptA}%
      \else
        \caption[\QCBOptB]{\QCBOptA}%
      \fi
    \fi
    \label{#5}%
  \fi
  \end{figure}%
 }%
%
%
%    \FRAME{ framedata f|i tbph x F|T }              %#1
%          { contentswidth (scalar)  }               %#2
%          { contentsheight (scalar) }               %#3
%          { vertical shift when in-line (scalar) }  %#4
%          { caption }                               %#5
%          { label }                                 %#6
%          { name }                                  %#7
%          { body }                                  %#8
%
%    framedata is a string which can contain the following
%    characters: idftbphxFT
%    Their meaning is as follows:
%             i, d or f : in-line, display, or floating
%             t,b,p,h   : LaTeX floating placement options
%             x         : fit contents box to contents
%             F or T    : Figure or Table. 
%                         Later this can expand
%                         to a more general float class.
%
%
\newcount\dispkind%

\def\makeactives{
  \catcode`\"=\active
  \catcode`\;=\active
  \catcode`\:=\active
  \catcode`\'=\active
  \catcode`\~=\active
}
\bgroup
   \makeactives
   \gdef\activesoff{%
      \def"{\string"}%
      \def;{\string;}%
      \def:{\string:}%
      \def'{\string'}%
      \def~{\string~}%
      %\bbl@deactivate{"}%
      %\bbl@deactivate{;}%
      %\bbl@deactivate{:}%
      %\bbl@deactivate{'}%
    }
\egroup

\def\FRAME#1#2#3#4#5#6#7#8{%
 \bgroup
 \ifnum\@msidraft=\@ne
   \wasdrafttrue
 \else
   \wasdraftfalse%
 \fi
 \def\LaTeXparams{}%
 \dispkind=\z@
 \def\LaTeXparams{}%
 \doFRAMEparams{#1}%
 \ifnum\dispkind=\z@\IFRAME{#2}{#3}{#4}{#7}{#8}{#5}\else
  \ifnum\dispkind=\@ne\DFRAME{#2}{#3}{#7}{#8}{#5}\else
   \ifnum\dispkind=\tw@
    \edef\@tempa{\noexpand\FFRAME{\LaTeXparams}}%
    \@tempa{#2}{#3}{#5}{#6}{#7}{#8}%
    \fi
   \fi
  \fi
  \ifwasdraft\@msidraft=1\else\@msidraft=0\fi{}%
  \egroup
 }%
%
% This macro added to let SW gobble a parameter that
% should not be passed on and expanded. 

\def\TEXUX#1{"texux"}

%
% Macros for text attributes:
%
\def\BF#1{{\bf {#1}}}%
\def\NEG#1{\leavevmode\hbox{\rlap{\thinspace/}{$#1$}}}%
%
%%%%%%%%%%%%%%%%%%%%%%%%%%%%%%%%%%%%%%%%%%%%%%%%%%%%%%%%%%%%%%%%%%%%%%%%
%
%
% macros for user - defined functions
\def\limfunc#1{\mathop{\rm #1}}%
\def\func#1{\mathop{\rm #1}\nolimits}%
% macro for unit names
\def\unit#1{\mathord{\thinspace\rm #1}}%

%
% miscellaneous 
\long\def\QQQ#1#2{%
     \long\expandafter\def\csname#1\endcsname{#2}}%
\@ifundefined{QTP}{\def\QTP#1{}}{}
\@ifundefined{QEXCLUDE}{\def\QEXCLUDE#1{}}{}
\@ifundefined{Qlb}{\def\Qlb#1{#1}}{}
\@ifundefined{Qlt}{\def\Qlt#1{#1}}{}
\def\QWE{}%
\long\def\QQA#1#2{}%
\def\QTR#1#2{{\csname#1\endcsname {#2}}}%
\long\def\TeXButton#1#2{#2}%
\long\def\QSubDoc#1#2{#2}%
\def\EXPAND#1[#2]#3{}%
\def\NOEXPAND#1[#2]#3{}%
\def\PROTECTED{}%
\def\LaTeXparent#1{}%
\def\ChildStyles#1{}%
\def\ChildDefaults#1{}%
\def\QTagDef#1#2#3{}%

% Constructs added with Scientific Notebook
\@ifundefined{correctchoice}{\def\correctchoice{\relax}}{}
\@ifundefined{HTML}{\def\HTML#1{\relax}}{}
\@ifundefined{TCIIcon}{\def\TCIIcon#1#2#3#4{\relax}}{}
\if@compatibility
  \typeout{Not defining UNICODE  U or CustomNote commands for LaTeX 2.09.}
\else
  \providecommand{\UNICODE}[2][]{\protect\rule{.1in}{.1in}}
  \providecommand{\U}[1]{\protect\rule{.1in}{.1in}}
  \providecommand{\CustomNote}[3][]{\marginpar{#3}}
\fi

\@ifundefined{lambdabar}{
      \def\lambdabar{\errmessage{You have used the lambdabar symbol. 
                      This is available for typesetting only in RevTeX styles.}}
   }{}

%
% Macros for style editor docs
\@ifundefined{StyleEditBeginDoc}{\def\StyleEditBeginDoc{\relax}}{}
%
% Macros for footnotes
\def\QQfnmark#1{\footnotemark}
\def\QQfntext#1#2{\addtocounter{footnote}{#1}\footnotetext{#2}}
%
% Macros for indexing.
%
\@ifundefined{TCIMAKEINDEX}{}{\makeindex}%
%
% Attempts to avoid problems with other styles
\@ifundefined{abstract}{%
 \def\abstract{%
  \if@twocolumn
   \section*{Abstract (Not appropriate in this style!)}%
   \else \small 
   \begin{center}{\bf Abstract\vspace{-.5em}\vspace{\z@}}\end{center}%
   \quotation 
   \fi
  }%
 }{%
 }%
\@ifundefined{endabstract}{\def\endabstract
  {\if@twocolumn\else\endquotation\fi}}{}%
\@ifundefined{maketitle}{\def\maketitle#1{}}{}%
\@ifundefined{affiliation}{\def\affiliation#1{}}{}%
\@ifundefined{proof}{\def\proof{\noindent{\bfseries Proof. }}}{}%
\@ifundefined{endproof}{\def\endproof{\mbox{\ \rule{.1in}{.1in}}}}{}%
\@ifundefined{newfield}{\def\newfield#1#2{}}{}%
\@ifundefined{chapter}{\def\chapter#1{\par(Chapter head:)#1\par }%
 \newcount\c@chapter}{}%
\@ifundefined{part}{\def\part#1{\par(Part head:)#1\par }}{}%
\@ifundefined{section}{\def\section#1{\par(Section head:)#1\par }}{}%
\@ifundefined{subsection}{\def\subsection#1%
 {\par(Subsection head:)#1\par }}{}%
\@ifundefined{subsubsection}{\def\subsubsection#1%
 {\par(Subsubsection head:)#1\par }}{}%
\@ifundefined{paragraph}{\def\paragraph#1%
 {\par(Subsubsubsection head:)#1\par }}{}%
\@ifundefined{subparagraph}{\def\subparagraph#1%
 {\par(Subsubsubsubsection head:)#1\par }}{}%
%%%%%%%%%%%%%%%%%%%%%%%%%%%%%%%%%%%%%%%%%%%%%%%%%%%%%%%%%%%%%%%%%%%%%%%%
% These symbols are not recognized by LaTeX
\@ifundefined{therefore}{\def\therefore{}}{}%
\@ifundefined{backepsilon}{\def\backepsilon{}}{}%
\@ifundefined{yen}{\def\yen{\hbox{\rm\rlap=Y}}}{}%
\@ifundefined{registered}{%
   \def\registered{\relax\ifmmode{}\r@gistered
                    \else$\m@th\r@gistered$\fi}%
 \def\r@gistered{^{\ooalign
  {\hfil\raise.07ex\hbox{$\scriptstyle\rm\text{R}$}\hfil\crcr
  \mathhexbox20D}}}}{}%
\@ifundefined{Eth}{\def\Eth{}}{}%
\@ifundefined{eth}{\def\eth{}}{}%
\@ifundefined{Thorn}{\def\Thorn{}}{}%
\@ifundefined{thorn}{\def\thorn{}}{}%
% A macro to allow any symbol that requires math to appear in text
\def\TEXTsymbol#1{\mbox{$#1$}}%
\@ifundefined{degree}{\def\degree{{}^{\circ}}}{}%
%
% macros for T3TeX files
\newdimen\theight
\@ifundefined{Column}{\def\Column{%
 \vadjust{\setbox\z@=\hbox{\scriptsize\quad\quad tcol}%
  \theight=\ht\z@\advance\theight by \dp\z@\advance\theight by \lineskip
  \kern -\theight \vbox to \theight{%
   \rightline{\rlap{\box\z@}}%
   \vss
   }%
  }%
 }}{}%
%
\@ifundefined{qed}{\def\qed{%
 \ifhmode\unskip\nobreak\fi\ifmmode\ifinner\else\hskip5\p@\fi\fi
 \hbox{\hskip5\p@\vrule width4\p@ height6\p@ depth1.5\p@\hskip\p@}%
 }}{}%
%
\@ifundefined{cents}{\def\cents{\hbox{\rm\rlap c/}}}{}%
\@ifundefined{tciLaplace}{\def\tciLaplace{\ensuremath{\mathcal{L}}}}{}%
\@ifundefined{tciFourier}{\def\tciFourier{\ensuremath{\mathcal{F}}}}{}%
\@ifundefined{textcurrency}{\def\textcurrency{\hbox{\rm\rlap xo}}}{}%
\@ifundefined{texteuro}{\def\texteuro{\hbox{\rm\rlap C=}}}{}%
\@ifundefined{euro}{\def\euro{\hbox{\rm\rlap C=}}}{}%
\@ifundefined{textfranc}{\def\textfranc{\hbox{\rm\rlap-F}}}{}%
\@ifundefined{textlira}{\def\textlira{\hbox{\rm\rlap L=}}}{}%
\@ifundefined{textpeseta}{\def\textpeseta{\hbox{\rm P\negthinspace s}}}{}%
%
\@ifundefined{miss}{\def\miss{\hbox{\vrule height2\p@ width 2\p@ depth\z@}}}{}%
%
\@ifundefined{vvert}{\def\vvert{\Vert}}{}%  %always translated to \left| or \right|
%
\@ifundefined{tcol}{\def\tcol#1{{\baselineskip=6\p@ \vcenter{#1}} \Column}}{}%
%
\@ifundefined{dB}{\def\dB{\hbox{{}}}}{}%        %dummy entry in column 
\@ifundefined{mB}{\def\mB#1{\hbox{$#1$}}}{}%   %column entry
\@ifundefined{nB}{\def\nB#1{\hbox{#1}}}{}%     %column entry (not math)
%
\@ifundefined{note}{\def\note{$^{\dag}}}{}%
%
\def\newfmtname{LaTeX2e}
% No longer load latexsym.  This is now handled by SWP, which uses amsfonts if necessary
%
\ifx\fmtname\newfmtname
  \DeclareOldFontCommand{\rm}{\normalfont\rmfamily}{\mathrm}
  \DeclareOldFontCommand{\sf}{\normalfont\sffamily}{\mathsf}
  \DeclareOldFontCommand{\tt}{\normalfont\ttfamily}{\mathtt}
  \DeclareOldFontCommand{\bf}{\normalfont\bfseries}{\mathbf}
  \DeclareOldFontCommand{\it}{\normalfont\itshape}{\mathit}
  \DeclareOldFontCommand{\sl}{\normalfont\slshape}{\@nomath\sl}
  \DeclareOldFontCommand{\sc}{\normalfont\scshape}{\@nomath\sc}
\fi

%
% Greek bold macros
% Redefine all of the math symbols 
% which might be bolded	 - there are 
% probably others to add to this list

\def\alpha{{\Greekmath 010B}}%
\def\beta{{\Greekmath 010C}}%
\def\gamma{{\Greekmath 010D}}%
\def\delta{{\Greekmath 010E}}%
\def\epsilon{{\Greekmath 010F}}%
\def\zeta{{\Greekmath 0110}}%
\def\eta{{\Greekmath 0111}}%
\def\theta{{\Greekmath 0112}}%
\def\iota{{\Greekmath 0113}}%
\def\kappa{{\Greekmath 0114}}%
\def\lambda{{\Greekmath 0115}}%
\def\mu{{\Greekmath 0116}}%
\def\nu{{\Greekmath 0117}}%
\def\xi{{\Greekmath 0118}}%
\def\pi{{\Greekmath 0119}}%
\def\rho{{\Greekmath 011A}}%
\def\sigma{{\Greekmath 011B}}%
\def\tau{{\Greekmath 011C}}%
\def\upsilon{{\Greekmath 011D}}%
\def\phi{{\Greekmath 011E}}%
\def\chi{{\Greekmath 011F}}%
\def\psi{{\Greekmath 0120}}%
\def\omega{{\Greekmath 0121}}%
\def\varepsilon{{\Greekmath 0122}}%
\def\vartheta{{\Greekmath 0123}}%
\def\varpi{{\Greekmath 0124}}%
\def\varrho{{\Greekmath 0125}}%
\def\varsigma{{\Greekmath 0126}}%
\def\varphi{{\Greekmath 0127}}%

\def\nabla{{\Greekmath 0272}}
\def\FindBoldGroup{%
   {\setbox0=\hbox{$\mathbf{x\global\edef\theboldgroup{\the\mathgroup}}$}}%
}

\def\Greekmath#1#2#3#4{%
    \if@compatibility
        \ifnum\mathgroup=\symbold
           \mathchoice{\mbox{\boldmath$\displaystyle\mathchar"#1#2#3#4$}}%
                      {\mbox{\boldmath$\textstyle\mathchar"#1#2#3#4$}}%
                      {\mbox{\boldmath$\scriptstyle\mathchar"#1#2#3#4$}}%
                      {\mbox{\boldmath$\scriptscriptstyle\mathchar"#1#2#3#4$}}%
        \else
           \mathchar"#1#2#3#4% 
        \fi 
    \else 
        \FindBoldGroup
        \ifnum\mathgroup=\theboldgroup % For 2e
           \mathchoice{\mbox{\boldmath$\displaystyle\mathchar"#1#2#3#4$}}%
                      {\mbox{\boldmath$\textstyle\mathchar"#1#2#3#4$}}%
                      {\mbox{\boldmath$\scriptstyle\mathchar"#1#2#3#4$}}%
                      {\mbox{\boldmath$\scriptscriptstyle\mathchar"#1#2#3#4$}}%
        \else
           \mathchar"#1#2#3#4% 
        \fi     	    
	  \fi}

\newif\ifGreekBold  \GreekBoldfalse
\let\SAVEPBF=\pbf
\def\pbf{\GreekBoldtrue\SAVEPBF}%
%

\@ifundefined{theorem}{\newtheorem{theorem}{Theorem}}{}
\@ifundefined{lemma}{\newtheorem{lemma}[theorem]{Lemma}}{}
\@ifundefined{corollary}{\newtheorem{corollary}[theorem]{Corollary}}{}
\@ifundefined{conjecture}{\newtheorem{conjecture}[theorem]{Conjecture}}{}
\@ifundefined{proposition}{\newtheorem{proposition}[theorem]{Proposition}}{}
\@ifundefined{axiom}{\newtheorem{axiom}{Axiom}}{}
\@ifundefined{remark}{\newtheorem{remark}{Remark}}{}
\@ifundefined{example}{\newtheorem{example}{Example}}{}
\@ifundefined{exercise}{\newtheorem{exercise}{Exercise}}{}
\@ifundefined{definition}{\newtheorem{definition}{Definition}}{}


\@ifundefined{mathletters}{%
  %\def\theequation{\arabic{equation}}
  \newcounter{equationnumber}  
  \def\mathletters{%
     \addtocounter{equation}{1}
     \edef\@currentlabel{\theequation}%
     \setcounter{equationnumber}{\c@equation}
     \setcounter{equation}{0}%
     \edef\theequation{\@currentlabel\noexpand\alph{equation}}%
  }
  \def\endmathletters{%
     \setcounter{equation}{\value{equationnumber}}%
  }
}{}

%Logos
\@ifundefined{BibTeX}{%
    \def\BibTeX{{\rm B\kern-.05em{\sc i\kern-.025em b}\kern-.08em
                 T\kern-.1667em\lower.7ex\hbox{E}\kern-.125emX}}}{}%
\@ifundefined{AmS}%
    {\def\AmS{{\protect\usefont{OMS}{cmsy}{m}{n}%
                A\kern-.1667em\lower.5ex\hbox{M}\kern-.125emS}}}{}%
\@ifundefined{AmSTeX}{\def\AmSTeX{\protect\AmS-\protect\TeX\@}}{}%
%

% This macro is a fix to eqnarray
\def\@@eqncr{\let\@tempa\relax
    \ifcase\@eqcnt \def\@tempa{& & &}\or \def\@tempa{& &}%
      \else \def\@tempa{&}\fi
     \@tempa
     \if@eqnsw
        \iftag@
           \@taggnum
        \else
           \@eqnnum\stepcounter{equation}%
        \fi
     \fi
     \global\tag@false
     \global\@eqnswtrue
     \global\@eqcnt\z@\cr}


\def\TCItag{\@ifnextchar*{\@TCItagstar}{\@TCItag}}
\def\@TCItag#1{%
    \global\tag@true
    \global\def\@taggnum{(#1)}%
    \global\def\@currentlabel{#1}}
\def\@TCItagstar*#1{%
    \global\tag@true
    \global\def\@taggnum{#1}%
    \global\def\@currentlabel{#1}}
%
%%%%%%%%%%%%%%%%%%%%%%%%%%%%%%%%%%%%%%%%%%%%%%%%%%%%%%%%%%%%%%%%%%%%%
%
\def\QATOP#1#2{{#1 \atop #2}}%
\def\QTATOP#1#2{{\textstyle {#1 \atop #2}}}%
\def\QDATOP#1#2{{\displaystyle {#1 \atop #2}}}%
\def\QABOVE#1#2#3{{#2 \above#1 #3}}%
\def\QTABOVE#1#2#3{{\textstyle {#2 \above#1 #3}}}%
\def\QDABOVE#1#2#3{{\displaystyle {#2 \above#1 #3}}}%
\def\QOVERD#1#2#3#4{{#3 \overwithdelims#1#2 #4}}%
\def\QTOVERD#1#2#3#4{{\textstyle {#3 \overwithdelims#1#2 #4}}}%
\def\QDOVERD#1#2#3#4{{\displaystyle {#3 \overwithdelims#1#2 #4}}}%
\def\QATOPD#1#2#3#4{{#3 \atopwithdelims#1#2 #4}}%
\def\QTATOPD#1#2#3#4{{\textstyle {#3 \atopwithdelims#1#2 #4}}}%
\def\QDATOPD#1#2#3#4{{\displaystyle {#3 \atopwithdelims#1#2 #4}}}%
\def\QABOVED#1#2#3#4#5{{#4 \abovewithdelims#1#2#3 #5}}%
\def\QTABOVED#1#2#3#4#5{{\textstyle 
   {#4 \abovewithdelims#1#2#3 #5}}}%
\def\QDABOVED#1#2#3#4#5{{\displaystyle 
   {#4 \abovewithdelims#1#2#3 #5}}}%
%
% Macros for text size operators:
%
\def\tint{\mathop{\textstyle \int}}%
\def\tiint{\mathop{\textstyle \iint }}%
\def\tiiint{\mathop{\textstyle \iiint }}%
\def\tiiiint{\mathop{\textstyle \iiiint }}%
\def\tidotsint{\mathop{\textstyle \idotsint }}%
\def\toint{\mathop{\textstyle \oint}}%
\def\tsum{\mathop{\textstyle \sum }}%
\def\tprod{\mathop{\textstyle \prod }}%
\def\tbigcap{\mathop{\textstyle \bigcap }}%
\def\tbigwedge{\mathop{\textstyle \bigwedge }}%
\def\tbigoplus{\mathop{\textstyle \bigoplus }}%
\def\tbigodot{\mathop{\textstyle \bigodot }}%
\def\tbigsqcup{\mathop{\textstyle \bigsqcup }}%
\def\tcoprod{\mathop{\textstyle \coprod }}%
\def\tbigcup{\mathop{\textstyle \bigcup }}%
\def\tbigvee{\mathop{\textstyle \bigvee }}%
\def\tbigotimes{\mathop{\textstyle \bigotimes }}%
\def\tbiguplus{\mathop{\textstyle \biguplus }}%
%
%
%Macros for display size operators:
%
\def\dint{\mathop{\displaystyle \int}}%
\def\diint{\mathop{\displaystyle \iint}}%
\def\diiint{\mathop{\displaystyle \iiint}}%
\def\diiiint{\mathop{\displaystyle \iiiint }}%
\def\didotsint{\mathop{\displaystyle \idotsint }}%
\def\doint{\mathop{\displaystyle \oint}}%
\def\dsum{\mathop{\displaystyle \sum }}%
\def\dprod{\mathop{\displaystyle \prod }}%
\def\dbigcap{\mathop{\displaystyle \bigcap }}%
\def\dbigwedge{\mathop{\displaystyle \bigwedge }}%
\def\dbigoplus{\mathop{\displaystyle \bigoplus }}%
\def\dbigodot{\mathop{\displaystyle \bigodot }}%
\def\dbigsqcup{\mathop{\displaystyle \bigsqcup }}%
\def\dcoprod{\mathop{\displaystyle \coprod }}%
\def\dbigcup{\mathop{\displaystyle \bigcup }}%
\def\dbigvee{\mathop{\displaystyle \bigvee }}%
\def\dbigotimes{\mathop{\displaystyle \bigotimes }}%
\def\dbiguplus{\mathop{\displaystyle \biguplus }}%

\if@compatibility\else
  % Always load amsmath in LaTeX2e mode
  \RequirePackage{amsmath}
\fi

\def\ExitTCILatex{\makeatother\endinput}

\bgroup
\ifx\ds@amstex\relax
   \message{amstex already loaded}\aftergroup\ExitTCILatex
\else
   \@ifpackageloaded{amsmath}%
      {\if@compatibility\message{amsmath already loaded}\fi\aftergroup\ExitTCILatex}
      {}
   \@ifpackageloaded{amstex}%
      {\if@compatibility\message{amstex already loaded}\fi\aftergroup\ExitTCILatex}
      {}
   \@ifpackageloaded{amsgen}%
      {\if@compatibility\message{amsgen already loaded}\fi\aftergroup\ExitTCILatex}
      {}
\fi
\egroup

%Exit if any of the AMS macros are already loaded.
%This is always the case for LaTeX2e mode.


%%%%%%%%%%%%%%%%%%%%%%%%%%%%%%%%%%%%%%%%%%%%%%%%%%%%%%%%%%%%%%%%%%%%%%%%%%
% NOTE: The rest of this file is read only if in LaTeX 2.09 compatibility
% mode. This section is used to define AMS-like constructs in the
% event they have not been defined.
%%%%%%%%%%%%%%%%%%%%%%%%%%%%%%%%%%%%%%%%%%%%%%%%%%%%%%%%%%%%%%%%%%%%%%%%%%
\typeout{TCILATEX defining AMS-like constructs in LaTeX 2.09 COMPATIBILITY MODE}
%%%%%%%%%%%%%%%%%%%%%%%%%%%%%%%%%%%%%%%%%%%%%%%%%%%%%%%%%%%%%%%%%%%%%%%%
%  Macros to define some AMS LaTeX constructs when 
%  AMS LaTeX has not been loaded
% 
% These macros are copied from the AMS-TeX package for doing
% multiple integrals.
%
\let\DOTSI\relax
\def\RIfM@{\relax\ifmmode}%
\def\FN@{\futurelet\next}%
\newcount\intno@
\def\iint{\DOTSI\intno@\tw@\FN@\ints@}%
\def\iiint{\DOTSI\intno@\thr@@\FN@\ints@}%
\def\iiiint{\DOTSI\intno@4 \FN@\ints@}%
\def\idotsint{\DOTSI\intno@\z@\FN@\ints@}%
\def\ints@{\findlimits@\ints@@}%
\newif\iflimtoken@
\newif\iflimits@
\def\findlimits@{\limtoken@true\ifx\next\limits\limits@true
 \else\ifx\next\nolimits\limits@false\else
 \limtoken@false\ifx\ilimits@\nolimits\limits@false\else
 \ifinner\limits@false\else\limits@true\fi\fi\fi\fi}%
\def\multint@{\int\ifnum\intno@=\z@\intdots@                          %1
 \else\intkern@\fi                                                    %2
 \ifnum\intno@>\tw@\int\intkern@\fi                                   %3
 \ifnum\intno@>\thr@@\int\intkern@\fi                                 %4
 \int}%                                                               %5
\def\multintlimits@{\intop\ifnum\intno@=\z@\intdots@\else\intkern@\fi
 \ifnum\intno@>\tw@\intop\intkern@\fi
 \ifnum\intno@>\thr@@\intop\intkern@\fi\intop}%
\def\intic@{%
    \mathchoice{\hskip.5em}{\hskip.4em}{\hskip.4em}{\hskip.4em}}%
\def\negintic@{\mathchoice
 {\hskip-.5em}{\hskip-.4em}{\hskip-.4em}{\hskip-.4em}}%
\def\ints@@{\iflimtoken@                                              %1
 \def\ints@@@{\iflimits@\negintic@
   \mathop{\intic@\multintlimits@}\limits                             %2
  \else\multint@\nolimits\fi                                          %3
  \eat@}%                                                             %4
 \else                                                                %5
 \def\ints@@@{\iflimits@\negintic@
  \mathop{\intic@\multintlimits@}\limits\else
  \multint@\nolimits\fi}\fi\ints@@@}%
\def\intkern@{\mathchoice{\!\!\!}{\!\!}{\!\!}{\!\!}}%
\def\plaincdots@{\mathinner{\cdotp\cdotp\cdotp}}%
\def\intdots@{\mathchoice{\plaincdots@}%
 {{\cdotp}\mkern1.5mu{\cdotp}\mkern1.5mu{\cdotp}}%
 {{\cdotp}\mkern1mu{\cdotp}\mkern1mu{\cdotp}}%
 {{\cdotp}\mkern1mu{\cdotp}\mkern1mu{\cdotp}}}%
%
%
%  These macros are for doing the AMS \text{} construct
%
\def\RIfM@{\relax\protect\ifmmode}
\def\text{\RIfM@\expandafter\text@\else\expandafter\mbox\fi}
\let\nfss@text\text
\def\text@#1{\mathchoice
   {\textdef@\displaystyle\f@size{#1}}%
   {\textdef@\textstyle\tf@size{\firstchoice@false #1}}%
   {\textdef@\textstyle\sf@size{\firstchoice@false #1}}%
   {\textdef@\textstyle \ssf@size{\firstchoice@false #1}}%
   \glb@settings}

\def\textdef@#1#2#3{\hbox{{%
                    \everymath{#1}%
                    \let\f@size#2\selectfont
                    #3}}}
\newif\iffirstchoice@
\firstchoice@true
%
%These are the AMS constructs for multiline limits.
%
\def\Let@{\relax\iffalse{\fi\let\\=\cr\iffalse}\fi}%
\def\vspace@{\def\vspace##1{\crcr\noalign{\vskip##1\relax}}}%
\def\multilimits@{\bgroup\vspace@\Let@
 \baselineskip\fontdimen10 \scriptfont\tw@
 \advance\baselineskip\fontdimen12 \scriptfont\tw@
 \lineskip\thr@@\fontdimen8 \scriptfont\thr@@
 \lineskiplimit\lineskip
 \vbox\bgroup\ialign\bgroup\hfil$\m@th\scriptstyle{##}$\hfil\crcr}%
\def\Sb{_\multilimits@}%
\def\endSb{\crcr\egroup\egroup\egroup}%
\def\Sp{^\multilimits@}%
\let\endSp\endSb
%
%
%These are AMS constructs for horizontal arrows
%
\newdimen\ex@
\ex@.2326ex
\def\rightarrowfill@#1{$#1\m@th\mathord-\mkern-6mu\cleaders
 \hbox{$#1\mkern-2mu\mathord-\mkern-2mu$}\hfill
 \mkern-6mu\mathord\rightarrow$}%
\def\leftarrowfill@#1{$#1\m@th\mathord\leftarrow\mkern-6mu\cleaders
 \hbox{$#1\mkern-2mu\mathord-\mkern-2mu$}\hfill\mkern-6mu\mathord-$}%
\def\leftrightarrowfill@#1{$#1\m@th\mathord\leftarrow
\mkern-6mu\cleaders
 \hbox{$#1\mkern-2mu\mathord-\mkern-2mu$}\hfill
 \mkern-6mu\mathord\rightarrow$}%
\def\overrightarrow{\mathpalette\overrightarrow@}%
\def\overrightarrow@#1#2{\vbox{\ialign{##\crcr\rightarrowfill@#1\crcr
 \noalign{\kern-\ex@\nointerlineskip}$\m@th\hfil#1#2\hfil$\crcr}}}%
\let\overarrow\overrightarrow
\def\overleftarrow{\mathpalette\overleftarrow@}%
\def\overleftarrow@#1#2{\vbox{\ialign{##\crcr\leftarrowfill@#1\crcr
 \noalign{\kern-\ex@\nointerlineskip}$\m@th\hfil#1#2\hfil$\crcr}}}%
\def\overleftrightarrow{\mathpalette\overleftrightarrow@}%
\def\overleftrightarrow@#1#2{\vbox{\ialign{##\crcr
   \leftrightarrowfill@#1\crcr
 \noalign{\kern-\ex@\nointerlineskip}$\m@th\hfil#1#2\hfil$\crcr}}}%
\def\underrightarrow{\mathpalette\underrightarrow@}%
\def\underrightarrow@#1#2{\vtop{\ialign{##\crcr$\m@th\hfil#1#2\hfil
  $\crcr\noalign{\nointerlineskip}\rightarrowfill@#1\crcr}}}%
\let\underarrow\underrightarrow
\def\underleftarrow{\mathpalette\underleftarrow@}%
\def\underleftarrow@#1#2{\vtop{\ialign{##\crcr$\m@th\hfil#1#2\hfil
  $\crcr\noalign{\nointerlineskip}\leftarrowfill@#1\crcr}}}%
\def\underleftrightarrow{\mathpalette\underleftrightarrow@}%
\def\underleftrightarrow@#1#2{\vtop{\ialign{##\crcr$\m@th
  \hfil#1#2\hfil$\crcr
 \noalign{\nointerlineskip}\leftrightarrowfill@#1\crcr}}}%
%%%%%%%%%%%%%%%%%%%%%

\def\qopnamewl@#1{\mathop{\operator@font#1}\nlimits@}
\let\nlimits@\displaylimits
\def\setboxz@h{\setbox\z@\hbox}


\def\varlim@#1#2{\mathop{\vtop{\ialign{##\crcr
 \hfil$#1\m@th\operator@font lim$\hfil\crcr
 \noalign{\nointerlineskip}#2#1\crcr
 \noalign{\nointerlineskip\kern-\ex@}\crcr}}}}

 \def\rightarrowfill@#1{\m@th\setboxz@h{$#1-$}\ht\z@\z@
  $#1\copy\z@\mkern-6mu\cleaders
  \hbox{$#1\mkern-2mu\box\z@\mkern-2mu$}\hfill
  \mkern-6mu\mathord\rightarrow$}
\def\leftarrowfill@#1{\m@th\setboxz@h{$#1-$}\ht\z@\z@
  $#1\mathord\leftarrow\mkern-6mu\cleaders
  \hbox{$#1\mkern-2mu\copy\z@\mkern-2mu$}\hfill
  \mkern-6mu\box\z@$}


\def\projlim{\qopnamewl@{proj\,lim}}
\def\injlim{\qopnamewl@{inj\,lim}}
\def\varinjlim{\mathpalette\varlim@\rightarrowfill@}
\def\varprojlim{\mathpalette\varlim@\leftarrowfill@}
\def\varliminf{\mathpalette\varliminf@{}}
\def\varliminf@#1{\mathop{\underline{\vrule\@depth.2\ex@\@width\z@
   \hbox{$#1\m@th\operator@font lim$}}}}
\def\varlimsup{\mathpalette\varlimsup@{}}
\def\varlimsup@#1{\mathop{\overline
  {\hbox{$#1\m@th\operator@font lim$}}}}

%
%Companion to stackrel
\def\stackunder#1#2{\mathrel{\mathop{#2}\limits_{#1}}}%
%
%
% These are AMS environments that will be defined to
% be verbatims if amstex has not actually been 
% loaded
%
%
\begingroup \catcode `|=0 \catcode `[= 1
\catcode`]=2 \catcode `\{=12 \catcode `\}=12
\catcode`\\=12 
|gdef|@alignverbatim#1\end{align}[#1|end[align]]
|gdef|@salignverbatim#1\end{align*}[#1|end[align*]]

|gdef|@alignatverbatim#1\end{alignat}[#1|end[alignat]]
|gdef|@salignatverbatim#1\end{alignat*}[#1|end[alignat*]]

|gdef|@xalignatverbatim#1\end{xalignat}[#1|end[xalignat]]
|gdef|@sxalignatverbatim#1\end{xalignat*}[#1|end[xalignat*]]

|gdef|@gatherverbatim#1\end{gather}[#1|end[gather]]
|gdef|@sgatherverbatim#1\end{gather*}[#1|end[gather*]]

|gdef|@gatherverbatim#1\end{gather}[#1|end[gather]]
|gdef|@sgatherverbatim#1\end{gather*}[#1|end[gather*]]


|gdef|@multilineverbatim#1\end{multiline}[#1|end[multiline]]
|gdef|@smultilineverbatim#1\end{multiline*}[#1|end[multiline*]]

|gdef|@arraxverbatim#1\end{arrax}[#1|end[arrax]]
|gdef|@sarraxverbatim#1\end{arrax*}[#1|end[arrax*]]

|gdef|@tabulaxverbatim#1\end{tabulax}[#1|end[tabulax]]
|gdef|@stabulaxverbatim#1\end{tabulax*}[#1|end[tabulax*]]


|endgroup
  

  
\def\align{\@verbatim \frenchspacing\@vobeyspaces \@alignverbatim
You are using the "align" environment in a style in which it is not defined.}
\let\endalign=\endtrivlist
 
\@namedef{align*}{\@verbatim\@salignverbatim
You are using the "align*" environment in a style in which it is not defined.}
\expandafter\let\csname endalign*\endcsname =\endtrivlist




\def\alignat{\@verbatim \frenchspacing\@vobeyspaces \@alignatverbatim
You are using the "alignat" environment in a style in which it is not defined.}
\let\endalignat=\endtrivlist
 
\@namedef{alignat*}{\@verbatim\@salignatverbatim
You are using the "alignat*" environment in a style in which it is not defined.}
\expandafter\let\csname endalignat*\endcsname =\endtrivlist




\def\xalignat{\@verbatim \frenchspacing\@vobeyspaces \@xalignatverbatim
You are using the "xalignat" environment in a style in which it is not defined.}
\let\endxalignat=\endtrivlist
 
\@namedef{xalignat*}{\@verbatim\@sxalignatverbatim
You are using the "xalignat*" environment in a style in which it is not defined.}
\expandafter\let\csname endxalignat*\endcsname =\endtrivlist




\def\gather{\@verbatim \frenchspacing\@vobeyspaces \@gatherverbatim
You are using the "gather" environment in a style in which it is not defined.}
\let\endgather=\endtrivlist
 
\@namedef{gather*}{\@verbatim\@sgatherverbatim
You are using the "gather*" environment in a style in which it is not defined.}
\expandafter\let\csname endgather*\endcsname =\endtrivlist


\def\multiline{\@verbatim \frenchspacing\@vobeyspaces \@multilineverbatim
You are using the "multiline" environment in a style in which it is not defined.}
\let\endmultiline=\endtrivlist
 
\@namedef{multiline*}{\@verbatim\@smultilineverbatim
You are using the "multiline*" environment in a style in which it is not defined.}
\expandafter\let\csname endmultiline*\endcsname =\endtrivlist


\def\arrax{\@verbatim \frenchspacing\@vobeyspaces \@arraxverbatim
You are using a type of "array" construct that is only allowed in AmS-LaTeX.}
\let\endarrax=\endtrivlist

\def\tabulax{\@verbatim \frenchspacing\@vobeyspaces \@tabulaxverbatim
You are using a type of "tabular" construct that is only allowed in AmS-LaTeX.}
\let\endtabulax=\endtrivlist

 
\@namedef{arrax*}{\@verbatim\@sarraxverbatim
You are using a type of "array*" construct that is only allowed in AmS-LaTeX.}
\expandafter\let\csname endarrax*\endcsname =\endtrivlist

\@namedef{tabulax*}{\@verbatim\@stabulaxverbatim
You are using a type of "tabular*" construct that is only allowed in AmS-LaTeX.}
\expandafter\let\csname endtabulax*\endcsname =\endtrivlist

% macro to simulate ams tag construct


% This macro is a fix to the equation environment
 \def\endequation{%
     \ifmmode\ifinner % FLEQN hack
      \iftag@
        \addtocounter{equation}{-1} % undo the increment made in the begin part
        $\hfil
           \displaywidth\linewidth\@taggnum\egroup \endtrivlist
        \global\tag@false
        \global\@ignoretrue   
      \else
        $\hfil
           \displaywidth\linewidth\@eqnnum\egroup \endtrivlist
        \global\tag@false
        \global\@ignoretrue 
      \fi
     \else   
      \iftag@
        \addtocounter{equation}{-1} % undo the increment made in the begin part
        \eqno \hbox{\@taggnum}
        \global\tag@false%
        $$\global\@ignoretrue
      \else
        \eqno \hbox{\@eqnnum}% $$ BRACE MATCHING HACK
        $$\global\@ignoretrue
      \fi
     \fi\fi
 } 

 \newif\iftag@ \tag@false
 
 \def\TCItag{\@ifnextchar*{\@TCItagstar}{\@TCItag}}
 \def\@TCItag#1{%
     \global\tag@true
     \global\def\@taggnum{(#1)}%
     \global\def\@currentlabel{#1}}
 \def\@TCItagstar*#1{%
     \global\tag@true
     \global\def\@taggnum{#1}%
     \global\def\@currentlabel{#1}}

  \@ifundefined{tag}{
     \def\tag{\@ifnextchar*{\@tagstar}{\@tag}}
     \def\@tag#1{%
         \global\tag@true
         \global\def\@taggnum{(#1)}}
     \def\@tagstar*#1{%
         \global\tag@true
         \global\def\@taggnum{#1}}
  }{}

\def\tfrac#1#2{{\textstyle {#1 \over #2}}}%
\def\dfrac#1#2{{\displaystyle {#1 \over #2}}}%
\def\binom#1#2{{#1 \choose #2}}%
\def\tbinom#1#2{{\textstyle {#1 \choose #2}}}%
\def\dbinom#1#2{{\displaystyle {#1 \choose #2}}}%

% Do not add anything to the end of this file.  
% The last section of the file is loaded only if 
% amstex has not been.
\makeatother
\endinput


\begin{document}


\subsection{陀螺的簡易數值模%
擬}

我們知道剛體轉動中%
從牛頓定律出發而得%
到的尤拉運動方程給%
出的是角速度在body座標%
的分量,一般力學的%
書上說明到角速度的%
尤拉運動方程,就會%
轉而求諸euler angle來得到Lagrangian%
,接著如果是陀螺的%
例子就以elliptical integral解出解%
析解,然後就可以模%
擬其運動。這樣做可%
以得到最完整的解析%
解。不過若只想要模%
擬其運動,並不是那%
麼在意有沒有得到解%
析解,我這邊分享一%
個只要得到body分量角速%
度的尤拉運動方程加%
上給定初始值,就可%
以數值模擬陀螺轉動%
,雖然無法得到解析%
解,但以此方法寫出%
的python模擬程式相較於%
Lagrangian方法簡單許多。並%
且此處所涉及的物理%
及body角速度的理解還可%
應用上許多相關領域%
。因此這邊我們呈現%
以此body角速度如何數值%
模擬完整的陀螺運動%
。

這裡以陀螺為例子,%
將陀螺的運動分解成%
為很多個t到t+dt時間的微%
小轉動,此作法有了%
兩個目的:

\begin{enumerate}
\item 證明Euler equation中的陀螺角%
速度在body的分量就是t到%
t+dt時間陀螺的body frame在space frame%
的轉動角速度及轉動%
矩陣。

\item 在數值解剛體轉動%
尤拉方程時,t到t+dt時間%
的微小座標運動將可%
以非常準確地以在t+dt時%
間的body角速度來近似,%
即轉動矩陣$[1+\omega _{b}(t+dt)]\times dt%
\footnote{%
The path order intergral of $\omega _{b}$ from time t to t+dt. This will be
discussed more in the text.}$,這裡將展示%
以此近似加上用簡單%
的四階Runge Kutta就可以解出%
很精確的陀螺運動。(%
目前一般陀螺運動都%
需要蠻高階的數值計%
算來避免其中一些守%
恆的運動量退化。此%
方法將以非常簡單的%
四階Runge Kutta來得到非常理%
想的結果。)
\end{enumerate}

\begin{summary}
一般認為解出陀螺的%
角速度在body上的分量並%
無太大用處,這裡我%
們證明了只要對陀螺%
物理及線性代換瞭解%
透徹,依然能夠用數%
值計算的方法完整模%
擬三維空間陀螺的運%
動。
\end{summary}

首先我們先討論向量%
變化量在不同觀測座%
標中的關係。由於當%
我實際在解這問題時%
我發現Goldstein classical mechanics書中描%
述的還不是那麼清楚%
,因此這邊寫上我認%
為可以補充書上的推%
導證明。

\begin{equation}
\left( \frac{d\vec{L}}{dt}\right) _{s}=\left( \frac{d\vec{L}}{dt}\right)
_{b}+\vec{\omega}\times \vec{L}
\end{equation}

此公式如何而來?此%
公式為一隨時間變動%
的向量在恆定座標與%
非恆定座標(此例為轉%
動中座標)之間線性變%
換的結果。

\FRAME{fthF}{1.6042in}{1.407in}{0in}{}{}{vecratechange.eps}{\special%
{language "Scientific Word";type "GRAPHIC";maintain-aspect-ratio
TRUE;display "USEDEF";valid_file "F";width 1.6042in;height 1.407in;depth
0in;original-width 1.5679in;original-height 1.3716in;cropleft "0";croptop
"1";cropright "1";cropbottom "0";filename '../../Google
Drive/researchstuff/digital machine aided drawing/canvas 11/coin
rolling/vecratechange.eps';file-properties "NPEU";}}\bigskip

在恆定座標S(space)中,在%
時間為t的時候一向量$%
\vec{A}$,過了時間dt後,觀%
察值為$\vec{A}+d\vec{A}$,%
\begin{eqnarray}
t &:&(\vec{A})_{s} \\
t+dt &:&\left( \vec{A}+d\vec{A}\right) _{s}
\end{eqnarray}%
在轉動座標b(body)中,時%
間為t的時候$\vec{A}$向量在b%
中的觀察值為$\left( \vec{A}\right) _{b}$%
(= $\Omega \times \left( \vec{A}\right) _{s}$ ,Ω是S frame%
到b frame的轉動矩陣)。時%
間為t+dt的時候body中觀察%
值為$\left( \vec{A}+d\vec{A}\right) _{b}$,%
\begin{eqnarray}
t &:&\left( \vec{A}\right) _{b} \\
t+dt &:&\left( \vec{A}+d\vec{A}\right) _{b}
\end{eqnarray}%
在t+dt時必須滿足%
\begin{equation}
\left( \vec{A}+d\vec{A}\right) _{b}=\underset{passive}{\left( \Omega
+d\Omega \right) }\times \left( \vec{A}+d\vec{A}\right) _{s}
\label{finiterotmatrix}
\end{equation}%
passive表示轉動矩陣取其%
被動含意,也就是轉%
坐標軸,上式展開後%
得%
\begin{equation}
\Omega \times \left( d\vec{A}\right) _{s}=\left( d\vec{A}\right)
_{b}+d\Omega \times \left( \vec{A}\right) _{s}
\end{equation}

現在若假設t時間S,b座%
標重合,則$\Omega $=Unity,$\left( \vec{A%
}\right) _{s}=\left( \vec{A}\right) _{b}$,則上式%
\begin{equation}
\left( d\vec{A}\right) _{s}=\left( d\vec{A}\right) _{b}+d\Omega \times
\left( \vec{A}\right) _{b}  \label{vectorrateofchange}
\end{equation}

\bigskip 我們將在陀螺的每%
一段t到t+dt的分解運動運%
用每個t時S,b座標重合%
的事實,也就是運用%
上\ref{vectorrateofchange}式。也就是%
我們將持續地改變space座%
標來符合這個條件,%
但是當然我們會跟蹤space%
座標每一個變換的位%
置來最終模擬陀螺運%
動‧(注意\ref{vectorrateofchange}式中$%
\Omega ,d\Omega $都是s到b的轉動矩%
陣)。

另外,在t時s,b重和情%
況下若我們加上考慮$%
\vec{A}$向量fixed在b座標中不%
變動(如body的xyz軸,在body座%
標中為固定值),並且%
考慮轉動矩陣的主動%
特性,則\ref{finiterotmatrix}式變成%
\begin{equation}
\left( \hat{z}_{s}\right) _{s}=\underset{active}{\left( 1+d\Omega \right) }%
\times \left( \hat{z}_{b}\right) _{s}  \label{baxesrot}
\end{equation}%
首先\ref{baxesrot}式很重要的%
一個特性是,當我們%
應用上轉動矩陣的主%
動特性,從\ref{baxesrot}式可%
以知道轉動矩陣$1+d\Omega $主%
動地把body的z軸$\left( \hat{z}_{b}\right) _{s}$%
轉到space的z軸$\left( \hat{z}_{s}\right) _{s}$%
,這也代表轉動矩陣$%
1+d\Omega $取其主動特性時其%
所屬的轉動矩陣作用%
在body z軸等同於body z 軸在space%
空間的轉動,也就是$%
\frac{d\Omega }{dt}$為body z軸在空間的%
角速度!(在時間為t到%
t+dt的時候),此事實對fixed%
在body上的任意向量都成%
立,所以body x, y軸也成立%
。另一個看法是,因%
為 space 與 body frame在t時重合,%
因此space的被動轉動矩陣%
$d\Omega $ = body的主動轉動矩陣$%
d\Omega $,稍後我們將會應%
用上這一重要的特性%
。這裡的結論是,主%
動特性的$d\Omega $等同於body%
在space空間的微小轉動。

\begin{remark}
Goldstein中$d\mathbf{\Omega }\times \mathbf{G}$中的\textbf{%
G}是在space座標,但因t時s,b%
兩座標重和,用s或body是%
都可以的。這也是為%
什麼goldstein p176(version 2)說明,\ref%
{vectorrateofchange}式中$\vec{A}$或$\vec{b}$向%
量沿著space或沿著body方向%
取分量都是可以的。%
而Goldstein後來用上陀螺後%
,是取body方向的分量沒%
錯。
\end{remark}

\begin{remark}
但我們必須強調,任%
意情況下,角速度$\left( \vec{%
\omega}\right) $在body轉動座標下的%
投影並不是body座標上觀%
察到的角速度!這裡%
我們是有條件的考慮t%
到t+dt時刻的t時刻s,b座標%
重和。
\end{remark}

\begin{remark}
角速度與Lie algebra相關連,%
SO(3) tangent,$\dot{\Omega}^{T}=-\dot{\Omega}$,$e^{\Omega }$%
是orthogonal matrix,wikipedia(orthogonal transformation)上%
有些線索。
\end{remark}

接下來用上陀螺,取%
陀螺的自旋軸為body z軸%
,取$\vec{A}$向量為陀螺角%
動量$\vec{L}$,以及對\ref%
{vectorrateofchange}式取微分即得到

\begin{equation}
\left( \vec{\Gamma}\right) _{s}=\left( \frac{d\vec{L}}{dt}\right)
_{s}=\left( \frac{d\vec{L}}{dt}\right) _{b}+d\dot{\Omega}\times \left( \vec{L%
}\right) _{b}  \label{newton1}
\end{equation}%
其中$\vec{\Gamma}$為Torque,$d\dot{\Omega}$我%
們已經證明是body xyz軸相%
對於space frame的轉動角速度%
,因euler theorem知轉動矩陣$%
d\Omega $可看成一向量,因%
此此向量 $d\Omega $即等於 $\left( 
\vec{\omega}\right) _{b}\cdot dt$,重新整理%
後得%
\begin{equation}
\left( \vec{\Gamma}\right) _{s}=\left( \frac{dI\vec{\omega}}{dt}\right) _{b}+%
\vec{\omega}\times \left( I\vec{\omega}\right) _{b}  \label{newton2}
\end{equation}%
(也可從微小轉動(infinitesimal
rotation)的向量相加特性考%
慮,我們可以考慮$d\Omega $%
的微小轉動是body x y z軸分%
別的微小轉動而組成%
,這代表我們可以寫$%
d\Omega =(\omega _{x}\hat{x}_{b}+\omega _{y}\hat{y}_{b}+\omega _{z}\hat{z}%
_{b})dt$,$d\Omega $即等於 $\left( \vec{\omega}%
\right) _{b}\cdot dt$)。另外在 body axis中$I$%
是 diagonal的,因此$\left( \vec{L}\right)
_{b}=I\times $ 角速度在body的分量$%
\vec{\omega}$,因此$(\vec{L})_{b}=I\times \vec{\omega}$%
。若加上考慮陀螺的%
條件 $I_{x}=I_{y}\neq I_{z}$,Eq.(\ref{newton2})可%
以寫成%
\begin{eqnarray}
\Gamma _{x} &=&I_{x}\dot{\omega}_{x}+(I_{z}-I_{y})\omega _{y}\omega _{z} \\
\Gamma _{y} &=&I_{y}\dot{\omega}_{y}+(I_{x}-I_{z})\omega _{x}\omega _{z} \\
\Gamma _{z} &=&I_{z}\dot{\omega}_{z}=0
\end{eqnarray}%
重新整理得%
\begin{eqnarray}
\dot{\omega}_{x} &=&-\frac{I_{z}-I_{y}}{I_{x}}\omega _{y}\omega _{z}+\frac{%
\Gamma _{x}}{I_{x}} \\
\dot{\omega}_{y} &=&-\frac{I_{x}-I_{z}}{I_{y}}\omega _{x}\omega _{z}+\frac{%
\Gamma _{y}}{I_{y}} \\
\dot{\omega}_{z} &=&0
\end{eqnarray}%
此方程組也可寫成%
\begin{equation}
\frac{d}{dt}\left[ 
\begin{array}{c}
\omega _{x} \\ 
\omega _{y} \\ 
\omega _{z}%
\end{array}%
\right] =\left[ 
\begin{array}{ccc}
0 & -\frac{I_{z}-I_{y}}{I_{x}} & 0 \\ 
-\frac{I_{x}-I_{z}}{I_{y}} & 0 & 0 \\ 
0 & 0 & 0%
\end{array}%
\right] \left[ 
\begin{array}{c}
\omega _{x} \\ 
\omega _{y} \\ 
\omega _{z}%
\end{array}%
\right] +\left[ 
\begin{array}{c}
\frac{\Gamma _{x}}{I_{x}} \\ 
\frac{\Gamma _{y}}{I_{y}} \\ 
\frac{\Gamma _{z}}{I_{z}}%
\end{array}%
\right]
\end{equation}%
以上的微分方程組可%
以用Ruge Kutta求出$\vec{\omega}(t)$,另%
外,此角速度雖然是%
沿著body軸的分量,但我%
們也證明了此角速度%
在任意一段t到t+dt的時間%
中代表了body座標軸在space%
座標上的轉動,接下%
來我們就說明如何以%
這個特性重構出陀螺%
在三維空間的運動。%
\bigskip

\FRAME{fthF}{3.0139in}{2.9006in}{0in}{}{}{top.eps}{\special{language
"Scientific Word";type "GRAPHIC";maintain-aspect-ratio TRUE;display
"USEDEF";valid_file "F";width 3.0139in;height 2.9006in;depth
0in;original-width 2.9698in;original-height 2.8565in;cropleft "0";croptop
"1";cropright "1";cropbottom "0";filename '../../Google
Drive/researchstuff/digital machine aided drawing/canvas 11/coin
rolling/top.eps';file-properties "NPEU";}}

由之前討論知道,雖%
然$\left( \vec{\omega}(t)\right) _{b}$是角速度%
沿著body軸的分量,並不%
代表是body座標中觀察到%
的角速度,但我們知%
道若只考慮t到t+dt時間,%
在t時間時s與b重和,在%
此條件下我們證明了$%
\left( \vec{\omega}(t)\right) _{b}$在t到t+dt時是body%
軸在space座標的轉動矩陣%
,也就是body軸的轉動速%
度,因此只要給定初%
始條件,接著一步一%
步的算出body軸在space frame中%
的位置,即可得到三%
維陀螺運動。\bigskip

\FRAME{fthF}{2.2433in}{2.7769in}{0in}{}{}{initialsetup.eps}{\special%
{language "Scientific Word";type "GRAPHIC";maintain-aspect-ratio
TRUE;display "USEDEF";valid_file "F";width 2.2433in;height 2.7769in;depth
0in;original-width 2.2035in;original-height 2.7345in;cropleft "0";croptop
"1";cropright "1";cropbottom "0";filename '../../Google
Drive/researchstuff/digital machine aided drawing/canvas 11/coin
rolling/initialsetup.eps';file-properties "NPEU";}}

假設陀螺起始位置已%
知$\left( z_{s}(t_{0})\right) $,此為body軸z%
軸且為單位向量,下%
標s代表的是觀測的frame%
。所以$z_{s}(t_{0})$代表時間%
是t$_{0}$的時候,z軸在space frame%
中的位置向量,$z_{s}(t_{0})$%
與$z_{s}$不一樣,$z_{s}$為恆%
定座標space frame z axis,見圖,%
我們可藉由一轉動矩%
陣$\mathbf{\Omega (orien)}$輕易改變陀%
螺的初始位置$z_{s}(t_{0})=\mathbf{\Omega
(}orien\mathbf{)}\times z_{s}$,orien為轉動向%
量。$z_{0}(t_{1})$則代表時間%
是t$_{1}$的時候z軸在t$_{0}$時%
的座標軸觀察到的位%
置向量,而$z_{0}(t_{0})=z_{1}(t_{1})=\left[ 
\begin{array}{ccc}
0 & 0 & 1%
\end{array}%
\right] $都是該時間上的座%
標軸上的單位軸向量%
。因此我們要得到的%
是$z_{s}(t_{1\symbol{126}N})$,即z軸時間%
上的變化在space中的向量%
值。在我們以數值法%
求得$\left( \vec{\omega}(t_{i})\right) _{b}$之後%
,接著可以用以下方%
法來求得$z_{s}(t_{i})$。首先%
先定義一些符號,這%
裡$\overset{active}{\overbrace{\mathbf{[}t_{i},t_{i-1}\mathbf{]}}}$%
代表t$_{i-1}$時的座標軸到t$%
_{i}$時的座標軸的轉動矩%
陣,取矩陣的主動特%
性,也就是$z_{i-1}(t_{i})=\overset{active}{%
\overbrace{\mathbf{[}t_{i},t_{i-1}\mathbf{]}}}\times z_{i-1}(t_{i-1})=%
\overset{active}{\overbrace{\mathbf{[}t_{i},t_{i-1}\mathbf{]}}}\times \left[ 
\begin{array}{ccc}
0 & 0 & 1%
\end{array}%
\right] $,當上面標示改成%
passive時,$\overset{passive}{\overbrace{\mathbf{[}t_{i},t_{i-1}%
\mathbf{]}}}$時取轉動矩陣的被%
動性質,也就是轉座%
標軸,因此$\overset{passive}{\overbrace{%
\mathbf{[}t_{i-1},t_{i-2}\mathbf{]}}}\times z_{i-1}(t_{i})=z_{i-2}(t_{i})$%
,因此%
\begin{eqnarray}
z_{s}(t_{i}) &=&\overset{passive}{\overbrace{\mathbf{[}t_{0},s\mathbf{]}}}%
\mathbf{\times }\overset{passive}{\overbrace{\mathbf{[}t_{1},t_{0}\mathbf{]}}%
}\cdots \\
&&\times \overset{passive}{\overbrace{\mathbf{[}t_{i-2},t_{i-3}\mathbf{]}}}%
\mathbf{\times }\underset{\text{t}_{i}\text{時的z在t}_{i-2}%
\text{時的座標的觀察值}}{%
\underbrace{\overset{passive}{\overbrace{\mathbf{[}t_{i-1},t_{i-2}\mathbf{]}}%
}\times \underset{\text{t}_{i}\text{時的z在t}_{i-1}\text{%
時的座標的觀察值}}{%
\underbrace{\overset{active}{\overbrace{\mathbf{[}t_{i},t_{i-1}\mathbf{]}}}%
\times \underset{\text{t}_{i-1}\text{時的z在t}_{i-1}\text{%
時的座標的觀察值,%
顯然為}\left[ 
\begin{array}{ccc}
0 & 0 & 1%
\end{array}%
\right] }{\underbrace{z_{i-1}(t_{i-1})}}}}}}  \label{bodytracking}
\end{eqnarray}%
以這方法我們可以求%
得所有$z_{s}(t_{i\text{, i=1}\sim N})$。但%
怎麼知道$\mathbf{[}t_{i},t_{i-1}\mathbf{]}$的%
轉動矩陣?這裡就是%
第二點個重點,我們%
近似t$_{i-1}$到t$_{i}$的座標運%
動為t$_{i-1}$的座標軸以$\vec{\omega%
}(t_{i})$的速度轉動了dt時間%
,因此,若$\mathbf{\Omega }(\vec{A})$代%
表一轉動矩陣\textbf{其轉%
軸在}$\vec{A}$方向且轉動角%
量值為$\left\vert \vec{A}\right\vert $\textbf{,%
則\ref{bodytracking}公式可寫成}%
\begin{equation}
z_{s}(t_{i})=\overset{passive}{\overbrace{\mathbf{\Omega (orien)}}}\mathbf{%
\times }\overset{passive}{\overbrace{\mathbf{\Omega }(\vec{\omega}%
(t_{1})\cdot dt)}}\cdots \overset{passive}{\overbrace{\mathbf{\Omega }(\vec{%
\omega}(t_{i-2})\cdot dt)}}\mathbf{\times }\overset{passive}{\overbrace{%
\mathbf{\Omega }(\vec{\omega}(t_{i-1})\cdot dt)}}\times \overset{active}{%
\overbrace{\mathbf{\Omega }(\vec{\omega}(t_{i})\cdot dt)}}\times
z_{i-1}(t_{i-1})
\end{equation}%
\bigskip

可以看出上面所有passive%
的矩陣的作用只是再%
把坐標軸從body frame轉回到%
space frame。因此若i=1$\sim N$我們%
可以以此求得$z_{s}(t_{i\text{, i=1}\sim
N})$,這樣我們就用上了%
之前求得的$\left( \vec{\omega}(t_{i})\right)
_{b}$。以此同方法可求得%
x與y軸的運動。以上以$%
\omega (t_{i})$來近似t到t+dt的轉動%
事實上包含更深的物%
理含意,我們是不是%
可以用$\omega (t_{i-1})$來做近似%
?以下將作解釋。

由於t到t+dt時的s,b座標重%
合,因此body軸從t到t+dt的%
轉動可以如下近似,%
由於s,b重合,$\Omega _{b}(t)=\Omega _{s}(t)$%
,我們先考慮陀螺沿%
著$\Omega _{b}(t)$轉了$exp(\Omega _{b}(t))$,但%
在t+dt時,$\Omega _{b}(t+dt)\neq \Omega _{b}(t)$,%
這代表從t到t+dt時,轉動%
向量在body座標上有變化%
,也因此我們不能夠%
單只考慮陀螺轉了$\Omega
_{b}(t)$而已,此額外向量%
的變化在t時space座標的%
向量表達式為$\Omega _{b}(t+dt)-\Omega
_{b}(t)=\Omega _{b}(t)+d\Omega _{b}(dt)-\Omega _{b}(t)=d\Omega
_{b}(dt)=d\Omega _{s}(dt)$,也是一個轉%
動向量,所以space空間中%
總共的轉動可以考慮%
成兩步,第一步轉$\Omega
_{s}(t)$,第二步轉$d\Omega _{s}(dt)$,%
寫成轉動矩陣%
\begin{equation}
\exp (\Omega _{b}(t))\times \exp (d\Omega _{b}(dt))=\exp (\Omega
_{b}(t)+d\Omega _{b}(dt))=\exp (\Omega _{b}(t+dt))
\end{equation}%
這代表我們只要考慮%
陀螺從t到t+dt的時候是轉%
了$\Omega _{b}(t+dt)$而不只是$\Omega _{b}(t)$%
,因此考慮$\Omega _{b}(t+dt)$我們%
就更準確的近似了這%
個轉動,以下的Python程%
式模擬會證明,考慮%
了$\Omega _{b}(t+dt)$給出的結果幾%
乎是完美的。

\begin{remark}
這裡要注意torque $\left( \Gamma _{s}\right) $%
雖然是力矩在space座標上%
的觀察值,但因為我%
們推導eq.\ref{newton2}時是利用%
上該方程在t到t+dt時,假%
設 s, b在 t時重合,這代%
表了我們考慮的space frame是%
必須與body frame在t時重和。%
這也代表了當考慮下%
一個$t^{\prime }$到$t^{\prime }+dt$時候,%
方程考慮的space frame會必須%
是在$t^{\prime }$時的body frame,也%
就是space frame必須持續的改%
變,這代表,若要考%
慮任意t到t+dt時刻,那再%
算$\vec{\omega}(t_{i+1})$時候torque應該要%
用$\left( \Gamma \right) _{b}$而不是用$\left(
\Gamma \right) _{s}$,因為我們的假%
設導致space frame必須持續的%
改變來符合eq.\ref{newton2}的假%
設。不過事實上這樣%
的考慮正是我們想要%
的結果,也就是eq.\ref{newton2}%
完全可以寫成在body frame的%
分量%
\begin{equation}
\left( \vec{\Gamma}\right) _{b}=\left( \frac{dI\vec{\omega}}{dt}\right) _{b}+%
\vec{\omega}\times \left( I\vec{\omega}\right) _{b}
\end{equation}%
,正是這樣才使的該%
方程容易求解。另外%
注意$\vec{L}$並不在body z軸的%
方向,這點可以從最%
後的模擬動畫看出來%
。
\end{remark}

\begin{remark}
要陀螺具有Precession and Nutation的%
動作,L/$\Delta L$必須要大,%
如果L小於$\Delta L$,則只會%
有陀螺質量受重力影%
響往下倒下的運動(不%
過這對檢查程式有沒%
有錯誤很有幫助!),理%
想上L至少要大於$\Delta L$,%
最好L大大於$\Delta L$。化成%
數值上的比較:這代%
表%
\begin{equation}
L\gg \Delta L\Rightarrow I\cdot 2\pi f\gg \vec{\Gamma}\Delta t\Rightarrow
I\cdot 2\pi f\gg \vec{r}\times \vec{F}\cdot 1/f\Rightarrow f\gg \sqrt{\frac{%
arm\cdot Mg\cdot \sin (\theta )}{2\pi I\cdot G}}
\end{equation}%
where $\theta $ is gyro's tilt angle and G is moment of inertial geometry
factor. 考慮$\Delta t$的量級大約%
是陀螺轉幾圈的時間%
(characteristic time),量級上約是$\sim
1/f$,若假設arm是10 cm, M = 1kg, g=10 m/s$^{2}$%
, I = 0.5M(0.05)$^{2}$,則f最少要10 Hertz以%
上。因此我們將以這%
些參數比較f = 1, 10, 50 Hertz所給%
出的陀螺運動。
\end{remark}

\begin{remark}
力矩給出的角速度是%
遵守右手定則(counterclockwise),%
所以rotation formula必須使用其%
active counterclockwise sense才能描述座標%
轉動,要小心,因大%
部分書上(如Goldstein)給的公%
式都是active clockwise(follow左手定%
則)(舉例如書上的Caley Klein
parameter rotation matrix),因此差一個%
負號。\bigskip
\end{remark}

以下將上述方法寫成%
python程式,並且畫圖模%
擬其xyz軸運動。

\underline{%
%TCIMACRO{%
%\hyperref{Python code can be found here.}{Python code can be found here.}{}{http://www.scribd.com/doc/214633604/Symmetric-top-numerical-simulation}}%
%BeginExpansion
\msihyperref{Python code can be found here.}{Python code can be found here.}{}{http://www.scribd.com/doc/214633604/Symmetric-top-numerical-simulation}%
%EndExpansion
}

\underline{%
%TCIMACRO{%
%\hyperref{3-D animation is here.}{3-D animation is here.}{}{http://tinypic.com/r/10cw9yf/8}}%
%BeginExpansion
\msihyperref{3-D animation is here.}{3-D animation is here.}{}{http://tinypic.com/r/10cw9yf/8}%
%EndExpansion
}

This document is prepared with Scientific Workplace 5.0 and typeset with Tex
Live 2013 (Xelatex). Date 4/2/2014

\FRAME{fhF}{5.5097in}{3.7135in}{0pt}{}{}{figure_xy.eps}{\special{language
"Scientific Word";type "GRAPHIC";maintain-aspect-ratio TRUE;display
"USEDEF";valid_file "F";width 5.5097in;height 3.7135in;depth
0pt;original-width 9.135in;original-height 6.1436in;cropleft "0";croptop
"1";cropright "1";cropbottom "0";filename
'coinfig/figure_xy.eps';file-properties "XNPEU";}}

\bigskip

\FRAME{fhF}{5.5097in}{3.7135in}{0pt}{}{}{figure_x.eps}{\special{language
"Scientific Word";type "GRAPHIC";maintain-aspect-ratio TRUE;display
"USEDEF";valid_file "F";width 5.5097in;height 3.7135in;depth
0pt;original-width 9.135in;original-height 6.1436in;cropleft "0";croptop
"1";cropright "1";cropbottom "0";filename
'coinfig/figure_x.eps';file-properties "XNPEU";}}

\FRAME{fhF}{5.5097in}{3.7135in}{0pt}{}{}{figure_y.eps}{\special{language
"Scientific Word";type "GRAPHIC";maintain-aspect-ratio TRUE;display
"USEDEF";valid_file "F";width 5.5097in;height 3.7135in;depth
0pt;original-width 9.135in;original-height 6.1436in;cropleft "0";croptop
"1";cropright "1";cropbottom "0";filename
'coinfig/figure_y.eps';file-properties "XNPEU";}}

\FRAME{fthF}{5.5097in}{3.7135in}{0pt}{}{\Qlb{figure_uniform}}{%
figure_uniform.eps}{\special{language "Scientific Word";type
"GRAPHIC";maintain-aspect-ratio TRUE;display "USEDEF";valid_file "F";width
5.5097in;height 3.7135in;depth 0pt;original-width 9.135in;original-height
6.1436in;cropleft "0";croptop "1";cropright "1";cropbottom "0";filename
'coinfig/figure_uniform.eps';file-properties "XNPEU";}}

陀螺等周速運動(Figure \ref%
{figure_uniform})的初始值條件如%
何計算呢?等周速的條%
件在Goldstein第二版5-77式給出%
\begin{equation}
Mgl=\dot{\phi}\left( I_{3}\omega _{3}-I_{1}\dot{\phi}\cos \theta _{0}\right)
\end{equation}%
,不過此式是由尤拉%
角(euler angles)給出,但我們%
需要的是anguler velocity along body的初%
始值,因此我們必須%
轉換尤拉角到anguler velocity along body%
,方法如下。上式中$%
\omega _{3}$即為我們的$\left( \omega
_{z}\right) _{b}$,這裡是20 Hz,$\theta _{0}$%
即為我們之前的orien向量%
所定,此模擬中是取45%
度角,由上式可求出%
兩組$\dot{\phi}(t_{0})$。另外尤拉%
角跟anugler velocity along body的關係式%
在Goldstein 4-125式給出%
\begin{eqnarray}
(\omega _{x})_{b} &=&\dot{\phi}\sin \theta \sin \psi +\dot{\theta}\cos \psi
\\
(\omega _{y})_{b} &=&\dot{\phi}\sin \theta \cos \psi -\dot{\theta}\sin \psi
\\
(\omega _{z})_{b} &=&\dot{\phi}\cos \theta +\dot{\psi}
\end{eqnarray}%
知道$\dot{\phi}(t_{0})$、$\theta _{0}$、$(\omega
_{z})_{b}$,我們由第三條求%
出$\dot{\psi}(t_{0})$,再把$\dot{\psi}(t_{0})$%
帶到第一二條後就可%
得到$(\omega _{x}(t_{0}),\omega _{y}(t_{0}))_{b}$,這%
樣我們就得到anguler velocity along body%
的初始值。因為$\dot{\phi}(t_{0})$%
有兩組,因此解出的%
貼體角速度也會有兩%
組,兩組的物理意義%
分別如下,一種情況%
是fast top,這個狀況相當%
於重力的影響遠小於%
總角動量$L$,因此這個%
特別的例子基本上相%
當於忽略重力,而陀%
螺基本上會像一個free top%
一樣進行precession。另一種%
狀況是slow top,也就是上%
面模擬結果中第四種%
的狀況,這裡提供的%
python程式所有情況都可%
以模擬。另外一個特%
殊的情況是在fast top的情%
形下,如果初始值$\theta
_{0}=0$,也就是陀螺z軸的%
起始狀態是垂直於水%
平面的,這樣的話陀%
螺幾乎會像靜止不動%
一樣,我們也叫這情%
況做sleeping top。

\end{document}
