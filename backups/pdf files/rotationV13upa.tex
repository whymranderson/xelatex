\documentclass[12pt,twoside]{article}
%\documentclass[12pt,a4paper]{article}
\usepackage{amsmath}
\usepackage{fontspec}
\usepackage{xeCJK}
\setmainfont{Times New Roman}
\setsansfont{Verdana}
\setmonofont{Courier New}
\setCJKmainfont[AutoFakeBold=1.5]{微軟正黑體}
\setCJKfamilyfont{kai}{標楷體}
\newcommand*{\kai}{\CJKfamily{kai}}
\usepackage[inner=1in,outer=0.6in,top=0.7in,bottom=1in]{geometry}
\usepackage{unicode-math}
\usepackage{graphicx}
\usepackage[usenames,dvipsnames]{xcolor}
\usepackage[hidelinks]{hyperref}
\usepackage{pgf}
\usepackage{pstricks,pst-node,pst-3dplot}
\usepackage{minted}
\usepackage{pdfpages}
\usepackage{mdframed}
\usepackage{fancyhdr}
\pagestyle{fancy}
\fancyhf{}
\fancyfoot[LE,RO]{\thepart-\thepage}
\renewcommand{\headrulewidth}{0pt}

\setcounter{MaxMatrixCols}{10}
%TCIDATA{OutputFilter=LATEX.DLL}
%TCIDATA{Version=5.00.0.2606}
%TCIDATA{<META NAME="SaveForMode" CONTENT="1">}
%TCIDATA{BibliographyScheme=Manual}
%TCIDATA{Created=Monday, January 13, 2014 11:43:31}
%TCIDATA{LastRevised=Sunday, March 22, 2015 12:09:34}
%TCIDATA{<META NAME="GraphicsSave" CONTENT="32">}
%TCIDATA{<META NAME="DocumentShell" CONTENT="International\Traditional Chinese Article">}
%TCIDATA{CSTFile=Traditional Chinese.cst}

\newtheorem{theorem}{Theorem}
\newtheorem{acknowledgement}[theorem]{Acknowledgement}
\newtheorem{algorithm}[theorem]{Algorithm}
\newtheorem{axiom}[theorem]{Axiom}
\newtheorem{case}[theorem]{Case}
\newtheorem{claim}[theorem]{Claim}
\newtheorem{conclusion}[theorem]{Conclusion}
\newtheorem{condition}[theorem]{Condition}
\newtheorem{conjecture}[theorem]{Conjecture}
\newtheorem{corollary}[theorem]{Corollary}
\newtheorem{criterion}[theorem]{Criterion}
\newtheorem{definition}[theorem]{Definition}
\newtheorem{example}[theorem]{Example}
\newtheorem{exercise}[theorem]{Exercise}
\newtheorem{lemma}[theorem]{Lemma}
\newtheorem{notation}[theorem]{Notation}
\newtheorem{problem}[theorem]{Problem}
\newtheorem{proposition}[theorem]{Proposition}
\newtheorem{remark}[theorem]{Remark}
\newtheorem{solution}[theorem]{Solution}
\newtheorem{summary}[theorem]{Summary}
\newenvironment{proof}[1][Proof]{\noindent\textbf{#1.} }{\ \rule{0.5em}{0.5em}}
% Macros for Scientific Word 4.0 documents saved with the LaTeX filter.
% Copyright (C) 2002 Mackichan Software, Inc.

\typeout{TCILATEX Macros for Scientific Word 5.0 <13 Feb 2003>.}
\typeout{NOTICE:  This macro file is NOT proprietary and may be 
freely copied and distributed.}
%
\makeatletter

%%%%%%%%%%%%%%%%%%%%%
% pdfTeX related.
\ifx\pdfoutput\relax\let\pdfoutput=\undefined\fi
\newcount\msipdfoutput
\ifx\pdfoutput\undefined
\else
 \ifcase\pdfoutput
 \else 
    \msipdfoutput=1
    \ifx\paperwidth\undefined
    \else
      \ifdim\paperheight=0pt\relax
      \else
        \pdfpageheight\paperheight
      \fi
      \ifdim\paperwidth=0pt\relax
      \else
        \pdfpagewidth\paperwidth
      \fi
    \fi
  \fi  
\fi

%%%%%%%%%%%%%%%%%%%%%
% FMTeXButton
% This is used for putting TeXButtons in the 
% frontmatter of a document. Add a line like
% \QTagDef{FMTeXButton}{101}{} to the filter 
% section of the cst being used. Also add a
% new section containing:
%     [f_101]
%     ALIAS=FMTexButton
%     TAG_TYPE=FIELD
%     TAG_LEADIN=TeX Button:
%
% It also works to put \defs in the preamble after 
% the \input tcilatex
\def\FMTeXButton#1{#1}
%
%%%%%%%%%%%%%%%%%%%%%%
% macros for time
\newcount\@hour\newcount\@minute\chardef\@x10\chardef\@xv60
\def\tcitime{
\def\@time{%
  \@minute\time\@hour\@minute\divide\@hour\@xv
  \ifnum\@hour<\@x 0\fi\the\@hour:%
  \multiply\@hour\@xv\advance\@minute-\@hour
  \ifnum\@minute<\@x 0\fi\the\@minute
  }}%

%%%%%%%%%%%%%%%%%%%%%%
% macro for hyperref and msihyperref
%\@ifundefined{hyperref}{\def\hyperref#1#2#3#4{#2\ref{#4}#3}}{}

\def\x@hyperref#1#2#3{%
   % Turn off various catcodes before reading parameter 4
   \catcode`\~ = 12
   \catcode`\$ = 12
   \catcode`\_ = 12
   \catcode`\# = 12
   \catcode`\& = 12
   \y@hyperref{#1}{#2}{#3}%
}

\def\y@hyperref#1#2#3#4{%
   #2\ref{#4}#3
   \catcode`\~ = 13
   \catcode`\$ = 3
   \catcode`\_ = 8
   \catcode`\# = 6
   \catcode`\& = 4
}

\@ifundefined{hyperref}{\let\hyperref\x@hyperref}{}
\@ifundefined{msihyperref}{\let\msihyperref\x@hyperref}{}




% macro for external program call
\@ifundefined{qExtProgCall}{\def\qExtProgCall#1#2#3#4#5#6{\relax}}{}
%%%%%%%%%%%%%%%%%%%%%%
%
% macros for graphics
%
\def\FILENAME#1{#1}%
%
\def\QCTOpt[#1]#2{%
  \def\QCTOptB{#1}
  \def\QCTOptA{#2}
}
\def\QCTNOpt#1{%
  \def\QCTOptA{#1}
  \let\QCTOptB\empty
}
\def\Qct{%
  \@ifnextchar[{%
    \QCTOpt}{\QCTNOpt}
}
\def\QCBOpt[#1]#2{%
  \def\QCBOptB{#1}%
  \def\QCBOptA{#2}%
}
\def\QCBNOpt#1{%
  \def\QCBOptA{#1}%
  \let\QCBOptB\empty
}
\def\Qcb{%
  \@ifnextchar[{%
    \QCBOpt}{\QCBNOpt}%
}
\def\PrepCapArgs{%
  \ifx\QCBOptA\empty
    \ifx\QCTOptA\empty
      {}%
    \else
      \ifx\QCTOptB\empty
        {\QCTOptA}%
      \else
        [\QCTOptB]{\QCTOptA}%
      \fi
    \fi
  \else
    \ifx\QCBOptA\empty
      {}%
    \else
      \ifx\QCBOptB\empty
        {\QCBOptA}%
      \else
        [\QCBOptB]{\QCBOptA}%
      \fi
    \fi
  \fi
}
\newcount\GRAPHICSTYPE
%\GRAPHICSTYPE 0 is for TurboTeX
%\GRAPHICSTYPE 1 is for DVIWindo (PostScript)
%%%(removed)%\GRAPHICSTYPE 2 is for psfig (PostScript)
\GRAPHICSTYPE=\z@
\def\GRAPHICSPS#1{%
 \ifcase\GRAPHICSTYPE%\GRAPHICSTYPE=0
   \special{ps: #1}%
 \or%\GRAPHICSTYPE=1
   \special{language "PS", include "#1"}%
%%%\or%\GRAPHICSTYPE=2
%%%  #1%
 \fi
}%
%
\def\GRAPHICSHP#1{\special{include #1}}%
%
% \graffile{ body }                                  %#1
%          { contentswidth (scalar)  }               %#2
%          { contentsheight (scalar) }               %#3
%          { vertical shift when in-line (scalar) }  %#4

\def\graffile#1#2#3#4{%
%%% \ifnum\GRAPHICSTYPE=\tw@
%%%  %Following if using psfig
%%%  \@ifundefined{psfig}{\input psfig.tex}{}%
%%%  \psfig{file=#1, height=#3, width=#2}%
%%% \else
  %Following for all others
  % JCS - added BOXTHEFRAME, see below
    \bgroup
	   \@inlabelfalse
       \leavevmode
       \@ifundefined{bbl@deactivate}{\def~{\string~}}{\activesoff}%
        \raise -#4 \BOXTHEFRAME{%
           \hbox to #2{\raise #3\hbox to #2{\null #1\hfil}}}%
    \egroup
}%
%
% A box for drafts
\def\draftbox#1#2#3#4{%
 \leavevmode\raise -#4 \hbox{%
  \frame{\rlap{\protect\tiny #1}\hbox to #2%
   {\vrule height#3 width\z@ depth\z@\hfil}%
  }%
 }%
}%
%
\newcount\@msidraft
\@msidraft=\z@
\let\nographics=\@msidraft
\newif\ifwasdraft
\wasdraftfalse

%  \GRAPHIC{ body }                                  %#1
%          { draft name }                            %#2
%          { contentswidth (scalar)  }               %#3
%          { contentsheight (scalar) }               %#4
%          { vertical shift when in-line (scalar) }  %#5
\def\GRAPHIC#1#2#3#4#5{%
   \ifnum\@msidraft=\@ne\draftbox{#2}{#3}{#4}{#5}%
   \else\graffile{#1}{#3}{#4}{#5}%
   \fi
}
%
\def\addtoLaTeXparams#1{%
    \edef\LaTeXparams{\LaTeXparams #1}}%
%
% JCS -  added a switch BoxFrame that can 
% be set by including X in the frame params.
% If set a box is drawn around the frame.

\newif\ifBoxFrame \BoxFramefalse
\newif\ifOverFrame \OverFramefalse
\newif\ifUnderFrame \UnderFramefalse

\def\BOXTHEFRAME#1{%
   \hbox{%
      \ifBoxFrame
         \frame{#1}%
      \else
         {#1}%
      \fi
   }%
}


\def\doFRAMEparams#1{\BoxFramefalse\OverFramefalse\UnderFramefalse\readFRAMEparams#1\end}%
\def\readFRAMEparams#1{%
 \ifx#1\end%
  \let\next=\relax
  \else
  \ifx#1i\dispkind=\z@\fi
  \ifx#1d\dispkind=\@ne\fi
  \ifx#1f\dispkind=\tw@\fi
  \ifx#1t\addtoLaTeXparams{t}\fi
  \ifx#1b\addtoLaTeXparams{b}\fi
  \ifx#1p\addtoLaTeXparams{p}\fi
  \ifx#1h\addtoLaTeXparams{h}\fi
  \ifx#1X\BoxFrametrue\fi
  \ifx#1O\OverFrametrue\fi
  \ifx#1U\UnderFrametrue\fi
  \ifx#1w
    \ifnum\@msidraft=1\wasdrafttrue\else\wasdraftfalse\fi
    \@msidraft=\@ne
  \fi
  \let\next=\readFRAMEparams
  \fi
 \next
 }%
%
%Macro for In-line graphics object
%   \IFRAME{ contentswidth (scalar)  }               %#1
%          { contentsheight (scalar) }               %#2
%          { vertical shift when in-line (scalar) }  %#3
%          { draft name }                            %#4
%          { body }                                  %#5
%          { caption}                                %#6


\def\IFRAME#1#2#3#4#5#6{%
      \bgroup
      \let\QCTOptA\empty
      \let\QCTOptB\empty
      \let\QCBOptA\empty
      \let\QCBOptB\empty
      #6%
      \parindent=0pt
      \leftskip=0pt
      \rightskip=0pt
      \setbox0=\hbox{\QCBOptA}%
      \@tempdima=#1\relax
      \ifOverFrame
          % Do this later
          \typeout{This is not implemented yet}%
          \show\HELP
      \else
         \ifdim\wd0>\@tempdima
            \advance\@tempdima by \@tempdima
            \ifdim\wd0 >\@tempdima
               \setbox1 =\vbox{%
                  \unskip\hbox to \@tempdima{\hfill\GRAPHIC{#5}{#4}{#1}{#2}{#3}\hfill}%
                  \unskip\hbox to \@tempdima{\parbox[b]{\@tempdima}{\QCBOptA}}%
               }%
               \wd1=\@tempdima
            \else
               \textwidth=\wd0
               \setbox1 =\vbox{%
                 \noindent\hbox to \wd0{\hfill\GRAPHIC{#5}{#4}{#1}{#2}{#3}\hfill}\\%
                 \noindent\hbox{\QCBOptA}%
               }%
               \wd1=\wd0
            \fi
         \else
            \ifdim\wd0>0pt
              \hsize=\@tempdima
              \setbox1=\vbox{%
                \unskip\GRAPHIC{#5}{#4}{#1}{#2}{0pt}%
                \break
                \unskip\hbox to \@tempdima{\hfill \QCBOptA\hfill}%
              }%
              \wd1=\@tempdima
           \else
              \hsize=\@tempdima
              \setbox1=\vbox{%
                \unskip\GRAPHIC{#5}{#4}{#1}{#2}{0pt}%
              }%
              \wd1=\@tempdima
           \fi
         \fi
         \@tempdimb=\ht1
         %\advance\@tempdimb by \dp1
         \advance\@tempdimb by -#2
         \advance\@tempdimb by #3
         \leavevmode
         \raise -\@tempdimb \hbox{\box1}%
      \fi
      \egroup%
}%
%
%Macro for Display graphics object
%   \DFRAME{ contentswidth (scalar)  }               %#1
%          { contentsheight (scalar) }               %#2
%          { draft label }                           %#3
%          { name }                                  %#4
%          { caption}                                %#5
\def\DFRAME#1#2#3#4#5{%
  \vspace\topsep
  \hfil\break
  \bgroup
     \leftskip\@flushglue
	 \rightskip\@flushglue
	 \parindent\z@
	 \parfillskip\z@skip
     \let\QCTOptA\empty
     \let\QCTOptB\empty
     \let\QCBOptA\empty
     \let\QCBOptB\empty
	 \vbox\bgroup
        \ifOverFrame 
           #5\QCTOptA\par
        \fi
        \GRAPHIC{#4}{#3}{#1}{#2}{\z@}%
        \ifUnderFrame 
           \break#5\QCBOptA
        \fi
	 \egroup
  \egroup
  \vspace\topsep
  \break
}%
%
%Macro for Floating graphic object
%   \FFRAME{ framedata f|i tbph x F|T }              %#1
%          { contentswidth (scalar)  }               %#2
%          { contentsheight (scalar) }               %#3
%          { caption }                               %#4
%          { label }                                 %#5
%          { draft name }                            %#6
%          { body }                                  %#7
\def\FFRAME#1#2#3#4#5#6#7{%
 %If float.sty loaded and float option is 'h', change to 'H'  (gp) 1998/09/05
  \@ifundefined{floatstyle}
    {%floatstyle undefined (and float.sty not present), no change
     \begin{figure}[#1]%
    }
    {%floatstyle DEFINED
	 \ifx#1h%Only the h parameter, change to H
      \begin{figure}[H]%
	 \else
      \begin{figure}[#1]%
	 \fi
	}
  \let\QCTOptA\empty
  \let\QCTOptB\empty
  \let\QCBOptA\empty
  \let\QCBOptB\empty
  \ifOverFrame
    #4
    \ifx\QCTOptA\empty
    \else
      \ifx\QCTOptB\empty
        \caption{\QCTOptA}%
      \else
        \caption[\QCTOptB]{\QCTOptA}%
      \fi
    \fi
    \ifUnderFrame\else
      \label{#5}%
    \fi
  \else
    \UnderFrametrue%
  \fi
  \begin{center}\GRAPHIC{#7}{#6}{#2}{#3}{\z@}\end{center}%
  \ifUnderFrame
    #4
    \ifx\QCBOptA\empty
      \caption{}%
    \else
      \ifx\QCBOptB\empty
        \caption{\QCBOptA}%
      \else
        \caption[\QCBOptB]{\QCBOptA}%
      \fi
    \fi
    \label{#5}%
  \fi
  \end{figure}%
 }%
%
%
%    \FRAME{ framedata f|i tbph x F|T }              %#1
%          { contentswidth (scalar)  }               %#2
%          { contentsheight (scalar) }               %#3
%          { vertical shift when in-line (scalar) }  %#4
%          { caption }                               %#5
%          { label }                                 %#6
%          { name }                                  %#7
%          { body }                                  %#8
%
%    framedata is a string which can contain the following
%    characters: idftbphxFT
%    Their meaning is as follows:
%             i, d or f : in-line, display, or floating
%             t,b,p,h   : LaTeX floating placement options
%             x         : fit contents box to contents
%             F or T    : Figure or Table. 
%                         Later this can expand
%                         to a more general float class.
%
%
\newcount\dispkind%

\def\makeactives{
  \catcode`\"=\active
  \catcode`\;=\active
  \catcode`\:=\active
  \catcode`\'=\active
  \catcode`\~=\active
}
\bgroup
   \makeactives
   \gdef\activesoff{%
      \def"{\string"}%
      \def;{\string;}%
      \def:{\string:}%
      \def'{\string'}%
      \def~{\string~}%
      %\bbl@deactivate{"}%
      %\bbl@deactivate{;}%
      %\bbl@deactivate{:}%
      %\bbl@deactivate{'}%
    }
\egroup

\def\FRAME#1#2#3#4#5#6#7#8{%
 \bgroup
 \ifnum\@msidraft=\@ne
   \wasdrafttrue
 \else
   \wasdraftfalse%
 \fi
 \def\LaTeXparams{}%
 \dispkind=\z@
 \def\LaTeXparams{}%
 \doFRAMEparams{#1}%
 \ifnum\dispkind=\z@\IFRAME{#2}{#3}{#4}{#7}{#8}{#5}\else
  \ifnum\dispkind=\@ne\DFRAME{#2}{#3}{#7}{#8}{#5}\else
   \ifnum\dispkind=\tw@
    \edef\@tempa{\noexpand\FFRAME{\LaTeXparams}}%
    \@tempa{#2}{#3}{#5}{#6}{#7}{#8}%
    \fi
   \fi
  \fi
  \ifwasdraft\@msidraft=1\else\@msidraft=0\fi{}%
  \egroup
 }%
%
% This macro added to let SW gobble a parameter that
% should not be passed on and expanded. 

\def\TEXUX#1{"texux"}

%
% Macros for text attributes:
%
\def\BF#1{{\bf {#1}}}%
\def\NEG#1{\leavevmode\hbox{\rlap{\thinspace/}{$#1$}}}%
%
%%%%%%%%%%%%%%%%%%%%%%%%%%%%%%%%%%%%%%%%%%%%%%%%%%%%%%%%%%%%%%%%%%%%%%%%
%
%
% macros for user - defined functions
\def\limfunc#1{\mathop{\rm #1}}%
\def\func#1{\mathop{\rm #1}\nolimits}%
% macro for unit names
\def\unit#1{\mathord{\thinspace\rm #1}}%

%
% miscellaneous 
\long\def\QQQ#1#2{%
     \long\expandafter\def\csname#1\endcsname{#2}}%
\@ifundefined{QTP}{\def\QTP#1{}}{}
\@ifundefined{QEXCLUDE}{\def\QEXCLUDE#1{}}{}
\@ifundefined{Qlb}{\def\Qlb#1{#1}}{}
\@ifundefined{Qlt}{\def\Qlt#1{#1}}{}
\def\QWE{}%
\long\def\QQA#1#2{}%
\def\QTR#1#2{{\csname#1\endcsname {#2}}}%
\long\def\TeXButton#1#2{#2}%
\long\def\QSubDoc#1#2{#2}%
\def\EXPAND#1[#2]#3{}%
\def\NOEXPAND#1[#2]#3{}%
\def\PROTECTED{}%
\def\LaTeXparent#1{}%
\def\ChildStyles#1{}%
\def\ChildDefaults#1{}%
\def\QTagDef#1#2#3{}%

% Constructs added with Scientific Notebook
\@ifundefined{correctchoice}{\def\correctchoice{\relax}}{}
\@ifundefined{HTML}{\def\HTML#1{\relax}}{}
\@ifundefined{TCIIcon}{\def\TCIIcon#1#2#3#4{\relax}}{}
\if@compatibility
  \typeout{Not defining UNICODE  U or CustomNote commands for LaTeX 2.09.}
\else
  \providecommand{\UNICODE}[2][]{\protect\rule{.1in}{.1in}}
  \providecommand{\U}[1]{\protect\rule{.1in}{.1in}}
  \providecommand{\CustomNote}[3][]{\marginpar{#3}}
\fi

\@ifundefined{lambdabar}{
      \def\lambdabar{\errmessage{You have used the lambdabar symbol. 
                      This is available for typesetting only in RevTeX styles.}}
   }{}

%
% Macros for style editor docs
\@ifundefined{StyleEditBeginDoc}{\def\StyleEditBeginDoc{\relax}}{}
%
% Macros for footnotes
\def\QQfnmark#1{\footnotemark}
\def\QQfntext#1#2{\addtocounter{footnote}{#1}\footnotetext{#2}}
%
% Macros for indexing.
%
\@ifundefined{TCIMAKEINDEX}{}{\makeindex}%
%
% Attempts to avoid problems with other styles
\@ifundefined{abstract}{%
 \def\abstract{%
  \if@twocolumn
   \section*{Abstract (Not appropriate in this style!)}%
   \else \small 
   \begin{center}{\bf Abstract\vspace{-.5em}\vspace{\z@}}\end{center}%
   \quotation 
   \fi
  }%
 }{%
 }%
\@ifundefined{endabstract}{\def\endabstract
  {\if@twocolumn\else\endquotation\fi}}{}%
\@ifundefined{maketitle}{\def\maketitle#1{}}{}%
\@ifundefined{affiliation}{\def\affiliation#1{}}{}%
\@ifundefined{proof}{\def\proof{\noindent{\bfseries Proof. }}}{}%
\@ifundefined{endproof}{\def\endproof{\mbox{\ \rule{.1in}{.1in}}}}{}%
\@ifundefined{newfield}{\def\newfield#1#2{}}{}%
\@ifundefined{chapter}{\def\chapter#1{\par(Chapter head:)#1\par }%
 \newcount\c@chapter}{}%
\@ifundefined{part}{\def\part#1{\par(Part head:)#1\par }}{}%
\@ifundefined{section}{\def\section#1{\par(Section head:)#1\par }}{}%
\@ifundefined{subsection}{\def\subsection#1%
 {\par(Subsection head:)#1\par }}{}%
\@ifundefined{subsubsection}{\def\subsubsection#1%
 {\par(Subsubsection head:)#1\par }}{}%
\@ifundefined{paragraph}{\def\paragraph#1%
 {\par(Subsubsubsection head:)#1\par }}{}%
\@ifundefined{subparagraph}{\def\subparagraph#1%
 {\par(Subsubsubsubsection head:)#1\par }}{}%
%%%%%%%%%%%%%%%%%%%%%%%%%%%%%%%%%%%%%%%%%%%%%%%%%%%%%%%%%%%%%%%%%%%%%%%%
% These symbols are not recognized by LaTeX
\@ifundefined{therefore}{\def\therefore{}}{}%
\@ifundefined{backepsilon}{\def\backepsilon{}}{}%
\@ifundefined{yen}{\def\yen{\hbox{\rm\rlap=Y}}}{}%
\@ifundefined{registered}{%
   \def\registered{\relax\ifmmode{}\r@gistered
                    \else$\m@th\r@gistered$\fi}%
 \def\r@gistered{^{\ooalign
  {\hfil\raise.07ex\hbox{$\scriptstyle\rm\text{R}$}\hfil\crcr
  \mathhexbox20D}}}}{}%
\@ifundefined{Eth}{\def\Eth{}}{}%
\@ifundefined{eth}{\def\eth{}}{}%
\@ifundefined{Thorn}{\def\Thorn{}}{}%
\@ifundefined{thorn}{\def\thorn{}}{}%
% A macro to allow any symbol that requires math to appear in text
\def\TEXTsymbol#1{\mbox{$#1$}}%
\@ifundefined{degree}{\def\degree{{}^{\circ}}}{}%
%
% macros for T3TeX files
\newdimen\theight
\@ifundefined{Column}{\def\Column{%
 \vadjust{\setbox\z@=\hbox{\scriptsize\quad\quad tcol}%
  \theight=\ht\z@\advance\theight by \dp\z@\advance\theight by \lineskip
  \kern -\theight \vbox to \theight{%
   \rightline{\rlap{\box\z@}}%
   \vss
   }%
  }%
 }}{}%
%
\@ifundefined{qed}{\def\qed{%
 \ifhmode\unskip\nobreak\fi\ifmmode\ifinner\else\hskip5\p@\fi\fi
 \hbox{\hskip5\p@\vrule width4\p@ height6\p@ depth1.5\p@\hskip\p@}%
 }}{}%
%
\@ifundefined{cents}{\def\cents{\hbox{\rm\rlap c/}}}{}%
\@ifundefined{tciLaplace}{\def\tciLaplace{\ensuremath{\mathcal{L}}}}{}%
\@ifundefined{tciFourier}{\def\tciFourier{\ensuremath{\mathcal{F}}}}{}%
\@ifundefined{textcurrency}{\def\textcurrency{\hbox{\rm\rlap xo}}}{}%
\@ifundefined{texteuro}{\def\texteuro{\hbox{\rm\rlap C=}}}{}%
\@ifundefined{euro}{\def\euro{\hbox{\rm\rlap C=}}}{}%
\@ifundefined{textfranc}{\def\textfranc{\hbox{\rm\rlap-F}}}{}%
\@ifundefined{textlira}{\def\textlira{\hbox{\rm\rlap L=}}}{}%
\@ifundefined{textpeseta}{\def\textpeseta{\hbox{\rm P\negthinspace s}}}{}%
%
\@ifundefined{miss}{\def\miss{\hbox{\vrule height2\p@ width 2\p@ depth\z@}}}{}%
%
\@ifundefined{vvert}{\def\vvert{\Vert}}{}%  %always translated to \left| or \right|
%
\@ifundefined{tcol}{\def\tcol#1{{\baselineskip=6\p@ \vcenter{#1}} \Column}}{}%
%
\@ifundefined{dB}{\def\dB{\hbox{{}}}}{}%        %dummy entry in column 
\@ifundefined{mB}{\def\mB#1{\hbox{$#1$}}}{}%   %column entry
\@ifundefined{nB}{\def\nB#1{\hbox{#1}}}{}%     %column entry (not math)
%
\@ifundefined{note}{\def\note{$^{\dag}}}{}%
%
\def\newfmtname{LaTeX2e}
% No longer load latexsym.  This is now handled by SWP, which uses amsfonts if necessary
%
\ifx\fmtname\newfmtname
  \DeclareOldFontCommand{\rm}{\normalfont\rmfamily}{\mathrm}
  \DeclareOldFontCommand{\sf}{\normalfont\sffamily}{\mathsf}
  \DeclareOldFontCommand{\tt}{\normalfont\ttfamily}{\mathtt}
  \DeclareOldFontCommand{\bf}{\normalfont\bfseries}{\mathbf}
  \DeclareOldFontCommand{\it}{\normalfont\itshape}{\mathit}
  \DeclareOldFontCommand{\sl}{\normalfont\slshape}{\@nomath\sl}
  \DeclareOldFontCommand{\sc}{\normalfont\scshape}{\@nomath\sc}
\fi

%
% Greek bold macros
% Redefine all of the math symbols 
% which might be bolded	 - there are 
% probably others to add to this list

\def\alpha{{\Greekmath 010B}}%
\def\beta{{\Greekmath 010C}}%
\def\gamma{{\Greekmath 010D}}%
\def\delta{{\Greekmath 010E}}%
\def\epsilon{{\Greekmath 010F}}%
\def\zeta{{\Greekmath 0110}}%
\def\eta{{\Greekmath 0111}}%
\def\theta{{\Greekmath 0112}}%
\def\iota{{\Greekmath 0113}}%
\def\kappa{{\Greekmath 0114}}%
\def\lambda{{\Greekmath 0115}}%
\def\mu{{\Greekmath 0116}}%
\def\nu{{\Greekmath 0117}}%
\def\xi{{\Greekmath 0118}}%
\def\pi{{\Greekmath 0119}}%
\def\rho{{\Greekmath 011A}}%
\def\sigma{{\Greekmath 011B}}%
\def\tau{{\Greekmath 011C}}%
\def\upsilon{{\Greekmath 011D}}%
\def\phi{{\Greekmath 011E}}%
\def\chi{{\Greekmath 011F}}%
\def\psi{{\Greekmath 0120}}%
\def\omega{{\Greekmath 0121}}%
\def\varepsilon{{\Greekmath 0122}}%
\def\vartheta{{\Greekmath 0123}}%
\def\varpi{{\Greekmath 0124}}%
\def\varrho{{\Greekmath 0125}}%
\def\varsigma{{\Greekmath 0126}}%
\def\varphi{{\Greekmath 0127}}%

\def\nabla{{\Greekmath 0272}}
\def\FindBoldGroup{%
   {\setbox0=\hbox{$\mathbf{x\global\edef\theboldgroup{\the\mathgroup}}$}}%
}

\def\Greekmath#1#2#3#4{%
    \if@compatibility
        \ifnum\mathgroup=\symbold
           \mathchoice{\mbox{\boldmath$\displaystyle\mathchar"#1#2#3#4$}}%
                      {\mbox{\boldmath$\textstyle\mathchar"#1#2#3#4$}}%
                      {\mbox{\boldmath$\scriptstyle\mathchar"#1#2#3#4$}}%
                      {\mbox{\boldmath$\scriptscriptstyle\mathchar"#1#2#3#4$}}%
        \else
           \mathchar"#1#2#3#4% 
        \fi 
    \else 
        \FindBoldGroup
        \ifnum\mathgroup=\theboldgroup % For 2e
           \mathchoice{\mbox{\boldmath$\displaystyle\mathchar"#1#2#3#4$}}%
                      {\mbox{\boldmath$\textstyle\mathchar"#1#2#3#4$}}%
                      {\mbox{\boldmath$\scriptstyle\mathchar"#1#2#3#4$}}%
                      {\mbox{\boldmath$\scriptscriptstyle\mathchar"#1#2#3#4$}}%
        \else
           \mathchar"#1#2#3#4% 
        \fi     	    
	  \fi}

\newif\ifGreekBold  \GreekBoldfalse
\let\SAVEPBF=\pbf
\def\pbf{\GreekBoldtrue\SAVEPBF}%
%

\@ifundefined{theorem}{\newtheorem{theorem}{Theorem}}{}
\@ifundefined{lemma}{\newtheorem{lemma}[theorem]{Lemma}}{}
\@ifundefined{corollary}{\newtheorem{corollary}[theorem]{Corollary}}{}
\@ifundefined{conjecture}{\newtheorem{conjecture}[theorem]{Conjecture}}{}
\@ifundefined{proposition}{\newtheorem{proposition}[theorem]{Proposition}}{}
\@ifundefined{axiom}{\newtheorem{axiom}{Axiom}}{}
\@ifundefined{remark}{\newtheorem{remark}{Remark}}{}
\@ifundefined{example}{\newtheorem{example}{Example}}{}
\@ifundefined{exercise}{\newtheorem{exercise}{Exercise}}{}
\@ifundefined{definition}{\newtheorem{definition}{Definition}}{}


\@ifundefined{mathletters}{%
  %\def\theequation{\arabic{equation}}
  \newcounter{equationnumber}  
  \def\mathletters{%
     \addtocounter{equation}{1}
     \edef\@currentlabel{\theequation}%
     \setcounter{equationnumber}{\c@equation}
     \setcounter{equation}{0}%
     \edef\theequation{\@currentlabel\noexpand\alph{equation}}%
  }
  \def\endmathletters{%
     \setcounter{equation}{\value{equationnumber}}%
  }
}{}

%Logos
\@ifundefined{BibTeX}{%
    \def\BibTeX{{\rm B\kern-.05em{\sc i\kern-.025em b}\kern-.08em
                 T\kern-.1667em\lower.7ex\hbox{E}\kern-.125emX}}}{}%
\@ifundefined{AmS}%
    {\def\AmS{{\protect\usefont{OMS}{cmsy}{m}{n}%
                A\kern-.1667em\lower.5ex\hbox{M}\kern-.125emS}}}{}%
\@ifundefined{AmSTeX}{\def\AmSTeX{\protect\AmS-\protect\TeX\@}}{}%
%

% This macro is a fix to eqnarray
\def\@@eqncr{\let\@tempa\relax
    \ifcase\@eqcnt \def\@tempa{& & &}\or \def\@tempa{& &}%
      \else \def\@tempa{&}\fi
     \@tempa
     \if@eqnsw
        \iftag@
           \@taggnum
        \else
           \@eqnnum\stepcounter{equation}%
        \fi
     \fi
     \global\tag@false
     \global\@eqnswtrue
     \global\@eqcnt\z@\cr}


\def\TCItag{\@ifnextchar*{\@TCItagstar}{\@TCItag}}
\def\@TCItag#1{%
    \global\tag@true
    \global\def\@taggnum{(#1)}%
    \global\def\@currentlabel{#1}}
\def\@TCItagstar*#1{%
    \global\tag@true
    \global\def\@taggnum{#1}%
    \global\def\@currentlabel{#1}}
%
%%%%%%%%%%%%%%%%%%%%%%%%%%%%%%%%%%%%%%%%%%%%%%%%%%%%%%%%%%%%%%%%%%%%%
%
\def\QATOP#1#2{{#1 \atop #2}}%
\def\QTATOP#1#2{{\textstyle {#1 \atop #2}}}%
\def\QDATOP#1#2{{\displaystyle {#1 \atop #2}}}%
\def\QABOVE#1#2#3{{#2 \above#1 #3}}%
\def\QTABOVE#1#2#3{{\textstyle {#2 \above#1 #3}}}%
\def\QDABOVE#1#2#3{{\displaystyle {#2 \above#1 #3}}}%
\def\QOVERD#1#2#3#4{{#3 \overwithdelims#1#2 #4}}%
\def\QTOVERD#1#2#3#4{{\textstyle {#3 \overwithdelims#1#2 #4}}}%
\def\QDOVERD#1#2#3#4{{\displaystyle {#3 \overwithdelims#1#2 #4}}}%
\def\QATOPD#1#2#3#4{{#3 \atopwithdelims#1#2 #4}}%
\def\QTATOPD#1#2#3#4{{\textstyle {#3 \atopwithdelims#1#2 #4}}}%
\def\QDATOPD#1#2#3#4{{\displaystyle {#3 \atopwithdelims#1#2 #4}}}%
\def\QABOVED#1#2#3#4#5{{#4 \abovewithdelims#1#2#3 #5}}%
\def\QTABOVED#1#2#3#4#5{{\textstyle 
   {#4 \abovewithdelims#1#2#3 #5}}}%
\def\QDABOVED#1#2#3#4#5{{\displaystyle 
   {#4 \abovewithdelims#1#2#3 #5}}}%
%
% Macros for text size operators:
%
\def\tint{\mathop{\textstyle \int}}%
\def\tiint{\mathop{\textstyle \iint }}%
\def\tiiint{\mathop{\textstyle \iiint }}%
\def\tiiiint{\mathop{\textstyle \iiiint }}%
\def\tidotsint{\mathop{\textstyle \idotsint }}%
\def\toint{\mathop{\textstyle \oint}}%
\def\tsum{\mathop{\textstyle \sum }}%
\def\tprod{\mathop{\textstyle \prod }}%
\def\tbigcap{\mathop{\textstyle \bigcap }}%
\def\tbigwedge{\mathop{\textstyle \bigwedge }}%
\def\tbigoplus{\mathop{\textstyle \bigoplus }}%
\def\tbigodot{\mathop{\textstyle \bigodot }}%
\def\tbigsqcup{\mathop{\textstyle \bigsqcup }}%
\def\tcoprod{\mathop{\textstyle \coprod }}%
\def\tbigcup{\mathop{\textstyle \bigcup }}%
\def\tbigvee{\mathop{\textstyle \bigvee }}%
\def\tbigotimes{\mathop{\textstyle \bigotimes }}%
\def\tbiguplus{\mathop{\textstyle \biguplus }}%
%
%
%Macros for display size operators:
%
\def\dint{\mathop{\displaystyle \int}}%
\def\diint{\mathop{\displaystyle \iint}}%
\def\diiint{\mathop{\displaystyle \iiint}}%
\def\diiiint{\mathop{\displaystyle \iiiint }}%
\def\didotsint{\mathop{\displaystyle \idotsint }}%
\def\doint{\mathop{\displaystyle \oint}}%
\def\dsum{\mathop{\displaystyle \sum }}%
\def\dprod{\mathop{\displaystyle \prod }}%
\def\dbigcap{\mathop{\displaystyle \bigcap }}%
\def\dbigwedge{\mathop{\displaystyle \bigwedge }}%
\def\dbigoplus{\mathop{\displaystyle \bigoplus }}%
\def\dbigodot{\mathop{\displaystyle \bigodot }}%
\def\dbigsqcup{\mathop{\displaystyle \bigsqcup }}%
\def\dcoprod{\mathop{\displaystyle \coprod }}%
\def\dbigcup{\mathop{\displaystyle \bigcup }}%
\def\dbigvee{\mathop{\displaystyle \bigvee }}%
\def\dbigotimes{\mathop{\displaystyle \bigotimes }}%
\def\dbiguplus{\mathop{\displaystyle \biguplus }}%

\if@compatibility\else
  % Always load amsmath in LaTeX2e mode
  \RequirePackage{amsmath}
\fi

\def\ExitTCILatex{\makeatother\endinput}

\bgroup
\ifx\ds@amstex\relax
   \message{amstex already loaded}\aftergroup\ExitTCILatex
\else
   \@ifpackageloaded{amsmath}%
      {\if@compatibility\message{amsmath already loaded}\fi\aftergroup\ExitTCILatex}
      {}
   \@ifpackageloaded{amstex}%
      {\if@compatibility\message{amstex already loaded}\fi\aftergroup\ExitTCILatex}
      {}
   \@ifpackageloaded{amsgen}%
      {\if@compatibility\message{amsgen already loaded}\fi\aftergroup\ExitTCILatex}
      {}
\fi
\egroup

%Exit if any of the AMS macros are already loaded.
%This is always the case for LaTeX2e mode.


%%%%%%%%%%%%%%%%%%%%%%%%%%%%%%%%%%%%%%%%%%%%%%%%%%%%%%%%%%%%%%%%%%%%%%%%%%
% NOTE: The rest of this file is read only if in LaTeX 2.09 compatibility
% mode. This section is used to define AMS-like constructs in the
% event they have not been defined.
%%%%%%%%%%%%%%%%%%%%%%%%%%%%%%%%%%%%%%%%%%%%%%%%%%%%%%%%%%%%%%%%%%%%%%%%%%
\typeout{TCILATEX defining AMS-like constructs in LaTeX 2.09 COMPATIBILITY MODE}
%%%%%%%%%%%%%%%%%%%%%%%%%%%%%%%%%%%%%%%%%%%%%%%%%%%%%%%%%%%%%%%%%%%%%%%%
%  Macros to define some AMS LaTeX constructs when 
%  AMS LaTeX has not been loaded
% 
% These macros are copied from the AMS-TeX package for doing
% multiple integrals.
%
\let\DOTSI\relax
\def\RIfM@{\relax\ifmmode}%
\def\FN@{\futurelet\next}%
\newcount\intno@
\def\iint{\DOTSI\intno@\tw@\FN@\ints@}%
\def\iiint{\DOTSI\intno@\thr@@\FN@\ints@}%
\def\iiiint{\DOTSI\intno@4 \FN@\ints@}%
\def\idotsint{\DOTSI\intno@\z@\FN@\ints@}%
\def\ints@{\findlimits@\ints@@}%
\newif\iflimtoken@
\newif\iflimits@
\def\findlimits@{\limtoken@true\ifx\next\limits\limits@true
 \else\ifx\next\nolimits\limits@false\else
 \limtoken@false\ifx\ilimits@\nolimits\limits@false\else
 \ifinner\limits@false\else\limits@true\fi\fi\fi\fi}%
\def\multint@{\int\ifnum\intno@=\z@\intdots@                          %1
 \else\intkern@\fi                                                    %2
 \ifnum\intno@>\tw@\int\intkern@\fi                                   %3
 \ifnum\intno@>\thr@@\int\intkern@\fi                                 %4
 \int}%                                                               %5
\def\multintlimits@{\intop\ifnum\intno@=\z@\intdots@\else\intkern@\fi
 \ifnum\intno@>\tw@\intop\intkern@\fi
 \ifnum\intno@>\thr@@\intop\intkern@\fi\intop}%
\def\intic@{%
    \mathchoice{\hskip.5em}{\hskip.4em}{\hskip.4em}{\hskip.4em}}%
\def\negintic@{\mathchoice
 {\hskip-.5em}{\hskip-.4em}{\hskip-.4em}{\hskip-.4em}}%
\def\ints@@{\iflimtoken@                                              %1
 \def\ints@@@{\iflimits@\negintic@
   \mathop{\intic@\multintlimits@}\limits                             %2
  \else\multint@\nolimits\fi                                          %3
  \eat@}%                                                             %4
 \else                                                                %5
 \def\ints@@@{\iflimits@\negintic@
  \mathop{\intic@\multintlimits@}\limits\else
  \multint@\nolimits\fi}\fi\ints@@@}%
\def\intkern@{\mathchoice{\!\!\!}{\!\!}{\!\!}{\!\!}}%
\def\plaincdots@{\mathinner{\cdotp\cdotp\cdotp}}%
\def\intdots@{\mathchoice{\plaincdots@}%
 {{\cdotp}\mkern1.5mu{\cdotp}\mkern1.5mu{\cdotp}}%
 {{\cdotp}\mkern1mu{\cdotp}\mkern1mu{\cdotp}}%
 {{\cdotp}\mkern1mu{\cdotp}\mkern1mu{\cdotp}}}%
%
%
%  These macros are for doing the AMS \text{} construct
%
\def\RIfM@{\relax\protect\ifmmode}
\def\text{\RIfM@\expandafter\text@\else\expandafter\mbox\fi}
\let\nfss@text\text
\def\text@#1{\mathchoice
   {\textdef@\displaystyle\f@size{#1}}%
   {\textdef@\textstyle\tf@size{\firstchoice@false #1}}%
   {\textdef@\textstyle\sf@size{\firstchoice@false #1}}%
   {\textdef@\textstyle \ssf@size{\firstchoice@false #1}}%
   \glb@settings}

\def\textdef@#1#2#3{\hbox{{%
                    \everymath{#1}%
                    \let\f@size#2\selectfont
                    #3}}}
\newif\iffirstchoice@
\firstchoice@true
%
%These are the AMS constructs for multiline limits.
%
\def\Let@{\relax\iffalse{\fi\let\\=\cr\iffalse}\fi}%
\def\vspace@{\def\vspace##1{\crcr\noalign{\vskip##1\relax}}}%
\def\multilimits@{\bgroup\vspace@\Let@
 \baselineskip\fontdimen10 \scriptfont\tw@
 \advance\baselineskip\fontdimen12 \scriptfont\tw@
 \lineskip\thr@@\fontdimen8 \scriptfont\thr@@
 \lineskiplimit\lineskip
 \vbox\bgroup\ialign\bgroup\hfil$\m@th\scriptstyle{##}$\hfil\crcr}%
\def\Sb{_\multilimits@}%
\def\endSb{\crcr\egroup\egroup\egroup}%
\def\Sp{^\multilimits@}%
\let\endSp\endSb
%
%
%These are AMS constructs for horizontal arrows
%
\newdimen\ex@
\ex@.2326ex
\def\rightarrowfill@#1{$#1\m@th\mathord-\mkern-6mu\cleaders
 \hbox{$#1\mkern-2mu\mathord-\mkern-2mu$}\hfill
 \mkern-6mu\mathord\rightarrow$}%
\def\leftarrowfill@#1{$#1\m@th\mathord\leftarrow\mkern-6mu\cleaders
 \hbox{$#1\mkern-2mu\mathord-\mkern-2mu$}\hfill\mkern-6mu\mathord-$}%
\def\leftrightarrowfill@#1{$#1\m@th\mathord\leftarrow
\mkern-6mu\cleaders
 \hbox{$#1\mkern-2mu\mathord-\mkern-2mu$}\hfill
 \mkern-6mu\mathord\rightarrow$}%
\def\overrightarrow{\mathpalette\overrightarrow@}%
\def\overrightarrow@#1#2{\vbox{\ialign{##\crcr\rightarrowfill@#1\crcr
 \noalign{\kern-\ex@\nointerlineskip}$\m@th\hfil#1#2\hfil$\crcr}}}%
\let\overarrow\overrightarrow
\def\overleftarrow{\mathpalette\overleftarrow@}%
\def\overleftarrow@#1#2{\vbox{\ialign{##\crcr\leftarrowfill@#1\crcr
 \noalign{\kern-\ex@\nointerlineskip}$\m@th\hfil#1#2\hfil$\crcr}}}%
\def\overleftrightarrow{\mathpalette\overleftrightarrow@}%
\def\overleftrightarrow@#1#2{\vbox{\ialign{##\crcr
   \leftrightarrowfill@#1\crcr
 \noalign{\kern-\ex@\nointerlineskip}$\m@th\hfil#1#2\hfil$\crcr}}}%
\def\underrightarrow{\mathpalette\underrightarrow@}%
\def\underrightarrow@#1#2{\vtop{\ialign{##\crcr$\m@th\hfil#1#2\hfil
  $\crcr\noalign{\nointerlineskip}\rightarrowfill@#1\crcr}}}%
\let\underarrow\underrightarrow
\def\underleftarrow{\mathpalette\underleftarrow@}%
\def\underleftarrow@#1#2{\vtop{\ialign{##\crcr$\m@th\hfil#1#2\hfil
  $\crcr\noalign{\nointerlineskip}\leftarrowfill@#1\crcr}}}%
\def\underleftrightarrow{\mathpalette\underleftrightarrow@}%
\def\underleftrightarrow@#1#2{\vtop{\ialign{##\crcr$\m@th
  \hfil#1#2\hfil$\crcr
 \noalign{\nointerlineskip}\leftrightarrowfill@#1\crcr}}}%
%%%%%%%%%%%%%%%%%%%%%

\def\qopnamewl@#1{\mathop{\operator@font#1}\nlimits@}
\let\nlimits@\displaylimits
\def\setboxz@h{\setbox\z@\hbox}


\def\varlim@#1#2{\mathop{\vtop{\ialign{##\crcr
 \hfil$#1\m@th\operator@font lim$\hfil\crcr
 \noalign{\nointerlineskip}#2#1\crcr
 \noalign{\nointerlineskip\kern-\ex@}\crcr}}}}

 \def\rightarrowfill@#1{\m@th\setboxz@h{$#1-$}\ht\z@\z@
  $#1\copy\z@\mkern-6mu\cleaders
  \hbox{$#1\mkern-2mu\box\z@\mkern-2mu$}\hfill
  \mkern-6mu\mathord\rightarrow$}
\def\leftarrowfill@#1{\m@th\setboxz@h{$#1-$}\ht\z@\z@
  $#1\mathord\leftarrow\mkern-6mu\cleaders
  \hbox{$#1\mkern-2mu\copy\z@\mkern-2mu$}\hfill
  \mkern-6mu\box\z@$}


\def\projlim{\qopnamewl@{proj\,lim}}
\def\injlim{\qopnamewl@{inj\,lim}}
\def\varinjlim{\mathpalette\varlim@\rightarrowfill@}
\def\varprojlim{\mathpalette\varlim@\leftarrowfill@}
\def\varliminf{\mathpalette\varliminf@{}}
\def\varliminf@#1{\mathop{\underline{\vrule\@depth.2\ex@\@width\z@
   \hbox{$#1\m@th\operator@font lim$}}}}
\def\varlimsup{\mathpalette\varlimsup@{}}
\def\varlimsup@#1{\mathop{\overline
  {\hbox{$#1\m@th\operator@font lim$}}}}

%
%Companion to stackrel
\def\stackunder#1#2{\mathrel{\mathop{#2}\limits_{#1}}}%
%
%
% These are AMS environments that will be defined to
% be verbatims if amstex has not actually been 
% loaded
%
%
\begingroup \catcode `|=0 \catcode `[= 1
\catcode`]=2 \catcode `\{=12 \catcode `\}=12
\catcode`\\=12 
|gdef|@alignverbatim#1\end{align}[#1|end[align]]
|gdef|@salignverbatim#1\end{align*}[#1|end[align*]]

|gdef|@alignatverbatim#1\end{alignat}[#1|end[alignat]]
|gdef|@salignatverbatim#1\end{alignat*}[#1|end[alignat*]]

|gdef|@xalignatverbatim#1\end{xalignat}[#1|end[xalignat]]
|gdef|@sxalignatverbatim#1\end{xalignat*}[#1|end[xalignat*]]

|gdef|@gatherverbatim#1\end{gather}[#1|end[gather]]
|gdef|@sgatherverbatim#1\end{gather*}[#1|end[gather*]]

|gdef|@gatherverbatim#1\end{gather}[#1|end[gather]]
|gdef|@sgatherverbatim#1\end{gather*}[#1|end[gather*]]


|gdef|@multilineverbatim#1\end{multiline}[#1|end[multiline]]
|gdef|@smultilineverbatim#1\end{multiline*}[#1|end[multiline*]]

|gdef|@arraxverbatim#1\end{arrax}[#1|end[arrax]]
|gdef|@sarraxverbatim#1\end{arrax*}[#1|end[arrax*]]

|gdef|@tabulaxverbatim#1\end{tabulax}[#1|end[tabulax]]
|gdef|@stabulaxverbatim#1\end{tabulax*}[#1|end[tabulax*]]


|endgroup
  

  
\def\align{\@verbatim \frenchspacing\@vobeyspaces \@alignverbatim
You are using the "align" environment in a style in which it is not defined.}
\let\endalign=\endtrivlist
 
\@namedef{align*}{\@verbatim\@salignverbatim
You are using the "align*" environment in a style in which it is not defined.}
\expandafter\let\csname endalign*\endcsname =\endtrivlist




\def\alignat{\@verbatim \frenchspacing\@vobeyspaces \@alignatverbatim
You are using the "alignat" environment in a style in which it is not defined.}
\let\endalignat=\endtrivlist
 
\@namedef{alignat*}{\@verbatim\@salignatverbatim
You are using the "alignat*" environment in a style in which it is not defined.}
\expandafter\let\csname endalignat*\endcsname =\endtrivlist




\def\xalignat{\@verbatim \frenchspacing\@vobeyspaces \@xalignatverbatim
You are using the "xalignat" environment in a style in which it is not defined.}
\let\endxalignat=\endtrivlist
 
\@namedef{xalignat*}{\@verbatim\@sxalignatverbatim
You are using the "xalignat*" environment in a style in which it is not defined.}
\expandafter\let\csname endxalignat*\endcsname =\endtrivlist




\def\gather{\@verbatim \frenchspacing\@vobeyspaces \@gatherverbatim
You are using the "gather" environment in a style in which it is not defined.}
\let\endgather=\endtrivlist
 
\@namedef{gather*}{\@verbatim\@sgatherverbatim
You are using the "gather*" environment in a style in which it is not defined.}
\expandafter\let\csname endgather*\endcsname =\endtrivlist


\def\multiline{\@verbatim \frenchspacing\@vobeyspaces \@multilineverbatim
You are using the "multiline" environment in a style in which it is not defined.}
\let\endmultiline=\endtrivlist
 
\@namedef{multiline*}{\@verbatim\@smultilineverbatim
You are using the "multiline*" environment in a style in which it is not defined.}
\expandafter\let\csname endmultiline*\endcsname =\endtrivlist


\def\arrax{\@verbatim \frenchspacing\@vobeyspaces \@arraxverbatim
You are using a type of "array" construct that is only allowed in AmS-LaTeX.}
\let\endarrax=\endtrivlist

\def\tabulax{\@verbatim \frenchspacing\@vobeyspaces \@tabulaxverbatim
You are using a type of "tabular" construct that is only allowed in AmS-LaTeX.}
\let\endtabulax=\endtrivlist

 
\@namedef{arrax*}{\@verbatim\@sarraxverbatim
You are using a type of "array*" construct that is only allowed in AmS-LaTeX.}
\expandafter\let\csname endarrax*\endcsname =\endtrivlist

\@namedef{tabulax*}{\@verbatim\@stabulaxverbatim
You are using a type of "tabular*" construct that is only allowed in AmS-LaTeX.}
\expandafter\let\csname endtabulax*\endcsname =\endtrivlist

% macro to simulate ams tag construct


% This macro is a fix to the equation environment
 \def\endequation{%
     \ifmmode\ifinner % FLEQN hack
      \iftag@
        \addtocounter{equation}{-1} % undo the increment made in the begin part
        $\hfil
           \displaywidth\linewidth\@taggnum\egroup \endtrivlist
        \global\tag@false
        \global\@ignoretrue   
      \else
        $\hfil
           \displaywidth\linewidth\@eqnnum\egroup \endtrivlist
        \global\tag@false
        \global\@ignoretrue 
      \fi
     \else   
      \iftag@
        \addtocounter{equation}{-1} % undo the increment made in the begin part
        \eqno \hbox{\@taggnum}
        \global\tag@false%
        $$\global\@ignoretrue
      \else
        \eqno \hbox{\@eqnnum}% $$ BRACE MATCHING HACK
        $$\global\@ignoretrue
      \fi
     \fi\fi
 } 

 \newif\iftag@ \tag@false
 
 \def\TCItag{\@ifnextchar*{\@TCItagstar}{\@TCItag}}
 \def\@TCItag#1{%
     \global\tag@true
     \global\def\@taggnum{(#1)}%
     \global\def\@currentlabel{#1}}
 \def\@TCItagstar*#1{%
     \global\tag@true
     \global\def\@taggnum{#1}%
     \global\def\@currentlabel{#1}}

  \@ifundefined{tag}{
     \def\tag{\@ifnextchar*{\@tagstar}{\@tag}}
     \def\@tag#1{%
         \global\tag@true
         \global\def\@taggnum{(#1)}}
     \def\@tagstar*#1{%
         \global\tag@true
         \global\def\@taggnum{#1}}
  }{}

\def\tfrac#1#2{{\textstyle {#1 \over #2}}}%
\def\dfrac#1#2{{\displaystyle {#1 \over #2}}}%
\def\binom#1#2{{#1 \choose #2}}%
\def\tbinom#1#2{{\textstyle {#1 \choose #2}}}%
\def\dbinom#1#2{{\displaystyle {#1 \choose #2}}}%

% Do not add anything to the end of this file.  
% The last section of the file is loaded only if 
% amstex has not been.
\makeatother
\endinput

\renewcommand{\today}{西元\number \year 年\ifcase \month \or 1月\or 2月\or 3月\or 4月\or 5月\or 6月\or 7月\or 8月\or 9月\or 10月\or 11月\or 12月\fi
} 

\begin{document}


%TCIMACRO{\TeXButton{gyrologo5 pgf image}{\input{gyrologo5.pgf}} }%
%BeginExpansion
\input{gyrologo5.pgf}
%EndExpansion
%TCIMACRO{\TeXButton{title pgf3 image}{%% Creator: Matplotlib, PGF backend
%%
%% To include the figure in your LaTeX document, write
%%   \input{<filename>.pgf}
%%
%% Make sure the required packages are loaded in your preamble
%%   \usepackage{pgf}
%%
%% Figures using additional raster images can only be included by \input if
%% they are in the same directory as the main LaTeX file. For loading figures
%% from other directories you can use the `import` package
%%   \usepackage{import}
%% and then include the figures with
%%   \import{<path to file>}{<filename>.pgf}
%%
%% Matplotlib used the following preamble
%%   \usepackage{fontspec}
%%   \setmainfont{Times New Roman}
%%   \setsansfont{Verdana}
%%   \setmonofont{Courier New}
%%
\begingroup%
\makeatletter%
\begin{pgfpicture}%
\pgfpathrectangle{\pgfpointorigin}{\pgfqpoint{5.000000in}{2.000000in}}%
\pgfusepath{use as bounding box}%
\begin{pgfscope}%
\pgfsetbuttcap%
\pgfsetroundjoin%
\definecolor{currentfill}{rgb}{1.000000,1.000000,1.000000}%
\pgfsetfillcolor{currentfill}%
\pgfsetlinewidth{0.000000pt}%
\definecolor{currentstroke}{rgb}{1.000000,1.000000,1.000000}%
\pgfsetstrokecolor{currentstroke}%
\pgfsetdash{}{0pt}%
\pgfpathmoveto{\pgfqpoint{0.000000in}{0.000000in}}%
\pgfpathlineto{\pgfqpoint{5.000000in}{0.000000in}}%
\pgfpathlineto{\pgfqpoint{5.000000in}{2.000000in}}%
\pgfpathlineto{\pgfqpoint{0.000000in}{2.000000in}}%
\pgfpathclose%
\pgfusepath{fill}%
\end{pgfscope}%
\begin{pgfscope}%
\definecolor{textcolor}{rgb}{0.501961,0.501961,0.501961}%
\pgfsetstrokecolor{textcolor}%
\pgfsetfillcolor{textcolor}%
\pgftext[x=-0.750000in,y=0.160000in,left,base]{{\sffamily\fontsize{28.000000}{33.600000}\selectfont {\bfseries {\kai 陀螺運動的電腦模擬}}}}%
\end{pgfscope}%
\begin{pgfscope}%
\pgftext[x=-0.200000in,y=0.680000in,left,base]{{\sffamily\fontsize{15.000000}{18.000000}\selectfont  \textcolor{gray}{深談貼體角速度在剛體轉動積分器與姿態估測的實際應用}}}%
\end{pgfscope}%
\begin{pgfscope}%
\pgftext[x=0.200000in,y=1.000000in,left,base]{{\sffamily\fontsize{12.000000}{14.400000}\selectfont 小良,台南,\today}}%
\end{pgfscope}%
\begin{pgfscope}%
\definecolor{textcolor}{rgb}{0.350000,0.350000,0.350000}%
\pgfsetstrokecolor{textcolor}%
\pgfsetfillcolor{textcolor}%
\pgftext[x=5.000000in,y=2.400000in,right,top]{{\sffamily\fontsize{13.000000}{15.600000}\selectfont 古典力學數值方法技術文件一}}%
\end{pgfscope}%
\end{pgfpicture}%
\makeatother%
\endgroup%
}}%
%BeginExpansion
%% Creator: Matplotlib, PGF backend
%%
%% To include the figure in your LaTeX document, write
%%   \input{<filename>.pgf}
%%
%% Make sure the required packages are loaded in your preamble
%%   \usepackage{pgf}
%%
%% Figures using additional raster images can only be included by \input if
%% they are in the same directory as the main LaTeX file. For loading figures
%% from other directories you can use the `import` package
%%   \usepackage{import}
%% and then include the figures with
%%   \import{<path to file>}{<filename>.pgf}
%%
%% Matplotlib used the following preamble
%%   \usepackage{fontspec}
%%   \setmainfont{Times New Roman}
%%   \setsansfont{Verdana}
%%   \setmonofont{Courier New}
%%
\begingroup%
\makeatletter%
\begin{pgfpicture}%
\pgfpathrectangle{\pgfpointorigin}{\pgfqpoint{5.000000in}{2.000000in}}%
\pgfusepath{use as bounding box}%
\begin{pgfscope}%
\pgfsetbuttcap%
\pgfsetroundjoin%
\definecolor{currentfill}{rgb}{1.000000,1.000000,1.000000}%
\pgfsetfillcolor{currentfill}%
\pgfsetlinewidth{0.000000pt}%
\definecolor{currentstroke}{rgb}{1.000000,1.000000,1.000000}%
\pgfsetstrokecolor{currentstroke}%
\pgfsetdash{}{0pt}%
\pgfpathmoveto{\pgfqpoint{0.000000in}{0.000000in}}%
\pgfpathlineto{\pgfqpoint{5.000000in}{0.000000in}}%
\pgfpathlineto{\pgfqpoint{5.000000in}{2.000000in}}%
\pgfpathlineto{\pgfqpoint{0.000000in}{2.000000in}}%
\pgfpathclose%
\pgfusepath{fill}%
\end{pgfscope}%
\begin{pgfscope}%
\definecolor{textcolor}{rgb}{0.501961,0.501961,0.501961}%
\pgfsetstrokecolor{textcolor}%
\pgfsetfillcolor{textcolor}%
\pgftext[x=-0.750000in,y=0.160000in,left,base]{{\sffamily\fontsize{28.000000}{33.600000}\selectfont {\bfseries {\kai 陀螺運動的電腦模擬}}}}%
\end{pgfscope}%
\begin{pgfscope}%
\pgftext[x=-0.200000in,y=0.680000in,left,base]{{\sffamily\fontsize{15.000000}{18.000000}\selectfont  \textcolor{gray}{深談貼體角速度在剛體轉動積分器與姿態估測的實際應用}}}%
\end{pgfscope}%
\begin{pgfscope}%
\pgftext[x=0.200000in,y=1.000000in,left,base]{{\sffamily\fontsize{12.000000}{14.400000}\selectfont 小良,台南,\today}}%
\end{pgfscope}%
\begin{pgfscope}%
\definecolor{textcolor}{rgb}{0.350000,0.350000,0.350000}%
\pgfsetstrokecolor{textcolor}%
\pgfsetfillcolor{textcolor}%
\pgftext[x=5.000000in,y=2.400000in,right,top]{{\sffamily\fontsize{13.000000}{15.600000}\selectfont 古典力學數值方法技術文件一}}%
\end{pgfscope}%
\end{pgfpicture}%
\makeatother%
\endgroup%
%
%EndExpansion

\part{導言\protect\bigskip}

當我們談到陀螺的三%
維運動,多數動力學%
的書上推導出貼體角%
速度的尤拉方程之後%
,就會求諸Euler angles來得到%
Lagrangian,接著用elliptical integral解出%
解析解。或者,以數%
值方法解其尤拉角的%
尤拉方程ODE,然後再以%
尤拉角模擬其運動。%
不過,若對尤拉方程%
透徹理解,只要有貼%
體角速度,也可以利%
用轉動向量﹝rotation vector﹞%
,用姿態估測的方法%
直接簡單地數值模擬%
剛體轉動,這樣就不%
需要用到Lagrangian或Euler angles。%
這個方法屬於姿態估%
測學中的方向餘弦遞%
推法﹝iteration of direction cosine matrix, DCM﹞%
。這裡將姿態﹝orientation 姿%
態 or attitude 航向,這裡統稱%
姿態。﹞動力學與姿%
態估測應用在剛體轉%
動的尤拉運動方程上%
,以姿態估測中的方%
向餘弦法來積分尤拉%
方程的貼體角速度轉%
動向量,以此達到模%
擬剛體的轉動,並且%
應用在陀螺的三維運%
動上。一般會覺得貼%
體角速度並沒有多大%
的用處,不過事實上%
在姿態動力學中,正%
是以正確地積分貼體%
角速度來得到姿態的%
運動。這邊不只介紹%
公式,還以淺顯易懂%
的方式給出姿態的運%
動方程,所以只要有%
基礎的線性代數矩陣%
知識,就可以掌握此%
方法。這裡詳述的這%
些理論在書籍上比較%
少見,在航太領域或%
電機的姿態控制領域%
有些講解,但也缺乏%
清楚的解釋,只給出%
複雜的公式,這樣在%
應用層面的時候會發%
生比較多不必要的試%
誤與嘗試。這裡我將%
這觀念連結錯綜複雜%
的東西清楚地寫下來%
,並且提供一個實作%
的例子,以供自己以%
及別人參考。

\bigskip

這裡詳述的方法可以%
廣泛的運用於任何剛%
體轉動或其尤拉方程%
。因此,此處所涉及%
的方向餘弦遞推的完%
整理解還可應用上其%
他相關領域。舉例來%
說,這裡在作陀螺模%
擬的姿態演算程式可%
以用在陀螺儀這類型%
的綁附式慣性感測器\cite%
[Ch 3.6.4]{titterton}上,積分測得的%
貼體角速度來作姿態%
估測,這將在第四章%
作介紹。仿間姿態演%
算法林立,但多數是%
沒有經過驗證的。這%
邊從根本的推導開始%
,一步步進階到實用%
的程式碼,每一步推%
導都有合理的邏輯,%
並且轉動矩陣都有標%
明轉動方向是遵守左%
手還是右手定則,以%
及標明轉動矩陣是取%
其主動性轉向量或是%
被動性轉坐標軸,這%
一點是所有書上或網%
路上的程式所共同缺%
乏的,而這在實際應%
用的時候尤其重要,%
這樣子推導出來的公%
式我們可以很方便的%
檢查其正確性,也可%
以很放心地應用在其%
他地方。另外,這邊%
提供與第三方理論及%
程式碼的驗證,證實%
我們公式及程式的正%
確性。這裡介紹的尤%
拉方程的數值化也可%
以應用上剛體物理模%
擬﹝simulation of rigid body﹞,這在%
電腦遊戲的物理引擎\cite%
[Ch 2.3]{pixarnote}中也有其廣泛的%
應用性。這邊我們呈%
現以這些方法數值模%
擬三維陀螺運動。這%
邊提供的詳細解說也%
適合當作大學物理,%
電機,機械或航太系%
在剛體運動的衍伸教%
材。

\bigskip

也有其他人做過類似%
的陀螺模擬,舉例如%
Wolfgang Christian教授\cite{wolfgangSimMeth},以%
及Eugene Bukitov教授\cite{eugene},不過%
都沒有我們這邊完整%
。比如說,Christian教授的%
程式無法改變$\omega _{2}$的參%
數,因此無法觀察到%
波紋及正圓兩種章動%
進動運動。Bukitov教授的%
參數只能在某區間裡%
以拉桿調整,而我們%
的程式沒有參數的限%
制,可以做任意運動%
狀態的模擬。Christian教授%
使用的是四元數遞推%
法。四元數法與我這%
裡的方向餘弦法在數%
學上是同義的。兩位%
教授都沒有提供跟其%
他方法做比較驗證的%
選項,我這裡是有跟%
Lagrange法的數值解做驗證%
比較。我下一步也希%
望能跟四元數法這類%
的數值解能做比較。%
Bukitov教授沒有公開他的%
演算法,因此不曉得%
他是用哪種方法模擬%
。很感謝Christian教授有提%
供java的source code,不過我還%
在思考怎麼將我的Python%
結果跟他的Java結果比較%
。值得一提的是,他%
有把他的Java程式寫成一%
個獨立的pc端執行檔,%
公開在comPADRE網路上,此%
程式是互動性的,使%
用起來非常的方便,%
這也是我的目標之一%
。四元數及方向餘弦%
法的理論可以在\cite{rapaport}%
文獻書中找到,姿態%
估測演算法的深度理%
論可參考文章\cite{savage},不%
過裡面並沒有提供實%
際的模擬。

這裡的實際應用例子%
也提供了一個,以尤%
拉方程來驗證姿態估%
測演算法的正確性的%
一個實例。即以尤拉%
方程解析解或數值解%
給出的貼體角速度來%
做姿態估測,結果再%
與原本的尤拉角解析%
結果直接比較,這樣%
比較的即是姿態估測%
演算法的估測能力,%
這會在文中詳述。這%
邊似乎是第一個完整%
結合尤拉方程與姿態%
估測並實際應用的例%
子。

這篇文章的架構是:

\begin{enumerate}
\item 第一部分為最完整%
的剛體轉動運動方程-%
牛頓尤拉﹝Newton-Euler or Euler﹞方%
程的推導證明。因為%
要做轉動軸的轉動向%
量數值積分必須要對%
尤拉方程有最正確的%
理解。這邊補充了Goldstein
Classical Mechanics\cite{goldstein}中的證明觀%
念跳來跳去的缺失,%
以及大多數的書上推%
導尤拉方程解釋模糊%
的地方。轉動理論多%
數的講述通常過於複%
雜,要不就是過於簡%
化,缺乏與基本原理%
的結合,並且大部分%
也無提供實際實例操%
作這非常重要的一環%
,這裡則提供完整的%
陀螺模擬實例與完整%
公開的code供練習。

\item 接著藉由第一段尤%
拉方程的推導來嚴謹%
的證明Euler equation中的貼體%
角速度﹝angular velocity along body frame﹞%
可直接用於建立剛體%
特徵軸與lab frame間的主動%
與被動轉動矩陣\thinspace ,%
並以此推導出方向餘%
弦法的主要遞推原理%
來積分剛體轉動,追%
蹤每一時刻的剛體特%
徵軸在lab frame的位置。並%
以python程式編寫演算法%
,並以python繪圖庫matplotlib來%
作3D動畫。

\item 接著說明了以貼體%
轉動向量來近似t到t+dt時%
間的微小轉動時我們%
將以在t+dt時間的貼體角%
速度來近似,即以$\vec{\omega}%
_{b}\left( t+dt\right) $來建立dt時間內%
的轉動矩陣$\footnote{%
The path order exponential of $\vec{\omega}_{b}$ from time t to t+dt. This
will be discussed more in the text.}$。這裡也比%
較一般常見的$\vec{\omega}_{b}\left(
t\right) $方法的結果來證明%
此優化所帶來精度上%
的提升。最後將以上%
方法應用上陀螺的三%
維運動,模擬結果將%
與文獻\cite{hasbun}的尤拉角法%
做比較。

\item 最後說明如何以以%
上的基礎來建立三種%
姿態估測演算法。其%
中一種即為慣性角速%
度感測器﹝IMU﹞的姿態%
解算基礎。另一種為%
利用力矩項的姿態估%
測演算法,適合用於%
電腦姿態模擬。最後%
一種為傳統的尤拉角%
加Lagrangian法。我們的程式%
可以在同一個動畫平%
台上同時比較這三種%
方法的正確性,因此%
相當於是一個姿態估%
測能力的比較平台。
\end{enumerate}

%TCIMACRO{\TeXButton{clearpage}{\clearpage}}%
%BeginExpansion
\clearpage%
%EndExpansion

\part{尤拉方程與姿態動%
力方程的推導}

\setcounter{page}{1}\bigskip

若你有認真思考過轉%
動的問題,你應該會%
曾經有過下面的疑問%
。

\begin{enumerate}
\item 牛頓尤拉方程\ref{liw}中%
的力矩項,明明在推%
導的時候是沿著固定%
的space座標取分量,為什%
麼在計算在應用的時%
候反而要沿著貼體的body%
座標取分量?﹝如Goldstein
classical mechanics第二版書中4-124公%
式﹞

\item 若我隨著轉動座標%
系運動,那麼我觀察%
到此座標系的轉動角%
速度應該是零,因為%
我隨著此座標系運動%
,但是body angular velocity偏偏又不%
是零?到底怎麼回事%
?那角度感測器angular rate sensor%
量到的也不是零,那%
它量到的到底又是甚%
麼東西,我們怎麼用%
它?

\item 貼體角速度﹝body angular velocity%
﹞明明是angular velocity沿著body投%
影的分量,為什麼在%
帶入公式的時候它會%
等於剛體特徵軸的軸%
轉動角速度?body angular velocity是%
不是等於轉動座標軸%
中\emph{觀測}到的angular velocity?

\item 與上一點非常相關%
,一般認為貼體角速%
度無太大用處,事實%
上在剛體運動中,貼%
體角速度因姿態動力%
學而有許多廣泛的應%
用性。
\end{enumerate}

這邊很少有書上會去%
詳細解釋清楚,這裡%
從最根本出發一次解%
決你的疑惑。

\bigskip

首先我們必須再次證%
明向量變化量在不同%
觀測座標中的關係,%
即公式\ref{eq1},這次你會%
發現在應用此公式的%
時候有不少地方是要%
注意的。由於當我實%
際在解這問題時我發%
現Goldstein classical mechanics書中還有幾%
點證明在應用層面還%
不夠清楚,讓我在應%
用的時候花了好多時%
間去釐清,因此這邊%
寫上我認為可以補充%
書上的推導證明。首%
先我們從以下公式開%
始

\begin{equation}
\left( \frac{d\vec{L}}{dt}\right) _{s}=\left( \frac{d\vec{L}}{dt}\right)
_{b}+\vec{\omega}\times \vec{L}  \label{eq1}
\end{equation}%
對此公式的理解將是%
整篇文章最重要的基%
礎。此公式如何而來%
?此公式為一隨時間%
變動的向量在恆定座%
標與非恆定座標(此例%
為轉動中座標)之間線%
性變換的結果。以下%
是此公式的推導。

\begin{figure}[th]
\caption{{}}
\label{firstfig}
\begin{center}
\fbox{\input{rateofchange.pgf}}
\end{center}
\end{figure}
\bigskip

首先考慮一恆定座標%
S(space),一轉動座標b(body),%
為了方便討論矩陣轉%
換的主被動性與座標%
轉換的左右手法則,%
我們這邊方便的先假%
設$\hat{S}_{x}, \hat{b}_{x}$兩軸重合%
,因此圖中顯示了body frame%
沿著$+\hat{S}_{x}$遵守右手定%
則逆時針轉了$\Omega $角度%
,依右手定則此角位%
移向量$\hat{\Omega}$會在$+\hat{S}_{x}$方%
向。但是接下來的推%
導以及所有公式都適%
用任意的座標旋轉,%
這邊是為了方便討論%
矩陣的主動被動的方%
向性,以及在之後的%
推導方便我們追蹤正%
負號以及矩陣主動被%
動意義的改變,因此%
在圖中做了一個方便%
我們思考的情形。另%
外,大部分書上在討%
論座標轉換時有時候%
給的公式是遵守左手%
定則,但這與物理定%
律所採納的右手定則%
相反,因此這邊我寫%
下完整的右手定則的%
推導,希望之後的人%
不需要像我一樣花了%
大半時間在轉換不同%
公式間左手右手定則%
帶來的正負號的改變%
。

依照圖\ref{firstfig}所示,我%
們可以寫下$\vec{A}$向量在S,b%
座標間的關係%
\begin{equation*}
\left( \vec{A}\right) _{b}=\underset{\text{passive, r.h.}}{\Omega }\left( 
\vec{A}\right) _{s}
\end{equation*}%
其中$\Omega $是s frame到b frame的座%
標轉換矩陣,因為是%
轉換座標軸,因此矩%
陣取被動含意,並且%
我們採用右手定則,%
因此逆時針方向為正%
方向。接下來只要有%
用到矩陣的運算我都%
會標明主被動及左右%
手(r.h. right-hand or l.h. left-hand),這樣我%
們可以很快速對照圖%
表來理解轉動方向,%
這很重要。

若我們考慮$\Omega $的角度%
很小$\Omega \rightarrow d\Omega $(infinitesimal rotation),%
則$d\Omega $矩陣與unity matrix相去%
不遠,可以寫成$1$(unity matrix) +$%
\epsilon $(infinitesimal matrix),$\epsilon $具有%
antisymmetric matrix的特性\cite[p. 169]{goldstein},%
帶入上式%
\begin{equation*}
\left( \vec{A}\right) _{b}=\underset{\text{passive, r.h.}}{\left( 1+\epsilon
\right) }\left( \vec{A}\right) _{s}
\end{equation*}%
infinitesimal matrix有個特性,很容%
易自行驗證,%
\begin{equation*}
\underset{\text{r.h., passive or active}}{\epsilon }=\left[ 
\begin{array}{ccc}
0 & \epsilon _{3}\geq 0 & -\epsilon _{2}\leq 0 \\ 
-\epsilon _{3} & 0 & \epsilon _{1}\geq 0 \\ 
\epsilon _{2} & -\epsilon _{1} & 0%
\end{array}%
\right] \text{, }\underset{\text{l.h., passive or active}}{\epsilon }=\left[ 
\begin{array}{ccc}
0 & -\epsilon _{3}\leq 0 & \epsilon _{2}\geq 0 \\ 
+\epsilon _{3} & 0 & -\epsilon _{1}\leq 0 \\ 
-\epsilon _{2} & \epsilon _{1} & 0%
\end{array}%
\right]
\end{equation*}

現在我們考慮$\vec{A}$是$+\hat{b}%
_{y}$軸的狀況,不過考慮%
相同矩陣$\left( 1+\epsilon \right) $的主%
動特性,也就是主動%
轉向量,這樣的話轉%
動方向會與原本的方%
向相反,變左手定則%
,我們會得到%
\begin{equation*}
\left( \hat{S}_{y}\right) _{s}=\underset{\text{active, l.h.}}{\left(
1+\epsilon \right) }\times \left( \hat{b}_{y}\right) _{s}
\end{equation*}%
整理一下%
\begin{equation*}
\left( \hat{b}_{y}\right) _{s}=\underset{\text{active, r.h.}}{\underbrace{%
\left[ \left( 1+\epsilon \right) \right] ^{T}}}\times \left( \hat{S}%
_{y}\right) _{s}=\underset{\text{active, r.h.}}{\left( 1-\epsilon \right) }%
\times \left( \hat{S}_{y}\right) _{s}
\end{equation*}%
代入上面r.h. $\epsilon $的公式(%
因$\epsilon $還是原本的矩陣)%
,整理一下%
\begin{equation*}
\left( \hat{b}_{y}\right) _{s}-\left( \hat{S}_{y}\right) _{s}=-\left[ 
\begin{array}{ccc}
0 & \epsilon _{3}\geq 0 & -\epsilon _{2}\leq 0 \\ 
-\epsilon _{3} & 0 & \epsilon _{1}\geq 0 \\ 
\epsilon _{2} & -\epsilon _{1} & 0%
\end{array}%
\right] \times \left( \hat{S}_{y}\right) _{s}
\end{equation*}%
利用向量外積,上式%
也可寫成%
\begin{equation*}
\left( \hat{b}_{y}\right) _{s}-\left( \hat{S}_{y}\right) _{s}=\left( \vec{%
\epsilon}\right) _{s}\times \left( \hat{S}_{y}\right) _{s}
\end{equation*}%
其中$\vec{\epsilon}=\left[ 
\begin{array}{c}
\epsilon _{1} \\ 
\epsilon _{2} \\ 
\epsilon _{3}%
\end{array}%
\right] _{s}$為一向量,在S frame中%
的分量為$\epsilon _{1}$,$\epsilon _{2}$%
,$\epsilon _{3}$。

現在我們將上式跟微%
小轉動公式Rodrigues rotation formula比%
較%
\begin{equation*}
\vec{r}^{\prime }-\vec{r}=d\vec{\Omega}\times \vec{r}
\end{equation*}%
$d\vec{\Omega}$是r到r'的r.h.角位移%
向量\thinspace ,因此我們得%
到$\vec{\epsilon}=d\vec{\Omega}$,$d\vec{\Omega}$就是%
s frame到b frame的角位移向量%
(follow r.h. rule)%
\begin{equation*}
\left( \hat{b}_{y}\right) _{s}-\left( \hat{S}_{y}\right) _{s}=\left( d\vec{%
\Omega}\right) _{s}\times \left( \hat{S}_{y}\right) _{s}
\end{equation*}%
這一點很重要,因為%
我們將證明此$\left( d\vec{\Omega}\right)
_{s}$跟接下來我們要推導%
的尤拉公式中的貼體%
角速度$\vec{\omega}$有直接相%
關性,並且以此來做%
我們模擬剛體轉動的%
基礎。

以上的討論是考慮$\vec{A}$%
向量不隨時間變動的%
情況,接下來我們必%
須討論$\vec{A}$以及b frame皆隨%
時間變動的狀況。

\begin{figure}[th]
\caption{Rate change of a vector observed in a inertial and non-inertial
frame.}
\label{ratevecfig}
\begin{center}
\fbox{\input{rateofchanget2tdt.pgf}}
\end{center}
\end{figure}

在時間t時我們令S與b frame%
重合,過了dt時間原本%
的$\vec{A}$向量加了一改變%
量$d\vec{A}$變成$\vec{A}^{\prime }$,並且%
b frame依右手定則轉動了%
一微小角度(infinitisemal rotation),%
在此前提下,向量$\vec{A}$%
在t時間符合%
\begin{equation}
\left( \vec{A}\right) _{s(t)}=\left( \vec{A}\right) _{b(t)}  \label{roc1}
\end{equation}%
接著,在t+dt時間$\vec{A}+d\vec{A}$%
向量在s與b frame間的關係%
為%
\begin{equation}
\left( \vec{A}^{\prime }\right) _{b(t+dt)}=\underset{\text{passive, r.h.}}{%
\Omega }\left( \vec{A}^{\prime }\right) _{s(t+dt)}  \label{roc1dt}
\end{equation}%
$\Omega $為s, b frame轉動矩陣(passive r.h.)%
,此$\Omega $矩陣與上一段$%
\vec{A}$不變動的情況的$\Omega $%
矩陣完全相同,我們%
取s到b frame的轉動為微小%
量,$\Omega \rightarrow d\Omega $,上式依%
之前所述的原理可寫%
成%
\begin{equation*}
\left( \vec{A}^{\prime }\right) _{b(t+dt)}=\underset{\text{passive, r.h.}}{%
\left( 1+\epsilon \right) }\left( \vec{A}^{\prime }\right) _{s(t+dt)}
\end{equation*}%
要強調這邊的$\epsilon $矩陣%
跟之前上一段的$\epsilon $矩%
陣是完全相同的,代%
入\ref{roc1dt}式重新整理上式%
\begin{equation}
\left( \vec{A}^{\prime }\right) _{s(t+dt)}=\left( \vec{A}^{\prime }\right)
_{b(t+dt)}-\epsilon \left( \vec{A}^{\prime }\right) _{s(t+dt)}  \label{roc2}
\end{equation}%
接著我們用\ref{roc2}式減去%
\ref{roc1}式,%
\begin{eqnarray}
\left( \vec{A}^{\prime }\right) _{s(t+dt)}-\left( \vec{A}\right) _{s(t)}
&=&\left( d\vec{A}\right) _{s}\text{ , (the change in observable A in space
frame)}  \notag \\
&=&\left( \left( \vec{A}^{\prime }\right) _{b(t+dt)}-\left( \vec{A}\right)
_{b(t)}\right) -\epsilon \left( \vec{A}^{\prime }\right) _{s(t+dt)}  \notag
\\
&=&\left( d\vec{A}\right) _{body}-\epsilon \left( \vec{A}^{\prime }\right)
_{s(t+dt)}  \notag \\
&=&\left( d\vec{A}\right) _{body}-\epsilon \left( \vec{A}+d\vec{A}\right)
_{s(t+dt)}  \label{roc4}
\end{eqnarray}%
注意,由於s frame式恆定%
座標因此s frame不變動,$%
s(t+dt)=s(t)$。忽略高階項$\epsilon
\left( d\vec{A}\right) _{s(t+dt)}$,重新整理%
成%
\begin{equation*}
\left( d\vec{A}\right) _{s}=\left( d\vec{A}\right) _{b}-\underset{\text{r.h.}%
}{\epsilon }\left( \vec{A}\right) _{s(t)}
\end{equation*}%
接下來我們只要記得%
我們的下標b frame總是在t+dt%
時的frame,s frame總是指在t時%
間的frame,我們將不再寫%
出frame所對應的時間。依%
之前所述原理代入r.h. $%
\epsilon $的公式,並且利用%
向量外積%
\begin{eqnarray}
\left( d\vec{A}\right) _{s} &=&\left( d\vec{A}\right) _{b}-\left[ 
\begin{array}{ccc}
0 & \epsilon _{3}\geq 0 & -\epsilon _{2}\leq 0 \\ 
-\epsilon _{3} & 0 & \epsilon _{1}\geq 0 \\ 
\epsilon _{2} & -\epsilon _{1} & 0%
\end{array}%
\right] \left( \vec{A}\right) _{s}  \notag \\
&=&\left( d\vec{A}\right) _{b}-\left( \vec{A}\right) _{s}\times \left( d\vec{%
\Omega}\right) _{s}  \label{roc3} \\
&=&\left( d\vec{A}\right) _{b}+\left( d\vec{\Omega}\right) _{s}\times \left( 
\vec{A}\right) _{s}  \notag
\end{eqnarray}%
因為這裡的$\epsilon $矩陣與%
上一段的$\epsilon $矩陣是一%
樣的,因此我們也可%
以用上之前轉動公式%
所推導的微小轉動矩%
陣$\epsilon $所對應的轉動向%
量$\left( d\vec{\Omega}\right) $,這樣我%
們就得到了rate of change of a
vector/observable in a rotating frame公式%
\begin{equation}
\left( d\vec{A}\right) _{s}=\left( d\vec{A}\right) _{b}+\left( d\vec{\Omega}%
\right) _{s}\times \left( \vec{A}\right) _{s}  \label{rateofdomega}
\end{equation}%
我們連結了不同觀測%
座標觀測到的物理變%
化量,並且所用到都%
是已知的物理量$\left( d\vec{\Omega}%
\right) _{s}$與$\left( \vec{A}\right) _{s}$。這邊%
要強調,因為這裡的$%
\epsilon $矩陣與上一段的$\epsilon $%
矩陣是一樣的,所以%
證明了$d\vec{\Omega}$所對應的%
向量就是s frame轉到b frame的%
角位移向量(r.h.),這樣%
強調的目的是,接下%
來$d\vec{\Omega}$所導出的貼體%
角速度,就是s frame轉到b
frame的角速度,因為這跟%
一般我們對貼體角速%
度的定義與認知並不%
一樣,這裡再次強調%
,我們必須考慮了t時%
間s與b frame重合,才能得%
到這結果。接下來會%
說明,也是因為如此%
,我們才能利用貼體%
角速度來作座標軸的%
轉動追蹤,因此此觀%
念至關重要。另外要%
注意的是$\vec{A}$與$d\vec{\Omega}$是%
沿著t時間的s frame取的投%
影量。這邊值得一提%
的是,傳統公式大多%
寫成%
\begin{equation*}
\left( d\vec{A}\right) _{s}=\left( d\vec{A}\right) _{b}+\left( d\vec{\Omega}%
\right) _{b}\times \left( \vec{A}\right) _{b}
\end{equation*}%
為什麼這邊的$d\vec{\Omega}$是%
沿著body取分量呢?注意%
\ref{roc3}式中當我們將微小%
轉動矩陣$\epsilon $寫成向量$d%
\vec{\Omega}$時,我們並沒有侷%
限此向量是定義在哪%
一個觀測座標,因此%
沿著s或b frame取都是可以%
的。另外\ref{roc4}式中我們%
忽略了高階項$\epsilon \left( d\vec{A}%
\right) _{s(t+dt)}$,因此留下了$\epsilon
\left( \vec{A}\right) _{s(t+dt)}=\epsilon \left( \vec{A}\right) _{s(t)}$%
,但其實我們也可以%
取另一個近似%
\begin{equation*}
-\epsilon \left( \vec{A}+d\vec{A}\right) _{s(t+dt)}=-\epsilon \left( \vec{A}%
\right) _{b(t+dt)}
\end{equation*}%
若我們考慮s到b frame的轉%
動非常的微小,我們%
只是做了一個不一樣%
的忽略方法。這樣的%
話,我們就得到傳統%
公式。這裡也是Goldstein\cite%
{goldstein}裡面的附註說明$\left( d%
\vec{\Omega}\right) \times \left( \vec{A}\right) $沿著s或b
frame取分量都是可以的,%
只要外積矩陣運算後%
出來的結果是一樣的%
就可以,不過他並沒%
有給出背後的原因。

\ref{rateofdomega}式取微分即得到%
一般常見的形式,這%
邊我們取沿s frame給出的%
公式,方便我們之後%
作數值模擬%
\begin{equation}
\left( \frac{d\vec{A}}{dt}\right) _{s}=\left( \frac{d\vec{A}}{dt}\right)
_{b}+\left( \vec{\omega}\right) _{s}\times \left( \vec{A}\right) _{s}
\label{rateofchange}
\end{equation}%
其中$\left( \vec{\omega}\right) _{s}$為$\left( \frac{d\vec{%
\Omega}}{dt}\right) _{s}$,根據我們之%
前對$d\vec{\Omega}$的定義與強%
調,我們知道$\left( \vec{\omega}\right)
_{s}$即為s frame到b frame的瞬時角%
速度。

嚴謹的定義了$d\vec{\Omega}$與$%
\left( \vec{\omega}\right) _{s}$後,我們接%
著需要討論如何從$\left( \vec{%
\omega}\right) _{s}$求回相對應的轉%
動矩陣,這邊你會認%
為,不是將$\left( \vec{\omega}\right) _{s}$%
的xyz分量帶入之前$1+\epsilon $%
矩陣中的$\epsilon _{1}\epsilon _{2}\epsilon _{3}$%
就可以了嗎,這樣是%
不行的,因為從之前%
微小轉動的推導可以%
看出,$\epsilon _{1}\epsilon _{2}\epsilon _{3}$是%
符合特定的antisymmetric matrix properties%
的,但任意的角速度%
向量$\left( \vec{\omega}\right) _{s}$可不然%
。這邊我們利用Calvin Klein
parameter來近似原本的轉動%
矩陣﹝CK parameters矩陣可從轉%
動公式Rodrigues rotation formula推導,%
它們數學上是同源的%
,參考\cite{goldstein}﹞,這邊%
我們給他一個代號$CK(d\vec{%
\Omega})$,當然,接下來只%
要是矩陣運算我們都%
會寫上$CK$的主被動及左%
右手性質,方便我們%
與圖對照與思考\footnote{物%
理上力矩給出的角速%
度是遵守右手定則%
(counterclockwise,不過說順或逆%
時針常常會造成困擾%
,因順逆時針方向是%
軸方向正對著我們自%
己的時候給出的方向%
。),所以CK矩陣必須使%
用其active r.h. sense才能描述正%
確向量轉動,要小心%
,因大部分書上(如Goldstein)%
給的公式都是active l.h. (舉例%
如書上的Caley Klein parameter rotation matrix)%
,因此差一個負號,%
這裡我也花了許多時%
間把文獻上所有公式%
轉成了正確的右手定%
則。}。%
\begin{eqnarray*}
\underset{\text{active, r.h.}}{CK(d\vec{\Omega})} &=&\left[ 
\begin{array}{ccc}
a^{2}+b^{2}-c^{2}-d^{2} & 2(bc-ad) & 2(bd+ac) \\ 
2(bc+ad) & a^{2}+c^{2}-b^{2}-d^{2} & 2(cd-ab) \\ 
2(bd-ac) & 2(cd+ab) & a^{2}+d^{2}-b^{2}-c^{2}%
\end{array}%
\right] \text{,} \\
\text{with }a &=&\cos \left( \frac{\left\vert d\vec{\Omega}\right\vert }{2}%
\right) \text{, b, c, d = component of }d\hat{\Omega}\cdot \sin \left( \frac{%
\left\vert d\vec{\Omega}\right\vert }{2}\right)
\end{eqnarray*}%
現在,我們一再強調$d%
\vec{\Omega}$所對應的是s frame轉動%
到b frame,因此我們建立%
的$CK(d\vec{\Omega})$矩陣具有以下%
的特性,根據圖\ref{ratevecfig}%
我們可以寫下,%
\begin{eqnarray}
\left( \vec{A}\right) _{s} &=&\underset{\text{passive, l.h.}}{CK(d\vec{\Omega%
})}\times \left( \vec{A}\right) _{b}  \label{frametrans} \\
\left( \hat{b}_{y}\right) _{s} &=&\underset{\text{active, r.h.}}{CK(d\vec{%
\Omega})}\times \left( \hat{S}_{y}\right) _{s}  \label{vecrot}
\end{eqnarray}%
這代表了,若我們知%
道s frame到b frame的轉動角度%
,我們就可以求出t+dt時%
間的xyz軸在t時間xyz軸的%
投影量。若我們知道%
的是s到b frame的瞬時角速%
度$\left( \vec{\omega}\right) _{s}$則可帶入$%
CK(\left( \vec{\omega}\right) _{s}\cdot dt)^{T}$來得到%
轉矩矩陣。以上兩式%
就是模擬或追蹤剛體%
的body frame的x,y,z軸轉動的基%
礎。

\begin{figure}[th]
\caption{How to apply rate-of-change-of-a-vector equation to numerically
simulate a true rotation. Here $S_{x}S_{y}b_{x}b_{y}$ are not shown.}
\label{szsbtdtfig}
\begin{center}
\fbox{\input{SzBz.pgf}}
\end{center}
\end{figure}

\bigskip

上述的微小轉動是只%
考慮t到t+dt時間內的變化%
,現在我們將用遞推%
並且discrete的方式,求出body
frame在實驗者處在的靜止%
座標的變化。因此現%
在我們設定一個真正%
的靜止座標Lab frame\thinspace ,見%
圖\ref{szsbtdtfig},此為真正的%
觀測者所處在的inertial frame%
。考慮任意一段微小%
轉動t到t+dt,在t時刻時%
我們將剛體的principle axes設%
定為S frame,再將t+dt時刻剛%
體的principle axes設定為b frame,這%
樣代表s frame到b frame就是剛%
體t到t+dt的轉動,並且s%
到b frame的瞬時角速度也%
是剛體轉動的瞬時角%
速度。我們重新將\ref{vecrot}%
式寫成%
\begin{equation}
\hat{y}_{s}\left( t+dt\right) =\underset{\text{active, r.h.}}{CK\left( \vec{%
\omega}_{s}\left( t\right) dt\right) }\times \hat{y}_{s}\left( t\right)
\label{vecrot05}
\end{equation}%
$\hat{y}_{s}\left( t\right) $現在為t時間%
剛體特徵軸$\hat{y}$在s frame(也%
是t時刻)的投影,這代%
表$\hat{y}_{s}\left( t\right) $即為單位向%
量$\left[ 
\begin{array}{ccc}
0 & 1 & 0%
\end{array}%
\right] $。現在我們再將上%
式寫成%
\begin{equation}
\hat{y}_{s}\left( t_{i+1}\right) =\underset{\text{active, r.h.}}{CK\left( 
\vec{\omega}_{s}\left( t_{i}\right) \Delta t\right) }\times \left[ 
\begin{array}{ccc}
0 & 1 & 0%
\end{array}%
\right]  \label{vecrot1}
\end{equation}

\begin{figure}[th]
\caption{Boby軸在每一分段t到t+dt%
的追蹤示意圖。}
\begin{center}
\fbox{\input{Zt0Zt1.pgf}}
\end{center}
\end{figure}

\begin{figure}[th]
\caption{陀螺的初始值設定%
。此初始值設定對應%
的尤拉角初始值﹝以$%
z-x^{\prime }-z^{\prime }$為轉軸順序分%
別以右手定則轉動$\protect%
\phi $,$\protect\theta $,$\protect\varphi $﹞為$%
\protect\phi _{0}=0$,$\protect\theta _{0}=-\left\vert \hat{\Omega}%
_{0}\right\vert $,$\protect\varphi _{0}=0$。在Hasbun%
教授的尤拉解法中需%
要此尤拉角初始值,%
來給出與我們方向餘%
弦法相同的初始條件%
。}
\begin{center}
\fbox{\input{orien.pgf}}
\end{center}
\end{figure}

假設陀螺特徵軸在lab frame%
的起始位置已知$\hat{x}\hat{y}\hat{z%
}_{lab}(t_{0})$,初始貼體角速%
度已知$\vec{\omega}_{s}(t_{0})$,$\hat{x}\hat{y}\hat{%
z}$的下標代表的是觀測%
的frame。首先我們將s frame放%
在$\left( \hat{x}\hat{y}\hat{z}_{lab}(t_{0})\right) $,這%
樣依照圖\ref{szsbtdtfig}及其所%
述原理,b frame的軸就是%
我們要求的$\hat{x}\hat{y}\hat{z}_{lab}(t_{1})$%
。我們先看$\hat{z}$軸,\ref%
{vecrot1}式告訴我們%
\begin{equation*}
\hat{z}_{0}\left( t_{1}\right) =\underset{\text{active, r.h.}}{\left[
CK\left( \vec{\omega}_{s}\left( t_{0}\right) dt\right) \right] }\times \hat{z%
}_{0}\left( t_{0}\right) \underset{\text{active, r.h.}}{=\left[ CK\left( 
\vec{\omega}_{s}\left( t_{0}\right) dt\right) \right] }\times \left[ 
\begin{array}{ccc}
0 & 0 & 1%
\end{array}%
\right]
\end{equation*}%
其中$\hat{z}_{0}\left( t_{0})\text{,}\hat{z}%
_{0}(t_{1}\right) $代表時間為$t_{0}$與$%
t_{1}$的\^{z}軸在$t_{0}$時間的座%
標軸(也就是s frame)的投影%
,因此$\hat{z}_{0}\left( t_{0}\right) $為單%
位向量$\left[ 
\begin{array}{ccc}
0 & 0 & 1%
\end{array}%
\right] $。這樣子我們求得%
下一個z軸的位置在$t_{0}$%
的投影,不過我們得%
轉回lab frame,我們假設lab frame%
的$xyz$軸到陀螺初始位置%
$\hat{x}\hat{y}\hat{z}_{lab}(t_{0})$的轉動向量%
是$\vec{\Omega}_{0}, $這樣我們可%
以用$\vec{\Omega}_{0}$輕易的改變%
陀螺初始位置,運用%
上\ref{frametrans}式%
\begin{equation*}
\hat{z}_{lab}\left( t_{1}\right) =\underset{\text{passive, l.h.}}{\left[
CK\left( \vec{\Omega}_{0}\right) \right] }\times \hat{z}_{0}\left(
t_{1}\right)
\end{equation*}%
注意這邊矩陣就取被%
動含意,結合以上兩%
式得到%
\begin{eqnarray*}
\hat{z}_{lab}\left( t_{1}\right) &=&\underset{\text{passive, l.h.}}{\left[
CK\left( \vec{\Omega}_{0}\right) \right] }\underset{\text{active, r.h.}}{%
\left[ CK\left( \vec{\omega}_{s}\left( t_{0}\right) dt\right) \right] }%
\times \hat{z}_{0}\left( t_{0}\right) \\
&=&\underset{\text{passive, l.h.}}{\left[ CK\left( \vec{\Omega}_{0}\right) %
\right] }\underset{\text{active, r.h.}}{\left[ CK\left( \vec{\omega}%
_{s}\left( t_{0}\right) dt\right) \right] }\times \left[ 
\begin{array}{ccc}
0 & 0 & 1%
\end{array}%
\right]
\end{eqnarray*}%
這樣我們就從t$_{0}$時間%
得到t$_{1}$時間陀螺z軸的%
位置。

接著若我們知道$\hat{z}_{lab}\left(
t_{i}\right) $,再從尤拉方程%
數值法解出的$\vec{\omega}_{s}\left(
t_{0},t_{1},\cdots ,t_{i}\right) $(接下來會說%
明),我們同樣可以求%
得$\hat{z}_{lab}\left( t_{i+1}\right) $,首先用%
\ref{vecrot}式%
\begin{eqnarray*}
\hat{z}_{i}\left( t_{i+1}\right) &=&\underset{\text{active, r.h.}}{\left[
CK\left( \vec{\omega}_{s}\left( t_{i}\right) dt\right) \right] }\times \hat{z%
}_{i}\left( t_{i}\right) \\
&=&\underset{\text{active, r.h.}}{\left[ CK\left( \vec{\omega}_{s}\left(
t_{i}\right) dt\right) \right] }\times \left[ 
\begin{array}{ccc}
0 & 0 & 1%
\end{array}%
\right]
\end{eqnarray*}%
再用\ref{frametrans}式轉回到lab frame%
\begin{eqnarray}
\hat{z}_{lab}\left( t_{i+1}\right) &=&\underset{\text{passive, l.h.}}{%
\underbrace{\left[ CK\left( lab\rightarrow t_{i}\right) \right] }}\times 
\hat{z}_{i}\left( t_{i+1}\right)  \notag \\
&=&\underset{\text{passive, l.h.}}{\underbrace{\left[ CK\left( \vec{\Omega}%
_{0}\right) \cdot CK\left( \vec{\omega}_{s}\left( t_{0}\right) dt\right)
\cdot CK\left( \vec{\omega}_{s}\left( t_{1}\right) dt\right) \cdot \cdots
\cdot CK\left( \vec{\omega}_{s}\left( t_{i-1}\right) dt\right) \right] }}%
\times  \notag \\
&&\hat{z}_{i}\left( t_{i+1}\right)  \notag \\
&=&\underset{\text{passive, l.h.}}{\underbrace{\left[ CK\left( \vec{\Omega}%
_{0}\right) \cdot CK\left( \vec{\omega}_{s}\left( t_{0}\right) dt\right)
\cdot CK\left( \vec{\omega}_{s}\left( t_{1}\right) dt\right) \cdot \cdots
\cdot CK\left( \vec{\omega}_{s}\left( t_{i-1}\right) dt\right) \right] }}%
\times  \notag \\
&&\underset{\text{active, r.h.}}{\underbrace{\left[ CK\left( \vec{\omega}%
_{s}\left( t_{i}\right) dt\right) \right] }}\times \left[ 
\begin{array}{ccc}
0 & 0 & 1%
\end{array}%
\right]  \label{iterateeq0}
\end{eqnarray}%
這裡用上不同時間微%
小轉動矩陣的commutive性質$%
\left( AB\right) C=A\left( BC\right) $。這樣我%
們就得到了所有時刻$%
\left( t_{0},t_{1},\cdots ,t_{i+1}\right) $z軸在lab frame位%
置的公式,同樣方法%
可求得x,y軸。可以看出%
上面所有passive的矩陣的%
作用只是再把坐標軸%
從body frame轉回到lab frame。若我%
們再用上轉動矩陣相%
乘即等於轉動向量相%
加的事實\cite{goldstein},上式%
也可寫成%
\begin{equation}
\hat{z}_{lab}\left( t_{i+1}\right) ==\left[ CK\left( \vec{\Omega}%
_{0}+\sum\limits_{m=0}^{i}\vec{\omega}_{s}\left( t_{m}\right) \right) \right]
\times \left[ 
\begin{array}{ccc}
0 & 0 & 1%
\end{array}%
\right]  \label{iterateeq2}
\end{equation}%
不過要注意,只有CK內%
包含的所有轉動向量%
都是要微小轉動向量%
此條公式才會成立。%
﹝在我們的Python程式中%
也包含有使用此公式%
的選項,讀者可以自%
由切換比較。用此公%
式的好處是,矩陣的%
相乘轉換成了矩陣的%
相加,在速度及記憶%
體的佔用上此近似會%
有優勢。不過,程式%
預設是使用\ref{iterateeq0}公式%
,因$\vec{\Omega}_{0}$量值不小。%
﹞另外這邊我們就省%
略了矩陣的主被動左%
右手的標示,但注意%
,我們的CK矩陣是定義%
右手定則,遵守物理%
定律。這裡若不強調%
會產生問題,因為很%
多書上給的是CK是左手%
公式。並且,若要了%
解公式,主被動及左%
右手的標示是必須保%
留並且對照圖表才有%
辦法了解公式如何而%
來以及轉動方向是如%
何,而這也是大部分%
書上所缺乏的。這樣%
,我們就得到了剛體%
特徵軸的追蹤公式,%
並且只要知道"貼體"角%
速度﹝注意並不是角%
速度在lab frame的觀察值,%
當然如果知道這個,%
我就不用寫這麼多了%
。﹞就可以求得。同%
理可得$\hat{x}_{lab}\left( t_{i+1}\right) $,$\hat{y}%
_{lab}\left( t_{i+1}\right) $。

若將$\hat{x}_{lab}\left( t_{i+1}\right) $,$\hat{y}%
_{lab}\left( t_{i+1}\right) $,$\hat{z}_{lab}\left( t_{i+1}\right) $%
的三個行組成一矩陣C%
,我們會發現\ref{iterateeq2}式%
即為導航書籍上常看%
到的方向餘弦遞推理%
論的姿態矩陣動力方%
程\cite{titterton}%
\begin{eqnarray}
C(t_{i+1}) &=&C(t_{i})\exp (\int \omega dt)  \label{iterationC} \\
\dot{C}(t_{i+1}) &=&C(t_{i})\times \omega (t)
\end{eqnarray}%
不過似乎沒有人願意%
花時間把此公式解釋%
清楚,比如說,\ref{iterateeq0}%
式若不經過我們每一%
步轉動矩陣都紀錄並%
寫下主被動與左右手%
性質,實際應用的時%
候我們將不知道是否%
要套用transpose矩陣而胡亂%
套用(很常發生),轉動%
向量的主動還是坐標%
系的被動特性以及轉%
動方向遵守是左手還%
是右手定則,這樣將%
造成實際應用時極大%
的困擾。因此我們知%
道,只給出\ref{iterationC}公式%
只是一小步,離實作%
層面還有非常大一段%
距離。此式中$\exp (\int \omega dt)$%
為path order exponential,關於此,%
以及要如何證明$\exp (\int \omega
dt)$是我們的CK矩陣,可%
以在\cite[Page 49]{tong}找到,並注%
意連結與$\sin $,$\cos $的泰%
勒展開的關係。

\bigskip

以上為轉動坐標系中%
的向量變化量與恆定%
坐標系中的向量變化%
量與轉動坐標系的角%
速度的關係推導,不%
過我們發現我們必須%
要知道所有時刻的貼%
體角速度$\vec{\omega}_{s}\left( t_{0}\cdots
t_{i}\right) $。若是在慣性感%
測器的應用,由於strap-down%
﹝捷聯式or綁附式﹞慣%
性感測器﹝IMU, inertial measurement unit%
﹞測量得的就是貼體%
角速度,因此我們可%
以直接將測量值帶入%
我們的公式,這就是%
捷聯式慣性感測器姿%
態估測演算法的一種%
演算法,也就是我們%
程式的C方法,我們會%
在第四部分說明。另%
一種方法是,我們也%
可以利用尤拉方程以%
遞推的方式數值求出$%
\vec{\omega}_{s}\left( t_{0}\cdots t_{i}\right) $,並且%
每遞推一次就以當前%
的$\vec{\omega}_{s}$更新陀螺的姿%
態,然後再以尤拉方%
程去遞推下一個$\vec{\omega}_{s}$%
,再更新陀螺的姿態%
,反覆遞推就會得到%
所有時刻的運動,此%
法為A方法,接下來我%
們將詳細說明此A方法%
。

\bigskip

將\ref{rateofchange}式應用上一段s%
到b frame,t到t+dt的微小轉動%
,並且考慮$\vec{A}$為剛體%
角動量$\vec{L}$,則我們得%
到%
\begin{equation}
\left( \Gamma \right) _{s}=\left( \frac{d\vec{L}}{dt}\right) _{s}=\left( 
\frac{d\vec{L}}{dt}\right) _{b}+\left( \vec{\omega}\right) _{s}\times \left( 
\vec{L}\right) _{s}  \label{liw}
\end{equation}%
這裡第一等號也用上%
牛頓定律。現在我們%
從\ref{rateofdomega}式知道$\left( d\vec{\Omega}%
\right) _{s}$是s到b frame的角位移,%
而經由我們之前剛體%
特徵軸的設定,s到b frame%
正是我們剛體特徵軸%
從t到t+dt的角位移,因此%
$\left( \frac{d\vec{\Omega}}{dt}\right) _{s}=\left( \vec{\omega}\right) _{s}$%
正是剛體的瞬時角速%
度﹝沿著t時間s frame取分%
量﹞,接著,因為s, b frame%
都是沿著body principle axes而取,%
因此沿s frame的角動量$\left( \vec{L%
}\right) _{s}$可以寫成%
\begin{equation*}
\left( \vec{L}\right) _{s}=\left[ 
\begin{array}{ccc}
I_{xx} & 0 & 0 \\ 
0 & I_{yy} & 0 \\ 
0 & 0 & I_{zz}%
\end{array}%
\right] \times \left( \vec{\omega}\right) _{s}
\end{equation*}%
並且沿著b frame的$\left( \frac{d\vec{L}}{dt}%
\right) _{b}$項也可寫成%
\begin{equation*}
\left( \frac{d\vec{L}}{dt}\right) _{b}=\left[ 
\begin{array}{ccc}
I_{xx} & 0 & 0 \\ 
0 & I_{yy} & 0 \\ 
0 & 0 & I_{zz}%
\end{array}%
\right] \times \frac{d\left( \vec{\omega}\right) _{b}}{dt}
\end{equation*}%
其中$I_{xx}$、$I_{yy}$、$I_{zz}$都不%
隨時間變動。再次注%
意$\left( \Gamma \right) _{s}$與$\left( \vec{\omega}\right) _{s}$%
與$\left( \vec{L}\right) _{s}$都是沿著t時%
刻的剛體特徵軸(也就%
是s frame)取的投影,並不%
是Lab frame的投影,這點要%
特別注意,基本上這%
代表,$\left( \Gamma \right) _{s}$是貼體%
的力矩,這裡也給出%
了文章開頭問題的解%
答!這裡大部分的書%
上都沒有給出恰當的%
原因,就連Goldstein在書中%
也只有說向量沿著任%
意座標取分量都可以%
,不過這並無法合理%
解釋,事實上是,我%
們假設了t時間s與b frame重%
和﹝Goldsten的證明也是採%
用此假設\cite{goldstein},我這%
裡只是將Goldstein上面的證%
明用淺顯的線性代數%
重新表達。﹞,見\ref{roc1}%
式,因此s frame是隨時間%
一直變動。如果用lab frame%
的$\Gamma $那麼就無法成功%
的數值化喔\footnote{注意因%
為s frame會持續的改變所%
以$\left( \vec{\Gamma}\right) _{s}$不可取$\left( 
\vec{\Gamma}\right) _{lab}$的值,同理$\left( 
\vec{\omega}\right) _{s}$也不是$\left( \vec{\omega}\right)
_{lab}$,兩者都必須經過%
轉換從lab轉到t時刻s frame。}%
。這邊我們證明了\ref{liw}%
式最後那一項中的兩%
個$\vec{\omega}$是相同的\footnote{但%
我們必須強調,任意%
情況下,角速度$\left( \vec{\omega}%
\right) $在body轉動座標下的投%
影並不是body座標上觀察%
到的角速度!這是很%
常見的錯誤,這裡我%
們是有條件的考慮t到t+dt%
時刻的t時刻s,b座標重和%
。}。$\frac{d\left( \vec{\omega}\right) _{b}}{dt}$即%
是$\frac{d\vec{\omega}_{b}(t+dt)-d\vec{\omega}_{s}(t)}{dt}$,%
由於s與b frame都是剛體特%
徵軸,都是貼體座標%
軸,只是在不同時間%
,以此展開\ref{liw}式,我%
們就得到所謂的牛頓%
尤拉公式(Newton-Euler's equation),或%
簡稱尤拉公式。%
\begin{eqnarray}
\Gamma _{x}(t) &=&I_{x}\dot{\omega}_{x}+(I_{z}-I_{y})\omega _{y}\left(
t\right) \omega _{z}\left( t\right)  \notag \\
\Gamma _{y}(t) &=&I_{y}\dot{\omega}_{y}+(I_{x}-I_{z})\omega _{x}\omega _{z}
\label{eulereqbody} \\
\Gamma _{z}(t) &=&I_{z}\dot{\omega}_{z}+(I_{x}-I_{y})\omega _{x}\omega _{y} 
\notag
\end{eqnarray}%
其中$\vec{\omega}=\left[ 
\begin{array}{ccc}
\omega _{x} & \omega _{y} & \omega _{z}%
\end{array}%
\right] $。注意$\vec{\Gamma}$及$\vec{\omega}$的%
x,y,z分量都是沿著t時刻%
的剛體特徵軸s frame取的%
分量,這點必須要再%
次強調。之後數值模%
擬的時候這點是非常%
重要的。基本上可以%
說,尤拉方程是我們%
假設\ref{roc1}式後所作的一%
種線性代數變換近似%
。

\bigskip

接下來應用上陀螺,%
若考慮陀螺的條件 $%
I_{x}=I_{y}\neq I_{z}$,\ref{eulereqbody}式可寫%
成%
\begin{equation}
\frac{d}{dt}\left[ 
\begin{array}{c}
\omega _{x} \\ 
\omega _{y} \\ 
\omega _{z}%
\end{array}%
\right] =\left[ 
\begin{array}{ccc}
0 & -\frac{I_{z}-I_{y}}{I_{x}}\omega _{z} & 0 \\ 
-\frac{I_{x}-I_{z}}{I_{y}}\omega _{z} & 0 & 0 \\ 
0 & 0 & 0%
\end{array}%
\right] \left[ 
\begin{array}{c}
\omega _{x} \\ 
\omega _{y} \\ 
\omega _{z}%
\end{array}%
\right] +\left[ 
\begin{array}{c}
\frac{\Gamma _{x}}{I_{x}} \\ 
\frac{\Gamma _{y}}{I_{y}} \\ 
\frac{\Gamma _{z}}{I_{z}}%
\end{array}%
\right] 
\end{equation}%
如之前所強調,右邊%
所有項都是在時間為t%
時刻的s frame取得值,也%
因此以上的微分方程%
組可以用普通數值由%
拉法或四階Ruge Kutta求出左%
側$\vec{\omega}_{b}(t+dt)$,也就是從$%
\vec{\omega}_{s}(t)$求得$\vec{\omega}_{b}(t+dt)$,而$%
\vec{\omega}_{b}(t+dt)$就是下一時間%
的$\vec{\omega}_{s}(t)$。不過對於任%
意的剛體轉動系統,%
只要能從\ref{eulereqbody}式右側$%
\vec{\omega}_{s}(t)$求得左側$\vec{\omega}_{b}(t+dt)$%
,都還是能夠適用接%
下來的模擬方法,也%
因此這裡描述的方法%
是具有任意一般性的%
,可以應用在任何的%
剛體轉動,包括asymmetric rotor%
的情況。有不少的數%
值方法可以解一般的%
非線性一階ODE尤拉方程%
\cite{matlab}。這裡我以Ruge Kutta四%
階法求解上式,來得%
到$\vec{\omega}_{s}\left( t_{0}\cdots t_{i}\right) $,並%
且寫成python程式,程式%
將在下一章介紹。這%
裡所使用的方法與原%
子分子運動模擬中的%
方法雷同,可以參考\cite%
{rapaport}。

\begin{figure}[th]
\caption{陀螺的對稱軸的單%
位向量的模擬軌跡圖%
。以$\protect\omega (t_{i})$或$\protect\omega (t_{i+1})$%
轉動向量來近似$t_{i}$到$%
t_{i+1}$時間的轉動的模擬%
結果,並與Hasbun教授的%
數值解B法比較。$\protect\omega %
(t_{i+1})$的結果與Hasbun教授的%
結果在此圖形中幾乎%
重疊,這代表以$\protect\omega %
(t_{i+1})$來近似比$\protect\omega (t_{i})$好%
很多。}
\label{wtiwti1}
\begin{center}
\fbox{\input{wti_wtiplus1.pgf}}
\end{center}
\end{figure}

\section{以$t_{i+1}$時間的轉動向%
量來近似$t_{i}$到$t_{i+1}$的轉%
動}

\bigskip

以上\ref{vecrot05}式中以$CK\left( \vec{\omega}%
_{s}(t_{i})dt\right) $來近似t$_{i}$到t$_{i+1}$%
的轉動事實上還不夠%
好,圖\ref{wtiwti1}顯示,以$\omega
(t_{i})$轉動向量近似的結%
果與Hasbun的正確結果偏%
差不少。這邊我提出%
以$CK\left( \vec{\omega}_{s}(t_{i+1})dt\right) $來近%
似t$_{i}$到t$_{i+1}$的轉動,圖\ref%
{wtiwti1}顯示模擬結果幾乎%
與Hasbun的正確解重合,%
至少此圖表上無法看%
出任何差別。為什麼%
有如此大的影響,以%
下我也嘗試提供物理%
解釋。這裡我們暫時%
假設$\vec{\omega}_{s}(t_{i})dt=\vec{\Omega}_{s}(t_{i})$,%
我們知道轉動向量在$%
t_{i+1}$跟$t_{i}$時刻在body frame中的%
向量值一般不會一樣%
,也就是$\vec{\Omega}_{i+1}(t_{i+1})\neq \vec{\Omega}%
_{i}(t_{i})$,這代表從$t_{i}$到$t_{i+1}$%
時,轉動向量在body座標%
上有變化,也因此我%
們不能夠單只考慮陀%
螺轉了$\vec{\Omega}_{s}(t_{i})$而已,%
此額外轉動向量的變%
化在$t_{i}$時s frame的向量值%
為$\Omega _{i+1}(t_{i+1})-\Omega _{i}(t_{i})=\Omega
_{i}(t_{i})+d\Omega _{i}(dt)-\Omega _{i}(t_{i})=d\Omega _{s}(dt)$,%
也是一個轉動向量,%
所以space空間中總共的轉%
動可以考慮成兩步,%
第一步轉$\Omega _{s}(t_{i})$,第二%
步轉$d\Omega _{s}(dt)$,寫成轉動%
矩陣%
\begin{equation}
CK(\Omega _{s}(t_{i}))\times CK(d\Omega _{s}(dt))=CK(\Omega
_{s}(t_{i})+d\Omega _{s}(dt))=CK(\Omega _{i+1}(t_{i+1}))
\end{equation}%
這代表我們只要考慮%
陀螺從t到t+dt的時候是轉%
了$\Omega _{s}(t+dt)$而不只是$\Omega _{s}(t)$%
,因此考慮$\Omega _{s}(t+dt)$我們%
就更準確的近似了這%
個轉動,以下的Python程%
式模擬會證明,考慮%
了$\Omega _{s}(t+dt)$給出的結果比$%
\Omega _{s}(t)$好非常多。若如%
此考慮則\ref{iterateeq0}式變成%
\begin{eqnarray}
\hat{z}_{lab}\left( t_{i+1}\right) &=&\left[ CK\left( \vec{\Omega}%
_{0}\right) \cdot CK\left( \vec{\omega}_{s}\left( t_{1}\right) dt\right)
\cdot CK\left( \vec{\omega}_{s}\left( t_{2}\right) dt\right) \cdot \cdots
\cdot CK\left( \vec{\omega}_{s}\left( t_{i+1}\right) dt\right) \right] \times
\notag \\
&&\left[ 
\begin{array}{ccc}
0 & 0 & 1%
\end{array}%
\right]  \label{iterationCK}
\end{eqnarray}%
此公式即為我Python程式%
中使用的DCM遞推的公式%
。相同方法可求得另%
外兩軸x,y的運動。

\bigskip

使用$\omega (t+dt)$來做t時間的%
轉動近似讓我們的程%
式碼變得非常簡單,%
比書中\cite[Page 301, Equation 10.24]{titterton}給%
出的轉動向量近似公%
式簡單太多了。簡單%
的程式碼代表之後在%
做速度優化,或轉寫%
成速度快的編譯式語%
言上,會簡單非常多%
。若需要更精確的近%
似,書\cite[Page 301, Equation 10.24]{titterton}中%
也給出轉動向量的近%
似公式,不過此公式%
頗為複雜,因此這邊%
先嘗試比較以我們$\vec{\omega}%
_{s}(t_{i+1})$方法與解析解的%
不同。下面就將上述%
方法寫成python程式,以%
尤拉方程數值解出貼%
體角速度,接著用遞%
推公式\ref{iterationCK}畫圖模擬%
其xyz軸運動。這邊劃出%
四種陀螺的經典運動%
,其模擬參數請參考%
下一章節的參數預設%
值。程式產生的3D動畫%
的演示請參考以下連%
結。

\begin{center}
\href{http://tinypic.com/r/wk20ch/8}{\underline{\color{blue}%
\smash{3D
animation: http://tinypic.com/r/wk20ch/8}}}
\end{center}

\begin{figure}[th]
\caption{{}左到右: (a) 尖點 (b) 環%
狀 (c) 波紋 (d) 正圓運動。}
\label{FourClassics}
\begin{center}
\fbox{\scalebox{1.2}[1.2]{\input{FourClassics.pgf}}}
\end{center}
\end{figure}

%TCIMACRO{\TeXButton{clearpage}{\clearpage}}%
%BeginExpansion
\clearpage%
%EndExpansion

\part{方向餘弦演算法Python程%
式碼之說明}

\setcounter{page}{1}\bigskip

\bigskip

程式經過幾次升級修%
改後,目前已經以物%
件導向方式編寫完成%
。主要是環繞著一個%
類別﹝class﹞名為RigidBodyObject來%
發展。此類別的宣告%
及其方法﹝演算法的%
主程式﹞被包在一個%
套件module裡,此套件為%
RGCordTransVXX.py檔,使用前先import
RGCordTransVXX,XX代表目前的版%
本。動畫及圖表產生%
的程式是整合在RBPlotFuncVXX.py%
此module檔中,因此也要import%
此module。我使用的Python版本%
為Python 2.7.5 - Anaconda 1.8.0 (32-bit),NumPy 1.7.1,%
SciPy 0.13.0,Matplotlib 1.3.1。不同版本%
的Python﹝如Python 3.x﹞會有一%
些指令的不同可能要%
做小修改,可以聯絡%
我我可以幫忙做修正%
。\texttt{使用者可可調整%
的}參數列在文件最後%
的程式使用說明書中%
。說明書名稱是GyroSoft Simulation
Documentation。以下為如何使用%
程式的一個最簡單範%
例,以及其輸出的陀%
螺動畫,我們將此範%
例存成一個Gyroscope-Demo.py檔,%
使用的時候只要執行%
此檔就可以,裡面的%
內容如下:

\bigskip

\definecolor{bg}{rgb}{0.95,0.95,0.95}%
%TCIMACRO{%
%\TeXButton{code example}{\begin{mdframed}[leftline=false, rightline=false,backgroundcolor=bg]
%\inputminted[linenos,fontsize=\footnotesize]{python}{../../Scripts/cordtrans/Gyroscope-Demo-LatexImport.py}
%\end{mdframed}}}%
%BeginExpansion
\begin{mdframed}[leftline=false, rightline=false,backgroundcolor=bg]
\inputminted[linenos,fontsize=\footnotesize]{python}{../../Scripts/cordtrans/Gyroscope-Demo-LatexImport.py}
\end{mdframed}%
%EndExpansion
\bigskip

模擬程式畫出剛體特%
徵軸的x, y, z軸單位向量%
,分別為藍紅綠線段%
,我們以一個正方形%
來代表剛體,正方形%
的中心點定在z軸的二%
分之一處,由於Python的%
Matplotlib三維繪圖程式庫mplot3d%
並沒有太多的3D繪圖支%
援,因此這邊只是以%
一個簡單的示意的正%
方形來代表陀螺,之%
後我也考慮將模擬結%
果輸出至專門做3D rendering的%
繪圖程式語言,如openGL%
,來做更精美的陀螺%
外觀。z軸單位向量的%
頂點對時間的軌跡圖%
,也就是locus,是以藍色%
的曲線表示。Hasbun教授%
的尤拉角法給出的z軸%
單位向量的locus,是以黑%
色的曲線表示。另外%
剛體的總角動量的單%
位向量以黑色線段表%
示,剛體的角速度除%
以角速度初始值量值%
的normalized向量以綠色線段%
表示。

\begin{figure}[th]
\caption{模擬條件(SI units): I=0.002; Is=0.0008;
g=9.8; M=1 ; arm=.04; spin freq= 20 Hz; Initial angle from vertical 54.57
degree; Sampling rate 2000 Hz.}
\begin{center}
\fbox{\input{AnimationFig.pgf}}
\end{center}
\end{figure}

\bigskip

如之前所提到,程式%
有三種不同的模擬陀%
螺運動的方法,可以%
互相比較互相驗證。%
這三種方法各自有其%
應用領域,我們會詳%
細解釋。這三種方法%
的流程圖我們整理在%
圖\ref{ABCmethodsIllustration}中,這樣可%
以幫助我們清楚的了%
解其差異性。

\begin{figure}[th]
\caption{Python程式所包含的ABC三%
種姿態解算法示意圖%
。}
\label{ABCmethodsIllustration}
\begin{center}
%\documentclass[12pt,twoside]{article}
%\usepackage[inner=1in,outer=0.6in,top=0.7in,bottom=1in]{geometry}
%\usepackage{xeCJK}
%\setmainfont{Times New Roman}
%\setsansfont{Verdana}
%\setmonofont{Courier New}                    % tt
%\setCJKmainfont{微軟正黑體}
%\setCJKfamilyfont{kai}{標楷體}		% for changing the title font in title.pgf -> have to manually 
%\usepackage{pgf}
%\usepackage{pstricks,pst-node}
%\begin{document}
\begin{pspicture}(0,12cm)(6.9in,9.3in)
%\psgrid
\footnotesize

\rput(3.2in,9in){
\rnode{A}{
\psframebox[fillcolor=white,fillstyle=solid,framearc=0.3]{
\parbox{2.5cm}{\centering 貼體角速度尤拉方程﹝非線性﹞}}}}

\rput(3.2in,8in){
%\nput{-90}{A}{
\rnode{B}{
\psframebox[fillcolor=white,fillstyle=solid,framearc=0.3]{
\parbox{3cm}{\centering 尤拉角尤拉方程﹝高度非線性﹞}}}}

\ncline[nodesep=3pt]{->}{A}{B} \trput{代入尤拉角}

\rput(2.9,20.5){
\rnode{C}{
\psframebox[fillcolor=white,fillstyle=solid,framearc=0.3]{
\parbox{5cm}{
\centering 給定初始貼角以及條件,直接ODE數值解貼角尤拉方程,得下一貼角後,update姿態DCM矩陣,得到新的姿態,再重複以上步驟。}}}}

\rput[Bl](0.2,22){\psscalebox{3}{A}}

\nccurve[angleA=180,angleB=90]{->}{A}{C} \nbput[npos=0.5]{ODE solver}

\rput(3.2in,7in){
\rnode{D}{
\psframebox[fillcolor=white,fillstyle=solid,framearc=0.3]{
\parbox{2.6cm}{\centering 尤拉角數值解 $\phi,\theta,\psi,\dot{\phi},\dot{\theta},\dot{\psi}$}}}}

\rput[Bl](6.1,21.1){\psscalebox{3}{B}}

\rput(3.2in,6.3in){
\rnode{D1}{
\psframebox[fillcolor=white,fillstyle=solid,framearc=0.3]{
\parbox{3cm}{\centering 得到貼體xyz軸向量在空間中隨時間的變化}}}}

\ncline[nodesep=3pt]{->}{B}{D} \trput{\parbox{1.2cm}{ODE solver}}
\ncline[nodesep=3pt]{->}{D}{D1}
\ncbox[boxsize=2.1cm,nodesep=5pt,linearc=0.2,linestyle=dashed]{B}{D1}

\rput(14,21.5){
\rnode{E}{
\psframebox[fillcolor=white,fillstyle=solid,framearc=0.3]{
\parbox{5cm}{\centering  轉成貼體角速度$\omega_{b}(t)$}}}}

\rput[Bl](11.1,22.1){\psscalebox{3}{C}}

%\ncline[nodesep=3pt]{->}{D}{E} \naput{\parbox{2cm}{轉body frame計算貼體角速度解}}
\nccurve[angleA=0,angleB=180]{->}{D}{E}

\rput(14,19.3){
\rnode{E1}{
\psframebox[fillcolor=white,fillstyle=solid,framearc=0.1]{
\parbox{6cm}{\centering 這裡的貼角雖然有尤拉角尤拉公式高度非線性的數值誤差,但因角速度$\omega_{b}(t_{i})$並沒有用上上一個姿態來計算,因此此角速度算是尤拉方程的正確解,假設ODE solver夠精確的話,因此$\omega_{b}$誤差並不不會累積。}}}}

\rput(14,17.2){
\rnode{E2}{
\psframebox[fillcolor=white,fillstyle=solid,framearc=0.3]{
\parbox{6cm}{\centering  以$\omega_{b}$做姿態矩陣DCM積分。}}}}

\rput(14,15.8){
\rnode{E3}{
\psframebox[fillcolor=white,fillstyle=solid,framearc=0.3]{
\parbox{6cm}{\centering  得到貼體xyz軸向量在空間中隨時間的變化}}}}

\ncline[nodesep=3pt]{->}{E}{E1}
\ncline[nodesep=3pt]{->}{E1}{E2}
\ncline[nodesep=3pt]{->}{E2}{E3}

\ncbox[boxsize=3.2cm,nodesep=5pt,linearc=0.2,linestyle=dashed]{E}{E3} 

\rput(12,13.5){
\rnode{F}{
\psframebox[fillcolor=white,fillstyle=solid,framearc=0.1]{
\parbox{5cm}{\centering 比較兩種方法給出的貼體(1,1,1)向量之間的角度差異,並討論在四種經典陀螺章進動中此差異有何變化。}}}}

\ncdiagg[nodesep=3pt,angleA=-90,angleB=45]{->}{E3}{F}
\ncdiagg[nodesep=3pt,angleA=-90,angleB=135]{->}{D1}{F}

\rput(2.9,18){
\rnode{C1}{
\psframebox[fillcolor=white,fillstyle=solid,framearc=0.3]{
\parbox{5cm}{
\centering 這裡的姿態誤差會因為用上上一個姿態位置來計算下一個貼角而造成$\omega_{b}$誤差的累積。}}}}

\rput(2.9,16){
\rnode{C2}{
\psframebox[fillcolor=white,fillstyle=solid,framearc=0.3]{
\parbox{5cm}{
\centering 得到貼體xyz軸向量在空間中隨時間的變化}}}}

\ncline[nodesep=3pt]{->}{C}{C1}
\ncline[nodesep=3pt]{->}{C1}{C2}
\ncbox[boxsize=2.85cm,nodesep=5pt,linearc=0.2,linestyle=dashed]{C}{C2}

\rput(5,13.5){
\rnode{G}{
\psframebox[fillcolor=white,fillstyle=solid,framearc=0.1]{
\parbox{5cm}{\centering 比較兩種方法給出的貼體(1,1,1)向量之間的角度差異,並討論在四種經典陀螺章進動中此差異有何變化。}}}}

\ncdiagg[nodesep=3pt,angleA=-90,angleB=45]{->}{D1}{G} %\trput{+} \tlput{-}
\ncdiagg[nodesep=3pt,angleA=-90,angleB=135]{->}{C2}{G} 

%\rput(14,4){
%\rnode{I}{
%\psframebox[fillcolor=white,fillstyle=solid]{
%test}}}


\rput[Br](16,12){Diagram created with PSTricks}

%\uput{5}[-90](14,4){
%\rnode{N}{
%\psframebox[fillcolor=white,fillstyle=solid,framearc=0.3]{
%適合中長時間姿態粗估}}}

%\rput(8.5,6){
%\rnode{J}{
%\psframebox[fillcolor=white,fillstyle=solid]{
%\scalebox{0.95}[0.95]{
%\input{compareDCMEulerAngle.pgf}
%}
%}}}
%
%\ncline[nodesep=3pt]{->}{G}{J} \nbput[npos=0.5]{blue}
%\ncline[nodesep=3pt]{->}{F}{J} \nbput[npos=0.5]{red}


\end{pspicture}


%\end{document}
\end{center}
\end{figure}

\bigskip

首先A方法為,數值解%
轉動座標的貼體角速%
度的尤拉方程﹝牛頓%
尤拉方程﹞,從初始%
值得到下一個貼體角%
速度,並以此貼體角%
速度依照第一章節推%
導的方向餘弦遞推公%
式\ref{iterationCK}求得下一個姿%
態,再以此姿態回到%
貼體尤拉方程解出下%
一個貼角,也就是重%
複以上步驟。這樣可%
以得到剛體xyz軸的運動%
﹝這邊我們是討論應%
用於陀螺的運動﹞。%
此方法我們姑且稱為%
\emph{力矩已知的貼體尤%
拉方程的姿態遞推},%
因為每一步遞推解貼%
體角速度時都需要知%
道貼體尤拉方程中的%
力矩項﹝這裡也呼應%
第一章我們所強調'貼%
體'力矩的證明是有其%
重要性﹞,而陀螺的%
力矩是我們已知的重%
力而來。由於此A方法%
是一種轉動向量的堆%
疊近似,因此會有誤%
差,而我們想要知道%
這誤差有多大,因此%
我們會跟以下的B方法-%
也就是傳統的數值解%
尤拉角尤拉方程﹝代%
入尤拉角後的牛頓尤%
拉方程﹞做比較。A方%
法以四階Runge Kutta直接數值%
解陀螺尤拉方程﹝%
topEOM(wi,torquei)\texttt{﹞}得到貼體角%
速度,接著用遞推公%
式\ref{iterateeq0}來得到陀螺特%
徵軸xyz軸的單位向量cordvec%
對時間的變化。cordvec陣%
列的維度shape=(N,3,7)\texttt{。其中%
維度為}7\texttt{的方向的首%
要三個陣列就是}xyz\texttt{軸%
。}DCM遞推的程式在DCMiter方%
程裡。

\bigskip

B方法,\emph{數值解尤拉%
角尤拉方程},就是將%
牛頓尤拉方程中的貼%
體角速度代換成尤拉%
角,然後進行ODE數值解%
。由於牛頓尤拉方程%
本身就呈現非線性,%
代入尤拉角後得到的%
微分方程組更是呈現%
高度非線性,因此取%
樣時間需要縮短來防%
止數值解發散。並且%
,解尤拉角尤拉方程%
組在$\theta =0^{\circ }$時因為會除%
以$\sin (\theta )$,因此在此角%
度附近會發散產生NaN的%
值。因此使用B方法時%
必須要遠離這個限制%
。不過,若ODE的取樣時%
間夠小﹝當計算量大%
,計算時間長並不是%
問題時﹞或誤差有作%
控制,此方法算是一%
種較精確的數值法,%
也因此我們將比較AB兩%
種方法給出的結果。

\bigskip

值得一提的是,B方法%
與使用Lagrange力學所得到%
的方程是可以reduce成一%
樣的微分方程組,所%
以基本上也可以說我%
們的B方法也是Lagrange法,%
關於如何證明兩個方%
法是一樣的,只要查%
詢Goldstein書上\cite[Page 216]{goldstein}的Equation
5-62與Eberly書上\cite[Page 158]{eberly}的Equation
3.50最終得到的是一樣的%
微分方程組即可得知%
。

\bigskip

B方法的程式碼來自Hasbun%
教授的著作\cite{hasbun},不過%
他的程式碼是Matlab寫成%
,因此為了能跟我的%
python程式碼比較,我將%
他的程式重寫成python程%
式碼,此方程叫做%
HasbunEulerEquationODEsolve,列在最後的%
程式說明書中。補充%
一提,Hasbun教授所採用%
的尤拉角定義與我的%
相同,都是依照z-x'-z'軸%
依右手定則轉$\phi -\theta -\varphi $%
角,也與Goldstein書上相同%
,因此比較上會比較%
方便,不需要做轉換%
。Hasbun教授的程式碼是%
尤拉角所寫出來的,%
所以會有奇異點問題%
,$\theta $在零度時會產生NaN%
。

\bigskip

A方法可以適用於任意%
的陀螺狀態,因此不%
會遇到以尤拉角表達%
的方法所遇到在九十%
度角所遭遇的奇異點%
問題。不過,B方法只%
要ODE數值解收斂,結果%
會是比較正確的,因%
為A方法的姿態方向餘%
弦遞推是一種空間轉%
動向量的堆疊近似,%
因此會有誤差,而B方%
法不會有此問題。當%
然,A方法適用任意的%
剛體轉動方程,可直%
接套用,而B的尤拉角%
法還得要人為的將力%
矩項以尤拉角表達,%
之後才能進行數值解%
,因此要看力矩項的%
複雜程度,因此各有%
其適用場合。

\bigskip

第三種C方法是,在我%
們從B方法得到尤拉角%
隨時間的數值解後,%
我們可以將尤拉角數%
值解轉到body frame計算出貼%
體角速度的數值解,%
這樣我們會得到一組$%
\omega _{b}(t_{i})$,$i=1\symbol{126}N$,接著我%
們可以用方向餘弦公%
式\ref{iterateeq0}一個一個把$t_{0}%
\symbol{126}t_{N}$的姿態求出來﹝%
或者說將$\omega _{b}(t_{i})dt$一個%
一個堆疊起來﹞。此%
方法適用的情形是,%
若只知道貼體角速度%
,力矩項未知,如飛%
機的姿態感測器,此%
時此方法是唯一可以%
估測物體姿態的方法%
。此法我們也說是以%
貼體角速度直接做姿%
態方向餘弦矩陣遞推%
法。這裡要注意與A方%
法不同的是,A方法有%
利用貼體尤拉方程以%
上一個姿態來計算下%
一個貼角,在以下一%
個貼角去求得新的姿%
態。因此A方法只要知%
道初始角速度,與外%
力力矩項的型式。C方%
法是只需要貼體角速%
度,不管是從量測得%
來,或者是從理論得%
來,C方法只負責姿態%
矩陣遞推。若加上噪%
音過濾,C方法即為角%
速度感測器解算姿態%
的公式,因此適用在%
感測器的姿態估測上%
面。因此我們這邊與B%
方法的比較也提供了%
一個平台,來驗證姿%
態估測演算的正確性%
與準確性。我們在下%
一章還會繼續討論姿%
態估測演算法。

\bigskip

我們將在範例與分析%
中詳細討論三種方法%
在四種經典運動中的%
表現,以及相互比較%
。

\section{A、C方法的理論驗證%
與第三方程式碼的比%
較}

與我們A方法非常接近%
的是Rapaport書上\cite[Page 232]{rapaport}的%
方向餘弦法,讀者只%
要留心比對即可觀察%
到兩者皆是微小轉動%
向量的應用,書上也%
有介紹另一種四元數%
法,我將會把我們的A%
方法的程式模擬結果%
﹝陀螺章動進動﹞與%
另一位Christian教授以四元%
數法寫成的完整的程%
式easy javascript simulation﹝同樣是陀%
螺的章進動模擬﹞作%
直接的系統性比較。(%
不過還有一個問題待%
解決:還不知道如何%
將Christian教授的Javascript運動結%
果作輸出。)

%TCIMACRO{\TeXButton{clearpage}{\clearpage}}%
%BeginExpansion
\clearpage%
%EndExpansion

\part{範例與分析}

\setcounter{page}{1}

以下的範例討論若是%
遇到我們沒有提到的%
參數,代表都是以預%
設值代入,使用者不%
須特別去改變,程式%
都有內建參數預設值%
。我們將參數的預設%
值列在table中。

\begin{equation*}
\text{%
%TCIMACRO{%
%\TeXButton{default parameter table}{\documentclass[12pt]{standalone}
%
%\documentclass{article}
%\usepackage{fontspec}
\usepackage{xeCJK}
\setmainfont{Times New Roman}
\setsansfont{Verdana}
\setmonofont{Courier New}
\setCJKmainfont{微軟正黑體}
%
\begin{document}

%\begin{table}[hb]
%\centering
\footnotesize
\begin{tabular}{| p{0.4\textwidth} | p{0.5\textwidth} |}

  %\multicolumn{2}{|c|}{RGCordTransV11.py} \\
   \hline
   parameter & description \\%\\ \cline{2-2}
   \hline
    M = 0.3 & mass of top in kg \\
    R = 0.025&default top is a disk with radius R in meters, unless otherwise specified in moment of inertia below\\
    L = 0.005&width of disk in m\\
    arm = 0.01&location of center of mass of top from origin in meters\\
    counter weight = 0.05 & mass that doesn't spin along symmetry axis, e.g. the gimble support part\\
    counter weight location from origin = 0.1&location of counter weight from origin\\
    Ix,Iy,Iz & one can set moment of inertia to overwrite the default. The default is calculated from above disk's parameters, see document\\    
    g = 9.8&gravity constant $m/s^2$\\
    freq = 20&top revolution speed in hertz, along symmetric axis\\
    tn=1.3&end of simulation time\\
    t0=0.0&start of time zero\\
    samplerate = 2000&rate of iteration in Hz\\
    classical case = 1&selection of four typical nutation and precession motions: 1,2,3,4\\
    orien = np.array([-np.pi/3,0,0])&starting orientation vector of top from lab xyz\\
   \hline
   
\end{tabular}
%\end{table}


\end{document}
}}%
%BeginExpansion
\documentclass[12pt]{standalone}
%
%\documentclass{article}
%\usepackage{fontspec}
\usepackage{xeCJK}
\setmainfont{Times New Roman}
\setsansfont{Verdana}
\setmonofont{Courier New}
\setCJKmainfont{微軟正黑體}
%
\begin{document}

%\begin{table}[hb]
%\centering
\footnotesize
\begin{tabular}{| p{0.4\textwidth} | p{0.5\textwidth} |}

  %\multicolumn{2}{|c|}{RGCordTransV11.py} \\
   \hline
   parameter & description \\%\\ \cline{2-2}
   \hline
    M = 0.3 & mass of top in kg \\
    R = 0.025&default top is a disk with radius R in meters, unless otherwise specified in moment of inertia below\\
    L = 0.005&width of disk in m\\
    arm = 0.01&location of center of mass of top from origin in meters\\
    counter weight = 0.05 & mass that doesn't spin along symmetry axis, e.g. the gimble support part\\
    counter weight location from origin = 0.1&location of counter weight from origin\\
    Ix,Iy,Iz & one can set moment of inertia to overwrite the default. The default is calculated from above disk's parameters, see document\\    
    g = 9.8&gravity constant $m/s^2$\\
    freq = 20&top revolution speed in hertz, along symmetric axis\\
    tn=1.3&end of simulation time\\
    t0=0.0&start of time zero\\
    samplerate = 2000&rate of iteration in Hz\\
    classical case = 1&selection of four typical nutation and precession motions: 1,2,3,4\\
    orien = np.array([-np.pi/3,0,0])&starting orientation vector of top from lab xyz\\
   \hline
   
\end{tabular}
%\end{table}


\end{document}
%
%EndExpansion
}
\end{equation*}

\begin{case}
方法A在尖點運動情況%
下,轉動頻率為1 Hz,$\theta
=1^{\circ }$情況下的陀螺動畫%
。
\end{case}

此範例是一個簡單的%
驗證若陀螺在幾乎不%
旋轉的情況﹝轉動速%
度只有1 Hz﹞,此情形相%
當於陀螺只受重力影%
響而有倒下的運動,%
因為我們的運動是只%
有假設陀螺與圓點的%
接觸點是固定不動的%
,我們並沒有地面來%
限制陀螺的運動,因%
此陀螺會如同單擺一%
樣,擺動到最低點後%
,再回復到原本的高%
度。

%TCIMACRO{%
%\TeXButton{case f0 1deg}{\begin{center}
%%\documentclass[12pt,twoside]{article}
%\usepackage[inner=1in,outer=0.6in,top=0.7in,bottom=1in]{geometry}
%\usepackage{pstricks,pst-3dplot}
%\usepackage{minted}
%\usepackage{mdframed}
%\begin{document}
%\definecolor{bg}{rgb}{0.95,0.95,0.95}
\psset{unit =1in}
\begin{pspicture}[showgrid=false](-1,-1)(1,1)
\psset{Alpha=90,Beta=20}
%\pstThreeDCoor[xMax=1,yMax=1,zMax=1]
\pstThreeDSphere[opacity=0.4,strokeopacity=0.2,linewidth=0.8pt](0,0,0){1}
\fileplotThreeD{./otherstuff/data_text_files/Case_A_f0_1deg_datafile.txt}%./otherstuff/

%A axes xyz
\fileplotThreeD[linewidth=2pt,linecolor=blue]{./otherstuff/data_text_files/Case_A_f0_1deg_square_datafile.txt}%./otherstuff/



\pstThreeDLine[linewidth=2pt,%
 linecolor=blue,arrows=->](0,0,0)(0,0.01745241,0.9998477 )

\end{pspicture}


%\end{document}

%\end{center}}}%
%BeginExpansion
\begin{center}
%\documentclass[12pt,twoside]{article}
%\usepackage[inner=1in,outer=0.6in,top=0.7in,bottom=1in]{geometry}
%\usepackage{pstricks,pst-3dplot}
%\usepackage{minted}
%\usepackage{mdframed}
%\begin{document}
%\definecolor{bg}{rgb}{0.95,0.95,0.95}
\psset{unit =1in}
\begin{pspicture}[showgrid=false](-1,-1)(1,1)
\psset{Alpha=90,Beta=20}
%\pstThreeDCoor[xMax=1,yMax=1,zMax=1]
\pstThreeDSphere[opacity=0.4,strokeopacity=0.2,linewidth=0.8pt](0,0,0){1}
\fileplotThreeD{./otherstuff/data_text_files/Case_A_f0_1deg_datafile.txt}%./otherstuff/

%A axes xyz
\fileplotThreeD[linewidth=2pt,linecolor=blue]{./otherstuff/data_text_files/Case_A_f0_1deg_square_datafile.txt}%./otherstuff/



\pstThreeDLine[linewidth=2pt,%
 linecolor=blue,arrows=->](0,0,0)(0,0.01745241,0.9998477 )

\end{pspicture}


%\end{document}

\end{center}%
%EndExpansion

\begin{case}
方法A在尖點運動下,$%
\theta =55^{\circ }$情況,重力常數%
改為g=20 $m/s^{2}$,的陀螺運%
動。
\end{case}

另一個簡單的範例,%
示範如何任意更改重%
力常數,同法可應用%
於更改其他常數,可%
更改的常數請參考文%
件最後的說明書中Rigid Body
Class的user adjustable parameters。重力增%
加後,明顯的尖點跟%
尖點中間的弧線運動%
區段變大變長了,這%
是很合理的。

%TCIMACRO{%
%\TeXButton{case g20}{\begin{center}
%%\documentclass[12pt,twoside]{article}
%\usepackage[inner=1in,outer=0.6in,top=0.7in,bottom=1in]{geometry}
%\usepackage{pstricks,pst-3dplot}
%\usepackage{minted}
%\usepackage{mdframed}
%\begin{document}
\definecolor{bg}{rgb}{0.95,0.95,0.95}
\psset{unit =1in}
\begin{pspicture}[showgrid=false](-1,-1)(1,1)
\psset{Alpha=45,Beta=20}
%\pstThreeDCoor[xMax=1,yMax=1,zMax=1]
\pstThreeDSphere[opacity=0.4,strokeopacity=0.2,linewidth=0.8pt](0,0,0){1}
\fileplotThreeD{./otherstuff/data_text_files/Case_A_g20_z_datafile.txt}%./otherstuff/

%A axes xyz
\fileplotThreeD[linewidth=2pt,linecolor=blue]{./otherstuff/data_text_files/Case_A_g20_square_datafile.txt}%./otherstuff/



\pstThreeDLine[linewidth=2pt,%
 linecolor=blue,arrows=->](0,0,0)(0,0.81915204,0.57357644)

\end{pspicture}

%\end{center}}}%
%BeginExpansion
\begin{center}
%\documentclass[12pt,twoside]{article}
%\usepackage[inner=1in,outer=0.6in,top=0.7in,bottom=1in]{geometry}
%\usepackage{pstricks,pst-3dplot}
%\usepackage{minted}
%\usepackage{mdframed}
%\begin{document}
\definecolor{bg}{rgb}{0.95,0.95,0.95}
\psset{unit =1in}
\begin{pspicture}[showgrid=false](-1,-1)(1,1)
\psset{Alpha=45,Beta=20}
%\pstThreeDCoor[xMax=1,yMax=1,zMax=1]
\pstThreeDSphere[opacity=0.4,strokeopacity=0.2,linewidth=0.8pt](0,0,0){1}
\fileplotThreeD{./otherstuff/data_text_files/Case_A_g20_z_datafile.txt}%./otherstuff/

%A axes xyz
\fileplotThreeD[linewidth=2pt,linecolor=blue]{./otherstuff/data_text_files/Case_A_g20_square_datafile.txt}%./otherstuff/



\pstThreeDLine[linewidth=2pt,%
 linecolor=blue,arrows=->](0,0,0)(0,0.81915204,0.57357644)

\end{pspicture}

\end{center}%
%EndExpansion

\begin{case}
這裡與實體陀螺儀做%
定性定量的觀察比較%
。,可以點選連結觀%
察我們用gyroscope.com公司所賣%
的陀螺儀所製作的影%
片\href{http://tinypic.com/r/2clcee/8}{\underline{\color{blue}%
\smash{http://tinypic.com/r/2clcee/8}}},觀察四%
種經典章動進動運動%
。然後再用以下的模%
擬code,模擬在相同的條%
件下,在電腦上觀察%
同樣四種章進動運動%
,並比較討論。我們%
可以觀察到,實體陀%
螺儀因有軸承摩擦存%
在而導致能量耗損很%
快,因此運動的damping很%
大,章進動的震幅只%
能持續幾秒就因阻尼%
而消失,電腦模擬就%
可以觀察到完整的運%
動。%
%TCIMACRO{%
%\TeXButton{real compare code plus fig}{%\documentclass[12pt,twoside]{article}
%\usepackage[inner=1in,outer=0.6in,top=0.7in,bottom=1in]{geometry}
%\usepackage{pstricks,pst-3dplot}
%\usepackage{minted}
%\usepackage{mdframed}
%\begin{document}
%\definecolor{bg}{rgb}{0.95,0.95,0.95}

\begin{mdframed}[leftline=false, rightline=false,backgroundcolor=bg]
\inputminted[linenos,fontsize=\footnotesize]{python}{../../Scripts/Developement/Development_zone/Gyroscope-Demo-RealCompare.py}
\end{mdframed}
\psset{unit =1in}
\begin{pspicture}[showgrid=false](-5in,-4in)(-4.099in,-4.01in)
\psset{Alpha=103,Beta=30}
%\pstThreeDCoor[xMax=1,yMax=1,zMax=1]
\pstThreeDSphere[opacity=0.4,strokeopacity=0.2,linewidth=0.8pt](0,0,0){1}
\fileplotThreeD{./otherstuff/data_text_files/Case_realcompare_datafile.txt}%

%A axes xyz
\pstThreeDLine[linewidth=2pt,%
 linecolor=red,arrows=->](0,0,0)(1,0,0)
\pstThreeDLine[linewidth=2pt,%
 linecolor=magenta,arrows=->](0,0,0)(0,0.5735764,-0.81915204)
\pstThreeDLine[linewidth=2pt,%
 linecolor=blue,arrows=->](0,0,0)(0,0.81915204,0.57357644)

\end{pspicture}


%\end{document}
}}%
%BeginExpansion
%\documentclass[12pt,twoside]{article}
%\usepackage[inner=1in,outer=0.6in,top=0.7in,bottom=1in]{geometry}
%\usepackage{pstricks,pst-3dplot}
%\usepackage{minted}
%\usepackage{mdframed}
%\begin{document}
%\definecolor{bg}{rgb}{0.95,0.95,0.95}

\begin{mdframed}[leftline=false, rightline=false,backgroundcolor=bg]
\inputminted[linenos,fontsize=\footnotesize]{python}{../../Scripts/Developement/Development_zone/Gyroscope-Demo-RealCompare.py}
\end{mdframed}
\psset{unit =1in}
\begin{pspicture}[showgrid=false](-5in,-4in)(-4.099in,-4.01in)
\psset{Alpha=103,Beta=30}
%\pstThreeDCoor[xMax=1,yMax=1,zMax=1]
\pstThreeDSphere[opacity=0.4,strokeopacity=0.2,linewidth=0.8pt](0,0,0){1}
\fileplotThreeD{./otherstuff/data_text_files/Case_realcompare_datafile.txt}%

%A axes xyz
\pstThreeDLine[linewidth=2pt,%
 linecolor=red,arrows=->](0,0,0)(1,0,0)
\pstThreeDLine[linewidth=2pt,%
 linecolor=magenta,arrows=->](0,0,0)(0,0.5735764,-0.81915204)
\pstThreeDLine[linewidth=2pt,%
 linecolor=blue,arrows=->](0,0,0)(0,0.81915204,0.57357644)

\end{pspicture}


%\end{document}
%
%EndExpansion
\end{case}

若要切換四種不同的%
經典運動,更改\texttt{b.setcase(1)}%
中的數字,1:尖點,2:環%
狀,3:波紋,4:正圓運動%
。若要更改陀螺初始%
的與space z軸的傾斜角,%
\texttt{b.orien =np.array([-np.radians(55),0,0])}中的數%
字即為角度in degree。由於%
陀螺儀有部分的質量%
是不繞行陀螺對稱軸%
轉動,如陀螺儀的鋁%
製外框,但這部分的%
質量還是會進行章動%
進動,因此跟理想上%
的理論假設有些微的%
差異。我們必須加入%
一個counter weight來描述這部%
分的質量的運動,這%
樣才能正確描述真實%
的運動。此counter weight質量%
是可以調整其量值大%
小,其質心到原點的%
距離,這樣一來我們%
可以做更真實且條件%
範圍更廣泛的測試陀%
螺運動。

\begin{case}
我們比較ABC法給出的陀%
螺$(1,1,1)$向量隨時間的變%
化,B方法由於是採用%
python的ode solver,此solver有確保%
其解有收斂(我們也可%
以調整\texttt{samplerate}來觀察B法%
結果確實不會有變化)%
,因此這邊我們將B法%
視為正確解,我們將%
觀察A與C法給出的$(1,1,1)$向%
量跟B法的差別,我們%
將畫出A法與B法給出的$%
(1,1,1)$向量間的夾角隨時%
間的變化(藍實線),與C%
法與B法給出的$(1,1,1)$向量%
間的夾角隨時間的變%
化(紅虛線)。我們分別%
針對四種經典陀螺章%
動進動狀況作比較,%
此圖在圖\ref{ABCcompareFig}中。
\end{case}

\begin{figure}[th]
\caption{AB與BC法的$(1,1,1)$向量比%
較圖。}
\label{ABCcompareFig}
\begin{center}
\input{CompareDCMEulerAngle.pgf}
\end{center}
\end{figure}

我們發現在第一種尖%
點運動a狀況中A法表現%
很好,在20秒A方法都可%
以與B法保持在0.5度左右%
的偏移誤差,這在轉%
動向量積分法中表現%
算是非常好的,有實%
際應用的價值。b經典%
運動下也還可以,不%
過在cd兩種運動中就越%
來越差了,在cd這兩種%
波紋與正圓兩種經典%
運動下,A與C法(1,1,1)向量%
都偏移發散的非常快%
,五秒就已經到了100 degree%
,這邊在實際應用上%
就會有困難度。不過%
,若是我們仔細觀察%
動畫結果,看A方法給%
出的xyz軸是如何隨時間%
發散的,我們發現事%
實上z軸對稱軸A與B方法%
給出的差異是非常小%
的,差比較多的是x與y%
軸,這一點是有趣的%
,物理上的原因還在%
理解當中,但,在實%
際的應用上我們或許%
可以利用這一點,當%
我們知道我們是在cd兩%
種運動狀況下,我們%
會知道,雖然x與y軸不%
準,但z軸對稱軸會是%
比較準確的,因此當%
我們喪失xy軸準度,我%
們還可以相信z軸。因%
此我們也針對cd兩種經%
典運動畫出AB兩法給出%
的z軸間角度差異隨時%
間的變化。

\begin{figure}[th]
\caption{cd兩種經典運動下AB兩%
法給出的兩個z軸間的%
角度差異。}
\begin{center}
\frame{\input{AB_methods_cdCases_zaxis_compare.pgf}}
\end{center}
\end{figure}

C方法在cd兩種運動狀況%
表現的情況會在以下%
例子中討論。有趣的%
是,當$\theta =90^{\circ }$時,在d%
經典運動(正圓運動)情%
況下,C方法的結果非%
常準確,這將在以下%
例子中呈現。目前還%
在了解其不同章進動%
及不同角度下為何會%
給出不同的差異結果%
。

\begin{case}
BC方法在正圓運動下的%
比較,我們考慮C方法%
在兩個初始角度﹝$\theta
=55^{\circ }$and$\theta =90^{\circ }$﹞的準確性%
。圖\ref{BC_d_5590}分別劃出在%
兩個角度下BC方法解出%
的陀螺z軸的軌跡圖locus plot%
,與BC方法解出的兩組%
陀螺在最後時間的xyz軸(%
紅黃藍),B組結果以實%
線劃出,C組結果以虛%
線畫出。我們會發現%
在55度的時候,C方法表%
現得並沒有很好,甚%
至可以說很差,但在90%
度的時候,C方法幾乎%
與B方法重合,表現得%
極好。這是非常有趣%
的一個現象,我還在%
嘗試理解為何在不同%
角度下C方法的準確度%
會有不同,若是能找%
出物理原因,我們說%
不定可以運用此現象%
於姿態控制當中,運%
用姿態解算法在某些%
特殊運動下會特別準%
確的優勢來做姿態控%
制。
\end{case}

\begin{figure}[th]
\caption{Comparing BC method in circular precession motion at two initial
angles.}
\label{BC_d_5590}
\begin{center}
%\documentclass{article}
%\usepackage{pstricks,pst-3dplot}
%\begin{document}

\psset{unit =1in}
%case 55
\begin{pspicture}[showgrid=false](-1,-1)(1,1)
\psset{Alpha=-90,Beta=30}
%\pstThreeDCoor[xMax=1,yMax=1,zMax=1]
\pstThreeDSphere[opacity=0.4,strokeopacity=0.2,linewidth=0.8pt](0,0,0){1}
\fileplotThreeD{./otherstuff/data_text_files/Case_BC_d_55_Bz_datafile.txt}
\fileplotThreeD[linestyle=dashed]{./otherstuff/data_text_files/Case_BC_d_55_Cz_datafile.txt}%

%B axes xyz
\pstThreeDLine[linewidth=2pt,%
 linecolor=red,arrows=->](0,0,0)(0.70746643,-0.20028796,-0.67777282)
\pstThreeDLine[linewidth=2pt,%
 linecolor=magenta,arrows=->](0,0,0)(0.6227078,0.6302105,0.4637561)
\pstThreeDLine[linewidth=2pt,%
 linecolor=blue,arrows=->](0,0,0)(0.33425478,-0.75014629,0.57057365)

%C axes xyz
\pstThreeDLine[linewidth=3pt, linestyle=dashed,
 linecolor=red,arrows=->](0,0,0)(0.47996987,0.51240539,-0.71208823)
\pstThreeDLine[linewidth=3pt, linestyle=dashed,
 linecolor=magenta,arrows=->](0,0,0)(-0.66391865,0.74273213,0.08695411)
\pstThreeDLine[linewidth=3pt, linestyle=dashed,
 linecolor=blue,arrows=->](0,0,0)(0.57344656,0.4310333,0.69668454)
\end{pspicture}
\hspace{1cm}
%case 90
\begin{pspicture}[showgrid=false](-1,-1)(1,1)
\psset{Alpha=-80,Beta=30}
%\pstThreeDCoor[xMax=1,yMax=1,zMax=1]
\pstThreeDSphere[opacity=0.4,strokeopacity=0.2,linewidth=0.8pt](0,0,0){1}
\fileplotThreeD{./otherstuff/data_text_files/Case_BC_d_90_Bz_datafile.txt}
\fileplotThreeD[linestyle=dashed]{./otherstuff/data_text_files/Case_BC_d_90_Cz_datafile.txt}%./otherstuff/

%B axes xyz
\pstThreeDLine[linewidth=2pt,%
 linecolor=red,arrows=->](0,0,0)(-0.99290292,-0.10100069, 0.06279052)
\pstThreeDLine[linewidth=2pt,%
 linecolor=magenta,arrows=->](0,0,0)(-0.06246816,-0.00635443,-0.99802673)
\pstThreeDLine[linewidth=2pt,%
 linecolor=blue,arrows=->](0,0,0)(1.01200390e-01,-9.94866062e-01,0)

%C axes xyz
\pstThreeDLine[linewidth=3pt, linestyle=dashed,
 linecolor=red,arrows=->](0,0,0)(-0.98705969,-0.10022169,0.12517503)
\pstThreeDLine[linewidth=3pt, linestyle=dashed,
 linecolor=magenta,arrows=->](0,0,0)(-0.12485473,-0.00948683,-0.99212968)
\pstThreeDLine[linewidth=3pt, linestyle=dashed,
 linecolor=blue,arrows=->](0,0,0)(0.10062043,-0.9949199,-0.00314908)
\end{pspicture}


%\end{document}

\end{center}
\end{figure}

C方法在a運動狀況下在20%
秒內雖然誤差比A方法%
大(圖\ref{ABCcompareFig}),但是由於%
A方法的誤差會持續累%
積,因此在四十秒後A%
方法就會有較大的誤%
差了。同樣的在b運動%
狀況下也有類似的狀%
況,五秒內A方法比較%
好,五秒後C方法比較%
好。那麼C方法的誤差%
會不會累積呢?那我%
們就要加長模擬的時%
間。

\begin{case}
加長模擬時間來觀察C%
方法的發散度。
\end{case}

\begin{case}
比較C方法使用不同轉%
動向量來近似所產生%
的不同,可以改進cd兩%
種運動的估測準確度%
。
\end{case}

以下也列出ABC方法的優%
劣比較與適用場合

%TCIMACRO{%
%\TeXButton{ABC compare table}{\begin{center}
%\input{Table-Tabular-ABCcompare.tex}
%\end{center}}}%
%BeginExpansion
\begin{center}
\input{Table-Tabular-ABCcompare.tex}
\end{center}%
%EndExpansion

%TCIMACRO{\TeXButton{clearpage}{\clearpage}}%
%BeginExpansion
\clearpage%
%EndExpansion

\part{在綁缚式慣性感測%
器姿態演算的應用}

\setcounter{page}{1}

我們知道在姿態感測%
的應用上,若是轉動%
軸不隨時間而變化,%
則姿態估測上會容易%
且準確非常多。如汽%
車的航向陀螺儀姿態%
感測器,由於其大部%
分是在平面上轉彎運%
動,其轉動軸多在垂%
直方向,因此姿態的%
估測上是容易且準確%
的。但若是轉動軸會%
隨時間而變化,如同%
人體、飛機、或陀螺%
的姿態估測,其姿態%
的準確度隨時間而發%
散速度是很快的。一%
般在作轉動軸變化的%
姿態估測上都比較困%
難,一般都會融合其%
他感測方法,如GPS、地%
磁、重力來輔助姿態%
的判定。而此篇討論%
一個重點就是是否能%
夠在無其他融合感測%
方法時,對轉動軸變%
化的姿態估測的發散%
度有所歸納,我們以%
轉動速度相當快﹝轉%
動軸變化也快且大﹞%
的陀螺為測試平台,%
與陀螺的尤拉角尤拉%
方程數值解作系統性%
比較,來看看姿態估%
測演算法﹝C方法﹞在%
極端的轉動條件下,%
準確程度如何。並且%
討論為何四種經典章%
動進動會有不同的發%
散程度,以及是否我%
們可以利用發散程度%
較小的情況並作實際%
的應用。因此此物件%
導向化的軟體平台適%
合作為驗證,比較姿%
態估測演算法的平台%
。

\bigskip

上面的DCM的程式可以獨%
立出來應用在感測器%
的姿態估測上。由於DCM%
是考慮貼體角速度$\vec{\omega}
$,因此主要應用在綁%
缚式慣性感測元件﹝%
strap-down IMU﹞上,此類包含%
範圍涵蓋陀螺儀角速%
度感測器,微機電角%
速度感測器,等諸如%
此類的元件。只要把%
元件讀到的貼體角速%
度資料輸出,然後正%
確的輸入到我們這邊%
的DCM遞推程式,就可以%
得到物件轉動的姿態%
。這邊大部分的書籍%
都無法從原理說起,%
都只有提供公式,而%
且也很少詳細的說明%
公式如何使用。不過%
要補充的是,我這邊%
只有轉動部分,並沒%
有平移部分的演算法%
,不過轉動姿態是姿%
態估測的基礎,平移%
的演算是相對容易很%
多的。並且,這邊也%
沒有噪音過濾的功能%
,因為任何的量測儀%
器都會有噪音,因此%
在輸入DCM遞推前,必須%
先經過過濾。一般常%
見的過濾方法是卡爾%
曼濾波,噪音濾波的%
討論是我下一步的目%
標。

\bigskip

\begin{case}
C方法加上噪音後的靜%
止測試。
\end{case}

我們使用C方法來測試%
在此狀況下C方法給出%
的姿態隨時間的運動%
是否也是靜止的。假%
設貼體角速度是零加%
上一個隨機的噪音$\vec{\omega}%
(t)=0+Noise(t)$,$Noise(t)$為一固定Amplitude%
乘上-1到1的隨機數,固%
定的Amplitude單位為radian per second,%
量值大小可從\texttt{%
b.IncludeNoiseInOmega(5)}中的數字5更改%
。我們比較兩個Amplitude,%
產生此動畫的程式如%
下:

\bigskip

%TCIMACRO{%
%\TeXButton{case noise code}{\begin{mdframed}[leftline=false, rightline=false,backgroundcolor=bg]
%\inputminted[linenos,fontsize=\footnotesize]{python}{../../Scripts/cordtrans/sketch-NoiseIncludeInCMethod.py}
%\end{mdframed}}}%
%BeginExpansion
\begin{mdframed}[leftline=false, rightline=false,backgroundcolor=bg]
\inputminted[linenos,fontsize=\footnotesize]{python}{../../Scripts/cordtrans/sketch-NoiseIncludeInCMethod.py}
\end{mdframed}%
%EndExpansion

\bigskip

我們畫出z軸與Z軸隨時%
間的軌跡圖,我們這%
邊將兩個Amplitude的結果劃%
出,左邊是1 radian per sec,右%
邊是5 radian per sec。這樣我們%
就可以跟市面上賣的%
陀螺儀及angular rate sensor做比較%
。57.32 degree = 1 radian,所以我們%
這邊是考慮噪音很大%
的陀螺儀,或是陀螺%
儀在環境溫度變化或%
機械震動很大的極端%
的例子。

%TCIMACRO{%
%\TeXButton{case still noise}{\begin{center}
%%\documentclass{article}
%\usepackage{pstricks,pst-3dplot}
%\begin{document}

\psset{unit =1in}
\begin{pspicture}[showgrid=false](-1,-1)(1,1)
\psset{Alpha=30,Beta=0}

%left lucid sphere
\pstThreeDSphere[opacity=0.4,strokeopacity=0.2,linewidth=0.8pt](0,0,0){1}

\fileplotThreeD{./otherstuff/data_text_files/Case_C_stillnoise_amp1_datafilex.txt}%
\fileplotThreeD{./otherstuff/data_text_files/Case_C_stillnoise_amp1_datafiley.txt}%./otherstuff/
\fileplotThreeD{./otherstuff/data_text_files/Case_C_stillnoise_amp1_datafilez.txt}%./otherstuff/

%left C axes xyz
\pstThreeDLine[linewidth=3pt, linestyle=dashed,
linecolor=red,arrows=->](0,0,0)(1,0,0)
\pstThreeDLine[linewidth=3pt, linestyle=dashed,
linecolor=magenta,arrows=->](0,0,0)(0,0.5,-0.866)
\pstThreeDLine[linewidth=3pt, linestyle=dashed,
 linecolor=blue,arrows=->](0,0,0)(0,0.8660254,0.5)
\end{pspicture}
\hspace{1cm}
\begin{pspicture}[showgrid=false](-1,-1)(1,1)
\psset{Alpha=30,Beta=0}
%\pstThreeDCoor[xMax=1,yMax=1,zMax=1]
\pstThreeDSphere[opacity=0.4,strokeopacity=0.2,linewidth=0.8pt](0,0,0){1}
\fileplotThreeD{./otherstuff/data_text_files/Case_C_stillnoise_amp2_datafilex.txt}%./otherstuff/
\fileplotThreeD{./otherstuff/data_text_files/Case_C_stillnoise_amp2_datafiley.txt}%./otherstuff/
\fileplotThreeD{./otherstuff/data_text_files/Case_C_stillnoise_amp2_datafilez.txt}%./otherstuff/

%C axes xyz
\pstThreeDLine[linewidth=3pt, linestyle=dashed,
linecolor=red,arrows=->](0,0,0)(1,0,0)
\pstThreeDLine[linewidth=3pt, linestyle=dashed,
linecolor=magenta,arrows=->](0,0,0)(0,0.5,-0.866)
\pstThreeDLine[linewidth=3pt, linestyle=dashed,
 linecolor=blue,arrows=->](0,0,0)(0,0.8660254,0.5)
\end{pspicture}


%\end{document}

%\end{center}}}%
%BeginExpansion
\begin{center}
%\documentclass{article}
%\usepackage{pstricks,pst-3dplot}
%\begin{document}

\psset{unit =1in}
\begin{pspicture}[showgrid=false](-1,-1)(1,1)
\psset{Alpha=30,Beta=0}

%left lucid sphere
\pstThreeDSphere[opacity=0.4,strokeopacity=0.2,linewidth=0.8pt](0,0,0){1}

\fileplotThreeD{./otherstuff/data_text_files/Case_C_stillnoise_amp1_datafilex.txt}%
\fileplotThreeD{./otherstuff/data_text_files/Case_C_stillnoise_amp1_datafiley.txt}%./otherstuff/
\fileplotThreeD{./otherstuff/data_text_files/Case_C_stillnoise_amp1_datafilez.txt}%./otherstuff/

%left C axes xyz
\pstThreeDLine[linewidth=3pt, linestyle=dashed,
linecolor=red,arrows=->](0,0,0)(1,0,0)
\pstThreeDLine[linewidth=3pt, linestyle=dashed,
linecolor=magenta,arrows=->](0,0,0)(0,0.5,-0.866)
\pstThreeDLine[linewidth=3pt, linestyle=dashed,
 linecolor=blue,arrows=->](0,0,0)(0,0.8660254,0.5)
\end{pspicture}
\hspace{1cm}
\begin{pspicture}[showgrid=false](-1,-1)(1,1)
\psset{Alpha=30,Beta=0}
%\pstThreeDCoor[xMax=1,yMax=1,zMax=1]
\pstThreeDSphere[opacity=0.4,strokeopacity=0.2,linewidth=0.8pt](0,0,0){1}
\fileplotThreeD{./otherstuff/data_text_files/Case_C_stillnoise_amp2_datafilex.txt}%./otherstuff/
\fileplotThreeD{./otherstuff/data_text_files/Case_C_stillnoise_amp2_datafiley.txt}%./otherstuff/
\fileplotThreeD{./otherstuff/data_text_files/Case_C_stillnoise_amp2_datafilez.txt}%./otherstuff/

%C axes xyz
\pstThreeDLine[linewidth=3pt, linestyle=dashed,
linecolor=red,arrows=->](0,0,0)(1,0,0)
\pstThreeDLine[linewidth=3pt, linestyle=dashed,
linecolor=magenta,arrows=->](0,0,0)(0,0.5,-0.866)
\pstThreeDLine[linewidth=3pt, linestyle=dashed,
 linecolor=blue,arrows=->](0,0,0)(0,0.8660254,0.5)
\end{pspicture}


%\end{document}

\end{center}%
%EndExpansion

\bigskip

\begin{case}
C方法加上噪音後,經%
過running average的濾波方式,%
來測試尖點運動情況%
下的估測能力。
\end{case}

\bigskip

**need thinking即時解尤拉方程%
然後作DCM姿態遞推的好%
處是,不用事先知道%
力矩項,因此會比尤%
拉角尤拉方程方法更%
適用於力矩會隨時間%
變動或力矩為未知的%
情況,舉例來說,我%
們可以用在機器人的%
恣態控制上面,又或%
者,可以用在有用到%
剛體轉動的遊戲物理%
引擎上面。這些狀況%
都是尤拉角尤拉方程%
中的力矩項無法事先%
預知,因此無法事先%
解其尤拉角尤拉方程%
。又或者,若是力矩%
項太過複雜,如無人%
機的空氣動力,也適%
合使用我們這邊所介%
紹的即時解尤拉方程%
並作姿態遞推,只要%
有角加速度計可以測%
量角加速度就可以。%
我們這邊所介紹的陀%
螺,是一個應用的例%
子,由於陀螺的轉動%
速度是每秒20轉,因此%
轉動速度很高,對於%
姿態遞推的精度要求%
算很高,因此陀螺適%
合當作驗證姿態遞推%
演算的非常好的例子%
,若陀螺能算的準,%
其他轉動速度不高的%
情況將會非常準。陀%
螺的情況我們可以比%
喻為,把機器人或無%
人機丟入龍捲風裡面%
,並且要作姿態控制%
。

%TCIMACRO{\TeXButton{clearpage}{\clearpage}}%
%BeginExpansion
\clearpage%
%EndExpansion

\part{補充}

陀螺等周速運動Figure \ref%
{FourClassics}(d)的初始值條件如%
何計算呢?等周速的條%
件在Goldstein第二版5-77式給出%
\begin{equation}
Mgl=\dot{\phi}\left( I_{3}\omega _{3}-I_{1}\dot{\phi}\cos \theta _{0}\right)
\end{equation}%
,不過此式是由尤拉%
角(euler angles)給出,但我們%
需要的是anguler velocity along body的初%
始值,因此我們必須%
轉換尤拉角到anguler velocity along body%
,方法如下。上式中$%
\omega _{3}$即為我們的$\left( \omega
_{z}\right) _{b}$,比如說是20 Hz,$%
\theta _{0}$即為我們之前的orien%
向量所定,由上式可%
求出兩組$\dot{\phi}(t_{0})$。另外%
尤拉角跟anugler velocity along body的關%
係式在書\cite[Chapter 4, Equation 125]{goldstein} 
給出%
\begin{eqnarray}
(\omega _{x})_{b} &=&\dot{\phi}\sin \theta \sin \psi +\dot{\theta}\cos \psi
\\
(\omega _{y})_{b} &=&\dot{\phi}\sin \theta \cos \psi -\dot{\theta}\sin \psi
\\
(\omega _{z})_{b} &=&\dot{\phi}\cos \theta +\dot{\psi}
\end{eqnarray}%
知道$\dot{\phi}(t_{0})$、$\theta _{0}$、$\dot{\theta}%
_{0}=0$及$\psi _{0}$,用一二條後%
就可得到$\omega _{xb}(t_{0})$與$\omega
_{yb}(t_{0})$,這樣我們就得到%
anguler velocity along body的初始值。因%
為$\dot{\phi}(t_{0})$有兩組,因此%
解出的貼體角速度也%
會有兩組,兩組的物%
理意義分別如下,一%
種情況是fast top,這個狀%
況相當於重力的影響%
遠小於總角動量$L$,因%
此這個特別的例子基%
本上相當於忽略重力%
,而陀螺基本上會像%
一個free top一樣進行precession。%
另一種狀況是slow top,也%
就是上面模擬結果中%
第四種的狀況,這裡%
提供的python程式所有情%
況都可以模擬。另外%
一個特殊的情況是在fast
top的情形下,如果初始%
值$\theta _{0}=0$,也就是陀螺z%
軸的起始狀態是垂直%
於水平面的,這樣的%
話陀螺幾乎會像靜止%
不動一樣,我們也叫%
這情況做sleeping top。這些計%
算經典狀況的程式方%
程被包含在setcase()方程裡%
。

\setcounter{page}{1}

\begin{remark}
要陀螺具有Precession and Nutation的%
動作,L/$\Delta L$必須要大,%
如果L小於$\Delta L$,則只會%
有陀螺質量受重力影%
響往下倒下的運動(不%
過這對檢查程式有沒%
有錯誤很有幫助!),理%
想上L至少要大於$\Delta L$,%
最好L大大於$\Delta L$。化成%
數值上的比較:這代%
表%
\begin{equation}
L\gg \Delta L\Rightarrow I\cdot 2\pi f\gg \vec{\Gamma}\Delta t\Rightarrow
I\cdot 2\pi f\gg \vec{r}\times \vec{F}\cdot 1/f\Rightarrow f\gg \sqrt{\frac{%
arm\cdot Mg\cdot \sin (\theta )}{2\pi I\cdot G}}
\end{equation}%
where $\theta $ is gyro's tilt angle and G is moment of inertial geometry
factor. 考慮$\Delta t$的量級大約%
是陀螺轉幾圈的時間%
(characteristic time),量級上約是$\sim
1/f$,若假設arm是10 cm, M = 1kg, g=10 m/s$^{2}$%
, I = 0.5M(0.05)$^{2}$,則f最少要10 Hertz以%
上。因此我們將以這%
些參數比較f = 1, 10, 50 Hertz所給%
出的陀螺運動。
\end{remark}

\begin{thebibliography}{99}
\bibitem{goldstein} Herbert Goldstein, \emph{Classical Mechanics}. Addison
Wesley, Massachusetts, 2nd Edition, 1980

\bibitem{tong} David Tong, \emph{Classical Dynamics University of Cambridge
Part II Mathematical Tripos.} Cambridge UK, 2004-2005, (Course note,
available on the web)

\bibitem{matlab} \href{http://www.mathworks.com/help/matlab/ordinary-differential-equations.html%
}{\underline{\color{blue}%
\smash{Matlab online documentation - Ordinary
differential equations.}}}, Matlab R2014a

\bibitem{xu} 徐小明 钟万勰,%
\textit{刚体动力学的四元%
数表示及保辛积分},%
《应用数学和力学》 2014%
, 35(1): 111

\bibitem{hasbun} Javier E. Hasbun, \emph{Classical Mechanics with Matlab
Appications.} Jones and Bartlett Publishers, London UK, 2009

\bibitem{titterton} D.H. Titterton and J.L. Weston, \emph{Strapdown Inertial
Navigation Technology}, Peter Peregrinus Ltd., London UK, 1997

\bibitem{pixarnote} David Baraff\textit{, }\href{http://graphics.cs.cmu.edu/courses/15-869-F08/lec/14/notesg.pdf%
}{\underline{\color{blue}%
\smash{Physically Based Modeling - Rigid Body
Simulation}}}, Pixar Animation Studios notes

\bibitem{wolfgangSimMeth} Harvey Gould and Jan Tobochnik and Wolfgang
Christian, \emph{An Introduction to Computer Simulation Methods Third Edition%
}, Addison-Wesley, 2006, \href{http://www.opensourcephysics.org/items/detail.cfm?ID=7375%
}{\underline{\color{blue}\smash{draft available on comPADRE}}}

\bibitem{rapaport} Dennis Rapaport, \emph{The Art of Molecular Dynamics
Simulation 2nd Edition}, Cambridge University Press, 2004

\bibitem{eugene} Eugene Butikov, \href{http://butikov.faculty.ifmo.ru/Applets/Gyroscope.pdf%
}{\underline{\color{blue}\smash{Precession and nutation of a gyroscope}}},
St. Petersburg, Russia, http://butikov.faculty.ifmo.ru/Applets/Gyroscope.pdf

\bibitem{savage} Paul Savage, \emph{Strapdown Inertial Navigation
Integration Algorithm Design Part 1: Attitude Algorithms}, Journal of
Guidance, Control, and Dynamics, Vol. 21, No. 1, January - February 1998

\bibitem{eberly} David Eberly, \emph{Game Physics,} 2nd Edition, CRC Press,
2010
\end{thebibliography}

\bigskip

\null
\vfill

\begin{center}
\rule{6in}{0.01in}
\end{center}

\noindent This document is prepared with Scientific Workplace 5.0 and
typeset with Tex Live 2013 (Xelatex).

\noindent\href{http://whymranderson.blogspot.tw/2014/03/how-to-convert-swp-50-special-unicode.html%
}{\underline{\color{blue}\smash{Blogpost: http://ppt.cc/yS2-}}}

\noindent\href{https://drive.google.com/file/d/0B96HmLH-SQVmM1dvYlFiQm9ESGM/edit?usp=sharing%
}{\underline{\color{blue}%
\smash{Python code:
https://drive.google.com/file/d/0B96HmLH-SQVmM1dvYlFiQm9ESGM/edit?usp=sharing}%
}}

\bigskip

%TCIMACRO{%
%\TeXButton{Include gyrodoc pdf pages}{\includepdf[pages=-]{../../Scripts/GyroDocs/_build/latex/GyroSoftdocumentation.pdf}}}%
%BeginExpansion
\includepdf[pages=-]{../../Scripts/GyroDocs/_build/latex/GyroSoftdocumentation.pdf}%
%EndExpansion

\end{document}
