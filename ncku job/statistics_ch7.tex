
\documentclass{article}
%%%%%%%%%%%%%%%%%%%%%%%%%%%%%%%%%%%%%%%%%%%%%%%%%%%%%%%%%%%%%%%%%%%%%%%%%%%%%%%%%%%%%%%%%%%%%%%%%%%%%%%%%%%%%%%%%%%%%%%%%%%%%%%%%%%%%%%%%%%%%%%%%%%%%%%%%%%%%%%%%%%%%%%%%%%%%%%%%%%%%%%%%%%%%%%%%%%%%%%%%%%%%%%%%%%%%%%%%%%%%%%%%%%%%%%%%%%%%%%%%%%%%%%%%%%%
\usepackage{amssymb}
\usepackage{amsfonts}
\usepackage{amsmath}
\usepackage{accents}
\usepackage[ignoreall,a4paper]{geometry}
\usepackage{fancyhdr}

\setcounter{MaxMatrixCols}{10}
%TCIDATA{OutputFilter=LATEX.DLL}
%TCIDATA{Version=5.00.0.2606}
%TCIDATA{<META NAME="SaveForMode" CONTENT="1">}
%TCIDATA{BibliographyScheme=Manual}
%TCIDATA{Created=Wednesday, November 25, 2015 15:33:37}
%TCIDATA{LastRevised=Thursday, December 31, 2015 11:58:46}
%TCIDATA{<META NAME="GraphicsSave" CONTENT="32">}
%TCIDATA{<META NAME="DocumentShell" CONTENT="Standard LaTeX\Blank - Standard LaTeX Article">}
%TCIDATA{CSTFile=40 LaTeX article.cst}
%TCIDATA{ComputeDefs=
%$W=\left( 1-\sigma \right) I$
%}


\newtheorem{theorem}{Theorem}
\newtheorem{acknowledgement}[theorem]{Acknowledgement}
\newtheorem{algorithm}[theorem]{Algorithm}
\newtheorem{axiom}[theorem]{Axiom}
\newtheorem{case}[theorem]{Case}
\newtheorem{claim}[theorem]{Claim}
\newtheorem{conclusion}[theorem]{Conclusion}
\newtheorem{condition}[theorem]{Condition}
\newtheorem{conjecture}[theorem]{Conjecture}
\newtheorem{corollary}[theorem]{Corollary}
\newtheorem{criterion}[theorem]{Criterion}
\newtheorem{definition}[theorem]{Definition}
\newtheorem{example}[theorem]{Example}
\newtheorem{exercise}[theorem]{Exercise}
\newtheorem{lemma}[theorem]{Lemma}
\newtheorem{notation}[theorem]{Notation}
\newtheorem{problem}[theorem]{Problem}
\newtheorem{proposition}[theorem]{Proposition}
\newtheorem{remark}[theorem]{Remark}
\newtheorem{solution}[theorem]{Solution}
\newtheorem{summary}[theorem]{Summary}
\newenvironment{proof}[1][Proof]{\noindent\textbf{#1.} }{\ \rule{0.5em}{0.5em}}
% Macros for Scientific Word 4.0 documents saved with the LaTeX filter.
% Copyright (C) 2002 Mackichan Software, Inc.

\typeout{TCILATEX Macros for Scientific Word 5.0 <13 Feb 2003>.}
\typeout{NOTICE:  This macro file is NOT proprietary and may be 
freely copied and distributed.}
%
\makeatletter

%%%%%%%%%%%%%%%%%%%%%
% pdfTeX related.
\ifx\pdfoutput\relax\let\pdfoutput=\undefined\fi
\newcount\msipdfoutput
\ifx\pdfoutput\undefined
\else
 \ifcase\pdfoutput
 \else 
    \msipdfoutput=1
    \ifx\paperwidth\undefined
    \else
      \ifdim\paperheight=0pt\relax
      \else
        \pdfpageheight\paperheight
      \fi
      \ifdim\paperwidth=0pt\relax
      \else
        \pdfpagewidth\paperwidth
      \fi
    \fi
  \fi  
\fi

%%%%%%%%%%%%%%%%%%%%%
% FMTeXButton
% This is used for putting TeXButtons in the 
% frontmatter of a document. Add a line like
% \QTagDef{FMTeXButton}{101}{} to the filter 
% section of the cst being used. Also add a
% new section containing:
%     [f_101]
%     ALIAS=FMTexButton
%     TAG_TYPE=FIELD
%     TAG_LEADIN=TeX Button:
%
% It also works to put \defs in the preamble after 
% the \input tcilatex
\def\FMTeXButton#1{#1}
%
%%%%%%%%%%%%%%%%%%%%%%
% macros for time
\newcount\@hour\newcount\@minute\chardef\@x10\chardef\@xv60
\def\tcitime{
\def\@time{%
  \@minute\time\@hour\@minute\divide\@hour\@xv
  \ifnum\@hour<\@x 0\fi\the\@hour:%
  \multiply\@hour\@xv\advance\@minute-\@hour
  \ifnum\@minute<\@x 0\fi\the\@minute
  }}%

%%%%%%%%%%%%%%%%%%%%%%
% macro for hyperref and msihyperref
%\@ifundefined{hyperref}{\def\hyperref#1#2#3#4{#2\ref{#4}#3}}{}

\def\x@hyperref#1#2#3{%
   % Turn off various catcodes before reading parameter 4
   \catcode`\~ = 12
   \catcode`\$ = 12
   \catcode`\_ = 12
   \catcode`\# = 12
   \catcode`\& = 12
   \y@hyperref{#1}{#2}{#3}%
}

\def\y@hyperref#1#2#3#4{%
   #2\ref{#4}#3
   \catcode`\~ = 13
   \catcode`\$ = 3
   \catcode`\_ = 8
   \catcode`\# = 6
   \catcode`\& = 4
}

\@ifundefined{hyperref}{\let\hyperref\x@hyperref}{}
\@ifundefined{msihyperref}{\let\msihyperref\x@hyperref}{}




% macro for external program call
\@ifundefined{qExtProgCall}{\def\qExtProgCall#1#2#3#4#5#6{\relax}}{}
%%%%%%%%%%%%%%%%%%%%%%
%
% macros for graphics
%
\def\FILENAME#1{#1}%
%
\def\QCTOpt[#1]#2{%
  \def\QCTOptB{#1}
  \def\QCTOptA{#2}
}
\def\QCTNOpt#1{%
  \def\QCTOptA{#1}
  \let\QCTOptB\empty
}
\def\Qct{%
  \@ifnextchar[{%
    \QCTOpt}{\QCTNOpt}
}
\def\QCBOpt[#1]#2{%
  \def\QCBOptB{#1}%
  \def\QCBOptA{#2}%
}
\def\QCBNOpt#1{%
  \def\QCBOptA{#1}%
  \let\QCBOptB\empty
}
\def\Qcb{%
  \@ifnextchar[{%
    \QCBOpt}{\QCBNOpt}%
}
\def\PrepCapArgs{%
  \ifx\QCBOptA\empty
    \ifx\QCTOptA\empty
      {}%
    \else
      \ifx\QCTOptB\empty
        {\QCTOptA}%
      \else
        [\QCTOptB]{\QCTOptA}%
      \fi
    \fi
  \else
    \ifx\QCBOptA\empty
      {}%
    \else
      \ifx\QCBOptB\empty
        {\QCBOptA}%
      \else
        [\QCBOptB]{\QCBOptA}%
      \fi
    \fi
  \fi
}
\newcount\GRAPHICSTYPE
%\GRAPHICSTYPE 0 is for TurboTeX
%\GRAPHICSTYPE 1 is for DVIWindo (PostScript)
%%%(removed)%\GRAPHICSTYPE 2 is for psfig (PostScript)
\GRAPHICSTYPE=\z@
\def\GRAPHICSPS#1{%
 \ifcase\GRAPHICSTYPE%\GRAPHICSTYPE=0
   \special{ps: #1}%
 \or%\GRAPHICSTYPE=1
   \special{language "PS", include "#1"}%
%%%\or%\GRAPHICSTYPE=2
%%%  #1%
 \fi
}%
%
\def\GRAPHICSHP#1{\special{include #1}}%
%
% \graffile{ body }                                  %#1
%          { contentswidth (scalar)  }               %#2
%          { contentsheight (scalar) }               %#3
%          { vertical shift when in-line (scalar) }  %#4

\def\graffile#1#2#3#4{%
%%% \ifnum\GRAPHICSTYPE=\tw@
%%%  %Following if using psfig
%%%  \@ifundefined{psfig}{\input psfig.tex}{}%
%%%  \psfig{file=#1, height=#3, width=#2}%
%%% \else
  %Following for all others
  % JCS - added BOXTHEFRAME, see below
    \bgroup
	   \@inlabelfalse
       \leavevmode
       \@ifundefined{bbl@deactivate}{\def~{\string~}}{\activesoff}%
        \raise -#4 \BOXTHEFRAME{%
           \hbox to #2{\raise #3\hbox to #2{\null #1\hfil}}}%
    \egroup
}%
%
% A box for drafts
\def\draftbox#1#2#3#4{%
 \leavevmode\raise -#4 \hbox{%
  \frame{\rlap{\protect\tiny #1}\hbox to #2%
   {\vrule height#3 width\z@ depth\z@\hfil}%
  }%
 }%
}%
%
\newcount\@msidraft
\@msidraft=\z@
\let\nographics=\@msidraft
\newif\ifwasdraft
\wasdraftfalse

%  \GRAPHIC{ body }                                  %#1
%          { draft name }                            %#2
%          { contentswidth (scalar)  }               %#3
%          { contentsheight (scalar) }               %#4
%          { vertical shift when in-line (scalar) }  %#5
\def\GRAPHIC#1#2#3#4#5{%
   \ifnum\@msidraft=\@ne\draftbox{#2}{#3}{#4}{#5}%
   \else\graffile{#1}{#3}{#4}{#5}%
   \fi
}
%
\def\addtoLaTeXparams#1{%
    \edef\LaTeXparams{\LaTeXparams #1}}%
%
% JCS -  added a switch BoxFrame that can 
% be set by including X in the frame params.
% If set a box is drawn around the frame.

\newif\ifBoxFrame \BoxFramefalse
\newif\ifOverFrame \OverFramefalse
\newif\ifUnderFrame \UnderFramefalse

\def\BOXTHEFRAME#1{%
   \hbox{%
      \ifBoxFrame
         \frame{#1}%
      \else
         {#1}%
      \fi
   }%
}


\def\doFRAMEparams#1{\BoxFramefalse\OverFramefalse\UnderFramefalse\readFRAMEparams#1\end}%
\def\readFRAMEparams#1{%
 \ifx#1\end%
  \let\next=\relax
  \else
  \ifx#1i\dispkind=\z@\fi
  \ifx#1d\dispkind=\@ne\fi
  \ifx#1f\dispkind=\tw@\fi
  \ifx#1t\addtoLaTeXparams{t}\fi
  \ifx#1b\addtoLaTeXparams{b}\fi
  \ifx#1p\addtoLaTeXparams{p}\fi
  \ifx#1h\addtoLaTeXparams{h}\fi
  \ifx#1X\BoxFrametrue\fi
  \ifx#1O\OverFrametrue\fi
  \ifx#1U\UnderFrametrue\fi
  \ifx#1w
    \ifnum\@msidraft=1\wasdrafttrue\else\wasdraftfalse\fi
    \@msidraft=\@ne
  \fi
  \let\next=\readFRAMEparams
  \fi
 \next
 }%
%
%Macro for In-line graphics object
%   \IFRAME{ contentswidth (scalar)  }               %#1
%          { contentsheight (scalar) }               %#2
%          { vertical shift when in-line (scalar) }  %#3
%          { draft name }                            %#4
%          { body }                                  %#5
%          { caption}                                %#6


\def\IFRAME#1#2#3#4#5#6{%
      \bgroup
      \let\QCTOptA\empty
      \let\QCTOptB\empty
      \let\QCBOptA\empty
      \let\QCBOptB\empty
      #6%
      \parindent=0pt
      \leftskip=0pt
      \rightskip=0pt
      \setbox0=\hbox{\QCBOptA}%
      \@tempdima=#1\relax
      \ifOverFrame
          % Do this later
          \typeout{This is not implemented yet}%
          \show\HELP
      \else
         \ifdim\wd0>\@tempdima
            \advance\@tempdima by \@tempdima
            \ifdim\wd0 >\@tempdima
               \setbox1 =\vbox{%
                  \unskip\hbox to \@tempdima{\hfill\GRAPHIC{#5}{#4}{#1}{#2}{#3}\hfill}%
                  \unskip\hbox to \@tempdima{\parbox[b]{\@tempdima}{\QCBOptA}}%
               }%
               \wd1=\@tempdima
            \else
               \textwidth=\wd0
               \setbox1 =\vbox{%
                 \noindent\hbox to \wd0{\hfill\GRAPHIC{#5}{#4}{#1}{#2}{#3}\hfill}\\%
                 \noindent\hbox{\QCBOptA}%
               }%
               \wd1=\wd0
            \fi
         \else
            \ifdim\wd0>0pt
              \hsize=\@tempdima
              \setbox1=\vbox{%
                \unskip\GRAPHIC{#5}{#4}{#1}{#2}{0pt}%
                \break
                \unskip\hbox to \@tempdima{\hfill \QCBOptA\hfill}%
              }%
              \wd1=\@tempdima
           \else
              \hsize=\@tempdima
              \setbox1=\vbox{%
                \unskip\GRAPHIC{#5}{#4}{#1}{#2}{0pt}%
              }%
              \wd1=\@tempdima
           \fi
         \fi
         \@tempdimb=\ht1
         %\advance\@tempdimb by \dp1
         \advance\@tempdimb by -#2
         \advance\@tempdimb by #3
         \leavevmode
         \raise -\@tempdimb \hbox{\box1}%
      \fi
      \egroup%
}%
%
%Macro for Display graphics object
%   \DFRAME{ contentswidth (scalar)  }               %#1
%          { contentsheight (scalar) }               %#2
%          { draft label }                           %#3
%          { name }                                  %#4
%          { caption}                                %#5
\def\DFRAME#1#2#3#4#5{%
  \vspace\topsep
  \hfil\break
  \bgroup
     \leftskip\@flushglue
	 \rightskip\@flushglue
	 \parindent\z@
	 \parfillskip\z@skip
     \let\QCTOptA\empty
     \let\QCTOptB\empty
     \let\QCBOptA\empty
     \let\QCBOptB\empty
	 \vbox\bgroup
        \ifOverFrame 
           #5\QCTOptA\par
        \fi
        \GRAPHIC{#4}{#3}{#1}{#2}{\z@}%
        \ifUnderFrame 
           \break#5\QCBOptA
        \fi
	 \egroup
  \egroup
  \vspace\topsep
  \break
}%
%
%Macro for Floating graphic object
%   \FFRAME{ framedata f|i tbph x F|T }              %#1
%          { contentswidth (scalar)  }               %#2
%          { contentsheight (scalar) }               %#3
%          { caption }                               %#4
%          { label }                                 %#5
%          { draft name }                            %#6
%          { body }                                  %#7
\def\FFRAME#1#2#3#4#5#6#7{%
 %If float.sty loaded and float option is 'h', change to 'H'  (gp) 1998/09/05
  \@ifundefined{floatstyle}
    {%floatstyle undefined (and float.sty not present), no change
     \begin{figure}[#1]%
    }
    {%floatstyle DEFINED
	 \ifx#1h%Only the h parameter, change to H
      \begin{figure}[H]%
	 \else
      \begin{figure}[#1]%
	 \fi
	}
  \let\QCTOptA\empty
  \let\QCTOptB\empty
  \let\QCBOptA\empty
  \let\QCBOptB\empty
  \ifOverFrame
    #4
    \ifx\QCTOptA\empty
    \else
      \ifx\QCTOptB\empty
        \caption{\QCTOptA}%
      \else
        \caption[\QCTOptB]{\QCTOptA}%
      \fi
    \fi
    \ifUnderFrame\else
      \label{#5}%
    \fi
  \else
    \UnderFrametrue%
  \fi
  \begin{center}\GRAPHIC{#7}{#6}{#2}{#3}{\z@}\end{center}%
  \ifUnderFrame
    #4
    \ifx\QCBOptA\empty
      \caption{}%
    \else
      \ifx\QCBOptB\empty
        \caption{\QCBOptA}%
      \else
        \caption[\QCBOptB]{\QCBOptA}%
      \fi
    \fi
    \label{#5}%
  \fi
  \end{figure}%
 }%
%
%
%    \FRAME{ framedata f|i tbph x F|T }              %#1
%          { contentswidth (scalar)  }               %#2
%          { contentsheight (scalar) }               %#3
%          { vertical shift when in-line (scalar) }  %#4
%          { caption }                               %#5
%          { label }                                 %#6
%          { name }                                  %#7
%          { body }                                  %#8
%
%    framedata is a string which can contain the following
%    characters: idftbphxFT
%    Their meaning is as follows:
%             i, d or f : in-line, display, or floating
%             t,b,p,h   : LaTeX floating placement options
%             x         : fit contents box to contents
%             F or T    : Figure or Table. 
%                         Later this can expand
%                         to a more general float class.
%
%
\newcount\dispkind%

\def\makeactives{
  \catcode`\"=\active
  \catcode`\;=\active
  \catcode`\:=\active
  \catcode`\'=\active
  \catcode`\~=\active
}
\bgroup
   \makeactives
   \gdef\activesoff{%
      \def"{\string"}%
      \def;{\string;}%
      \def:{\string:}%
      \def'{\string'}%
      \def~{\string~}%
      %\bbl@deactivate{"}%
      %\bbl@deactivate{;}%
      %\bbl@deactivate{:}%
      %\bbl@deactivate{'}%
    }
\egroup

\def\FRAME#1#2#3#4#5#6#7#8{%
 \bgroup
 \ifnum\@msidraft=\@ne
   \wasdrafttrue
 \else
   \wasdraftfalse%
 \fi
 \def\LaTeXparams{}%
 \dispkind=\z@
 \def\LaTeXparams{}%
 \doFRAMEparams{#1}%
 \ifnum\dispkind=\z@\IFRAME{#2}{#3}{#4}{#7}{#8}{#5}\else
  \ifnum\dispkind=\@ne\DFRAME{#2}{#3}{#7}{#8}{#5}\else
   \ifnum\dispkind=\tw@
    \edef\@tempa{\noexpand\FFRAME{\LaTeXparams}}%
    \@tempa{#2}{#3}{#5}{#6}{#7}{#8}%
    \fi
   \fi
  \fi
  \ifwasdraft\@msidraft=1\else\@msidraft=0\fi{}%
  \egroup
 }%
%
% This macro added to let SW gobble a parameter that
% should not be passed on and expanded. 

\def\TEXUX#1{"texux"}

%
% Macros for text attributes:
%
\def\BF#1{{\bf {#1}}}%
\def\NEG#1{\leavevmode\hbox{\rlap{\thinspace/}{$#1$}}}%
%
%%%%%%%%%%%%%%%%%%%%%%%%%%%%%%%%%%%%%%%%%%%%%%%%%%%%%%%%%%%%%%%%%%%%%%%%
%
%
% macros for user - defined functions
\def\limfunc#1{\mathop{\rm #1}}%
\def\func#1{\mathop{\rm #1}\nolimits}%
% macro for unit names
\def\unit#1{\mathord{\thinspace\rm #1}}%

%
% miscellaneous 
\long\def\QQQ#1#2{%
     \long\expandafter\def\csname#1\endcsname{#2}}%
\@ifundefined{QTP}{\def\QTP#1{}}{}
\@ifundefined{QEXCLUDE}{\def\QEXCLUDE#1{}}{}
\@ifundefined{Qlb}{\def\Qlb#1{#1}}{}
\@ifundefined{Qlt}{\def\Qlt#1{#1}}{}
\def\QWE{}%
\long\def\QQA#1#2{}%
\def\QTR#1#2{{\csname#1\endcsname {#2}}}%
\long\def\TeXButton#1#2{#2}%
\long\def\QSubDoc#1#2{#2}%
\def\EXPAND#1[#2]#3{}%
\def\NOEXPAND#1[#2]#3{}%
\def\PROTECTED{}%
\def\LaTeXparent#1{}%
\def\ChildStyles#1{}%
\def\ChildDefaults#1{}%
\def\QTagDef#1#2#3{}%

% Constructs added with Scientific Notebook
\@ifundefined{correctchoice}{\def\correctchoice{\relax}}{}
\@ifundefined{HTML}{\def\HTML#1{\relax}}{}
\@ifundefined{TCIIcon}{\def\TCIIcon#1#2#3#4{\relax}}{}
\if@compatibility
  \typeout{Not defining UNICODE  U or CustomNote commands for LaTeX 2.09.}
\else
  \providecommand{\UNICODE}[2][]{\protect\rule{.1in}{.1in}}
  \providecommand{\U}[1]{\protect\rule{.1in}{.1in}}
  \providecommand{\CustomNote}[3][]{\marginpar{#3}}
\fi

\@ifundefined{lambdabar}{
      \def\lambdabar{\errmessage{You have used the lambdabar symbol. 
                      This is available for typesetting only in RevTeX styles.}}
   }{}

%
% Macros for style editor docs
\@ifundefined{StyleEditBeginDoc}{\def\StyleEditBeginDoc{\relax}}{}
%
% Macros for footnotes
\def\QQfnmark#1{\footnotemark}
\def\QQfntext#1#2{\addtocounter{footnote}{#1}\footnotetext{#2}}
%
% Macros for indexing.
%
\@ifundefined{TCIMAKEINDEX}{}{\makeindex}%
%
% Attempts to avoid problems with other styles
\@ifundefined{abstract}{%
 \def\abstract{%
  \if@twocolumn
   \section*{Abstract (Not appropriate in this style!)}%
   \else \small 
   \begin{center}{\bf Abstract\vspace{-.5em}\vspace{\z@}}\end{center}%
   \quotation 
   \fi
  }%
 }{%
 }%
\@ifundefined{endabstract}{\def\endabstract
  {\if@twocolumn\else\endquotation\fi}}{}%
\@ifundefined{maketitle}{\def\maketitle#1{}}{}%
\@ifundefined{affiliation}{\def\affiliation#1{}}{}%
\@ifundefined{proof}{\def\proof{\noindent{\bfseries Proof. }}}{}%
\@ifundefined{endproof}{\def\endproof{\mbox{\ \rule{.1in}{.1in}}}}{}%
\@ifundefined{newfield}{\def\newfield#1#2{}}{}%
\@ifundefined{chapter}{\def\chapter#1{\par(Chapter head:)#1\par }%
 \newcount\c@chapter}{}%
\@ifundefined{part}{\def\part#1{\par(Part head:)#1\par }}{}%
\@ifundefined{section}{\def\section#1{\par(Section head:)#1\par }}{}%
\@ifundefined{subsection}{\def\subsection#1%
 {\par(Subsection head:)#1\par }}{}%
\@ifundefined{subsubsection}{\def\subsubsection#1%
 {\par(Subsubsection head:)#1\par }}{}%
\@ifundefined{paragraph}{\def\paragraph#1%
 {\par(Subsubsubsection head:)#1\par }}{}%
\@ifundefined{subparagraph}{\def\subparagraph#1%
 {\par(Subsubsubsubsection head:)#1\par }}{}%
%%%%%%%%%%%%%%%%%%%%%%%%%%%%%%%%%%%%%%%%%%%%%%%%%%%%%%%%%%%%%%%%%%%%%%%%
% These symbols are not recognized by LaTeX
\@ifundefined{therefore}{\def\therefore{}}{}%
\@ifundefined{backepsilon}{\def\backepsilon{}}{}%
\@ifundefined{yen}{\def\yen{\hbox{\rm\rlap=Y}}}{}%
\@ifundefined{registered}{%
   \def\registered{\relax\ifmmode{}\r@gistered
                    \else$\m@th\r@gistered$\fi}%
 \def\r@gistered{^{\ooalign
  {\hfil\raise.07ex\hbox{$\scriptstyle\rm\text{R}$}\hfil\crcr
  \mathhexbox20D}}}}{}%
\@ifundefined{Eth}{\def\Eth{}}{}%
\@ifundefined{eth}{\def\eth{}}{}%
\@ifundefined{Thorn}{\def\Thorn{}}{}%
\@ifundefined{thorn}{\def\thorn{}}{}%
% A macro to allow any symbol that requires math to appear in text
\def\TEXTsymbol#1{\mbox{$#1$}}%
\@ifundefined{degree}{\def\degree{{}^{\circ}}}{}%
%
% macros for T3TeX files
\newdimen\theight
\@ifundefined{Column}{\def\Column{%
 \vadjust{\setbox\z@=\hbox{\scriptsize\quad\quad tcol}%
  \theight=\ht\z@\advance\theight by \dp\z@\advance\theight by \lineskip
  \kern -\theight \vbox to \theight{%
   \rightline{\rlap{\box\z@}}%
   \vss
   }%
  }%
 }}{}%
%
\@ifundefined{qed}{\def\qed{%
 \ifhmode\unskip\nobreak\fi\ifmmode\ifinner\else\hskip5\p@\fi\fi
 \hbox{\hskip5\p@\vrule width4\p@ height6\p@ depth1.5\p@\hskip\p@}%
 }}{}%
%
\@ifundefined{cents}{\def\cents{\hbox{\rm\rlap c/}}}{}%
\@ifundefined{tciLaplace}{\def\tciLaplace{\ensuremath{\mathcal{L}}}}{}%
\@ifundefined{tciFourier}{\def\tciFourier{\ensuremath{\mathcal{F}}}}{}%
\@ifundefined{textcurrency}{\def\textcurrency{\hbox{\rm\rlap xo}}}{}%
\@ifundefined{texteuro}{\def\texteuro{\hbox{\rm\rlap C=}}}{}%
\@ifundefined{euro}{\def\euro{\hbox{\rm\rlap C=}}}{}%
\@ifundefined{textfranc}{\def\textfranc{\hbox{\rm\rlap-F}}}{}%
\@ifundefined{textlira}{\def\textlira{\hbox{\rm\rlap L=}}}{}%
\@ifundefined{textpeseta}{\def\textpeseta{\hbox{\rm P\negthinspace s}}}{}%
%
\@ifundefined{miss}{\def\miss{\hbox{\vrule height2\p@ width 2\p@ depth\z@}}}{}%
%
\@ifundefined{vvert}{\def\vvert{\Vert}}{}%  %always translated to \left| or \right|
%
\@ifundefined{tcol}{\def\tcol#1{{\baselineskip=6\p@ \vcenter{#1}} \Column}}{}%
%
\@ifundefined{dB}{\def\dB{\hbox{{}}}}{}%        %dummy entry in column 
\@ifundefined{mB}{\def\mB#1{\hbox{$#1$}}}{}%   %column entry
\@ifundefined{nB}{\def\nB#1{\hbox{#1}}}{}%     %column entry (not math)
%
\@ifundefined{note}{\def\note{$^{\dag}}}{}%
%
\def\newfmtname{LaTeX2e}
% No longer load latexsym.  This is now handled by SWP, which uses amsfonts if necessary
%
\ifx\fmtname\newfmtname
  \DeclareOldFontCommand{\rm}{\normalfont\rmfamily}{\mathrm}
  \DeclareOldFontCommand{\sf}{\normalfont\sffamily}{\mathsf}
  \DeclareOldFontCommand{\tt}{\normalfont\ttfamily}{\mathtt}
  \DeclareOldFontCommand{\bf}{\normalfont\bfseries}{\mathbf}
  \DeclareOldFontCommand{\it}{\normalfont\itshape}{\mathit}
  \DeclareOldFontCommand{\sl}{\normalfont\slshape}{\@nomath\sl}
  \DeclareOldFontCommand{\sc}{\normalfont\scshape}{\@nomath\sc}
\fi

%
% Greek bold macros
% Redefine all of the math symbols 
% which might be bolded	 - there are 
% probably others to add to this list

\def\alpha{{\Greekmath 010B}}%
\def\beta{{\Greekmath 010C}}%
\def\gamma{{\Greekmath 010D}}%
\def\delta{{\Greekmath 010E}}%
\def\epsilon{{\Greekmath 010F}}%
\def\zeta{{\Greekmath 0110}}%
\def\eta{{\Greekmath 0111}}%
\def\theta{{\Greekmath 0112}}%
\def\iota{{\Greekmath 0113}}%
\def\kappa{{\Greekmath 0114}}%
\def\lambda{{\Greekmath 0115}}%
\def\mu{{\Greekmath 0116}}%
\def\nu{{\Greekmath 0117}}%
\def\xi{{\Greekmath 0118}}%
\def\pi{{\Greekmath 0119}}%
\def\rho{{\Greekmath 011A}}%
\def\sigma{{\Greekmath 011B}}%
\def\tau{{\Greekmath 011C}}%
\def\upsilon{{\Greekmath 011D}}%
\def\phi{{\Greekmath 011E}}%
\def\chi{{\Greekmath 011F}}%
\def\psi{{\Greekmath 0120}}%
\def\omega{{\Greekmath 0121}}%
\def\varepsilon{{\Greekmath 0122}}%
\def\vartheta{{\Greekmath 0123}}%
\def\varpi{{\Greekmath 0124}}%
\def\varrho{{\Greekmath 0125}}%
\def\varsigma{{\Greekmath 0126}}%
\def\varphi{{\Greekmath 0127}}%

\def\nabla{{\Greekmath 0272}}
\def\FindBoldGroup{%
   {\setbox0=\hbox{$\mathbf{x\global\edef\theboldgroup{\the\mathgroup}}$}}%
}

\def\Greekmath#1#2#3#4{%
    \if@compatibility
        \ifnum\mathgroup=\symbold
           \mathchoice{\mbox{\boldmath$\displaystyle\mathchar"#1#2#3#4$}}%
                      {\mbox{\boldmath$\textstyle\mathchar"#1#2#3#4$}}%
                      {\mbox{\boldmath$\scriptstyle\mathchar"#1#2#3#4$}}%
                      {\mbox{\boldmath$\scriptscriptstyle\mathchar"#1#2#3#4$}}%
        \else
           \mathchar"#1#2#3#4% 
        \fi 
    \else 
        \FindBoldGroup
        \ifnum\mathgroup=\theboldgroup % For 2e
           \mathchoice{\mbox{\boldmath$\displaystyle\mathchar"#1#2#3#4$}}%
                      {\mbox{\boldmath$\textstyle\mathchar"#1#2#3#4$}}%
                      {\mbox{\boldmath$\scriptstyle\mathchar"#1#2#3#4$}}%
                      {\mbox{\boldmath$\scriptscriptstyle\mathchar"#1#2#3#4$}}%
        \else
           \mathchar"#1#2#3#4% 
        \fi     	    
	  \fi}

\newif\ifGreekBold  \GreekBoldfalse
\let\SAVEPBF=\pbf
\def\pbf{\GreekBoldtrue\SAVEPBF}%
%

\@ifundefined{theorem}{\newtheorem{theorem}{Theorem}}{}
\@ifundefined{lemma}{\newtheorem{lemma}[theorem]{Lemma}}{}
\@ifundefined{corollary}{\newtheorem{corollary}[theorem]{Corollary}}{}
\@ifundefined{conjecture}{\newtheorem{conjecture}[theorem]{Conjecture}}{}
\@ifundefined{proposition}{\newtheorem{proposition}[theorem]{Proposition}}{}
\@ifundefined{axiom}{\newtheorem{axiom}{Axiom}}{}
\@ifundefined{remark}{\newtheorem{remark}{Remark}}{}
\@ifundefined{example}{\newtheorem{example}{Example}}{}
\@ifundefined{exercise}{\newtheorem{exercise}{Exercise}}{}
\@ifundefined{definition}{\newtheorem{definition}{Definition}}{}


\@ifundefined{mathletters}{%
  %\def\theequation{\arabic{equation}}
  \newcounter{equationnumber}  
  \def\mathletters{%
     \addtocounter{equation}{1}
     \edef\@currentlabel{\theequation}%
     \setcounter{equationnumber}{\c@equation}
     \setcounter{equation}{0}%
     \edef\theequation{\@currentlabel\noexpand\alph{equation}}%
  }
  \def\endmathletters{%
     \setcounter{equation}{\value{equationnumber}}%
  }
}{}

%Logos
\@ifundefined{BibTeX}{%
    \def\BibTeX{{\rm B\kern-.05em{\sc i\kern-.025em b}\kern-.08em
                 T\kern-.1667em\lower.7ex\hbox{E}\kern-.125emX}}}{}%
\@ifundefined{AmS}%
    {\def\AmS{{\protect\usefont{OMS}{cmsy}{m}{n}%
                A\kern-.1667em\lower.5ex\hbox{M}\kern-.125emS}}}{}%
\@ifundefined{AmSTeX}{\def\AmSTeX{\protect\AmS-\protect\TeX\@}}{}%
%

% This macro is a fix to eqnarray
\def\@@eqncr{\let\@tempa\relax
    \ifcase\@eqcnt \def\@tempa{& & &}\or \def\@tempa{& &}%
      \else \def\@tempa{&}\fi
     \@tempa
     \if@eqnsw
        \iftag@
           \@taggnum
        \else
           \@eqnnum\stepcounter{equation}%
        \fi
     \fi
     \global\tag@false
     \global\@eqnswtrue
     \global\@eqcnt\z@\cr}


\def\TCItag{\@ifnextchar*{\@TCItagstar}{\@TCItag}}
\def\@TCItag#1{%
    \global\tag@true
    \global\def\@taggnum{(#1)}%
    \global\def\@currentlabel{#1}}
\def\@TCItagstar*#1{%
    \global\tag@true
    \global\def\@taggnum{#1}%
    \global\def\@currentlabel{#1}}
%
%%%%%%%%%%%%%%%%%%%%%%%%%%%%%%%%%%%%%%%%%%%%%%%%%%%%%%%%%%%%%%%%%%%%%
%
\def\QATOP#1#2{{#1 \atop #2}}%
\def\QTATOP#1#2{{\textstyle {#1 \atop #2}}}%
\def\QDATOP#1#2{{\displaystyle {#1 \atop #2}}}%
\def\QABOVE#1#2#3{{#2 \above#1 #3}}%
\def\QTABOVE#1#2#3{{\textstyle {#2 \above#1 #3}}}%
\def\QDABOVE#1#2#3{{\displaystyle {#2 \above#1 #3}}}%
\def\QOVERD#1#2#3#4{{#3 \overwithdelims#1#2 #4}}%
\def\QTOVERD#1#2#3#4{{\textstyle {#3 \overwithdelims#1#2 #4}}}%
\def\QDOVERD#1#2#3#4{{\displaystyle {#3 \overwithdelims#1#2 #4}}}%
\def\QATOPD#1#2#3#4{{#3 \atopwithdelims#1#2 #4}}%
\def\QTATOPD#1#2#3#4{{\textstyle {#3 \atopwithdelims#1#2 #4}}}%
\def\QDATOPD#1#2#3#4{{\displaystyle {#3 \atopwithdelims#1#2 #4}}}%
\def\QABOVED#1#2#3#4#5{{#4 \abovewithdelims#1#2#3 #5}}%
\def\QTABOVED#1#2#3#4#5{{\textstyle 
   {#4 \abovewithdelims#1#2#3 #5}}}%
\def\QDABOVED#1#2#3#4#5{{\displaystyle 
   {#4 \abovewithdelims#1#2#3 #5}}}%
%
% Macros for text size operators:
%
\def\tint{\mathop{\textstyle \int}}%
\def\tiint{\mathop{\textstyle \iint }}%
\def\tiiint{\mathop{\textstyle \iiint }}%
\def\tiiiint{\mathop{\textstyle \iiiint }}%
\def\tidotsint{\mathop{\textstyle \idotsint }}%
\def\toint{\mathop{\textstyle \oint}}%
\def\tsum{\mathop{\textstyle \sum }}%
\def\tprod{\mathop{\textstyle \prod }}%
\def\tbigcap{\mathop{\textstyle \bigcap }}%
\def\tbigwedge{\mathop{\textstyle \bigwedge }}%
\def\tbigoplus{\mathop{\textstyle \bigoplus }}%
\def\tbigodot{\mathop{\textstyle \bigodot }}%
\def\tbigsqcup{\mathop{\textstyle \bigsqcup }}%
\def\tcoprod{\mathop{\textstyle \coprod }}%
\def\tbigcup{\mathop{\textstyle \bigcup }}%
\def\tbigvee{\mathop{\textstyle \bigvee }}%
\def\tbigotimes{\mathop{\textstyle \bigotimes }}%
\def\tbiguplus{\mathop{\textstyle \biguplus }}%
%
%
%Macros for display size operators:
%
\def\dint{\mathop{\displaystyle \int}}%
\def\diint{\mathop{\displaystyle \iint}}%
\def\diiint{\mathop{\displaystyle \iiint}}%
\def\diiiint{\mathop{\displaystyle \iiiint }}%
\def\didotsint{\mathop{\displaystyle \idotsint }}%
\def\doint{\mathop{\displaystyle \oint}}%
\def\dsum{\mathop{\displaystyle \sum }}%
\def\dprod{\mathop{\displaystyle \prod }}%
\def\dbigcap{\mathop{\displaystyle \bigcap }}%
\def\dbigwedge{\mathop{\displaystyle \bigwedge }}%
\def\dbigoplus{\mathop{\displaystyle \bigoplus }}%
\def\dbigodot{\mathop{\displaystyle \bigodot }}%
\def\dbigsqcup{\mathop{\displaystyle \bigsqcup }}%
\def\dcoprod{\mathop{\displaystyle \coprod }}%
\def\dbigcup{\mathop{\displaystyle \bigcup }}%
\def\dbigvee{\mathop{\displaystyle \bigvee }}%
\def\dbigotimes{\mathop{\displaystyle \bigotimes }}%
\def\dbiguplus{\mathop{\displaystyle \biguplus }}%

\if@compatibility\else
  % Always load amsmath in LaTeX2e mode
  \RequirePackage{amsmath}
\fi

\def\ExitTCILatex{\makeatother\endinput}

\bgroup
\ifx\ds@amstex\relax
   \message{amstex already loaded}\aftergroup\ExitTCILatex
\else
   \@ifpackageloaded{amsmath}%
      {\if@compatibility\message{amsmath already loaded}\fi\aftergroup\ExitTCILatex}
      {}
   \@ifpackageloaded{amstex}%
      {\if@compatibility\message{amstex already loaded}\fi\aftergroup\ExitTCILatex}
      {}
   \@ifpackageloaded{amsgen}%
      {\if@compatibility\message{amsgen already loaded}\fi\aftergroup\ExitTCILatex}
      {}
\fi
\egroup

%Exit if any of the AMS macros are already loaded.
%This is always the case for LaTeX2e mode.


%%%%%%%%%%%%%%%%%%%%%%%%%%%%%%%%%%%%%%%%%%%%%%%%%%%%%%%%%%%%%%%%%%%%%%%%%%
% NOTE: The rest of this file is read only if in LaTeX 2.09 compatibility
% mode. This section is used to define AMS-like constructs in the
% event they have not been defined.
%%%%%%%%%%%%%%%%%%%%%%%%%%%%%%%%%%%%%%%%%%%%%%%%%%%%%%%%%%%%%%%%%%%%%%%%%%
\typeout{TCILATEX defining AMS-like constructs in LaTeX 2.09 COMPATIBILITY MODE}
%%%%%%%%%%%%%%%%%%%%%%%%%%%%%%%%%%%%%%%%%%%%%%%%%%%%%%%%%%%%%%%%%%%%%%%%
%  Macros to define some AMS LaTeX constructs when 
%  AMS LaTeX has not been loaded
% 
% These macros are copied from the AMS-TeX package for doing
% multiple integrals.
%
\let\DOTSI\relax
\def\RIfM@{\relax\ifmmode}%
\def\FN@{\futurelet\next}%
\newcount\intno@
\def\iint{\DOTSI\intno@\tw@\FN@\ints@}%
\def\iiint{\DOTSI\intno@\thr@@\FN@\ints@}%
\def\iiiint{\DOTSI\intno@4 \FN@\ints@}%
\def\idotsint{\DOTSI\intno@\z@\FN@\ints@}%
\def\ints@{\findlimits@\ints@@}%
\newif\iflimtoken@
\newif\iflimits@
\def\findlimits@{\limtoken@true\ifx\next\limits\limits@true
 \else\ifx\next\nolimits\limits@false\else
 \limtoken@false\ifx\ilimits@\nolimits\limits@false\else
 \ifinner\limits@false\else\limits@true\fi\fi\fi\fi}%
\def\multint@{\int\ifnum\intno@=\z@\intdots@                          %1
 \else\intkern@\fi                                                    %2
 \ifnum\intno@>\tw@\int\intkern@\fi                                   %3
 \ifnum\intno@>\thr@@\int\intkern@\fi                                 %4
 \int}%                                                               %5
\def\multintlimits@{\intop\ifnum\intno@=\z@\intdots@\else\intkern@\fi
 \ifnum\intno@>\tw@\intop\intkern@\fi
 \ifnum\intno@>\thr@@\intop\intkern@\fi\intop}%
\def\intic@{%
    \mathchoice{\hskip.5em}{\hskip.4em}{\hskip.4em}{\hskip.4em}}%
\def\negintic@{\mathchoice
 {\hskip-.5em}{\hskip-.4em}{\hskip-.4em}{\hskip-.4em}}%
\def\ints@@{\iflimtoken@                                              %1
 \def\ints@@@{\iflimits@\negintic@
   \mathop{\intic@\multintlimits@}\limits                             %2
  \else\multint@\nolimits\fi                                          %3
  \eat@}%                                                             %4
 \else                                                                %5
 \def\ints@@@{\iflimits@\negintic@
  \mathop{\intic@\multintlimits@}\limits\else
  \multint@\nolimits\fi}\fi\ints@@@}%
\def\intkern@{\mathchoice{\!\!\!}{\!\!}{\!\!}{\!\!}}%
\def\plaincdots@{\mathinner{\cdotp\cdotp\cdotp}}%
\def\intdots@{\mathchoice{\plaincdots@}%
 {{\cdotp}\mkern1.5mu{\cdotp}\mkern1.5mu{\cdotp}}%
 {{\cdotp}\mkern1mu{\cdotp}\mkern1mu{\cdotp}}%
 {{\cdotp}\mkern1mu{\cdotp}\mkern1mu{\cdotp}}}%
%
%
%  These macros are for doing the AMS \text{} construct
%
\def\RIfM@{\relax\protect\ifmmode}
\def\text{\RIfM@\expandafter\text@\else\expandafter\mbox\fi}
\let\nfss@text\text
\def\text@#1{\mathchoice
   {\textdef@\displaystyle\f@size{#1}}%
   {\textdef@\textstyle\tf@size{\firstchoice@false #1}}%
   {\textdef@\textstyle\sf@size{\firstchoice@false #1}}%
   {\textdef@\textstyle \ssf@size{\firstchoice@false #1}}%
   \glb@settings}

\def\textdef@#1#2#3{\hbox{{%
                    \everymath{#1}%
                    \let\f@size#2\selectfont
                    #3}}}
\newif\iffirstchoice@
\firstchoice@true
%
%These are the AMS constructs for multiline limits.
%
\def\Let@{\relax\iffalse{\fi\let\\=\cr\iffalse}\fi}%
\def\vspace@{\def\vspace##1{\crcr\noalign{\vskip##1\relax}}}%
\def\multilimits@{\bgroup\vspace@\Let@
 \baselineskip\fontdimen10 \scriptfont\tw@
 \advance\baselineskip\fontdimen12 \scriptfont\tw@
 \lineskip\thr@@\fontdimen8 \scriptfont\thr@@
 \lineskiplimit\lineskip
 \vbox\bgroup\ialign\bgroup\hfil$\m@th\scriptstyle{##}$\hfil\crcr}%
\def\Sb{_\multilimits@}%
\def\endSb{\crcr\egroup\egroup\egroup}%
\def\Sp{^\multilimits@}%
\let\endSp\endSb
%
%
%These are AMS constructs for horizontal arrows
%
\newdimen\ex@
\ex@.2326ex
\def\rightarrowfill@#1{$#1\m@th\mathord-\mkern-6mu\cleaders
 \hbox{$#1\mkern-2mu\mathord-\mkern-2mu$}\hfill
 \mkern-6mu\mathord\rightarrow$}%
\def\leftarrowfill@#1{$#1\m@th\mathord\leftarrow\mkern-6mu\cleaders
 \hbox{$#1\mkern-2mu\mathord-\mkern-2mu$}\hfill\mkern-6mu\mathord-$}%
\def\leftrightarrowfill@#1{$#1\m@th\mathord\leftarrow
\mkern-6mu\cleaders
 \hbox{$#1\mkern-2mu\mathord-\mkern-2mu$}\hfill
 \mkern-6mu\mathord\rightarrow$}%
\def\overrightarrow{\mathpalette\overrightarrow@}%
\def\overrightarrow@#1#2{\vbox{\ialign{##\crcr\rightarrowfill@#1\crcr
 \noalign{\kern-\ex@\nointerlineskip}$\m@th\hfil#1#2\hfil$\crcr}}}%
\let\overarrow\overrightarrow
\def\overleftarrow{\mathpalette\overleftarrow@}%
\def\overleftarrow@#1#2{\vbox{\ialign{##\crcr\leftarrowfill@#1\crcr
 \noalign{\kern-\ex@\nointerlineskip}$\m@th\hfil#1#2\hfil$\crcr}}}%
\def\overleftrightarrow{\mathpalette\overleftrightarrow@}%
\def\overleftrightarrow@#1#2{\vbox{\ialign{##\crcr
   \leftrightarrowfill@#1\crcr
 \noalign{\kern-\ex@\nointerlineskip}$\m@th\hfil#1#2\hfil$\crcr}}}%
\def\underrightarrow{\mathpalette\underrightarrow@}%
\def\underrightarrow@#1#2{\vtop{\ialign{##\crcr$\m@th\hfil#1#2\hfil
  $\crcr\noalign{\nointerlineskip}\rightarrowfill@#1\crcr}}}%
\let\underarrow\underrightarrow
\def\underleftarrow{\mathpalette\underleftarrow@}%
\def\underleftarrow@#1#2{\vtop{\ialign{##\crcr$\m@th\hfil#1#2\hfil
  $\crcr\noalign{\nointerlineskip}\leftarrowfill@#1\crcr}}}%
\def\underleftrightarrow{\mathpalette\underleftrightarrow@}%
\def\underleftrightarrow@#1#2{\vtop{\ialign{##\crcr$\m@th
  \hfil#1#2\hfil$\crcr
 \noalign{\nointerlineskip}\leftrightarrowfill@#1\crcr}}}%
%%%%%%%%%%%%%%%%%%%%%

\def\qopnamewl@#1{\mathop{\operator@font#1}\nlimits@}
\let\nlimits@\displaylimits
\def\setboxz@h{\setbox\z@\hbox}


\def\varlim@#1#2{\mathop{\vtop{\ialign{##\crcr
 \hfil$#1\m@th\operator@font lim$\hfil\crcr
 \noalign{\nointerlineskip}#2#1\crcr
 \noalign{\nointerlineskip\kern-\ex@}\crcr}}}}

 \def\rightarrowfill@#1{\m@th\setboxz@h{$#1-$}\ht\z@\z@
  $#1\copy\z@\mkern-6mu\cleaders
  \hbox{$#1\mkern-2mu\box\z@\mkern-2mu$}\hfill
  \mkern-6mu\mathord\rightarrow$}
\def\leftarrowfill@#1{\m@th\setboxz@h{$#1-$}\ht\z@\z@
  $#1\mathord\leftarrow\mkern-6mu\cleaders
  \hbox{$#1\mkern-2mu\copy\z@\mkern-2mu$}\hfill
  \mkern-6mu\box\z@$}


\def\projlim{\qopnamewl@{proj\,lim}}
\def\injlim{\qopnamewl@{inj\,lim}}
\def\varinjlim{\mathpalette\varlim@\rightarrowfill@}
\def\varprojlim{\mathpalette\varlim@\leftarrowfill@}
\def\varliminf{\mathpalette\varliminf@{}}
\def\varliminf@#1{\mathop{\underline{\vrule\@depth.2\ex@\@width\z@
   \hbox{$#1\m@th\operator@font lim$}}}}
\def\varlimsup{\mathpalette\varlimsup@{}}
\def\varlimsup@#1{\mathop{\overline
  {\hbox{$#1\m@th\operator@font lim$}}}}

%
%Companion to stackrel
\def\stackunder#1#2{\mathrel{\mathop{#2}\limits_{#1}}}%
%
%
% These are AMS environments that will be defined to
% be verbatims if amstex has not actually been 
% loaded
%
%
\begingroup \catcode `|=0 \catcode `[= 1
\catcode`]=2 \catcode `\{=12 \catcode `\}=12
\catcode`\\=12 
|gdef|@alignverbatim#1\end{align}[#1|end[align]]
|gdef|@salignverbatim#1\end{align*}[#1|end[align*]]

|gdef|@alignatverbatim#1\end{alignat}[#1|end[alignat]]
|gdef|@salignatverbatim#1\end{alignat*}[#1|end[alignat*]]

|gdef|@xalignatverbatim#1\end{xalignat}[#1|end[xalignat]]
|gdef|@sxalignatverbatim#1\end{xalignat*}[#1|end[xalignat*]]

|gdef|@gatherverbatim#1\end{gather}[#1|end[gather]]
|gdef|@sgatherverbatim#1\end{gather*}[#1|end[gather*]]

|gdef|@gatherverbatim#1\end{gather}[#1|end[gather]]
|gdef|@sgatherverbatim#1\end{gather*}[#1|end[gather*]]


|gdef|@multilineverbatim#1\end{multiline}[#1|end[multiline]]
|gdef|@smultilineverbatim#1\end{multiline*}[#1|end[multiline*]]

|gdef|@arraxverbatim#1\end{arrax}[#1|end[arrax]]
|gdef|@sarraxverbatim#1\end{arrax*}[#1|end[arrax*]]

|gdef|@tabulaxverbatim#1\end{tabulax}[#1|end[tabulax]]
|gdef|@stabulaxverbatim#1\end{tabulax*}[#1|end[tabulax*]]


|endgroup
  

  
\def\align{\@verbatim \frenchspacing\@vobeyspaces \@alignverbatim
You are using the "align" environment in a style in which it is not defined.}
\let\endalign=\endtrivlist
 
\@namedef{align*}{\@verbatim\@salignverbatim
You are using the "align*" environment in a style in which it is not defined.}
\expandafter\let\csname endalign*\endcsname =\endtrivlist




\def\alignat{\@verbatim \frenchspacing\@vobeyspaces \@alignatverbatim
You are using the "alignat" environment in a style in which it is not defined.}
\let\endalignat=\endtrivlist
 
\@namedef{alignat*}{\@verbatim\@salignatverbatim
You are using the "alignat*" environment in a style in which it is not defined.}
\expandafter\let\csname endalignat*\endcsname =\endtrivlist




\def\xalignat{\@verbatim \frenchspacing\@vobeyspaces \@xalignatverbatim
You are using the "xalignat" environment in a style in which it is not defined.}
\let\endxalignat=\endtrivlist
 
\@namedef{xalignat*}{\@verbatim\@sxalignatverbatim
You are using the "xalignat*" environment in a style in which it is not defined.}
\expandafter\let\csname endxalignat*\endcsname =\endtrivlist




\def\gather{\@verbatim \frenchspacing\@vobeyspaces \@gatherverbatim
You are using the "gather" environment in a style in which it is not defined.}
\let\endgather=\endtrivlist
 
\@namedef{gather*}{\@verbatim\@sgatherverbatim
You are using the "gather*" environment in a style in which it is not defined.}
\expandafter\let\csname endgather*\endcsname =\endtrivlist


\def\multiline{\@verbatim \frenchspacing\@vobeyspaces \@multilineverbatim
You are using the "multiline" environment in a style in which it is not defined.}
\let\endmultiline=\endtrivlist
 
\@namedef{multiline*}{\@verbatim\@smultilineverbatim
You are using the "multiline*" environment in a style in which it is not defined.}
\expandafter\let\csname endmultiline*\endcsname =\endtrivlist


\def\arrax{\@verbatim \frenchspacing\@vobeyspaces \@arraxverbatim
You are using a type of "array" construct that is only allowed in AmS-LaTeX.}
\let\endarrax=\endtrivlist

\def\tabulax{\@verbatim \frenchspacing\@vobeyspaces \@tabulaxverbatim
You are using a type of "tabular" construct that is only allowed in AmS-LaTeX.}
\let\endtabulax=\endtrivlist

 
\@namedef{arrax*}{\@verbatim\@sarraxverbatim
You are using a type of "array*" construct that is only allowed in AmS-LaTeX.}
\expandafter\let\csname endarrax*\endcsname =\endtrivlist

\@namedef{tabulax*}{\@verbatim\@stabulaxverbatim
You are using a type of "tabular*" construct that is only allowed in AmS-LaTeX.}
\expandafter\let\csname endtabulax*\endcsname =\endtrivlist

% macro to simulate ams tag construct


% This macro is a fix to the equation environment
 \def\endequation{%
     \ifmmode\ifinner % FLEQN hack
      \iftag@
        \addtocounter{equation}{-1} % undo the increment made in the begin part
        $\hfil
           \displaywidth\linewidth\@taggnum\egroup \endtrivlist
        \global\tag@false
        \global\@ignoretrue   
      \else
        $\hfil
           \displaywidth\linewidth\@eqnnum\egroup \endtrivlist
        \global\tag@false
        \global\@ignoretrue 
      \fi
     \else   
      \iftag@
        \addtocounter{equation}{-1} % undo the increment made in the begin part
        \eqno \hbox{\@taggnum}
        \global\tag@false%
        $$\global\@ignoretrue
      \else
        \eqno \hbox{\@eqnnum}% $$ BRACE MATCHING HACK
        $$\global\@ignoretrue
      \fi
     \fi\fi
 } 

 \newif\iftag@ \tag@false
 
 \def\TCItag{\@ifnextchar*{\@TCItagstar}{\@TCItag}}
 \def\@TCItag#1{%
     \global\tag@true
     \global\def\@taggnum{(#1)}%
     \global\def\@currentlabel{#1}}
 \def\@TCItagstar*#1{%
     \global\tag@true
     \global\def\@taggnum{#1}%
     \global\def\@currentlabel{#1}}

  \@ifundefined{tag}{
     \def\tag{\@ifnextchar*{\@tagstar}{\@tag}}
     \def\@tag#1{%
         \global\tag@true
         \global\def\@taggnum{(#1)}}
     \def\@tagstar*#1{%
         \global\tag@true
         \global\def\@taggnum{#1}}
  }{}

\def\tfrac#1#2{{\textstyle {#1 \over #2}}}%
\def\dfrac#1#2{{\displaystyle {#1 \over #2}}}%
\def\binom#1#2{{#1 \choose #2}}%
\def\tbinom#1#2{{\textstyle {#1 \choose #2}}}%
\def\dbinom#1#2{{\displaystyle {#1 \choose #2}}}%

% Do not add anything to the end of this file.  
% The last section of the file is loaded only if 
% amstex has not been.
\makeatother
\endinput

\DeclareMathAccent{\wtilde}{\mathord}{largesymbols}{"65}
\pagestyle{fancy}
\fancyfoot[C]{\thepage}


\begin{document}


\setcounter{part}{6} \setcounter{page}{38}

\bigskip

\part{Sensitivity Analysis}

\begin{equation*}
%TCIMACRO{\TeXButton{underaccent_Y}{\underaccent{\wtilde}{Y}}}%
%BeginExpansion
\underaccent{\wtilde}{Y}%
%EndExpansion
=x%
%TCIMACRO{\TeXButton{underaccent_beta}{\underaccent{\wtilde}{\beta}}}%
%BeginExpansion
\underaccent{\wtilde}{\beta}%
%EndExpansion
+%
%TCIMACRO{\TeXButton{underaccent_epsilon}{\underaccent{\wtilde}{\epsilon}}}%
%BeginExpansion
\underaccent{\wtilde}{\epsilon}%
%EndExpansion
\quad E\left( 
%TCIMACRO{\TeXButton{underaccent_epsilon}{\underaccent{\wtilde}{\epsilon}}}%
%BeginExpansion
\underaccent{\wtilde}{\epsilon}%
%EndExpansion
\right)
\end{equation*}%
\begin{equation*}
Cov\left( 
%TCIMACRO{\TeXButton{underaccent_epsilon}{\underaccent{\wtilde}{\epsilon}}}%
%BeginExpansion
\underaccent{\wtilde}{\epsilon}%
%EndExpansion
\right) =\sigma ^{2}I\quad \hat{\beta}=\left( x^{\dagger }x^{-1}\right)
x^{\dagger }Y
\end{equation*}

$x\hat{\beta}\longrightarrow x\beta $, $%
%TCIMACRO{\TeXButton{underaccent_e}{\underaccent{\wtilde}{e}}}%
%BeginExpansion
\underaccent{\wtilde}{e}%
%EndExpansion
=\left( 
%TCIMACRO{\TeXButton{underaccent_Y}{\underaccent{\wtilde}{Y}}}%
%BeginExpansion
\underaccent{\wtilde}{Y}%
%EndExpansion
-x\hat{\beta}\right) \longrightarrow $,$%
%TCIMACRO{\TeXButton{underaccent_epsilon}{\underaccent{\wtilde}{\epsilon}}}%
%BeginExpansion
\underaccent{\wtilde}{\epsilon}%
%EndExpansion
$ use $%
%TCIMACRO{\TeXButton{underaccent_e}{\underaccent{\wtilde}{e}}}%
%BeginExpansion
\underaccent{\wtilde}{e}%
%EndExpansion
$ behaviour to understand the model's propernous.

\begin{eqnarray*}
\hat{Y} &=&x\hat{\beta} \\
&=&x\left( x^{\dagger }x^{-1}\right) x^{\dagger }%
%TCIMACRO{\TeXButton{underaccent_Y}{\underaccent{\wtilde}{Y}}}%
%BeginExpansion
\underaccent{\wtilde}{Y}%
%EndExpansion
\quad \text{rank(x)=k} \\
&=&PY \\
&=&\left[ P_{ij}\right] 
%TCIMACRO{\TeXButton{underaccent_Y}{\underaccent{\wtilde}{Y}} }%
%BeginExpansion
\underaccent{\wtilde}{Y}
%EndExpansion
\\
&=&P_{i}%
%TCIMACRO{\TeXButton{underaccent_Y}{\underaccent{\wtilde}{Y}} }%
%BeginExpansion
\underaccent{\wtilde}{Y}
%EndExpansion
\\
&=&P_{i}\left[ 
\begin{array}{ccc}
Y_{1} & \cdots & Y_{n}%
\end{array}%
\right] ^{\dagger }
\end{eqnarray*}%
P: prediction matrix (i.e. $\hat{Y}=PY$), I-P: residual matrix (i.e. $%
%TCIMACRO{\TeXButton{underaccent_e}{\underaccent{\wtilde}{e}}}%
%BeginExpansion
\underaccent{\wtilde}{e}%
%EndExpansion
=\left( I-P\right) Y$)

\begin{equation*}
%TCIMACRO{\TeXButton{underaccent_e}{\underaccent{\wtilde}{e}}}%
%BeginExpansion
\underaccent{\wtilde}{e}%
%EndExpansion
=%
%TCIMACRO{\TeXButton{underaccent_Y}{\underaccent{\wtilde}{Y}}}%
%BeginExpansion
\underaccent{\wtilde}{Y}%
%EndExpansion
-%
%TCIMACRO{\TeXButton{underaccent_Y_hat}{\underaccent{\wtilde}{\hat{Y}}}}%
%BeginExpansion
\underaccent{\wtilde}{\hat{Y}}%
%EndExpansion
=%
%TCIMACRO{\TeXButton{underaccent_Y}{\underaccent{\wtilde}{Y}}}%
%BeginExpansion
\underaccent{\wtilde}{Y}%
%EndExpansion
-x\hat{\beta}
\end{equation*}%
\begin{equation*}
=\left[ I-x\left( x^{\dagger }x^{-1}\right) x^{\dagger }\right] 
%TCIMACRO{\TeXButton{underaccent_Y}{\underaccent{\wtilde}{Y}}}%
%BeginExpansion
\underaccent{\wtilde}{Y}%
%EndExpansion
=\left( I-P\right) 
%TCIMACRO{\TeXButton{underaccent_Y}{\underaccent{\wtilde}{Y}}}%
%BeginExpansion
\underaccent{\wtilde}{Y}%
%EndExpansion
=\underset{%
\begin{array}{c}
\text{indempotent} \\ 
\text{matrix}%
\end{array}%
}{M}%
%TCIMACRO{\TeXButton{underaccent_Y}{\underaccent{\wtilde}{Y}}}%
%BeginExpansion
\underaccent{\wtilde}{Y}%
%EndExpansion
\end{equation*}

\bigskip

\begin{itemize}
\item What are the similarities and disimilarities of $%
%TCIMACRO{\TeXButton{underaccent_e}{\underaccent{\wtilde}{e}}}%
%BeginExpansion
\underaccent{\wtilde}{e}%
%EndExpansion
$ and $%
%TCIMACRO{\TeXButton{underaccent_epsilon}{\underaccent{\wtilde}{\epsilon}}}%
%BeginExpansion
\underaccent{\wtilde}{\epsilon}%
%EndExpansion
$%
\begin{equation*}
\chi =\left[ x_{ij}\right] \quad \text{i=1}\cdots \text{n, j=1}\cdots \text{K%
}\qquad P_{ij}=%
%TCIMACRO{\TeXButton{underaccent_x}{\underaccent{\wtilde}{x}}}%
%BeginExpansion
\underaccent{\wtilde}{x}%
%EndExpansion
_{i}^{\dagger }\left( x^{\dagger }x^{-1}\right) ^{-1}%
%TCIMACRO{\TeXButton{underaccent_x}{\underaccent{\wtilde}{x}}}%
%BeginExpansion
\underaccent{\wtilde}{x}%
%EndExpansion
_{j}
\end{equation*}%
\begin{equation*}
\hat{Y}_{1}=P_{1}%
%TCIMACRO{\TeXButton{underaccent_Y}{\underaccent{\wtilde}{Y}}}%
%BeginExpansion
\underaccent{\wtilde}{Y}%
%EndExpansion
=\tsum\limits_{j=1}^{n}P_{ij}Y_{j}=P_{11}Y_{1}+\tsum\limits_{j\neq
1}^{n}P_{ij}Y_{j}
\end{equation*}%
$P_{ij}$: leverage $\frac{\partial \hat{Y}_{i}}{\partial Y_{i}}=P_{ii}$%
\begin{eqnarray*}
Cov\left( \hat{Y}\right) &=&Cov\left( x\hat{\beta}\right) =xCov\left( \hat{%
\beta}\right) x^{\dagger }=x\left( x^{\dagger }x^{-1}\right) ^{-1}\sigma
^{2}x^{\dagger } \\
&=&\sigma ^{2}x\left( x^{\dagger }x^{-1}\right) ^{-1}x^{\dagger }=P\sigma
^{2} \\
&=&\left[ 
\begin{array}{cccc}
P_{11} &  &  &  \\ 
& P_{22} & P_{ij} &  \\ 
&  & \ddots &  \\ 
&  &  & P_{44}%
\end{array}%
\right] \sigma ^{2}
\end{eqnarray*}%
$\left( \text{i.e. the bigger P}_{ii}\text{, the bigger }\hat{Y}_{i}\right) $%
\begin{eqnarray*}
Var\left( 
%TCIMACRO{\TeXButton{underaccent_e}{\underaccent{\wtilde}{e}}}%
%BeginExpansion
\underaccent{\wtilde}{e}%
%EndExpansion
\right) &=&\left( I-P\right) Var\left( 
%TCIMACRO{\TeXButton{underaccent_Y}{\underaccent{\wtilde}{Y}}}%
%BeginExpansion
\underaccent{\wtilde}{Y}%
%EndExpansion
\right) \left( I-P\right) =\sigma ^{2}\left( I-P\right) \\
&=&\sigma ^{2}\left[ 
\begin{array}{cccc}
1-P_{11} &  &  &  \\ 
& 1-P_{22} & -P_{ij} &  \\ 
&  & \ddots &  \\ 
&  &  & 1-P_{44}%
\end{array}%
\right]
\end{eqnarray*}%
\begin{eqnarray*}
Var\left( \hat{Y}_{i}\right) &=&P_{ii}\sigma ^{2} \\
Var\left( \hat{\varepsilon}_{i}\right) &=&\left( 1-P_{ii}\right) \sigma ^{2}
\\
Cov\left( \hat{\varepsilon}_{i},\hat{\varepsilon}_{j}\right) &=&\sigma
^{2}\left( -P_{ij}\right)
\end{eqnarray*}%
Note: the bigger P$_{ii}$, the bigger $\hat{Y}_{i}$, but the smaller $%
Var\left( \hat{\varepsilon}_{i}\right) $, theoretically not suppose to
happen.
\end{itemize}

\bigskip

\begin{equation*}
Y_{i}=\beta _{0}+\beta _{1}x_{i}+\varepsilon _{i}=\beta _{0}^{\ast }+\beta
_{1}\left( x_{i}-\bar{x}\right) +\varepsilon _{i}
\end{equation*}%
\begin{eqnarray*}
P &=&x\left( x^{\dagger }x^{-1}\right) ^{-1}x^{\dagger } \\
&=&\left[ 
\begin{array}{cc}
1 & x_{1}-\bar{x} \\ 
\vdots & \vdots \\ 
1 & x_{n}-\bar{x}%
\end{array}%
\right] \left[ \left[ 
\begin{array}{cc}
1 & x_{1}-\bar{x} \\ 
\vdots & \vdots \\ 
1 & x_{n}-\bar{x}%
\end{array}%
\right] ^{\dagger }\left[ 
\begin{array}{cc}
1 & x_{1}-\bar{x} \\ 
\vdots & \vdots \\ 
1 & x_{n}-\bar{x}%
\end{array}%
\right] \right] \left[ 
\begin{array}{cc}
1 & x_{1}-\bar{x} \\ 
\vdots & \vdots \\ 
1 & x_{n}-\bar{x}%
\end{array}%
\right] ^{\dagger } \\
&=&\left[ 
\begin{array}{cc}
1 & x_{1}-\bar{x} \\ 
\vdots & \vdots \\ 
1 & x_{n}-\bar{x}%
\end{array}%
\right] \left[ 
\begin{array}{cc}
n & 0 \\ 
0 & \tsum \left( x_{i}-\bar{x}\right) ^{2}%
\end{array}%
\right] ^{-1}\left[ 
\begin{array}{cc}
1 & x_{1}-\bar{x} \\ 
\vdots & \vdots \\ 
1 & x_{n}-\bar{x}%
\end{array}%
\right] ^{\dagger } \\
&=&\left[ 
\begin{array}{ccc}
\frac{1}{n}+\frac{\left( x_{1}-\bar{x}\right) ^{2}}{\tsum \left( x_{i}-\bar{x%
}\right) ^{2}} &  &  \\ 
& \ddots &  \\ 
&  & \frac{1}{n}+\frac{\left( x_{n}-\bar{x}\right) ^{2}}{\tsum \left( x_{i}-%
\bar{x}\right) ^{2}}%
\end{array}%
\right]
\end{eqnarray*}

\bigskip

\begin{equation*}
\because \hat{Y}_{1}=P_{11}Y_{1}+\tsum\limits_{j=2}^{n}P_{ij}Y_{j}
\end{equation*}%
if $P_{11}$ is bigger than contribution is bigger, meaning if $P_{11}$ is
bigger than regression will approach $Y_{1}$. This means the variance of $%
Y_{1}$ is bigger.

If one point is different than the others, than the point will affect \
regression result more, dangerous!%
\begin{equation*}
P=x\left( x^{\dagger }x^{-1}\right) ^{-1}x^{\dagger }=x_{1}\left( x^{\dagger
}x^{-1}\right) ^{-1}x_{1}^{\dagger }+\left( I-P_{1}\right) \left(
x_{2}^{\dagger }\left( I-P_{1}\right) x_{2}\right) ^{-1}x_{2}^{\dagger
}\left( I-P_{1}\right)
\end{equation*}%
$P_{1}=x_{1}\left( x_{1}^{\dagger }x_{1}^{-1}\right) ^{-1}x_{1}^{\dagger }$

\bigskip

$x=\left[ 
\begin{array}{cc}
x_{1} & x_{2}%
\end{array}%
\right] $

If%
\begin{equation*}
x_{1}=\left[ 
\begin{array}{c}
1 \\ 
\vdots \\ 
1%
\end{array}%
\right] \quad x_{2}=\left[ 
\begin{array}{c}
x_{1} \\ 
\vdots \\ 
x_{n}%
\end{array}%
\right]
\end{equation*}

\begin{equation*}
P=\left[ 
\begin{array}{ccc}
\frac{1}{n} &  & \frac{1}{n} \\ 
& \ddots &  \\ 
\frac{1}{n} &  & \frac{1}{n}%
\end{array}%
\right] +\left[ 
\begin{array}{c}
x_{1}-\bar{x} \\ 
\vdots \\ 
x_{n}-\bar{x}%
\end{array}%
\right] \frac{1}{\tsum \left( x_{i}-\bar{x}\right) ^{2}}\left[ 
\begin{array}{ccc}
x_{1}-\bar{x} & \cdots & x_{n}-\bar{x}%
\end{array}%
\right]
\end{equation*}

P is leverage, P$_{ij}$ will reflect on estimation. P$_{ij}$ bigger than
effect is bigger, ....

\bigskip

P=PP (idempotency)%
\begin{equation*}
P_{ii}=\tsum\limits_{j=1}^{n}P_{ij}^{2}=P_{ii}^{2}+\tsum%
\limits_{j=2}^{n}P_{ij}^{2}\Longrightarrow 0\leq P_{ii}\leq 1
\end{equation*}%
\begin{equation*}
\left( 
\begin{array}{c}
P_{ii}\rightarrow 1\text{, meaning outlier possible} \\ 
P_{ii}\rightarrow 0\text{, meaning closer to mean}%
\end{array}%
\right)
\end{equation*}

\bigskip

Also $P_{ii}=P_{ii}^{2}+P_{ij}^{2}+\tsum\limits_{k=i,j}P_{ik}^{2}$ (j fixed)%
\begin{eqnarray*}
&\Rightarrow &P_{ij}^{2}\leq P_{ii}^{2}\left( 1-P_{ii}^{2}\right) \leq \frac{%
1}{4} \\
&\Rightarrow &-\frac{1}{2}\leq P_{ij}\leq \frac{1}{2}
\end{eqnarray*}

\begin{itemize}
\item Results

\begin{enumerate}
\item If $P_{ii}=1$, or $0\Rightarrow P_{ij}=0$%
\begin{equation*}
\hat{Y}_{1}=P_{11}Y_{1}+\tsum\limits_{j=2}^{n}P_{ij}Y_{j}=Y_{1}\quad e_{1}=0
\end{equation*}

\item $P_{ii}\times P_{jj}-P_{ij}^{2}\geq 0$%
\begin{equation*}
P=\left[ 
\begin{array}{cc}
P_{11} & P_{12} \\ 
P_{21} & P_{22}%
\end{array}%
\right] \text{ nonegative}
\end{equation*}

\item $\left( 1-P_{ii}\right) \times \left( 1-P_{jj}\right) -P_{ij}^{2}\geq
0 $, I-P: nonegative

\item $P_{ii}+\frac{e_{i}^{2}}{%
%TCIMACRO{\TeXButton{underaccent_e}{\underaccent{\wtilde}{e}}}%
%BeginExpansion
\underaccent{\wtilde}{e}%
%EndExpansion
^{\dagger }%
%TCIMACRO{\TeXButton{underaccent_e}{\underaccent{\wtilde}{e}}}%
%BeginExpansion
\underaccent{\wtilde}{e}%
%EndExpansion
}\leq 1$

\begin{itemize}
\item use this to diagnose outlier

\item diagnose reg- design

\item cook distant emphasizes $P_{ii}$

\item if $P_{ii}$ is big than $\frac{e_{i}^{2}}{%
%TCIMACRO{\TeXButton{underaccent_e}{\underaccent{\wtilde}{e}}}%
%BeginExpansion
\underaccent{\wtilde}{e}%
%EndExpansion
^{\dagger }%
%TCIMACRO{\TeXButton{underaccent_e}{\underaccent{\wtilde}{e}}}%
%BeginExpansion
\underaccent{\wtilde}{e}%
%EndExpansion
}$ is small.
\end{itemize}

\begin{proof}
\begin{equation*}
Z=\left( x,Y\right) \quad P_{x}=x\left( x^{\dagger }x^{-1}\right)
^{-1}x^{\dagger }
\end{equation*}%
\begin{eqnarray*}
P_{z} &=&x\left( x^{\dagger }x^{-1}\right) ^{-1}x^{\dagger }+\left(
I-P_{x}\right) Y\left( Y^{\dagger }\left( I-P_{x}\right) Y\right) ^{-1}\left[
\left( I-P_{x}\right) Y\right] ^{\dagger } \\
&=&P_{x}+\frac{e_{i}^{2}}{%
%TCIMACRO{\TeXButton{underaccent_e}{\underaccent{\wtilde}{e}}}%
%BeginExpansion
\underaccent{\wtilde}{e}%
%EndExpansion
^{\dagger }%
%TCIMACRO{\TeXButton{underaccent_e}{\underaccent{\wtilde}{e}}}%
%BeginExpansion
\underaccent{\wtilde}{e}%
%EndExpansion
}
\end{eqnarray*}%
\begin{equation*}
P_{ii}+\frac{e_{i}^{2}}{%
%TCIMACRO{\TeXButton{underaccent_e}{\underaccent{\wtilde}{e}}}%
%BeginExpansion
\underaccent{\wtilde}{e}%
%EndExpansion
^{\dagger }%
%TCIMACRO{\TeXButton{underaccent_e}{\underaccent{\wtilde}{e}}}%
%BeginExpansion
\underaccent{\wtilde}{e}%
%EndExpansion
}=P_{ii}^{\left( 2\right) }\leq 1
\end{equation*}%
Assume two variables%
\begin{equation*}
\text{one-dim: }\frac{\left( x_{i}-\bar{x}\right) ^{2}}{\tsum \left( x_{i}-%
\bar{x}\right) ^{2}}
\end{equation*}%
although distance the same, 1,2 have different meanings, may consider
relavent variant.%
\begin{equation*}
\Pr \left( \left( 
%TCIMACRO{\TeXButton{underaccent_x}{\underaccent{\wtilde}{x}}}%
%BeginExpansion
\underaccent{\wtilde}{x}%
%EndExpansion
-\mu \right) ^{\dagger }\tsum\nolimits^{-1}\left( 
%TCIMACRO{\TeXButton{underaccent_x}{\underaccent{\wtilde}{x}}}%
%BeginExpansion
\underaccent{\wtilde}{x}%
%EndExpansion
-\mu \right) \leq \chi _{0}^{2}\right) =0.95
\end{equation*}%
a large eig. value means long axis, small eig. value means short axis. Along
the short axis it is more likely to have large $P_{ii}$ value.
\end{proof}
\end{enumerate}
\end{itemize}

\bigskip

\begin{itemize}
\item Case of multiple regression%
\begin{equation*}
\left( \lambda _{i},e_{i}^{\ast }\right) :\text{eigenvalue-eigenvector pari
of }x^{\dagger }x
\end{equation*}%
\begin{equation*}
\theta _{ij}:\text{the angle between the colum vector }%
%TCIMACRO{\TeXButton{underaccent_x}{\underaccent{\wtilde}{x}}}%
%BeginExpansion
\underaccent{\wtilde}{x}%
%EndExpansion
_{i}\text{ and }%
%TCIMACRO{\TeXButton{underaccent_e}{\underaccent{\wtilde}{e}}}%
%BeginExpansion
\underaccent{\wtilde}{e}%
%EndExpansion
^{\ast }
\end{equation*}%
\begin{eqnarray*}
P_{ij} &=&%
%TCIMACRO{\TeXButton{underaccent_x}{\underaccent{\wtilde}{x}}}%
%BeginExpansion
\underaccent{\wtilde}{x}%
%EndExpansion
_{i}^{\dagger }\left( x^{\dagger }x^{-1}\right) ^{-1}%
%TCIMACRO{\TeXButton{underaccent_x}{\underaccent{\wtilde}{x}}}%
%BeginExpansion
\underaccent{\wtilde}{x}%
%EndExpansion
_{j} \\
&=&%
%TCIMACRO{\TeXButton{underaccent_x}{\underaccent{\wtilde}{x}}}%
%BeginExpansion
\underaccent{\wtilde}{x}%
%EndExpansion
_{i}^{\dagger }P\Lambda ^{-1}P^{\dagger }%
%TCIMACRO{\TeXButton{underaccent_x}{\underaccent{\wtilde}{x}}}%
%BeginExpansion
\underaccent{\wtilde}{x}%
%EndExpansion
_{j} \\
&=&\tsum\limits_{r=1}^{k}%
%TCIMACRO{\TeXButton{underaccent_x}{\underaccent{\wtilde}{x}}}%
%BeginExpansion
\underaccent{\wtilde}{x}%
%EndExpansion
_{i}^{\dagger }\frac{1}{\lambda r}%
%TCIMACRO{\TeXButton{underaccent_e}{\underaccent{\wtilde}{e}}}%
%BeginExpansion
\underaccent{\wtilde}{e}%
%EndExpansion
_{r}^{\dagger }%
%TCIMACRO{\TeXButton{underaccent_e}{\underaccent{\wtilde}{e}}}%
%BeginExpansion
\underaccent{\wtilde}{e}%
%EndExpansion
_{r}^{\ast \dagger }%
%TCIMACRO{\TeXButton{underaccent_x}{\underaccent{\wtilde}{x}}}%
%BeginExpansion
\underaccent{\wtilde}{x}%
%EndExpansion
_{j} \\
&=&\tsum\limits_{r=1}^{k}\frac{1}{\lambda r}%
%TCIMACRO{\TeXButton{underaccent_x}{\underaccent{\wtilde}{x}}}%
%BeginExpansion
\underaccent{\wtilde}{x}%
%EndExpansion
_{i}^{\dagger }%
%TCIMACRO{\TeXButton{underaccent_e}{\underaccent{\wtilde}{e}}}%
%BeginExpansion
\underaccent{\wtilde}{e}%
%EndExpansion
_{r}^{\dagger }%
%TCIMACRO{\TeXButton{underaccent_e}{\underaccent{\wtilde}{e}}}%
%BeginExpansion
\underaccent{\wtilde}{e}%
%EndExpansion
_{r}^{\ast \dagger }%
%TCIMACRO{\TeXButton{underaccent_x}{\underaccent{\wtilde}{x}}}%
%BeginExpansion
\underaccent{\wtilde}{x}%
%EndExpansion
_{j} \\
&=&\left\Vert 
%TCIMACRO{\TeXButton{underaccent_x}{\underaccent{\wtilde}{x}}}%
%BeginExpansion
\underaccent{\wtilde}{x}%
%EndExpansion
_{j}\right\Vert \left\Vert 
%TCIMACRO{\TeXButton{underaccent_x}{\underaccent{\wtilde}{x}}}%
%BeginExpansion
\underaccent{\wtilde}{x}%
%EndExpansion
_{j}\right\Vert \tsum\limits_{r=1}^{k}\frac{1}{\lambda r}\cos \theta
_{ir}\cos \theta _{jr}
\end{eqnarray*}%
$\cos \theta _{ir}=\frac{x_{i}^{\dagger }%
%TCIMACRO{\TeXButton{underaccent_x}{\underaccent{\wtilde}{x}}}%
%BeginExpansion
\underaccent{\wtilde}{x}%
%EndExpansion
_{j}^{\ast }}{\left\Vert 
%TCIMACRO{\TeXButton{underaccent_x}{\underaccent{\wtilde}{x}}}%
%BeginExpansion
\underaccent{\wtilde}{x}%
%EndExpansion
_{i}\right\Vert \left\Vert 
%TCIMACRO{\TeXButton{underaccent_e}{\underaccent{\wtilde}{e}}}%
%BeginExpansion
\underaccent{\wtilde}{e}%
%EndExpansion
_{j}^{\ast }\right\Vert }$

\begin{enumerate}
\item $P_{ii}$ large means $\left\Vert 
%TCIMACRO{\TeXButton{underaccent_x}{\underaccent{\wtilde}{x}}}%
%BeginExpansion
\underaccent{\wtilde}{x}%
%EndExpansion
_{i}\right\Vert \cdot \left\Vert 
%TCIMACRO{\TeXButton{underaccent_x}{\underaccent{\wtilde}{x}}}%
%BeginExpansion
\underaccent{\wtilde}{x}%
%EndExpansion
_{i}\right\Vert $ is relatively larger than other

\item $x_{i}$ and the smallest eig-value of eig-vector are parralel $\left(
\cos \theta =1\right) $
\end{enumerate}
\end{itemize}

\bigskip

Q: when k dim, how to declare this value to be the largest?%
\begin{equation*}
\text{If i=j }\Rightarrow P_{ii}=\left\Vert 
%TCIMACRO{\TeXButton{underaccent_x}{\underaccent{\wtilde}{x}}}%
%BeginExpansion
\underaccent{\wtilde}{x}%
%EndExpansion
_{i}\right\Vert ^{2}\tsum\limits_{r=1}^{k}\frac{1}{\lambda r}\left( \cos
\theta _{ir}\right) ^{2}
\end{equation*}%
$P=\left[ P_{ij}\right] $, some people take $\frac{2h\left( P\right) }{n}=%
\frac{2k}{n}$ to compare. (i.e. if $P_{ij}>2\frac{k}{n}$, it means $P_{ij}$
is large)

\bigskip

Check the good and bad of the model

\begin{equation*}
\begin{array}{c}
\text{leverage: }P_{ii} \\ 
\text{residual: }e_{i}=Y_{i}-\hat{Y}_{i}%
\end{array}%
\end{equation*}%
measure of distance of $%
%TCIMACRO{\TeXButton{underaccent_x}{\underaccent{\wtilde}{x}}}%
%BeginExpansion
\underaccent{\wtilde}{x}%
%EndExpansion
_{i}$ away from $%
%TCIMACRO{\TeXButton{underaccent_x}{\underaccent{\wtilde}{x}}}%
%BeginExpansion
\underaccent{\wtilde}{x}%
%EndExpansion
$%
\begin{equation*}
\text{multiple rows\quad }\chi _{n\times k}=\left[ 
\begin{array}{c}
%TCIMACRO{\TeXButton{underaccent_x}{\underaccent{\wtilde}{x}}}%
%BeginExpansion
\underaccent{\wtilde}{x}%
%EndExpansion
_{1}^{\dagger } \\ 
\vdots \\ 
%TCIMACRO{\TeXButton{underaccent_x}{\underaccent{\wtilde}{x}}}%
%BeginExpansion
\underaccent{\wtilde}{x}%
%EndExpansion
_{n}^{\dagger }%
\end{array}%
\right]
\end{equation*}

Let $I=\left\{ j:%
%TCIMACRO{\TeXButton{underaccent_x}{\underaccent{\wtilde}{x}}}%
%BeginExpansion
\underaccent{\wtilde}{x}%
%EndExpansion
_{i}=%
%TCIMACRO{\TeXButton{underaccent_x}{\underaccent{\wtilde}{x}}}%
%BeginExpansion
\underaccent{\wtilde}{x}%
%EndExpansion
_{j}\right\} $ (size of I=a)

then%
\begin{equation*}
P_{ii}=\tsum\limits_{j=1}^{n}P_{ij}^{2}=aP_{ii}^{2}+\tsum\limits_{j\neq
I}P_{ij}^{2}\geq aP_{ii}^{2}\Rightarrow P_{ii}\leq \frac{1}{a}
\end{equation*}%
if data has duplicate measure

\bigskip

\begin{itemize}
\item Let $\left( 
%TCIMACRO{\TeXButton{underaccent_Y}{\underaccent{\wtilde}{Y}}}%
%BeginExpansion
\underaccent{\wtilde}{Y}%
%EndExpansion
_{\left( i\right) },x_{\left( i\right) }\right) $ denote the vector $\left(
Y_{i},%
%TCIMACRO{\TeXButton{underaccent_x}{\underaccent{\wtilde}{x}}}%
%BeginExpansion
\underaccent{\wtilde}{x}%
%EndExpansion
_{i}^{\dagger }\right) $ is omitted. Write $\left( \chi ^{\dagger }\right)
_{k\times n}=\left( 
%TCIMACRO{\TeXButton{underaccent_chi}{\underaccent{\wtilde}{\chi}}}%
%BeginExpansion
\underaccent{\wtilde}{\chi}%
%EndExpansion
_{1},\cdots ,%
%TCIMACRO{\TeXButton{underaccent_chi}{\underaccent{\wtilde}{\chi}}}%
%BeginExpansion
\underaccent{\wtilde}{\chi}%
%EndExpansion
_{n}\right) $, then%
\begin{equation*}
x^{\dagger }x=\tsum\limits_{i=1}^{n}%
%TCIMACRO{\TeXButton{underaccent_chi}{\underaccent{\wtilde}{\chi}}}%
%BeginExpansion
\underaccent{\wtilde}{\chi}%
%EndExpansion
_{i}%
%TCIMACRO{\TeXButton{underaccent_chi}{\underaccent{\wtilde}{\chi}}}%
%BeginExpansion
\underaccent{\wtilde}{\chi}%
%EndExpansion
_{i}^{\dagger }=x_{\left( i\right) }^{\dagger }x_{\left( i\right) }+%
%TCIMACRO{\TeXButton{underaccent_chi}{\underaccent{\wtilde}{\chi}}}%
%BeginExpansion
\underaccent{\wtilde}{\chi}%
%EndExpansion
_{i}%
%TCIMACRO{\TeXButton{underaccent_chi}{\underaccent{\wtilde}{\chi}}}%
%BeginExpansion
\underaccent{\wtilde}{\chi}%
%EndExpansion
_{i}^{\dagger }
\end{equation*}%
\newline
Assume rank($x_{\left( i\right) }$)=k\newline
if $%
%TCIMACRO{\TeXButton{underaccent_chi}{\underaccent{\wtilde}{\chi}}}%
%BeginExpansion
\underaccent{\wtilde}{\chi}%
%EndExpansion
_{i}^{\dagger }\left( x_{\left( i\right) }^{\dagger }x_{\left( i\right)
}\right) ^{-1}%
%TCIMACRO{\TeXButton{underaccent_chi}{\underaccent{\wtilde}{\chi}}}%
%BeginExpansion
\underaccent{\wtilde}{\chi}%
%EndExpansion
_{i}\neq 1$, then by A.18(?)%
\begin{equation*}
\left( x_{\left( i\right) }^{\dagger }x_{\left( i\right) }\right)
^{-1}=\left( x^{\dagger }x\right) ^{-1}+\frac{\left( x^{\dagger }x\right)
^{-1}%
%TCIMACRO{\TeXButton{underaccent_chi}{\underaccent{\wtilde}{\chi}}}%
%BeginExpansion
\underaccent{\wtilde}{\chi}%
%EndExpansion
_{i}%
%TCIMACRO{\TeXButton{underaccent_chi}{\underaccent{\wtilde}{\chi}}}%
%BeginExpansion
\underaccent{\wtilde}{\chi}%
%EndExpansion
_{i}^{\dagger }\left( x^{\dagger }x\right) ^{-1}}{1-x_{\left( i\right)
}^{\dagger }\left( x^{\dagger }x\right) ^{-1}x_{\left( i\right) }}
\end{equation*}%
\begin{equation*}
\left[ 
\begin{array}{c}
\text{if }x_{\left( i\right) }^{\dagger }\left( x^{\dagger }x\right)
^{-1}x_{\left( i\right) }\neq 1 \\ 
x^{\dagger }x=x_{\left( i\right) }^{\dagger }x_{\left( i\right) }+%
%TCIMACRO{\TeXButton{underaccent_chi}{\underaccent{\wtilde}{\chi}}}%
%BeginExpansion
\underaccent{\wtilde}{\chi}%
%EndExpansion
_{i}%
%TCIMACRO{\TeXButton{underaccent_chi}{\underaccent{\wtilde}{\chi}}}%
%BeginExpansion
\underaccent{\wtilde}{\chi}%
%EndExpansion
_{i}^{\dagger } \\ 
\Rightarrow \text{above equation}%
\end{array}%
\right]
\end{equation*}%
\begin{equation*}
%TCIMACRO{\TeXButton{underaccent_chi}{\underaccent{\wtilde}{\chi}}}%
%BeginExpansion
\underaccent{\wtilde}{\chi}%
%EndExpansion
_{r}^{\dagger }\left( x_{\left( i\right) }^{\dagger }x_{\left( i\right)
}\right) ^{-1}%
%TCIMACRO{\TeXButton{underaccent_chi}{\underaccent{\wtilde}{\chi}}}%
%BeginExpansion
\underaccent{\wtilde}{\chi}%
%EndExpansion
_{r}=%
%TCIMACRO{\TeXButton{underaccent_chi}{\underaccent{\wtilde}{\chi}}}%
%BeginExpansion
\underaccent{\wtilde}{\chi}%
%EndExpansion
_{r}^{\dagger }\left[ \left( x^{\dagger }x\right) ^{-1}+\frac{\left(
x^{\dagger }x\right) ^{-1}%
%TCIMACRO{\TeXButton{underaccent_chi}{\underaccent{\wtilde}{\chi}}}%
%BeginExpansion
\underaccent{\wtilde}{\chi}%
%EndExpansion
_{i}%
%TCIMACRO{\TeXButton{underaccent_chi}{\underaccent{\wtilde}{\chi}}}%
%BeginExpansion
\underaccent{\wtilde}{\chi}%
%EndExpansion
_{i}^{\dagger }\left( x^{\dagger }x\right) ^{-1}}{1-P_{ii}}\right] 
%TCIMACRO{\TeXButton{underaccent_chi}{\underaccent{\wtilde}{\chi}}}%
%BeginExpansion
\underaccent{\wtilde}{\chi}%
%EndExpansion
_{r}
\end{equation*}%
\begin{equation*}
P_{rr\left( i\right) }=P_{rr}+\frac{P_{ri}^{2}}{1-P_{ii}}
\end{equation*}%
take away ith the leverage of rr position (rr can be different)\newline
$\Longrightarrow P_{rr\left( i\right) }$ may be large if either $P_{rr}$
large or $P_{ii}$ large and/or $P_{ri}$ large\newline
For example: $P_{22}=0.4$, $P_{22\left( 1\right) }=0.8$, $\Rightarrow $1 and
2 can be in same class $\rightarrow $ masking effect
\end{itemize}

\bigskip

\bigskip

\paragraph{7.3 Measure based on residuals $e_{i}$ ($\frac{e_{r}}{\protect%
\sigma _{i}}$) different kind of $e_{i}$}

\begin{enumerate}
\item normalized residuals: $\frac{e_{i}}{\sqrt{%
%TCIMACRO{\TeXButton{underaccent_e}{\underaccent{\wtilde}{e}}}%
%BeginExpansion
\underaccent{\wtilde}{e}%
%EndExpansion
^{\dagger }%
%TCIMACRO{\TeXButton{underaccent_e}{\underaccent{\wtilde}{e}}}%
%BeginExpansion
\underaccent{\wtilde}{e}%
%EndExpansion
}}=a_{i}$

\item standardized residuals: $\frac{e_{i}}{\sqrt{\frac{%
%TCIMACRO{\TeXButton{underaccent_e}{\underaccent{\wtilde}{e}}}%
%BeginExpansion
\underaccent{\wtilde}{e}%
%EndExpansion
^{\dagger }%
%TCIMACRO{\TeXButton{underaccent_e}{\underaccent{\wtilde}{e}}}%
%BeginExpansion
\underaccent{\wtilde}{e}%
%EndExpansion
}{n-k}}}=b_{i}$ (MSE)\newline
$b_{i}=\sqrt{n-k}a_{i}$, $r_{i}=\frac{b_{i}}{\sqrt{1-P_{ii}}}=\sqrt{\frac{n-k%
}{1-P_{ii}}}a_{i}$%
\begin{equation*}
%TCIMACRO{\TeXButton{underaccent_Y}{\underaccent{\wtilde}{Y}}}%
%BeginExpansion
\underaccent{\wtilde}{Y}%
%EndExpansion
-\hat{Y}=%
%TCIMACRO{\TeXButton{underaccent_e}{\underaccent{\wtilde}{e}}}%
%BeginExpansion
\underaccent{\wtilde}{e}%
%EndExpansion
=\left( I-P\right) 
%TCIMACRO{\TeXButton{underaccent_Y}{\underaccent{\wtilde}{Y}}}%
%BeginExpansion
\underaccent{\wtilde}{Y}%
%EndExpansion
\end{equation*}%
\begin{equation*}
Var\left( 
%TCIMACRO{\TeXButton{underaccent_e}{\underaccent{\wtilde}{e}}}%
%BeginExpansion
\underaccent{\wtilde}{e}%
%EndExpansion
\right) =\left( I-P\right) \sigma ^{2}=\left[ 
\begin{array}{ccc}
1-P_{11} &  & -P_{ij} \\ 
& \ddots &  \\ 
&  & 1-P_{nn}%
\end{array}%
\right] \sigma ^{2}
\end{equation*}%
If $P_{ij}\approx 0$ and $P_{ii}\approx P_{jj}$, $i\neq j$, then reflects $%
%TCIMACRO{\TeXButton{underaccent_epsilon}{\underaccent{\wtilde}{\epsilon}}}%
%BeginExpansion
\underaccent{\wtilde}{\epsilon}%
%EndExpansion
\sim \left( o,\sigma ^{2}I\right) $ similar, but in general not this case

\item internally studentized: $r_{i}=\frac{e_{i}}{\sqrt{\left(
1-P_{ii}\right) \text{\c{S}}^{2}}}\quad $\c{S}$^{2}=\frac{%
%TCIMACRO{\TeXButton{underaccent_e}{\underaccent{\wtilde}{e}}}%
%BeginExpansion
\underaccent{\wtilde}{e}%
%EndExpansion
^{\dagger }%
%TCIMACRO{\TeXButton{underaccent_e}{\underaccent{\wtilde}{e}}}%
%BeginExpansion
\underaccent{\wtilde}{e}%
%EndExpansion
}{n-k}$

\item externally studentized: $\frac{e_{i}}{\sqrt{\left( 1-P_{ii}\right) 
\text{\c{S}}_{\left( i\right) }^{2}}}$\newline
where \c{S}$_{\left( i\right) }^{2}=\frac{%
%TCIMACRO{\TeXButton{underaccent_Y}{\underaccent{\wtilde}{Y}}}%
%BeginExpansion
\underaccent{\wtilde}{Y}%
%EndExpansion
_{\left( i\right) }^{\dagger }\left[ I-P_{\left( i\right) }\right] 
%TCIMACRO{\TeXButton{underaccent_Y}{\underaccent{\wtilde}{Y}}}%
%BeginExpansion
\underaccent{\wtilde}{Y}%
%EndExpansion
_{\left( i\right) }}{n-k-1}$%
\begin{equation*}
\begin{array}{ccc}
e_{i\left( i\right) } & e_{i} & \text{cook's distance} \\ 
\hat{Y}_{i\left( i\right) }=%
%TCIMACRO{\TeXButton{underaccent_x}{\underaccent{\wtilde}{x}}}%
%BeginExpansion
\underaccent{\wtilde}{x}%
%EndExpansion
_{i}^{\dagger }%
%TCIMACRO{%
%\TeXButton{underaccent_beta_hat}{\underaccent{\wtilde}{\hat{\beta}}}}%
%BeginExpansion
\underaccent{\wtilde}{\hat{\beta}}%
%EndExpansion
_{\left( i\right) } & \hat{\beta} & D_{i}=\left( 
%TCIMACRO{%
%\TeXButton{underaccent_beta_hat}{\underaccent{\wtilde}{\hat{\beta}}}}%
%BeginExpansion
\underaccent{\wtilde}{\hat{\beta}}%
%EndExpansion
_{\left( i\right) }-\hat{\beta}\right) ^{\dagger }x^{\dagger }x\left( 
%TCIMACRO{%
%\TeXButton{underaccent_beta_hat}{\underaccent{\wtilde}{\hat{\beta}}}}%
%BeginExpansion
\underaccent{\wtilde}{\hat{\beta}}%
%EndExpansion
_{\left( i\right) }-\hat{\beta}\right) \\ 
%TCIMACRO{%
%\TeXButton{underaccent_beta_hat}{\underaccent{\wtilde}{\hat{\beta}}}}%
%BeginExpansion
\underaccent{\wtilde}{\hat{\beta}}%
%EndExpansion
_{\left( i\right) }=x_{\left( i\right) }\left[ x_{\left( i\right) }^{\dagger
}x_{\left( i\right) }\right] ^{-1}x_{\left( i\right) }^{\dagger }Y_{\left(
i\right) } &  & =\left( x%
%TCIMACRO{%
%\TeXButton{underaccent_beta_hat}{\underaccent{\wtilde}{\hat{\beta}}}}%
%BeginExpansion
\underaccent{\wtilde}{\hat{\beta}}%
%EndExpansion
_{\left( i\right) }-x\hat{\beta}\right) ^{\dagger }\left( x%
%TCIMACRO{%
%\TeXButton{underaccent_beta_hat}{\underaccent{\wtilde}{\hat{\beta}}}}%
%BeginExpansion
\underaccent{\wtilde}{\hat{\beta}}%
%EndExpansion
_{\left( i\right) }-x\hat{\beta}\right)%
\end{array}%
\end{equation*}%
\begin{eqnarray*}
\hat{\varepsilon}_{i\left( i\right) } &=&\frac{\hat{\varepsilon}_{i}}{%
1-P_{ii}} \\
b-\hat{\beta}_{\left( i\right) } &=&\left( x^{\dagger }x\right) ^{-1}\frac{%
x_{i}\hat{\varepsilon}_{i}}{1-P_{ii}}
\end{eqnarray*}
\end{enumerate}

\bigskip

\bigskip

\paragraph{7.3.3 Outlier for influential observation}

\begin{equation*}
\text{Leverage }%
%TCIMACRO{\TeXButton{underaccent_x}{\underaccent{\wtilde}{x}}}%
%BeginExpansion
\underaccent{\wtilde}{x}%
%EndExpansion
_{i}^{\dagger }\left[ x^{\dagger }x\right] ^{-1}%
%TCIMACRO{\TeXButton{underaccent_x}{\underaccent{\wtilde}{x}}}%
%BeginExpansion
\underaccent{\wtilde}{x}%
%EndExpansion
_{\left( i\right) }=P_{ii}\text{, }0\leq P_{ii}\leq 1
\end{equation*}%
P$_{ii}$ has outlier.

If a subject have the same $%
%TCIMACRO{\TeXButton{underaccent_x}{\underaccent{\wtilde}{x}}}%
%BeginExpansion
\underaccent{\wtilde}{x}%
%EndExpansion
$, then $P_{ii}\leq \frac{1}{a}$, $0\leq P_{ii}\leq \frac{1}{a}$

\bigskip

$P_{rr\left( i\right) }:$ the rth subject of the leverage after taking away
the ith subject.

\begin{equation*}
P_{rr\left( i\right) }=P_{rr}+\frac{P_{ri}^{2}}{1-P_{ii}}\quad r\neq i
\end{equation*}

\begin{equation*}
Cov\left( 
%TCIMACRO{\TeXButton{underaccent_e}{\underaccent{\wtilde}{e}}}%
%BeginExpansion
\underaccent{\wtilde}{e}%
%EndExpansion
\right) =\left( I-P\right) \sigma ^{2}=\left[ 
\begin{array}{ccc}
1-P_{11} &  & -P_{ij} \\ 
& \ddots &  \\ 
&  & 1-P_{nn}%
\end{array}%
\right] \sigma ^{2}
\end{equation*}

$x\overset{\text{dianose}}{\longleftarrow }P_{ii}$, $Y\overset{\text{dianose}%
}{\longleftarrow }e_{i}$

\begin{eqnarray*}
e_{i\left( i\right) } &=&Y_{i}-%
%TCIMACRO{\TeXButton{underaccent_x}{\underaccent{\wtilde}{x}}}%
%BeginExpansion
\underaccent{\wtilde}{x}%
%EndExpansion
_{i}^{\dagger }\hat{\beta}_{\left( i\right) } \\
&=&y_{i}-%
%TCIMACRO{\TeXButton{underaccent_x}{\underaccent{\wtilde}{x}}}%
%BeginExpansion
\underaccent{\wtilde}{x}%
%EndExpansion
_{i}^{\dagger }\left[ x_{\left( i\right) }^{\dagger }x_{\left( i\right) }%
\right] ^{-1}x_{\left( i\right) }^{\dagger }%
%TCIMACRO{\TeXButton{underaccent_Y}{\underaccent{\wtilde}{Y}}}%
%BeginExpansion
\underaccent{\wtilde}{Y}%
%EndExpansion
_{\left( i\right) } \\
&=&y_{i}-%
%TCIMACRO{\TeXButton{underaccent_x}{\underaccent{\wtilde}{x}}}%
%BeginExpansion
\underaccent{\wtilde}{x}%
%EndExpansion
_{i}^{\dagger }\left( x^{\dagger }x-%
%TCIMACRO{\TeXButton{underaccent_x}{\underaccent{\wtilde}{x}}}%
%BeginExpansion
\underaccent{\wtilde}{x}%
%EndExpansion
_{i}%
%TCIMACRO{\TeXButton{underaccent_x}{\underaccent{\wtilde}{x}}}%
%BeginExpansion
\underaccent{\wtilde}{x}%
%EndExpansion
_{i}^{\dagger }\right) ^{-1}\left( x^{\dagger }Y-%
%TCIMACRO{\TeXButton{underaccent_x}{\underaccent{\wtilde}{x}}}%
%BeginExpansion
\underaccent{\wtilde}{x}%
%EndExpansion
_{i}y_{i}\right) \\
&=&y_{i}-%
%TCIMACRO{\TeXButton{underaccent_x}{\underaccent{\wtilde}{x}}}%
%BeginExpansion
\underaccent{\wtilde}{x}%
%EndExpansion
_{i}^{\dagger }\left[ \left( x^{\dagger }x\right) ^{-1}+\frac{\left(
x^{\dagger }x\right) ^{-1}x_{i}x_{i}^{\dagger }\left( x^{\dagger }x\right)
^{-1}}{1-P_{ii}}\right] \left( x^{\dagger }%
%TCIMACRO{\TeXButton{underaccent_Y}{\underaccent{\wtilde}{Y}}}%
%BeginExpansion
\underaccent{\wtilde}{Y}%
%EndExpansion
-%
%TCIMACRO{\TeXButton{underaccent_x}{\underaccent{\wtilde}{x}}}%
%BeginExpansion
\underaccent{\wtilde}{x}%
%EndExpansion
_{i}y_{i}\right) \\
&=&y_{i}-%
%TCIMACRO{\TeXButton{underaccent_x}{\underaccent{\wtilde}{x}}}%
%BeginExpansion
\underaccent{\wtilde}{x}%
%EndExpansion
_{i}^{\dagger }\left( x^{\dagger }x\right) ^{-1}x^{\dagger }Y+%
%TCIMACRO{\TeXButton{underaccent_x}{\underaccent{\wtilde}{x}}}%
%BeginExpansion
\underaccent{\wtilde}{x}%
%EndExpansion
_{i}^{\dagger }\left( x^{\dagger }x\right) ^{-1}%
%TCIMACRO{\TeXButton{underaccent_x}{\underaccent{\wtilde}{x}}}%
%BeginExpansion
\underaccent{\wtilde}{x}%
%EndExpansion
_{i}y_{i} \\
&&-\frac{%
%TCIMACRO{\TeXButton{underaccent_x}{\underaccent{\wtilde}{x}}}%
%BeginExpansion
\underaccent{\wtilde}{x}%
%EndExpansion
_{i}^{\dagger }\left( x^{\dagger }x\right) ^{-1}%
%TCIMACRO{\TeXButton{underaccent_x}{\underaccent{\wtilde}{x}}}%
%BeginExpansion
\underaccent{\wtilde}{x}%
%EndExpansion
_{i}%
%TCIMACRO{\TeXButton{underaccent_x}{\underaccent{\wtilde}{x}}}%
%BeginExpansion
\underaccent{\wtilde}{x}%
%EndExpansion
_{i}^{\dagger }\left( x^{\dagger }x\right) ^{-1}x^{\dagger }Y}{1-P_{ii}} \\
&&+\frac{%
%TCIMACRO{\TeXButton{underaccent_x}{\underaccent{\wtilde}{x}}}%
%BeginExpansion
\underaccent{\wtilde}{x}%
%EndExpansion
_{i}^{\dagger }\left( x^{\dagger }x\right) ^{-1}x_{i}x_{i}^{\dagger }\left(
x^{\dagger }x\right) ^{-1}%
%TCIMACRO{\TeXButton{underaccent_x}{\underaccent{\wtilde}{x}}}%
%BeginExpansion
\underaccent{\wtilde}{x}%
%EndExpansion
_{i}y_{i}}{1-P_{ii}} \\
&=&y_{i}-\underset{\hat{Y}_{i}}{\underbrace{%
%TCIMACRO{\TeXButton{underaccent_x}{\underaccent{\wtilde}{x}}}%
%BeginExpansion
\underaccent{\wtilde}{x}%
%EndExpansion
_{i}^{\dagger }\hat{\beta}}}+P_{ii}y_{i}-\frac{P_{ii}}{1-P_{ii}}%
%TCIMACRO{\TeXButton{underaccent_x}{\underaccent{\wtilde}{x}}}%
%BeginExpansion
\underaccent{\wtilde}{x}%
%EndExpansion
_{i}^{\dagger }\hat{\beta}+\frac{P_{ii}y_{i}}{1-P_{ii}} \\
&=&y_{i}+P_{ii}y_{i}+\frac{P_{ii}^{2}y_{i}}{1-P_{ii}}-\hat{y}_{i}-\frac{%
P_{ii}}{1-P_{ii}}\hat{y}_{i} \\
&=&\frac{\left( 1-P_{ii}\right) \left( 1+P_{ii}\right) Y_{i}+P_{ii}^{2}Y_{i}%
}{1-P_{ii}}-\frac{\hat{Y}_{i}-P_{ii}\hat{Y}_{i}+P_{ii}\hat{Y}_{i}}{1-P_{ii}}
\\
&=&\frac{Y_{i}-P_{ii}^{2}Y_{i}+P_{ii}^{2}Y_{i}}{1-P_{ii}}-\frac{\hat{Y}_{i}}{%
1-P_{ii}} \\
&=&\frac{1}{1-P_{ii}}\left( y_{i}-\hat{y}_{i}\right) \\
&=&\frac{e_{i}}{1-P_{ii}}
\end{eqnarray*}

\bigskip

\bigskip

\paragraph{7.5 Cook's Distance}

\begin{equation*}
C_{i}=\frac{\left( 
%TCIMACRO{%
%\TeXButton{underaccent_beta_hat}{\underaccent{\wtilde}{\hat{\beta}}}}%
%BeginExpansion
\underaccent{\wtilde}{\hat{\beta}}%
%EndExpansion
-%
%TCIMACRO{%
%\TeXButton{underaccent_beta_hat}{\underaccent{\wtilde}{\hat{\beta}}}}%
%BeginExpansion
\underaccent{\wtilde}{\hat{\beta}}%
%EndExpansion
_{\left( i\right) }\right) ^{\dagger }x^{\dagger }x\left( 
%TCIMACRO{%
%\TeXButton{underaccent_beta_hat}{\underaccent{\wtilde}{\hat{\beta}}}}%
%BeginExpansion
\underaccent{\wtilde}{\hat{\beta}}%
%EndExpansion
-%
%TCIMACRO{%
%\TeXButton{underaccent_beta_hat}{\underaccent{\wtilde}{\hat{\beta}}}}%
%BeginExpansion
\underaccent{\wtilde}{\hat{\beta}}%
%EndExpansion
_{\left( i\right) }\right) }{k\text{\c{S}}^{2}}
\end{equation*}

\bigskip

Note: $\frac{\left( \hat{\beta}-\beta \right) ^{\dagger }\left( \left(
x^{\dagger }x\right) ^{-1}\hat{\sigma}^{2}\right) ^{-1}\left( \hat{\beta}%
-\beta \right) }{k}\sim F_{k,n-k}=\frac{\left( \hat{\beta}-\beta \right)
^{\dagger }x^{\dagger }x\left( \hat{\beta}-\beta \right) }{k\text{\c{S}}^{2}}
$

\bigskip

estimate whether some value have markable effect on estimation of regression
coefficient, normally look for $F_{k,n-k}(0.2)$, $F_{k,n-k}\left( 0.5\right) 
$

\bigskip

\begin{eqnarray*}
\left( \hat{\beta}-\hat{\beta}_{\left( i\right) }\right) &=&\left(
x^{\dagger }x\right) ^{-1}x^{\dagger }Y-\left( x_{\left( i\right) }^{\dagger
}x_{\left( i\right) }\right) ^{-1}x_{\left( i\right) }Y_{\left( i\right) } \\
&=&\left( x^{\dagger }x\right) ^{-1}x^{\dagger }Y-\left[ \left( x^{\dagger
}x\right) ^{-1}+\frac{\left( x^{\dagger }x\right) ^{-1}%
%TCIMACRO{\TeXButton{underaccent_x}{\underaccent{\wtilde}{x}}}%
%BeginExpansion
\underaccent{\wtilde}{x}%
%EndExpansion
_{i}%
%TCIMACRO{\TeXButton{underaccent_x}{\underaccent{\wtilde}{x}}}%
%BeginExpansion
\underaccent{\wtilde}{x}%
%EndExpansion
_{i}^{\dagger }\left( x^{\dagger }x\right) ^{-1}}{1-P_{ii}}\right] \left(
x^{\dagger }%
%TCIMACRO{\TeXButton{underaccent_Y}{\underaccent{\wtilde}{Y}}}%
%BeginExpansion
\underaccent{\wtilde}{Y}%
%EndExpansion
-%
%TCIMACRO{\TeXButton{underaccent_x}{\underaccent{\wtilde}{x}}}%
%BeginExpansion
\underaccent{\wtilde}{x}%
%EndExpansion
_{i}y_{i}\right) \\
&=&\left( x^{\dagger }x\right) ^{-1}%
%TCIMACRO{\TeXButton{underaccent_x}{\underaccent{\wtilde}{x}}}%
%BeginExpansion
\underaccent{\wtilde}{x}%
%EndExpansion
_{i}y_{i}-\frac{\left( x^{\dagger }x\right) ^{-1}%
%TCIMACRO{\TeXButton{underaccent_x}{\underaccent{\wtilde}{x}}}%
%BeginExpansion
\underaccent{\wtilde}{x}%
%EndExpansion
_{i}%
%TCIMACRO{\TeXButton{underaccent_x}{\underaccent{\wtilde}{x}}}%
%BeginExpansion
\underaccent{\wtilde}{x}%
%EndExpansion
_{i}^{\dagger }\left( x^{\dagger }x\right) ^{-1}}{1-P_{ii}}\left( x^{\dagger
}Y-%
%TCIMACRO{\TeXButton{underaccent_x}{\underaccent{\wtilde}{x}}}%
%BeginExpansion
\underaccent{\wtilde}{x}%
%EndExpansion
_{i}y_{i}\right) \\
&=&\left( x^{\dagger }x\right) ^{-1}%
%TCIMACRO{\TeXButton{underaccent_x}{\underaccent{\wtilde}{x}}}%
%BeginExpansion
\underaccent{\wtilde}{x}%
%EndExpansion
_{i}\left[ y_{i}-\frac{%
%TCIMACRO{\TeXButton{underaccent_x}{\underaccent{\wtilde}{x}}}%
%BeginExpansion
\underaccent{\wtilde}{x}%
%EndExpansion
_{i}^{\dagger }\left( x^{\dagger }x\right) ^{-1}}{1-P_{ii}}\left( x^{\dagger
}Y-%
%TCIMACRO{\TeXButton{underaccent_x}{\underaccent{\wtilde}{x}}}%
%BeginExpansion
\underaccent{\wtilde}{x}%
%EndExpansion
_{i}y_{i}\right) \right] \\
&=&\left( x^{\dagger }x\right) ^{-1}%
%TCIMACRO{\TeXButton{underaccent_x}{\underaccent{\wtilde}{x}}}%
%BeginExpansion
\underaccent{\wtilde}{x}%
%EndExpansion
_{i}\left[ \frac{y_{i}-P_{ii}y_{i}-%
%TCIMACRO{\TeXButton{underaccent_x}{\underaccent{\wtilde}{x}}}%
%BeginExpansion
\underaccent{\wtilde}{x}%
%EndExpansion
_{i}^{\dagger }\left( x^{\dagger }x\right) ^{-1}x^{\dagger }Y+%
%TCIMACRO{\TeXButton{underaccent_x}{\underaccent{\wtilde}{x}}}%
%BeginExpansion
\underaccent{\wtilde}{x}%
%EndExpansion
_{i}^{\dagger }\left( x^{\dagger }x\right) ^{-1}%
%TCIMACRO{\TeXButton{underaccent_x}{\underaccent{\wtilde}{x}}}%
%BeginExpansion
\underaccent{\wtilde}{x}%
%EndExpansion
_{i}y_{i}}{1-P_{ii}}\right] \\
&=&\left( x^{\dagger }x\right) ^{-1}%
%TCIMACRO{\TeXButton{underaccent_x}{\underaccent{\wtilde}{x}}}%
%BeginExpansion
\underaccent{\wtilde}{x}%
%EndExpansion
_{i}\frac{y_{i}-%
%TCIMACRO{\TeXButton{underaccent_x}{\underaccent{\wtilde}{x}}}%
%BeginExpansion
\underaccent{\wtilde}{x}%
%EndExpansion
_{i}^{\dagger }%
%TCIMACRO{%
%\TeXButton{underaccent_beta_hat}{\underaccent{\wtilde}{\hat{\beta}}}}%
%BeginExpansion
\underaccent{\wtilde}{\hat{\beta}}%
%EndExpansion
}{1-P_{ii}} \\
&=&\left( x^{\dagger }x\right) ^{-1}%
%TCIMACRO{\TeXButton{underaccent_x}{\underaccent{\wtilde}{x}}}%
%BeginExpansion
\underaccent{\wtilde}{x}%
%EndExpansion
_{i}\frac{e_{i}}{1-P_{ii}}
\end{eqnarray*}

\bigskip

Rewrite%
\begin{eqnarray*}
C_{i} &=&\frac{e_{i}^{2}%
%TCIMACRO{\TeXButton{underaccent_x}{\underaccent{\wtilde}{x}}}%
%BeginExpansion
\underaccent{\wtilde}{x}%
%EndExpansion
_{i}^{\dagger }\left( x^{\dagger }x\right) ^{-1}x^{\dagger }x\left(
x^{\dagger }x\right) ^{-1}%
%TCIMACRO{\TeXButton{underaccent_x}{\underaccent{\wtilde}{x}}}%
%BeginExpansion
\underaccent{\wtilde}{x}%
%EndExpansion
_{i}}{\left( 1-P_{ii}\right) ^{2}k\text{\c{S}}^{2}} \\
&=&\frac{e_{i}^{2}}{\left( 1-P_{ii}\right) ^{2}}\times \frac{P_{ii}}{k\text{%
\c{S}}^{2}} \\
&=&\frac{P_{ii}}{\left( 1-P_{ii}\right) ^{2}}\times \frac{e_{i}^{2}}{k\text{%
\c{S}}^{2}}
\end{eqnarray*}

\bigskip

use $C_{i}$ to estimate outlier. $\frac{P_{ii}}{\left( 1-P_{ii}\right) ^{2}}$
has larger control power among the terms.

\bigskip

\bigskip

\paragraph{Mean Shift Model}

\begin{equation*}
%TCIMACRO{\TeXButton{underaccent_Y}{\underaccent{\wtilde}{Y}}}%
%BeginExpansion
\underaccent{\wtilde}{Y}%
%EndExpansion
=x%
%TCIMACRO{\TeXButton{underaccent_beta}{\underaccent{\wtilde}{\beta}}}%
%BeginExpansion
\underaccent{\wtilde}{\beta}%
%EndExpansion
+\delta 
%TCIMACRO{\TeXButton{underaccent_e}{\underaccent{\wtilde}{e}}}%
%BeginExpansion
\underaccent{\wtilde}{e}%
%EndExpansion
_{i}+%
%TCIMACRO{\TeXButton{underaccent_epsilon}{\underaccent{\wtilde}{\epsilon}}}%
%BeginExpansion
\underaccent{\wtilde}{\epsilon}%
%EndExpansion
\end{equation*}%
where $%
%TCIMACRO{\TeXButton{underaccent_e}{\underaccent{\wtilde}{e}}}%
%BeginExpansion
\underaccent{\wtilde}{e}%
%EndExpansion
_{i}=\left[ 0,\cdots ,1,\cdots ,0\right] ^{\dagger }$

\begin{equation*}
\begin{array}{c}
H_{0}:\delta =0\text{, (i.e. }E\left( 
%TCIMACRO{\TeXButton{underaccent_Y}{\underaccent{\wtilde}{Y}}}%
%BeginExpansion
\underaccent{\wtilde}{Y}%
%EndExpansion
\right) =x%
%TCIMACRO{\TeXButton{underaccent_beta}{\underaccent{\wtilde}{\beta}}}%
%BeginExpansion
\underaccent{\wtilde}{\beta}%
%EndExpansion
\text{)} \\ 
H_{1}:\delta \neq 0\text{, (i.e. }E\left( 
%TCIMACRO{\TeXButton{underaccent_Y}{\underaccent{\wtilde}{Y}}}%
%BeginExpansion
\underaccent{\wtilde}{Y}%
%EndExpansion
\right) =x%
%TCIMACRO{\TeXButton{underaccent_beta}{\underaccent{\wtilde}{\beta}}}%
%BeginExpansion
\underaccent{\wtilde}{\beta}%
%EndExpansion
+\delta 
%TCIMACRO{\TeXButton{underaccent_e}{\underaccent{\wtilde}{e}}}%
%BeginExpansion
\underaccent{\wtilde}{e}%
%EndExpansion
_{i}\text{)}%
\end{array}%
\end{equation*}

\bigskip

Use likelihood ratio test statistic $F=\frac{\text{\c{S}\c{S}}E\left(
H_{0}\right) -\text{\c{S}\c{S}}E\left( H_{1}\right) }{\frac{\text{\c{S}\c{S}}%
E\left( H_{1}\right) }{n-k-1}}$

\begin{eqnarray*}
\text{\c{S}\c{S}}E\left( H_{0}\right) &=&%
%TCIMACRO{\TeXButton{underaccent_Y}{\underaccent{\wtilde}{Y}}}%
%BeginExpansion
\underaccent{\wtilde}{Y}%
%EndExpansion
^{\dagger }\left( I-P\right) 
%TCIMACRO{\TeXButton{underaccent_Y}{\underaccent{\wtilde}{Y}}}%
%BeginExpansion
\underaccent{\wtilde}{Y}%
%EndExpansion
=\left( n-k\right) \text{\c{S}}^{2} \\
\text{\c{S}\c{S}}E\left( H_{1}\right) &=&%
%TCIMACRO{\TeXButton{underaccent_Y}{\underaccent{\wtilde}{Y}}}%
%BeginExpansion
\underaccent{\wtilde}{Y}%
%EndExpansion
^{\dagger }\left( I-P-\frac{\left( I-P\right) 
%TCIMACRO{\TeXButton{underaccent_e}{\underaccent{\wtilde}{e}}}%
%BeginExpansion
\underaccent{\wtilde}{e}%
%EndExpansion
_{i}%
%TCIMACRO{\TeXButton{underaccent_e}{\underaccent{\wtilde}{e}}}%
%BeginExpansion
\underaccent{\wtilde}{e}%
%EndExpansion
_{i}^{\dagger }\left( I-P\right) }{1-P_{ii}}\right) 
%TCIMACRO{\TeXButton{underaccent_Y}{\underaccent{\wtilde}{Y}}}%
%BeginExpansion
\underaccent{\wtilde}{Y}%
%EndExpansion
\end{eqnarray*}

\begin{eqnarray*}
\text{i.e. }P^{\prime } &=&\left[ x,%
%TCIMACRO{\TeXButton{underaccent_e}{\underaccent{\wtilde}{e}}}%
%BeginExpansion
\underaccent{\wtilde}{e}%
%EndExpansion
_{i}\right] \left[ \left( x,%
%TCIMACRO{\TeXButton{underaccent_e}{\underaccent{\wtilde}{e}}}%
%BeginExpansion
\underaccent{\wtilde}{e}%
%EndExpansion
_{i}\right) ^{\dagger }\left( x,%
%TCIMACRO{\TeXButton{underaccent_e}{\underaccent{\wtilde}{e}}}%
%BeginExpansion
\underaccent{\wtilde}{e}%
%EndExpansion
_{i}\right) \right] ^{-1}\left[ x,%
%TCIMACRO{\TeXButton{underaccent_e}{\underaccent{\wtilde}{e}}}%
%BeginExpansion
\underaccent{\wtilde}{e}%
%EndExpansion
_{i}\right] \\
&=&x\left( x^{\dagger }x\right) ^{-1}x+\frac{\left( I-P\right) 
%TCIMACRO{\TeXButton{underaccent_e}{\underaccent{\wtilde}{e}}}%
%BeginExpansion
\underaccent{\wtilde}{e}%
%EndExpansion
_{i}%
%TCIMACRO{\TeXButton{underaccent_e}{\underaccent{\wtilde}{e}}}%
%BeginExpansion
\underaccent{\wtilde}{e}%
%EndExpansion
_{i}^{\dagger }\left( I-P\right) }{e_{i}^{\dagger }\left( 1-P_{ii}\right) 
%TCIMACRO{\TeXButton{underaccent_e}{\underaccent{\wtilde}{e}}}%
%BeginExpansion
\underaccent{\wtilde}{e}%
%EndExpansion
_{i}}
\end{eqnarray*}

\begin{equation*}
I-P^{\prime }=I-P-\frac{\left( I-P\right) 
%TCIMACRO{\TeXButton{underaccent_e}{\underaccent{\wtilde}{e}}}%
%BeginExpansion
\underaccent{\wtilde}{e}%
%EndExpansion
_{i}%
%TCIMACRO{\TeXButton{underaccent_e}{\underaccent{\wtilde}{e}}}%
%BeginExpansion
\underaccent{\wtilde}{e}%
%EndExpansion
_{i}^{\dagger }\left( I-P\right) }{e_{i}^{\dagger }\left( 1-P_{ii}\right) 
%TCIMACRO{\TeXButton{underaccent_e}{\underaccent{\wtilde}{e}}}%
%BeginExpansion
\underaccent{\wtilde}{e}%
%EndExpansion
_{i}}
\end{equation*}

\bigskip

\begin{eqnarray*}
F &=&\frac{\text{\c{S}\c{S}}E\left( H_{0}\right) -\text{\c{S}\c{S}}E\left(
H_{1}\right) }{\frac{\text{\c{S}\c{S}}E\left( H_{1}\right) }{n-k-1}}=\frac{%
\frac{e_{i}^{2}}{1-P_{ii}}}{\text{\c{S}}_{\left( i\right) }^{2}}=\frac{%
e_{i}^{2}}{\left( 1-P_{ii}\right) \text{\c{S}}_{\left( i\right) }^{2}} \\
&=&\left( \frac{e_{i}}{\sqrt{\left( 1-P_{ii}\right) \text{\c{S}}_{\left(
i\right) }^{2}}}\right) ^{2}=\left( r_{i}^{\ast }\right) ^{2}\quad \text{%
external student residual}
\end{eqnarray*}

\bigskip

disadvantage:%
\begin{eqnarray*}
&&\left( r_{i}^{\ast }\right) ^{2}\overset{H_{0}}{\leadsto }F_{1,n-k-1} \\
&\leadsto &F_{1,n-k-1}\left( \delta ^{2}\left( 1-P_{ii}\right) \sigma
^{2}\right)
\end{eqnarray*}

\bigskip

\begin{equation*}
\frac{P_{ii}}{\left( 1-P_{ii}\right) ^{2}k\text{\c{S}}^{2}}\quad \text{(cook)%
}
\end{equation*}%
one more 1-P$_{ii}$, if far enough, it can be enhanced, 1-P$_{ii}$ is very
small.

\bigskip

Internal residual:%
\begin{eqnarray*}
r_{i} &=&\frac{e_{i}}{\text{\c{S}}_{\left( i\right) }\sqrt{\left(
1-P_{ii}\right) }} \\
r_{i}^{2} &=&\frac{1}{\left( 1-P_{ii}\right) }\times \left( \frac{e_{i}}{%
\text{\c{S}}_{\left( i\right) }}\right) ^{2}=\frac{n-k}{1-P_{ii}}\times 
\frac{e_{i}^{2}}{\tsum e_{i}^{2}}
\end{eqnarray*}

External residual:%
\begin{eqnarray*}
\left( r_{i}^{\ast }\right) ^{2} &=&\left( \frac{e_{i}}{\text{\c{S}}_{\left(
i\right) }\sqrt{\left( 1-P_{ii}\right) }}\right) ^{2} \\
&&\left( r_{i}^{\ast }\right) ^{2}\overset{H_{0}}{\leadsto }F_{1,n-k-1}
\end{eqnarray*}

\begin{eqnarray*}
r_{i}^{\ast } &=&r_{i}\sqrt{\frac{n-k-1}{n-k-r_{i}^{2}}}\text{, see range, }%
0\leq \left( r_{i}^{\ast }\right) ^{2}<\infty  \\
&=&\sqrt{\frac{r_{i}^{2}}{n-k-r_{i}^{2}}\left( n-k-1\right) }
\end{eqnarray*}%
\begin{equation*}
\Rightarrow \left( r_{i}^{\ast }\right) ^{2}=\frac{r_{i}^{2}}{n-k-r_{i}^{2}}%
\left( n-k-1\right) \overset{r_{i}^{2}\rightarrow n-k}{\longrightarrow }%
\infty 
\end{equation*}

If hope can find obvious span, although $\alpha $ is 1-1, but $\because
\left( r_{i}^{\ast }\right) ^{2}$ from $0\sim \infty $ has a obvious sense. $%
r_{i}^{\ast }$ is ??, can use percentile as a reference.

$0<\frac{r_{i}^{2}}{n-k}<1$ is $Beta\left( \frac{1}{2},\frac{n-k-1}{2}%
\right) $, it is harder to find percentile for comparison.

\begin{equation*}
F=\frac{\left( n-k\right) \text{\c{S}}^{2}-\left( n-k-1\right) \text{\c{S}}%
_{\left( i\right) }^{2}}{\text{\c{S}}_{\left( i\right) }^{2}}=\frac{e_{i}^{2}%
}{\text{\c{S}}_{\left( i\right) }^{2}\left( 1-P_{ii}\right) }
\end{equation*}%
\begin{eqnarray*}
&\Longrightarrow &\left( n-k\right) \frac{\text{\c{S}}^{2}}{\text{\c{S}}%
_{\left( i\right) }^{2}}-\left( n-k-1\right) =\frac{e_{i}^{2}}{\text{\c{S}}%
_{\left( i\right) }^{2}\left( 1-P_{ii}\right) } \\
&\Longrightarrow &\left( n-k\right) \text{\c{S}}^{2}-\left( n-k-1\right) 
\text{\c{S}}_{\left( i\right) }^{2}=\frac{e_{i}^{2}}{\left( 1-P_{ii}\right) }
\\
&\Longrightarrow &\text{\c{S}}_{\left( i\right) }^{2}=\frac{n-k}{n-k-1}\text{%
\c{S}}^{2}-\frac{e_{i}^{2}}{\left( n-k-1\right) \left( 1-P_{ii}\right) } \\
&=&\text{\c{S}}^{2}\left( \frac{n-k}{n-k-1}-\frac{r_{i}^{2}\left(
1-P_{ii}\right) }{\left( n-k-1\right) \left( 1-P_{ii}\right) }\right) \\
&=&\text{\c{S}}^{2}\left( \frac{n-k-r_{i}^{2}}{n-k-1}\right)
\end{eqnarray*}

\begin{equation*}
r_{i}=\frac{e_{i}}{\text{\c{S}}\sqrt{1-P_{ii}}}\quad r_{i}^{\ast }=\frac{%
e_{i}}{\text{\c{S}}_{\left( i\right) }\sqrt{\left( 1-P_{ii}\right) }}
\end{equation*}

\begin{equation*}
r_{i}^{\ast }=\frac{e_{i}}{\text{\c{S}}\sqrt{\frac{n-k-r_{i}^{2}}{n-k-1}}%
\sqrt{\left( 1-P_{ii}\right) }}=\frac{e_{i}}{\text{\c{S}}\sqrt{\left(
1-P_{ii}\right) }}\times \sqrt{\frac{n-k-1}{n-k-r_{i}^{2}}}=r_{i}\sqrt{\frac{%
n-k-1}{n-k-r_{i}^{2}}}
\end{equation*}%
\begin{equation*}
\text{\c{S}}_{\left( i\right) }=\frac{SSE_{\left( i\right) }}{n-k-1}
\end{equation*}

Note:%
\begin{equation*}
F=r_{\ast }^{2}=\frac{e_{i}^{2}}{\text{\c{S}}_{\left( i\right) }^{2}\left(
1-P_{ii}\right) }=\frac{\left( n-k\right) \text{\c{S}}^{2}-\left(
n-k-1\right) \text{\c{S}}_{\left( i\right) }^{2}}{\text{\c{S}}_{\left(
i\right) }^{2}}
\end{equation*}%
\begin{equation*}
\therefore \text{\c{S}}_{\left( i\right) }^{2}=\frac{\left( n-k\right) \text{%
\c{S}}^{2}-\frac{e_{i}^{2}}{\left( 1-P_{ii}\right) }}{n-k-1}
\end{equation*}

\end{document}
