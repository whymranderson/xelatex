
\documentclass{article}
%%%%%%%%%%%%%%%%%%%%%%%%%%%%%%%%%%%%%%%%%%%%%%%%%%%%%%%%%%%%%%%%%%%%%%%%%%%%%%%%%%%%%%%%%%%%%%%%%%%%%%%%%%%%%%%%%%%%%%%%%%%%%%%%%%%%%%%%%%%%%%%%%%%%%%%%%%%%%%%%%%%%%%%%%%%%%%%%%%%%%%%%%%%%%%%%%%%%%%%%%%%%%%%%%%%%%%%%%%%%%%%%%%%%%%%%%%%%%%%%%%%%%%%%%%%%
\usepackage{amssymb}
\usepackage{amsfonts}
\usepackage{amsmath}
\usepackage{accents}
\usepackage[ignoreall,a4paper]{geometry}
\usepackage{fancyhdr}

\setcounter{MaxMatrixCols}{10}
%TCIDATA{OutputFilter=LATEX.DLL}
%TCIDATA{Version=5.00.0.2606}
%TCIDATA{<META NAME="SaveForMode" CONTENT="1">}
%TCIDATA{BibliographyScheme=Manual}
%TCIDATA{Created=Wednesday, November 25, 2015 15:33:37}
%TCIDATA{LastRevised=Friday, December 25, 2015 12:50:01}
%TCIDATA{<META NAME="GraphicsSave" CONTENT="32">}
%TCIDATA{<META NAME="DocumentShell" CONTENT="Standard LaTeX\Blank - Standard LaTeX Article">}
%TCIDATA{CSTFile=40 LaTeX article.cst}
%TCIDATA{ComputeDefs=
%$W=\left( 1-\sigma \right) I$
%}


\newtheorem{theorem}{Theorem}
\newtheorem{acknowledgement}[theorem]{Acknowledgement}
\newtheorem{algorithm}[theorem]{Algorithm}
\newtheorem{axiom}[theorem]{Axiom}
\newtheorem{case}[theorem]{Case}
\newtheorem{claim}[theorem]{Claim}
\newtheorem{conclusion}[theorem]{Conclusion}
\newtheorem{condition}[theorem]{Condition}
\newtheorem{conjecture}[theorem]{Conjecture}
\newtheorem{corollary}[theorem]{Corollary}
\newtheorem{criterion}[theorem]{Criterion}
\newtheorem{definition}[theorem]{Definition}
\newtheorem{example}[theorem]{Example}
\newtheorem{exercise}[theorem]{Exercise}
\newtheorem{lemma}[theorem]{Lemma}
\newtheorem{notation}[theorem]{Notation}
\newtheorem{problem}[theorem]{Problem}
\newtheorem{proposition}[theorem]{Proposition}
\newtheorem{remark}[theorem]{Remark}
\newtheorem{solution}[theorem]{Solution}
\newtheorem{summary}[theorem]{Summary}
\newenvironment{proof}[1][Proof]{\noindent\textbf{#1.} }{\ \rule{0.5em}{0.5em}}
\input{../tcilatex}
\DeclareMathAccent{\wtilde}{\mathord}{largesymbols}{"65}
\pagestyle{fancy}
\fancyfoot[C]{\thepage}


\begin{document}


\setcounter{part}{6} \setcounter{page}{38}

\bigskip

\part{Sensitivity Analysis}

\begin{equation*}
%TCIMACRO{\TeXButton{underaccent_Y}{\underaccent{\wtilde}{Y}}}%
%BeginExpansion
\underaccent{\wtilde}{Y}%
%EndExpansion
=x%
%TCIMACRO{\TeXButton{underaccent_beta}{\underaccent{\wtilde}{\beta}}}%
%BeginExpansion
\underaccent{\wtilde}{\beta}%
%EndExpansion
+%
%TCIMACRO{\TeXButton{underaccent_epsilon}{\underaccent{\wtilde}{\epsilon}}}%
%BeginExpansion
\underaccent{\wtilde}{\epsilon}%
%EndExpansion
\quad E\left( 
%TCIMACRO{\TeXButton{underaccent_epsilon}{\underaccent{\wtilde}{\epsilon}}}%
%BeginExpansion
\underaccent{\wtilde}{\epsilon}%
%EndExpansion
\right)
\end{equation*}%
\begin{equation*}
Cov\left( 
%TCIMACRO{\TeXButton{underaccent_epsilon}{\underaccent{\wtilde}{\epsilon}}}%
%BeginExpansion
\underaccent{\wtilde}{\epsilon}%
%EndExpansion
\right) =\sigma ^{2}I\quad \hat{\beta}=\left( x^{\dagger }x^{-1}\right)
x^{\dagger }Y
\end{equation*}

$x\hat{\beta}\longrightarrow x\beta $, $%
%TCIMACRO{\TeXButton{underaccent_e}{\underaccent{\wtilde}{e}}}%
%BeginExpansion
\underaccent{\wtilde}{e}%
%EndExpansion
=\left( 
%TCIMACRO{\TeXButton{underaccent_Y}{\underaccent{\wtilde}{Y}}}%
%BeginExpansion
\underaccent{\wtilde}{Y}%
%EndExpansion
-x\hat{\beta}\right) \longrightarrow $,$%
%TCIMACRO{\TeXButton{underaccent_epsilon}{\underaccent{\wtilde}{\epsilon}}}%
%BeginExpansion
\underaccent{\wtilde}{\epsilon}%
%EndExpansion
$ use $%
%TCIMACRO{\TeXButton{underaccent_e}{\underaccent{\wtilde}{e}}}%
%BeginExpansion
\underaccent{\wtilde}{e}%
%EndExpansion
$ behaviour to understand the model's propernous.

\begin{eqnarray*}
\hat{Y} &=&x\hat{\beta} \\
&=&x\left( x^{\dagger }x^{-1}\right) x^{\dagger }%
%TCIMACRO{\TeXButton{underaccent_Y}{\underaccent{\wtilde}{Y}}}%
%BeginExpansion
\underaccent{\wtilde}{Y}%
%EndExpansion
\quad \text{rank(x)=k} \\
&=&PY \\
&=&\left[ P_{ij}\right] 
%TCIMACRO{\TeXButton{underaccent_Y}{\underaccent{\wtilde}{Y}} }%
%BeginExpansion
\underaccent{\wtilde}{Y}
%EndExpansion
\\
&=&P_{i}%
%TCIMACRO{\TeXButton{underaccent_Y}{\underaccent{\wtilde}{Y}} }%
%BeginExpansion
\underaccent{\wtilde}{Y}
%EndExpansion
\\
&=&P_{i}\left[ 
\begin{array}{ccc}
Y_{1} & \cdots  & Y_{n}%
\end{array}%
\right] ^{\dagger }
\end{eqnarray*}%
P: prediction matrix (i.e. $\hat{Y}=PY$), I-P: residual matrix (i.e. $%
%TCIMACRO{\TeXButton{underaccent_e}{\underaccent{\wtilde}{e}}}%
%BeginExpansion
\underaccent{\wtilde}{e}%
%EndExpansion
=\left( I-P\right) Y$)

\begin{equation*}
%TCIMACRO{\TeXButton{underaccent_e}{\underaccent{\wtilde}{e}}}%
%BeginExpansion
\underaccent{\wtilde}{e}%
%EndExpansion
=%
%TCIMACRO{\TeXButton{underaccent_Y}{\underaccent{\wtilde}{Y}}}%
%BeginExpansion
\underaccent{\wtilde}{Y}%
%EndExpansion
-%
%TCIMACRO{\TeXButton{underaccent_Y_hat}{\underaccent{\wtilde}{\hat{Y}}}}%
%BeginExpansion
\underaccent{\wtilde}{\hat{Y}}%
%EndExpansion
=%
%TCIMACRO{\TeXButton{underaccent_Y}{\underaccent{\wtilde}{Y}}}%
%BeginExpansion
\underaccent{\wtilde}{Y}%
%EndExpansion
-x\hat{\beta}
\end{equation*}%
\begin{equation*}
=\left[ I-x\left( x^{\dagger }x^{-1}\right) x^{\dagger }\right] 
%TCIMACRO{\TeXButton{underaccent_Y}{\underaccent{\wtilde}{Y}}}%
%BeginExpansion
\underaccent{\wtilde}{Y}%
%EndExpansion
=\left( I-P\right) 
%TCIMACRO{\TeXButton{underaccent_Y}{\underaccent{\wtilde}{Y}}}%
%BeginExpansion
\underaccent{\wtilde}{Y}%
%EndExpansion
=\underset{%
\begin{array}{c}
\text{indempotent} \\ 
\text{matrix}%
\end{array}%
}{M}%
%TCIMACRO{\TeXButton{underaccent_Y}{\underaccent{\wtilde}{Y}}}%
%BeginExpansion
\underaccent{\wtilde}{Y}%
%EndExpansion
\end{equation*}

\bigskip

\begin{itemize}
\item What are the similarities and disimilarities of $%
%TCIMACRO{\TeXButton{underaccent_e}{\underaccent{\wtilde}{e}}}%
%BeginExpansion
\underaccent{\wtilde}{e}%
%EndExpansion
$ and $%
%TCIMACRO{\TeXButton{underaccent_epsilon}{\underaccent{\wtilde}{\epsilon}}}%
%BeginExpansion
\underaccent{\wtilde}{\epsilon}%
%EndExpansion
$%
\begin{equation*}
\chi =\left[ x_{ij}\right] \quad \text{i=1}\cdots \text{n, j=1}\cdots \text{K%
}\qquad P_{ij}=%
%TCIMACRO{\TeXButton{underaccent_x}{\underaccent{\wtilde}{x}}}%
%BeginExpansion
\underaccent{\wtilde}{x}%
%EndExpansion
_{i}^{\dagger }\left( x^{\dagger }x^{-1}\right) ^{-1}%
%TCIMACRO{\TeXButton{underaccent_x}{\underaccent{\wtilde}{x}}}%
%BeginExpansion
\underaccent{\wtilde}{x}%
%EndExpansion
_{j}
\end{equation*}%
\begin{equation*}
\hat{Y}_{1}=P_{1}%
%TCIMACRO{\TeXButton{underaccent_Y}{\underaccent{\wtilde}{Y}}}%
%BeginExpansion
\underaccent{\wtilde}{Y}%
%EndExpansion
=\tsum\limits_{j=1}^{n}P_{ij}Y_{j}=P_{11}Y_{1}+\tsum\limits_{j\neq
1}^{n}P_{ij}Y_{j}
\end{equation*}%
$P_{ij}$: leverage $\frac{\partial \hat{Y}_{i}}{\partial Y_{i}}=P_{ii}$%
\begin{eqnarray*}
Cov\left( \hat{Y}\right) &=&Cov\left( x\hat{\beta}\right) =xCov\left( \hat{%
\beta}\right) x^{\dagger }=x\left( x^{\dagger }x^{-1}\right) ^{-1}\sigma
^{2}x^{\dagger } \\
&=&\sigma ^{2}x\left( x^{\dagger }x^{-1}\right) ^{-1}x^{\dagger }=P\sigma
^{2} \\
&=&\left[ 
\begin{array}{cccc}
P_{11} &  &  &  \\ 
& P_{22} & P_{ij} &  \\ 
&  & \ddots &  \\ 
&  &  & P_{44}%
\end{array}%
\right] \sigma ^{2}
\end{eqnarray*}%
$\left( \text{i.e. the bigger P}_{ii}\text{, the bigger }\hat{Y}_{i}\right) $%
\begin{eqnarray*}
Var\left( 
%TCIMACRO{\TeXButton{underaccent_e}{\underaccent{\wtilde}{e}}}%
%BeginExpansion
\underaccent{\wtilde}{e}%
%EndExpansion
\right) &=&\left( I-P\right) Var\left( 
%TCIMACRO{\TeXButton{underaccent_Y}{\underaccent{\wtilde}{Y}}}%
%BeginExpansion
\underaccent{\wtilde}{Y}%
%EndExpansion
\right) \left( I-P\right) =\sigma ^{2}\left( I-P\right) \\
&=&\sigma ^{2}\left[ 
\begin{array}{cccc}
1-P_{11} &  &  &  \\ 
& 1-P_{22} & -P_{ij} &  \\ 
&  & \ddots &  \\ 
&  &  & 1-P_{44}%
\end{array}%
\right]
\end{eqnarray*}%
\begin{eqnarray*}
Var\left( \hat{Y}_{i}\right) &=&P_{ii}\sigma ^{2} \\
Var\left( \hat{\varepsilon}_{i}\right) &=&\left( 1-P_{ii}\right) \sigma ^{2}
\\
Cov\left( \hat{\varepsilon}_{i},\hat{\varepsilon}_{j}\right) &=&\sigma
^{2}\left( -P_{ij}\right)
\end{eqnarray*}%
Note: the bigger P$_{ii}$, the bigger $\hat{Y}_{i}$, but the smaller $%
Var\left( \hat{\varepsilon}_{i}\right) $, theoretically not suppose to
happen.
\end{itemize}

\bigskip

\begin{equation*}
Y_{i}=\beta _{0}+\beta _{1}x_{i}+\varepsilon _{i}=\beta _{0}^{\ast }+\beta
_{1}\left( x_{i}-\bar{x}\right) +\varepsilon _{i}
\end{equation*}%
\begin{eqnarray*}
P &=&x\left( x^{\dagger }x^{-1}\right) ^{-1}x^{\dagger } \\
&=&\left[ 
\begin{array}{cc}
1 & x_{1}-\bar{x} \\ 
\vdots  & \vdots  \\ 
1 & x_{n}-\bar{x}%
\end{array}%
\right] \left[ \left[ 
\begin{array}{cc}
1 & x_{1}-\bar{x} \\ 
\vdots  & \vdots  \\ 
1 & x_{n}-\bar{x}%
\end{array}%
\right] ^{\dagger }\left[ 
\begin{array}{cc}
1 & x_{1}-\bar{x} \\ 
\vdots  & \vdots  \\ 
1 & x_{n}-\bar{x}%
\end{array}%
\right] \right] \left[ 
\begin{array}{cc}
1 & x_{1}-\bar{x} \\ 
\vdots  & \vdots  \\ 
1 & x_{n}-\bar{x}%
\end{array}%
\right] ^{\dagger } \\
&=&\left[ 
\begin{array}{cc}
1 & x_{1}-\bar{x} \\ 
\vdots  & \vdots  \\ 
1 & x_{n}-\bar{x}%
\end{array}%
\right] \left[ 
\begin{array}{cc}
n & 0 \\ 
0 & \tsum \left( x_{i}-\bar{x}\right) ^{2}%
\end{array}%
\right] ^{-1}\left[ 
\begin{array}{cc}
1 & x_{1}-\bar{x} \\ 
\vdots  & \vdots  \\ 
1 & x_{n}-\bar{x}%
\end{array}%
\right] ^{\dagger } \\
&=&\left[ 
\begin{array}{ccc}
\frac{1}{n}+\frac{\left( x_{1}-\bar{x}\right) ^{2}}{\tsum \left( x_{i}-\bar{x%
}\right) ^{2}} &  &  \\ 
& \ddots  &  \\ 
&  & \frac{1}{n}+\frac{\left( x_{n}-\bar{x}\right) ^{2}}{\tsum \left( x_{i}-%
\bar{x}\right) ^{2}}%
\end{array}%
\right] 
\end{eqnarray*}

\bigskip

\begin{equation*}
\because \hat{Y}_{1}=P_{11}Y_{1}+\tsum\limits_{j=2}^{n}P_{ij}Y_{j}
\end{equation*}%
if $P_{11}$ is bigger than contribution is bigger, meaning if $P_{11}$ is
bigger than regression will approach $Y_{1}$. This means the variance of $%
Y_{1}$ is bigger.

If one point is different than the others, than the point will affect \
regression result more, dangerous!%
\begin{equation*}
P=x\left( x^{\dagger }x^{-1}\right) ^{-1}x^{\dagger }=x_{1}\left( x^{\dagger
}x^{-1}\right) ^{-1}x_{1}^{\dagger }+\left( I-P_{1}\right) \left(
x_{2}^{\dagger }\left( I-P_{1}\right) x_{2}\right) ^{-1}x_{2}^{\dagger
}\left( I-P_{1}\right)
\end{equation*}%
$P_{1}=x_{1}\left( x_{1}^{\dagger }x_{1}^{-1}\right) ^{-1}x_{1}^{\dagger }$

\bigskip

$x=\left[ 
\begin{array}{cc}
x_{1} & x_{2}%
\end{array}%
\right] $

If%
\begin{equation*}
x_{1}=\left[ 
\begin{array}{c}
1 \\ 
\vdots \\ 
1%
\end{array}%
\right] \quad x_{2}=\left[ 
\begin{array}{c}
x_{1} \\ 
\vdots \\ 
x_{2}%
\end{array}%
\right]
\end{equation*}

\begin{equation*}
P=\left[ 
\begin{array}{ccc}
\frac{1}{n} &  & \frac{1}{n} \\ 
& \ddots &  \\ 
\frac{1}{n} &  & \frac{1}{n}%
\end{array}%
\right] +\left[ 
\begin{array}{c}
x_{1}-\bar{x} \\ 
\vdots \\ 
x_{n}-\bar{x}%
\end{array}%
\right] \frac{1}{\tsum \left( x_{i}-\bar{x}\right) ^{2}}\left[ 
\begin{array}{ccc}
x_{1}-\bar{x} & \cdots & x_{n}-\bar{x}%
\end{array}%
\right]
\end{equation*}

P is leverage, P$_{ij}$ will reflect on estimation. P$_{ij}$ bigger than
effect is bigger, ....

\bigskip

P=PP (idempotency)%
\begin{equation*}
P_{ii}=\tsum\limits_{j=1}^{n}P_{ij}^{2}=P_{ii}^{2}+\tsum%
\limits_{j=2}^{n}P_{ij}^{2}\Longrightarrow 0\leq P_{ii}\leq 1
\end{equation*}%
\begin{equation*}
\left( 
\begin{array}{c}
P_{ii}\rightarrow 1\text{, meaning outlier possible} \\ 
P_{ii}\rightarrow 0\text{, meaning closer to mean}%
\end{array}%
\right)
\end{equation*}

\bigskip

Also $P_{ii}=P_{ii}^{2}+P_{ij}^{2}+\tsum\limits_{k=i,j}P_{ik}^{2}$ (j fixed)%
\begin{eqnarray*}
&\Rightarrow &P_{ij}^{2}\leq P_{ii}^{2}\left( 1-P_{ii}^{2}\right) \leq \frac{%
1}{4} \\
&\Rightarrow &-\frac{1}{2}\leq P_{ij}\leq \frac{1}{2}
\end{eqnarray*}

\begin{itemize}
\item Results

\begin{enumerate}
\item If $P_{ii}=1$, or $0\Rightarrow P_{ij}=0$%
\begin{equation*}
\hat{Y}_{1}=P_{11}Y_{1}+\tsum\limits_{j=2}^{n}P_{ij}Y_{j}=Y_{1}\quad e_{1}=0
\end{equation*}

\item $P_{ii}\times P_{jj}-P_{ij}^{2}\geq 0$%
\begin{equation*}
P=\left[ 
\begin{array}{cc}
P_{11} & P_{12} \\ 
P_{21} & P_{22}%
\end{array}%
\right] \text{ nonegative}
\end{equation*}

\item $\left( 1-P_{ii}\right) \times \left( 1-P_{jj}\right) -P_{ij}^{2}\geq
0 $, I-P: nonegative

\item $P_{ii}+\frac{e_{i}^{2}}{%
%TCIMACRO{\TeXButton{underaccent_e}{\underaccent{\wtilde}{e}}}%
%BeginExpansion
\underaccent{\wtilde}{e}%
%EndExpansion
^{\dagger }%
%TCIMACRO{\TeXButton{underaccent_e}{\underaccent{\wtilde}{e}}}%
%BeginExpansion
\underaccent{\wtilde}{e}%
%EndExpansion
}\leq 1$

\begin{itemize}
\item use this to diagnose outlier

\item diagnose reg- design

\item cook distant emphasizes $P_{ii}$

\item if $P_{ii}$ is big than $\frac{e_{i}^{2}}{%
%TCIMACRO{\TeXButton{underaccent_e}{\underaccent{\wtilde}{e}}}%
%BeginExpansion
\underaccent{\wtilde}{e}%
%EndExpansion
^{\dagger }%
%TCIMACRO{\TeXButton{underaccent_e}{\underaccent{\wtilde}{e}}}%
%BeginExpansion
\underaccent{\wtilde}{e}%
%EndExpansion
}$ is small.
\end{itemize}

\begin{proof}
\begin{equation*}
Z=\left( x,Y\right) \quad P_{x}=x\left( x^{\dagger }x^{-1}\right)
^{-1}x^{\dagger }
\end{equation*}
\end{proof}
\end{enumerate}
\end{itemize}

\end{document}
