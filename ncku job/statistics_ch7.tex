
\documentclass{article}
%%%%%%%%%%%%%%%%%%%%%%%%%%%%%%%%%%%%%%%%%%%%%%%%%%%%%%%%%%%%%%%%%%%%%%%%%%%%%%%%%%%%%%%%%%%%%%%%%%%%%%%%%%%%%%%%%%%%%%%%%%%%%%%%%%%%%%%%%%%%%%%%%%%%%%%%%%%%%%%%%%%%%%%%%%%%%%%%%%%%%%%%%%%%%%%%%%%%%%%%%%%%%%%%%%%%%%%%%%%%%%%%%%%%%%%%%%%%%%%%%%%%%%%%%%%%
\usepackage{amssymb}
\usepackage{amsfonts}
\usepackage{amsmath}
\usepackage{accents}
\usepackage[ignoreall,a4paper]{geometry}
\usepackage{fancyhdr}

\setcounter{MaxMatrixCols}{10}
%TCIDATA{OutputFilter=LATEX.DLL}
%TCIDATA{Version=5.00.0.2606}
%TCIDATA{<META NAME="SaveForMode" CONTENT="1">}
%TCIDATA{BibliographyScheme=Manual}
%TCIDATA{Created=Wednesday, November 25, 2015 15:33:37}
%TCIDATA{LastRevised=Tuesday, December 29, 2015 16:30:19}
%TCIDATA{<META NAME="GraphicsSave" CONTENT="32">}
%TCIDATA{<META NAME="DocumentShell" CONTENT="Standard LaTeX\Blank - Standard LaTeX Article">}
%TCIDATA{CSTFile=40 LaTeX article.cst}
%TCIDATA{ComputeDefs=
%$W=\left( 1-\sigma \right) I$
%}


\newtheorem{theorem}{Theorem}
\newtheorem{acknowledgement}[theorem]{Acknowledgement}
\newtheorem{algorithm}[theorem]{Algorithm}
\newtheorem{axiom}[theorem]{Axiom}
\newtheorem{case}[theorem]{Case}
\newtheorem{claim}[theorem]{Claim}
\newtheorem{conclusion}[theorem]{Conclusion}
\newtheorem{condition}[theorem]{Condition}
\newtheorem{conjecture}[theorem]{Conjecture}
\newtheorem{corollary}[theorem]{Corollary}
\newtheorem{criterion}[theorem]{Criterion}
\newtheorem{definition}[theorem]{Definition}
\newtheorem{example}[theorem]{Example}
\newtheorem{exercise}[theorem]{Exercise}
\newtheorem{lemma}[theorem]{Lemma}
\newtheorem{notation}[theorem]{Notation}
\newtheorem{problem}[theorem]{Problem}
\newtheorem{proposition}[theorem]{Proposition}
\newtheorem{remark}[theorem]{Remark}
\newtheorem{solution}[theorem]{Solution}
\newtheorem{summary}[theorem]{Summary}
\newenvironment{proof}[1][Proof]{\noindent\textbf{#1.} }{\ \rule{0.5em}{0.5em}}
\input{../tcilatex}
\DeclareMathAccent{\wtilde}{\mathord}{largesymbols}{"65}
\pagestyle{fancy}
\fancyfoot[C]{\thepage}


\begin{document}


\setcounter{part}{6} \setcounter{page}{38}

\bigskip

\part{Sensitivity Analysis}

\begin{equation*}
%TCIMACRO{\TeXButton{underaccent_Y}{\underaccent{\wtilde}{Y}}}%
%BeginExpansion
\underaccent{\wtilde}{Y}%
%EndExpansion
=x%
%TCIMACRO{\TeXButton{underaccent_beta}{\underaccent{\wtilde}{\beta}}}%
%BeginExpansion
\underaccent{\wtilde}{\beta}%
%EndExpansion
+%
%TCIMACRO{\TeXButton{underaccent_epsilon}{\underaccent{\wtilde}{\epsilon}}}%
%BeginExpansion
\underaccent{\wtilde}{\epsilon}%
%EndExpansion
\quad E\left( 
%TCIMACRO{\TeXButton{underaccent_epsilon}{\underaccent{\wtilde}{\epsilon}}}%
%BeginExpansion
\underaccent{\wtilde}{\epsilon}%
%EndExpansion
\right)
\end{equation*}%
\begin{equation*}
Cov\left( 
%TCIMACRO{\TeXButton{underaccent_epsilon}{\underaccent{\wtilde}{\epsilon}}}%
%BeginExpansion
\underaccent{\wtilde}{\epsilon}%
%EndExpansion
\right) =\sigma ^{2}I\quad \hat{\beta}=\left( x^{\dagger }x^{-1}\right)
x^{\dagger }Y
\end{equation*}

$x\hat{\beta}\longrightarrow x\beta $, $%
%TCIMACRO{\TeXButton{underaccent_e}{\underaccent{\wtilde}{e}}}%
%BeginExpansion
\underaccent{\wtilde}{e}%
%EndExpansion
=\left( 
%TCIMACRO{\TeXButton{underaccent_Y}{\underaccent{\wtilde}{Y}}}%
%BeginExpansion
\underaccent{\wtilde}{Y}%
%EndExpansion
-x\hat{\beta}\right) \longrightarrow $,$%
%TCIMACRO{\TeXButton{underaccent_epsilon}{\underaccent{\wtilde}{\epsilon}}}%
%BeginExpansion
\underaccent{\wtilde}{\epsilon}%
%EndExpansion
$ use $%
%TCIMACRO{\TeXButton{underaccent_e}{\underaccent{\wtilde}{e}}}%
%BeginExpansion
\underaccent{\wtilde}{e}%
%EndExpansion
$ behaviour to understand the model's propernous.

\begin{eqnarray*}
\hat{Y} &=&x\hat{\beta} \\
&=&x\left( x^{\dagger }x^{-1}\right) x^{\dagger }%
%TCIMACRO{\TeXButton{underaccent_Y}{\underaccent{\wtilde}{Y}}}%
%BeginExpansion
\underaccent{\wtilde}{Y}%
%EndExpansion
\quad \text{rank(x)=k} \\
&=&PY \\
&=&\left[ P_{ij}\right] 
%TCIMACRO{\TeXButton{underaccent_Y}{\underaccent{\wtilde}{Y}} }%
%BeginExpansion
\underaccent{\wtilde}{Y}
%EndExpansion
\\
&=&P_{i}%
%TCIMACRO{\TeXButton{underaccent_Y}{\underaccent{\wtilde}{Y}} }%
%BeginExpansion
\underaccent{\wtilde}{Y}
%EndExpansion
\\
&=&P_{i}\left[ 
\begin{array}{ccc}
Y_{1} & \cdots & Y_{n}%
\end{array}%
\right] ^{\dagger }
\end{eqnarray*}%
P: prediction matrix (i.e. $\hat{Y}=PY$), I-P: residual matrix (i.e. $%
%TCIMACRO{\TeXButton{underaccent_e}{\underaccent{\wtilde}{e}}}%
%BeginExpansion
\underaccent{\wtilde}{e}%
%EndExpansion
=\left( I-P\right) Y$)

\begin{equation*}
%TCIMACRO{\TeXButton{underaccent_e}{\underaccent{\wtilde}{e}}}%
%BeginExpansion
\underaccent{\wtilde}{e}%
%EndExpansion
=%
%TCIMACRO{\TeXButton{underaccent_Y}{\underaccent{\wtilde}{Y}}}%
%BeginExpansion
\underaccent{\wtilde}{Y}%
%EndExpansion
-%
%TCIMACRO{\TeXButton{underaccent_Y_hat}{\underaccent{\wtilde}{\hat{Y}}}}%
%BeginExpansion
\underaccent{\wtilde}{\hat{Y}}%
%EndExpansion
=%
%TCIMACRO{\TeXButton{underaccent_Y}{\underaccent{\wtilde}{Y}}}%
%BeginExpansion
\underaccent{\wtilde}{Y}%
%EndExpansion
-x\hat{\beta}
\end{equation*}%
\begin{equation*}
=\left[ I-x\left( x^{\dagger }x^{-1}\right) x^{\dagger }\right] 
%TCIMACRO{\TeXButton{underaccent_Y}{\underaccent{\wtilde}{Y}}}%
%BeginExpansion
\underaccent{\wtilde}{Y}%
%EndExpansion
=\left( I-P\right) 
%TCIMACRO{\TeXButton{underaccent_Y}{\underaccent{\wtilde}{Y}}}%
%BeginExpansion
\underaccent{\wtilde}{Y}%
%EndExpansion
=\underset{%
\begin{array}{c}
\text{indempotent} \\ 
\text{matrix}%
\end{array}%
}{M}%
%TCIMACRO{\TeXButton{underaccent_Y}{\underaccent{\wtilde}{Y}}}%
%BeginExpansion
\underaccent{\wtilde}{Y}%
%EndExpansion
\end{equation*}

\bigskip

\begin{itemize}
\item What are the similarities and disimilarities of $%
%TCIMACRO{\TeXButton{underaccent_e}{\underaccent{\wtilde}{e}}}%
%BeginExpansion
\underaccent{\wtilde}{e}%
%EndExpansion
$ and $%
%TCIMACRO{\TeXButton{underaccent_epsilon}{\underaccent{\wtilde}{\epsilon}}}%
%BeginExpansion
\underaccent{\wtilde}{\epsilon}%
%EndExpansion
$%
\begin{equation*}
\chi =\left[ x_{ij}\right] \quad \text{i=1}\cdots \text{n, j=1}\cdots \text{K%
}\qquad P_{ij}=%
%TCIMACRO{\TeXButton{underaccent_x}{\underaccent{\wtilde}{x}}}%
%BeginExpansion
\underaccent{\wtilde}{x}%
%EndExpansion
_{i}^{\dagger }\left( x^{\dagger }x^{-1}\right) ^{-1}%
%TCIMACRO{\TeXButton{underaccent_x}{\underaccent{\wtilde}{x}}}%
%BeginExpansion
\underaccent{\wtilde}{x}%
%EndExpansion
_{j}
\end{equation*}%
\begin{equation*}
\hat{Y}_{1}=P_{1}%
%TCIMACRO{\TeXButton{underaccent_Y}{\underaccent{\wtilde}{Y}}}%
%BeginExpansion
\underaccent{\wtilde}{Y}%
%EndExpansion
=\tsum\limits_{j=1}^{n}P_{ij}Y_{j}=P_{11}Y_{1}+\tsum\limits_{j\neq
1}^{n}P_{ij}Y_{j}
\end{equation*}%
$P_{ij}$: leverage $\frac{\partial \hat{Y}_{i}}{\partial Y_{i}}=P_{ii}$%
\begin{eqnarray*}
Cov\left( \hat{Y}\right) &=&Cov\left( x\hat{\beta}\right) =xCov\left( \hat{%
\beta}\right) x^{\dagger }=x\left( x^{\dagger }x^{-1}\right) ^{-1}\sigma
^{2}x^{\dagger } \\
&=&\sigma ^{2}x\left( x^{\dagger }x^{-1}\right) ^{-1}x^{\dagger }=P\sigma
^{2} \\
&=&\left[ 
\begin{array}{cccc}
P_{11} &  &  &  \\ 
& P_{22} & P_{ij} &  \\ 
&  & \ddots &  \\ 
&  &  & P_{44}%
\end{array}%
\right] \sigma ^{2}
\end{eqnarray*}%
$\left( \text{i.e. the bigger P}_{ii}\text{, the bigger }\hat{Y}_{i}\right) $%
\begin{eqnarray*}
Var\left( 
%TCIMACRO{\TeXButton{underaccent_e}{\underaccent{\wtilde}{e}}}%
%BeginExpansion
\underaccent{\wtilde}{e}%
%EndExpansion
\right) &=&\left( I-P\right) Var\left( 
%TCIMACRO{\TeXButton{underaccent_Y}{\underaccent{\wtilde}{Y}}}%
%BeginExpansion
\underaccent{\wtilde}{Y}%
%EndExpansion
\right) \left( I-P\right) =\sigma ^{2}\left( I-P\right) \\
&=&\sigma ^{2}\left[ 
\begin{array}{cccc}
1-P_{11} &  &  &  \\ 
& 1-P_{22} & -P_{ij} &  \\ 
&  & \ddots &  \\ 
&  &  & 1-P_{44}%
\end{array}%
\right]
\end{eqnarray*}%
\begin{eqnarray*}
Var\left( \hat{Y}_{i}\right) &=&P_{ii}\sigma ^{2} \\
Var\left( \hat{\varepsilon}_{i}\right) &=&\left( 1-P_{ii}\right) \sigma ^{2}
\\
Cov\left( \hat{\varepsilon}_{i},\hat{\varepsilon}_{j}\right) &=&\sigma
^{2}\left( -P_{ij}\right)
\end{eqnarray*}%
Note: the bigger P$_{ii}$, the bigger $\hat{Y}_{i}$, but the smaller $%
Var\left( \hat{\varepsilon}_{i}\right) $, theoretically not suppose to
happen.
\end{itemize}

\bigskip

\begin{equation*}
Y_{i}=\beta _{0}+\beta _{1}x_{i}+\varepsilon _{i}=\beta _{0}^{\ast }+\beta
_{1}\left( x_{i}-\bar{x}\right) +\varepsilon _{i}
\end{equation*}%
\begin{eqnarray*}
P &=&x\left( x^{\dagger }x^{-1}\right) ^{-1}x^{\dagger } \\
&=&\left[ 
\begin{array}{cc}
1 & x_{1}-\bar{x} \\ 
\vdots & \vdots \\ 
1 & x_{n}-\bar{x}%
\end{array}%
\right] \left[ \left[ 
\begin{array}{cc}
1 & x_{1}-\bar{x} \\ 
\vdots & \vdots \\ 
1 & x_{n}-\bar{x}%
\end{array}%
\right] ^{\dagger }\left[ 
\begin{array}{cc}
1 & x_{1}-\bar{x} \\ 
\vdots & \vdots \\ 
1 & x_{n}-\bar{x}%
\end{array}%
\right] \right] \left[ 
\begin{array}{cc}
1 & x_{1}-\bar{x} \\ 
\vdots & \vdots \\ 
1 & x_{n}-\bar{x}%
\end{array}%
\right] ^{\dagger } \\
&=&\left[ 
\begin{array}{cc}
1 & x_{1}-\bar{x} \\ 
\vdots & \vdots \\ 
1 & x_{n}-\bar{x}%
\end{array}%
\right] \left[ 
\begin{array}{cc}
n & 0 \\ 
0 & \tsum \left( x_{i}-\bar{x}\right) ^{2}%
\end{array}%
\right] ^{-1}\left[ 
\begin{array}{cc}
1 & x_{1}-\bar{x} \\ 
\vdots & \vdots \\ 
1 & x_{n}-\bar{x}%
\end{array}%
\right] ^{\dagger } \\
&=&\left[ 
\begin{array}{ccc}
\frac{1}{n}+\frac{\left( x_{1}-\bar{x}\right) ^{2}}{\tsum \left( x_{i}-\bar{x%
}\right) ^{2}} &  &  \\ 
& \ddots &  \\ 
&  & \frac{1}{n}+\frac{\left( x_{n}-\bar{x}\right) ^{2}}{\tsum \left( x_{i}-%
\bar{x}\right) ^{2}}%
\end{array}%
\right]
\end{eqnarray*}

\bigskip

\begin{equation*}
\because \hat{Y}_{1}=P_{11}Y_{1}+\tsum\limits_{j=2}^{n}P_{ij}Y_{j}
\end{equation*}%
if $P_{11}$ is bigger than contribution is bigger, meaning if $P_{11}$ is
bigger than regression will approach $Y_{1}$. This means the variance of $%
Y_{1}$ is bigger.

If one point is different than the others, than the point will affect \
regression result more, dangerous!%
\begin{equation*}
P=x\left( x^{\dagger }x^{-1}\right) ^{-1}x^{\dagger }=x_{1}\left( x^{\dagger
}x^{-1}\right) ^{-1}x_{1}^{\dagger }+\left( I-P_{1}\right) \left(
x_{2}^{\dagger }\left( I-P_{1}\right) x_{2}\right) ^{-1}x_{2}^{\dagger
}\left( I-P_{1}\right)
\end{equation*}%
$P_{1}=x_{1}\left( x_{1}^{\dagger }x_{1}^{-1}\right) ^{-1}x_{1}^{\dagger }$

\bigskip

$x=\left[ 
\begin{array}{cc}
x_{1} & x_{2}%
\end{array}%
\right] $

If%
\begin{equation*}
x_{1}=\left[ 
\begin{array}{c}
1 \\ 
\vdots \\ 
1%
\end{array}%
\right] \quad x_{2}=\left[ 
\begin{array}{c}
x_{1} \\ 
\vdots \\ 
x_{n}%
\end{array}%
\right]
\end{equation*}

\begin{equation*}
P=\left[ 
\begin{array}{ccc}
\frac{1}{n} &  & \frac{1}{n} \\ 
& \ddots &  \\ 
\frac{1}{n} &  & \frac{1}{n}%
\end{array}%
\right] +\left[ 
\begin{array}{c}
x_{1}-\bar{x} \\ 
\vdots \\ 
x_{n}-\bar{x}%
\end{array}%
\right] \frac{1}{\tsum \left( x_{i}-\bar{x}\right) ^{2}}\left[ 
\begin{array}{ccc}
x_{1}-\bar{x} & \cdots & x_{n}-\bar{x}%
\end{array}%
\right]
\end{equation*}

P is leverage, P$_{ij}$ will reflect on estimation. P$_{ij}$ bigger than
effect is bigger, ....

\bigskip

P=PP (idempotency)%
\begin{equation*}
P_{ii}=\tsum\limits_{j=1}^{n}P_{ij}^{2}=P_{ii}^{2}+\tsum%
\limits_{j=2}^{n}P_{ij}^{2}\Longrightarrow 0\leq P_{ii}\leq 1
\end{equation*}%
\begin{equation*}
\left( 
\begin{array}{c}
P_{ii}\rightarrow 1\text{, meaning outlier possible} \\ 
P_{ii}\rightarrow 0\text{, meaning closer to mean}%
\end{array}%
\right)
\end{equation*}

\bigskip

Also $P_{ii}=P_{ii}^{2}+P_{ij}^{2}+\tsum\limits_{k=i,j}P_{ik}^{2}$ (j fixed)%
\begin{eqnarray*}
&\Rightarrow &P_{ij}^{2}\leq P_{ii}^{2}\left( 1-P_{ii}^{2}\right) \leq \frac{%
1}{4} \\
&\Rightarrow &-\frac{1}{2}\leq P_{ij}\leq \frac{1}{2}
\end{eqnarray*}

\begin{itemize}
\item Results

\begin{enumerate}
\item If $P_{ii}=1$, or $0\Rightarrow P_{ij}=0$%
\begin{equation*}
\hat{Y}_{1}=P_{11}Y_{1}+\tsum\limits_{j=2}^{n}P_{ij}Y_{j}=Y_{1}\quad e_{1}=0
\end{equation*}

\item $P_{ii}\times P_{jj}-P_{ij}^{2}\geq 0$%
\begin{equation*}
P=\left[ 
\begin{array}{cc}
P_{11} & P_{12} \\ 
P_{21} & P_{22}%
\end{array}%
\right] \text{ nonegative}
\end{equation*}

\item $\left( 1-P_{ii}\right) \times \left( 1-P_{jj}\right) -P_{ij}^{2}\geq
0 $, I-P: nonegative

\item $P_{ii}+\frac{e_{i}^{2}}{%
%TCIMACRO{\TeXButton{underaccent_e}{\underaccent{\wtilde}{e}}}%
%BeginExpansion
\underaccent{\wtilde}{e}%
%EndExpansion
^{\dagger }%
%TCIMACRO{\TeXButton{underaccent_e}{\underaccent{\wtilde}{e}}}%
%BeginExpansion
\underaccent{\wtilde}{e}%
%EndExpansion
}\leq 1$

\begin{itemize}
\item use this to diagnose outlier

\item diagnose reg- design

\item cook distant emphasizes $P_{ii}$

\item if $P_{ii}$ is big than $\frac{e_{i}^{2}}{%
%TCIMACRO{\TeXButton{underaccent_e}{\underaccent{\wtilde}{e}}}%
%BeginExpansion
\underaccent{\wtilde}{e}%
%EndExpansion
^{\dagger }%
%TCIMACRO{\TeXButton{underaccent_e}{\underaccent{\wtilde}{e}}}%
%BeginExpansion
\underaccent{\wtilde}{e}%
%EndExpansion
}$ is small.
\end{itemize}

\begin{proof}
\begin{equation*}
Z=\left( x,Y\right) \quad P_{x}=x\left( x^{\dagger }x^{-1}\right)
^{-1}x^{\dagger }
\end{equation*}%
\begin{eqnarray*}
P_{z} &=&x\left( x^{\dagger }x^{-1}\right) ^{-1}x^{\dagger }+\left(
I-P_{x}\right) Y\left( Y^{\dagger }\left( I-P_{x}\right) Y\right) ^{-1}\left[
\left( I-P_{x}\right) Y\right] ^{\dagger } \\
&=&P_{x}+\frac{e_{i}^{2}}{%
%TCIMACRO{\TeXButton{underaccent_e}{\underaccent{\wtilde}{e}}}%
%BeginExpansion
\underaccent{\wtilde}{e}%
%EndExpansion
^{\dagger }%
%TCIMACRO{\TeXButton{underaccent_e}{\underaccent{\wtilde}{e}}}%
%BeginExpansion
\underaccent{\wtilde}{e}%
%EndExpansion
}
\end{eqnarray*}%
\begin{equation*}
P_{ii}+\frac{e_{i}^{2}}{%
%TCIMACRO{\TeXButton{underaccent_e}{\underaccent{\wtilde}{e}}}%
%BeginExpansion
\underaccent{\wtilde}{e}%
%EndExpansion
^{\dagger }%
%TCIMACRO{\TeXButton{underaccent_e}{\underaccent{\wtilde}{e}}}%
%BeginExpansion
\underaccent{\wtilde}{e}%
%EndExpansion
}=P_{ii}^{\left( 2\right) }\leq 1
\end{equation*}%
Assume two variables%
\begin{equation*}
\text{one-dim: }\frac{\left( x_{i}-\bar{x}\right) ^{2}}{\tsum \left( x_{i}-%
\bar{x}\right) ^{2}}
\end{equation*}%
although distance the same, 1,2 have different meanings, may consider
relavent variant.%
\begin{equation*}
\Pr \left( \left( 
%TCIMACRO{\TeXButton{underaccent_x}{\underaccent{\wtilde}{x}}}%
%BeginExpansion
\underaccent{\wtilde}{x}%
%EndExpansion
-\mu \right) ^{\dagger }\tsum\nolimits^{-1}\left( 
%TCIMACRO{\TeXButton{underaccent_x}{\underaccent{\wtilde}{x}}}%
%BeginExpansion
\underaccent{\wtilde}{x}%
%EndExpansion
-\mu \right) \leq \chi _{0}^{2}\right) =0.95
\end{equation*}%
a large eig. value means long axis, small eig. value means short axis. Along
the short axis it is more likely to have large $P_{ii}$ value.
\end{proof}
\end{enumerate}
\end{itemize}

\bigskip

\begin{itemize}
\item Case of multiple regression%
\begin{equation*}
\left( \lambda _{i},e_{i}^{\ast }\right) :\text{eigenvalue-eigenvector pari
of }x^{\dagger }x
\end{equation*}%
\begin{equation*}
\theta _{ij}:\text{the angle between the colum vector }%
%TCIMACRO{\TeXButton{underaccent_x}{\underaccent{\wtilde}{x}}}%
%BeginExpansion
\underaccent{\wtilde}{x}%
%EndExpansion
_{i}\text{ and }%
%TCIMACRO{\TeXButton{underaccent_e}{\underaccent{\wtilde}{e}}}%
%BeginExpansion
\underaccent{\wtilde}{e}%
%EndExpansion
^{\ast }
\end{equation*}%
\begin{eqnarray*}
P_{ij} &=&%
%TCIMACRO{\TeXButton{underaccent_x}{\underaccent{\wtilde}{x}}}%
%BeginExpansion
\underaccent{\wtilde}{x}%
%EndExpansion
_{i}^{\dagger }\left( x^{\dagger }x^{-1}\right) ^{-1}%
%TCIMACRO{\TeXButton{underaccent_x}{\underaccent{\wtilde}{x}}}%
%BeginExpansion
\underaccent{\wtilde}{x}%
%EndExpansion
_{j} \\
&=&%
%TCIMACRO{\TeXButton{underaccent_x}{\underaccent{\wtilde}{x}}}%
%BeginExpansion
\underaccent{\wtilde}{x}%
%EndExpansion
_{i}^{\dagger }P\Lambda ^{-1}P^{\dagger }%
%TCIMACRO{\TeXButton{underaccent_x}{\underaccent{\wtilde}{x}}}%
%BeginExpansion
\underaccent{\wtilde}{x}%
%EndExpansion
_{j} \\
&=&\tsum\limits_{r=1}^{k}%
%TCIMACRO{\TeXButton{underaccent_x}{\underaccent{\wtilde}{x}}}%
%BeginExpansion
\underaccent{\wtilde}{x}%
%EndExpansion
_{i}^{\dagger }\frac{1}{\lambda r}%
%TCIMACRO{\TeXButton{underaccent_e}{\underaccent{\wtilde}{e}}}%
%BeginExpansion
\underaccent{\wtilde}{e}%
%EndExpansion
_{r}^{\dagger }%
%TCIMACRO{\TeXButton{underaccent_e}{\underaccent{\wtilde}{e}}}%
%BeginExpansion
\underaccent{\wtilde}{e}%
%EndExpansion
_{r}^{\ast \dagger }%
%TCIMACRO{\TeXButton{underaccent_x}{\underaccent{\wtilde}{x}}}%
%BeginExpansion
\underaccent{\wtilde}{x}%
%EndExpansion
_{j} \\
&=&\tsum\limits_{r=1}^{k}\frac{1}{\lambda r}%
%TCIMACRO{\TeXButton{underaccent_x}{\underaccent{\wtilde}{x}}}%
%BeginExpansion
\underaccent{\wtilde}{x}%
%EndExpansion
_{i}^{\dagger }%
%TCIMACRO{\TeXButton{underaccent_e}{\underaccent{\wtilde}{e}}}%
%BeginExpansion
\underaccent{\wtilde}{e}%
%EndExpansion
_{r}^{\dagger }%
%TCIMACRO{\TeXButton{underaccent_e}{\underaccent{\wtilde}{e}}}%
%BeginExpansion
\underaccent{\wtilde}{e}%
%EndExpansion
_{r}^{\ast \dagger }%
%TCIMACRO{\TeXButton{underaccent_x}{\underaccent{\wtilde}{x}}}%
%BeginExpansion
\underaccent{\wtilde}{x}%
%EndExpansion
_{j} \\
&=&\left\Vert 
%TCIMACRO{\TeXButton{underaccent_x}{\underaccent{\wtilde}{x}}}%
%BeginExpansion
\underaccent{\wtilde}{x}%
%EndExpansion
_{j}\right\Vert \left\Vert 
%TCIMACRO{\TeXButton{underaccent_x}{\underaccent{\wtilde}{x}}}%
%BeginExpansion
\underaccent{\wtilde}{x}%
%EndExpansion
_{j}\right\Vert \tsum\limits_{r=1}^{k}\frac{1}{\lambda r}\cos \theta
_{ir}\cos \theta _{jr}
\end{eqnarray*}%
$\cos \theta _{ir}=\frac{x_{i}^{\dagger }%
%TCIMACRO{\TeXButton{underaccent_x}{\underaccent{\wtilde}{x}}}%
%BeginExpansion
\underaccent{\wtilde}{x}%
%EndExpansion
_{j}^{\ast }}{\left\Vert 
%TCIMACRO{\TeXButton{underaccent_x}{\underaccent{\wtilde}{x}}}%
%BeginExpansion
\underaccent{\wtilde}{x}%
%EndExpansion
_{i}\right\Vert \left\Vert 
%TCIMACRO{\TeXButton{underaccent_e}{\underaccent{\wtilde}{e}}}%
%BeginExpansion
\underaccent{\wtilde}{e}%
%EndExpansion
_{j}^{\ast }\right\Vert }$

\begin{enumerate}
\item $P_{ii}$ large means $\left\Vert 
%TCIMACRO{\TeXButton{underaccent_x}{\underaccent{\wtilde}{x}}}%
%BeginExpansion
\underaccent{\wtilde}{x}%
%EndExpansion
_{i}\right\Vert \cdot \left\Vert 
%TCIMACRO{\TeXButton{underaccent_x}{\underaccent{\wtilde}{x}}}%
%BeginExpansion
\underaccent{\wtilde}{x}%
%EndExpansion
_{i}\right\Vert $ is relatively larger than other

\item $x_{i}$ and the smallest eig-value of eig-vector are parralel $\left(
\cos \theta =1\right) $
\end{enumerate}
\end{itemize}

\bigskip

Q: when k dim, how to declare this value to be the largest?%
\begin{equation*}
\text{If i=j }\Rightarrow P_{ii}=\left\Vert 
%TCIMACRO{\TeXButton{underaccent_x}{\underaccent{\wtilde}{x}}}%
%BeginExpansion
\underaccent{\wtilde}{x}%
%EndExpansion
_{i}\right\Vert ^{2}\tsum\limits_{r=1}^{k}\frac{1}{\lambda r}\left( \cos
\theta _{ir}\right) ^{2}
\end{equation*}%
$P=\left[ P_{ij}\right] $, some people take $\frac{2h\left( P\right) }{n}=%
\frac{2k}{n}$ to compare. (i.e. if $P_{ij}>2\frac{k}{n}$, it means $P_{ij}$
is large)

\bigskip

Check the good and bad of the model

\begin{equation*}
\begin{array}{c}
\text{leverage: }P_{ii} \\ 
\text{residual: }e_{i}=Y_{i}-\hat{Y}_{i}%
\end{array}%
\end{equation*}%
measure of distance of $%
%TCIMACRO{\TeXButton{underaccent_x}{\underaccent{\wtilde}{x}}}%
%BeginExpansion
\underaccent{\wtilde}{x}%
%EndExpansion
_{i}$ away from $%
%TCIMACRO{\TeXButton{underaccent_x}{\underaccent{\wtilde}{x}}}%
%BeginExpansion
\underaccent{\wtilde}{x}%
%EndExpansion
$%
\begin{equation*}
\text{multiple rows\quad }\chi _{n\times k}=\left[ 
\begin{array}{c}
%TCIMACRO{\TeXButton{underaccent_x}{\underaccent{\wtilde}{x}}}%
%BeginExpansion
\underaccent{\wtilde}{x}%
%EndExpansion
_{1}^{\dagger } \\ 
\vdots \\ 
%TCIMACRO{\TeXButton{underaccent_x}{\underaccent{\wtilde}{x}}}%
%BeginExpansion
\underaccent{\wtilde}{x}%
%EndExpansion
_{n}^{\dagger }%
\end{array}%
\right]
\end{equation*}

Let $I=\left\{ j:%
%TCIMACRO{\TeXButton{underaccent_x}{\underaccent{\wtilde}{x}}}%
%BeginExpansion
\underaccent{\wtilde}{x}%
%EndExpansion
_{i}=%
%TCIMACRO{\TeXButton{underaccent_x}{\underaccent{\wtilde}{x}}}%
%BeginExpansion
\underaccent{\wtilde}{x}%
%EndExpansion
_{j}\right\} $ (size of I=a)

then%
\begin{equation*}
P_{ii}=\tsum\limits_{j=1}^{n}P_{ij}^{2}=aP_{ii}^{2}+\tsum\limits_{j\neq
I}P_{ij}^{2}\geq aP_{ii}^{2}\Rightarrow P_{ii}\leq \frac{1}{a}
\end{equation*}%
if data has duplicate measure

\bigskip

\begin{itemize}
\item Let $\left( 
%TCIMACRO{\TeXButton{underaccent_Y}{\underaccent{\wtilde}{Y}}}%
%BeginExpansion
\underaccent{\wtilde}{Y}%
%EndExpansion
_{\left( i\right) },x_{\left( i\right) }\right) $ denote the vector $\left(
Y_{i},%
%TCIMACRO{\TeXButton{underaccent_x}{\underaccent{\wtilde}{x}}}%
%BeginExpansion
\underaccent{\wtilde}{x}%
%EndExpansion
_{i}^{\dagger }\right) $ is omitted. Write $\left( \chi ^{\dagger }\right)
_{k\times n}=\left( 
%TCIMACRO{\TeXButton{underaccent_chi}{\underaccent{\wtilde}{\chi}}}%
%BeginExpansion
\underaccent{\wtilde}{\chi}%
%EndExpansion
_{1},\cdots ,%
%TCIMACRO{\TeXButton{underaccent_chi}{\underaccent{\wtilde}{\chi}}}%
%BeginExpansion
\underaccent{\wtilde}{\chi}%
%EndExpansion
_{n}\right) $, then%
\begin{equation*}
x^{\dagger }x=\tsum\limits_{i=1}^{n}%
%TCIMACRO{\TeXButton{underaccent_chi}{\underaccent{\wtilde}{\chi}}}%
%BeginExpansion
\underaccent{\wtilde}{\chi}%
%EndExpansion
_{i}%
%TCIMACRO{\TeXButton{underaccent_chi}{\underaccent{\wtilde}{\chi}}}%
%BeginExpansion
\underaccent{\wtilde}{\chi}%
%EndExpansion
_{i}^{\dagger }=x_{\left( i\right) }^{\dagger }x_{\left( i\right) }+%
%TCIMACRO{\TeXButton{underaccent_chi}{\underaccent{\wtilde}{\chi}}}%
%BeginExpansion
\underaccent{\wtilde}{\chi}%
%EndExpansion
_{i}%
%TCIMACRO{\TeXButton{underaccent_chi}{\underaccent{\wtilde}{\chi}}}%
%BeginExpansion
\underaccent{\wtilde}{\chi}%
%EndExpansion
_{i}^{\dagger }
\end{equation*}%
\newline
Assume rank($x_{\left( i\right) }$)=k\newline
if $%
%TCIMACRO{\TeXButton{underaccent_chi}{\underaccent{\wtilde}{\chi}}}%
%BeginExpansion
\underaccent{\wtilde}{\chi}%
%EndExpansion
_{i}^{\dagger }\left( x_{\left( i\right) }^{\dagger }x_{\left( i\right)
}\right) ^{-1}%
%TCIMACRO{\TeXButton{underaccent_chi}{\underaccent{\wtilde}{\chi}}}%
%BeginExpansion
\underaccent{\wtilde}{\chi}%
%EndExpansion
_{i}\neq 1$, then by A.18(?)%
\begin{equation*}
\left( x_{\left( i\right) }^{\dagger }x_{\left( i\right) }\right)
^{-1}=\left( x^{\dagger }x\right) ^{-1}+\frac{\left( x^{\dagger }x\right)
^{-1}%
%TCIMACRO{\TeXButton{underaccent_chi}{\underaccent{\wtilde}{\chi}}}%
%BeginExpansion
\underaccent{\wtilde}{\chi}%
%EndExpansion
_{i}%
%TCIMACRO{\TeXButton{underaccent_chi}{\underaccent{\wtilde}{\chi}}}%
%BeginExpansion
\underaccent{\wtilde}{\chi}%
%EndExpansion
_{i}^{\dagger }\left( x^{\dagger }x\right) ^{-1}}{1-x_{\left( i\right)
}^{\dagger }\left( x^{\dagger }x\right) ^{-1}x_{\left( i\right) }}
\end{equation*}%
\begin{equation*}
\left[ 
\begin{array}{c}
\text{if }x_{\left( i\right) }^{\dagger }\left( x^{\dagger }x\right)
^{-1}x_{\left( i\right) }\neq 1 \\ 
x^{\dagger }x=x_{\left( i\right) }^{\dagger }x_{\left( i\right) }+%
%TCIMACRO{\TeXButton{underaccent_chi}{\underaccent{\wtilde}{\chi}}}%
%BeginExpansion
\underaccent{\wtilde}{\chi}%
%EndExpansion
_{i}%
%TCIMACRO{\TeXButton{underaccent_chi}{\underaccent{\wtilde}{\chi}}}%
%BeginExpansion
\underaccent{\wtilde}{\chi}%
%EndExpansion
_{i}^{\dagger } \\ 
\Rightarrow \text{above equation}%
\end{array}%
\right]
\end{equation*}%
\begin{equation*}
%TCIMACRO{\TeXButton{underaccent_chi}{\underaccent{\wtilde}{\chi}}}%
%BeginExpansion
\underaccent{\wtilde}{\chi}%
%EndExpansion
_{r}^{\dagger }\left( x_{\left( i\right) }^{\dagger }x_{\left( i\right)
}\right) ^{-1}%
%TCIMACRO{\TeXButton{underaccent_chi}{\underaccent{\wtilde}{\chi}}}%
%BeginExpansion
\underaccent{\wtilde}{\chi}%
%EndExpansion
_{r}=%
%TCIMACRO{\TeXButton{underaccent_chi}{\underaccent{\wtilde}{\chi}}}%
%BeginExpansion
\underaccent{\wtilde}{\chi}%
%EndExpansion
_{r}^{\dagger }\left[ \left( x^{\dagger }x\right) ^{-1}+\frac{\left(
x^{\dagger }x\right) ^{-1}%
%TCIMACRO{\TeXButton{underaccent_chi}{\underaccent{\wtilde}{\chi}}}%
%BeginExpansion
\underaccent{\wtilde}{\chi}%
%EndExpansion
_{i}%
%TCIMACRO{\TeXButton{underaccent_chi}{\underaccent{\wtilde}{\chi}}}%
%BeginExpansion
\underaccent{\wtilde}{\chi}%
%EndExpansion
_{i}^{\dagger }\left( x^{\dagger }x\right) ^{-1}}{1-P_{ii}}\right] 
%TCIMACRO{\TeXButton{underaccent_chi}{\underaccent{\wtilde}{\chi}}}%
%BeginExpansion
\underaccent{\wtilde}{\chi}%
%EndExpansion
_{r}
\end{equation*}%
\begin{equation*}
P_{rr\left( i\right) }=P_{rr}+\frac{P_{ri}^{2}}{1-P_{ii}}
\end{equation*}%
take away ith the leverage of rr position (rr can be different)\newline
$\Longrightarrow P_{rr\left( i\right) }$ may be large if either $P_{rr}$
large or $P_{ii}$ large and/or $P_{ri}$ large\newline
For example: $P_{22}=0.4$, $P_{22\left( 1\right) }=0.8$, $\Rightarrow $1 and
2 can be in same class $\rightarrow $ masking effect
\end{itemize}

\bigskip

\bigskip

\paragraph{7.3 Measure based on residuals $e_{i}$ ($\frac{e_{r}}{\protect%
\sigma _{i}}$) different kind of $e_{i}$}

\begin{enumerate}
\item normalized residuals: $\frac{e_{i}}{\sqrt{%
%TCIMACRO{\TeXButton{underaccent_e}{\underaccent{\wtilde}{e}}}%
%BeginExpansion
\underaccent{\wtilde}{e}%
%EndExpansion
^{\dagger }%
%TCIMACRO{\TeXButton{underaccent_e}{\underaccent{\wtilde}{e}}}%
%BeginExpansion
\underaccent{\wtilde}{e}%
%EndExpansion
}}=a_{i}$

\item standardized residuals: $\frac{e_{i}}{\sqrt{\frac{%
%TCIMACRO{\TeXButton{underaccent_e}{\underaccent{\wtilde}{e}}}%
%BeginExpansion
\underaccent{\wtilde}{e}%
%EndExpansion
^{\dagger }%
%TCIMACRO{\TeXButton{underaccent_e}{\underaccent{\wtilde}{e}}}%
%BeginExpansion
\underaccent{\wtilde}{e}%
%EndExpansion
}{n-k}}}=b_{i}$ (MSE)\newline
$b_{i}=\sqrt{n-k}a_{i}$, $r_{i}=\frac{b_{i}}{\sqrt{1-P_{ii}}}=\sqrt{\frac{n-k%
}{1-P_{ii}}}a_{i}$%
\begin{equation*}
%TCIMACRO{\TeXButton{underaccent_Y}{\underaccent{\wtilde}{Y}}}%
%BeginExpansion
\underaccent{\wtilde}{Y}%
%EndExpansion
-\hat{Y}=%
%TCIMACRO{\TeXButton{underaccent_e}{\underaccent{\wtilde}{e}}}%
%BeginExpansion
\underaccent{\wtilde}{e}%
%EndExpansion
=\left( I-P\right) 
%TCIMACRO{\TeXButton{underaccent_Y}{\underaccent{\wtilde}{Y}}}%
%BeginExpansion
\underaccent{\wtilde}{Y}%
%EndExpansion
\end{equation*}%
\begin{equation*}
Var\left( 
%TCIMACRO{\TeXButton{underaccent_e}{\underaccent{\wtilde}{e}}}%
%BeginExpansion
\underaccent{\wtilde}{e}%
%EndExpansion
\right) =\left( I-P\right) \sigma ^{2}=\left[ 
\begin{array}{ccc}
1-P_{11} &  & -P_{ij} \\ 
& \ddots &  \\ 
&  & 1-P_{nn}%
\end{array}%
\right] \sigma ^{2}
\end{equation*}%
If $P_{ij}\approx 0$ and $P_{ii}\approx P_{jj}$, $i\neq j$, then reflects $%
%TCIMACRO{\TeXButton{underaccent_epsilon}{\underaccent{\wtilde}{\epsilon}}}%
%BeginExpansion
\underaccent{\wtilde}{\epsilon}%
%EndExpansion
\sim \left( o,\sigma ^{2}I\right) $ similar, but in general not this case

\item internally studentized: $r_{i}=\frac{e_{i}}{\sqrt{\left(
1-P_{ii}\right) \text{\c{S}}^{2}}}\quad $\c{S}$^{2}=\frac{%
%TCIMACRO{\TeXButton{underaccent_e}{\underaccent{\wtilde}{e}}}%
%BeginExpansion
\underaccent{\wtilde}{e}%
%EndExpansion
^{\dagger }%
%TCIMACRO{\TeXButton{underaccent_e}{\underaccent{\wtilde}{e}}}%
%BeginExpansion
\underaccent{\wtilde}{e}%
%EndExpansion
}{n-k}$

\item externally studentized: $\frac{e_{i}}{\sqrt{\left( 1-P_{ii}\right) 
\text{\c{S}}_{\left( i\right) }^{2}}}$\newline
where \c{S}$_{\left( i\right) }^{2}=\frac{%
%TCIMACRO{\TeXButton{underaccent_Y}{\underaccent{\wtilde}{Y}}}%
%BeginExpansion
\underaccent{\wtilde}{Y}%
%EndExpansion
_{\left( i\right) }^{\dagger }\left[ I-P_{\left( i\right) }\right] 
%TCIMACRO{\TeXButton{underaccent_Y}{\underaccent{\wtilde}{Y}}}%
%BeginExpansion
\underaccent{\wtilde}{Y}%
%EndExpansion
_{\left( i\right) }}{n-k-1}$%
\begin{equation*}
\begin{array}{ccc}
e_{i\left( i\right) } & e_{i} & \text{cook's distance} \\ 
\hat{Y}_{i\left( i\right) }=%
%TCIMACRO{\TeXButton{underaccent_x}{\underaccent{\wtilde}{x}}}%
%BeginExpansion
\underaccent{\wtilde}{x}%
%EndExpansion
_{i}^{\dagger }%
%TCIMACRO{%
%\TeXButton{underaccent_beta_hat}{\underaccent{\wtilde}{\hat{\beta}}}}%
%BeginExpansion
\underaccent{\wtilde}{\hat{\beta}}%
%EndExpansion
_{\left( i\right) } & \hat{\beta} & D_{i}=\left( 
%TCIMACRO{%
%\TeXButton{underaccent_beta_hat}{\underaccent{\wtilde}{\hat{\beta}}}}%
%BeginExpansion
\underaccent{\wtilde}{\hat{\beta}}%
%EndExpansion
_{\left( i\right) }-\hat{\beta}\right) ^{\dagger }x^{\dagger }x\left( 
%TCIMACRO{%
%\TeXButton{underaccent_beta_hat}{\underaccent{\wtilde}{\hat{\beta}}}}%
%BeginExpansion
\underaccent{\wtilde}{\hat{\beta}}%
%EndExpansion
_{\left( i\right) }-\hat{\beta}\right) \\ 
%TCIMACRO{%
%\TeXButton{underaccent_beta_hat}{\underaccent{\wtilde}{\hat{\beta}}}}%
%BeginExpansion
\underaccent{\wtilde}{\hat{\beta}}%
%EndExpansion
_{\left( i\right) }=x_{\left( i\right) }\left[ x_{\left( i\right) }^{\dagger
}x_{\left( i\right) }\right] ^{-1}x_{\left( i\right) }^{\dagger }Y_{\left(
i\right) } &  & =\left( x%
%TCIMACRO{%
%\TeXButton{underaccent_beta_hat}{\underaccent{\wtilde}{\hat{\beta}}}}%
%BeginExpansion
\underaccent{\wtilde}{\hat{\beta}}%
%EndExpansion
_{\left( i\right) }-x\hat{\beta}\right) ^{\dagger }\left( x%
%TCIMACRO{%
%\TeXButton{underaccent_beta_hat}{\underaccent{\wtilde}{\hat{\beta}}}}%
%BeginExpansion
\underaccent{\wtilde}{\hat{\beta}}%
%EndExpansion
_{\left( i\right) }-x\hat{\beta}\right)%
\end{array}%
\end{equation*}%
\begin{eqnarray*}
\hat{\varepsilon}_{i\left( i\right) } &=&\frac{\hat{\varepsilon}_{i}}{%
1-P_{ii}} \\
b-\hat{\beta}_{\left( i\right) } &=&\left( x^{\dagger }x\right) ^{-1}\frac{%
x_{i}\hat{\varepsilon}_{i}}{1-P_{ii}}
\end{eqnarray*}
\end{enumerate}

\bigskip

\bigskip

\paragraph{7.3.3 Outlier for influential observation}

\begin{equation*}
\text{Leverage }%
%TCIMACRO{\TeXButton{underaccent_x}{\underaccent{\wtilde}{x}}}%
%BeginExpansion
\underaccent{\wtilde}{x}%
%EndExpansion
_{i}^{\dagger }\left[ x^{\dagger }x\right] ^{-1}%
%TCIMACRO{\TeXButton{underaccent_x}{\underaccent{\wtilde}{x}}}%
%BeginExpansion
\underaccent{\wtilde}{x}%
%EndExpansion
_{\left( i\right) }=P_{ii}\text{, }0\leq P_{ii}\leq 1
\end{equation*}%
P$_{ii}$ has outlier.

If a subject have the same $%
%TCIMACRO{\TeXButton{underaccent_x}{\underaccent{\wtilde}{x}}}%
%BeginExpansion
\underaccent{\wtilde}{x}%
%EndExpansion
$, then $P_{ii}\leq \frac{1}{a}$, $0\leq P_{ii}\leq \frac{1}{a}$

\bigskip

$P_{rr\left( i\right) }:$ the rth subject of the leverage after taking away
the ith subject.

\begin{equation*}
P_{rr\left( i\right) }=P_{rr}+\frac{P_{ri}^{2}}{1-P_{ii}}\quad r\neq i
\end{equation*}

\begin{equation*}
Cov\left( 
%TCIMACRO{\TeXButton{underaccent_e}{\underaccent{\wtilde}{e}}}%
%BeginExpansion
\underaccent{\wtilde}{e}%
%EndExpansion
\right) =\left( I-P\right) \sigma ^{2}=\left[ 
\begin{array}{ccc}
1-P_{11} &  & -P_{ij} \\ 
& \ddots &  \\ 
&  & 1-P_{nn}%
\end{array}%
\right] \sigma ^{2}
\end{equation*}

$x\overset{\text{dianose}}{\longleftarrow }P_{ii}$, $Y\overset{\text{dianose}%
}{\longleftarrow }e_{i}$

\begin{eqnarray*}
e_{i\left( i\right) } &=&Y_{i}-%
%TCIMACRO{\TeXButton{underaccent_x}{\underaccent{\wtilde}{x}}}%
%BeginExpansion
\underaccent{\wtilde}{x}%
%EndExpansion
_{i}^{\dagger }\hat{\beta}_{\left( i\right) } \\
&=&y_{i}-%
%TCIMACRO{\TeXButton{underaccent_x}{\underaccent{\wtilde}{x}}}%
%BeginExpansion
\underaccent{\wtilde}{x}%
%EndExpansion
_{i}^{\dagger }\left[ x_{\left( i\right) }^{\dagger }x_{\left( i\right) }%
\right] ^{-1}x_{\left( i\right) }^{\dagger }%
%TCIMACRO{\TeXButton{underaccent_Y}{\underaccent{\wtilde}{Y}}}%
%BeginExpansion
\underaccent{\wtilde}{Y}%
%EndExpansion
_{\left( i\right) } \\
&=&y_{i}-%
%TCIMACRO{\TeXButton{underaccent_x}{\underaccent{\wtilde}{x}}}%
%BeginExpansion
\underaccent{\wtilde}{x}%
%EndExpansion
_{i}^{\dagger }\left( x^{\dagger }x-%
%TCIMACRO{\TeXButton{underaccent_x}{\underaccent{\wtilde}{x}}}%
%BeginExpansion
\underaccent{\wtilde}{x}%
%EndExpansion
_{i}%
%TCIMACRO{\TeXButton{underaccent_x}{\underaccent{\wtilde}{x}}}%
%BeginExpansion
\underaccent{\wtilde}{x}%
%EndExpansion
_{i}^{\dagger }\right) ^{-1}\left( x^{\dagger }Y-%
%TCIMACRO{\TeXButton{underaccent_x}{\underaccent{\wtilde}{x}}}%
%BeginExpansion
\underaccent{\wtilde}{x}%
%EndExpansion
_{i}y_{i}\right) \\
&=&y_{i}-%
%TCIMACRO{\TeXButton{underaccent_x}{\underaccent{\wtilde}{x}}}%
%BeginExpansion
\underaccent{\wtilde}{x}%
%EndExpansion
_{i}^{\dagger }\left[ \left( x^{\dagger }x\right) ^{-1}+\frac{\left(
x^{\dagger }x\right) ^{-1}x_{i}x_{i}^{\dagger }\left( x^{\dagger }x\right)
^{-1}}{1-P_{ii}}\right] \left( x^{\dagger }%
%TCIMACRO{\TeXButton{underaccent_Y}{\underaccent{\wtilde}{Y}}}%
%BeginExpansion
\underaccent{\wtilde}{Y}%
%EndExpansion
-%
%TCIMACRO{\TeXButton{underaccent_x}{\underaccent{\wtilde}{x}}}%
%BeginExpansion
\underaccent{\wtilde}{x}%
%EndExpansion
_{i}y_{i}\right) \\
&=&y_{i}-%
%TCIMACRO{\TeXButton{underaccent_x}{\underaccent{\wtilde}{x}}}%
%BeginExpansion
\underaccent{\wtilde}{x}%
%EndExpansion
_{i}^{\dagger }\left( x^{\dagger }x\right) ^{-1}x^{\dagger }Y+%
%TCIMACRO{\TeXButton{underaccent_x}{\underaccent{\wtilde}{x}}}%
%BeginExpansion
\underaccent{\wtilde}{x}%
%EndExpansion
_{i}^{\dagger }\left( x^{\dagger }x\right) ^{-1}%
%TCIMACRO{\TeXButton{underaccent_x}{\underaccent{\wtilde}{x}}}%
%BeginExpansion
\underaccent{\wtilde}{x}%
%EndExpansion
_{i}y_{i} \\
&&-\frac{%
%TCIMACRO{\TeXButton{underaccent_x}{\underaccent{\wtilde}{x}}}%
%BeginExpansion
\underaccent{\wtilde}{x}%
%EndExpansion
_{i}^{\dagger }\left( x^{\dagger }x\right) ^{-1}%
%TCIMACRO{\TeXButton{underaccent_x}{\underaccent{\wtilde}{x}}}%
%BeginExpansion
\underaccent{\wtilde}{x}%
%EndExpansion
_{i}%
%TCIMACRO{\TeXButton{underaccent_x}{\underaccent{\wtilde}{x}}}%
%BeginExpansion
\underaccent{\wtilde}{x}%
%EndExpansion
_{i}^{\dagger }\left( x^{\dagger }x\right) ^{-1}x^{\dagger }Y}{1-P_{ii}} \\
&&+\frac{%
%TCIMACRO{\TeXButton{underaccent_x}{\underaccent{\wtilde}{x}}}%
%BeginExpansion
\underaccent{\wtilde}{x}%
%EndExpansion
_{i}^{\dagger }\left( x^{\dagger }x\right) ^{-1}x_{i}x_{i}^{\dagger }\left(
x^{\dagger }x\right) ^{-1}%
%TCIMACRO{\TeXButton{underaccent_x}{\underaccent{\wtilde}{x}}}%
%BeginExpansion
\underaccent{\wtilde}{x}%
%EndExpansion
_{i}y_{i}}{1-P_{ii}} \\
&=&y_{i}-\underset{\hat{Y}_{i}}{\underbrace{%
%TCIMACRO{\TeXButton{underaccent_x}{\underaccent{\wtilde}{x}}}%
%BeginExpansion
\underaccent{\wtilde}{x}%
%EndExpansion
_{i}^{\dagger }\hat{\beta}}}+P_{ii}y_{i}-\frac{P_{ii}}{1-P_{ii}}%
%TCIMACRO{\TeXButton{underaccent_x}{\underaccent{\wtilde}{x}}}%
%BeginExpansion
\underaccent{\wtilde}{x}%
%EndExpansion
_{i}^{\dagger }\hat{\beta}+\frac{P_{ii}y_{i}}{1-P_{ii}} \\
&=&y_{i}+P_{ii}y_{i}+\frac{P_{ii}^{2}y_{i}}{1-P_{ii}}-\hat{y}_{i}-\frac{%
P_{ii}}{1-P_{ii}}\hat{y}_{i} \\
&=&\frac{\left( 1-P_{ii}\right) \left( 1+P_{ii}\right) Y_{i}+P_{ii}^{2}Y_{i}%
}{1-P_{ii}}-\frac{\hat{Y}_{i}-P_{ii}\hat{Y}_{i}+P_{ii}\hat{Y}_{i}}{1-P_{ii}}
\\
&=&\frac{Y_{i}-P_{ii}^{2}Y_{i}+P_{ii}^{2}Y_{i}}{1-P_{ii}}-\frac{\hat{Y}_{i}}{%
1-P_{ii}} \\
&=&\frac{1}{1-P_{ii}}\left( y_{i}-\hat{y}_{i}\right) \\
&=&\frac{e_{i}}{1-P_{ii}}
\end{eqnarray*}

\bigskip

\bigskip

\paragraph{7.5 Cook's Distance}

\begin{equation*}
C_{i}=\frac{\left( 
%TCIMACRO{%
%\TeXButton{underaccent_beta_hat}{\underaccent{\wtilde}{\hat{\beta}}}}%
%BeginExpansion
\underaccent{\wtilde}{\hat{\beta}}%
%EndExpansion
-%
%TCIMACRO{%
%\TeXButton{underaccent_beta_hat}{\underaccent{\wtilde}{\hat{\beta}}}}%
%BeginExpansion
\underaccent{\wtilde}{\hat{\beta}}%
%EndExpansion
_{\left( i\right) }\right) ^{\dagger }x^{\dagger }x\left( 
%TCIMACRO{%
%\TeXButton{underaccent_beta_hat}{\underaccent{\wtilde}{\hat{\beta}}}}%
%BeginExpansion
\underaccent{\wtilde}{\hat{\beta}}%
%EndExpansion
-%
%TCIMACRO{%
%\TeXButton{underaccent_beta_hat}{\underaccent{\wtilde}{\hat{\beta}}}}%
%BeginExpansion
\underaccent{\wtilde}{\hat{\beta}}%
%EndExpansion
_{\left( i\right) }\right) }{k\text{\c{S}}^{2}}
\end{equation*}

\bigskip

Note: $\frac{\left( \hat{\beta}-\beta \right) ^{\dagger }\left( \left(
x^{\dagger }x\right) ^{-1}\hat{\sigma}^{2}\right) ^{-1}\left( \hat{\beta}%
-\beta \right) }{k}\sim F_{k,n-k}=\frac{\left( \hat{\beta}-\beta \right)
^{\dagger }x^{\dagger }x\left( \hat{\beta}-\beta \right) }{k\text{\c{S}}^{2}}
$

\bigskip

estimate whether some value have markable effect on estimation of regression
coefficient, normally look for $F_{k,n-k}(0.2)$, $F_{k,n-k}\left( 0.5\right) 
$

\bigskip

\begin{eqnarray*}
\left( \hat{\beta}-\hat{\beta}_{\left( i\right) }\right) &=&\left(
x^{\dagger }x\right) ^{-1}x^{\dagger }Y-\left( x_{\left( i\right) }^{\dagger
}x_{\left( i\right) }\right) ^{-1}x_{\left( i\right) }Y_{\left( i\right) } \\
&=&\left( x^{\dagger }x\right) ^{-1}x^{\dagger }Y-\left[ \left( x^{\dagger
}x\right) ^{-1}+\frac{\left( x^{\dagger }x\right) ^{-1}%
%TCIMACRO{\TeXButton{underaccent_x}{\underaccent{\wtilde}{x}}}%
%BeginExpansion
\underaccent{\wtilde}{x}%
%EndExpansion
_{i}%
%TCIMACRO{\TeXButton{underaccent_x}{\underaccent{\wtilde}{x}}}%
%BeginExpansion
\underaccent{\wtilde}{x}%
%EndExpansion
_{i}^{\dagger }\left( x^{\dagger }x\right) ^{-1}}{1-P_{ii}}\right] \left(
x^{\dagger }%
%TCIMACRO{\TeXButton{underaccent_Y}{\underaccent{\wtilde}{Y}}}%
%BeginExpansion
\underaccent{\wtilde}{Y}%
%EndExpansion
-%
%TCIMACRO{\TeXButton{underaccent_x}{\underaccent{\wtilde}{x}}}%
%BeginExpansion
\underaccent{\wtilde}{x}%
%EndExpansion
_{i}y_{i}\right) \\
&=&\left( x^{\dagger }x\right) ^{-1}%
%TCIMACRO{\TeXButton{underaccent_x}{\underaccent{\wtilde}{x}}}%
%BeginExpansion
\underaccent{\wtilde}{x}%
%EndExpansion
_{i}y_{i}-\frac{\left( x^{\dagger }x\right) ^{-1}%
%TCIMACRO{\TeXButton{underaccent_x}{\underaccent{\wtilde}{x}}}%
%BeginExpansion
\underaccent{\wtilde}{x}%
%EndExpansion
_{i}%
%TCIMACRO{\TeXButton{underaccent_x}{\underaccent{\wtilde}{x}}}%
%BeginExpansion
\underaccent{\wtilde}{x}%
%EndExpansion
_{i}^{\dagger }\left( x^{\dagger }x\right) ^{-1}}{1-P_{ii}}\left( x^{\dagger
}Y-%
%TCIMACRO{\TeXButton{underaccent_x}{\underaccent{\wtilde}{x}}}%
%BeginExpansion
\underaccent{\wtilde}{x}%
%EndExpansion
_{i}y_{i}\right) \\
&=&\left( x^{\dagger }x\right) ^{-1}%
%TCIMACRO{\TeXButton{underaccent_x}{\underaccent{\wtilde}{x}}}%
%BeginExpansion
\underaccent{\wtilde}{x}%
%EndExpansion
_{i}\left[ y_{i}-\frac{%
%TCIMACRO{\TeXButton{underaccent_x}{\underaccent{\wtilde}{x}}}%
%BeginExpansion
\underaccent{\wtilde}{x}%
%EndExpansion
_{i}^{\dagger }\left( x^{\dagger }x\right) ^{-1}}{1-P_{ii}}\left( x^{\dagger
}Y-%
%TCIMACRO{\TeXButton{underaccent_x}{\underaccent{\wtilde}{x}}}%
%BeginExpansion
\underaccent{\wtilde}{x}%
%EndExpansion
_{i}y_{i}\right) \right] \\
&=&\left( x^{\dagger }x\right) ^{-1}%
%TCIMACRO{\TeXButton{underaccent_x}{\underaccent{\wtilde}{x}}}%
%BeginExpansion
\underaccent{\wtilde}{x}%
%EndExpansion
_{i}\left[ \frac{y_{i}-P_{ii}y_{i}-%
%TCIMACRO{\TeXButton{underaccent_x}{\underaccent{\wtilde}{x}}}%
%BeginExpansion
\underaccent{\wtilde}{x}%
%EndExpansion
_{i}^{\dagger }\left( x^{\dagger }x\right) ^{-1}x^{\dagger }Y+%
%TCIMACRO{\TeXButton{underaccent_x}{\underaccent{\wtilde}{x}}}%
%BeginExpansion
\underaccent{\wtilde}{x}%
%EndExpansion
_{i}^{\dagger }\left( x^{\dagger }x\right) ^{-1}%
%TCIMACRO{\TeXButton{underaccent_x}{\underaccent{\wtilde}{x}}}%
%BeginExpansion
\underaccent{\wtilde}{x}%
%EndExpansion
_{i}y_{i}}{1-P_{ii}}\right] \\
&=&\left( x^{\dagger }x\right) ^{-1}%
%TCIMACRO{\TeXButton{underaccent_x}{\underaccent{\wtilde}{x}}}%
%BeginExpansion
\underaccent{\wtilde}{x}%
%EndExpansion
_{i}\frac{y_{i}-%
%TCIMACRO{\TeXButton{underaccent_x}{\underaccent{\wtilde}{x}}}%
%BeginExpansion
\underaccent{\wtilde}{x}%
%EndExpansion
_{i}^{\dagger }%
%TCIMACRO{%
%\TeXButton{underaccent_beta_hat}{\underaccent{\wtilde}{\hat{\beta}}}}%
%BeginExpansion
\underaccent{\wtilde}{\hat{\beta}}%
%EndExpansion
}{1-P_{ii}} \\
&=&\left( x^{\dagger }x\right) ^{-1}%
%TCIMACRO{\TeXButton{underaccent_x}{\underaccent{\wtilde}{x}}}%
%BeginExpansion
\underaccent{\wtilde}{x}%
%EndExpansion
_{i}\frac{e_{i}}{1-P_{ii}}
\end{eqnarray*}

\bigskip

Rewrite%
\begin{eqnarray*}
C_{i} &=&\frac{e_{i}^{2}%
%TCIMACRO{\TeXButton{underaccent_x}{\underaccent{\wtilde}{x}}}%
%BeginExpansion
\underaccent{\wtilde}{x}%
%EndExpansion
_{i}^{\dagger }\left( x^{\dagger }x\right) ^{-1}x^{\dagger }x\left(
x^{\dagger }x\right) ^{-1}%
%TCIMACRO{\TeXButton{underaccent_x}{\underaccent{\wtilde}{x}}}%
%BeginExpansion
\underaccent{\wtilde}{x}%
%EndExpansion
_{i}}{\left( 1-P_{ii}\right) ^{2}k\text{\c{S}}^{2}} \\
&=&\frac{e_{i}^{2}}{\left( 1-P_{ii}\right) ^{2}}\times \frac{P_{ii}}{k\text{%
\c{S}}^{2}} \\
&=&\frac{P_{ii}}{\left( 1-P_{ii}\right) ^{2}}\times \frac{e_{i}^{2}}{k\text{%
\c{S}}^{2}}
\end{eqnarray*}

\bigskip

use $C_{i}$ to estimate outlier. $\frac{P_{ii}}{\left( 1-P_{ii}\right) ^{2}}$
has larger control power among the terms.

\end{document}
