
\documentclass{article}
%%%%%%%%%%%%%%%%%%%%%%%%%%%%%%%%%%%%%%%%%%%%%%%%%%%%%%%%%%%%%%%%%%%%%%%%%%%%%%%%%%%%%%%%%%%%%%%%%%%%%%%%%%%%%%%%%%%%%%%%%%%%%%%%%%%%%%%%%%%%%%%%%%%%%%%%%%%%%%%%%%%%%%%%%%%%%%%%%%%%%%%%%%%%%%%%%%%%%%%%%%%%%%%%%%%%%%%%%%%%%%%%%%%%%%%%%%%%%%%%%%%%%%%%%%%%
\usepackage{amsfonts}
\usepackage{amsmath}
\usepackage{accents}
\usepackage[ignoreall]{geometry}
\usepackage{fancyhdr}

\setcounter{MaxMatrixCols}{10}
%TCIDATA{OutputFilter=LATEX.DLL}
%TCIDATA{Version=5.00.0.2606}
%TCIDATA{<META NAME="SaveForMode" CONTENT="1">}
%TCIDATA{BibliographyScheme=Manual}
%TCIDATA{Created=Wednesday, November 25, 2015 15:33:37}
%TCIDATA{LastRevised=Saturday, November 28, 2015 11:01:14}
%TCIDATA{<META NAME="GraphicsSave" CONTENT="32">}
%TCIDATA{<META NAME="DocumentShell" CONTENT="Standard LaTeX\Blank - Standard LaTeX Article">}
%TCIDATA{CSTFile=40 LaTeX article.cst}

\newtheorem{theorem}{Theorem}
\newtheorem{acknowledgement}[theorem]{Acknowledgement}
\newtheorem{algorithm}[theorem]{Algorithm}
\newtheorem{axiom}[theorem]{Axiom}
\newtheorem{case}[theorem]{Case}
\newtheorem{claim}[theorem]{Claim}
\newtheorem{conclusion}[theorem]{Conclusion}
\newtheorem{condition}[theorem]{Condition}
\newtheorem{conjecture}[theorem]{Conjecture}
\newtheorem{corollary}[theorem]{Corollary}
\newtheorem{criterion}[theorem]{Criterion}
\newtheorem{definition}[theorem]{Definition}
\newtheorem{example}[theorem]{Example}
\newtheorem{exercise}[theorem]{Exercise}
\newtheorem{lemma}[theorem]{Lemma}
\newtheorem{notation}[theorem]{Notation}
\newtheorem{problem}[theorem]{Problem}
\newtheorem{proposition}[theorem]{Proposition}
\newtheorem{remark}[theorem]{Remark}
\newtheorem{solution}[theorem]{Solution}
\newtheorem{summary}[theorem]{Summary}
\newenvironment{proof}[1][Proof]{\noindent\textbf{#1.} }{\ \rule{0.5em}{0.5em}}
\input{../tcilatex}
\DeclareMathAccent{\wtilde}{\mathord}{largesymbols}{"65}
\pagestyle{fancy}
\fancyfoot[C]{\thepage}

%\input{tcilatex}

\begin{document}


\section{Minimax Estimation}

\begin{equation}
\underset{\hat{\beta}\in \mathbb{A}}{\min }\,\underset{\beta \in \mathbb{B}}{%
\sup }\quad R\left( \underaccent{\wtilde}{\hat{\beta}}^{\ast },%
\underaccent{\wtilde}{\beta},A\right) =\underset{\beta \in \mathbb{B}}{\sup }%
\left( \underaccent{\wtilde}{\hat{\beta}}^{\ast },\underaccent{\wtilde}{%
\beta},A\right) \Rightarrow \hat{\beta}^{\ast }\text{ is minimax}
\end{equation}

\bigskip

\begin{theorem}
Inequality Restrictions
\end{theorem}

Restriction on $\underaccent{\wtilde}{\beta}$: $A\underaccent{\wtilde}{\beta}%
\leq a\qquad a:$know vector(by priori knowledge)

\begin{equation}
\min \,\text{\c{S}}\left( \underaccent{\wtilde}{\beta}\right) =\left( Y-x%
\underaccent{\wtilde}{\beta}\right) ^{T}\left( Y-x\underaccent{\wtilde}{%
\beta}\right)
\end{equation}

Assume $a_{i}\leq \beta _{i}\leq b_{i}$ ($a_{i}$, $b_{i}$ known)

take%
\begin{equation}
\frac{\left\vert \beta _{i}-\frac{\left( a_{i}+b_{i}\right) }{2}\right\vert 
}{\frac{\left( b_{i}-a_{i}\right) }{2}}\leq 1\qquad \forall \quad i
\end{equation}

We want an ellipsoid $\left( \underaccent{\wtilde}{\beta}-%
\underaccent{\wtilde}{\beta}_{0}\right) ^{^{\prime }}T\left( %
\underaccent{\wtilde}{\beta}-\underaccent{\wtilde}{\beta}_{0}\right) \leq 1$%
, which encloses the and fullness the following condition:

\begin{enumerate}
\item The ellipsoid and cuboid have the same center point%
\begin{equation}
(i.e.)\quad \underaccent{\wtilde}{\beta}_{0}=\frac{1}{2}\left(
a_{1}+b_{1},\cdots ,a_{k}+b_{k}\right) ^{T}
\end{equation}

\item The axes of ellipsoid are parallel the coordiated so%
\begin{equation}
T=diag\left( t_{1},\cdots ,t_{k}\right)
\end{equation}

\item The corner points of the cuboin are on the surface of the ellipsoid%
\begin{equation}
(i.e.)\quad \dsum\limits_{i=1}^{k}\left( \frac{a_{i}-b_{i}}{2}\right)
^{2}t_{i}=1
\end{equation}

\item The dllipsoid has minimum volume%
\begin{eqnarray}
\min \,V_{k} &=&C_{k}\dprod\limits_{i=1}^{k}t_{i}^{\frac{1}{2}} \\
&&C_{k}\text{ dependent on the dimension K}  \notag \\
s.t.\qquad \dsum\limits_{i=1}^{k}\left( \frac{a_{i}-b_{i}}{2}\right)
^{2}t_{i} &=&1
\end{eqnarray}
\end{enumerate}

\bigskip

take \~{V}$_{k}=\dprod\limits_{i=1}^{k}t_{i}^{-1}-\lambda \left(
\dsum\limits_{i=1}^{k}\left( \frac{a_{i}-b_{i}}{2}\right) ^{2}t_{i}-1\right) 
$

\begin{equation}
\left\{ 
\begin{array}{c}
\frac{\partial \tilde{V}_{k}}{\partial t_{j}}=\dprod\limits_{i\neq
j}^{k}t_{j}^{-1}-\lambda \left( \frac{a_{i}-b_{i}}{2}\right) ^{2}=0 \\ 
\frac{\partial \tilde{V}_{k}}{\partial t_{j}}=\dsum\limits_{{}}^{{}}\left( 
\frac{a_{i}-b_{i}}{2}\right) ^{2}t_{i}-1=0%
\end{array}%
\right.
\end{equation}

\begin{equation}
\Rightarrow \left\{ 
\begin{array}{c}
\lambda =-t_{j}^{-2}\dprod\limits_{i\neq j}^{k}t_{i}^{-1}\left( \frac{2}{%
a_{j}-b_{j}}\right) ^{2} \\ 
=-t_{j}^{-1}\dprod\limits_{i=1}^{k}t_{i}^{-1}\left( \frac{2}{a_{j}-b_{j}}%
\right) ^{2}%
\end{array}%
\right.
\end{equation}

\begin{eqnarray}
&\Rightarrow &t_{i}\left[ \frac{a_{i}-b_{i}}{2}\right] ^{2}=t_{j}\left[ 
\frac{a_{j}-b_{j}}{2}\right] ^{2} \\
\dsum\limits_{i=1}^{k}\left( \frac{a_{i}-b_{i}}{2}\right) ^{2}t_{i}
&=&\dsum\limits_{j=1}^{k}\left( \frac{a_{j}-b_{j}}{2}\right) ^{2}t_{j}
\end{eqnarray}

\begin{eqnarray}
&\Rightarrow &\dsum\limits_{i=1}^{k}\left( \frac{a_{i}-b_{i}}{2}\right)
^{2}t_{i}=1 \\
&\Rightarrow &\dsum\limits_{j=1}^{k}\left( \frac{a_{j}-b_{j}}{2}\right)
^{2}t_{j}=1 \\
&\Rightarrow &t_{j}=\frac{4}{k\left( a_{j}-b_{j}\right) ^{2}}\qquad
j=1,2,3,\cdots ,k
\end{eqnarray}

\bigskip

$\left( \underaccent{\wtilde}{\beta}-\underaccent{\wtilde}{\beta}_{0}\right)
^{T^{\prime }}T\left( \underaccent{\wtilde}{\beta}-\underaccent{\wtilde}{%
\beta}_{0}\right) =1$ the optimal which contain the cuboid, has the center
point vetor%
\begin{eqnarray}
\beta _{0}^{T} &=&\frac{1}{2}\left( a_{1}+b_{1},\cdots ,a_{k}+b_{k}\right) \\
T &=&diag\frac{4}{K}\left( \left( b_{1}-a_{1}\right) ^{-2},\cdots ,\left(
b_{k}-a_{k}\right) ^{-2}\right)
\end{eqnarray}

Note:

\begin{enumerate}
\item The ellipsoid has a larger volume than the cuboid

\item The transition to an ellipsoid as a priori information represents a
weakening, but comes with an earlier mathematical handling.
\end{enumerate}

\section{The Minimax Principle}

Consider the \emph{quodratic} risk $R_{1}\left( \underaccent{\wtilde}{\hat{%
\beta}},\underaccent{\wtilde}{\beta},A\right) =E\left[ \left( 
%TCIMACRO{%
%\TeXButton{underaccent_beta_hat}{\underaccent{\wtilde}{\hat{\beta}}}}%
%BeginExpansion
\underaccent{\wtilde}{\hat{\beta}}%
%EndExpansion
-%
%TCIMACRO{\TeXButton{underaccent_beta}{\underaccent{\wtilde}{\beta}}}%
%BeginExpansion
\underaccent{\wtilde}{\beta}%
%EndExpansion
\right) ^{T}A\left( \hat{\beta}-\beta \right) \right] =tr\left[ AM\left( %
\underaccent{\wtilde}{\hat{\beta}},\underaccent{\wtilde}{\beta}\right) %
\right] $

Let B$\left( \beta \right) \subset \mathbb{R}^{k}$ be a convex region of a
priori restriction for $\underaccent{\wtilde}{\beta}$\bigskip

\begin{definition}
An estimator $b^{\ast }\in \left\{ \hat{\beta}\right\} $ is called a \textsf{%
minimax estimator} of $%
%TCIMACRO{\TeXButton{underaccent_beta}{\underaccent{\wtilde}{\beta}}}%
%BeginExpansion
\underaccent{\wtilde}{\beta}%
%EndExpansion
$ if%
\begin{equation}
\underset{\left\{ 
%TCIMACRO{%
%\TeXButton{underaccent_beta_hat}{\underaccent{\wtilde}{\hat{\beta}}}}%
%BeginExpansion
\underaccent{\wtilde}{\hat{\beta}}%
%EndExpansion
\right\} }{\min }\,\underset{\beta \in \mathbb{B}}{\sup }\quad R_{1}\left( %
\underaccent{\wtilde}{\hat{\beta}},\underaccent{\wtilde}{\beta},A\right) =%
\underset{\beta \in \mathbb{B}}{\sup }R_{1}\left( b^{\ast },%
\underaccent{\wtilde}{\beta},A\right)
\end{equation}
\end{definition}

\bigskip

\subsubsection{Linear Minimax Estimator}

Consider $\hat{\beta}=C%
%TCIMACRO{\TeXButton{underaccent_Y}{\underaccent{\wtilde}{Y}}}%
%BeginExpansion
\underaccent{\wtilde}{Y}%
%EndExpansion
$

\begin{eqnarray}
R_{1}\left( CY,\beta ,A\right)  &=&E\left\{ \left( CY-B\right) ^{T}A\left(
CY-B\right) \right\}   \notag \\
&=&E\left\{ \left( \underbrace{CY-C\chi \beta }+\underbrace{C\chi \beta -B}%
\right) ^{T}A\left( \underbrace{CY-C\chi \beta }+\underbrace{C\chi \beta -B}%
\right) \right\}   \notag \\
&=&\left( C\chi 
%TCIMACRO{\TeXButton{underaccent_beta}{\underaccent{\wtilde}{\beta}}}%
%BeginExpansion
\underaccent{\wtilde}{\beta}%
%EndExpansion
-%
%TCIMACRO{\TeXButton{underaccent_beta}{\underaccent{\wtilde}{\beta}}}%
%BeginExpansion
\underaccent{\wtilde}{\beta}%
%EndExpansion
\right) ^{T}A\left( C\chi 
%TCIMACRO{\TeXButton{underaccent_beta}{\underaccent{\wtilde}{\beta}}}%
%BeginExpansion
\underaccent{\wtilde}{\beta}%
%EndExpansion
-%
%TCIMACRO{\TeXButton{underaccent_beta}{\underaccent{\wtilde}{\beta}}}%
%BeginExpansion
\underaccent{\wtilde}{\beta}%
%EndExpansion
\right)   \notag \\
&&+E\left\{ \left( CY-C\chi \beta \right) ^{T}A\left( CY-C\chi \beta \right)
\right\}   \notag \\
&=&\beta ^{T}\left( C\chi -I\right) A\left( C\chi -I\right) \beta +E\left\{
C^{T}\underbrace{\left( Y-\chi \beta \right) ^{T}A\left( Y-\chi \beta
\right) }C\right\}   \notag \\
&=&\beta ^{T}\left( Cx-I\right) ^{T}A\left( Cx-I\right) \beta +tr\left\{
ACC^{T}\underbrace{E\left( \hat{\beta}-\beta \right) \left( \hat{\beta}%
-\beta \right) ^{T}}\right\}   \notag \\
&=&\underset{1}{\underbrace{\beta ^{T}\left( Cx-I\right) ^{T}A\left(
Cx-I\right) \beta }}+tr\left\{ ACC^{T}\sigma ^{2}I\right\} 
\end{eqnarray}

\begin{eqnarray*}
1 &=&\beta ^{T}T^{\frac{1}{2}}\underset{\tilde{A}}{\underbrace{\left[ T^{-%
\frac{1}{2}}\left( Cx-I\right) ^{T}A\left( Cx-I\right) T^{-\frac{1}{2}}%
\right] }}T^{\frac{1}{2}}\beta  \\
&=&%
%TCIMACRO{\TeXButton{underaccent_beta}{\underaccent{\wtilde}{\beta}}}%
%BeginExpansion
\underaccent{\wtilde}{\beta}%
%EndExpansion
^{T}T^{\frac{1}{2}}\tilde{A}T^{\frac{1}{2}}%
%TCIMACRO{\TeXButton{underaccent_beta}{\underaccent{\wtilde}{\beta}}}%
%BeginExpansion
\underaccent{\wtilde}{\beta}%
%EndExpansion
\end{eqnarray*}%
hence $R_{1}\left( CY,\beta ,A\right) =%
%TCIMACRO{\TeXButton{underaccent_beta}{\underaccent{\wtilde}{\beta}}}%
%BeginExpansion
\underaccent{\wtilde}{\beta}%
%EndExpansion
^{T}T^{\frac{1}{2}}\tilde{A}T^{\frac{1}{2}}%
%TCIMACRO{\TeXButton{underaccent_beta}{\underaccent{\wtilde}{\beta}}}%
%BeginExpansion
\underaccent{\wtilde}{\beta}%
%EndExpansion
+tr\left\{ ACC^{T}\sigma ^{2}I\right\} $

\begin{equation}
\underset{\beta ^{T}T%
%TCIMACRO{\TeXButton{underaccent_beta}{\underaccent{\wtilde}{\beta}}}%
%BeginExpansion
\underaccent{\wtilde}{\beta}%
%EndExpansion
\leq k}{\sup }\qquad R_{1}\left( CI,\beta ,A\right) =tr\left\{ ACC^{T}\sigma
^{2}I\right\} +k\lambda _{\max }\left( \tilde{A}\right) 
\end{equation}%
want to a closed from for C

\bigskip 

Consider $A=aa^{T}$, then $\left( \hat{\beta}-\beta \right) ^{T}%
%TCIMACRO{\TeXButton{underaccent_a}{\underaccent{\wtilde}{a}}}%
%BeginExpansion
\underaccent{\wtilde}{a}%
%EndExpansion
%TCIMACRO{\TeXButton{underaccent_a}{\underaccent{\wtilde}{a}}}%
%BeginExpansion
\underaccent{\wtilde}{a}%
%EndExpansion
^{T}\left( \hat{\beta}-\beta \right) =\left( a^{T}\hat{\beta}-a^{T}\beta
\right) ^{T}\left( a^{T}\hat{\beta}-a^{T}\beta \right) $

for this $A=aa^{T}$, 
\begin{equation}
\tilde{A}=\left[ T^{-\frac{1}{2}}\left( Cx-I\right) ^{T}%
%TCIMACRO{\TeXButton{underaccent_a}{\underaccent{\wtilde}{a}}}%
%BeginExpansion
\underaccent{\wtilde}{a}%
%EndExpansion
\right] \left[ 
%TCIMACRO{\TeXButton{underaccent_a}{\underaccent{\wtilde}{a}}}%
%BeginExpansion
\underaccent{\wtilde}{a}%
%EndExpansion
^{T}\left( Cx-I\right) T^{-\frac{1}{2}}\right] =\tilde{a}\tilde{a}^{T}
\end{equation}

Note:{}

\begin{enumerate}
\item aa$^{T}$ is monnegative $\tilde{a}\tilde{a}^{T}x=\lambda x$, take $x=%
\tilde{a}$, $\lambda =\tilde{a}^{T}\tilde{a}$

\item $rank\left( aa^{T}\right) =1$, meaning eigenvalue = 1 is the only one,
the rest are zeros
\end{enumerate}

\bigskip 

\paragraph{November 24}

\begin{equation*}
R_{1}=E\left( CC%
%TCIMACRO{\TeXButton{underaccent_Y}{\underaccent{\wtilde}{Y}}}%
%BeginExpansion
\underaccent{\wtilde}{Y}%
%EndExpansion
-%
%TCIMACRO{\TeXButton{underaccent_beta}{\underaccent{\wtilde}{\beta}}}%
%BeginExpansion
\underaccent{\wtilde}{\beta}%
%EndExpansion
\right) ^{T}A\left( CC%
%TCIMACRO{\TeXButton{underaccent_Y}{\underaccent{\wtilde}{Y}}}%
%BeginExpansion
\underaccent{\wtilde}{Y}%
%EndExpansion
-%
%TCIMACRO{\TeXButton{underaccent_beta}{\underaccent{\wtilde}{\beta}}}%
%BeginExpansion
\underaccent{\wtilde}{\beta}%
%EndExpansion
\right) \qquad \beta ^{T}T\beta \leq k\qquad \text{no closed form}
\end{equation*}%
\bigskip 

\part{The Generalized Linear Regression Model}

\bigskip

\begin{itemize}
\item R$_{1}$-optional estimators

\item R$_{2}$-optional estimators

\item R$_{3}$-optional estimators
\end{itemize}

\section{Aitken estimator (in fact is the weight LSE)}

\end{document}
