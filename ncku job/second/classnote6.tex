
\documentclass{article}
%%%%%%%%%%%%%%%%%%%%%%%%%%%%%%%%%%%%%%%%%%%%%%%%%%%%%%%%%%%%%%%%%%%%%%%%%%%%%%%%%%%%%%%%%%%%%%%%%%%%%%%%%%%%%%%%%%%%%%%%%%%%%%%%%%%%%%%%%%%%%%%%%%%%%%%%%%%%%%%%%%%%%%%%%%%%%%%%%%%%%%%%%%%%%%%%%%%%%%%%%%%%%%%%%%%%%%%%%%%%%%%%%%%%%%%%%%%%%%%%%%%%%%%%%%%%
\usepackage{amssymb}
\usepackage{amsfonts}
\usepackage{amsmath}
\usepackage{accents}
\usepackage[ignoreall,a4paper]{geometry}
\usepackage{fancyhdr}

\setcounter{MaxMatrixCols}{10}
%TCIDATA{OutputFilter=LATEX.DLL}
%TCIDATA{Version=5.00.0.2606}
%TCIDATA{<META NAME="SaveForMode" CONTENT="1">}
%TCIDATA{BibliographyScheme=Manual}
%TCIDATA{Created=Wednesday, November 25, 2015 15:33:37}
%TCIDATA{LastRevised=Wednesday, May 11, 2016 17:01:56}
%TCIDATA{<META NAME="GraphicsSave" CONTENT="32">}
%TCIDATA{<META NAME="DocumentShell" CONTENT="Standard LaTeX\Blank - Standard LaTeX Article">}
%TCIDATA{CSTFile=40 LaTeX article.cst}
%TCIDATA{ComputeDefs=
%$W=\left( 1-\sigma \right) I$
%}


\newtheorem{theorem}{Theorem}
\newtheorem{acknowledgement}[theorem]{Acknowledgement}
\newtheorem{algorithm}[theorem]{Algorithm}
\newtheorem{axiom}[theorem]{Axiom}
\newtheorem{case}[theorem]{Case}
\newtheorem{claim}[theorem]{Claim}
\newtheorem{conclusion}[theorem]{Conclusion}
\newtheorem{condition}[theorem]{Condition}
\newtheorem{conjecture}[theorem]{Conjecture}
\newtheorem{corollary}[theorem]{Corollary}
\newtheorem{criterion}[theorem]{Criterion}
\newtheorem{definition}[theorem]{Definition}
\newtheorem{example}[theorem]{Example}
\newtheorem{exercise}[theorem]{Exercise}
\newtheorem{lemma}[theorem]{Lemma}
\newtheorem{notation}[theorem]{Notation}
\newtheorem{problem}[theorem]{Problem}
\newtheorem{proposition}[theorem]{Proposition}
\newtheorem{remark}[theorem]{Remark}
\newtheorem{solution}[theorem]{Solution}
\newtheorem{summary}[theorem]{Summary}
\newenvironment{proof}[1][Proof]{\noindent\textbf{#1.} }{\ \rule{0.5em}{0.5em}}
\input{../../tcilatex}
\DeclareMathAccent{\wtilde}{\mathord}{largesymbols}{"65}
\pagestyle{fancy}
\fancyfoot[C]{\thepage}

\input{tcilatex}

\begin{document}


\subsubsection{Inference}

\paragraph{Testing models}

\bigskip

Consider the linear model 
\begin{equation*}
\mathbb{Y}=\mathbb{X}\boldsymbol{\beta }+\boldsymbol{\varepsilon \quad }%
\cdots \boldsymbol{\quad }\text{full model}
\end{equation*}%
where $\gamma \left( \mathbb{X}\right) =\gamma \leq p$, $E\left( \boldsymbol{%
\varepsilon }\right) =0$, $Var\left( \boldsymbol{\varepsilon }\right)
=\sigma ^{2}I$

\bigskip

Consider the linear model%
\begin{equation*}
\boldsymbol{y}=\mathbb{W}\boldsymbol{\gamma }+\boldsymbol{\varepsilon \quad }%
\cdots \boldsymbol{\quad }\text{redued model}
\end{equation*}%
where $C\left( \mathbb{W}\right) \subset C\left( \mathbb{X}\right) $, $%
E\left( \boldsymbol{\varepsilon }\right) =0$, $Var\left( \boldsymbol{%
\varepsilon }\right) =\sigma ^{2}I$

\bigskip

note:

\begin{enumerate}
\item the estimation space of R.M. is smaller than in the F.M.

\item The goal is to test whether or not the reduced model is also correct.

\begin{itemize}
\item If R.M. is correct, there is no reason not to use it.

\item Smaller models are easier to interpret.
\end{itemize}

\item Let $P_{\mathbb{X}}$ and $P_{\mathbb{W}}$ denote the ppm onto $C\left( 
\mathbb{X}\right) $ and $C\left( \mathbb{W}\right) $. Because $C\left( 
\mathbb{W}\right) \subset C\left( \mathbb{X}\right) $, we know that $P_{%
\mathbb{X}}-P_{\mathbb{W}}$ is the ppm onto $C\left( P_{\mathbb{X}}-P_{%
\mathbb{W}}\right) =\underset{\maltese }{\underline{C\left( \mathbb{W}%
\right) ^{\perp }C\left( \mathbb{X}\right) }}$
\end{enumerate}

\bigskip

\begin{lemma}
Assume $P_{\mathbb{X}}$ and $P_{\mathbb{W}}$ are projection matries and $%
P_{2}-P_{1}$ is P.D., then

\begin{enumerate}
\item $P_{1}P_{2}=P_{2}P_{1}=P_{2}$

\item $P_{1}-P_{2}$ is a p.m.
\end{enumerate}
\end{lemma}

\bigskip

note:

\begin{enumerate}
\item $P_{1}$ is a p.m onto $C\left( \mathbb{X}\right) $

\item $P_{2}$ is a p.m onto $C\left( \mathbb{W}\right) \subset C\left( 
\mathbb{X}\right) $

\item $P_{1}-P_{2}$ is a p.m onto the orthogonal complement of $\mathbb{W}$
with in $\mathbb{X}$
\end{enumerate}

\bigskip

\frame{figure}

\bigskip

Under F.M., our estimate for $E\left( \boldsymbol{y}\right) =\boldsymbol{%
x\beta }$ is $P_{\mathbb{X}}\mathbb{Y}$

Under R.M., our estimate is $E\left( \boldsymbol{y}\right) =\mathbb{W}%
\boldsymbol{\gamma }$ is $P_{\mathbb{W}}\mathbb{Y}$

\begin{enumerate}
\item If R.M. is correct, then the $P_{\mathbb{X}}$ and $P_{\mathbb{W}}$ are
estimates the same thing. (i.e. $P_{\mathbb{X}}\mathbb{Y}-P_{\mathbb{W}}%
\mathbb{Y=}\left( P_{\mathbb{X}}-P_{\mathbb{W}}\right) \mathbb{Y}$ should be
small.)

\item the size of $\left( P_{\mathbb{X}}-P_{\mathbb{W}}\right) \mathbb{Y}$ is%
\begin{equation*}
\frac{\left[ \left( P_{\mathbb{X}}-P_{\mathbb{W}}\right) \mathbb{Y}\right]
^{\dagger }\left[ \left( P_{\mathbb{X}}-P_{\mathbb{W}}\right) \mathbb{Y}%
\right] }{\gamma \left( P_{\mathbb{X}}-P_{\mathbb{W}}\right) }=\frac{\mathbb{%
Y}^{\dagger }\left( P_{\mathbb{X}}-P_{\mathbb{W}}\right) \mathbb{Y}}{\mathbb{%
\gamma }\left( P_{\mathbb{X}}-P_{\mathbb{W}}\right) }
\end{equation*}
\end{enumerate}

\bigskip

If R.M. is correct, then%
\begin{eqnarray*}
E\left( \frac{\mathbb{Y}^{\dagger }\left( P_{\mathbb{X}}-P_{\mathbb{W}%
}\right) \mathbb{Y}}{\mathbb{\gamma }\left( P_{\mathbb{X}}-P_{\mathbb{W}%
}\right) }\right) &=&\frac{1}{\gamma ^{\ast }}E\left( \mathbb{Y}^{\dagger
}\left( P_{\mathbb{X}}-P_{\mathbb{W}}\right) \mathbb{Y}\right) \\
&=&\frac{1}{\gamma ^{\ast }}\left\{ tr\left[ \left( P_{\mathbb{X}}-P_{%
\mathbb{W}}\right) \sigma ^{2}I\right] +\right. \\
&&\left. \left( W\gamma \right) ^{\dagger }\left( P_{\mathbb{X}}-P_{\mathbb{W%
}}\right) \left( W\gamma \right) \right\} \\
&=&\frac{1}{\gamma ^{\ast }}\left\{ \sigma ^{2}tr\left( P_{\mathbb{X}}-P_{%
\mathbb{W}}\right) +\right. \\
&&\left. \gamma ^{\dagger }W^{\dagger }\left( P_{\mathbb{X}}-P_{\mathbb{W}%
}\right) W\gamma \right\} \\
&=&\frac{1}{\gamma ^{\ast }}\left\{ \sigma ^{2}\gamma \left( P_{\mathbb{X}%
}-P_{\mathbb{W}}\right) \right\} \\
&=&\sigma ^{2}\quad \left( \text{U.E. of }\sigma ^{2}\right)
\end{eqnarray*}

\bigskip

If R.M. is not correct, then%
\begin{eqnarray*}
E\left( \frac{\mathbb{Y}^{\dagger }\left( P_{\mathbb{X}}-P_{\mathbb{W}%
}\right) \mathbb{Y}}{\mathbb{\gamma }^{\ast }}\right) &=&\frac{1}{\gamma
^{\ast }}\left\{ tr\left[ \left( P_{\mathbb{X}}-P_{\mathbb{W}}\right) \sigma
^{2}I\right] \right. \\
&&+\left. \left( \mathbb{X}\beta \right) ^{\dagger }\left( P_{\mathbb{X}}-P_{%
\mathbb{W}}\right) \left( \mathbb{X}\beta \right) \right\} \\
&=&\sigma ^{2}+\left( \mathbb{X}\beta \right) ^{\dagger }\left( P_{\mathbb{X}%
}-P_{\mathbb{W}}\right) \mathbb{X}\beta
\end{eqnarray*}%
\begin{equation*}
MSE=\frac{\mathbb{Y}^{\dagger }\left( I-P_{\mathbb{X}}\right) \mathbb{Y}}{%
\mathbb{\gamma }\left( I-P_{\mathbb{X}}\right) }=\frac{1}{n-\gamma }\mathbb{Y%
}^{\dagger }\left( I-P_{\mathbb{X}}\right) \mathbb{Y}
\end{equation*}%
\begin{equation*}
E\left( MSE\right) =\sigma ^{2}
\end{equation*}

\bigskip

Test Statistic

\bigskip

To test R.M. vs F.M., we use%
\begin{eqnarray*}
F &=&\frac{\mathbb{Y}^{\dagger }\left( P_{\mathbb{X}}-P_{\mathbb{W}}\right) 
\mathbb{Y}/\gamma ^{\ast }}{\mathbb{Y}^{\dagger }\left( I-P_{\mathbb{X}%
}\right) \mathbb{Y}/\left( n-\gamma \right) } \\
&=&\left\{ 
\begin{tabular}{l}
$\simeq 1\quad \text{, R.M. is correct}$ \\ 
$>1\quad \text{, R.M. is not correct}$%
\end{tabular}%
\right.  \\
&=&\frac{\text{MSR}}{\text{MSE}}
\end{eqnarray*}

\end{document}
