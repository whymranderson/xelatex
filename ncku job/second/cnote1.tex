
\documentclass{article}
%%%%%%%%%%%%%%%%%%%%%%%%%%%%%%%%%%%%%%%%%%%%%%%%%%%%%%%%%%%%%%%%%%%%%%%%%%%%%%%%%%%%%%%%%%%%%%%%%%%%%%%%%%%%%%%%%%%%%%%%%%%%%%%%%%%%%%%%%%%%%%%%%%%%%%%%%%%%%%%%%%%%%%%%%%%%%%%%%%%%%%%%%%%%%%%%%%%%%%%%%%%%%%%%%%%%%%%%%%%%%%%%%%%%%%%%%%%%%%%%%%%%%%%%%%%%
\usepackage{amssymb}
\usepackage{amsfonts}
\usepackage{amsmath}
\usepackage{accents}
\usepackage[ignoreall,a4paper]{geometry}
\usepackage{fancyhdr}

\setcounter{MaxMatrixCols}{10}
%TCIDATA{OutputFilter=LATEX.DLL}
%TCIDATA{Version=5.00.0.2606}
%TCIDATA{<META NAME="SaveForMode" CONTENT="1">}
%TCIDATA{BibliographyScheme=Manual}
%TCIDATA{Created=Wednesday, November 25, 2015 15:33:37}
%TCIDATA{LastRevised=Tuesday, May 17, 2016 15:42:17}
%TCIDATA{<META NAME="GraphicsSave" CONTENT="32">}
%TCIDATA{<META NAME="DocumentShell" CONTENT="Standard LaTeX\Blank - Standard LaTeX Article">}
%TCIDATA{CSTFile=40 LaTeX article.cst}
%TCIDATA{ComputeDefs=
%$W=\left( 1-\sigma \right) I$
%}


\newtheorem{theorem}{Theorem}
\newtheorem{acknowledgement}[theorem]{Acknowledgement}
\newtheorem{algorithm}[theorem]{Algorithm}
\newtheorem{axiom}[theorem]{Axiom}
\newtheorem{case}[theorem]{Case}
\newtheorem{claim}[theorem]{Claim}
\newtheorem{conclusion}[theorem]{Conclusion}
\newtheorem{condition}[theorem]{Condition}
\newtheorem{conjecture}[theorem]{Conjecture}
\newtheorem{corollary}[theorem]{Corollary}
\newtheorem{criterion}[theorem]{Criterion}
\newtheorem{definition}[theorem]{Definition}
\newtheorem{example}[theorem]{Example}
\newtheorem{exercise}[theorem]{Exercise}
\newtheorem{lemma}[theorem]{Lemma}
\newtheorem{notation}[theorem]{Notation}
\newtheorem{problem}[theorem]{Problem}
\newtheorem{proposition}[theorem]{Proposition}
\newtheorem{remark}[theorem]{Remark}
\newtheorem{solution}[theorem]{Solution}
\newtheorem{summary}[theorem]{Summary}
\newenvironment{proof}[1][Proof]{\noindent\textbf{#1.} }{\ \rule{0.5em}{0.5em}}
\input{../../tcilatex}
\DeclareMathAccent{\wtilde}{\mathord}{largesymbols}{"65}
\pagestyle{fancy}
\fancyfoot[C]{\thepage}
\input{tcilatex}

\begin{document}


\section{Linear Mixed Models}

\paragraph{The random effects model}

Consider the model%
\begin{equation*}
Y_{ij}=\mu +a_{i}+\varepsilon _{ij}
\end{equation*}%
\begin{eqnarray*}
i &=&1,\cdots ,a \\
j &=&1,\cdots ,n
\end{eqnarray*}%
where $a_{i}\sim \mathcal{N}\left( 0,\sigma _{a}^{2}\right) $ and $%
\varepsilon _{ij}\sim \mathcal{N}\left( 0,\sigma ^{2}\right) $, and the $%
\varepsilon _{ij}$ and $a_{i}$ are $JI+I$ independent random variables.

\begin{example}
Suppose that we are interested in studying the output in numbers of parts
turned out by the workers in a factory. A large number of workers are
available, and we choose $I$ of them at random, asking each to work $J$
different two-hour time periods.
\end{example}

\begin{example}
Suppose that a pharmaceutical company wishes to test a new experimental
drug. The drug is to be applied to patients with particular disease and the
response is a measure of the improvement in their status. A sample of $I=10$
clinics is selected at random from a large population of clinics and, within
each clinic, a random sample of $J=5$ patients is selected.
\end{example}

\bigskip 

The matrix form

\begin{eqnarray*}
Y_{11} &=&\mu +\alpha _{1}+\varepsilon _{11} \\
Y_{12} &=&\mu +\alpha _{1}+\varepsilon _{12} \\
&&\vdots  \\
Y_{1n} &=&\mu +\alpha _{1}+\varepsilon _{1n} \\
&&\vdots  \\
Y_{a1} &=&\mu +\alpha _{I}+\varepsilon _{I1} \\
&&\vdots  \\
Y_{an} &=&\mu +\alpha _{I}+\varepsilon _{In}
\end{eqnarray*}%
\begin{equation*}
\Leftrightarrow \left[ 
\begin{array}{c}
Y_{11} \\ 
\vdots  \\ 
Y_{1n} \\ 
\vdots  \\ 
Y_{I1} \\ 
\vdots  \\ 
Y_{IJ}%
\end{array}%
\right] =\left[ {}\right] 
\end{equation*}

\end{document}
