
\documentclass{article}
%%%%%%%%%%%%%%%%%%%%%%%%%%%%%%%%%%%%%%%%%%%%%%%%%%%%%%%%%%%%%%%%%%%%%%%%%%%%%%%%%%%%%%%%%%%%%%%%%%%%%%%%%%%%%%%%%%%%%%%%%%%%%%%%%%%%%%%%%%%%%%%%%%%%%%%%%%%%%%%%%%%%%%%%%%%%%%%%%%%%%%%%%%%%%%%%%%%%%%%%%%%%%%%%%%%%%%%%%%%%%%%%%%%%%%%%%%%%%%%%%%%%%%%%%%%%
\usepackage{amssymb}
\usepackage{amsfonts}
\usepackage{amsmath}
\usepackage{accents}
\usepackage[ignoreall,a4paper]{geometry}
\usepackage{fancyhdr}

\setcounter{MaxMatrixCols}{10}
%TCIDATA{OutputFilter=LATEX.DLL}
%TCIDATA{Version=5.00.0.2606}
%TCIDATA{<META NAME="SaveForMode" CONTENT="1">}
%TCIDATA{BibliographyScheme=Manual}
%TCIDATA{Created=Wednesday, November 25, 2015 15:33:37}
%TCIDATA{LastRevised=Friday, April 22, 2016 17:28:48}
%TCIDATA{<META NAME="GraphicsSave" CONTENT="32">}
%TCIDATA{<META NAME="DocumentShell" CONTENT="Standard LaTeX\Blank - Standard LaTeX Article">}
%TCIDATA{CSTFile=40 LaTeX article.cst}
%TCIDATA{ComputeDefs=
%$W=\left( 1-\sigma \right) I$
%}


\newtheorem{theorem}{Theorem}
\newtheorem{acknowledgement}[theorem]{Acknowledgement}
\newtheorem{algorithm}[theorem]{Algorithm}
\newtheorem{axiom}[theorem]{Axiom}
\newtheorem{case}[theorem]{Case}
\newtheorem{claim}[theorem]{Claim}
\newtheorem{conclusion}[theorem]{Conclusion}
\newtheorem{condition}[theorem]{Condition}
\newtheorem{conjecture}[theorem]{Conjecture}
\newtheorem{corollary}[theorem]{Corollary}
\newtheorem{criterion}[theorem]{Criterion}
\newtheorem{definition}[theorem]{Definition}
\newtheorem{example}[theorem]{Example}
\newtheorem{exercise}[theorem]{Exercise}
\newtheorem{lemma}[theorem]{Lemma}
\newtheorem{notation}[theorem]{Notation}
\newtheorem{problem}[theorem]{Problem}
\newtheorem{proposition}[theorem]{Proposition}
\newtheorem{remark}[theorem]{Remark}
\newtheorem{solution}[theorem]{Solution}
\newtheorem{summary}[theorem]{Summary}
\newenvironment{proof}[1][Proof]{\noindent\textbf{#1.} }{\ \rule{0.5em}{0.5em}}
\input{../../tcilatex}
\DeclareMathAccent{\wtilde}{\mathord}{largesymbols}{"65}
\pagestyle{fancy}
\fancyfoot[C]{\thepage}


\begin{document}


\setcounter{part}{1} \setcounter{page}{1}

\begin{example}
Consider the model%
\begin{equation*}
\mathbb{Y=X}%
%TCIMACRO{\TeXButton{beta_dstroke}{\beta\mkern-9.6mu \beta}}%
%BeginExpansion
\beta\mkern-9.6mu \beta%
%EndExpansion
+%
%TCIMACRO{\TeXButton{epsilon_dstroke}{\epsilon\mkern-10mu \epsilon}}%
%BeginExpansion
\epsilon\mkern-10mu \epsilon%
%EndExpansion
\end{equation*}%
where $\mathbb{X}$ is $n\times p$ matrix with $\gamma \left( \mathbb{X}%
\right) =\gamma \leq p<n$ and $%
%TCIMACRO{\TeXButton{epsilon_dstroke}{\epsilon\mkern-10mu \epsilon}}%
%BeginExpansion
\epsilon\mkern-10mu \epsilon%
%EndExpansion
\curvearrowright N_{n}\left( 0,\sigma ^{2}I_{n}\right) $. Let $P_{\mathbb{X}%
}=\mathbb{X}\left( \mathbb{X}^{\dagger }\mathbb{X}\right) ^{-1}\mathbb{X}%
^{\dagger }$ denote the p.p.m onto $c\left( \mathbb{X}\right) $. We know
that $\mathbb{Y\sim }N_{n}\left( \mathbb{X}%
%TCIMACRO{\TeXButton{beta_dstroke}{\beta\mkern-9.6mu \beta}}%
%BeginExpansion
\beta\mkern-9.6mu \beta%
%EndExpansion
,\sigma ^{2}I_{n}\right) $. Consider the (uncorrected) partition%
\begin{eqnarray*}
\underset{\text{SST}}{\underbrace{\mathbb{Y}^{\dagger }\mathbb{Y}}} &\mathbb{%
=}&\mathbb{Y}^{\dagger }I\mathbb{Y=Y}^{\dagger }\left( P_{\mathbb{X}}+I-P_{%
\mathbb{X}}\right) \mathbb{Y} \\
&=&\underset{\text{SSR}}{\underbrace{\mathbb{Y}^{\dagger }P_{\mathbb{X}}%
\mathbb{Y}}}\mathbb{+}\underset{\text{SSE}}{\underbrace{\mathbb{Y}^{\dagger
}\left( I-P_{\mathbb{X}}\right) \mathbb{Y}}}
\end{eqnarray*}

\begin{enumerate}
\item 
\begin{equation*}
\frac{\text{SSE}}{\sigma ^{2}}=\frac{\mathbb{Y}^{\dagger }\left( I-P_{%
\mathbb{X}}\right) \mathbb{Y}}{\sigma ^{2}}=\mathbb{Y}^{\dagger }\underset{%
\mathbb{A}}{\underbrace{\sigma ^{-2}\left( I-P_{\mathbb{X}}\right) }}\mathbb{%
Y=Y}^{\dagger }\mathbb{AY}
\end{equation*}%
$\because 
%TCIMACRO{\TeXButton{sum_dstroke}{\tsum\mkern-20mu\tsum}}%
%BeginExpansion
\tsum\mkern-20mu\tsum%
%EndExpansion
=\sigma ^{2}I$ and $\mathbb{A}%
%TCIMACRO{\TeXButton{sum_dstroke}{\tsum\mkern-20mu\tsum}}%
%BeginExpansion
\tsum\mkern-20mu\tsum%
%EndExpansion
=\sigma ^{-2}\left( I-P_{\mathbb{X}}\right) \sigma ^{2}I=I-P_{\mathbb{X}}$
is idempotent and $\gamma \left( \mathbb{A}%
%TCIMACRO{\TeXButton{sum_dstroke}{\tsum\mkern-20mu\tsum}}%
%BeginExpansion
\tsum\mkern-20mu\tsum%
%EndExpansion
\right) =\gamma \left( I-P_{\mathbb{X}}\right) =n-\gamma $%
\begin{eqnarray*}
&\therefore &\frac{\mathbb{Y}^{\dagger }\left( I-P_{\mathbb{X}}\right) 
\mathbb{Y}}{\sigma ^{2}}\backsim \chi _{\underset{\gamma \left( \mathbb{A}%
%TCIMACRO{\TeXButton{sum_dstroke}{\tsum\mkern-20mu\tsum}}%
%BeginExpansion
\tsum\mkern-20mu\tsum%
%EndExpansion
\right) }{\underbrace{n-\gamma }}}^{2}\left( \frac{\left( \mathbb{X}%
%TCIMACRO{\TeXButton{beta_dstroke}{\beta\mkern-9.6mu \beta}}%
%BeginExpansion
\beta\mkern-9.6mu \beta%
%EndExpansion
\right) ^{\dagger }\mathbb{A}\left( \mathbb{X}%
%TCIMACRO{\TeXButton{beta_dstroke}{\beta\mkern-9.6mu \beta}}%
%BeginExpansion
\beta\mkern-9.6mu \beta%
%EndExpansion
\right) }{2}\right)  \\
&=&\chi _{n-\gamma }^{2}\left( 0\right) 
\end{eqnarray*}

\item 
\begin{equation*}
\frac{\text{SSR}}{\sigma ^{2}}=\frac{\mathbb{Y}^{\dagger }P_{\mathbb{X}}%
\mathbb{Y}}{\sigma ^{2}}=\mathbb{Y}^{\dagger }\sigma ^{2}P_{\mathbb{X}}%
\mathbb{Y=Y}^{\dagger }\mathbb{BY}
\end{equation*}%
$\because \mathbb{B}%
%TCIMACRO{\TeXButton{sum_dstroke}{\tsum\mkern-20mu\tsum}}%
%BeginExpansion
\tsum\mkern-20mu\tsum%
%EndExpansion
=\sigma ^{-2}P_{\mathbb{X}}\sigma ^{2}I=P_{\mathbb{X}}$ is idempotent and $%
\gamma \left( \mathbb{B}%
%TCIMACRO{\TeXButton{sum_dstroke}{\tsum\mkern-20mu\tsum}}%
%BeginExpansion
\tsum\mkern-20mu\tsum%
%EndExpansion
\right) =\gamma \left( P_{\mathbb{X}}\right) =\gamma =\gamma \left( \mathbb{X%
}\right) $%
\begin{equation*}
\therefore \frac{\mathbb{Y}^{\dagger }P_{\mathbb{X}}\mathbb{Y}}{\sigma ^{2}}%
\backsim \chi _{\gamma \Leftrightarrow \gamma \left( \mathbb{B}%
%TCIMACRO{\TeXButton{sum_dstroke}{\tsum\mkern-20mu\tsum}}%
%BeginExpansion
\tsum\mkern-20mu\tsum%
%EndExpansion
\right) }^{2}\left( \frac{\left( \mathbb{X}%
%TCIMACRO{\TeXButton{beta_dstroke}{\beta\mkern-9.6mu \beta}}%
%BeginExpansion
\beta\mkern-9.6mu \beta%
%EndExpansion
\right) ^{\dagger }\mathbb{A}\left( \mathbb{X}%
%TCIMACRO{\TeXButton{beta_dstroke}{\beta\mkern-9.6mu \beta}}%
%BeginExpansion
\beta\mkern-9.6mu \beta%
%EndExpansion
\right) }{2}\right) _{\ast }
\end{equation*}
\end{enumerate}
\end{example}

\end{document}
