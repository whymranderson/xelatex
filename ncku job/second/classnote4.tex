
\documentclass{article}
%%%%%%%%%%%%%%%%%%%%%%%%%%%%%%%%%%%%%%%%%%%%%%%%%%%%%%%%%%%%%%%%%%%%%%%%%%%%%%%%%%%%%%%%%%%%%%%%%%%%%%%%%%%%%%%%%%%%%%%%%%%%%%%%%%%%%%%%%%%%%%%%%%%%%%%%%%%%%%%%%%%%%%%%%%%%%%%%%%%%%%%%%%%%%%%%%%%%%%%%%%%%%%%%%%%%%%%%%%%%%%%%%%%%%%%%%%%%%%%%%%%%%%%%%%%%
\usepackage{amssymb}
\usepackage{amsfonts}
\usepackage{amsmath}
\usepackage{accents}
\usepackage[ignoreall,a4paper]{geometry}
\usepackage{fancyhdr}

\setcounter{MaxMatrixCols}{10}
%TCIDATA{OutputFilter=LATEX.DLL}
%TCIDATA{Version=5.00.0.2606}
%TCIDATA{<META NAME="SaveForMode" CONTENT="1">}
%TCIDATA{BibliographyScheme=Manual}
%TCIDATA{Created=Wednesday, November 25, 2015 15:33:37}
%TCIDATA{LastRevised=Monday, April 25, 2016 20:02:03}
%TCIDATA{<META NAME="GraphicsSave" CONTENT="32">}
%TCIDATA{<META NAME="DocumentShell" CONTENT="Standard LaTeX\Blank - Standard LaTeX Article">}
%TCIDATA{CSTFile=40 LaTeX article.cst}
%TCIDATA{ComputeDefs=
%$W=\left( 1-\sigma \right) I$
%}


\newtheorem{theorem}{Theorem}
\newtheorem{acknowledgement}[theorem]{Acknowledgement}
\newtheorem{algorithm}[theorem]{Algorithm}
\newtheorem{axiom}[theorem]{Axiom}
\newtheorem{case}[theorem]{Case}
\newtheorem{claim}[theorem]{Claim}
\newtheorem{conclusion}[theorem]{Conclusion}
\newtheorem{condition}[theorem]{Condition}
\newtheorem{conjecture}[theorem]{Conjecture}
\newtheorem{corollary}[theorem]{Corollary}
\newtheorem{criterion}[theorem]{Criterion}
\newtheorem{definition}[theorem]{Definition}
\newtheorem{example}[theorem]{Example}
\newtheorem{exercise}[theorem]{Exercise}
\newtheorem{lemma}[theorem]{Lemma}
\newtheorem{notation}[theorem]{Notation}
\newtheorem{problem}[theorem]{Problem}
\newtheorem{proposition}[theorem]{Proposition}
\newtheorem{remark}[theorem]{Remark}
\newtheorem{solution}[theorem]{Solution}
\newtheorem{summary}[theorem]{Summary}
\newenvironment{proof}[1][Proof]{\noindent\textbf{#1.} }{\ \rule{0.5em}{0.5em}}
\input{../../tcilatex}
\DeclareMathAccent{\wtilde}{\mathord}{largesymbols}{"65}
\pagestyle{fancy}
\fancyfoot[C]{\thepage}



\begin{document}


\setcounter{part}{1} \setcounter{page}{1}

\begin{example}
Consider the model%
\begin{equation*}
\mathbb{Y=X}%
%TCIMACRO{\TeXButton{beta_dstroke}{\beta\mkern-9.6mu \beta}}%
%BeginExpansion
\beta\mkern-9.6mu \beta%
%EndExpansion
+%
%TCIMACRO{\TeXButton{epsilon_dstroke}{\epsilon\mkern-10mu \epsilon}}%
%BeginExpansion
\epsilon\mkern-10mu \epsilon%
%EndExpansion
\end{equation*}%
where $\mathbb{X}$ is $n\times p$ matrix with $\gamma \left( \mathbb{X}%
\right) =\gamma \leq p<n$ and $%
%TCIMACRO{\TeXButton{epsilon_dstroke}{\epsilon\mkern-10mu \epsilon}}%
%BeginExpansion
\epsilon\mkern-10mu \epsilon%
%EndExpansion
\curvearrowright N_{n}\left( 0,\sigma ^{2}I_{n}\right) $. Let $P_{\mathbb{X}%
}=\mathbb{X}\left( \mathbb{X}^{\dagger }\mathbb{X}\right) ^{-1}\mathbb{X}%
^{\dagger }$ denote the p.p.m onto $c\left( \mathbb{X}\right) $. We know
that $\mathbb{Y\sim }N_{n}\left( \mathbb{X}%
%TCIMACRO{\TeXButton{beta_dstroke}{\beta\mkern-9.6mu \beta}}%
%BeginExpansion
\beta\mkern-9.6mu \beta%
%EndExpansion
,\sigma ^{2}I_{n}\right) $. Consider the (uncorrected) partition%
\begin{eqnarray*}
\underset{\text{SST}}{\underbrace{\mathbb{Y}^{\dagger }\mathbb{Y}}} &\mathbb{%
=}&\mathbb{Y}^{\dagger }I\mathbb{Y=Y}^{\dagger }\left( P_{\mathbb{X}}+I-P_{%
\mathbb{X}}\right) \mathbb{Y} \\
&=&\underset{\text{SSR}}{\underbrace{\mathbb{Y}^{\dagger }P_{\mathbb{X}}%
\mathbb{Y}}}\mathbb{+}\underset{\text{SSE}}{\underbrace{\mathbb{Y}^{\dagger
}\left( I-P_{\mathbb{X}}\right) \mathbb{Y}}}
\end{eqnarray*}

\begin{enumerate}
\item 
\begin{equation*}
\frac{\text{SSE}}{\sigma ^{2}}=\frac{\mathbb{Y}^{\dagger }\left( I-P_{%
\mathbb{X}}\right) \mathbb{Y}}{\sigma ^{2}}=\mathbb{Y}^{\dagger }\underset{%
\mathbb{A}}{\underbrace{\sigma ^{-2}\left( I-P_{\mathbb{X}}\right) }}\mathbb{%
Y=Y}^{\dagger }\mathbb{AY}
\end{equation*}%
$\because 
%TCIMACRO{\TeXButton{sum_dstroke}{\tsum\mkern-20mu\tsum}}%
%BeginExpansion
\tsum\mkern-20mu\tsum%
%EndExpansion
=\sigma ^{2}I$ and $\mathbb{A}%
%TCIMACRO{\TeXButton{sum_dstroke}{\tsum\mkern-20mu\tsum}}%
%BeginExpansion
\tsum\mkern-20mu\tsum%
%EndExpansion
=\sigma ^{-2}\left( I-P_{\mathbb{X}}\right) \sigma ^{2}I=I-P_{\mathbb{X}}$
is idempotent and $\gamma \left( \mathbb{A}%
%TCIMACRO{\TeXButton{sum_dstroke}{\tsum\mkern-20mu\tsum}}%
%BeginExpansion
\tsum\mkern-20mu\tsum%
%EndExpansion
\right) =\gamma \left( I-P_{\mathbb{X}}\right) =n-\gamma $%
\begin{eqnarray*}
&\therefore &\frac{\mathbb{Y}^{\dagger }\left( I-P_{\mathbb{X}}\right) 
\mathbb{Y}}{\sigma ^{2}}\backsim \chi _{\underset{\gamma \left( \mathbb{A}%
%TCIMACRO{\TeXButton{sum_dstroke}{\tsum\mkern-20mu\tsum}}%
%BeginExpansion
\tsum\mkern-20mu\tsum%
%EndExpansion
\right) }{\underbrace{n-\gamma }}}^{2}\left( \frac{\left( \mathbb{X}%
%TCIMACRO{\TeXButton{beta_dstroke}{\beta\mkern-9.6mu \beta}}%
%BeginExpansion
\beta\mkern-9.6mu \beta%
%EndExpansion
\right) ^{\dagger }\mathbb{A}\left( \mathbb{X}%
%TCIMACRO{\TeXButton{beta_dstroke}{\beta\mkern-9.6mu \beta}}%
%BeginExpansion
\beta\mkern-9.6mu \beta%
%EndExpansion
\right) }{2}\right) _{\_\_\frac{\mu ^{\dagger }A\mu }{2}} \\
&=&\chi _{n-\gamma }^{2}\left( 0\right)
\end{eqnarray*}

\item 
\begin{equation*}
\frac{\text{SSR}}{\sigma ^{2}}=\frac{\mathbb{Y}^{\dagger }P_{\mathbb{X}}%
\mathbb{Y}}{\sigma ^{2}}=\mathbb{Y}^{\dagger }\sigma ^{2}P_{\mathbb{X}}%
\mathbb{Y=Y}^{\dagger }\mathbb{BY}
\end{equation*}%
$\because \mathbb{B}%
%TCIMACRO{\TeXButton{sum_dstroke}{\tsum\mkern-20mu\tsum}}%
%BeginExpansion
\tsum\mkern-20mu\tsum%
%EndExpansion
=\sigma ^{-2}P_{\mathbb{X}}\sigma ^{2}I=P_{\mathbb{X}}$ is idempotent and $%
\gamma \left( \mathbb{B}%
%TCIMACRO{\TeXButton{sum_dstroke}{\tsum\mkern-20mu\tsum}}%
%BeginExpansion
\tsum\mkern-20mu\tsum%
%EndExpansion
\right) =\gamma \left( P_{\mathbb{X}}\right) =\gamma =\gamma \left( \mathbb{X%
}\right) $%
\begin{equation*}
\therefore \frac{\mathbb{Y}^{\dagger }P_{\mathbb{X}}\mathbb{Y}}{\sigma ^{2}}%
\backsim \chi _{\gamma \Leftrightarrow \gamma \left( \mathbb{B}%
%TCIMACRO{\TeXButton{sum_dstroke}{\tsum\mkern-20mu\tsum}}%
%BeginExpansion
\tsum\mkern-20mu\tsum%
%EndExpansion
\right) }^{2}\left( \underset{\left( \ast \right) }{\underbrace{\frac{\left( 
\mathbb{X}%
%TCIMACRO{\TeXButton{beta_dstroke}{\beta\mkern-9.6mu \beta}}%
%BeginExpansion
\beta\mkern-9.6mu \beta%
%EndExpansion
\right) ^{\dagger }\mathbb{A}\left( \mathbb{X}%
%TCIMACRO{\TeXButton{beta_dstroke}{\beta\mkern-9.6mu \beta}}%
%BeginExpansion
\beta\mkern-9.6mu \beta%
%EndExpansion
\right) }{2}}}\right)
\end{equation*}%
\begin{eqnarray*}
\left( \ast \right) &\Rightarrow &\left( \mathbb{X}%
%TCIMACRO{\TeXButton{beta_dstroke}{\beta\mkern-9.6mu \beta}}%
%BeginExpansion
\beta\mkern-9.6mu \beta%
%EndExpansion
\right) ^{\dagger }\underset{\mathbb{B}}{\underbrace{\sigma ^{-2}P_{\mathbb{X%
}}}}\left( \mathbb{X}%
%TCIMACRO{\TeXButton{beta_dstroke}{\beta\mkern-9.6mu \beta}}%
%BeginExpansion
\beta\mkern-9.6mu \beta%
%EndExpansion
\right) \\
&=&\sigma ^{-2}%
%TCIMACRO{\TeXButton{beta_dstroke}{\beta\mkern-9.6mu \beta}}%
%BeginExpansion
\beta\mkern-9.6mu \beta%
%EndExpansion
^{\dagger }\mathbb{X}^{\dagger }P_{\mathbb{X}}\mathbb{X}%
%TCIMACRO{\TeXButton{beta_dstroke}{\beta\mkern-9.6mu \beta} }%
%BeginExpansion
\beta\mkern-9.6mu \beta
%EndExpansion
\\
&=&\sigma ^{-2}%
%TCIMACRO{\TeXButton{beta_dstroke}{\beta\mkern-9.6mu \beta}}%
%BeginExpansion
\beta\mkern-9.6mu \beta%
%EndExpansion
^{\dagger }\mathbb{X}^{\dagger }\mathbb{X}%
%TCIMACRO{\TeXButton{beta_dstroke}{\beta\mkern-9.6mu \beta}}%
%BeginExpansion
\beta\mkern-9.6mu \beta%
%EndExpansion
\end{eqnarray*}%
\begin{equation*}
\therefore \frac{\mathbb{Y}^{\dagger }P_{\mathbb{X}}\mathbb{Y}}{\sigma ^{2}}%
\backsim \chi _{\gamma }^{2}\left( \frac{%
%TCIMACRO{\TeXButton{beta_dstroke}{\beta\mkern-9.6mu \beta}}%
%BeginExpansion
\beta\mkern-9.6mu \beta%
%EndExpansion
^{\dagger }\mathbb{X}^{\dagger }\mathbb{X}%
%TCIMACRO{\TeXButton{beta_dstroke}{\beta\mkern-9.6mu \beta}}%
%BeginExpansion
\beta\mkern-9.6mu \beta%
%EndExpansion
}{2\sigma ^{2}}\right)
\end{equation*}
\end{enumerate}
\end{example}

\bigskip

note: $\mathbb{X}%
%TCIMACRO{\TeXButton{beta_dstroke}{\beta\mkern-9.6mu \beta}}%
%BeginExpansion
\beta\mkern-9.6mu \beta%
%EndExpansion
=\boldsymbol{0}\Longleftrightarrow \left( \text{i.e. }%
%TCIMACRO{\TeXButton{beta_dstroke}{\beta\mkern-9.6mu \beta}}%
%BeginExpansion
\beta\mkern-9.6mu \beta%
%EndExpansion
=0\right) $

\begin{enumerate}
\item $\frac{%
%TCIMACRO{\TeXButton{beta_dstroke}{\beta\mkern-9.6mu \beta}}%
%BeginExpansion
\beta\mkern-9.6mu \beta%
%EndExpansion
^{\dagger }\mathbb{X}^{\dagger }\mathbb{X}%
%TCIMACRO{\TeXButton{beta_dstroke}{\beta\mkern-9.6mu \beta}}%
%BeginExpansion
\beta\mkern-9.6mu \beta%
%EndExpansion
}{2\sigma ^{2}}=0$

\item $\frac{\mathbb{Y}^{\dagger }\left( I-P_{\mathbb{X}}\right) \mathbb{Y}}{%
\sigma ^{2}}$ and $\frac{\mathbb{Y}^{\dagger }P_{\mathbb{X}}\mathbb{Y}}{%
\sigma ^{2}}$ have central $\chi ^{2}$ distribution
\end{enumerate}

\bigskip

\paragraph{Indepence of quadratic forms}

\begin{theorem}
Suppose that $\mathbb{Y}\sim N_{p}\left( \mathbb{%
%TCIMACRO{\TeXButton{mu_dstroke}{\mu\mkern-12.3mu\mu}}%
%BeginExpansion
\mu\mkern-12.3mu\mu%
%EndExpansion
},%
%TCIMACRO{\TeXButton{sum_dstroke}{\tsum\mkern-20mu\tsum}}%
%BeginExpansion
\tsum\mkern-20mu\tsum%
%EndExpansion
\right) $. If $\mathbb{B}%
%TCIMACRO{\TeXButton{sum_dstroke}{\tsum\mkern-20mu\tsum}}%
%BeginExpansion
\tsum\mkern-20mu\tsum%
%EndExpansion
\mathbb{A}=\boldsymbol{0}$, then $\mathbb{Y}^{\dagger }A\mathbb{Y}$ and $%
\mathbb{BY}$ are independent.

\begin{proof}

\begin{itemize}
\item[Case 1] We may assume taht $\mathbb{A}$ is symmetric%
\begin{equation*}
\Rightarrow \mathbb{A}=\mathbb{QAQ}^{\dagger }
\end{equation*}%
where $\mathbb{D}=diag\left( \lambda _{1},\cdots ,\lambda _{p}\right) $ and $%
\mathbb{Q}$ is orthogonal.\newline
\newline
We know that $r\leq p$ of the eigenvalues $\lambda _{1},\cdots ,\lambda _{p}$
are nonzero where $\gamma \left( \mathbb{A}\right) =r\leq p$ of the
eigenvalues $\lambda _{1},\cdots ,\lambda _{p}$ are nonzero where $\gamma
\left( \mathbb{A}\right) =\gamma $%
\begin{equation*}
\mathbb{A}=\mathbb{QDQ}^{\dagger }=\left[ 
\begin{array}{cc}
\mathbb{P}_{1} & \mathbb{P}_{2}%
\end{array}%
\right] \left[ 
\begin{array}{cc}
\mathbb{D}_{1} & \boldsymbol{0} \\ 
\boldsymbol{0} & \boldsymbol{0}%
\end{array}%
\right] \left[ 
\begin{array}{c}
\mathbb{P}_{1}^{\dagger } \\ 
\mathbb{P}_{2}^{\dagger }%
\end{array}%
\right] =\mathbb{P}_{1}\mathbb{D}_{1}\mathbb{P}_{1}^{\dagger }
\end{equation*}%
where $\mathbb{D}_{1}=diag\left( \lambda _{1},\cdots ,\lambda _{\gamma
}\right) $%
\begin{equation*}
\therefore \mathbb{Y^{\dagger }AY}=\mathbb{Y^{\dagger }QDQ^{\dagger }Y}=%
\mathbb{Y^{\dagger }\mathbb{P}}_{1}D_{1}\mathbb{P^{\dagger }Y}=\left( 
\mathbb{P}_{1}^{\dagger }\mathbb{Y}\right) ^{\dagger }D_{1}\left( \mathbb{%
P^{\dagger }Y}\right) 
\end{equation*}%
\begin{eqnarray*}
&\because &\left[ 
\begin{array}{c}
\mathbb{BY} \\ 
\mathbb{P}_{1}^{\dagger }\mathbb{Y}%
\end{array}%
\right] =\left[ 
\begin{array}{c}
\mathbb{B} \\ 
\mathbb{P}_{1}^{\dagger }%
\end{array}%
\right] \mathbb{Y} \\
&\sim &N\left( \left[ 
\begin{array}{c}
\mathbb{B}\mu  \\ 
\mathbb{P}_{1}^{\dagger }\mu 
\end{array}%
\right] ,\left[ 
\begin{array}{cc}
\mathbb{B}%
%TCIMACRO{\TeXButton{sum_dstroke}{\tsum\mkern-20mu\tsum}}%
%BeginExpansion
\tsum\mkern-20mu\tsum%
%EndExpansion
\mathbb{B}^{\dagger } & \mathbb{B}%
%TCIMACRO{\TeXButton{sum_dstroke}{\tsum\mkern-20mu\tsum}}%
%BeginExpansion
\tsum\mkern-20mu\tsum%
%EndExpansion
\mathbb{P}_{1} \\ 
\mathbb{P}_{1}^{\dagger }%
%TCIMACRO{\TeXButton{sum_dstroke}{\tsum\mkern-20mu\tsum}}%
%BeginExpansion
\tsum\mkern-20mu\tsum%
%EndExpansion
\mathbb{B}_{1}^{\dagger } & \mathbb{P}_{1}^{\dagger }%
%TCIMACRO{\TeXButton{sum_dstroke}{\tsum\mkern-20mu\tsum}}%
%BeginExpansion
\tsum\mkern-20mu\tsum%
%EndExpansion
\mathbb{P}_{1}%
\end{array}%
\right] \right) 
\end{eqnarray*}%
\newline
\newline
Suppose $\mathbb{B}%
%TCIMACRO{\TeXButton{sum_dstroke}{\tsum\mkern-20mu\tsum}}%
%BeginExpansion
\tsum\mkern-20mu\tsum%
%EndExpansion
\mathbb{A}=\boldsymbol{0}$%
\begin{eqnarray*}
&\Rightarrow &0=\mathbb{B}%
%TCIMACRO{\TeXButton{sum_dstroke}{\tsum\mkern-20mu\tsum}}%
%BeginExpansion
\tsum\mkern-20mu\tsum%
%EndExpansion
\mathbb{A} \\
&=&\mathbb{B}%
%TCIMACRO{\TeXButton{sum_dstroke}{\tsum\mkern-20mu\tsum}}%
%BeginExpansion
\tsum\mkern-20mu\tsum%
%EndExpansion
\mathbb{\mathbb{P}}_{1}\mathbb{D}_{1}\mathbb{P}_{1}^{\dagger } \\
&=&\mathbb{B}%
%TCIMACRO{\TeXButton{sum_dstroke}{\tsum\mkern-20mu\tsum}}%
%BeginExpansion
\tsum\mkern-20mu\tsum%
%EndExpansion
\mathbb{\mathbb{P}}_{1}\mathbb{D}_{1}\underset{I_{r}}{\underbrace{\mathbb{P}%
_{1}^{\dagger }\mathbb{P}}} \\
&=&\mathbb{B}%
%TCIMACRO{\TeXButton{sum_dstroke}{\tsum\mkern-20mu\tsum}}%
%BeginExpansion
\tsum\mkern-20mu\tsum%
%EndExpansion
\mathbb{\mathbb{P}}_{1}\underset{I_{r}}{\underbrace{\mathbb{D}_{1}\mathbb{D}%
_{1}^{-1}}} \\
&=&\mathbb{B}%
%TCIMACRO{\TeXButton{sum_dstroke}{\tsum\mkern-20mu\tsum}}%
%BeginExpansion
\tsum\mkern-20mu\tsum%
%EndExpansion
\mathbb{\mathbb{P}}_{1}
\end{eqnarray*}%
\newline
Thus, $Cov\left( \mathbb{BY},\mathbb{P}_{1}^{\dagger }\mathbb{Y}\right) =%
\mathbb{B}%
%TCIMACRO{\TeXButton{sum_dstroke}{\tsum\mkern-20mu\tsum}}%
%BeginExpansion
\tsum\mkern-20mu\tsum%
%EndExpansion
\mathbb{\mathbb{P}}_{1}=\boldsymbol{0}$\newline
\newline
$\therefore \mathbb{P}^{\dagger }\mathbb{Y}$ and $\mathbb{BY}$ are
independent%
\begin{equation*}
\mathbb{Y^{\dagger }AY}=\left( \mathbb{P}_{1}^{\dagger }\mathbb{Y}\right) 
\mathbb{D}_{1}\left( \mathbb{P}_{1}^{\dagger }\mathbb{Y}\right) \text{ is a
function of }\mathbb{P}_{1}^{\dagger }\mathbb{Y}
\end{equation*}%
\newline
$\therefore \mathbb{Y^{\dagger }AY}$ and $\mathbb{BY}$ are independent%
\newline
\newline

\item[Case 2] We may assume that $\mathbb{A}$ is symmetric and idempotent.%
\begin{equation*}
\therefore \mathbb{Y^{\dagger }AY=Y^{\dagger }A}^{\dagger }\mathbb{AY}%
=\left( \mathbb{AY}\right) \mathbb{^{\dagger }AY}
\end{equation*}%
\newline
\newline
If $\mathbb{B}\tsum \mathbb{A}=\boldsymbol{0}$, we have%
\begin{equation*}
\mathbb{B}\tsum \mathbb{A}=Cov\left( \mathbb{BY},\mathbb{AY}\right) =%
\boldsymbol{0}
\end{equation*}%
$\therefore \mathbb{BY}$ and $\mathbb{AY}$ are indep.\newline
\newline
Thus, $\mathbb{BY}$ and $\underset{\mathbb{Y^{\dagger }AY}}{\underbrace{%
\text{the function of }\mathbb{AY}}}$ are indep.\newline
\newline
$\mathbb{BY\quad \amalg \quad Y^{\dagger }AY}$
\end{itemize}
\end{proof}
\end{theorem}

\end{document}
