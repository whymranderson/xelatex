
\documentclass{article}
%%%%%%%%%%%%%%%%%%%%%%%%%%%%%%%%%%%%%%%%%%%%%%%%%%%%%%%%%%%%%%%%%%%%%%%%%%%%%%%%%%%%%%%%%%%%%%%%%%%%%%%%%%%%%%%%%%%%%%%%%%%%%%%%%%%%%%%%%%%%%%%%%%%%%%%%%%%%%%%%%%%%%%%%%%%%%%%%%%%%%%%%%%%%%%%%%%%%%%%%%%%%%%%%%%%%%%%%%%%%%%%%%%%%%%%%%%%%%%%%%%%%%%%%%%%%
\usepackage{amssymb}
\usepackage{amsfonts}
\usepackage{amsmath}
\usepackage{accents}
\usepackage[ignoreall,a4paper]{geometry}
\usepackage{fancyhdr}

\setcounter{MaxMatrixCols}{10}
%TCIDATA{OutputFilter=LATEX.DLL}
%TCIDATA{Version=5.00.0.2606}
%TCIDATA{<META NAME="SaveForMode" CONTENT="1">}
%TCIDATA{BibliographyScheme=Manual}
%TCIDATA{Created=Wednesday, November 25, 2015 15:33:37}
%TCIDATA{LastRevised=Sunday, May 01, 2016 11:01:06}
%TCIDATA{<META NAME="GraphicsSave" CONTENT="32">}
%TCIDATA{<META NAME="DocumentShell" CONTENT="Standard LaTeX\Blank - Standard LaTeX Article">}
%TCIDATA{CSTFile=40 LaTeX article.cst}
%TCIDATA{ComputeDefs=
%$W=\left( 1-\sigma \right) I$
%}


\newtheorem{theorem}{Theorem}
\newtheorem{acknowledgement}[theorem]{Acknowledgement}
\newtheorem{algorithm}[theorem]{Algorithm}
\newtheorem{axiom}[theorem]{Axiom}
\newtheorem{case}[theorem]{Case}
\newtheorem{claim}[theorem]{Claim}
\newtheorem{conclusion}[theorem]{Conclusion}
\newtheorem{condition}[theorem]{Condition}
\newtheorem{conjecture}[theorem]{Conjecture}
\newtheorem{corollary}[theorem]{Corollary}
\newtheorem{criterion}[theorem]{Criterion}
\newtheorem{definition}[theorem]{Definition}
\newtheorem{example}[theorem]{Example}
\newtheorem{exercise}[theorem]{Exercise}
\newtheorem{lemma}[theorem]{Lemma}
\newtheorem{notation}[theorem]{Notation}
\newtheorem{problem}[theorem]{Problem}
\newtheorem{proposition}[theorem]{Proposition}
\newtheorem{remark}[theorem]{Remark}
\newtheorem{solution}[theorem]{Solution}
\newtheorem{summary}[theorem]{Summary}
\newenvironment{proof}[1][Proof]{\noindent\textbf{#1.} }{\ \rule{0.5em}{0.5em}}
\input{../../tcilatex}
\DeclareMathAccent{\wtilde}{\mathord}{largesymbols}{"65}
\pagestyle{fancy}
\fancyfoot[C]{\thepage}



\begin{document}


\setcounter{part}{1} \setcounter{page}{1}

\begin{example}
Consider the model%
\begin{equation*}
\mathbb{Y=X}%
%TCIMACRO{\TeXButton{beta_dstroke}{\beta\mkern-9.6mu \beta}}%
%BeginExpansion
\beta\mkern-9.6mu \beta%
%EndExpansion
+%
%TCIMACRO{\TeXButton{epsilon_dstroke}{\epsilon\mkern-10mu \epsilon}}%
%BeginExpansion
\epsilon\mkern-10mu \epsilon%
%EndExpansion
\end{equation*}%
where $\mathbb{X}$ is $n\times p$ matrix with $\gamma \left( \mathbb{X}%
\right) =\gamma \leq p<n$ and $%
%TCIMACRO{\TeXButton{epsilon_dstroke}{\epsilon\mkern-10mu \epsilon}}%
%BeginExpansion
\epsilon\mkern-10mu \epsilon%
%EndExpansion
\curvearrowright N_{n}\left( 0,\sigma ^{2}I_{n}\right) $. Let $P_{\mathbb{X}%
}=\mathbb{X}\left( \mathbb{X}^{\dagger }\mathbb{X}\right) ^{-1}\mathbb{X}%
^{\dagger }$ denote the p.p.m onto $c\left( \mathbb{X}\right) $. We know
that $\mathbb{Y\sim }N_{n}\left( \mathbb{X}%
%TCIMACRO{\TeXButton{beta_dstroke}{\beta\mkern-9.6mu \beta}}%
%BeginExpansion
\beta\mkern-9.6mu \beta%
%EndExpansion
,\sigma ^{2}I_{n}\right) $. Consider the (uncorrected) partition%
\begin{eqnarray*}
\underset{\text{SST}}{\underbrace{\mathbb{Y}^{\dagger }\mathbb{Y}}} &\mathbb{%
=}&\mathbb{Y}^{\dagger }I\mathbb{Y=Y}^{\dagger }\left( P_{\mathbb{X}}+I-P_{%
\mathbb{X}}\right) \mathbb{Y} \\
&=&\underset{\text{SSR}}{\underbrace{\mathbb{Y}^{\dagger }P_{\mathbb{X}}%
\mathbb{Y}}}\mathbb{+}\underset{\text{SSE}}{\underbrace{\mathbb{Y}^{\dagger
}\left( I-P_{\mathbb{X}}\right) \mathbb{Y}}}
\end{eqnarray*}

\begin{enumerate}
\item 
\begin{equation*}
\frac{\text{SSE}}{\sigma ^{2}}=\frac{\mathbb{Y}^{\dagger }\left( I-P_{%
\mathbb{X}}\right) \mathbb{Y}}{\sigma ^{2}}=\mathbb{Y}^{\dagger }\underset{%
\mathbb{A}}{\underbrace{\sigma ^{-2}\left( I-P_{\mathbb{X}}\right) }}\mathbb{%
Y=Y}^{\dagger }\mathbb{AY}
\end{equation*}%
$\because 
%TCIMACRO{\TeXButton{sum_dstroke}{\tsum\mkern-20mu\tsum}}%
%BeginExpansion
\tsum\mkern-20mu\tsum%
%EndExpansion
=\sigma ^{2}I$ and $\mathbb{A}%
%TCIMACRO{\TeXButton{sum_dstroke}{\tsum\mkern-20mu\tsum}}%
%BeginExpansion
\tsum\mkern-20mu\tsum%
%EndExpansion
=\sigma ^{-2}\left( I-P_{\mathbb{X}}\right) \sigma ^{2}I=I-P_{\mathbb{X}}$
is idempotent and $\gamma \left( \mathbb{A}%
%TCIMACRO{\TeXButton{sum_dstroke}{\tsum\mkern-20mu\tsum}}%
%BeginExpansion
\tsum\mkern-20mu\tsum%
%EndExpansion
\right) =\gamma \left( I-P_{\mathbb{X}}\right) =n-\gamma $%
\begin{eqnarray*}
&\therefore &\frac{\mathbb{Y}^{\dagger }\left( I-P_{\mathbb{X}}\right) 
\mathbb{Y}}{\sigma ^{2}}\backsim \chi _{\underset{\gamma \left( \mathbb{A}%
\tsum \right) }{\underbrace{n-\gamma }}}^{2}\left( \frac{\left( \mathbb{X}%
%TCIMACRO{\TeXButton{beta_dstroke}{\beta\mkern-9.6mu \beta}}%
%BeginExpansion
\beta\mkern-9.6mu \beta%
%EndExpansion
\right) ^{\dagger }\mathbb{A}\left( \mathbb{X}%
%TCIMACRO{\TeXButton{beta_dstroke}{\beta\mkern-9.6mu \beta}}%
%BeginExpansion
\beta\mkern-9.6mu \beta%
%EndExpansion
\right) }{2}\right) _{\_\_\frac{\mu ^{\dagger }A\mu }{2}} \\
&=&\chi _{n-\gamma }^{2}\left( 0\right) 
\end{eqnarray*}

\item 
\begin{equation*}
\frac{\text{SSR}}{\sigma ^{2}}=\frac{\mathbb{Y}^{\dagger }P_{\mathbb{X}}%
\mathbb{Y}}{\sigma ^{2}}=\mathbb{Y}^{\dagger }\sigma ^{2}P_{\mathbb{X}}%
\mathbb{Y=Y}^{\dagger }\mathbb{BY}
\end{equation*}%
$\because \mathbb{B}%
%TCIMACRO{\TeXButton{sum_dstroke}{\tsum\mkern-20mu\tsum}}%
%BeginExpansion
\tsum\mkern-20mu\tsum%
%EndExpansion
=\sigma ^{-2}P_{\mathbb{X}}\sigma ^{2}I=P_{\mathbb{X}}$ is idempotent and $%
\gamma \left( \mathbb{B}%
%TCIMACRO{\TeXButton{sum_dstroke}{\tsum\mkern-20mu\tsum}}%
%BeginExpansion
\tsum\mkern-20mu\tsum%
%EndExpansion
\right) =\gamma \left( P_{\mathbb{X}}\right) =\gamma =\gamma \left( \mathbb{X%
}\right) $%
\begin{equation*}
\therefore \frac{\mathbb{Y}^{\dagger }P_{\mathbb{X}}\mathbb{Y}}{\sigma ^{2}}%
\backsim \chi _{\gamma \Leftrightarrow \gamma \left( \mathbb{B}\tsum \right)
}^{2}\left( \underset{\left( \ast \right) }{\underbrace{\frac{\left( \mathbb{%
X}%
%TCIMACRO{\TeXButton{beta_dstroke}{\beta\mkern-9.6mu \beta}}%
%BeginExpansion
\beta\mkern-9.6mu \beta%
%EndExpansion
\right) ^{\dagger }\mathbb{A}\left( \mathbb{X}%
%TCIMACRO{\TeXButton{beta_dstroke}{\beta\mkern-9.6mu \beta}}%
%BeginExpansion
\beta\mkern-9.6mu \beta%
%EndExpansion
\right) }{2}}}\right) 
\end{equation*}%
\begin{eqnarray*}
\left( \ast \right)  &\Rightarrow &\left( \mathbb{X}%
%TCIMACRO{\TeXButton{beta_dstroke}{\beta\mkern-9.6mu \beta}}%
%BeginExpansion
\beta\mkern-9.6mu \beta%
%EndExpansion
\right) ^{\dagger }\underset{\mathbb{B}}{\underbrace{\sigma ^{-2}P_{\mathbb{X%
}}}}\left( \mathbb{X}%
%TCIMACRO{\TeXButton{beta_dstroke}{\beta\mkern-9.6mu \beta}}%
%BeginExpansion
\beta\mkern-9.6mu \beta%
%EndExpansion
\right)  \\
&=&\sigma ^{-2}%
%TCIMACRO{\TeXButton{beta_dstroke}{\beta\mkern-9.6mu \beta}}%
%BeginExpansion
\beta\mkern-9.6mu \beta%
%EndExpansion
^{\dagger }\mathbb{X}^{\dagger }P_{\mathbb{X}}\mathbb{X}%
%TCIMACRO{\TeXButton{beta_dstroke}{\beta\mkern-9.6mu \beta} }%
%BeginExpansion
\beta\mkern-9.6mu \beta
%EndExpansion
\\
&=&\sigma ^{-2}%
%TCIMACRO{\TeXButton{beta_dstroke}{\beta\mkern-9.6mu \beta}}%
%BeginExpansion
\beta\mkern-9.6mu \beta%
%EndExpansion
^{\dagger }\mathbb{X}^{\dagger }\mathbb{X}%
%TCIMACRO{\TeXButton{beta_dstroke}{\beta\mkern-9.6mu \beta}}%
%BeginExpansion
\beta\mkern-9.6mu \beta%
%EndExpansion
\end{eqnarray*}%
\begin{equation*}
\therefore \frac{\mathbb{Y}^{\dagger }P_{\mathbb{X}}\mathbb{Y}}{\sigma ^{2}}%
\backsim \chi _{\gamma }^{2}\left( \frac{%
%TCIMACRO{\TeXButton{beta_dstroke}{\beta\mkern-9.6mu \beta}}%
%BeginExpansion
\beta\mkern-9.6mu \beta%
%EndExpansion
^{\dagger }\mathbb{X}^{\dagger }\mathbb{X}%
%TCIMACRO{\TeXButton{beta_dstroke}{\beta\mkern-9.6mu \beta}}%
%BeginExpansion
\beta\mkern-9.6mu \beta%
%EndExpansion
}{2\sigma ^{2}}\right) 
\end{equation*}
\end{enumerate}
\end{example}

\bigskip

note: $\mathbb{X}%
%TCIMACRO{\TeXButton{beta_dstroke}{\beta\mkern-9.6mu \beta}}%
%BeginExpansion
\beta\mkern-9.6mu \beta%
%EndExpansion
=\boldsymbol{0}\Longleftrightarrow \left( \text{i.e. }%
%TCIMACRO{\TeXButton{beta_dstroke}{\beta\mkern-9.6mu \beta}}%
%BeginExpansion
\beta\mkern-9.6mu \beta%
%EndExpansion
=0\right) $

\begin{enumerate}
\item $\frac{%
%TCIMACRO{\TeXButton{beta_dstroke}{\beta\mkern-9.6mu \beta}}%
%BeginExpansion
\beta\mkern-9.6mu \beta%
%EndExpansion
^{\dagger }\mathbb{X}^{\dagger }\mathbb{X}%
%TCIMACRO{\TeXButton{beta_dstroke}{\beta\mkern-9.6mu \beta}}%
%BeginExpansion
\beta\mkern-9.6mu \beta%
%EndExpansion
}{2\sigma ^{2}}=0$

\item $\frac{\mathbb{Y}^{\dagger }\left( I-P_{\mathbb{X}}\right) \mathbb{Y}}{%
\sigma ^{2}}$ and $\frac{\mathbb{Y}^{\dagger }P_{\mathbb{X}}\mathbb{Y}}{%
\sigma ^{2}}$ have central $\chi ^{2}$ distribution
\end{enumerate}

\bigskip

\paragraph{Indepence of quadratic forms}

\begin{theorem}
Suppose that $\mathbb{Y}\sim N_{p}\left( \mathbb{%
%TCIMACRO{\TeXButton{mu_dstroke}{\mu\mkern-12.3mu\mu}}%
%BeginExpansion
\mu\mkern-12.3mu\mu%
%EndExpansion
},%
%TCIMACRO{\TeXButton{sum_dstroke}{\tsum\mkern-20mu\tsum}}%
%BeginExpansion
\tsum\mkern-20mu\tsum%
%EndExpansion
\right) $. If $\mathbb{B}%
%TCIMACRO{\TeXButton{sum_dstroke}{\tsum\mkern-20mu\tsum}}%
%BeginExpansion
\tsum\mkern-20mu\tsum%
%EndExpansion
\mathbb{A}=\boldsymbol{0}$, then $\mathbb{Y}^{\dagger }A\mathbb{Y}$ and $%
\mathbb{BY}$ are independent.

\begin{proof}
\begin{itemize}
\item[Case 1] We may assume taht $\mathbb{A}$ is symmetric%
\begin{equation*}
\Rightarrow \mathbb{A}=\mathbb{QAQ}^{\dagger }
\end{equation*}%
where $\mathbb{D}=diag\left( \lambda _{1},\cdots ,\lambda _{p}\right) $ and $%
\mathbb{Q}$ is orthogonal.\newline
\newline
We know that $r\leq p$ of the eigenvalues $\lambda _{1},\cdots ,\lambda _{p}$
are nonzero where $\gamma \left( \mathbb{A}\right) =r\leq p$ of the
eigenvalues $\lambda _{1},\cdots ,\lambda _{p}$ are nonzero where $\gamma
\left( \mathbb{A}\right) =\gamma $%
\begin{equation*}
\mathbb{A}=\mathbb{QDQ}^{\dagger }=\left[ 
\begin{array}{cc}
\mathbb{P}_{1} & \mathbb{P}_{2}%
\end{array}%
\right] \left[ 
\begin{array}{cc}
\mathbb{D}_{1} & \boldsymbol{0} \\ 
\boldsymbol{0} & \boldsymbol{0}%
\end{array}%
\right] \left[ 
\begin{array}{c}
\mathbb{P}_{1}^{\dagger } \\ 
\mathbb{P}_{2}^{\dagger }%
\end{array}%
\right] =\mathbb{P}_{1}\mathbb{D}_{1}\mathbb{P}_{1}^{\dagger }
\end{equation*}%
where $\mathbb{D}_{1}=diag\left( \lambda _{1},\cdots ,\lambda _{\gamma
}\right) $%
\begin{equation*}
\therefore \mathbb{Y^{\dagger }AY}=\mathbb{Y^{\dagger }QDQ^{\dagger }Y}=%
\mathbb{Y^{\dagger }\mathbb{P}}_{1}D_{1}\mathbb{P^{\dagger }Y}=\left( 
\mathbb{P}_{1}^{\dagger }\mathbb{Y}\right) ^{\dagger }D_{1}\left( \mathbb{%
P^{\dagger }Y}\right)
\end{equation*}%
\begin{eqnarray*}
&\because &\left[ 
\begin{array}{c}
\mathbb{BY} \\ 
\mathbb{P}_{1}^{\dagger }\mathbb{Y}%
\end{array}%
\right] =\left[ 
\begin{array}{c}
\mathbb{B} \\ 
\mathbb{P}_{1}^{\dagger }%
\end{array}%
\right] \mathbb{Y} \\
&\sim &N\left( \left[ 
\begin{array}{c}
\mathbb{B}\mu \\ 
\mathbb{P}_{1}^{\dagger }\mu%
\end{array}%
\right] ,\left[ 
\begin{array}{cc}
\mathbb{B}%
%TCIMACRO{\TeXButton{sum_dstroke}{\tsum\mkern-20mu\tsum}}%
%BeginExpansion
\tsum\mkern-20mu\tsum%
%EndExpansion
\mathbb{B}^{\dagger } & \mathbb{B}%
%TCIMACRO{\TeXButton{sum_dstroke}{\tsum\mkern-20mu\tsum}}%
%BeginExpansion
\tsum\mkern-20mu\tsum%
%EndExpansion
\mathbb{P}_{1} \\ 
\mathbb{P}_{1}^{\dagger }%
%TCIMACRO{\TeXButton{sum_dstroke}{\tsum\mkern-20mu\tsum}}%
%BeginExpansion
\tsum\mkern-20mu\tsum%
%EndExpansion
\mathbb{B}_{1}^{\dagger } & \mathbb{P}_{1}^{\dagger }%
%TCIMACRO{\TeXButton{sum_dstroke}{\tsum\mkern-20mu\tsum}}%
%BeginExpansion
\tsum\mkern-20mu\tsum%
%EndExpansion
\mathbb{P}_{1}%
\end{array}%
\right] \right)
\end{eqnarray*}%
\newline
\newline
Suppose $\mathbb{B}%
%TCIMACRO{\TeXButton{sum_dstroke}{\tsum\mkern-20mu\tsum}}%
%BeginExpansion
\tsum\mkern-20mu\tsum%
%EndExpansion
\mathbb{A}=\boldsymbol{0}$%
\begin{eqnarray*}
&\Rightarrow &0=\mathbb{B}%
%TCIMACRO{\TeXButton{sum_dstroke}{\tsum\mkern-20mu\tsum}}%
%BeginExpansion
\tsum\mkern-20mu\tsum%
%EndExpansion
\mathbb{A} \\
&=&\mathbb{B}%
%TCIMACRO{\TeXButton{sum_dstroke}{\tsum\mkern-20mu\tsum}}%
%BeginExpansion
\tsum\mkern-20mu\tsum%
%EndExpansion
\mathbb{\mathbb{P}}_{1}\mathbb{D}_{1}\mathbb{P}_{1}^{\dagger } \\
&=&\mathbb{B}%
%TCIMACRO{\TeXButton{sum_dstroke}{\tsum\mkern-20mu\tsum}}%
%BeginExpansion
\tsum\mkern-20mu\tsum%
%EndExpansion
\mathbb{\mathbb{P}}_{1}\mathbb{D}_{1}\underset{I_{r}}{\underbrace{\mathbb{P}%
_{1}^{\dagger }\mathbb{P}}} \\
&=&\mathbb{B}%
%TCIMACRO{\TeXButton{sum_dstroke}{\tsum\mkern-20mu\tsum}}%
%BeginExpansion
\tsum\mkern-20mu\tsum%
%EndExpansion
\mathbb{\mathbb{P}}_{1}\underset{I_{r}}{\underbrace{\mathbb{D}_{1}\mathbb{D}%
_{1}^{-1}}} \\
&=&\mathbb{B}%
%TCIMACRO{\TeXButton{sum_dstroke}{\tsum\mkern-20mu\tsum}}%
%BeginExpansion
\tsum\mkern-20mu\tsum%
%EndExpansion
\mathbb{\mathbb{P}}_{1}
\end{eqnarray*}%
\newline
Thus, $Cov\left( \mathbb{BY},\mathbb{P}_{1}^{\dagger }\mathbb{Y}\right) =%
\mathbb{B}%
%TCIMACRO{\TeXButton{sum_dstroke}{\tsum\mkern-20mu\tsum}}%
%BeginExpansion
\tsum\mkern-20mu\tsum%
%EndExpansion
\mathbb{\mathbb{P}}_{1}=\boldsymbol{0}$\newline
\newline
$\therefore \mathbb{P}^{\dagger }\mathbb{Y}$ and $\mathbb{BY}$ are
independent%
\begin{equation*}
\mathbb{Y^{\dagger }AY}=\left( \mathbb{P}_{1}^{\dagger }\mathbb{Y}\right) 
\mathbb{D}_{1}\left( \mathbb{P}_{1}^{\dagger }\mathbb{Y}\right) \text{ is a
function of }\mathbb{P}_{1}^{\dagger }\mathbb{Y}
\end{equation*}%
\newline
$\therefore \mathbb{Y^{\dagger }AY}$ and $\mathbb{BY}$ are independent%
\newline
\newline

\item[Case 2] We may assume that $\mathbb{A}$ is symmetric and idempotent.%
\begin{equation*}
\therefore \mathbb{Y^{\dagger }AY=Y^{\dagger }A}^{\dagger }\mathbb{AY}%
=\left( \mathbb{AY}\right) \mathbb{^{\dagger }AY}
\end{equation*}%
\newline
\newline
If $\mathbb{B}\tsum \mathbb{A}=\boldsymbol{0}$, we have%
\begin{equation*}
\mathbb{B}\tsum \mathbb{A}=Cov\left( \mathbb{BY},\mathbb{AY}\right) =%
\boldsymbol{0}
\end{equation*}%
$\therefore \mathbb{BY}$ and $\mathbb{AY}$ are indep.\newline
\newline
Thus, $\mathbb{BY}$ and $\underset{\mathbb{Y^{\dagger }AY}}{\underbrace{%
\text{the function of }\mathbb{AY}}}$ are indep.\newline
\newline
$\mathbb{BY\quad \amalg \quad Y^{\dagger }AY}$
\end{itemize}
\end{proof}
\end{theorem}

\bigskip

\begin{example}
$\mathbb{Y}\sim N_{n}\left( \mathbb{%
%TCIMACRO{\TeXButton{mu_dstroke}{\mu\mkern-12.3mu\mu}}%
%BeginExpansion
\mu\mkern-12.3mu\mu%
%EndExpansion
}\boldsymbol{1},\sigma ^{2}I_{n}\right) $ $\left( \text{i.e. }Y_{i}\overset{%
\text{iid}}{\sim }N\left( \mu ,\sigma ^{2}\right) \right) $%
\begin{equation*}
\left( n-1\right) S^{2}=\mathbb{Y}^{\dagger }\left( I-\frac{1}{n}J\right) 
\mathbb{Y=Y}^{\dagger }\mathbb{AY}
\end{equation*}%
\begin{equation*}
\bar{Y}=\frac{1}{n}\boldsymbol{1}^{\dagger }\mathbb{Y}=\mathbb{BY}
\end{equation*}%
\begin{equation*}
\mathbb{B}%
%TCIMACRO{\TeXButton{sum_dstroke}{\tsum\mkern-20mu\tsum}}%
%BeginExpansion
\tsum\mkern-20mu\tsum%
%EndExpansion
\mathbb{A}=\frac{1}{n}\boldsymbol{1}^{\dagger }\sigma ^{2}I\left( I-\frac{1}{%
n}J\right) =\frac{\sigma ^{2}}{n}\boldsymbol{1}^{\dagger }\left( I-\frac{1}{n%
}J\right) =\boldsymbol{0}
\end{equation*}%
\begin{eqnarray*}
&\therefore &\left( n-1\right) S^{2}\text{ and }\bar{Y}\text{ are indep} \\
&\Rightarrow &S^{2}\text{ and }\bar{Y}
\end{eqnarray*}
\end{example}

\bigskip

\begin{theorem}
Suppose $\mathbb{Y}\sim N_{n}\left( \mathbb{%
%TCIMACRO{\TeXButton{mu_dstroke}{\mu\mkern-12.3mu\mu}}%
%BeginExpansion
\mu\mkern-12.3mu\mu%
%EndExpansion
},%
%TCIMACRO{\TeXButton{sum_dstroke}{\tsum\mkern-20mu\tsum}}%
%BeginExpansion
\tsum\mkern-20mu\tsum%
%EndExpansion
\right) $. If $\mathbb{B}%
%TCIMACRO{\TeXButton{sum_dstroke}{\tsum\mkern-20mu\tsum}}%
%BeginExpansion
\tsum\mkern-20mu\tsum%
%EndExpansion
\mathbb{A}=0$, then $\underset{\Leftrightarrow P_{\mathbb{X}}}{\underbrace{%
\mathbb{Y}^{\dagger }\mathbb{AY}}}$ and $\underset{\rightarrow \left( I-P_{%
\mathbb{X}}\right) }{\underbrace{\mathbb{Y}^{\dagger }\mathbb{BY}}}$ are
independent.

\begin{proof}

\begin{itemize}
\item[Case 1] (Symmetric)%
\begin{eqnarray*}
\mathbb{A} &=&\mathbb{PDP}^{\dagger }=\mathbb{P}_{1}D_{1}\mathbb{P}%
_{1}^{\dagger } \\
\mathbb{B} &=&\mathbb{QRQ}^{\dagger }=\mathbb{Q}_{1}\mathbb{R}_{1}\mathbb{Q}%
_{1}^{\dagger }
\end{eqnarray*}%
where%
\begin{eqnarray*}
\mathbb{D}_{1} &=&diag\left( \lambda _{1},\cdots ,\lambda _{r}\right) \text{
and }r\left( \mathbb{A}\right) =\gamma \\
\mathbb{R}_{1} &=&diag\left( \lambda _{1},\cdots ,\lambda _{t}\right) \text{
and }r\left( \mathbb{B}\right) =t
\end{eqnarray*}%
$\because \mathbb{P}$ and $\mathbb{Q}$ are orthogonal\newline
\newline
$\Rightarrow \mathbb{P}_{1}^{\dagger }\mathbb{P}_{1}=I_{r}$ and $\mathbb{R}%
_{1}^{\dagger }\mathbb{R}_{1}=I_{t}$\newline
\newline
Suppose that $\mathbb{B}%
%TCIMACRO{\TeXButton{sum_dstroke}{\tsum\mkern-20mu\tsum}}%
%BeginExpansion
\tsum\mkern-20mu\tsum%
%EndExpansion
\mathbb{A}=\boldsymbol{0}$, then%
\begin{eqnarray*}
\boldsymbol{0} &=&\mathbb{B}%
%TCIMACRO{\TeXButton{sum_dstroke}{\tsum\mkern-20mu\tsum}}%
%BeginExpansion
\tsum\mkern-20mu\tsum%
%EndExpansion
\mathbb{A}=Q_{1}\mathbb{R}_{1}Q_{1}^{\dagger }%
%TCIMACRO{\TeXButton{sum_dstroke}{\tsum\mkern-20mu\tsum}}%
%BeginExpansion
\tsum\mkern-20mu\tsum%
%EndExpansion
\mathbb{P}_{1}\mathbb{D}_{1}\mathbb{P}_{1}^{\dagger } \\
&=&Q_{1}^{\dagger }Q_{1}\mathbb{R}_{1}Q^{\dagger }%
%TCIMACRO{\TeXButton{sum_dstroke}{\tsum\mkern-20mu\tsum}}%
%BeginExpansion
\tsum\mkern-20mu\tsum%
%EndExpansion
\mathbb{P}_{1}\mathbb{D}_{1}\mathbb{P}_{1}^{\dagger }\mathbb{P}_{1} \\
&=&\mathbb{R}_{1}Q^{\dagger }%
%TCIMACRO{\TeXButton{sum_dstroke}{\tsum\mkern-20mu\tsum}}%
%BeginExpansion
\tsum\mkern-20mu\tsum%
%EndExpansion
\mathbb{P}_{1}\mathbb{D}_{1} \\
&=&\mathbb{R}_{1}^{-1}\quad \cdots \quad \mathbb{D}_{1}^{-1} \\
&=&Q_{1}^{\dagger }\tsum \mathbb{P}_{1}=Cov\left( Q_{1}^{\dagger }\mathbb{Y},%
\mathbb{P}_{1}^{\dagger }\mathbb{Y}\right) =\boldsymbol{0}
\end{eqnarray*}%
\begin{equation*}
\Rightarrow \left[ 
\begin{array}{c}
\mathbb{P}_{1}^{\dagger } \\ 
\mathbb{Q}_{1}^{\dagger }%
\end{array}%
\right] \mathbb{Y}\sim N\left( \left[ 
\begin{array}{c}
\mathbb{P}_{1}^{\dagger }\mu \\ 
\mathbb{Q}_{1}^{\dagger }\mu%
\end{array}%
\right] ,\left[ 
\begin{array}{cc}
P_{1}^{\dagger }\tsum P_{1} & \boldsymbol{0} \\ 
\boldsymbol{0} & Q_{1}^{\dagger }\tsum Q_{1}%
\end{array}%
\right] \right)
\end{equation*}%
$\therefore \mathbb{P}_{1}^{\dagger }\mathbb{Y}$ and $\mathbb{Q}%
_{1}^{\dagger }\mathbb{Y}$ are jointly normal and uncorrelated\newline
\newline
$\therefore \mathbb{P}_{1}^{\dagger }\mathbb{Y}$ and $\mathbb{Q}%
_{1}^{\dagger }\mathbb{Y}$ are indep $\Rightarrow \mathbb{Y^{\dagger }AY}$
and $\mathbb{Y^{\dagger }BY}$ are indep.\newline
\newline

\item[Case 2] 
\begin{eqnarray*}
\mathbb{Y^{\dagger }AY} &=&\mathbb{Y^{\dagger }A}^{\dagger }\mathbb{AY}=%
\overset{\mathbb{Y}^{\ast }}{\left( \mathbb{AY}\right) \mathbb{^{\dagger }}}%
\overset{\mathbb{Y}^{\ast }}{\left( \mathbb{AY}\right) } \\
\mathbb{Y^{\dagger }BY} &=&\quad \mathbb{\cdots }\quad =\left( \mathbb{BY}%
\right) \mathbb{^{\dagger }}\left( \mathbb{BY}\right)
\end{eqnarray*}%
\begin{equation*}
Cov\left( \mathbb{BY},\mathbb{AY}\right) =\mathbb{B}%
%TCIMACRO{\TeXButton{sum_dstroke}{\tsum\mkern-20mu\tsum}}%
%BeginExpansion
\tsum\mkern-20mu\tsum%
%EndExpansion
\mathbb{A}=\boldsymbol{0}
\end{equation*}%
$\therefore \mathbb{BY}$ and $\mathbb{AY}$ are indep\newline
\newline
$\Rightarrow \overset{\mathbb{Y^{\dagger }BY}}{\left( \mathbb{BY}\right) 
\mathbb{^{\dagger }}\left( \mathbb{BY}\right) }$ and $\overset{\mathbb{%
Y^{\dagger }AY}}{\left( \mathbb{AY}\right) \mathbb{^{\dagger }}\left( 
\mathbb{AY}\right) }$ are indep
\end{itemize}
\end{proof}
\end{theorem}

\bigskip

\begin{example}
Consider $\mathbb{Y=X}%
%TCIMACRO{\TeXButton{beta_dstroke}{\beta\mkern-9.6mu \beta}}%
%BeginExpansion
\beta\mkern-9.6mu \beta%
%EndExpansion
+%
%TCIMACRO{\TeXButton{epsilon_dstroke}{\epsilon\mkern-10mu \epsilon}}%
%BeginExpansion
\epsilon\mkern-10mu \epsilon%
%EndExpansion
$\qquad $%
%TCIMACRO{\TeXButton{epsilon_dstroke}{\epsilon\mkern-10mu \epsilon}}%
%BeginExpansion
\epsilon\mkern-10mu \epsilon%
%EndExpansion
\sim N_{n}\left( \boldsymbol{0},\sigma ^{2}I_{n}\right) $%
\begin{equation*}
P_{\mathbb{X}}=\mathbb{X}\left( \mathbb{X}^{\dagger }\mathbb{X}\right) ^{-1}%
\mathbb{X}^{\dagger }\mathbb{\quad }\left( \text{i.e. }\mathbb{Y}\sim
N_{n}\left( \mathbb{X}%
%TCIMACRO{\TeXButton{beta_dstroke}{\beta\mkern-9.6mu \beta}}%
%BeginExpansion
\beta\mkern-9.6mu \beta%
%EndExpansion
,\sigma ^{2}I_{n}\right) \right)
\end{equation*}%
\begin{equation*}
\frac{\mathbb{Y}^{\dagger }\left( 1-P_{\mathbb{X}}\right) \mathbb{Y}}{\sigma
^{2}}\sim \chi _{n-r}^{2}\qquad \left[ \mathbb{Y}^{\dagger }\mathbb{AY}\sim
\chi _{n-r}^{2}\right]
\end{equation*}%
and%
\begin{equation*}
\frac{\mathbb{Y}^{\dagger }P_{\mathbb{X}}\mathbb{Y}}{\sigma ^{2}}\sim \chi
_{r}^{2}\left( \frac{\left( \mathbb{X}%
%TCIMACRO{\TeXButton{beta_dstroke}{\beta\mkern-9.6mu \beta}}%
%BeginExpansion
\beta\mkern-9.6mu \beta%
%EndExpansion
\right) ^{\dagger }\left( \mathbb{X}%
%TCIMACRO{\TeXButton{beta_dstroke}{\beta\mkern-9.6mu \beta}}%
%BeginExpansion
\beta\mkern-9.6mu \beta%
%EndExpansion
\right) }{2\sigma ^{2}}\right)
\end{equation*}%
\begin{equation*}
\left[ \mathbb{Y}^{\dagger }\mathbb{BY}\sim \chi _{r}^{2}\left( \frac{\left( 
\mathbb{X}%
%TCIMACRO{\TeXButton{beta_dstroke}{\beta\mkern-9.6mu \beta}}%
%BeginExpansion
\beta\mkern-9.6mu \beta%
%EndExpansion
\right) ^{\dagger }\left( \mathbb{X}%
%TCIMACRO{\TeXButton{beta_dstroke}{\beta\mkern-9.6mu \beta}}%
%BeginExpansion
\beta\mkern-9.6mu \beta%
%EndExpansion
\right) }{2\sigma ^{2}}\right) \right]
\end{equation*}%
\newline
\newline
$\therefore \mathbb{B}%
%TCIMACRO{\TeXButton{sum_dstroke}{\tsum\mkern-20mu\tsum}}%
%BeginExpansion
\tsum\mkern-20mu\tsum%
%EndExpansion
\mathbb{A}=\sigma ^{-2}P_{\mathbb{X}}\sigma ^{2}I\left( I-P_{\mathbb{X}%
}\right) \sigma ^{-2}=\boldsymbol{0}$%
\begin{equation*}
\therefore \frac{\mathbb{Y}^{\dagger }\left( 1-P_{\mathbb{X}}\right) \mathbb{%
Y}}{\sigma ^{2}}\text{ and }\frac{\mathbb{Y}^{\dagger }P_{\mathbb{X}}\mathbb{%
Y}}{\sigma ^{2}}\text{ are indep.}
\end{equation*}%
\newline
\newline
\begin{equation*}
F=\frac{\frac{\mathbb{Y}^{\dagger }P_{\mathbb{X}}\mathbb{Y}/\sigma ^{2}}{%
\gamma }}{\frac{\mathbb{Y}^{\dagger }\left( 1-P_{\mathbb{X}}\right) \mathbb{Y%
}/\sigma ^{2}}{n-\gamma }}=\frac{\text{MSR}}{\text{MSE}}\sim F_{r,n-r}\left( 
\frac{\left( \mathbb{X}%
%TCIMACRO{\TeXButton{beta_dstroke}{\beta\mkern-9.6mu \beta}}%
%BeginExpansion
\beta\mkern-9.6mu \beta%
%EndExpansion
\right) ^{\dagger }\left( \mathbb{X}%
%TCIMACRO{\TeXButton{beta_dstroke}{\beta\mkern-9.6mu \beta}}%
%BeginExpansion
\beta\mkern-9.6mu \beta%
%EndExpansion
\right) }{2\sigma ^{2}}\right)
\end{equation*}
\end{example}

\bigskip

\bigskip

\subsubsection{Cochran's Theorem}

\begin{theorem}
Suppose that $\mathbb{Y}\sim N_{n}\left( \mathbb{\mu },\sigma ^{2}I\right) $%
. Suppose that $\mathbb{A}_{1},\cdots ,\mathbb{A}_{k}$ are $n\times n$
symmetric and idempotent matrix where $\gamma \left( \mathbb{A}_{i}\right)
=\gamma _{i}\quad ,\quad i=1,\cdots ,k$.\newline
\newline
If $A_{1}+\ldots +A_{k}=I_{n}$, then%
\begin{eqnarray*}
&&\frac{\mathbb{Y}^{\dagger }A_{1}\mathbb{Y}}{\sigma ^{2}},\cdots ,\frac{%
\mathbb{Y}^{\dagger }A_{k}\mathbb{Y}}{\sigma ^{2}}\text{ follow indep }\chi
_{\gamma _{i}}^{2}\left( \lambda _{i}\right) \text{ distribution where} \\
\lambda _{i} &=&\frac{\mu ^{\dagger }A_{i}\mu }{\sigma ^{2}} \\
i &=&1,\cdots ,k\text{ and }\tsum\limits_{i=1}^{k}\gamma _{i}=n
\end{eqnarray*}%
\newline
\newline
note: If $A_{1}+\ldots +A_{k}=I_{n}$, then $A_{i}$ independent implies that%
\begin{equation*}
A_{i}A_{j}=\boldsymbol{0}\quad \forall \quad i\neq j
\end{equation*}
\end{theorem}

\bigskip

\begin{example}
\begin{eqnarray*}
\mathbb{Y} &\mathbb{=}&\mathbb{X}%
%TCIMACRO{\TeXButton{beta_dstroke}{\beta\mkern-9.6mu \beta}}%
%BeginExpansion
\beta\mkern-9.6mu \beta%
%EndExpansion
+%
%TCIMACRO{\TeXButton{epsilon_dstroke}{\epsilon\mkern-10mu \epsilon} }%
%BeginExpansion
\epsilon\mkern-10mu \epsilon
%EndExpansion
\\
&=&\left[ 
\begin{array}{cccc}
\mathbb{X}_{0} & \mathbb{X}_{1} & \cdots & \mathbb{X}_{k}%
\end{array}%
\right] \left[ 
\begin{array}{c}
%TCIMACRO{\TeXButton{beta_dstroke}{\beta\mkern-9.6mu \beta}}%
%BeginExpansion
\beta\mkern-9.6mu \beta%
%EndExpansion
_{0} \\ 
\vdots \\ 
%TCIMACRO{\TeXButton{beta_dstroke}{\beta\mkern-9.6mu \beta}}%
%BeginExpansion
\beta\mkern-9.6mu \beta%
%EndExpansion
_{k}%
\end{array}%
\right] +%
%TCIMACRO{\TeXButton{epsilon_dstroke}{\epsilon\mkern-10mu \epsilon}}%
%BeginExpansion
\epsilon\mkern-10mu \epsilon%
%EndExpansion
\end{eqnarray*}%
\newline
Now consider fitting each of the $k$ submodels%
\begin{eqnarray*}
\mathbb{Y} &\mathbb{=}&\mathbb{X}_{0}%
%TCIMACRO{\TeXButton{beta_dstroke}{\beta\mkern-9.6mu \beta}}%
%BeginExpansion
\beta\mkern-9.6mu \beta%
%EndExpansion
_{0}+%
%TCIMACRO{\TeXButton{epsilon_dstroke}{\epsilon\mkern-10mu \epsilon} }%
%BeginExpansion
\epsilon\mkern-10mu \epsilon
%EndExpansion
\\
\mathbb{Y} &\mathbb{=}&\mathbb{X}_{0}%
%TCIMACRO{\TeXButton{beta_dstroke}{\beta\mkern-9.6mu \beta}}%
%BeginExpansion
\beta\mkern-9.6mu \beta%
%EndExpansion
_{0}+\mathbb{X}_{1}%
%TCIMACRO{\TeXButton{beta_dstroke}{\beta\mkern-9.6mu \beta}}%
%BeginExpansion
\beta\mkern-9.6mu \beta%
%EndExpansion
_{1}+%
%TCIMACRO{\TeXButton{epsilon_dstroke}{\epsilon\mkern-10mu \epsilon} }%
%BeginExpansion
\epsilon\mkern-10mu \epsilon
%EndExpansion
\\
&&\vdots \\
\mathbb{Y} &\mathbb{=}&\mathbb{X}_{0}%
%TCIMACRO{\TeXButton{beta_dstroke}{\beta\mkern-9.6mu \beta}}%
%BeginExpansion
\beta\mkern-9.6mu \beta%
%EndExpansion
_{0}+\cdots +\mathbb{X}_{k-1}%
%TCIMACRO{\TeXButton{beta_dstroke}{\beta\mkern-9.6mu \beta}}%
%BeginExpansion
\beta\mkern-9.6mu \beta%
%EndExpansion
_{k-1}+%
%TCIMACRO{\TeXButton{epsilon_dstroke}{\epsilon\mkern-10mu \epsilon}}%
%BeginExpansion
\epsilon\mkern-10mu \epsilon%
%EndExpansion
\end{eqnarray*}%
\newline
\newline
Let $R\left( 
%TCIMACRO{\TeXButton{beta_dstroke}{\beta\mkern-9.6mu \beta}}%
%BeginExpansion
\beta\mkern-9.6mu \beta%
%EndExpansion
_{0},\cdots ,%
%TCIMACRO{\TeXButton{beta_dstroke}{\beta\mkern-9.6mu \beta}}%
%BeginExpansion
\beta\mkern-9.6mu \beta%
%EndExpansion
_{i}\right) $ denote the refression (model) sum of squares from fitting the
ith submodels for $i=0,\cdots ,k$%
\begin{eqnarray*}
R\left( 
%TCIMACRO{\TeXButton{beta_dstroke}{\beta\mkern-9.6mu \beta}}%
%BeginExpansion
\beta\mkern-9.6mu \beta%
%EndExpansion
_{0},\cdots ,%
%TCIMACRO{\TeXButton{beta_dstroke}{\beta\mkern-9.6mu \beta}}%
%BeginExpansion
\beta\mkern-9.6mu \beta%
%EndExpansion
_{i}\right) &=&\mathbb{Y}^{\dagger }\left[ \mathbb{X}_{0},\cdots ,\mathbb{X}%
_{i}\right] \left[ \left( \mathbb{X}_{0},\cdots ,\mathbb{X}_{i}\right)
^{\dagger }\left( \mathbb{X}_{0},\cdots ,\mathbb{X}_{i}\right) \right] ^{-1}
\\
&&\left( \mathbb{X}_{0},\cdots ,\mathbb{X}_{i}\right) ^{\dagger }\mathbb{Y}
\\
&=&\mathbb{Y}^{\dagger }P_{\mathbb{X}_{i}^{\ast }}\mathbb{Y}
\end{eqnarray*}%
where $P_{\mathbb{X}_{i}^{\ast }}$ is the p.p.m onto $\left( C\mathbb{X}%
_{i}^{\ast }\right) $, $\mathbb{X}_{i}^{\ast }=\left[ \mathbb{X}_{0},\cdots ,%
\mathbb{X}_{i}\right] $\newline
\newline
note that%
\begin{equation*}
C\left( \mathbb{X}_{0}^{\ast }\right) \subset C\left( \mathbb{X}_{1}^{\ast
}\right) \subset \cdots \subset C\left( \mathbb{X}_{k-1}^{\ast }\right)
\subset C\left( \mathbb{X}\right)
\end{equation*}
\newline
\newline
\begin{eqnarray*}
\mathbb{Y}^{\dagger }\mathbb{Y} &=&\mathbb{Y}^{\dagger }I\mathbb{Y} \\
&=&\mathbb{Y}^{\dagger }\left( P_{\mathbb{X}_{0}^{\ast }}+P_{\mathbb{X}%
_{1}^{\ast }}-P_{\mathbb{X}_{0}^{\ast }}+\cdots +P_{\mathbb{X}}-P_{\mathbb{X}%
_{i-1}^{\ast }}+I-P_{\mathbb{X}}\right) \mathbb{Y} \\
&=&\mathbb{Y}^{\dagger }P_{\mathbb{X}_{0}^{\ast }}\mathbb{Y}+\mathbb{Y}%
^{\dagger }\left( P_{\mathbb{X}_{1}^{\ast }}-P_{\mathbb{X}_{0}^{\ast
}}\right) \mathbb{Y}+\cdots \\
&&+\mathbb{Y}^{\dagger }\left( P_{\mathbb{X}}-P_{\mathbb{X}_{i-1}^{\ast
}}\right) \mathbb{Y}+\mathbb{Y}^{\dagger }\left( I-P_{\mathbb{X}}\right) 
\mathbb{Y} \\
&=&\mathbb{Y}^{\dagger }\mathbb{A}_{0}\mathbb{Y+Y}^{\dagger }\mathbb{A}_{1}%
\mathbb{Y+\cdots +Y}^{\dagger }\mathbb{A}_{k}\mathbb{Y+Y}^{\dagger }\mathbb{A%
}_{k+1}\mathbb{Y}
\end{eqnarray*}%
\newline
\newline
note:

\begin{enumerate}
\item $\mathbb{A}_{0}+\cdots +\mathbb{A}_{k+1}=I_{n}$

\item $\mathbb{A}_{i}$ are symmetric $i=0,\cdots ,k+1$ and\newline
$\mathbb{A}_{i}$ is idempotent for $i=1,\cdots ,k$ (why)
\end{enumerate}
\end{example}

\bigskip

Let $S_{0}=\gamma \left( \mathbb{A}_{0}\right) =\gamma \left( \mathbb{X}%
_{0}\right) $%
\begin{eqnarray*}
S_{i} &=&\gamma \left( A_{i}\right) =tr\left( A_{i}\right) =tr\left( P_{%
\mathbb{X}_{i}^{\ast }}-P_{\mathbb{X}_{i-1}^{\ast }}\right) \\
&=&tr\left( P_{\mathbb{X}_{i}^{\ast }}\right) -tr\left( \mathbb{X}%
_{i-1}^{\ast }\right)
\end{eqnarray*}%
\begin{equation*}
S_{k+1}=\gamma \left( A_{k+1}\right) =n-\gamma \left( \mathbb{X}\right)
=n-\gamma
\end{equation*}

\bigskip

By Cochran's Thm%
\begin{equation*}
\frac{1}{\sigma ^{2}}\mathbb{Y}^{\dagger }\mathbb{A}_{0}\mathbb{Y\sim \chi }%
_{S_{0}}^{2}\left( \frac{\left( \mathbb{X}%
%TCIMACRO{\TeXButton{beta_dstroke}{\beta\mkern-9.6mu \beta}}%
%BeginExpansion
\beta\mkern-9.6mu \beta%
%EndExpansion
\right) ^{\dagger }\mathbb{A}_{0}\left( \mathbb{X}%
%TCIMACRO{\TeXButton{beta_dstroke}{\beta\mkern-9.6mu \beta}}%
%BeginExpansion
\beta\mkern-9.6mu \beta%
%EndExpansion
\right) }{2\sigma ^{2}}\right)
\end{equation*}%
\begin{equation*}
\left. 
\begin{array}{c}
\frac{1}{\sigma ^{2}}\mathbb{Y}^{\dagger }\mathbb{A}_{0}\mathbb{Y\sim \chi }%
_{S_{0}}^{2}\left( \frac{\left( \mathbb{X}%
%TCIMACRO{\TeXButton{beta_dstroke}{\beta\mkern-9.6mu \beta}}%
%BeginExpansion
\beta\mkern-9.6mu \beta%
%EndExpansion
\right) ^{\dagger }\mathbb{A}_{0}\left( \mathbb{X}%
%TCIMACRO{\TeXButton{beta_dstroke}{\beta\mkern-9.6mu \beta}}%
%BeginExpansion
\beta\mkern-9.6mu \beta%
%EndExpansion
\right) }{2\sigma ^{2}}\right) \\ 
\frac{1}{\sigma ^{2}}\mathbb{Y}^{\dagger }\mathbb{A}_{i}\mathbb{Y\sim \chi }%
_{S_{i}}^{2}\left( \frac{\left( \mathbb{X}%
%TCIMACRO{\TeXButton{beta_dstroke}{\beta\mkern-9.6mu \beta}}%
%BeginExpansion
\beta\mkern-9.6mu \beta%
%EndExpansion
\right) ^{\dagger }\mathbb{A}_{i}\left( \mathbb{X}%
%TCIMACRO{\TeXButton{beta_dstroke}{\beta\mkern-9.6mu \beta}}%
%BeginExpansion
\beta\mkern-9.6mu \beta%
%EndExpansion
\right) }{2\sigma ^{2}}\right) \\ 
\frac{1}{\sigma ^{2}}\mathbb{Y}^{\dagger }\mathbb{A}_{k+1}\mathbb{Y\sim \chi 
}_{n-\gamma }^{2}\left( 0\right)%
\end{array}%
\right] \quad \text{indep.}
\end{equation*}

\end{document}
