
\documentclass{article}
%%%%%%%%%%%%%%%%%%%%%%%%%%%%%%%%%%%%%%%%%%%%%%%%%%%%%%%%%%%%%%%%%%%%%%%%%%%%%%%%%%%%%%%%%%%%%%%%%%%%%%%%%%%%%%%%%%%%%%%%%%%%%%%%%%%%%%%%%%%%%%%%%%%%%%%%%%%%%%%%%%%%%%%%%%%%%%%%%%%%%%%%%%%%%%%%%%%%%%%%%%%%%%%%%%%%%%%%%%%%%%%%%%%%%%%%%%%%%%%%%%%%%%%%%%%%
\usepackage{amssymb}
\usepackage{amsfonts}
\usepackage{amsmath}
\usepackage{accents}
\usepackage[ignoreall,a4paper]{geometry}
\usepackage{fancyhdr}

\setcounter{MaxMatrixCols}{10}
%TCIDATA{OutputFilter=LATEX.DLL}
%TCIDATA{Version=5.00.0.2606}
%TCIDATA{<META NAME="SaveForMode" CONTENT="1">}
%TCIDATA{BibliographyScheme=Manual}
%TCIDATA{Created=Wednesday, November 25, 2015 15:33:37}
%TCIDATA{LastRevised=Tuesday, April 19, 2016 10:19:49}
%TCIDATA{<META NAME="GraphicsSave" CONTENT="32">}
%TCIDATA{<META NAME="DocumentShell" CONTENT="Standard LaTeX\Blank - Standard LaTeX Article">}
%TCIDATA{CSTFile=40 LaTeX article.cst}
%TCIDATA{ComputeDefs=
%$W=\left( 1-\sigma \right) I$
%}


\newtheorem{theorem}{Theorem}
\newtheorem{acknowledgement}[theorem]{Acknowledgement}
\newtheorem{algorithm}[theorem]{Algorithm}
\newtheorem{axiom}[theorem]{Axiom}
\newtheorem{case}[theorem]{Case}
\newtheorem{claim}[theorem]{Claim}
\newtheorem{conclusion}[theorem]{Conclusion}
\newtheorem{condition}[theorem]{Condition}
\newtheorem{conjecture}[theorem]{Conjecture}
\newtheorem{corollary}[theorem]{Corollary}
\newtheorem{criterion}[theorem]{Criterion}
\newtheorem{definition}[theorem]{Definition}
\newtheorem{example}[theorem]{Example}
\newtheorem{exercise}[theorem]{Exercise}
\newtheorem{lemma}[theorem]{Lemma}
\newtheorem{notation}[theorem]{Notation}
\newtheorem{problem}[theorem]{Problem}
\newtheorem{proposition}[theorem]{Proposition}
\newtheorem{remark}[theorem]{Remark}
\newtheorem{solution}[theorem]{Solution}
\newtheorem{summary}[theorem]{Summary}
\newenvironment{proof}[1][Proof]{\noindent\textbf{#1.} }{\ \rule{0.5em}{0.5em}}
\input{../../tcilatex}
\DeclareMathAccent{\wtilde}{\mathord}{largesymbols}{"65}
\pagestyle{fancy}
\fancyfoot[C]{\thepage}



\begin{document}


\setcounter{part}{1} \setcounter{page}{1}

Def $\mathbb{A}_{p\times p}=\left[ 
\begin{array}{cc}
I_{r} & \mathbb{O}_{r\times \left( p-r\right) }%
\end{array}%
\right] $, \newline
Then%
\begin{equation*}
\mathbb{Y=AY\sim }N_{r}\left( \mathbb{A%
%TCIMACRO{\TeXButton{mu_dstroke}{\mu\mkern-12.6mu\mu}}%
%BeginExpansion
\mu\mkern-12.6mu\mu%
%EndExpansion
},\mathbb{A}\tsum \mathbb{A}^{\dagger }\right)
\end{equation*}%
where%
\begin{equation*}
\mathbb{A%
%TCIMACRO{\TeXButton{mu_dstroke}{\mu\mkern-12.3mu\mu}}%
%BeginExpansion
\mu\mkern-12.3mu\mu%
%EndExpansion
=}\left[ 
\begin{array}{cc}
I_{r} & \mathbb{O}%
\end{array}%
\right] \left[ 
\begin{array}{c}
I_{r} \\ 
\mathbb{O}%
\end{array}%
\right] =\mathbb{%
%TCIMACRO{\TeXButton{mu_dstroke}{\mu\mkern-12.3mu\mu}}%
%BeginExpansion
\mu\mkern-12.3mu\mu%
%EndExpansion
}_{1}
\end{equation*}%
\begin{eqnarray*}
\mathbb{A%
%TCIMACRO{\TeXButton{sum_dstroke}{\tsum\mkern-20mu\tsum}}%
%BeginExpansion
\tsum\mkern-20mu\tsum%
%EndExpansion
A}^{\dagger } &=&\left[ 
\begin{array}{cc}
I_{r} & \mathbb{O}%
\end{array}%
\right] \mathbb{%
%TCIMACRO{\TeXButton{sum_dstroke}{\tsum\mkern-20mu\tsum}}%
%BeginExpansion
\tsum\mkern-20mu\tsum%
%EndExpansion
}\left[ 
\begin{array}{c}
I_{r} \\ 
\mathbb{O}%
\end{array}%
\right] \\
&=&\left[ 
\begin{array}{cc}
\tsum\nolimits_{11} & \tsum\nolimits_{12}%
\end{array}%
\right] \left[ 
\begin{array}{c}
I_{r} \\ 
\mathbb{O}%
\end{array}%
\right] =\tsum\nolimits_{11}
\end{eqnarray*}%
\begin{equation*}
\therefore \mathbb{Y}_{1}\sim N\left( 
\begin{array}{cc}
\mu _{r} & \tsum\nolimits_{11}%
\end{array}%
\right)
\end{equation*}

\bigskip

note: If $\mathbb{Y}\sim N_{p}\left( 
\begin{array}{cc}
\mathbb{%
%TCIMACRO{\TeXButton{mu_dstroke}{\mu\mkern-12.3mu\mu}}%
%BeginExpansion
\mu\mkern-12.3mu\mu%
%EndExpansion
} & \mathbb{%
%TCIMACRO{\TeXButton{sum_dstroke}{\tsum\mkern-20mu\tsum}}%
%BeginExpansion
\tsum\mkern-20mu\tsum%
%EndExpansion
}%
\end{array}%
\right) $ $\underset{\Rightarrow }{\text{then }}\mathbb{Y}_{i}\sim N\left( 
\begin{array}{cc}
\mu _{i} & \sigma _{i}^{2}%
\end{array}%
\right) $\newline
$i=1,2,\cdots ,p$

\bigskip

\begin{theorem}
Suppose that $\mathbb{Y}\sim N_{p}\left( 
\begin{array}{cc}
\mathbb{%
%TCIMACRO{\TeXButton{mu_dstroke}{\mu\mkern-12.3mu\mu}}%
%BeginExpansion
\mu\mkern-12.3mu\mu%
%EndExpansion
} & \mathbb{%
%TCIMACRO{\TeXButton{sum_dstroke}{\tsum\mkern-20mu\tsum}}%
%BeginExpansion
\tsum\mkern-20mu\tsum%
%EndExpansion
}%
\end{array}%
\right) $, where%
\begin{equation*}
\mathbb{Y}=\left[ 
\begin{array}{c}
Y_{1} \\ 
\vdots \\ 
Y_{m}%
\end{array}%
\right] ,\quad \mathbb{%
%TCIMACRO{\TeXButton{mu_dstroke}{\mu\mkern-12.3mu\mu}}%
%BeginExpansion
\mu\mkern-12.3mu\mu%
%EndExpansion
}=\left[ 
\begin{array}{c}
\mu _{1} \\ 
\vdots \\ 
\mu _{m}%
\end{array}%
\right] ,\quad \text{and}
\end{equation*}%
\begin{equation*}
\mathbb{%
%TCIMACRO{\TeXButton{sum_dstroke}{\tsum\mkern-20mu\tsum}}%
%BeginExpansion
\tsum\mkern-20mu\tsum%
%EndExpansion
}=\left[ 
\begin{array}{ccc}
\tsum\nolimits_{11} &  &  \\ 
& \ddots &  \\ 
&  & \tsum\nolimits_{mm}%
\end{array}%
\right]
\end{equation*}%
Then $\mathbb{Y}_{1},\cdots ,\mathbb{Y}_{m}$ are joint independent if and
only if $\mathbb{%
%TCIMACRO{\TeXButton{sum_dstroke}{\tsum\mkern-20mu\tsum}}%
%BeginExpansion
\tsum\mkern-20mu\tsum%
%EndExpansion
}_{ij}=\boldsymbol{0\quad \forall i\neq j}$
\end{theorem}

\bigskip

note: $\mathbb{Y}\sim N_{p}\left( 
\begin{array}{cc}
\mathbb{%
%TCIMACRO{\TeXButton{mu_dstroke}{\mu\mkern-12.3mu\mu}}%
%BeginExpansion
\mu\mkern-12.3mu\mu%
%EndExpansion
} & \mathbb{%
%TCIMACRO{\TeXButton{sum_dstroke}{\tsum\mkern-20mu\tsum}}%
%BeginExpansion
\tsum\mkern-20mu\tsum%
%EndExpansion
}%
\end{array}%
\right) $, then any two individual variables $Y_{i}$ and $Y_{j}$ are
independent if $\sigma _{ij}=0$

\bigskip

\begin{theorem}
Suppose that $\mathbb{Y}\sim N_{p}\left( 
\begin{array}{cc}
\mathbb{%
%TCIMACRO{\TeXButton{mu_dstroke}{\mu\mkern-12.3mu\mu}}%
%BeginExpansion
\mu\mkern-12.3mu\mu%
%EndExpansion
} & \mathbb{%
%TCIMACRO{\TeXButton{sum_dstroke}{\tsum\mkern-20mu\tsum}}%
%BeginExpansion
\tsum\mkern-20mu\tsum%
%EndExpansion
}%
\end{array}%
\right) $ and let $\mathbb{Y}_{1}=%
%TCIMACRO{\TeXButton{a_dstroke}{a\mkern-11mu a}}%
%BeginExpansion
a\mkern-11mu a%
%EndExpansion
_{1}\mathbb{+B}_{1}\mathbb{Y}$ and $\mathbb{Y}_{2}=%
%TCIMACRO{\TeXButton{a_dstroke}{a\mkern-11mu a}}%
%BeginExpansion
a\mkern-11mu a%
%EndExpansion
_{2}\mathbb{+B}_{2}\mathbb{Y}$, for nonrundom conformable $a_{i}$ and $%
\mathbb{B}_{i}\quad i=1,2.$ Then $\mathbb{Y}_{1}$ and $\mathbb{Y}_{2}$ are
independent if and only if $\mathbb{B}_{1}\tsum \mathbb{B}_{2}^{\dagger }=%
\boldsymbol{0}$

\begin{proof}
\begin{eqnarray*}
\mathbb{Y}^{\prime } &=&\left[ 
\begin{array}{c}
\mathbb{Y}_{1} \\ 
\mathbb{Y}_{2}%
\end{array}%
\right] =\left[ 
\begin{array}{c}
%TCIMACRO{\TeXButton{a_dstroke}{a\mkern-11mu a}}%
%BeginExpansion
a\mkern-11mu a%
%EndExpansion
_{1} \\ 
%TCIMACRO{\TeXButton{a_dstroke}{a\mkern-11mu a}}%
%BeginExpansion
a\mkern-11mu a%
%EndExpansion
_{2}%
\end{array}%
\right] +\left[ 
\begin{array}{c}
\mathbb{B}_{1} \\ 
\mathbb{B}_{2}%
\end{array}%
\right] \mathbb{Y} \\
&=&%
%TCIMACRO{\TeXButton{a_dstroke}{a\mkern-11mu a}}%
%BeginExpansion
a\mkern-11mu a%
%EndExpansion
+\mathbb{BY}
\end{eqnarray*}%
\newline
$\because $ $\mathbb{Y}^{\prime }$ is a linear combination of $\mathbb{Y}$%
\newline
$\therefore $ $\mathbb{Y}^{\prime }$ follows a multivariate normal \ dist.%
\newline
(i.e. $\mathbb{Y}_{1}$ and $\mathbb{Y}_{2}$ are joint normal)%
\begin{eqnarray*}
Cov\left( \mathbb{Y}_{1},\mathbb{Y}_{2}\right) &=&Cov\left( \mathbb{B}_{1}%
\mathbb{Y}+%
%TCIMACRO{\TeXButton{a_dstroke}{a\mkern-11mu a}}%
%BeginExpansion
a\mkern-11mu a%
%EndExpansion
_{1},\mathbb{B}_{2}\mathbb{Y}+%
%TCIMACRO{\TeXButton{a_dstroke}{a\mkern-11mu a}}%
%BeginExpansion
a\mkern-11mu a%
%EndExpansion
_{2}\right) \\
&=&Cov\left( \mathbb{B}_{1}\mathbb{Y},\mathbb{B}_{2}\mathbb{Y}\right) \\
&=&\mathbb{B}_{1}\tsum \mathbb{B}_{2}^{\dagger }=\boldsymbol{0}
\end{eqnarray*}
\end{proof}
\end{theorem}

\begin{example}
Consider the model $\mathbb{Y=X}%
%TCIMACRO{\TeXButton{beta_dstroke}{\beta\mkern-9.6mu \beta}}%
%BeginExpansion
\beta\mkern-9.6mu \beta%
%EndExpansion
+%
%TCIMACRO{\TeXButton{epsilon_dstroke}{\epsilon\mkern-10mu \epsilon}}%
%BeginExpansion
\epsilon\mkern-10mu \epsilon%
%EndExpansion
$, 
\begin{equation*}
%TCIMACRO{\TeXButton{epsilon_dstroke}{\epsilon\mkern-10mu \epsilon}}%
%BeginExpansion
\epsilon\mkern-10mu \epsilon%
%EndExpansion
\sim N_{n}\left( \boldsymbol{0,}\sigma ^{2}I\right)
\end{equation*}%
\begin{equation*}
\mathbb{\hat{Y}\sim }N_{n}\left( \mathbb{X}%
%TCIMACRO{\TeXButton{beta_dstroke}{\beta\mkern-9.6mu \beta}}%
%BeginExpansion
\beta\mkern-9.6mu \beta%
%EndExpansion
,\sigma ^{2}I\right)
\end{equation*}%
\begin{equation*}
\left[ 
\begin{array}{c}
\mathbb{\hat{Y}} \\ 
%TCIMACRO{\TeXButton{e_dstroke_hat}{\hat{e\mkern-10.5mu e}}}%
%BeginExpansion
\hat{e\mkern-10.5mu e}%
%EndExpansion
\end{array}%
\right] ^{\prime }=\left[ 
\begin{array}{c}
P_{\mathbb{X}}\mathbb{Y} \\ 
\left( I-P_{\mathbb{X}}\right) \mathbb{Y}%
\end{array}%
\right] =\left[ 
\begin{array}{c}
P_{\mathbb{X}} \\ 
I-P_{\mathbb{X}}%
\end{array}%
\right] \mathbb{Y}
\end{equation*}%
is a l.c. of $\mathbb{Y}$\newline
Thus, $\mathbb{\hat{Y}}$ and $%
%TCIMACRO{\TeXButton{e_dstroke_hat}{\hat{e\mkern-10.5mu e}}}%
%BeginExpansion
\hat{e\mkern-10.5mu e}%
%EndExpansion
$ are jointly normal\newline
\newline
We know that $\mathbb{\hat{Y}}$ and $%
%TCIMACRO{\TeXButton{e_dstroke_hat}{\hat{e\mkern-10.5mu e}}}%
%BeginExpansion
\hat{e\mkern-10.5mu e}%
%EndExpansion
$ are indep%
\begin{eqnarray*}
&\because &Cov\left( \mathbb{\hat{Y}},%
%TCIMACRO{\TeXButton{e_dstroke_hat}{\hat{e\mkern-10.5mu e}}}%
%BeginExpansion
\hat{e\mkern-10.5mu e}%
%EndExpansion
\right) =Cov\left( P_{\mathbb{X}}\mathbb{Y},\left( I-P_{\mathbb{X}}\right) 
\mathbb{Y}\right) \\
&=&P_{\mathbb{X}}\sigma ^{2}I\left( I-P_{\mathbb{X}}\right) =\boldsymbol{0}
\end{eqnarray*}
\end{example}

\bigskip

\paragraph{Distribution of Quadric forms}

\begin{equation*}
\mathbb{Y}^{\mathbb{\dagger }}\mathbb{AY\sim }\fbox{?}\text{\quad where\quad 
}\mathbb{Y}\sim N_{p}\left( 
\begin{array}{cc}
\mathbb{%
%TCIMACRO{\TeXButton{mu_dstroke}{\mu\mkern-12.3mu\mu}}%
%BeginExpansion
\mu\mkern-12.3mu\mu%
%EndExpansion
} & \mathbb{%
%TCIMACRO{\TeXButton{sum_dstroke}{\tsum\mkern-20mu\tsum}}%
%BeginExpansion
\tsum\mkern-20mu\tsum%
%EndExpansion
}%
\end{array}%
\right)
\end{equation*}

\subparagraph{Sum of squares}

\begin{eqnarray*}
\tsum\limits_{i=1}^{n}\left( y_{i}-\bar{y}\right) ^{2}
&=&\tsum\limits_{i=1}^{n}y_{i}^{2}-n\bar{y}^{2} \\
&=&\mathbb{Y}^{\mathbb{\dagger }}\mathbb{Y-Y}^{\mathbb{\dagger }}\frac{1}{n}%
\mathbb{JY} \\
&=&\mathbb{Y}^{\mathbb{\dagger }}\left( I-\frac{1}{n}\mathbb{J}\right) 
\mathbb{Y}
\end{eqnarray*}

properties:

\begin{enumerate}
\item $I=\left( I-\frac{1}{n}\mathbb{J}\right) +\frac{1}{n}\mathbb{J}$

\item $I$, $\left( I-\frac{1}{n}\mathbb{J}\right) $ and $\frac{1}{n}\mathbb{J%
}$ are independent (why)
\end{enumerate}

\bigskip

\begin{theorem}
If $\mathbb{Y}$ is a random vector with $E\left( \mathbb{Y}\right) =\mathbb{%
%TCIMACRO{\TeXButton{mu_dstroke}{\mu\mkern-12.3mu\mu}}%
%BeginExpansion
\mu\mkern-12.3mu\mu%
%EndExpansion
}$ and $Var\left( \mathbb{Y}\right) =\mathbb{%
%TCIMACRO{\TeXButton{sum_dstroke}{\tsum\mkern-20mu\tsum}}%
%BeginExpansion
\tsum\mkern-20mu\tsum%
%EndExpansion
}$ and if $\mathbb{A}$ is a symmetric matrix then%
\begin{equation*}
E\left( \mathbb{Y}^{\mathbb{\dagger }}\mathbb{A}%
%TCIMACRO{\TeXButton{y_dstroke}{y\mkern-11mu y}}%
%BeginExpansion
y\mkern-11mu y%
%EndExpansion
\right) =tr\left( \mathbb{A%
%TCIMACRO{\TeXButton{sum_dstroke}{\tsum\mkern-20mu\tsum}}%
%BeginExpansion
\tsum\mkern-20mu\tsum%
%EndExpansion
}\right) +\mathbb{%
%TCIMACRO{\TeXButton{mu_dstroke}{\mu\mkern-12.3mu\mu}}%
%BeginExpansion
\mu\mkern-12.3mu\mu%
%EndExpansion
}^{\dagger }\mathbb{A%
%TCIMACRO{\TeXButton{sum_dstroke}{\tsum\mkern-20mu\tsum}}%
%BeginExpansion
\tsum\mkern-20mu\tsum%
%EndExpansion
}
\end{equation*}

\begin{proof}
tool:%
\begin{eqnarray*}
\mathbb{%
%TCIMACRO{\TeXButton{sum_dstroke}{\tsum\mkern-20mu\tsum}}%
%BeginExpansion
\tsum\mkern-20mu\tsum%
%EndExpansion
}_{%
%TCIMACRO{\TeXButton{y_dstroke}{y\mkern-11mu y}}%
%BeginExpansion
y\mkern-11mu y%
%EndExpansion
} &=&E\left[ \left( 
%TCIMACRO{\TeXButton{y_dstroke}{y\mkern-11mu y}}%
%BeginExpansion
y\mkern-11mu y%
%EndExpansion
-\mathbb{%
%TCIMACRO{\TeXButton{mu_dstroke}{\mu\mkern-12.3mu\mu}}%
%BeginExpansion
\mu\mkern-12.3mu\mu%
%EndExpansion
}\right) \left( 
%TCIMACRO{\TeXButton{y_dstroke}{y\mkern-11mu y}}%
%BeginExpansion
y\mkern-11mu y%
%EndExpansion
-\mathbb{%
%TCIMACRO{\TeXButton{mu_dstroke}{\mu\mkern-12.3mu\mu}}%
%BeginExpansion
\mu\mkern-12.3mu\mu%
%EndExpansion
}\right) ^{\dagger }\right] \\
&=&E\left[ 
%TCIMACRO{\TeXButton{y_dstroke}{y\mkern-11mu y}}%
%BeginExpansion
y\mkern-11mu y%
%EndExpansion
%TCIMACRO{\TeXButton{y_dstroke}{y\mkern-11mu y}}%
%BeginExpansion
y\mkern-11mu y%
%EndExpansion
^{\dagger }-%
%TCIMACRO{\TeXButton{y_dstroke}{y\mkern-11mu y}}%
%BeginExpansion
y\mkern-11mu y%
%EndExpansion
\mathbb{%
%TCIMACRO{\TeXButton{mu_dstroke}{\mu\mkern-12.3mu\mu}}%
%BeginExpansion
\mu\mkern-12.3mu\mu%
%EndExpansion
}^{\dagger }-\mathbb{%
%TCIMACRO{\TeXButton{mu_dstroke}{\mu\mkern-12.3mu\mu}}%
%BeginExpansion
\mu\mkern-12.3mu\mu%
%EndExpansion
}%
%TCIMACRO{\TeXButton{y_dstroke}{y\mkern-11mu y}}%
%BeginExpansion
y\mkern-11mu y%
%EndExpansion
^{\dagger }+\mathbb{%
%TCIMACRO{\TeXButton{mu_dstroke}{\mu\mkern-12.3mu\mu}}%
%BeginExpansion
\mu\mkern-12.3mu\mu%
%EndExpansion
%TCIMACRO{\TeXButton{mu_dstroke}{\mu\mkern-12.3mu\mu}}%
%BeginExpansion
\mu\mkern-12.3mu\mu%
%EndExpansion
}^{\dagger }\right] \\
&=&E\left[ \left( 
%TCIMACRO{\TeXButton{y_dstroke}{y\mkern-11mu y}}%
%BeginExpansion
y\mkern-11mu y%
%EndExpansion
%TCIMACRO{\TeXButton{y_dstroke}{y\mkern-11mu y}}%
%BeginExpansion
y\mkern-11mu y%
%EndExpansion
^{\dagger }\right) -\mathbb{%
%TCIMACRO{\TeXButton{mu_dstroke}{\mu\mkern-12.3mu\mu}}%
%BeginExpansion
\mu\mkern-12.3mu\mu%
%EndExpansion
%TCIMACRO{\TeXButton{mu_dstroke}{\mu\mkern-12.3mu\mu}}%
%BeginExpansion
\mu\mkern-12.3mu\mu%
%EndExpansion
}^{\dagger }\right]
\end{eqnarray*}%
\begin{equation*}
\therefore E\left( 
%TCIMACRO{\TeXButton{y_dstroke}{y\mkern-11mu y}}%
%BeginExpansion
y\mkern-11mu y%
%EndExpansion
%TCIMACRO{\TeXButton{y_dstroke}{y\mkern-11mu y}}%
%BeginExpansion
y\mkern-11mu y%
%EndExpansion
^{\dagger }\right) =\mathbb{%
%TCIMACRO{\TeXButton{sum_dstroke}{\tsum\mkern-20mu\tsum}}%
%BeginExpansion
\tsum\mkern-20mu\tsum%
%EndExpansion
+%
%TCIMACRO{\TeXButton{mu_dstroke}{\mu\mkern-12.3mu\mu}}%
%BeginExpansion
\mu\mkern-12.3mu\mu%
%EndExpansion
%TCIMACRO{\TeXButton{mu_dstroke}{\mu\mkern-12.3mu\mu}}%
%BeginExpansion
\mu\mkern-12.3mu\mu%
%EndExpansion
}^{\dagger }
\end{equation*}%
\begin{eqnarray*}
E\left( 
%TCIMACRO{\TeXButton{y_dstroke}{y\mkern-11mu y}}%
%BeginExpansion
y\mkern-11mu y%
%EndExpansion
^{\mathbb{\dagger }}\mathbb{A}%
%TCIMACRO{\TeXButton{y_dstroke}{y\mkern-11mu y}}%
%BeginExpansion
y\mkern-11mu y%
%EndExpansion
\right) &=&E\left( tr\left( 
%TCIMACRO{\TeXButton{y_dstroke}{y\mkern-11mu y}}%
%BeginExpansion
y\mkern-11mu y%
%EndExpansion
^{\mathbb{\dagger }}\mathbb{A}%
%TCIMACRO{\TeXButton{y_dstroke}{y\mkern-11mu y}}%
%BeginExpansion
y\mkern-11mu y%
%EndExpansion
\right) \right) \\
&=&E\left( tr\left( \mathbb{A}%
%TCIMACRO{\TeXButton{y_dstroke}{y\mkern-11mu y}}%
%BeginExpansion
y\mkern-11mu y%
%EndExpansion
%TCIMACRO{\TeXButton{y_dstroke}{y\mkern-11mu y}}%
%BeginExpansion
y\mkern-11mu y%
%EndExpansion
^{\mathbb{\dagger }}\right) \right) \\
&=&tr\left( E\left( \mathbb{A}%
%TCIMACRO{\TeXButton{y_dstroke}{y\mkern-11mu y}}%
%BeginExpansion
y\mkern-11mu y%
%EndExpansion
%TCIMACRO{\TeXButton{y_dstroke}{y\mkern-11mu y}}%
%BeginExpansion
y\mkern-11mu y%
%EndExpansion
^{\mathbb{\dagger }}\right) \right) \\
&=&tr\left( \mathbb{AE}\left( 
%TCIMACRO{\TeXButton{y_dstroke}{y\mkern-11mu y}}%
%BeginExpansion
y\mkern-11mu y%
%EndExpansion
%TCIMACRO{\TeXButton{y_dstroke}{y\mkern-11mu y}}%
%BeginExpansion
y\mkern-11mu y%
%EndExpansion
^{\mathbb{\dagger }}\right) \right) \\
&=&tr\left( \mathbb{A}\left( \mathbb{%
%TCIMACRO{\TeXButton{sum_dstroke}{\tsum\mkern-20mu\tsum}}%
%BeginExpansion
\tsum\mkern-20mu\tsum%
%EndExpansion
+%
%TCIMACRO{\TeXButton{mu_dstroke}{\mu\mkern-12.3mu\mu}}%
%BeginExpansion
\mu\mkern-12.3mu\mu%
%EndExpansion
%TCIMACRO{\TeXButton{mu_dstroke}{\mu\mkern-12.3mu\mu}}%
%BeginExpansion
\mu\mkern-12.3mu\mu%
%EndExpansion
}^{\dagger }\right) \right) \\
&=&tr\left( \mathbb{A%
%TCIMACRO{\TeXButton{sum_dstroke}{\tsum\mkern-20mu\tsum}}%
%BeginExpansion
\tsum\mkern-20mu\tsum%
%EndExpansion
+A%
%TCIMACRO{\TeXButton{mu_dstroke}{\mu\mkern-12.3mu\mu}}%
%BeginExpansion
\mu\mkern-12.3mu\mu%
%EndExpansion
%TCIMACRO{\TeXButton{mu_dstroke}{\mu\mkern-12.3mu\mu}}%
%BeginExpansion
\mu\mkern-12.3mu\mu%
%EndExpansion
}^{\dagger }\right) \\
&=&tr\left( \mathbb{A%
%TCIMACRO{\TeXButton{sum_dstroke}{\tsum\mkern-20mu\tsum}}%
%BeginExpansion
\tsum\mkern-20mu\tsum%
%EndExpansion
}\right) +tr\left( \mathbb{A%
%TCIMACRO{\TeXButton{mu_dstroke}{\mu\mkern-12.3mu\mu}}%
%BeginExpansion
\mu\mkern-12.3mu\mu%
%EndExpansion
%TCIMACRO{\TeXButton{mu_dstroke}{\mu\mkern-12.3mu\mu}}%
%BeginExpansion
\mu\mkern-12.3mu\mu%
%EndExpansion
}^{\dagger }\right) \\
&=&tr\left( \mathbb{A%
%TCIMACRO{\TeXButton{sum_dstroke}{\tsum\mkern-20mu\tsum}}%
%BeginExpansion
\tsum\mkern-20mu\tsum%
%EndExpansion
}\right) +tr\left( \mathbb{%
%TCIMACRO{\TeXButton{mu_dstroke}{\mu\mkern-12.3mu\mu}}%
%BeginExpansion
\mu\mkern-12.3mu\mu%
%EndExpansion
}^{\dagger }\mathbb{A%
%TCIMACRO{\TeXButton{mu_dstroke}{\mu\mkern-12.3mu\mu}}%
%BeginExpansion
\mu\mkern-12.3mu\mu%
%EndExpansion
}\right) \\
&=&tr\left( \mathbb{A%
%TCIMACRO{\TeXButton{sum_dstroke}{\tsum\mkern-20mu\tsum}}%
%BeginExpansion
\tsum\mkern-20mu\tsum%
%EndExpansion
}\right) +\mathbb{%
%TCIMACRO{\TeXButton{mu_dstroke}{\mu\mkern-12.3mu\mu}}%
%BeginExpansion
\mu\mkern-12.3mu\mu%
%EndExpansion
}^{\dagger }\mathbb{A%
%TCIMACRO{\TeXButton{mu_dstroke}{\mu\mkern-12.3mu\mu}}%
%BeginExpansion
\mu\mkern-12.3mu\mu%
%EndExpansion
}
\end{eqnarray*}
\end{proof}
\end{theorem}

\bigskip

\begin{theorem}
Suppose that $\mathbb{Y}\sim N_{p}\left( \mathbb{%
%TCIMACRO{\TeXButton{mu_dstroke}{\mu\mkern-12.3mu\mu}}%
%BeginExpansion
\mu\mkern-12.3mu\mu%
%EndExpansion
},I_{p}\right) $ and define%
\begin{equation*}
\mathbb{Y}^{\mathbb{\dagger }}\mathbb{Y=Y}^{\dagger }I\mathbb{Y=}%
\tsum\limits_{i=1}^{p}Y_{i}^{2}\sim \chi _{p}^{2}\left( \frac{\mathbb{%
%TCIMACRO{\TeXButton{mu_dstroke}{\mu\mkern-12.3mu\mu}}%
%BeginExpansion
\mu\mkern-12.3mu\mu%
%EndExpansion
}^{\dagger }\mathbb{%
%TCIMACRO{\TeXButton{mu_dstroke}{\mu\mkern-12.3mu\mu}}%
%BeginExpansion
\mu\mkern-12.3mu\mu%
%EndExpansion
}}{2}\right)
\end{equation*}%
Then%
\begin{equation*}
\mathbb{Y}^{\mathbb{\dagger }}\mathbb{AY\sim }\chi _{s}^{2}\left( \frac{%
\mathbb{%
%TCIMACRO{\TeXButton{mu_dstroke}{\mu\mkern-12.3mu\mu}}%
%BeginExpansion
\mu\mkern-12.3mu\mu%
%EndExpansion
}^{\dagger }\mathbb{A%
%TCIMACRO{\TeXButton{mu_dstroke}{\mu\mkern-12.3mu\mu}}%
%BeginExpansion
\mu\mkern-12.3mu\mu%
%EndExpansion
}}{2}\right)
\end{equation*}%
where $\mathbb{A}$ is idempotent of $r\left( \mathbb{A}\right) =S$
\end{theorem}

\bigskip

remark: The $p\times p$ symmetric matrix $\mathbb{A}$ is idempotent of rank $%
S$ iff $\exists $ $p\times s$ matrix $\mathbb{P}$ $\lambda t$

\begin{enumerate}
\item $\underset{p\times p}{\mathbb{A}}\mathbb{=}\underset{p\times s}{%
\mathbb{P}}\underset{\quad s\times p}{\mathbb{P}^{\dagger }}$

\item $\underset{s\times p}{\mathbb{P}^{\dagger }}\underset{\quad p\times s}{%
\mathbb{P}}\mathbb{=}I_{s}$
\end{enumerate}

\begin{proof}
$\mathbb{Y}^{\mathbb{\dagger }}\mathbb{AY=Y}^{\mathbb{\dagger }}\mathbb{PP}%
^{\dagger }\mathbb{Y=}\left( \underset{\mathbb{Y}^{\ast \dagger }}{%
\underbrace{\mathbb{P}^{\dagger }\mathbb{Y}}}\right) ^{\dagger }\left( 
\underset{\mathbb{Y}^{\ast }}{\underbrace{\mathbb{P}^{\dagger }\mathbb{Y}}}%
\right) $\newline
\newline
$\because \mathbb{P}^{\dagger }\mathbb{Y}$ is a linear combination of $%
\mathbb{Y}$ and $\mathbb{Y}\sim N_{p}\left( \mathbb{%
%TCIMACRO{\TeXButton{mu_dstroke}{\mu\mkern-12.3mu\mu}}%
%BeginExpansion
\mu\mkern-12.3mu\mu%
%EndExpansion
},I\right) $%
\begin{equation*}
\therefore \mathbb{P}^{\dagger }\mathbb{Y\sim }N_{s}\left( \mathbb{P}%
^{\dagger }\mathbb{%
%TCIMACRO{\TeXButton{mu_dstroke}{\mu\mkern-12.3mu\mu}}%
%BeginExpansion
\mu\mkern-12.3mu\mu%
%EndExpansion
},\underset{\mathbb{P}^{\dagger }\mathbb{P}=I_{s}}{\underbrace{\mathbb{P}%
^{\dagger }I\left( \mathbb{P}^{\dagger }\right) ^{\dagger }}}\right) 
\end{equation*}%
\begin{equation*}
\therefore \left( \mathbb{P}^{\dagger }\mathbb{Y}\right) ^{\dagger }\left( 
\mathbb{P}^{\dagger }\mathbb{Y}\right) \sim \chi _{s}^{2}\left( \underset{%
\frac{\mathbb{%
%TCIMACRO{\TeXButton{mu_dstroke}{\mu\mkern-12.3mu\mu}}%
%BeginExpansion
\mu\mkern-12.3mu\mu%
%EndExpansion
}^{\dagger }\mathbb{A%
%TCIMACRO{\TeXButton{mu_dstroke}{\mu\mkern-12.3mu\mu}}%
%BeginExpansion
\mu\mkern-12.3mu\mu%
%EndExpansion
}}{2}}{\underbrace{\frac{\left( \mathbb{P}^{\dagger }\mathbb{%
%TCIMACRO{\TeXButton{mu_dstroke}{\mu\mkern-12.3mu\mu}}%
%BeginExpansion
\mu\mkern-12.3mu\mu%
%EndExpansion
}\right) ^{\dagger }\left( \mathbb{P}^{\dagger }\mathbb{%
%TCIMACRO{\TeXButton{mu_dstroke}{\mu\mkern-12.3mu\mu}}%
%BeginExpansion
\mu\mkern-12.3mu\mu%
%EndExpansion
}\right) }{2}}}\right) 
\end{equation*}
\end{proof}

\bigskip 

\begin{theorem}
Suppose that $\mathbb{Y}\sim N_{p}\left( \mathbb{%
%TCIMACRO{\TeXButton{mu_dstroke}{\mu\mkern-12.3mu\mu}}%
%BeginExpansion
\mu\mkern-12.3mu\mu%
%EndExpansion
},\mathbb{%
%TCIMACRO{\TeXButton{sum_dstroke}{\tsum\mkern-20mu\tsum}}%
%BeginExpansion
\tsum\mkern-20mu\tsum%
%EndExpansion
}\right) $ where $\gamma \left( \mathbb{%
%TCIMACRO{\TeXButton{sum_dstroke}{\tsum\mkern-20mu\tsum}}%
%BeginExpansion
\tsum\mkern-20mu\tsum%
%EndExpansion
}\right) =p$. If $\mathbb{A%
%TCIMACRO{\TeXButton{sum_dstroke}{\tsum\mkern-20mu\tsum}}%
%BeginExpansion
\tsum\mkern-20mu\tsum%
%EndExpansion
}$ is idempotent of rank $S$, then 
\begin{equation*}
\mathbb{Y}^{\mathbb{\dagger }}\mathbb{AY\sim }\chi _{s}^{2}\left( \frac{%
\mathbb{%
%TCIMACRO{\TeXButton{mu_dstroke}{\mu\mkern-12.3mu\mu}}%
%BeginExpansion
\mu\mkern-12.3mu\mu%
%EndExpansion
}^{\dagger }\mathbb{A%
%TCIMACRO{\TeXButton{mu_dstroke}{\mu\mkern-12.3mu\mu}}%
%BeginExpansion
\mu\mkern-12.3mu\mu%
%EndExpansion
}}{2}\right) 
\end{equation*}

\begin{proof}
$\because \tsum\nolimits^{-\frac{1}{2}}\mathbb{Y}$ is a linear combination
of $\mathbb{Y}$ and 
\begin{equation*}
\mathbb{Y}\sim N_{p}\left( \mathbb{%
%TCIMACRO{\TeXButton{mu_dstroke}{\mu\mkern-12.3mu\mu}}%
%BeginExpansion
\mu\mkern-12.3mu\mu%
%EndExpansion
},\mathbb{%
%TCIMACRO{\TeXButton{sum_dstroke}{\tsum\mkern-20mu\tsum}}%
%BeginExpansion
\tsum\mkern-20mu\tsum%
%EndExpansion
}\right) 
\end{equation*}%
\begin{eqnarray*}
&\therefore &\tsum\nolimits^{-\frac{1}{2}}\mathbb{Y\sim }N_{p}\left( \mathbb{%
%TCIMACRO{\TeXButton{sum_dstroke}{\tsum\mkern-20mu\tsum}}%
%BeginExpansion
\tsum\mkern-20mu\tsum%
%EndExpansion
}^{-\frac{1}{2}}\mathbb{%
%TCIMACRO{\TeXButton{mu_dstroke}{\mu\mkern-12.3mu\mu}}%
%BeginExpansion
\mu\mkern-12.3mu\mu%
%EndExpansion
},\tsum\nolimits^{\frac{1}{2}}\tsum \tsum\nolimits^{-\frac{1}{2}}\right)  \\
&=&N_{p}\left( \mathbb{%
%TCIMACRO{\TeXButton{sum_dstroke}{\tsum\mkern-20mu\tsum}}%
%BeginExpansion
\tsum\mkern-20mu\tsum%
%EndExpansion
}^{-\frac{1}{2}}\mathbb{%
%TCIMACRO{\TeXButton{mu_dstroke}{\mu\mkern-12.3mu\mu}}%
%BeginExpansion
\mu\mkern-12.3mu\mu%
%EndExpansion
},I_{p}\right) 
\end{eqnarray*}%
\begin{eqnarray*}
\mathbb{Y}^{\mathbb{\dagger }}\mathbb{AY} &\mathbb{=Y}&^{\mathbb{\dagger }%
}\tsum\nolimits^{-\frac{1}{2}}\tsum\nolimits^{\frac{1}{2}}\mathbb{A}%
\tsum\nolimits^{\frac{1}{2}}\tsum\nolimits^{-\frac{1}{2}}\mathbb{Y} \\
&=&\left( \tsum\nolimits^{-\frac{1}{2}}\mathbb{Y}\right) ^{\dagger }\underset%
{\mathbb{B}}{\underbrace{\tsum\nolimits^{\frac{1}{2}}\mathbb{A}%
\tsum\nolimits^{\frac{1}{2}}}}\left( \tsum\nolimits^{-\frac{1}{2}}\mathbb{Y}%
\right) 
\end{eqnarray*}%
\begin{equation*}
\gamma \left( \mathbb{B}\right) =\gamma \left( \tsum\nolimits^{\frac{1}{2}}%
\mathbb{A}\tsum\nolimits^{\frac{1}{2}}\right) =\gamma \left( \mathbb{A}\tsum
\right) =S
\end{equation*}%
and $\mathbb{B}$ is idempotent (why)
\end{proof}
\end{theorem}

\end{document}
