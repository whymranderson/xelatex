
\documentclass{article}
%%%%%%%%%%%%%%%%%%%%%%%%%%%%%%%%%%%%%%%%%%%%%%%%%%%%%%%%%%%%%%%%%%%%%%%%%%%%%%%%%%%%%%%%%%%%%%%%%%%%%%%%%%%%%%%%%%%%%%%%%%%%%%%%%%%%%%%%%%%%%%%%%%%%%%%%%%%%%%%%%%%%%%%%%%%%%%%%%%%%%%%%%%%%%%%%%%%%%%%%%%%%%%%%%%%%%%%%%%%%%%%%%%%%%%%%%%%%%%%%%%%%%%%%%%%%
\usepackage{amssymb}
\usepackage{amsfonts}
\usepackage{amsmath}
\usepackage{accents}
\usepackage[ignoreall,a4paper]{geometry}
\usepackage{fancyhdr}

\setcounter{MaxMatrixCols}{10}
%TCIDATA{OutputFilter=LATEX.DLL}
%TCIDATA{Version=5.00.0.2606}
%TCIDATA{<META NAME="SaveForMode" CONTENT="1">}
%TCIDATA{BibliographyScheme=Manual}
%TCIDATA{Created=Wednesday, November 25, 2015 15:33:37}
%TCIDATA{LastRevised=Monday, July 04, 2016 17:00:46}
%TCIDATA{<META NAME="GraphicsSave" CONTENT="32">}
%TCIDATA{<META NAME="DocumentShell" CONTENT="Standard LaTeX\Blank - Standard LaTeX Article">}
%TCIDATA{CSTFile=40 LaTeX article.cst}
%TCIDATA{ComputeDefs=
%$W=\left( 1-\sigma \right) I$
%}


\newtheorem{theorem}{Theorem}
\newtheorem{acknowledgement}[theorem]{Acknowledgement}
\newtheorem{algorithm}[theorem]{Algorithm}
\newtheorem{axiom}[theorem]{Axiom}
\newtheorem{case}[theorem]{Case}
\newtheorem{claim}[theorem]{Claim}
\newtheorem{conclusion}[theorem]{Conclusion}
\newtheorem{condition}[theorem]{Condition}
\newtheorem{conjecture}[theorem]{Conjecture}
\newtheorem{corollary}[theorem]{Corollary}
\newtheorem{criterion}[theorem]{Criterion}
\newtheorem{definition}[theorem]{Definition}
\newtheorem{example}[theorem]{Example}
\newtheorem{exercise}[theorem]{Exercise}
\newtheorem{lemma}[theorem]{Lemma}
\newtheorem{notation}[theorem]{Notation}
\newtheorem{problem}[theorem]{Problem}
\newtheorem{proposition}[theorem]{Proposition}
\newtheorem{remark}[theorem]{Remark}
\newtheorem{solution}[theorem]{Solution}
\newtheorem{summary}[theorem]{Summary}
\newenvironment{proof}[1][Proof]{\noindent\textbf{#1.} }{\ \rule{0.5em}{0.5em}}
\input{../../tcilatex}
\DeclareMathAccent{\wtilde}{\mathord}{largesymbols}{"65}
\pagestyle{fancy}
\fancyfoot[C]{\thepage}



\begin{document}


\begin{theorem}
If $\mathbf{Y}\sim N_{n}\left( \mathbf{X\beta },\sigma ^{2}I\right) $, then
the F test for $H_{0}:\mathbf{C\beta =0}$ is equivalent to the likelihood
ratio test.

\begin{proof}
As $\mathbf{C\beta =0}$. Let%
\begin{eqnarray*}
V &=&l\left( \mathbf{\beta },\sigma ^{2}\right) +\mathbf{\lambda }^{\dagger
}\left( \mathbf{C\beta -0}\right) \\
&=&-\frac{n}{2}\ln 2\pi -\frac{n}{2}\ln \sigma ^{2}-\frac{\left( \mathbf{%
Y-X\beta }\right) ^{\dagger }\left( \mathbf{Y-X\beta }\right) }{2\hat{\sigma}%
^{2}}+\mathbf{\lambda }^{\dagger }\left( \mathbf{C\beta -0}\right)
\end{eqnarray*}%
\begin{equation*}
\begin{array}{c}
\frac{\partial V}{\partial \mathbf{\beta }}=0 \\ 
\frac{\partial V}{\partial \sigma ^{2}}=0 \\ 
\frac{\partial V}{\partial \mathbf{\lambda }}=0%
\end{array}%
\Rightarrow \left[ 
\begin{tabular}{l}
$\left( 2\mathbf{X}^{\dagger }\mathbf{Y-2X}^{\dagger }\mathbf{X\hat{\beta}}%
\right) /2\hat{\sigma}^{2}+\mathbf{C}^{\mathbf{\dagger }}\mathbf{\lambda }=%
\mathbf{0}$ \\ 
$-\frac{n}{2\sigma ^{2}}+\frac{1}{2\left( \sigma ^{2}\right) ^{2}}\left( 
\mathbf{Y-X\hat{\beta}}\right) ^{\dagger }\left( \mathbf{Y-X\hat{\beta}}%
\right) =0$ \\ 
$\mathbf{C\beta }=\mathbf{0}$%
\end{tabular}%
\right.
\end{equation*}%
\begin{equation*}
\Rightarrow \hat{\beta}_{0}=\hat{\beta}-\left( \mathbf{X}^{\dagger }\mathbf{X%
}\right) ^{-1}\mathbf{C}^{\dagger }\left[ \mathbf{C}\left( \mathbf{X}%
^{\dagger }\mathbf{X}\right) ^{-1}\mathbf{C}^{\dagger }\right] ^{-1}\mathbf{C%
}\hat{\beta}
\end{equation*}%
\begin{eqnarray*}
\hat{\sigma}_{0}^{2} &=&\hat{\sigma}^{2}+\frac{1}{n}\left( \mathbf{C\hat{%
\beta}}\right) ^{\dagger }\left[ \mathbf{C}\left( \mathbf{X}^{\dagger }%
\mathbf{X}\right) ^{-1}\mathbf{C}^{\dagger }\right] ^{-1}\mathbf{C\hat{\beta}%
} \\
&=&\frac{1}{n}\left( \mathbf{Y-X}\hat{\beta}_{0}\right) ^{\dagger }\left( 
\mathbf{Y-X}\hat{\beta}_{0}\right)
\end{eqnarray*}%
\newline
Recall%
\begin{equation*}
\frac{\partial }{\partial \mathbf{x}}\left\{ \mathbf{a}^{\dagger }\mathbf{x}%
\right\} =\frac{\partial }{\partial \mathbf{x}}\left\{ \mathbf{x}^{\dagger }%
\mathbf{a}\right\} =\mathbf{a}
\end{equation*}%
\begin{equation*}
\text{max}_{H_{0}}L\left( \mathbf{\beta },\sigma ^{2}\right) =L\left( 
\mathbf{\hat{\beta}}_{0},\hat{\sigma}_{0}^{2}\right) =\left( 2\pi \right) ^{-%
\frac{n}{2}}\left( \hat{\sigma}_{0}^{2}\right) ^{-\frac{n}{2}}e^{-\frac{%
\left( \mathbf{Y-X\hat{\beta}}_{0}\right) ^{\dagger }\left( \mathbf{Y-X\hat{%
\beta}}_{0}\right) }{2\hat{\sigma}_{0}^{2}}}
\end{equation*}%
\begin{eqnarray*}
\text{LR} &=&\frac{\text{max}_{H_{0}}L\left( \mathbf{\beta },\sigma
^{2}\right) }{\text{max}_{H_{1}}L\left( \mathbf{\beta },\sigma ^{2}\right) }=%
\left[ \frac{\text{SSE}}{\text{SSE}+\left( \mathbf{C\hat{\beta}}\right)
^{\dagger }\left[ \mathbf{C}\left( \mathbf{X}^{\dagger }\mathbf{X}\right)
^{-1}\mathbf{C}^{\dagger }\right] ^{-1}\mathbf{C\hat{\beta}}}\right] ^{\frac{%
n}{2}} \\
&=&\left[ \frac{1}{1+\frac{\text{SSH}}{\text{SSE}}}\right] ^{\frac{n}{2}}=%
\left[ \frac{1}{1+F\frac{q}{n-p}}\right] ^{\frac{n}{2}}
\end{eqnarray*}%
where%
\begin{equation*}
\begin{tabular}{l}
SSH $=\left( \mathbf{C\hat{\beta}}\right) ^{\dagger }\left[ \mathbf{C}\left( 
\mathbf{X}^{\dagger }\mathbf{X}\right) ^{-1}\mathbf{C}^{\dagger }\right]
^{-1}\mathbf{C\hat{\beta}}$ \\ 
SSE $=\left( \mathbf{Y-X\hat{\beta}}\right) ^{\dagger }\left( \mathbf{Y-X%
\hat{\beta}}\right) $ \\ 
F $=\frac{\text{SSH}/q}{\text{SSE}/n-p}$%
\end{tabular}%
\end{equation*}
\end{proof}
\end{theorem}

\bigskip

\bigskip

\subsection{Confidence Intervals and Prediction Intervals for future obs.}

\begin{description}
\item[Goal] Consider the linear model $\mathbf{Y=X\beta +\epsilon }$ where $%
\mathbf{X}$ is $n\times p$ matrix with $r\left( \mathbf{X}\right) =r<p\leq n$
and $\epsilon \sim N_{n}\left( 0,\sigma ^{2}I\right) $. Suppose that $%
\mathbf{\lambda }^{\dagger }\mathbf{\beta }$ is estimable, that is $\mathbf{%
\lambda }^{\dagger }=\mathbf{a}^{\dagger }\mathbf{X}$ for some vector $%
\mathbf{a}$. The goal is to write a $100\left( 1-\alpha \right) $ percent
confidence interval for $\mathbf{\lambda }^{\dagger }\mathbf{\beta }$.
\end{description}

\bigskip

LSE of $\mathbf{\lambda }^{\dagger }\mathbf{\beta }$ is $\mathbf{\lambda }%
^{\dagger }\mathbf{\hat{\beta}}$ and we have%
\begin{equation*}
\mathbf{\lambda }^{\dagger }\mathbf{\hat{\beta}}\sim N_{1}\left( \mathbf{%
\lambda }^{\dagger }\mathbf{\beta },\sigma ^{2}\mathbf{\lambda }^{\dagger
}\left( \mathbf{X}^{\dagger }\mathbf{X}\right) ^{-1}\mathbf{\lambda }\right)
\end{equation*}%
Thus,%
\begin{equation*}
Z=\frac{\mathbf{\lambda }^{\dagger }\mathbf{\hat{\beta}}-\mathbf{\lambda }%
^{\dagger }\mathbf{\beta }}{\sqrt{\sigma ^{2}\mathbf{\lambda }^{\dagger
}\left( \mathbf{X}^{\dagger }\mathbf{X}\right) ^{-1}\mathbf{\lambda }}}\sim
N\left( 0,1\right)
\end{equation*}%
\begin{eqnarray*}
t &=&\frac{\mathbf{\lambda }^{\dagger }\hat{\beta}\mathbf{-\lambda }%
^{\dagger }\hat{\beta}}{\sqrt{\text{MSE }\mathbf{\lambda }^{\dagger }\left( 
\mathbf{X}^{\dagger }\mathbf{X}\right) ^{-1}\mathbf{\lambda }}},\quad \text{%
MSE}=\frac{\mathbf{Y}^{\dagger }\left( I-P_{\mathbf{X}}\right) \mathbf{Y}}{%
n-r} \\
&=&\frac{\frac{\mathbf{\lambda }^{\dagger }\mathbf{\hat{\beta}-\lambda }%
^{\dagger }\mathbf{\beta }}{\sqrt{\sigma ^{2}\mathbf{\lambda }^{\dagger
}\left( \mathbf{X}^{\dagger }\mathbf{X}\right) ^{-1}\mathbf{\lambda }}}}{%
\sqrt{\frac{\mathbf{Y}^{\dagger }\left( I-P_{\mathbf{X}}\right) \mathbf{Y}%
/\sigma ^{2}}{n-r}}}=\frac{Z}{\sqrt{\frac{\chi _{n-r}^{2}}{n-r}}}
\end{eqnarray*}

\bigskip

note:%
\begin{eqnarray*}
\mathbf{\lambda }^{\dagger }\mathbf{\hat{\beta}} &=&\mathbf{\lambda }%
^{\dagger }\left( \mathbf{X}^{\dagger }\mathbf{X}\right) ^{-1}\mathbf{X}%
^{\dagger }\mathbf{Y} \\
&=&\mathbf{a}^{\dagger }\mathbf{X}\left( \mathbf{X}^{\dagger }\mathbf{X}%
\right) ^{-1}\mathbf{X}^{\dagger }\mathbf{Y}=\mathbf{a}^{\dagger }P_{\mathbf{%
X}}\mathbf{Y}
\end{eqnarray*}%
\begin{eqnarray*}
&\because &\mathbf{a}^{\dagger }P_{\mathbf{X}}\sigma ^{2}I\left( I-P_{%
\mathbf{X}}\right) =0 \\
&\therefore &\mathbf{\lambda }^{\dagger }\mathbf{\hat{\beta}}\text{ and }%
\mathbf{Y}^{\dagger }\left( I-P_{\mathbf{X}}\right) \mathbf{Y}\text{ are
indep}
\end{eqnarray*}%
\begin{equation*}
P_{r}\left\{ -t_{n-r,\alpha /2}\leq \frac{\mathbf{\lambda }^{\dagger }%
\mathbf{\hat{\beta}-\lambda }^{\dagger }\mathbf{\beta }}{\sqrt{\text{MSE }%
\mathbf{\lambda }^{\dagger }\left( \mathbf{X}^{\dagger }\mathbf{X}\right)
^{-1}\mathbf{\lambda }}}\leq t_{n-r,\alpha /2}\right\} =1-\alpha 
\end{equation*}%
\newline
\newline
$\Rightarrow \mathbf{\lambda }^{\dagger }\mathbf{\hat{\beta}}\pm
t_{n-r,\alpha /2}\sqrt{\text{MSE }\mathbf{\lambda }^{\dagger }\left( \mathbf{%
X}^{\dagger }\mathbf{X}\right) ^{-1}\mathbf{\lambda }}$ is a $100\left(
1-\alpha \right) $ percent C.I for $\mathbf{\lambda }^{\dagger }\mathbf{%
\beta }$.\newline
\newline

$\lambda =\left[ 0,0,\cdots ,1,0,0,\cdots \right] \quad g_{jj}$

\bigskip 

\paragraph{Prediction Interval}

\begin{equation*}
\begin{tabular}{l}
confidence interval: for parameter $\left( E\left( \mathbf{Y}\right) =%
\mathbf{X\beta }\right) $ \\ 
prediction interval: for random variable $\left( y_{0}=\mathbf{X}%
_{0}^{\dagger }\mathbf{\beta }+\varepsilon _{0}\right) $%
\end{tabular}%
\end{equation*}%
\newline
We predict $y_{0}$ by $\hat{y}_{0}=\mathbf{X}_{0}^{\dagger }\mathbf{\hat{%
\beta}}$, $\hat{y}_{0}$ and $y_{0}$ are indep (why)%
\begin{eqnarray*}
Var\left( y_{0}-\hat{y}_{0}\right) &=&Var\left( y_{0}-\mathbf{X}%
_{0}^{\dagger }\mathbf{\hat{\beta}}\right) \\
&=&Var\left( \mathbf{X}_{0}^{\dagger }\mathbf{\beta }+\varepsilon _{0}-%
\mathbf{X}_{0}^{\dagger }\mathbf{\hat{\beta}}\right) \\
&=&Var\left( \varepsilon _{0}\right) -Var\left( \mathbf{X}_{0}^{\dagger }%
\mathbf{\hat{\beta}}\right) \\
&=&\sigma ^{2}+\mathbf{X}_{0}^{\dagger }Var\left( \mathbf{\hat{\beta}}%
\right) \mathbf{X}_{0} \\
&=&\sigma ^{2}\left( 1+\mathbf{X}_{0}^{\dagger }\left( \mathbf{X}^{\dagger }%
\mathbf{X}\right) ^{-1}\mathbf{X}_{0}\right)
\end{eqnarray*}%
\begin{eqnarray*}
&&\frac{\left( y_{0}-\hat{y}_{0}\right) -E\left( y_{0}-\hat{y}_{0}\right) }{%
\sqrt{Var\left( y_{0}-\hat{y}_{0}\right) }} \\
&=&\frac{y_{0}-\hat{y}_{0}}{\sqrt{\sigma ^{2}\left( 1+\mathbf{X}%
_{0}^{\dagger }\left( \mathbf{X}^{\dagger }\mathbf{X}\right) ^{-1}\mathbf{X}%
_{0}\right) }}\sim Z
\end{eqnarray*}%
\begin{equation*}
\frac{y_{0}-\hat{y}_{0}}{\sqrt{\text{MSE}\left( 1+\mathbf{X}_{0}^{\dagger
}\left( \mathbf{X}^{\dagger }\mathbf{X}\right) ^{-1}\mathbf{X}_{0}\right) }}%
\sim t_{n-r}
\end{equation*}%
$\therefore \left( 1-\alpha \right) \%$ P.I for $y_{0}$ is%
\begin{equation*}
\hat{y}_{0}\pm t_{n-r,\alpha /2}\sqrt{\text{MSE}\left( 1+\mathbf{X}%
_{0}^{\dagger }\left( \mathbf{X}^{\dagger }\mathbf{X}\right) ^{-1}\mathbf{X}%
_{0}\right) }
\end{equation*}

\bigskip

\paragraph{Confidence Interval for $\protect\sigma ^{2}$}

\bigskip 

\begin{equation*}
\because \frac{\mathbf{Y}^{\dagger }\left( I-P_{\mathbf{X}}\right) \mathbf{Y}%
}{\sigma ^{2}}\sim \chi _{n-r}^{2}
\end{equation*}%
\begin{eqnarray*}
&\Rightarrow &\Pr \left\{ \chi _{n-r,1-\frac{\alpha }{2}}^{2}\leq \frac{%
\mathbf{Y}^{\dagger }\left( I-P_{\mathbf{X}}\right) \mathbf{Y}}{\sigma ^{2}}%
\leq \chi _{n-r,\frac{\alpha }{2}}^{2}\right\} =1-\alpha \\
&\Rightarrow &\Pr \left\{ \chi _{n-r,1-\frac{\alpha }{2}}^{2}\leq \frac{%
\text{MSE}\times \left( n-r\right) }{\sigma ^{2}}\leq \chi _{n-r,\frac{%
\alpha }{2}}^{2}\right\} =1-\alpha
\end{eqnarray*}%
\begin{equation*}
\left[ \frac{\left( n-r\right) \times \text{MSE}}{\chi _{n-r,\frac{\alpha }{2%
}}^{2}},\frac{\left( n-r\right) \times \text{MSE}}{\chi _{n-r,1-\frac{\alpha 
}{2}}^{2}}\right] \text{ is the }\left( 1-\alpha \right) \%\text{ C.I for }%
\sigma ^{2}
\end{equation*}

\end{document}
