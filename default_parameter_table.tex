\documentclass[12pt]{standalone}
%
%\documentclass{article}
%\usepackage{fontspec}
\usepackage{xeCJK}
\setmainfont{Times New Roman}
\setsansfont{Verdana}
\setmonofont{Courier New}
\setCJKmainfont{微軟正黑體}
%
\begin{document}

%\begin{table}[hb]
%\centering
\footnotesize
\begin{tabular}{| p{0.4\textwidth} | p{0.5\textwidth} |}

  %\multicolumn{2}{|c|}{RGCordTransV11.py} \\
   \hline
   parameter & description \\%\\ \cline{2-2}
   \hline
    M = 0.3 & mass of top in kg \\
    R = 0.025&default top is a disk with radius R in meters, unless otherwise specified in moment of inertia below\\
    L = 0.005&width of disk in m\\
    arm = 0.01&location of center of mass of top from origin in meters\\
    counter weight = 0.05 & mass that doesn't spin along symmetry axis, e.g. the gimble support part\\
    counter weight location from origin = 0.1&location of counter weight from origin\\
    Ix,Iy,Iz & one can set moment of inertia to overwrite the default. The default is calculated from above disk's parameters, see document\\    
    g = 9.8&gravity constant $m/s^2$\\
    freq = 20&top revolution speed in hertz, along symmetric axis\\
    tn=1.3&end of simulation time\\
    t0=0.0&start of time zero\\
    samplerate = 2000&rate of iteration in Hz\\
    classical case = 1&selection of four typical nutation and precession motions: 1,2,3,4\\
    orien = np.array([-np.pi/3,0,0])&starting orientation vector of top from lab xyz\\
   \hline
   
\end{tabular}
%\end{table}


\end{document}
