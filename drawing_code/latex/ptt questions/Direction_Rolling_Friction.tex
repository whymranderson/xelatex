
\documentclass{article}
%%%%%%%%%%%%%%%%%%%%%%%%%%%%%%%%%%%%%%%%%%%%%%%%%%%%%%%%%%%%%%%%%%%%%%%%%%%%%%%%%%%%%%%%%%%%%%%%%%%%%%%%%%%%%%%%%%%%%%%%%%%%%%%%%%%%%%%%%%%%%%%%%%%%%%%%%%%%%%%%%%%%%%%%%%%%%%%%%%%%%%%%%%%%%%%%%%%%%%%%%%%%%%%%%%%%%%%%%%%%%%%%%%%%%%%%%%%%%%%%%%%%%%%%%%%%
\usepackage{geometry}
\usepackage{graphicx}

%TCIDATA{OutputFilter=LATEX.DLL}
%TCIDATA{Version=5.00.0.2606}
%TCIDATA{<META NAME="SaveForMode" CONTENT="1">}
%TCIDATA{BibliographyScheme=Manual}
%TCIDATA{Created=Wednesday, April 06, 2016 11:13:15}
%TCIDATA{LastRevised=Wednesday, April 06, 2016 11:30:39}
%TCIDATA{<META NAME="GraphicsSave" CONTENT="32">}
%TCIDATA{<META NAME="DocumentShell" CONTENT="Standard LaTeX\Blank - Standard LaTeX Article">}
%TCIDATA{CSTFile=40 LaTeX article.cst}

\newtheorem{theorem}{Theorem}
\newtheorem{acknowledgement}[theorem]{Acknowledgement}
\newtheorem{algorithm}[theorem]{Algorithm}
\newtheorem{axiom}[theorem]{Axiom}
\newtheorem{case}[theorem]{Case}
\newtheorem{claim}[theorem]{Claim}
\newtheorem{conclusion}[theorem]{Conclusion}
\newtheorem{condition}[theorem]{Condition}
\newtheorem{conjecture}[theorem]{Conjecture}
\newtheorem{corollary}[theorem]{Corollary}
\newtheorem{criterion}[theorem]{Criterion}
\newtheorem{definition}[theorem]{Definition}
\newtheorem{example}[theorem]{Example}
\newtheorem{exercise}[theorem]{Exercise}
\newtheorem{lemma}[theorem]{Lemma}
\newtheorem{notation}[theorem]{Notation}
\newtheorem{problem}[theorem]{Problem}
\newtheorem{proposition}[theorem]{Proposition}
\newtheorem{remark}[theorem]{Remark}
\newtheorem{solution}[theorem]{Solution}
\newtheorem{summary}[theorem]{Summary}
\newenvironment{proof}[1][Proof]{\noindent\textbf{#1.} }{\ \rule{0.5em}{0.5em}}
\input{tcilatex}

\begin{document}


\part{The direction of rolling friction w/o slippling}

How to determine the direction of friction force acting on a rolling object?
This is important and is ensential to solving the dynamics of rolling
motions.

\begin{case}
Round object freely rolling down the hill\newline
\newline
The only force that makes the object rotate is friction so friction has to
go up the hill. This friction force is exerted on the wheel by the slope.%
\FRAME{fthF}{3.0398in}{3.0398in}{0in}{}{}{whatever.png}{\special{language
"Scientific Word";type "GRAPHIC";maintain-aspect-ratio TRUE;display
"USEDEF";valid_file "F";width 3.0398in;height 3.0398in;depth
0in;original-width 6.0001in;original-height 6.0001in;cropleft "0";croptop
"1";cropright "1";cropbottom "0";filename
'../../../../Scripts/cordtrans/cases_fig_only/whatever.png';file-properties
"XNPEU";}}
\end{case}

\begin{case}
Object is forced to roll up the hill initially but external force is removed
once the object is going upward. We are considering the later part of the
motion when the external force is removed, so only gravitation is in place.
The wheel is still rolling up the hill.\newline
\newline
The rotation of the object slows down as it climbs up the hill. Friction is
the only force that produces a torque to slow down the rotation. So it needs
to go against the rotating direction. So the friction force acting on the
wheel is up the hill.
\end{case}

\begin{case}
\end{case}

\end{document}
