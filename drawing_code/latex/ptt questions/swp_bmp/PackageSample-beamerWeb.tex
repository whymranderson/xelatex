
\documentclass[notes=show,beamer]{beamer}
%%%%%%%%%%%%%%%%%%%%%%%%%%%%%%%%%%%%%%%%%%%%%%%%%%%%%%%%%%%%%%%%%%%%%%%%%%%%%%%%%%%%%%%%%%%%%%%%%%%%%%%%%%%%%%%%%%%%%%%%%%%%%%%%%%%%%%%%%%%%%%%%%%%%%%%%%%%%%%%%%%%%%%%%%%%%%%%%%%%%%%%%%%%%%%%%%%%%%%%%%%%%%%%%%%%%%%%%%%%%%%%%%%%%%%%%%%%%%%%%%%%%%%%%%%%%
\usepackage{mathpazo}
\usepackage{hyperref}
\usepackage{multimedia}

%TCIDATA{OutputFilter=LATEX.DLL}
%TCIDATA{Version=5.50.0.2953}
%TCIDATA{<META NAME="SaveForMode" CONTENT="1">}
%TCIDATA{BibliographyScheme=Manual}
%TCIDATA{Created=Monday, March 27, 2006 16:48:17}
%TCIDATA{LastRevised=Monday, November 27, 2006 13:01:24}
%TCIDATA{<META NAME="GraphicsSave" CONTENT="32">}
%TCIDATA{<META NAME="DocumentShell" CONTENT="Other Documents\SW\Slides - Beamer">}
%TCIDATA{Language=American English}
%TCIDATA{CSTFile=beamer.cst}

\newenvironment{stepenumerate}{\begin{enumerate}[<+->]}{\end{enumerate}}
\newenvironment{stepitemize}{\begin{itemize}[<+->]}{\end{itemize} }
\newenvironment{stepenumeratewithalert}{\begin{enumerate}[<+-| alert@+>]}{\end{enumerate}}
\newenvironment{stepitemizewithalert}{\begin{itemize}[<+-| alert@+>]}{\end{itemize} }
\usetheme{Madrid}
\input{tcilatex}
\begin{document}

\title[]{Creating Beamer presentations in \textsl{Scientific WorkPlace }and 
\textsl{Scientific Word}}
\author{MacKichan Software Technical Support}
\date{March 2006}
\maketitle

\section{Creating Beamer presentations in Scientific WorkPlace and
Scientific Word}

\subsection{What is Beamer?}

%TCIMACRO{\TeXButton{BeginFrame}{\begin{frame}}}%
%BeginExpansion
\begin{frame}%
%EndExpansion

\QTR{frametitle}{What is Beamer?}

\begin{itemize}
\item Beamer is a 
%TCIMACRO{\TeXButton{LaTeX}{\LaTeX{}} }%
%BeginExpansion
\LaTeX{}
%EndExpansion
document class that produces presentations and transparency slides.

\item Beamer presentations feature

\begin{itemize}
\item \textsc{pdf}%
%TCIMACRO{\TeXButton{LaTeX}{\LaTeX{}} }%
%BeginExpansion
\LaTeX{}
%EndExpansion
output.

\item Global and local control of layout, color, and fonts.

\item List items that can appear one at a time.

\item Overlays and dynamic transitions between slides.

\item Standard 
%TCIMACRO{\TeXButton{LaTeX}{\LaTeX{}} }%
%BeginExpansion
\LaTeX{}
%EndExpansion
constructs.

\item Typeset text, mathematics like this $\frac{-b\pm \sqrt{b^{2}-4ac}}{2a}$%
, and graphics.
\end{itemize}
\end{itemize}

\FRAME{dtbpF}{0.2888in}{0.3105in}{0pt}{}{}{logo.wmf}{\special{language
"Scientific Word";type "GRAPHIC";maintain-aspect-ratio TRUE;display
"PICT";valid_file "F";width 0.2888in;height 0.3105in;depth
0pt;original-width 0.2603in;original-height 0.2811in;cropleft "0";croptop
"1";cropright "1";cropbottom "0";filename
'graphics/logo.wmf';file-properties "XNPEU";}}

%TCIMACRO{\TeXButton{Transition: Box Out}{\transboxout}}%
%BeginExpansion
\transboxout%
%EndExpansion
%TCIMACRO{\TeXButton{EndFrame}{\end{frame}}}%
%BeginExpansion
\end{frame}%
%EndExpansion

\subsection{Basic tasks in creating a Beamer presentation}

%TCIMACRO{\TeXButton{BeginFrame}{\begin{frame}}}%
%BeginExpansion
\begin{frame}%
%EndExpansion

\QTR{frametitle}{Basic tasks in creating a Beamer presentation}

\begin{enumerate}
\item Start a new document with the Slides - Beamer shell in the Other
Documents shell directory.

\item Choose a presentation theme to define the appearance of the
presentation.

\item Create a frame for each slide in the presentation.

\item Organize information in lists and columns.

\item Create transitions between frames.

\item Add graphics and animations.

\item Typeset the presentation with \textsc{pdf}\LaTeX{}.
\end{enumerate}

%TCIMACRO{\TeXButton{Transition: Box Out}{\transboxout}}%
%BeginExpansion
\transboxout%
%EndExpansion
%TCIMACRO{\TeXButton{EndFrame}{\end{frame}}}%
%BeginExpansion
\end{frame}%
%EndExpansion

\subsection{Using presentation themes}

%TCIMACRO{\TeXButton{BeginFrame}{\begin{frame}}}%
%BeginExpansion
\begin{frame}%
%EndExpansion

\QTR{frametitle}{Using presentation themes}

\begin{stepitemize}
\item Beamer \textit{presentation themes} automatically define all aspects
of a presentation's layout and appearance.

\item Use a presentation theme to format your presentation quickly.

\item Beamer has five types of presentation themes:

\begin{itemize}
\item 
%TCIMACRO{\qhyperref{}{}{nav}{Without navigation bars}}%
%BeginExpansion
\hyperref{}{}{nav}{Without navigation bars}%
%EndExpansion

\item 
%TCIMACRO{\qhyperref{}{}{tree}{With tree-like navigation bars}}%
%BeginExpansion
\hyperref{}{}{tree}{With tree-like navigation bars}%
%EndExpansion

\item 
%TCIMACRO{\qhyperref{}{}{toc}{With a table of contents sidebar}}%
%BeginExpansion
\hyperref{}{}{toc}{With a table of contents sidebar}%
%EndExpansion

\item 
%TCIMACRO{\qhyperref{}{}{mini}{With a mini-frame navigation bar}}%
%BeginExpansion
\hyperref{}{}{mini}{With a mini-frame navigation bar}%
%EndExpansion

\item \hyperlink{sec}{With section and subsection tables}
\end{itemize}

\item This sample presentation uses the Madrid theme, which is without
navigation bars.

\item Sample any of the themes by adding the command \texttt{\TEXTsymbol{%
\backslash}usetheme\{themename\}} to the preamble of this document,
replacing any existing \texttt{\TEXTsymbol{\backslash}usetheme} command.
\end{stepitemize}

%TCIMACRO{\TeXButton{Transition: Box Out}{\transboxout}}%
%BeginExpansion
\transboxout%
%EndExpansion
%TCIMACRO{\TeXButton{EndFrame}{\end{frame}}}%
%BeginExpansion
\end{frame}%
%EndExpansion

%TCIMACRO{\TeXButton{BeginFrame}{\begin{frame}}}%
%BeginExpansion
\begin{frame}%
%EndExpansion

\hypertarget{nav}{}\QTR{frametitle}{Using presentation themes}

\QTR{framesubtitle}{Themes without navigation bars}

\begin{itemize}
\item \textbf{Default}: a professional, no-nonsense theme

\item \textbf{Bergen}: divides the frame into vertical boxes

\item \textbf{Boadilla}: gives much information in little space

\item \textbf{Madrid}: similar to Boadilla but with fancier item icons

\item \textbf{Pittsburgh}: professional, with right-flush titles

\item \textbf{Rochester}: a professional theme with a horizontal header panel
\end{itemize}

%TCIMACRO{\TeXButton{Transition: Box Out}{\transboxout}}%
%BeginExpansion
\transboxout%
%EndExpansion
%TCIMACRO{\TeXButton{EndFrame}{\end{frame}}}%
%BeginExpansion
\end{frame}%
%EndExpansion

%TCIMACRO{\TeXButton{BeginFrame}{\begin{frame}}}%
%BeginExpansion
\begin{frame}%
%EndExpansion

\hypertarget{tree}{}\QTR{frametitle}{Using presentation themes}

\QTR{framesubtitle}{Themes that display a tree-like navigation bar}

\begin{itemize}
\item \textbf{Antibes}: a theme with a strong appearance, with a navigation
bar at the top

\item \textbf{JuanLesPins}: a smoother, rounder version of Antibes

\item \textbf{Montpellier}: professional, with basic navigational hints at
the bottom
\end{itemize}

%TCIMACRO{\TeXButton{Transition: Box Out}{\transboxout}}%
%BeginExpansion
\transboxout%
%EndExpansion
%TCIMACRO{\TeXButton{EndFrame}{\end{frame}}}%
%BeginExpansion
\end{frame}%
%EndExpansion

%TCIMACRO{\TeXButton{BeginFrame}{\begin{frame}}}%
%BeginExpansion
\begin{frame}%
%EndExpansion

\hypertarget{toc}{}\QTR{frametitle}{Using presentation themes}

\QTR{framesubtitle}{Themes with a table of contents sidebar}

\begin{itemize}
\item \textbf{Berkeley}: a professional theme with a with a horizontal
header panel

\item \textbf{PaloAlto}: a less forceful form of Berkeley

\item \textbf{Goettingen}: with a full table of contents on the right and no
header panel

\item \textbf{Marburg}: a very strong variation of Goettingen

\item \textbf{Hannover}: TOC sidebar on the left balances right-flush titles
\end{itemize}

%TCIMACRO{\TeXButton{Transition: Box Out}{\transboxout}}%
%BeginExpansion
\transboxout%
%EndExpansion
%TCIMACRO{\TeXButton{EndFrame}{\end{frame}}}%
%BeginExpansion
\end{frame}%
%EndExpansion

%TCIMACRO{\TeXButton{BeginFrame}{\begin{frame}}}%
%BeginExpansion
\begin{frame}%
%EndExpansion

\hypertarget{mini}{}\QTR{frametitle}{Using presentation themes}

\QTR{framesubtitle}{Themes with a mini-frame navigation bar}

\begin{itemize}
\item \textbf{Berlin}: a theme with strong rectangular areas and a
navigation\ bar at the top

\item \textbf{Ilmenau}: a variation of Berlin

\item \textbf{Dresden}: a variation of Ilmenau

\item \textbf{Darmstadt}: similar to Dresden with rounded theorem boxes and
icons

\item \textbf{Frankfurt}: like Darmstadt but without subsection information

\item \textbf{Singapore}: a softer appearance; divides headings from text
with shading

\item \textbf{Szeged}: softer, with strong horizontal lines
\end{itemize}

%TCIMACRO{\TeXButton{Transition: Box Out}{\transboxout}}%
%BeginExpansion
\transboxout%
%EndExpansion
%TCIMACRO{\TeXButton{EndFrame}{\end{frame}}}%
%BeginExpansion
\end{frame}%
%EndExpansion

%TCIMACRO{\TeXButton{BeginFrame}{\begin{frame}}}%
%BeginExpansion
\begin{frame}%
%EndExpansion

\hypertarget{sec}{}\QTR{frametitle}{Using presentation themes}

\QTR{framesubtitle}{Themes with section and subsection tables}

\begin{itemize}
\item \textbf{Copenhagen}: shows current section and subsection at top,
title and author at bottom; no shadows

\item \textbf{Luebeck}: a boxier variation of Copenhagen

\item \textbf{Malmoe}: a more professional variation of Copenhagen

\item \textbf{Warsaw}: a variation of Copenhagen, with a strong appearance
\end{itemize}

%TCIMACRO{\TeXButton{Transition: Box Out}{\transboxout}}%
%BeginExpansion
\transboxout%
%EndExpansion
%TCIMACRO{\TeXButton{EndFrame}{\end{frame}}}%
%BeginExpansion
\end{frame}%
%EndExpansion

%TCIMACRO{\TeXButton{BeginFrame}{\begin{frame}}}%
%BeginExpansion
\begin{frame}%
%EndExpansion

\QTR{frametitle}{Using outer and inner themes}

Instead of using complete presentation themes, you can define presentation
elements separately.

\begin{itemize}
\item Use \textbf{Outer themes }to define the overall frame layout, borders,
headers, sidebars, footers, and navigation bars.

\item Use \textbf{Inner themes }to define the appearance of title pages,
lists, blocks of text, theorems and proofs, figures, tables, footnotes, and
bibliography entries.

\item For full information, see
TCITeX/TeX/LaTeX/contrib/beamer/doc/beameruserguide.pdf provided as part of
the downloaded support.
\end{itemize}

%TCIMACRO{\TeXButton{Transition: Box Out}{\transboxout}}%
%BeginExpansion
\transboxout%
%EndExpansion
%TCIMACRO{\TeXButton{EndFrame}{\end{frame}}}%
%BeginExpansion
\end{frame}%
%EndExpansion

\subsection{Creating frames}

%TCIMACRO{\TeXButton{BeginFrame}{\begin{frame}}}%
%BeginExpansion
\begin{frame}%
%EndExpansion

\QTR{frametitle}{Creating frames}

\begin{stepitemize}
\item All the information in a Beamer\emph{\ }presentation is contained in 
\textit{frames.}

\item Each frame corresponds to a single presentation slide.

\item To create frames in a Beamer document,

\begin{stepenumerate}
\item Apply a frame fragment:

\begin{stepitemize}
\item The \textbf{Frame with title and subtitle} fragment starts and ends a
new frame and includes a title and subtitle.

\item The \textbf{Frame with title }fragment starts and ends a new frame and
includes a title.

\item The \textbf{Frame} fragment starts and ends a new frame.
\end{stepitemize}

\item Place the text for the frame between the BeginFrame and EndFrame
fields.

\item Enter the frame title and subtitle.

If you used the Frame fragment, apply the Frame title and Frame subtitle
text tags as necessary.
\end{stepenumerate}
\end{stepitemize}

%TCIMACRO{\TeXButton{Transition: Box Out}{\transboxout}}%
%BeginExpansion
\transboxout%
%EndExpansion
%TCIMACRO{\TeXButton{EndFrame}{\end{frame}}}%
%BeginExpansion
\end{frame}%
%EndExpansion

\subsection{Suppressing frame headlines, footlines, and sidebars}

%TCIMACRO{\TeXButton{BeginFrame[plain]}{\begin{frame}[plain]}}%
%BeginExpansion
\begin{frame}[plain]%
%EndExpansion

\QTR{frametitle}{Suppressing frame headlines, footlines, and sidebars}

\begin{itemize}
\item Having a frame without the usual frame elements is useful:

\begin{itemize}
\item The text area is slightly larger.

\item Large graphics may look nicer.
\end{itemize}

\item To create a frame that suppresses the headlines, footlines, and
sidebars,

\begin{enumerate}
\item Apply the BeginFrame fragment to start a frame.

\item Double-click the fragment to open the TeX field.

\item In the entry area, place the insertion point at the end of the 
\TEXTsymbol{\backslash}begin\{frame\} command and type \textbf{[plain]}.

\item Choose \textbf{OK}.

\item Apply the EndFrame fragment to end the frame.

\item Place the text for the frame between the two fragments.
\end{enumerate}

\item To illustrate, the headline, footline, and sidebars have been
suppressed in this frame.
\end{itemize}

%TCIMACRO{\TeXButton{EndFrame}{\end{frame}}}%
%BeginExpansion
\end{frame}%
%EndExpansion

\subsection{Breaking frames automatically}

%TCIMACRO{%
%\TeXButton{BeginFrame[allowframebreaks]}{\begin{frame}[allowframebreaks]}}%
%BeginExpansion
\begin{frame}[allowframebreaks]%
%EndExpansion

\QTR{frametitle}{Breaking frames automatically}

\begin{itemize}
\item When information extends beyond the boundaries of a single slide, you
can use a Beamer option to automatically create additional slides within the
frame.

\item To break the frame automatically,

\begin{enumerate}
\item Apply the BeginFrame fragment to start a frame.

\item Double-click the fragment to open the TeX field.

\item In the entry area, place the insertion point at the end of the 
\TEXTsymbol{\backslash}begin\{frame\} command and type \textbf{%
[allowframebreaks]}.

\item Choose \textbf{OK}.

\item Apply the EndFrame fragment to end the frame.

\item Place the text for the frame between the two fragments.
\end{enumerate}

\item Beamer modifies the Frame title to indicate which of several slides is
displayed. Here, you see "Breaking frames automatically I".

\item To illustrate automatic frame breaking, this frame contains
information that continues to an additional slide.

\begin{itemize}
\item The BeginFrame fragment for this frame contains [allowframebreaks].

\item No other changes to the document are necessary.

\item Beamer automatically creates as many additional slides as are needed
to contain the information in the frame.

\item Note the designation of a slide number in the Frame title. Now you see
"Breaking frames automatically II".
\end{itemize}
\end{itemize}

%TCIMACRO{\TeXButton{Transition: Box Out}{\transboxout}}%
%BeginExpansion
\transboxout%
%EndExpansion
%TCIMACRO{\TeXButton{EndFrame}{\end{frame}}}%
%BeginExpansion
\end{frame}%
%EndExpansion

\subsection{Working with lists}

%TCIMACRO{\TeXButton{BeginFrame}{\begin{frame}}}%
%BeginExpansion
\begin{frame}%
%EndExpansion

\QTR{frametitle}{Working with lists}

\begin{itemize}
\item Use lists to organize information on slides.

\item Types of lists:

\begin{itemize}
\item Standard Numbered and Bullet lists

\item Step lists

\item Alert step lists
\end{itemize}
\end{itemize}

%TCIMACRO{\TeXButton{Transition: Box Out}{\transboxout}}%
%BeginExpansion
\transboxout%
%EndExpansion
%TCIMACRO{\TeXButton{EndFrame}{\end{frame}}}%
%BeginExpansion
\end{frame}%
%EndExpansion

%TCIMACRO{\TeXButton{BeginFrame}{\begin{frame}}}%
%BeginExpansion
\begin{frame}%
%EndExpansion

\QTR{frametitle}{Working with lists}

\QTR{framesubtitle}{Step lists}

\begin{stepitemize}
\item \textbf{Step lists} enhance presentations by displaying the list items
one at a time.

\item Enter numbered step lists with the Step Numbered List Item tag.

\item Enter bulleted step lists with the Step Bullet List Item tag.
\end{stepitemize}

%TCIMACRO{\TeXButton{Transition: Box Out}{\transboxout}}%
%BeginExpansion
\transboxout%
%EndExpansion
%TCIMACRO{\TeXButton{EndFrame}{\end{frame}}}%
%BeginExpansion
\end{frame}%
%EndExpansion

%TCIMACRO{\TeXButton{BeginFrame}{\begin{frame}}}%
%BeginExpansion
\begin{frame}%
%EndExpansion

\QTR{frametitle}{Working with lists}

\QTR{framesubtitle}{Alert step lists}

\begin{stepitemizewithalert}
\item \textbf{Alert step lists} enhance presentations by highlighting the
most recently displayed item in a step list.

\item Enter numbered alert lists with the Alert Step Numbered List Item tag.

\item Enter bulleted alert lists with the Alert Step Bullet List Item tag.
\end{stepitemizewithalert}

%TCIMACRO{\TeXButton{Transition: Box Out}{\transboxout}}%
%BeginExpansion
\transboxout%
%EndExpansion
%TCIMACRO{\TeXButton{EndFrame}{\end{frame}}}%
%BeginExpansion
\end{frame}%
%EndExpansion

%TCIMACRO{\TeXButton{BeginFrame}{\begin{frame}}}%
%BeginExpansion
\begin{frame}%
%EndExpansion

\QTR{frametitle}{Working with lists}

\QTR{framesubtitle}{Display of nested lists}

\begin{itemize}
\item Instead of stepping through a list item by item, you can effectively
display a top-level list item together with its associated subitems.%
%TCIMACRO{\TeXButton{Pause}{\pause}}%
%BeginExpansion
\pause%
%EndExpansion

\item For each top-level list item,

\begin{enumerate}
\item Create a standard Bullet List or Numbered List Item, with subitems.

\item Place the \textbf{Pause} fragment at the end of the last subitem.
\end{enumerate}

Beamer automatically displays the top-level item and its subitems, then
pauses before displaying the next top-level item.%
%TCIMACRO{\TeXButton{Pause}{\pause}}%
%BeginExpansion
\pause%
%EndExpansion

\item This frame illustrates the use of the Pause fragment to display a
nested list.

\begin{enumerate}
\item First subitem

\item Second subitem%
%TCIMACRO{\TeXButton{Pause}{\pause}}%
%BeginExpansion
\pause%
%EndExpansion
\end{enumerate}

\item Another top-level item

\begin{enumerate}
\item First subitem

\item Final subitem
\end{enumerate}
\end{itemize}

%TCIMACRO{\TeXButton{Transition: Box Out}{\transboxout}}%
%BeginExpansion
\transboxout%
%EndExpansion
%TCIMACRO{\TeXButton{EndFrame}{\end{frame}}}%
%BeginExpansion
\end{frame}%
%EndExpansion

%TCIMACRO{%
%\TeXButton{Change meaning of the Alert Step Bullet list}{\renewenvironment{stepitemizewithalert}{\begin{itemize}[<0-| alert@+>]}{\end{itemize} }}}%
%BeginExpansion
\renewenvironment{stepitemizewithalert}{\begin{itemize}[<0-| alert@+>]}{\end{itemize} }%
%EndExpansion

%TCIMACRO{\TeXButton{BeginFrame}{\begin{frame}}}%
%BeginExpansion
\begin{frame}%
%EndExpansion

\QTR{frametitle}{Working with lists}

\QTR{framesubtitle}{Modification of Alert step lists}

\begin{stepitemizewithalert}
\item You can modify Alert step lists to display all items in the list at
once but highlight the items as you step through the list.

\item You can also modify the lists to change the colors used for list items.

\item This frame displays all the items in this list at the same time using
the default colors.

\item Items are highlighted as you move from one to the next.
\end{stepitemizewithalert}

%TCIMACRO{\TeXButton{EndFrame}{\end{frame}}}%
%BeginExpansion
\end{frame}%
%EndExpansion

%TCIMACRO{%
%\TeXButton{Change meaning of the Alert Step Bullet list}{\renewenvironment{stepitemizewithalert}{\begingroup\begin{itemize}[<0-| alert@+>]
%\setbeamercolor{alerted text}{fg=black}
%\setbeamercolor{unhighlighted text}{fg=black!40}\usebeamercolor[fg]{unhighlighted text}
%}{\end{itemize}\endgroup }}}%
%BeginExpansion
\renewenvironment{stepitemizewithalert}{\begingroup\begin{itemize}[<0-| alert@+>]
\setbeamercolor{alerted text}{fg=black}
\setbeamercolor{unhighlighted text}{fg=black!40}\usebeamercolor[fg]{unhighlighted text}
}{\end{itemize}\endgroup }%
%EndExpansion

%TCIMACRO{\TeXButton{BeginFrame}{\begin{frame}}}%
%BeginExpansion
\begin{frame}%
%EndExpansion

\QTR{frametitle}{Working with lists}

\QTR{framesubtitle}{Modification of Alert step lists}

\begin{stepitemizewithalert}
\item This frame displays all the items in this list at the same time using
a gray scale instead of the default colors.

\item Items are highlighted as you step through the list.

\item The frame displays the highlighted item in black and displays other
items using a 40\% gray scale.

\item A TeX command placed before the BeginFrame fragment specifies the
colors.
\end{stepitemizewithalert}

%TCIMACRO{\TeXButton{EndFrame}{\end{frame}}}%
%BeginExpansion
\end{frame}%
%EndExpansion

%TCIMACRO{%
%\TeXButton{Change meaning of the Alert Step Bullet list}{\renewenvironment{stepitemizewithalert}{\begingroup\begin{itemize}[<0-| alert@+>]
%\setbeamercolor{alerted text}{fg=black}
%\setbeamercolor{unhighlighted text}{fg=black!20}\usebeamercolor[fg]{unhighlighted text}
%}{\end{itemize}\endgroup }}}%
%BeginExpansion
\renewenvironment{stepitemizewithalert}{\begingroup\begin{itemize}[<0-| alert@+>]
\setbeamercolor{alerted text}{fg=black}
\setbeamercolor{unhighlighted text}{fg=black!20}\usebeamercolor[fg]{unhighlighted text}
}{\end{itemize}\endgroup }%
%EndExpansion

%TCIMACRO{\TeXButton{BeginFrame}{\begin{frame}}}%
%BeginExpansion
\begin{frame}%
%EndExpansion

\QTR{frametitle}{Working with lists}

\QTR{framesubtitle}{Modification of Alert step lists}

\begin{stepitemizewithalert}
\item This frame displays all the items in this list at the same time using
a lighter gray scale.

\item Items are highlighted as you step through the list.

\item The frame displays highlighted items in black and displays other items
using a 20\% gray scale. Compare it to the previous frame.

\item A TeX command placed before the BeginFrame fragment specifies the
colors.
\end{stepitemizewithalert}

%TCIMACRO{\TeXButton{EndFrame}{\end{frame}}}%
%BeginExpansion
\end{frame}%
%EndExpansion

%TCIMACRO{%
%\TeXButton{Reset default for Alert Step Bullet list}{\renewenvironment{stepitemizewithalert}{\begin{itemize}[<+-| alert@+>]}{\end{itemize} }}}%
%BeginExpansion
\renewenvironment{stepitemizewithalert}{\begin{itemize}[<+-| alert@+>]}{\end{itemize} }%
%EndExpansion

%TCIMACRO{\TeXButton{BeginFrame}{\begin{frame}[allowframebreaks]}}%
%BeginExpansion
\begin{frame}[allowframebreaks]%
%EndExpansion

\QTR{frametitle}{Working with lists}

\QTR{framesubtitle}{Modification of Alert step lists}

\begin{itemize}
\item You can modify Alert step lists globally or individually.

\item To modify all Alert step lists in your document,

\begin{enumerate}
\item In your document preamble, find the \TEXTsymbol{\backslash}%
newenvironment\{stepenumeratewithalert\} command (for Alert numbered lists)\
or the \TEXTsymbol{\backslash}newenvironment\{stepitemizewithalert\} command
(for Alert bullet lists).

\item Replace the command with the command(s) necessary for the changes you
want to make, and choose \textbf{OK}.
\end{enumerate}

\item To modify individual Alert step lists,

\begin{enumerate}
\item Place the insertion point before the BeginFrame statement for the
frame containing the lists you want to modify.

\item Enter a TeX field and type the command(s) necessary for the changes
you want to make, then choose \textbf{OK}.
\end{enumerate}

\item See the package documentation for detailed information about package
commands. You can also use the commands in this sample document as models.
Remember that when you place commands in the preamble, you must use 
\TEXTsymbol{\backslash}newcommand statements instead of \TEXTsymbol{%
\backslash}renewcommand statements.
\end{itemize}

%TCIMACRO{\TeXButton{EndFrame}{\end{frame}}}%
%BeginExpansion
\end{frame}%
%EndExpansion

\subsection{Adding transitions}

%TCIMACRO{\TeXButton{BeginFrame}{\begin{frame}}}%
%BeginExpansion
\begin{frame}%
%EndExpansion

\QTR{frametitle}{Adding transitions}

\begin{stepitemize}
\item Beamer enhances presentations with dynamic transitions between frames.

\item Apply fragments to create these types of dynamic transitions:

\begin{stepitemize}
\item \textbf{Horizontal and vertical blinds}: Blinds Horizontal, Blinds
Vertical

\item \textbf{Boxes}:\ Box In, Box Out

\item \textbf{Dissolves}: Dissolve, Glitter, Wipe

\item \textbf{Horizontal fades:} Split Horizontal In, Split Horizontal Out

\item \textbf{Vertical fades: }Split Vertical In, Split Vertical Out
\end{stepitemize}

\item Each fragment controls the transition from the frame that precedes it
to the frame that contains it.

\item This sample presentation uses the Box Out transition.
\end{stepitemize}

%TCIMACRO{\TeXButton{Transition: Box Out}{\transboxout}}%
%BeginExpansion
\transboxout%
%EndExpansion
%TCIMACRO{\TeXButton{EndFrame}{\end{frame}}}%
%BeginExpansion
\end{frame}%
%EndExpansion

\subsection{Working with color}

%TCIMACRO{\TeXButton{BeginFrame}{\begin{frame}}}%
%BeginExpansion
\begin{frame}%
%EndExpansion

\QTR{frametitle}{Working with color}

\begin{itemize}
\item Because specifying color for individual elements is complex, use
Beamer \textit{color themes} to define the use of all color in a
presentation:

\begin{itemize}
\item \textbf{Default color theme} has black text, white background, blue
structural elements, alerts in red, examples in dark green.

\item \textbf{Complete color themes }specify all colors for all parts of a
frame. Available color themes: albatross, beetle, crane, dove, fly, seagull,
wolverine, beaver.

\item \textbf{Special purpose color theme }defines colors for sidebars.

\item \textbf{Inner color themes }define colors for inner elements. Named
for flowers.

\item \textbf{Outer color themes} define colors for outer elements. Named
for sea animals.
\end{itemize}

\item This presentation uses the default color theme.

\item To use a different color theme, add the command \texttt{\TEXTsymbol{%
\backslash}usecolortheme\{colorthemename\}} to the preamble, replacing any
existing \texttt{\TEXTsymbol{\backslash}usecolortheme} command.
\end{itemize}

%TCIMACRO{\TeXButton{Transition: Box Out}{\transboxout}}%
%BeginExpansion
\transboxout%
%EndExpansion
%TCIMACRO{\TeXButton{EndFrame}{\end{frame}}}%
%BeginExpansion
\end{frame}%
%EndExpansion

\subsection{Working with fonts}

%TCIMACRO{\TeXButton{BeginFrame}{\begin{frame}}}%
%BeginExpansion
\begin{frame}%
%EndExpansion

\QTR{frametitle}{Working with fonts}

\begin{itemize}
\item Beamer \textit{font themes }define the use of fonts in a presentation:

\begin{itemize}
\item \textbf{Default }font theme uses a sans serif font in various sizes.

\item \textbf{Serif }uses a serif font in various sizes.

\item \textbf{Professional }suppresses Beamer font control in favor of
installed fonts.

\item \textbf{Structurebold, structureitalicserif, }and \textbf{%
structuresmallcapsserif }change the font used for structural elements such
as headers, footers, and sidebars.
\end{itemize}

\item This presentation uses the default font scheme.

\item To use a different font theme, add the command \texttt{\TEXTsymbol{%
\backslash}usefonttheme\{fontthemename\}} to the preamble of your document,
replacing any existing \texttt{\TEXTsymbol{\backslash}usefonttheme} command.
\end{itemize}

%TCIMACRO{\TeXButton{Transition: Box Out}{\transboxout}}%
%BeginExpansion
\transboxout%
%EndExpansion
%TCIMACRO{\TeXButton{EndFrame}{\end{frame}}}%
%BeginExpansion
\end{frame}%
%EndExpansion

\subsection{Using columns}

%TCIMACRO{\TeXButton{BeginFrame}{\begin{frame}}}%
%BeginExpansion
\begin{frame}%
%EndExpansion

\QTR{frametitle}{Using columns}

Beamer supports multiple columns of text.

%TCIMACRO{\TeXButton{BeginColumns}{\begin{columns}[5cm]}}%
%BeginExpansion
\begin{columns}[5cm]%
%EndExpansion
%TCIMACRO{\TeXButton{Column}{\column{5cm}}}%
%BeginExpansion
\column{5cm}%
%EndExpansion

\begin{stepitemizewithalert}
\item To begin columns, apply the BeginColumns fragment.

\item Revise the fragment to set the column width.

\item The default fragment uses a column width of 6 cm.
\end{stepitemizewithalert}

%TCIMACRO{\TeXButton{Column}{\column{5cm}}}%
%BeginExpansion
\column{5cm}%
%EndExpansion

\begin{stepitemizewithalert}
\item To begin later columns, apply and revise the Column fragment.

\item Columns can contain inline graphics and movies.

\item To end the last column, apply the EndColumns fragment.
\end{stepitemizewithalert}

%TCIMACRO{\TeXButton{EndColumns}{\end{columns}}}%
%BeginExpansion
\end{columns}%
%EndExpansion

%TCIMACRO{\TeXButton{Transition: Box Out}{\transboxout}}%
%BeginExpansion
\transboxout%
%EndExpansion
%TCIMACRO{\TeXButton{EndFrame}{\end{frame}}}%
%BeginExpansion
\end{frame}%
%EndExpansion

\subsection{Adding graphics}

%TCIMACRO{\TeXButton{BeginFrame}{\begin{frame}}}%
%BeginExpansion
\begin{frame}%
%EndExpansion

\QTR{frametitle}{Adding graphics}

\begin{itemize}
\item Frames can contain graphics \FRAME{itbpF}{20.875pt}{22.4375pt}{2pt}{}{%
}{logo.wmf}{\special{language "Scientific Word";type
"GRAPHIC";maintain-aspect-ratio TRUE;display "PICT";valid_file "F";width
20.875pt;height 22.4375pt;depth 2pt;original-width 0.2603in;original-height
0.2811in;cropleft "0";croptop "1";cropright "1";cropbottom "0";filename
'graphics/logo.wmf';file-properties "XNPEU";}} and movies.

\item Columns provide support for laying out graphics and text:
\end{itemize}

%TCIMACRO{\TeXButton{BeginColumns}{\begin{columns}[5cm]}}%
%BeginExpansion
\begin{columns}[5cm]%
%EndExpansion

%TCIMACRO{\TeXButton{Column}{\column{5cm}}}%
%BeginExpansion
\column{5cm}%
%EndExpansion

\FRAME{itbpF}{2.4889in}{1.6613in}{0in}{}{}{tubeknot.jpg}{\special{language
"Scientific Word";type "GRAPHIC";maintain-aspect-ratio TRUE;display
"USEDEF";valid_file "F";width 2.4889in;height 1.6613in;depth
0in;original-width 2.4483in;original-height 1.625in;cropleft "0";croptop
"1";cropright "1";cropbottom "0";filename
'graphics/TubeKnot.jpg';file-properties "XNPEU";}}

%TCIMACRO{\TeXButton{Column}{\column{5cm}}}%
%BeginExpansion
\column{5cm}%
%EndExpansion

\begin{theorem}
A picture is worth 1000 words.
\end{theorem}

\begin{proof}
Look to the left.
\end{proof}

%TCIMACRO{\TeXButton{EndColumns}{\end{columns}}}%
%BeginExpansion
\end{columns}%
%EndExpansion

%TCIMACRO{\TeXButton{Transition: Box Out}{\transboxout}}%
%BeginExpansion
\transboxout%
%EndExpansion
%TCIMACRO{\TeXButton{EndFrame}{\end{frame}}}%
%BeginExpansion
\end{frame}%
%EndExpansion

\subsection{Embedding animations}

%TCIMACRO{\TeXButton{BeginFrame}{\begin{frame}}}%
%BeginExpansion
\begin{frame}%
%EndExpansion

\QTR{frametitle}{Embedding animations}

\begin{itemize}
\item Beamer presentations can include graphics and animations:\bigskip
\end{itemize}

%TCIMACRO{\TeXButton{BeginColumns}{\begin{columns}[5cm]}}%
%BeginExpansion
\begin{columns}[5cm]%
%EndExpansion

%TCIMACRO{\TeXButton{Column}{\column{5cm}}}%
%BeginExpansion
\column{5cm}%
%EndExpansion

%TCIMACRO{%
%\TeXButton{Movie}{\movie[palindrome,borderwidth=1pt,width=2in,height=1.5in]{Click me!}{TubeKnot.avi}}}%
%BeginExpansion
\movie[palindrome,borderwidth=1pt,width=2in,height=1.5in]{Click me!}{TubeKnot.avi}%
%EndExpansion

%TCIMACRO{\TeXButton{Column}{\column{5cm}}}%
%BeginExpansion
\column{5cm}%
%EndExpansion

\begin{theorem}
An animation is worth 1,000,000 words.
\end{theorem}

\begin{proof}
Click to the left.
\end{proof}

%TCIMACRO{\TeXButton{EndColumns}{\end{columns}}}%
%BeginExpansion
\end{columns}%
%EndExpansion
\bigskip

{\scriptsize This slide shows an AVI\ file of an animated plot generated in 
\textsl{SWP }and exported from VCAM.}

%TCIMACRO{\TeXButton{Transition: Box Out}{\transboxout}}%
%BeginExpansion
\transboxout%
%EndExpansion
%TCIMACRO{\TeXButton{EndFrame}{\end{frame}}}%
%BeginExpansion
\end{frame}%
%EndExpansion

\subsection{Floating graphics and tables}

%TCIMACRO{\TeXButton{BeginFrame}{\begin{frame}}}%
%BeginExpansion
\begin{frame}%
%EndExpansion

\QTR{frametitle}{Floating graphics and tables}

Usually, floating graphics and tables are automatically labelled as Figure
and Table along with a figure and table number. With Beamer, no number is
printed, since numbers make little sense in a normal presentation. You can
override the default setting to allow automatic numbering. Add the following
to the document preamble:

\texttt{\TEXTsymbol{\backslash}setbeamertemplate\{caption\}[numbered]}

%TCIMACRO{\TeXButton{Transition: Box Out}{\transboxout}}%
%BeginExpansion
\transboxout%
%EndExpansion
%TCIMACRO{\TeXButton{EndFrame}{\end{frame}}}%
%BeginExpansion
\end{frame}%
%EndExpansion

\subsection{Setting class options}

%TCIMACRO{\TeXButton{BeginFrame}{\begin{frame}}}%
%BeginExpansion
\begin{frame}%
%EndExpansion

\QTR{frametitle}{Setting class options}

Use class options to

\begin{itemize}
\item Set the base font size for the presentation.

\item Set text alignment.

\item Set equation numbering.

\item Set print quality.

\item Format displayed equations.

\item Create a presentation, handout, or set of transparencies.

\item Hide or display notes.
\end{itemize}

%TCIMACRO{\TeXButton{Transition: Box Out}{\transboxout}}%
%BeginExpansion
\transboxout%
%EndExpansion
%TCIMACRO{\TeXButton{EndFrame}{\end{frame}}}%
%BeginExpansion
\end{frame}%
%EndExpansion

%TCIMACRO{\TeXButton{BeginFrame}{\begin{frame}}}%
%BeginExpansion
\begin{frame}%
%EndExpansion

\QTR{frametitle}{Setting class options}

\QTR{framesubtitle}{Notes}

\note{Here is a Beamer note.}

\begin{itemize}
\item This sample document is originally supplied with the notes class
option set to Show.

\item This frame contains a note so that you can test the notes options.

\item To see the note, scroll to the next frame
\end{itemize}

%TCIMACRO{\TeXButton{Transition: Box Out}{\transboxout}}%
%BeginExpansion
\transboxout%
%EndExpansion
%TCIMACRO{\TeXButton{EndFrame}{\end{frame}}}%
%BeginExpansion
\end{frame}%
%EndExpansion

\subsection{Learn more about Beamer}

%TCIMACRO{\TeXButton{BeginFrame}{\begin{frame}}}%
%BeginExpansion
\begin{frame}%
%EndExpansion

\QTR{frametitle}{Learn more about Beamer}

\begin{itemize}
\item This\ sample document, the Slides - Beamer shell, and the associated
fragments provide basic support for Beamer in \textsl{SWP }and \textsl{SW}.

\item For complete information, read the Beamer User Guide (a .pdf file):

See TCITeX/TeX/LaTeX/contrib/beamer/doc/beameruserguide.pdf in your program
installation.

\item For support, contact \textbf{support@mackichan.com}.
\end{itemize}

%TCIMACRO{\TeXButton{Transition: Box Out}{\transboxout}}%
%BeginExpansion
\transboxout%
%EndExpansion
%TCIMACRO{\TeXButton{EndFrame}{\end{frame}}}%
%BeginExpansion
\end{frame}%
%EndExpansion

\end{document}
