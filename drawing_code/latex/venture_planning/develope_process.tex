% EPC flow charts
% Author: Fabian Schuh
\documentclass{minimal}
\usepackage{xeCJK} % 分開設置中英文字型
\setCJKmainfont{微軟正黑體} % 設定中文字型

\usepackage{pgf}
\usepackage{tikz}
%%%<
\usepackage{verbatim}
\usepackage[active,tightpage]{preview}
\PreviewEnvironment{tikzpicture}
\setlength\PreviewBorder{5pt}%
%%%>

\begin{comment}
:Title:  EPC flow charts
:Grid: 1x2


\end{comment}
%\usepackage[utf8]{inputenc}
\usetikzlibrary{arrows,automata}
\usetikzlibrary{positioning}


\tikzset{
    state/.style={
           rectangle,
           rounded corners,
           draw=black, very thick,
           minimum height=2em,
           inner sep=2pt,
           text centered,
           },
}


\begin{document}
\tikzstyle{abstract}=[rectangle, draw=black, rounded corners,  anchor=center, text width=3cm,text centered,rectangle split, rectangle split parts=2]

\begin{tikzpicture}[item/.style={rectangle, draw=black, rounded corners,  anchor=center, text width=5cm,text centered,rectangle split, rectangle split parts=2}]

\begin{scope}[node distance=3cm]
\node [item](a) at (1,1){%
            \textbf{更改對應的影響}
            \nodepart{second}
            \begin{enumerate}
            \item 1. 每一個項目的更改都會影響相關項目,影響會傳播,傳播後可能要與原更動項目來回好幾次才會達到最佳的情況,很花時間。
  	 \item 2. 很多解決問題的步驟值得記錄下來並做成網頁放上網站,也很花時間。
  	 \item 3. 已經很多影響傳播做成一鑑完成按鍵,如E->B,A->B,demo舉例。不過還有很多還沒有。
	 \end{enumerate}
};
\node [item,left=of a]   (manual)     {%
            \textbf{B. 技術手冊}
            \nodepart{second}
            \begin{enumerate}
            \item 1. 教學文件【軟體操作說明,程式函式說明書需要一直更新編輯很花時間】
  	 \item 2. 有保持一個程式演進進程文字紀錄,一段時間就要撰寫紀錄也相當搞剛。
  	 \item 3. 文件的3D圖形製作花時間,文件中圖形位置擺放花時間(MSword比LaTeX更花時間!)
  	 \item 4. 實體書DIY製作,一旦有更新,要重作實體書,至今六七本,需非常多列印,列印至少上千張紙,編輯社影印社不會告訴你的秘辛。做出一本書的繁瑣度。更新->校稿->重新列印,反覆發生。
	 \end{enumerate}
 };
\node [item,right=of a]      (extend)    {%
            \textbf{E. 擴充性}
            \nodepart{second}
            \begin{enumerate}
  	 \item 1. 準確度,A法與compedra上的ejss陀螺模擬比較
  	 \item 2. 陀螺環不會停止旋轉的原因
  	 \item 3. 加Unity3D遊戲物件轉動模擬演示、第一人稱飛機遊戲滑鼠控制器演示。(皆為C法的應用)
	 \end{enumerate}
  };
\node [item,above=of a] (animation){
            \textbf{A. 動畫製作}
            \nodepart{second}
            \begin{enumerate}
  	 \item 1. 美觀化?(加光線追蹤)
  	 \item --GUI使用者介面部分--
  	 \item a. OpenGL animation關閉後要手動重啟一個新個interpreter,因為glut的關係,怎麼改進?
  	 \item b. GUI使用者自訂參數 passing in paras 還沒完成
	 \end{enumerate}
        };
\node [item,above left=of a] (beauty){%
            \textbf{美觀性}
            \nodepart{second}
            \begin{enumerate}
            \item 1. ABF項目的美術設計相當花時間
  	 \item 2. 設計需要靈感,需要反覆構圖,需要時間發酵
  	 \item 3. warowl, immortalHD都會遇到這問題
  	 \item --目標--
  	 \item a. spinning cube換成白色加上光影
	 \end{enumerate}
  };
\node [item,above right=of a] (website){%
            \textbf{F. 網站建立}
            \nodepart{second}
            \begin{enumerate}
            \item 1. blogger無讓讀者瀏覽相關文章gadget
  	 \item 2. blogger圖片需一張一張上傳很麻煩,但好處是非常穩定。
  	 \item 3. Django架站,維護,更新都花時間。pythonanywhere只能拉,所以需要所有更改都要先上傳或推至github,然後再從pythonanywhere去github拉檔案。這也花時間。
  	 \item 4. 將A或B的內容製作成網站的圖片或動畫影片很花時間,還要上傳至vimeo,通常都要等很久。OpenGL的動畫製作是存一張張png再用ffmpeg存成mp4,在上傳至vimeo,需要時間。
	 \end{enumerate}
  };
\node [item,below = of a,xshift=-2cm](VC) {%
            \textbf{C. 檔案備份與管理}
            \nodepart{second}
            \begin{enumerate}
            \item 1. git學習很花時間
  	 \item 2. 檔案規劃思考與整理花時間
	 \end{enumerate}
};
\node [item,below=of VC] (hire){
            \textbf{徵夥伴}
            \nodepart{second}
            \begin{enumerate}
            \item 1. 徵有興趣一起完成此套教材的夥伴
	 \end{enumerate}
        };

\node [item,below right=of a,xshift=-2cm](outreach) {%
            \textbf{D. 推廣}
            \nodepart{second}
            \begin{enumerate}
            \item 1. 以B的內容來製作翻轉教學影片,拍影片編輯影片要花九倍時間
  	 \item 2. 下鄉教學的機會找不太到?,職訓?青輔? 台南市內處處碰壁
  	 \item 3. 文宣,廣告(又要列印)
  	 \item 若有回饋,則要來回修正上面許多相關項目(如藍哥回饋三法比較圖,irslab kuku提浮點數誤差累積,此兩項目已完成。)
	 \end{enumerate}
 	};
 
\node [align=center, below=of outreach] (powered) {Powered by TikZ \& PGF \\[-2pt] Modified from Fabian Schuh's EPC flow charts  \\[-2pt] http://www.texample.net/tikz/examples/epc-flow-charts/ };

\end{scope}

\path (animation) 	edge[<->,>=stealth']  node[auto]{BCDF} (a)
	(manual)	edge[<->,>=stealth'] node[auto]{CDF}		(a)
	(outreach)	edge[<->,>=stealth'] node[auto]{ABCF}		(a)
	(website)	edge[<->,>=stealth'] node[auto]{C}		(a)
	(beauty)	edge[<->,>=stealth'] node[auto]{ABDF}		(a)
	(VC)		edge[<->,>=stealth'] node[auto]{ABDEF}		(a)
	(extend)	edge[<->,>=stealth'] node[auto]{ABCDF}		(a);
\end{tikzpicture}

\begin{tikzpicture}[->,>=stealth', node distance=3cm]

 % Position of animation 
\node (animation) [abstract]
        {
            \textbf{動畫製作OpenGL}
            \nodepart{second}
            \begin{enumerate}
            \item 1. 擴充性
  	 \item 2. 平板應用
	 \end{enumerate}
	 % enumerate has issue in tikz, search display mode related? minipage?
        };
        
 % State: ACK with different content
 \node (ACK)  [abstract,below left= of animation, text width = 5cm]
 {%
            \textbf{技術手冊}
            \nodepart{second}
            \begin{enumerate}
            \item A. 教學文件
  	 \item B. 軟體操作手冊
  	 \item C. 程式函式說明書
  	 \item (編輯很花時間)
  	 \item 3D圖形製作,文件中圖形位置擺放花時間(MSword>LaTeX)
	 \end{enumerate}
 };
 
 % STATE animationREP
 \node[state,
  below left =  of ACK,
  anchor=center,
  text width=3cm] (animationREP) 
 {%
 \begin{tabular}{l}
  \textbf{animationRep}\\
  \parbox{2.8cm}{Dekrementiere Slotzähler}
 \end{tabular}
 };

 % STATE EPC
 \node[abstract, below right= of animation] (EPC) 
 {%
            \textbf{網站建立}
            \nodepart{second}
            \begin{enumerate}
            \item 1. blogger無讓讀者瀏覽相關文章gadget
  	 \item 2. 圖片需一張一張上傳很麻煩
  	 \item 3. Django架站花時間,維護,更新
	 \end{enumerate}
  };

 % draw the paths and and print some Text below/above the graph
 \path (animation) 	edge[bend left=20]  node[anchor=south,above]{$SC_n=0$}
                                    node[anchor=north,below]{$RN_{16}$} (ACK)
 (animation)     	edge[bend right=20] node[anchor=south,above]{$SC_n\neq 0$} (animationREP)
 (ACK)       	edge                                                     (EPC)
 (EPC)       	edge[bend left]                                          (animationREP)
 (animationREP)  	edge[loop below]    node[anchor=north,below]{$SC_n\neq 0$} (animationREP)
 (animationREP)  	edge                node[anchor=left,right]{$SC_n = 0$} (ACK);

\end{tikzpicture}


\end{document}