% EPC flow charts
% Author: Fabian Schuh
\documentclass{article}
\usepackage{myflowchart}
\begin{document}

\begin{tikzpicture}

\begin{scope}[node distance=5mm and 5mm]

\node [ wideitem=2](a) at (1,1) {%
            \textbf{未完成方向}
            \nodepart[text width = 16.5cm]{two}
            \begin{enumerate}
            	\item 3rd normalization written down, try implemanting? and how to compare?
           \end{enumerate}
            };


\node [above = of a, align = center] (title){\includegraphics[width=0.3\textwidth]{../../../figs/csgo_db_logo.png}\\ \LARGE GO事業部規劃};
            
\node [ wideitem=2, below = of a](unique_db) {%
            \textbf{資料庫賽程管理特色}
            \nodepart[text width = 16.5cm]{two}
            \begin{enumerate}
            	\item 能有個網頁可以馬上看到想看的結果,不用在上去Liquipedia或HLTV然後要點老半天找老半天,才找的到喜歡的隊伍的賽程。
            	\item 可選擇模組輸入或腳本輸入
            	\begin{enumerate}
            		\item 腳本輸入使用tts\_creation\_template.py檔。此步驟在隊伍的資料庫選擇上並沒有用上賽事單位的隊伍簡寫判斷,而是按照一排好的隊伍list,來做賽程輸入,若資料庫找不到隊伍,代表list中的隊伍名稱找不到對應的資料庫資料。這個狀況時,程式會允許使用者手動輸入正確的隊伍名稱。因此此方法適合比賽不多的賽事,不需要去跑格式判斷程式。編輯隊伍list的方法如下,從官網右鍵複製包含隊伍名稱及比賽時間的賽程表,貼上vim,用vim的快速組合鍵刪除所有不需要的東西,讓每一行剩下一個隊伍名稱,並且相連的兩個隊伍就是對戰的兩個隊伍。然後用我們note中整理的vim快速組合鍵將每個名稱加上"及",讓他變成一個python string list。編輯好後複製後就可以貼進tts\_creation\_template.py檔中。見我網站的db note page。
            		\item 模組輸入,網頁上複製包含該網頁所有隊伍及對戰時間的表格後,貼上一py檔,去跑test parsing.py,會自動分析格式取出隊伍名稱及對戰時間的程式。這對大型賽事較方便。如ESL pro League的巡迴賽,賽事有很多很多,就不適合手動輸入。目前在test\_scraping資料夾。
            	\end{enumerate}
           \end{enumerate}
            };


%\end{scope}
%\end{tikzpicture}
%
%%second page
%\begin{tikzpicture}
%\begin{scope}[node distance=5mm and 5mm]
\node [ wideitem=2, below =of unique_db](unique_editing) {%
            \textbf{網頁編輯特色}
            \nodepart[text width = 16.5cm]{two}
            \begin{enumerate}
            	\item upcoming post已整合成可用文書處理系統LYX編輯,不須處理raw html檔。
            	\item upcoming post目前是網路上少數可查找單一隊伍未來賽事的公開網頁,比HLTV、Liquipedia更方便。
           \end{enumerate}
            };

\end{scope}
\end{tikzpicture}
\newpage

%third page
\begin{tikzpicture}
\begin{scope}[node distance=5mm and 5mm]

	\node [ wideitem=2](gitlog) at (1,1) {%
            \textbf{last 10 git commits}
            \nodepart[text width = 16.5cm]{two}
            \footnotesize \verbatiminput{mysite.log}
            };

\end{scope}
\end{tikzpicture}

\end{document}
