% EPC flow charts
% Author: Fabian Schuh
\documentclass{article}
\usepackage{myflowchart}

\begin{document}

\begin{tikzpicture}

\begin{scope}[node distance=5mm and 5mm]

\node [ item=2](a) at (1,1) {%
            \textbf{2017未完成的較複雜項目}
            \nodepart[text width = 8.2cm]{two}
            \begin{enumerate}
            	\item 車燈gyro儀
             	\item django doc improve $\bullet\bullet\circ$
		\begin{enumerate}
			\item 加新model到full example
			\item 可小修m2m section(extra field那段)
			\item django static file那邊的說明可以再清楚一點?說明django static file的做法是,他希望你把需要伺服的圖檔放在你引用此圖的檔案(如你的html pages)的相同資料夾下,建立一個名為static的資料夾,然後圖檔放在此static資料夾之下(可以在子目錄下)。你說,那我有好多個html檔案都在不同地方,不同資料夾裡面,這些檔案都引用不同的圖檔的話,那不是有一堆static資料夾四散各地?是的,就是這樣。django在執行collectstatic指令時,就是把四散各處的static資料夾以及裡面的需要伺服的圖檔,全部copy到同一個static資料夾下,而此static資料夾會放在我們指定的被伺服的資料夾。你說,那html檔裡的圖檔連結又不是指向這個被伺服的資料夾,對,他會指向之前四散各地的static資料夾連結嘛。但django還會做一個步驟。他會把在html中遇到的圖檔連結,只要遇到有static字串包含在裡面,就會把static前面的超連結字串取代為我們設定的伺服連結。嘿,這樣就可以連到需要的圖檔囉,對吧。劃出例子?這也是為什麼在local資料夾放置圖片時,最好就做好分類(這也是docs說明中為何在連結名稱static folder中會重複分類名稱的原因),這樣django在收集static資料夾時就也會維持同樣的分類。
		\end{enumerate}

            	\item 滾筒洗衣機要清洗
            	\item 印表機level up
		\begin{enumerate}
			\item 等十分鐘熱了再印顏色較深
			\item 圖形有白色條紋,見fidget gyro圖
		\end{enumerate}
		\item 有水電的地面工作區域 $\bullet\bullet\bullet\circ\circ\circ$

                \item vr,gyro體驗館,原力推starwar?
		\item 乾洗液(目前溫水加乾淨布擦)
                \item 好筆的墨水,整理墨水post,入網站。

                \item bank order checks
                \item 英文整理還有訂下一個時間去查

		\item 筆記本的強力夾邊設計及製作,要不會卡其他書以及書包,現在的夾子會卡到然後破壞其他書及書包。
		\item 低收補助寫下文章
	    \end{enumerate}
            };

\node [ item=2, right = of a.north east, anchor = north west](continue) {%
            \textbf{continue..}
            \nodepart[text width = 8.2cm]{two}
            \begin{enumerate}
		\setcounter{enumi}{22}
		\item T60相關
		\begin{enumerate}
			\item T60電池找(型號已找到,4-17-2017 note)
			\item 液晶模組diy post未完成的另一半
		\end{enumerate}

            	\item swp加link要變容易。
            	\item 仿宋體試5/16/17
            	\item science fair typeset continue.
            	\item 特殊的tex學習法寫下,排版是一門專業
		\begin{enumerate}
			\item xeCJK模組改良,自動抓取常用的中文字型,讓初學者可以引用package後就可以使用,不用再去設定字形。這會讓中文化更加方便容易與普及。多數網路例子都是英文編譯模式,中文模式要自己選字形,但怎麼選?中文使用設置要變容易,預設大家都有的幾個字型?learn ifexist isthenesle,check ifexist in moderncv package。
			\item 心法:學tex人會覺得,常常會遇到連一個簡單的問題都會覺得怎麼那麼麻煩,比如說啊我今天要最快速排出一個非常簡單的文件,所以用usepackage{minimal}應該就夠了,我只需要最簡單的排版就好,結果發現一直編譯不過,卡在字形大小tiny出錯。結果居然是在minimal package裡面沒有定義tiny字形大小。心法就是,就像做硬體一樣,要做一精密精細的作品所需要的工具本來就很多很繁瑣,有時你認為簡單的東西,只要有一個小地方你沒有想到,但這小地方其實需要不少工具,那就不能怪工具不好,是你自己沒想到而已。多數人以為做東西很簡單,那是因為他們只會用幾樣很簡單的工具,就想要應付各式各樣的狀況,結果只做得出一些很爛很醜又不耐用的作品。做東西都是如此,忌氣急敗壞。心法。
		\end{enumerate}            	
		\item gui module sphinx doc comment補齊與整理。
		\item 目前matplotlib2pgf text label margin是用annotation,但text method好像有offset還是leftright margin可用?
		


           \end{enumerate}
            };
\end{scope}
\end{tikzpicture}


\begin{tikzpicture}
\begin{scope}[node distance=5mm and 5mm]


\node [ item=2](b) at (1,1) {%
	\textbf{2013未完成項目}
	\nodepart[text width = 8.2cm]{two}
	\begin{enumerate}
		\item 5/30/13 滾動角解析式停住
		\item 8/26/13 嘗試compost加入DIYpost,並加並沒有那麼久
		\item 10/1/13 事實上一點點收入是有達成的,list
			\begin{enumerate}
			\item 2017 3-8 五個月做早餐麵包加餅乾。兩個禮拜做五次,每次食材成本估250,預估可賣三倍價錢,因此賺兩倍,等於淨賺500,每次所花時間約兩小時。等於一小時淨賺250。因此一個月做十次約淨賺5000,五個月大約預估扣掉食材成本賺了兩萬五。
			\item 2016 三個月助理臨時工,兩萬一。python顧問,三千。
			\item 2014-2015 排版一萬八,翻譯一萬。
			\end{enumerate}
		\item 12/15/13, 10/1/13, 想要以高技術賺錢或想以技術養一個家,換個說法就行的通了,想要以高技術來交換以物易物,這是很合理的,因為高技術代表高效率。這是因為現在獲得錢的交換條件不平等。
		\item 12/26/13 值得做coin滾動實驗。
	\end{enumerate}
        };

\node [ item=2, below = of b](2017) {%
	\textbf{2016未完成項目}
	\nodepart[text width = 8.2cm]{two}
	\begin{enumerate}
		\item 2016
	\end{enumerate}
        };

\end{scope}
\end{tikzpicture}

\begin{tikzpicture}
\begin{scope}[node distance=5mm and 5mm]


\node [ item=2](texshop) at (1,1) {%
	\textbf{疊格未完成項目}
	\nodepart[text width = 8.2cm]{two}
		\begin{enumerate}
		\item service page
		
		\begin{enumerate}
		\item flowchart更新
		
		\begin{enumerate}
		\item 圖形的link要修, 1/17/18
		\item 改只放一頁scribd包含所有作品,並以title間頁間隔
		\end{enumerate}
		\end{enumerate}
		\item CV first page font blur problem
		\item GS note TOC add and fix
		\item xeCJK預設字型mac win linux
		\item sandisk git xelatex backup
		\item buy a texlive CD
		\item xelatex資料夾獨立
		
		\begin{enumerate}
		\item get rid of unwanted files using git clone (2 hours estimate?)
		\end{enumerate}
		\item 檢視7/18/17standalone, subfile
		\item 7/17/17前列印flowchart,修gs dev flowchart, fix/craft title, tex flowchart未完成事項
		\item share community: tikz, LYX doc
		
		\begin{enumerate}

		\item math.lyx,pdf,加上空格框,像companion book一樣才會方便閱讀理解。$\bullet\bullet\circ$
		\item before tex compile, hope cmd batch can close the pdf file\\
		\href{http://www.xtremevbtalk.com/general/183310-print-close-pdf-command-line.html}{http://www.xtremevbtalk.com/general/183310-print-close-pdf-command-line.html}
		\end{enumerate}
		\item annotate docs
		
		\begin{enumerate}
		\item alphaphi圖,或fig\_2\_af\_powered.png,加入documentation example。並且把code整理出來。
		\item blogger的3d向量繪圖文整理,整合入documentation後,做成網頁?
		\item 2017/12/4 note, circle\_arc需要加
		
		\begin{enumerate}
		\item 不需要axis向量,axis = np.cross(startv\_n,endv\_n)
		\item add 鈍角選項,ob angle? 
		\end{enumerate}
		\end{enumerate}
		\item tetra\_premise\_1.py 要修成用toolbox
		\item share pstricks and tikz graphs
		\item logo3D OpenGL engrave continue
		\item backup flashdrive
		\end{enumerate}
        };

\node [ item=2, right = of texshop.north east, anchor = north west](webshop) {%
	\textbf{website未完成項目}
	\nodepart[text width = 8.2cm]{two}
		\begin{enumerate}
		\item add events, see 1/21/18$\bullet\bullet\bullet\bullet\circ\circ$
		
		\begin{itemize}
		\item 目前可列出liqui右欄文字,但還須work out a complete solution
		\item apply etree.parse and getiterator and elt.attrib.has\_key('class')
		in\\
		\href{http://infohost.nmt.edu/tcc/help/pubs/pylxml/web/ElementTree-getiterator.html}{http://infohost.nmt.edu/tcc/help/pubs/pylxml/web/ElementTree-getiterator.html}\\
		9. element's attributes\\
		9.8 use get to retrieve attribute value\\
		read til 9.11
		The hope here is we only
		need to feed in the liquipedia page's weblink plus a little confirming
		check.
		\end{itemize}
		\item 7/17/17前GO段入dbnote
		\item GO建立輸入流程圖。
		\item 修沒出現賽程
		\item team logic改重複輸入直到找到
		
		%\subsection{my website}
		\item dad's book upload, then add paypal, when?
		\item try authentication ( may take some time)
		
		\begin{itemize}
		\item add pwd to recipe and workshop\_note
		
		\begin{itemize}
		\item 若用此方法則db file不能從local machine一起git push到production machine。看來得要在server端重建,資料可以在重建後慢慢run
		script加入。但script都在submodule中,若要一起upload則會公開,要公開否?
		\end{itemize}
		\end{itemize}
		\item gyro page make into LYX file
		\item grab\_log.sh automization
		\item create墨水post,找,白板attempt心得
		\item TOC移除section,subsection headings
            	\item {{varljust:''10''}}在html中只算一個空白。網路上可加這個filter,還需查怎麼裝filter,但若裝了,server端pythonanywhere也要裝,花時間。應推薦加入新版django,一勞永逸。
            	\item all coming matches 隊伍重複出現,divisibleby:'2'還有問題
		\item team page轉換成LYX編輯,換banner headline?
		\item team page有過期賽程繼續顯示問題,upcoming page有未顯示所有未來隊伍比賽問題。
		\item db\_note整理,看如何推銷與賣(可能太難,因為不是你的領域,離你太遙遠,無認識的人)。
		\item gitlogs page目前還需要跑grab\_gitlogs.bat去收集並產生文字檔,是否可以自動化,一禮拜做一次?
		\item csgo event scraping continue?
		\item db模型改良,足球聯賽應用。但現在又變成大會定時間,是否可應用上大會定?
           \end{enumerate}

       };



\end{scope}
\end{tikzpicture}

%page 4
\begin{tikzpicture}
\begin{scope}[node distance=5mm and 5mm]


\node [ item=2](toolshop) at (1,1) {%
	\textbf{工具箱未完成項目}
	\nodepart[text width = 8.2cm]{two}
		\subsubsection*{LYX}
		\begin{enumerate}
		\item report community about removing anaconda run into export problem.
		keywords: elyxer location, magick convert, command system encoding.
		1/20/18, 如果另外裝elyxer,記得移除LYX內的elyxer。要不然會有bug。實例。
		\item 9/3/17, 8/2/17(可入service page介紹), 7/13/17後
		\item customization continue
		\end{enumerate}

		\subsubsection*{Python}
		\begin{enumerate}
		\item 2018
		
		\begin{enumerate}
		\item django docs
		
		\begin{enumerate}
		\item how to wrap 72 chars in rebase's editor?
		\item how to wrap 72 chars in gitGUI
		\item how many chars should i wrap in docs' paragraphs?
		\end{enumerate}
		\end{enumerate}
		\item 2017
		
		\begin{enumerate}
		\item 灌spyder? idle
		\item matplotlib border margin mplot3d export png
		
		\begin{enumerate}
		\item stackexchange tick/12 suggestion
		\end{enumerate}
		\item python import
		
		\begin{enumerate}
		\item 查為何sys.append('/下一層')不行,要sys.append('./下一層')。
		\item python的help()有些功能的說明不多,如str的大小比對>或gt\_就沒說明。\\
		以help查module如OpenGL,大小寫有差。\\
		OpenGL的說明不多。\\
		cmd指令pydoc numpy可列出套件的說明行
		\end{enumerate}
		\end{enumerate}
		\end{enumerate}
		
       };

\node [ item=2, right = of toolshop.north east, anchor = north west](bakery) {%
	\textbf{烘焙未完成項目}
	\nodepart[text width = 8.2cm]{two}
		\begin{enumerate}
		\item 食譜待整理2017: 12/3, 11/27, 8/19, 8/14, 7/11, 12/15
		\item 買烤盤
		\item 烤箱製作貼紙
		\end{enumerate}
	};

\node [ item=2, below = of bakery](GSS) {%
	\textbf{GSS未完成項目}
	\nodepart[text width = 8.2cm]{two}
		\begin{enumerate}
		\item unpack修成fix
		\item note若要印A4(如出去印,或letter紙印完),可能需要不少時間調整,可以怎麼準備?5/9/16		
		\item 慢速轉10幾圈,差五度內。快速轉十圈,可差到30度。可能原因見5/10/17。run\_avg有問題?
            	\item load GUI demo setting 1234.
		\item py2exe繼續否?
		\item 滾動碰撞模型建立持續否?是。
		\item note有不少要修改,然後要重印note $\bullet\bullet\bullet\circ\circ\circ$
		\item 單軸角速度計進階版證明。8/16/17
		\item $L,\omega,z$互相繞的想法有錯,make a demo,8/16/17
		\item $\omega,z$ not centered around L check
		\end{enumerate}
	
       };
\node [ item=2, below = of GSS](archi) {%
	\textbf{archived}
	\nodepart[text width = 8.2cm]{two}
            \begin{enumerate}
            	\item 可繼續印以及可以擦掉的printer
		\item 調墨水,加墨水
		\item 看pdf layout,收集。完成度$\bullet\bullet\circ$
            	\item 椅子削平回家教人
                \item 回spyder+manage.py as startupscript問題
            \end{enumerate}
		
       };


\end{scope}
\end{tikzpicture}

%page 5
\begin{tikzpicture}
\begin{scope}[node distance=5mm and 5mm]


\node [ item=4](fancee) at (1,1) {%
	\textbf{F\&C待完成}
            \nodepart{two}
            \begin{enumerate}
            	\item thinkpad T60 電池找
            	\item SMEG烤箱時間溫度表製作
            	\item compressor 排水閥, still no clue, dangerous
		\item 補褲$\bullet\bullet\bullet\circ$
		\item 相機,手機插電
		\item 電繡片問正興老闆
		\item note裝訂強力邊改良
		\item 印表機淡,修,感光鼓,電棒?
		\item 相機接ps
		\item 電動起子接ps
		\item 床墊上4f, call公司
           \end{enumerate}
	   \nodepart{three}\textbf{DIY未完成}
	   \nodepart{four}
           \begin{enumerate}
		\item 手寫板加一硬紙讓能翻頁寫
		\item 單槓-角鋼?啞管?做單槓
		\item gimbal繼續1/12/16
		\item T60電池找
		\item 墨水調
		\item 找cargo crane box
		\item 修compressor,銘山,空壓機排水閥修+換油
		\item Lathe,\protect\href{http://timeway.en.e-cantonfair.com/}{http://timeway.en.e-cantonfair.com/}
           \end{enumerate}
            };
\end{scope}
\end{tikzpicture}

\end{document}
