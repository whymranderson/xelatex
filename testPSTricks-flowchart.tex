%\documentclass[12pt,twoside]{article}
%\usepackage[inner=1in,outer=0.6in,top=0.7in,bottom=1in]{geometry}
%\usepackage{fontspec}
%\usepackage{xeCJK}
%\setmainfont{Times New Roman}
%\setsansfont{Verdana}
%\setmonofont{Courier New}                    % tt
%\setCJKmainfont{微軟正黑體}
%\setCJKfamilyfont{kai}{標楷體}		% for changing the title font in title.pgf -> have to manually 
%\usepackage{pgf}
%\usepackage{pstricks,pst-node}
%\begin{document}

\begin{pspicture}(6in,8in)
%\psgrid

\rput(7,19){
\rnode{A}{
\psframebox[fillcolor=white,fillstyle=solid,framearc=0.3]{
\parbox{2.5cm}{\centering 貼體角速度尤拉方程﹝非線性﹞}}}}

\rput(7,16){
\rnode{B}{
\psframebox[fillcolor=white,fillstyle=solid,framearc=0.3]{
\parbox{3cm}{\centering 尤拉角尤拉方程﹝高度非線性﹞}}}}

\ncline[nodesep=3pt]{->}{A}{B} \trput{代入尤拉角}

\rput(3,16){
\rnode{C}{
\psframebox[fillcolor=white,fillstyle=solid,framearc=0.3]{
\parbox{3.5cm}{
\centering 直接ODE解貼角,求得下一貼角後,建立DCM矩陣遞推得到下一姿態,以此姿態再重複此步驟。}}}}

\nccurve[angleA=180,angleB=90]{->}{A}{C} \nbput[npos=0.5]{ODE solver}

\rput(7,13){
\rnode{D}{
\psframebox[fillcolor=white,fillstyle=solid,framearc=0.3]{
\parbox{2.6cm}{\centering 尤拉角數值解 $\phi,\theta,\psi,\dot{\phi},\dot{\theta},\dot{\psi}$}}}}

\ncline[nodesep=3pt]{->}{B}{D} \trput{\parbox{1cm}{ODE solver}}

\rput(12,13){
\rnode{E}{
\psframebox[fillcolor=white,fillstyle=solid,framearc=0.3]{
\parbox{2cm}{\centering  $\omega_{bx}$}}}}

\ncline[nodesep=3pt]{->}{D}{E} \naput{\parbox{1cm}{轉body frame}}

\rput(12,9){\ovalnode{F}{比較z軸差異}}

\ncdiagg[nodesep=3pt,angleA=-90,angleB=90,armA=2cm]{->}{E}{F}
\ncdiagg[nodesep=3pt,angleA=-90,angleB=135,armA=1.5cm]{->}{D}{F}

\rput(6,9){\ovalnode{G}{比較z軸差異}}

\rput[tl](11,19){
\rnode{note}{
\parbox{4cm}{\centering 從初始值做DCM遞推來作t時間的模擬。這裡雖然$\omega_{b}$有高度非線性造成的較大數值誤差,不過這邊做模擬的時候只是把所有的轉動角堆疊/積分起來,因此$\omega_{b}$的誤差不會放大累積。}}}

\rput[tl](12.5,11.5){
\rnode{star}{$\Large \star$
}}

\nccurve[nodesep=3pt,angleA=280,angleB=45]{->}{note}{star}

\ncdiagg[nodesep=3pt,angleA=-90,angleB=135,armA=1.5cm]{->}{D}{G} \trput{+} \tlput{-}
\ncdiagg[nodesep=3pt,angleA=-90,angleB=135,armA=3.5cm]{->}{C}{G} 

\cnodeput(1,9){H}{+}

\nccurve[nodesep=3pt,angleA=200,angleB=15]{->}{E}{H}
\ncdiagg[nodesep=3pt,angleA=-90,angleB=45,armA=3.5cm]{->}{C}{H} \tlput{\parbox{2.9cm}{這裡每一組$\omega_{b}$的誤差會因為用上上一個姿態位置而包含上一個$\omega_{b}$的誤差,因此誤差會累積而放大。}}

\rput(14,4){
\rnode{I}{
\psframebox[fillcolor=white,fillstyle=solid]{
\input{compare_smallfig_case1.pgf}}}}

\ncline[nodesep=3pt]{->}{F}{I} \naput{\parbox{3cm}{比較兩種方法給出的z軸間的夾角}}

\nput{-90}{I}{
\parbox{4.5cm}{結論:四十秒誤差已達2度,不過四十秒後誤差增加斜率較左邊方法小,不過誤差還是會增加。}}

\uput{5}[-90](14,4){
\rnode{N}{
\psframebox[fillcolor=white,fillstyle=solid,framearc=0.3]{
適合中長時間姿態粗估}}}

\rput(7,4){
\rnode{J}{
\psframebox[fillcolor=white,fillstyle=solid]{
\input{compare_smallfig_case2.pgf}}}}

\ncline[nodesep=3pt]{->}{G}{J} \naput{\parbox{3cm}{比較兩種方法給出的z軸間的夾角}}

\nput{-90}{J}{
\rnode{M}{
\parbox{4.5cm}{結論:四十秒內誤差非常小優於右側方法,可惜四十秒後誤差因累積而增加幅度較大。}}}

\uput{5}[-90](7,4){
\rnode{O}{
\psframebox[fillcolor=white,fillstyle=solid,framearc=0.3]{
適合短時間高精度姿態估測}}}


\end{pspicture}


%\end{document}