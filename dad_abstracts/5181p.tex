%%%%%%%%%%%%%%%%%%%%%%%%%%%%%%%%%%%%%%%%%%%%%%%%%%%%%%%%%%%%%%%%%
%Paper:    5181p.tex                  PUT YOUR PAPER FILE NAME
%%%%%%%%%%%%%%%%%%%%%%%%%%%%%%%%%%%%%%%%%%%%%%%%%%%%%%%%%%%%%%%%%
\documentclass[runningheads]{svmult}

\usepackage{makeidx}   % allows index generation
\usepackage{graphicx}  % standard LaTeX graphics tool
                       % for including eps-figure files
\usepackage{subeqnar}  % subnumbers individual equations
                       % within an array
\usepackage{multicol}  % used for the two-column index
%\usepackage{cropmark} % cropmarks for pages without
                       % pagenumbers - only needed when manuscript
                       % is printed from paper and not from data
\usepackage{physprbb}  % modified textarea for proceedings,
                       % lecture notes, and the like.
\makeindex             % used for the subject index
                       % please use the style sprmidx.sty with
                       % your makeindex program

%%upright Greek letters (example below: upright "mu")
\newcommand{\greeksym}[1]{{\usefont{U}{psy}{m}{n}#1}}
\newcommand{\umu}{\mbox{\greeksym{m}}}
\newcommand{\udelta}{\mbox{\greeksym{d}}}
\newcommand{\uDelta}{\mbox{\greeksym{D}}}
\newcommand{\uPi}{\mbox{\greeksym{P}}}
%%%%%%%%%%%%%%%%%%%%%%%%%%%%%%%%%%%%%%%%%%%%%%%%%%%%%%%%%%%%%
%%%%%%%%%%%%%%%%%%%%%%%%%%%%%%%%%%%%%%%%%%%%%%%%%%%%%%%%%%%%%
\begin{document}
%
\title*{Numerical Simulation of Blast Wave \protect\newline Propagation in a Double-Bent Duct}
\toctitle{Numerical Study of Blast wave Propagation in a Double-Bent Duct}
% allows explicit linebreak for the table of content
%
%
\titlerunning{Paper 5181: Blast Wave Propagation in a Double-Bent Duct}
%
\author{S.M.~Liang\inst{1}
\and K.C.~Weng\inst{1}
\and K.~Takayama\inst{2}}
%
\authorrunning{S.M.~Liang et al.}
%
%
\institute{National Cheng Kung University, Tainan 701, Taiwan, R.O.C.
\and Tohoku University, Sendai 980-8577, Japan}

\maketitle              % typesets the title of the contribution
\index{Blast wave}
\index{Vortex}
\index{Diffraction}
\index{Liang, S.M.}\index{Weng, K.C.}\index{Takayama, K.}
\begin{abstract}
\index{abstract} A high-resolution Euler solver has been used to investigate the flow field induced by a blast wave in a double-bent duct. A blast wave/vortex interaction and shock/shock interactions are also studied. A pair of vortices with an opposite rotating direction is found. The trajectories of the induced main vortices are explored for different initial pressure ratios. It was found that the vortex could deform the shock front of the blast wave because of compression and expansion involved in their interaction. Moreover, after blast wave diffraction around the first convex corner, a locally maximum wall pressure could occur at the second concave corner and a globally maximum wall pressure at the first concave corner, resulting in a large loading at the compression corners.
\end{abstract}

\section{Introduction}
%
Engineers and scientists are quite interested in the problem of blast wave propagation in ducts, since the phenomenon of blast wave propagation takes place, such as in the exhaust pipes of internal combustion engines or inside a building due to explosions. A fundamental study of blast wave propagation in a duct is helpful in understanding of blast wave dynamics. In this study we investigated the unsteady flow of planar blast wave propagation in a double-bent duct with two $90^\circ$ turns. The problem involved a complicated unsteady flow field of blast wave/vortex interactions, shock/shock interactions, and blast wave reflections.

Since a blast wave has the jump characteristics of a shock wave associated with expansion waves behind it, a weighted essentially non-oscillatory (WENO) scheme of Jiang and Shu \cite{5181pref1} is employed in this study. The WENO scheme possesses a high-resolution ability of capturing discontinuities. The working fluid is assumed to be air. The duct geometry is schematically shown in Fig.~\ref{5181plab1}. Two distance parameters, \(s_1\) and \(s_2\), are chosen. The parameter \(s_1\) is used to measure the distance from the first convex corner along the lower wall. The parameter \(s_2\)  is the distance from the first concave corner along the upper wall. Thus \(s_1=0\) denotes the first convex corner, \(s_1=1\) for the second concave corner, \(s_2=0\) for the first concave corner, and \(s_2=1\) for the second convex corner. At the duct outlet, \(s_1=2.5\) denotes the lower wall exit and \(s_2=2\) for the upper wall exit.
%
\begin{figure}
\begin{center}
\includegraphics[width=0.5\textwidth]{5181pf1.eps}
\end{center}
\caption{A schematic diagram of a double-bent duct}
\label{5181plab1}
\end{figure}

\section{Mathematical Formulation and Numerical Method}

   The Euler equations are used in this study. The high-order numerical scheme of Jiang and Shu \cite{5181pref1} in a finite volume approach with an extension to curvilinear coordinates is used for solving the Euler equations, namely, a fourth-order Runge-Kutta method is used for time integration and a fifth-order WENO scheme is used for spatial discretization. 

\section{Results and Discussion}

\subsection{Code Validation}

To validate the accuracy of the present solver, the problem of shock wave propagation through the double-bend duct is computed. The computed result is compared with the holographic interferogram that was obtained by Yang et al.~\cite{5181pref2} . Figure 2 shows the comparison of the computed isopycnics with the holographic picture at the dimensionless time \(t=3.70\). The dimensionless time $t$ corresponds to the dimensional time of \(\tilde{t}=200\ \mu\rm s\) in the experiment. At this instant, the incident shock has experienced two bends. These two convex corners (\(s_1=0 , s_2=1\)) have induced diffracted shock waves and vortices near the corners, as shown in Fig.~\ref{5181plab2}. By comparing Figs.~\ref{5181plab2}(a) and \ref{5181plab2}(b), we can see that good agreement is obtained.

%
\begin{figure}[h]
\begin{minipage}{0.48\textwidth}
  \includegraphics[width=\textwidth]{5181pf2.eps}
  \caption{Comparison of (a) computed isopycnics with (b) holographic interferograms, $M_s=1.08$, $t=3.70$}
\label{5181plab2}
\end{minipage}  
\hfil
\begin{minipage}[t]{0.48\textwidth}
\vspace*{-50mm}
  \includegraphics[width=\textwidth]{5181pf3.eps}
  \caption{The flow pattern of isopycnic, \(P_h/P_l=40\), $t=4$}
\label{5181plab3}
\includegraphics*[width=\textwidth]{5181pf4.eps}
\caption{The recorded maximum wall pressures on the lower wall}
\label{5181plab4}
\end{minipage}
\end{figure}
%

\subsection{Blast Wave Propagation in the Double-Bent Duct}

   In order to generate a planar blast wave, we consider a high-pressure region, located near \(x=0\), as shown in Fig.~\ref{5181plab1}. The region has a width of $R/40$, where $R$ is the duct width. A blast wave is generated by rupture of a diaphragm that separates the high- and  low-pressure regions. The temperatures in the both regions were chosen to be the same. The blast wave will propagate into the downstream duct. The initial pressure ratio \(P_h / P_l\) is assumed to be 40, which can generate a blast wave with intensity of shock Mach number of \(M_s=1.55\)  right before the first convex corner.

\subsubsection*{Flow Field}

   Figure \ref{5181plab3} shows the isopycnics at $t=4$. We can see that the diffracted wave $D_2$ is moving to the duct exit. The vortices induced by the convex corners are clearly seen. The reflected wave of the diffracted wave $D_2$ is denoted by the wave \(R_4\). Other reflected waves have encountered two or three reflections. The evolution of these waves is complicated and omitted here. It was found that there was a contact surface which occurred at approximately \(x=0.4\).

\subsubsection*{Wall Pressure Distribution}

   Since the blast wave propagation is unsteady and nonlinear, the flow properties, in particular the wall pressures, are changing with time. The maximum wall pressure is the highest wall pressure recorded at a fixed point for \(t \leq  4 \) . We only recorded the maximum wall pressures at every grid point on the upper and lower walls after the first $90^\circ$ turn. Figure \ref{5181plab4} shows the recorded maximum wall pressures on the lower $s_1$ wall for different initial pressure ratios. We chose the initial pressure ratio to be \(P_h/P_l=20 \), 40 and 60. These ratios produced a shock Mach number of 1.45 for \(P_h/P_l=20 \), 1.55 for \(P_h/P_l=40 \), and 1.75 for \(P_h/P_l=60 \) right before the blast wave diffraction around the first convex corner. Obviously, a larger value of \(P_h/P_l \) produces a stronger blast wave. We use the case of \(P_h/P_l=40 \) to describe the variation of the maximum wall pressures, since other cases produced a similar result. From Fig. 4, we can see that there is a peak with value of  \(P_{max}/P_0=2.41 \) occurring in the vicinity of the first convex corner, where $P_0$ denotes the undisturbed pressure. Away from the corner the maximum wall pressure is decreased to 1.20 which occurs at \(s_1 = 0.057 \). For \(s_1 > 0.13 \), the maximum wall pressure monotonically decreases from 1.70 to 1.29 at \(s_1 = 0.77 \). For \(s_1 > 0.77 \), the maximum wall pressure starts to increase to the second peak with \(P_{max}/P_0=1.77 \) at \(s_1 = 1 \), the second concave corner. For \(s_1 > 1 \), the maximum wall pressure with two jumps is monotonically decreased to the initial pressure at downstream. The pressure jump was caused by a reflection of blast wave. Moreover, it was found that the maximum wall pressure in the vicinity of the two convex corners is the lowest value of maximum wall pressures, and the maximum wall pressure at the second concave corner is locally a peak value, compared with those at the neighboring regions.

\subsection{Blast Wave/Vortex Interaction}

\subsubsection*{Vorticity Generation}

The vorticity production in the double-bent duct is completely due to blast wave diffraction. We can use a vorticity transport equation to explain the vorticity generation along the particle path. For an inviscid, two-dimensional flow with shock waves, we can easily derive the following vorticity transport equation:
$$\frac{D\vec{\omega}}{Dt}=-\vec{\omega}(\nabla\cdot\vec{V})-\frac{\nabla p\times\nabla\rho}{\rho^2}$$ 
where $\vec{\omega}$ denotes the vorticity, $\vec\omega=\nabla\times\vec{V}$, and $\nabla$ the gradient operator. The mechanism of vorticity production is attributed to two parts. One is due to flow compressibility -- the divergence of $\vec{V}$, $\nabla\cdot\vec{V}$, that is, the fluid dilatation. The other is due to the baroclinic effect, $\nabla p\times \nabla \rho$. During  blast wave diffraction, the flow near the corner is in expansion, and the wave front is being elongated, resulting in a curved non-uniform wave front. Consequently, the flow behind the curved wave front has pressure and density gradients, and generates a non-zero value of $\nabla p\times \nabla \rho$.

\subsubsection*{Blast Wave Deformation}

   When the reflected blast wave interacts with the vortex caused by the first convex corner, the blast wave encounters two different processes of interaction, resulting in blast wave deformation, as shown in Fig. 5. The wave deformation involves two processes -- compression and expansion. The upper part of the blast wave in the vortex region encounters a compression (deceleration) effect because the vortex flow direction is in a clockwise direction, which is opposite the direction of the moving blast wave. The lower part of the blast wave in the vortex region encounters an expansion (acceleration) effect because the flow direction is the same as the direction of the moving blast wave. Since the lower part of the blast wave in the vortex region moves faster than the upper part, the lower part will move faster away from the vortex core than the upper part. 
\begin{figure}
\begin{center}
\includegraphics[width=.5\textwidth]{5181pf5.eps}
\end{center}
\caption{Pressure contours near the first convex corner when the blast wave is interacting with the induced vortex for \(P_h/P_l=40\), $t = 2.525$}
\label{5181plab5}
\end{figure}

\subsubsection*{Effect of Blast Wave Intensity}

   We plotted the velocity field and the vorticity contour near the first convex corner for different values of \(p_h/p_l\). It was found that there are two vortices generated with opposite rotation. The upstream vortex has a counterclockwise rotation, and the downstream vortex has a clockwise rotation. The downstream vortex intensity is stronger than the upstream one. Moreover, the more intense the blast wave, the further downstream and the stronger the vortex. 

\section{Conclusions}

A reasonably accurate Euler solver was used to investigate the flow fields induced by blast waves in a double-bent duct. blast wave/vortex interactions were studied. The formation of the induced vortex was investigated and reported. A pair of vortices with an opposite rotating direction is found. For a stronger blast wave, the two vortices consist of one stronger vortex and one weaker vortex. The stronger vortex is located further downstream than the weaker one. It was found that the vortex could deform the blast wave because of compression and expansion involved in their interaction. Moreover, after blast wave diffraction around the first convex corner, there is a locally maximum pressure at the first compression (concave) corner and a globally maximum pressure at the second compression (concave) corner. In other words, the duct wall has a larger loading near the compression corners than the neighboring regions.
\\\\
{\small{\bf Acknowledgment.}
The support for this work under the contract of National Science Council, NSC 89-2612-E-006-018, is gratefully acknowledged.
}

\begin{thebibliography}{2}

\addcontentsline{toc}{section}{References}

\bibitem{5181pref1} Jiang GS, Shu CW (1996) Efficient implementation of weighted ENO schemes. J Comp Phys 126(1):202--228

\bibitem{5181pref2} Yang J, Onodera O, Takayama K (1994) Holographic interferometric investigation of shock wave diffraction. J Vizualization Soc Jpn 14(S1):85--88

\end{thebibliography}

\end{document}